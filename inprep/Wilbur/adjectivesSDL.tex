%\documentclass[ number=5
			   ,series=sidl
			   ,isbn=xxx-x-xxxxxx-xx-x
			   ,url=http://langsci-press.org/catalog/book/17
			   ,output=long   % long|short|inprep              
			   %,blackandwhite
			   %,smallfont
			   ,draftmode   
			  ]{LSP/langsci}                          

\usepackage{LSP/lsp-styles/lsp-gb4e}		% verhindert Komma bei mehrfachen Fußnoten?
                                                      
\usepackage{layout}
\usepackage{lipsum}

%%%% ABOVE FOR LangSciPress %%%%
%%%% ABOVE FOR LangSciPress %%%%
%%%% ABOVE FOR LangSciPress %%%%
\usepackage{libertine}%work-around solution for rendering problematic characters ʦ, ͡  (mostly in \textbf{})

\usepackage{longtable}%Double-lines (\hline\hline) aren’t typeset properly in ‘longtable’-environment across several pages! conflict with other package (maybe xcolor with option ‘tables’?)

\usepackage{multirow}

\usepackage{array} %allows, among other things, centering column content in a table while also specifying width, creates new column style "x" for center-alignment, "y" for right-alignment
\newcolumntype{x}[1]{>{\centering\hspace{0pt}}p{#1}}%
\newcolumntype{y}[1]{>{\raggedleft\hspace{0pt}}p{#1}}%

\usepackage[]{placeins}%using \FloatBarrier command, all floats still floating at that point will be typeset, and cannot cross that boundary. the option here \usepackage[section]{placeins} automatically adds \FloatBarrier to every \section command (only works for \section commands, nothing lower than that!)
%\usepackage{afterpage}%by using the command \afterpage{\clearpage}, all floats will appear, but no new page will be started, thus avoiding bad page breaks around floats

\usepackage{vowel} %for vowel space chart


%%%IS THIS NECESSARY??
%%%%following allows you to refer to footnotes (from http://anthony.liekens.net/index.php/LaTeX/MultipleFootnoteReferences)
%\newcommand{\footnoteremember}[2]{
%  \footnote{#2}
%  \newcounter{#1}
%  \setcounter{#1}{\value{footnote}}
%} \newcommand{\footnoterecall}[1]{
%  \footnotemark[\value{#1}]} 
%%%%previous allows you to refer to footnotes: use \footnoteremember{referenceText} in footnote, then \footnoterecall{referenceText} to refer.

\usepackage{tikz}%
\usetikzlibrary{plothandlers,matrix,decorations.text,shapes.arrows,shadows,chains,positioning,scopes}

\usepackage{synttree} %zeichnet linguistische Bäume
\branchheight{36pt}%sets height between rows in synttree

\usepackage{lscape}%used for landscape pages in index (list of recordings)

\usepackage{polyglossia}
\setmainlanguage{english}


%%%TAKE OUT FOR FINAL VERSION:
%%%TAKE OUT FOR FINAL VERSION:
%%%TAKE OUT FOR FINAL VERSION:

%%%%following readjusts margin text!
%\setlength{\marginparwidth}{20mm}
%\let\oldmarginpar\marginpar
%\renewcommand\marginpar[1]{\-\oldmarginpar[\raggedleft\footnotesize\vspace{-7pt}\color{red}\It{→ #1}]%
%{\raggedright\footnotesize\vspace{-7pt}\color{red}\It{→ #1}}}
%%%%previous readjusts margin text!

%%%The following lines set depth of ToC (LSP default is only 3 levels)!
%%%\renewcommand{\contentsname}{Table of Contents} % überschrift des inhaltsverzeichnisses
%\setcounter{secnumdepth}{5}%sets how deep section/subsection/subsubsections are numbered
%\setcounter{tocdepth}{5}%sets the depth of the ToC %but this doesn't seem to work!!!
%% new commands for LSP book (Grammar of Pite Saami, by J. Wilbur)

\newcommand{\PS}{Pite Saami}
\newcommand{\PSDP}{Pite Saami Documentation Project}
\newcommand{\WLP}{Wordlist Project}

\newcommand{\HANG}{\everypar{\hangindent15pt \hangafter1}}%also useful for table cells
\newcommand{\FB}{\FloatBarrier}%shortcut for this command to print all floats w/o pagebreak

\newcommand{\REF}[1]{(\ref{#1})}%adds parenthesis around the reference number, particularly useful for examples.%\Ref had clash with LSP!
\newcommand{\dline}{\hline\hline}%makes a double line in a table
\newcommand{\superS}[1]{\textsuperscript{#1}}%adds superscript element
\newcommand{\sub}[1]{$_{#1}$}%adds subscript element
\newcommand{\Sc}[1]{\textsc{#1}}%shortcut for small capitals (not to be confused with \sc, which changes the font from that point on)
\newcommand{\It}[1]{\textit{#1}}%shortcut for italics (not to be confused with \it, which changes the font from that point on)
\newcommand{\Bf}[1]{\textbf{#1}}%shortcut for bold (not to be confused with \bf, which changes the font from that point on)
\newcommand{\BfIt}[1]{\textbf{\textit{#1}}}
\newcommand{\BfSc}[1]{\textbf{\textsc{#1}}}
\newcommand{\Tn}[1]{\textnormal{#1}}%shortcut for normal text (undo italics, bolt, etc.)
\newcommand{\MC}{\multicolumn}%shortcut for multicolumn command in tabular environment - only replaces command, not variables!
\newcommand{\MR}{\multirow}%shortcut for multicolumn command in tabular environment - only replaces command, not variables!
\newcommand{\TILDE}{∼}%U+223C %OLD:~}%shortcut for tilde%command ‘\Tilde’ clashes with LSP!%
\newcommand{\BS}{\textbackslash}%backslash
\newcommand{\Red}[1]{{\color{red}{#1}}}%for red text
\newcommand{\Blue}[1]{{\color{blue}{#1}}}%for blue text
\newcommand{\PLUS}{+}%nicer looking plus symbol
\newcommand{\MINUS}{-}%nicer looking plus symbol
%    Was die Pfeile betrifft, kannst Du mal \Rightarrow \mapsto \textrightarrow probieren und dann \mathbf \boldsymbol oder \pbm dazutun.
\newcommand{\ARROW}{\textrightarrow}%→%dieser dicke Pfeil ➜ wird nicht von der LSP-Font unterstützt: %\newcommand{\ARROW}{{\fontspec{DejaVu Sans}➜}}
\newcommand{\DARROW}{\textleftrightarrow}%↔︎%DoubleARROW
\newcommand{\BULLET}{•}%
%%✓ does not exist in the default LSP font!
\newcommand{\CH}{\checkmark}%%\newcommand{\CH}{\fontspec{Arial Unicode MS}✓}%CH as in CHeck
%%following used to separate alternation forms for consonant gradation and umlaut patterns:
\newcommand{\Div}{‑}%↔︎⬌⟷⬄⟺⇔%non-breaking hyphen: ‑  
\newcommand{\QUES}{\textsuperscript{?}}%marks questionable/uncertain forms

\newcommand{\jvh}{\mbox{\It{j}-suffix} vowel harmony}%
%\newcommand{\Ptcl}{\Sc{ptcl} }%just shortcut for glossing ‘particle’
%\newcommand{\ATTR}{{\Sc{attributive}}}%shortcut for ATTRIBUTIVE in small caps
%\newcommand{\PRED}{{\Sc{predicative}}}%shortcut for PREDICATIVE in small caps
%\newcommand{\COMP}{{\Sc{comparative}}}%shortcut for COMPARATIVE in small caps
%\newcommand{\SUPERL}{{\Sc{superlative}}}%shortcut for SUPERLATIVE in small caps
\newcommand{\SG}{{\Sc{singular}}}%shortcut for SINGULAR in small caps
\newcommand{\DU}{{\Sc{dual}}}%shortcut for DUAL in small caps
\newcommand{\PL}{{\Sc{plural}}}%shortcut for PLURAL in small caps
%\newcommand{\NOM}{{\Sc{nominative}}}%shortcut for NOMINATIVE in small caps
%\newcommand{\ACC}{{\Sc{accusative}}}%shortcut for ACCUSATIVE in small caps
%\newcommand{\GEN}{{\Sc{genitive}}}%shortcut for GENITIVE in small caps
%\newcommand{\ILL}{{\Sc{illative}}}%shortcut for ILLATIVE in small caps
%\newcommand{\INESS}{{\Sc{inessive}}}%shortcut for INESSIVE in small caps
\newcommand{\ELAT}{{\Sc{elative}}}%shortcut for ELATIVE in small caps
%\newcommand{\COM}{{\Sc{comitative}}}%shortcut for COMITATIVE in small caps
%\newcommand{\ABESS}{{\Sc{abessive}}}%shortcut for ABESSIVE in small caps
%\newcommand{\ESS}{{\Sc{essive}}}%shortcut for ESSIVE in small caps
%\newcommand{\DIM}{{\Sc{diminutive}}}%shortcut for DIMINUTIVE in small caps
%\newcommand{\ORD}{{\Sc{ordinal}}}%shortcut for ORDINAL in small caps
%\newcommand{\CARD}{{\Sc{cardinal}}}%shortcut for CARDINAL in small caps
%\newcommand{\PROX}{{\Sc{proximal}}}%shortcut for PROXIMAL in small caps
%\newcommand{\DIST}{{\Sc{distal}}}%shortcut for DISTAL in small caps
%\newcommand{\RMT}{{\Sc{remote}}}%shortcut for REMOTE in small caps
%\newcommand{\REFL}{{\Sc{reflexive}}}%shortcut for REFLEXIVE in small caps
%\newcommand{\PRS}{{\Sc{present}}}%shortcut for PRESENT in small caps
%\newcommand{\PST}{{\Sc{past}}}%shortcut for PAST in small caps
%\newcommand{\IMP}{{\Sc{imperative}}}%shortcut for IMPERATIVE in small caps
%\newcommand{\POT}{{\Sc{potential}}}%shortcut for POTENTIAL in small caps
\newcommand{\PROG}{{\Sc{progressive}}}%shortcut for PROGRESSIVE in small caps
\newcommand{\PRF}{{\Sc{perfect}}}%shortcut for PERFECT in small caps
\newcommand{\INF}{{\Sc{infinitive}}}%shortcut for INFINITIVE in small caps
%\newcommand{\NEG}{{\Sc{negative}}}%shortcut for NEGATIVE in small caps
\newcommand{\CONNEG}{{\Sc{connegative}}}%shortcut for CONNEGATIVE in small caps
\newcommand{\ATTRs}{{\Sc{attr}}}%shortcut for ATTR in small caps
\newcommand{\PREDs}{{\Sc{pred}}}%shortcut for PRED in small caps
%\newcommand{\COMPs}{{\Sc{comp}}}%shortcut for COMP in small caps
%\newcommand{\SUPERLs}{{\Sc{superl}}}%shortcut for SUPERL in small caps
\newcommand{\SGs}{{\Sc{sg}}}%shortcut for SG in small caps
\newcommand{\DUs}{{\Sc{du}}}%shortcut for DU in small caps
\newcommand{\PLs}{{\Sc{pl}}}%shortcut for PL in small caps
\newcommand{\NOMs}{{\Sc{nom}}}%shortcut for NOM in small caps
\newcommand{\ACCs}{{\Sc{acc}}}%shortcut for ACC in small caps
\newcommand{\GENs}{{\Sc{gen}}}%shortcut for GEN in small caps
\newcommand{\ILLs}{{\Sc{ill}}}%shortcut for ILL in small caps
\newcommand{\INESSs}{{\Sc{iness}}}%shortcut for INESS in small caps
\newcommand{\ELATs}{{\Sc{elat}}}%shortcut for ELAT in small caps
\newcommand{\COMs}{{\Sc{com}}}%shortcut for COM in small caps
\newcommand{\ABESSs}{{\Sc{abess}}}%shortcut for ABESS in small caps
\newcommand{\ESSs}{{\Sc{ess}}}%shortcut for ESS in small caps
%\newcommand{\DIMs}{{\Sc{dim}}}%shortcut for DIM in small caps
%\newcommand{\ORDs}{{\Sc{ord}}}%shortcut for ORD in small caps
%\newcommand{\CARDs}{{\Sc{card}}}%shortcut for CARD in small caps
\newcommand{\PROXs}{{\Sc{prox}}}%shortcut for PROX in small caps
\newcommand{\DISTs}{{\Sc{dist}}}%shortcut for DIST in small caps
\newcommand{\RMTs}{{\Sc{rmt}}}%shortcut for RMT in small caps
\newcommand{\REFLs}{{\Sc{refl}}}%shortcut for REFL in small caps
\newcommand{\PRSs}{{\Sc{prs}}}%shortcut for PRS in small caps
\newcommand{\PSTs}{{\Sc{pst}}}%shortcut for PST in small caps
\newcommand{\IMPs}{{\Sc{imp}}}%shortcut for IMP in small caps
\newcommand{\POTs}{{\Sc{pot}}}%shortcut for POT in small caps
\newcommand{\PROGs}{{\Sc{prog}}}%shortcut for PROG in small caps
\newcommand{\PRFs}{{\Sc{prf}}}%shortcut for PRF in small caps
\newcommand{\INFs}{{\Sc{inf}}}%shortcut for INF in small caps
\newcommand{\NEGs}{{\Sc{neg}}}%shortcut for NEG in small caps
\newcommand{\CONNEGs}{{\Sc{conneg}}}%shortcut for CONNEG in small caps

\newcommand{\subNP}{{\footnotesize\sub{NP}}}%shortcut for NP (nominal phrase) in subscript
\newcommand{\subVC}{{\footnotesize\sub{VC}}}%shortcut for VC (verb complex) in subscript
\newcommand{\subAP}{{\footnotesize\sub{AP}}}%shortcut for NP (adjectival phrase) in subscript
\newcommand{\subAdvP}{{\footnotesize\sub{AdvP}}}%shortcut for AdvP (adverbial phrase) in subscript
\newcommand{\subPP}{{\footnotesize\sub{PP}}}%shortcut for NP (postpoistional phrase) in subscript

\newcommand{\ipa}[1]{{\fontspec{Linux Libertine}#1}}%specifying font for IPA characters

\newcommand{\SEC}{§}%standardize section symbol and spacing afterwards
%\newcommand{\SEC}{§\,}%

\newcommand{\Nth}{{\footnotesize(\It{n})}}%used in table of numerals in ADJ chapter

%%newcommands for tables in introductionSDL.tex:
\newcommand{\cliticExs}[3]{\Tn{\begin{tabular}{p{28mm} c p{28mm} p{35mm}}\It{#1}&\ARROW &\It{#2} & ‘#3’\\\end{tabular}}}%specifically for the two clitic examples
\newcommand{\Grapheme}[1]{\It{#1}}%formatting for graphemes in orthography tables
%%new command for the section on orthographic examples; syntax: #1=orthography, #2=phonology, #3=gloss
\newcommand{\SpellEx}[3]{\Tn{\begin{tabular}{p{70pt} p{70pt} l}\ipa{/#2/}&\It{#1}& ‘#3’ \\\end{tabular}}}%formatting for orthographic examples (intro-Chapter)


%%new transl tier in gb4e; syntax: #1=free translation (in single quotes), #2=additional comments, z.B. literal meaning:
\newcommand{\Transl}[2]{\trans\Tn{‘#1’ #2}}%new transl tier in gb4e;
\newcommand{\TranslMulti}[2]{\trans\hspace{12pt}\Tn{‘#1’ #2}}%new transl tier in gb4e for a dialog to be included under a single example number


%% used for examples in the Prosody and Segmental phonology chapters:
\newcommand{\PhonGloss}[7]{%PhonGloss = Phonology Gloss;
%pattern: \PhonGloss{label}{phonemic}{phonetic}{orthographic}{gloss}{recording}{utterance}
\ea\label{#1}
\Tn{\begin{tabular}[t]{p{30mm} l}
\ipa{/#2/}	& \It{#4} \\
\ipa{[#3]}	&\HANG ‘#5’\\%no table row can start with square brackets! thus the workaround with \MC
\end{tabular}\hfill\hyperlink{#6}{{\small\textnormal[pit#6#7]}}%\index{Z\Red{rec}!\Red{pit#6}}\index{Z\Red{utt}!\Red{pit#6#7} \Blue{Phon}}
}
\z}
\newcommand{\PhonGlossWL}[6]{%PhonGloss = Phonology Gloss for words from WORDLIST, not from corpus!;
%pattern: \PhonGloss{label}{phonemic}{phonetic}{orthographic}{gloss}{wordListNumber}
\ea\label{#1}
\Tn{\begin{tabular}[t]{p{30mm} l}
\ipa{/#2/}	& \It{#4} \\
\ipa{[#3]}	&\HANG ‘#5’\\%no table row can start with square brackets! thus the workaround with \MC
\end{tabular}\hfill\hyperlink{explExs}{{\small\textnormal[#6]}}%\index{Z\Red{wl}!\Red{#6}\Blue{Phon}}
}
\z}

%%for derivation examples in the derivational morphology chapter!
%syntax: \DerivExam{#1}{#2}{#3}{#4}{#5}{#6}
%#1: base, #2: base-gloss, #3: derived form, #4: derived form gloss, #5: derived form translation, #6: pit-recording, #7: utterance number
\newcommand{\DW}{28mm}%for following three commands, to align arrows throughout
%%%%OLD:
%%%\newcommand{\DerivExam}[7]{\Tn{\begin{tabular}[t]{p{\DW}cl}\It{#1}&\ARROW&\It{#3}\\#2&&#4\\\end{tabular}\hfill\pbox{.3\textwidth}{\hfill‘#5’\\\hbox{}\hfill\hyperlink{pit#6}{{\small\textnormal[pit#6.#7]}}}
%%%%\index{Z\Red{rec}!\Red{pit#6}}\index{Z\Red{utt}!\Red{pit#6.#7}}
%%%}}
%NEW:
\newcommand{\DerivExam}[7]{\Tn{
\begin{tabular}[t]{p{\DW}x{5mm}l}\It{#1}&\ARROW&\It{#3}\\\end{tabular}\hfill‘#5’\\
\hspace{1mm}\begin{tabular}[t]{p{\DW}x{5mm}l}#2&&#4\\\end{tabular}\hfill\hyperlink{pit#6}{{\small\textnormal[pit#6.#7]}}
%\index{Z\Red{rec}!\Red{pit#6}}\index{Z\Red{utt}!\Red{pit#6.#7}}
}}
%%same as above, but supress any reference to a specific utterance
\newcommand{\DerivExamX}[7]{\Tn{
\begin{tabular}[t]{p{\DW}x{5mm}l}\It{#1}&\ARROW&\It{#3}\\\end{tabular}\hfill‘#5’\\
\hspace{1mm}\begin{tabular}[t]{p{\DW}x{5mm}l}#2&&#4\\\end{tabular}\hfill\hyperlink{pit#6}{{\small\textnormal[pit#6]\It{e}}}
%\index{Z\Red{rec}!\Red{pit#6}}\index{Z\Red{utt}!\Red{pit#6.#7}}
}}
\newcommand{\DerivExamWL}[6]{\Tn{
\begin{tabular}[t]{p{\DW}x{5mm}l}\It{#1}&\ARROW&\It{#3}\\\end{tabular}\hfill‘#5’\\
\hspace{1mm}\begin{tabular}[t]{p{\DW}x{5mm}l}#2&&#4\\\end{tabular}\hfill\hyperlink{explExs}{{\small\textnormal[#6]}}
%\index{Z\Red{wl}!\Red{#6}}
}}


%formatting of corpus source information (after \transl in gb4e-environments):
\newcommand{\Corpus}[2]{\hspace*{1pt}\hfill{\small\mbox{\hyperlink{pit#1}{\Tn{[pit#1.#2]}}}}%\index{Z\Red{rec}!\Red{pit#1}}\index{Z\Red{utt}!\Red{pit#1.#2}}
}%
\newcommand{\CorpusE}[2]{\hspace*{1pt}\hfill{\small\mbox{\hyperlink{pit#1}{\Tn{[pit#1.#2]}}\It{e}}}%\index{Z\Red{rec}!\Red{pit#1}}\index{Z\Red{utt}!\Red{pit#1.#2}\Blue{-E}}
}%
%%as above, but necessary for recording names which include an underline because the first variable in \href understands _ but the second variable requires \_
\newcommand{\CorpusLink}[3]{\hspace*{1pt}\hfill{\small\mbox{\hyperlink{pit#1}{\Tn{[pit#2.#3]}}}}%\index{Z\Red{rec}!\Red{pit#2}}\index{Z\Red{utt}!\Red{pit#2.#3}}
}%
%%as above, but for newer recordings which begin with sje20 instead of pit
\newcommand{\CorpusSJE}[2]{\hspace*{1pt}\hfill{\small\mbox{\hyperlink{sje20#1}{\Tn{[sje20#1.#2]}}}}%\index{Z\Red{rec}!\Red{sje20#1}}\index{Z\Red{utt}!\Red{sje20#1.#2}}
}%
\newcommand{\CorpusSJEE}[2]{\hspace*{1pt}\hfill{\small\mbox{\hyperlink{sje20#1}{\Tn{[sje20#1.#2]}}\It{e}}}%\index{Z\Red{rec}!\Red{sje20#1}}\index{Z\Red{utt}!\Red{sje20#1.#2}\Blue{-E}}
}%











%%hyphenation points for line breaks
%%add to TeX file before \begin{document} with:
%%%%hyphenation points for line breaks
%%add to TeX file before \begin{document} with:
%%%%hyphenation points for line breaks
%%add to TeX file before \begin{document} with:
%%\include{hyphenationSDL}
\hyphenation{
ab-es-sive
affri-ca-te
affri-ca-tes
Ahka-javv-re
al-ve-o-lar
com-ple-ments
%check this:
de-cad-es
fri-ca-tive
fri-ca-tives
gemi-nate
gemi-nates
gra-pheme
gra-phemes
ho-mo-pho-nous
ho-mor-ga-nic
mor-pho-syn-tac-tic
or-tho-gra-phic
pho-neme
pho-ne-mes
phra-ses
post-po-si-tion
post-po-si-tion-al
pre-as-pi-ra-te
pre-as-pi-ra-ted
pre-as-pi-ra-tion
seg-ment
un-voiced
wor-king-ver-sion
}
\hyphenation{
ab-es-sive
affri-ca-te
affri-ca-tes
Ahka-javv-re
al-ve-o-lar
com-ple-ments
%check this:
de-cad-es
fri-ca-tive
fri-ca-tives
gemi-nate
gemi-nates
gra-pheme
gra-phemes
ho-mo-pho-nous
ho-mor-ga-nic
mor-pho-syn-tac-tic
or-tho-gra-phic
pho-neme
pho-ne-mes
phra-ses
post-po-si-tion
post-po-si-tion-al
pre-as-pi-ra-te
pre-as-pi-ra-ted
pre-as-pi-ra-tion
seg-ment
un-voiced
wor-king-ver-sion
}
\hyphenation{
ab-es-sive
affri-ca-te
affri-ca-tes
Ahka-javv-re
al-ve-o-lar
com-ple-ments
%check this:
de-cad-es
fri-ca-tive
fri-ca-tives
gemi-nate
gemi-nates
gra-pheme
gra-phemes
ho-mo-pho-nous
ho-mor-ga-nic
mor-pho-syn-tac-tic
or-tho-gra-phic
pho-neme
pho-ne-mes
phra-ses
post-po-si-tion
post-po-si-tion-al
pre-as-pi-ra-te
pre-as-pi-ra-ted
pre-as-pi-ra-tion
seg-ment
un-voiced
wor-king-ver-sion
}\begin{document}\tableofcontents\clearpage

%%%%%%%%%%%%%%%%%%%%%%%%%%%%%%%% ALL THE ABOVE TO BE COMMENTED OUT FOR COMPLETE DOCUMENT! %%%%%%%%%%%


\chapter{Adjectivals}\label{adjectivesIntro}%\index{adjectives}\index{parts of speech!adjectives}
Adjectivals in \PS\ are defined syntactically by their ability to head an adjectival phrase (AP). They can be divided into four sub-categories based on both syntactic and morphological behavior, as summarized in Table \vref{adjectivalsSummary}.
\begin{table}\centering
\caption{Summary of syntactic and morphological features for the four types of adjectivals}\label{adjectivalsSummary}
\resizebox{\columnwidth}{!}{
\begin{tabular}{c p{140pt} p{140pt}}%\hline
					&\MC{1}{c}{\It{syntax}}		&\MC{1}{c}{\It{morphology}}	\\\hline
\It{attributive adjectives}	& attributive position within an NP	& no inflection (except in elliptic constructions)\\\hline%, morphologically, form their own sub-group by not inflecting}
\It{predicative adjectives}	& predicative position (complement of \It{årrot} ‘be’)	&inflect for number\\\hline% (like nominals)
\It{demonstratives}		& initial attributive position within an NP	& inflect for number \& case\\\hline%, morphologically, form their own sub-group by not inflecting}
\It{numerals}			& attributive or predicative position	& never inflect\\\hline
\end{tabular}}
\end{table}

While attributive adjectives generally do not inflect, in elliptical phrases in which the head of an NP is not realized overtly, they do inflect for case and number. 
Predicative adjectives are marked for number, and are morphologically similar to nominals. 
Demonstratives agree in number and case with the noun they modify. 
Numerals, on the other hand, are consistently uninflected. 
Finally, the two types of adjectives form an open sub-class, while numerals are a closed sub-class. 

The rest of this chapter covers adjectives, demonstratives and numerals as follows: §\,\ref{adjectivesATTR} provides a description of attributive adjectives, %(\ref{ADJinHeadlessNPs}) describes adjectives in headless NPs, 
while §\,\ref{predADJ} deals with predicative adjectives, before §\,\ref{notePredNounsAdjs} takes up the formal relationship between these two types. 
§\,\ref{compSuperlADJs} then goes on to describe comparative and superlative forms, 
before §\,\ref{comparingNPs} illustrates the implementation of such forms in making comparisons. 
Syntactic restrictions on the adjectives corresponding to ‘small, little’ are described in §\,\ref{smallADJs}. 
Quantifiers (a semantic sub-class of adjectives) are discussed in §\,\ref{quantifiers}, while 
demonstratives are presented in §\,\ref{demonstratives}. 
Finally, §\,\ref{numerals} covers numerals. %, which essentially function like other adjective and predicate adjective forms. %are morphosyntactically like other adjectives and predicate adjective forms, but semantically like numerals in that they quantify the referents of the nouns they modify. 
%Finally, demonstrative adjectives, which inflect like nouns, not adjectives, are described briefly in §\,\ref{demonstrativeAdjectives}.





\section{Attributive adjectives}\label{adjectivesATTR}\index{adjectives}\index{parts of speech!adjectives}\index{attribution}\index{attributive adjectives}
Attributive adjectives form the head of an AP modifying the head of the matrix nominal phrase, and are normally not subject to inflectional morphology. 
%Adjectives in \PS\ form a subclass of nouns which can modify the nominal head of an NP in attributive position. Unlike the nouns they modify, adjectives in attributive position do not inflect for case or number. 
As part of an attributive AP, an attributive adjective occurs before the head noun it modifies, but after a demonstrative, if present (cf. §\,\ref{nominalPhrases} on the structure of NPs). Examples are provided in \REF{adjEx1} through \REF{adjEx3}. 
\ea\label{adjEx1}
\glll	dat lä tjähppis båtsoj ja villges åjjve\\
	d-a-t lä tjähppis båtsoj ja villges åjjve\\
	\Sc{dem}-\Sc{dist}-\Sc{nom.sg} be\BS\Sc{3sg.prs} black reindeer\BS\Sc{nom.sg} and white head\BS\Sc{nom.sg}\\\nopagebreak
\Transl{it is a black reindeer and a white head}{}	\Corpus{100405b}{043}
\z
\ea\label{adjEx2}
\glll	guolle'l nåv njalga bäbbmo\\
	guolle=l nåv njalga bäbbmo\\
	fish\BS\Sc{nom.sg}=be\BS\Sc{3sg.prs} so tasty food\BS\Sc{nom.sg}\\\nopagebreak
\Transl{fish is such tasty food}{}	\Corpus{100310b}{025}
\z
\ea\label{adjEx3}
\glll	dat villges båtsoj\\
	d-a-t villges båtsoj\\
	\Sc{dem}-\Sc{dist}-\Sc{nom.sg} white reindeer\BS\Sc{nom.sg}\\\nopagebreak
\Transl{that white reindeer}{}	\CorpusE{090930a}{014}
%\glll	jå, månnå aj mujhtav gu lev unna mánatj\\
%	jå månnå aj mujhta-v gu le-v unna mána-tj\\
%	yes \Sc{1sg.nom} also remember-\Sc{1sg.prs} when be-\Sc{1sg.prs} small child-\Sc{dim}\BS\Sc{nom.sg}\\
%\trans	‘yes, I also remember when I was a small child’	\Corpus{080924.632}
\z

As the head of an AP, attributive adjectives can be modified by adverbs of grade, as illustrated by the AP \It{hoj buorak} ‘really good’ in \REF{adjEx4}.
\ea\label{adjEx4}
\glll	ja dat lä årrom hoj buorak giesse\\
	ja d-a-t lä årro-m hoj buorak giesse\\
	and \Sc{dem}-\Sc{dist}-\Sc{nom.sg} be\BS\Sc{3sg.prs} be-\Sc{prf} really good summer\BS\Sc{nom.sg}\\\nopagebreak
\Transl{and it has been a really good summer}{}	\Corpus{080909}{009}
\z

A number of adjectives end in \It{-s} (cf. the two adjectives \It{tjähppes} ‘black’ and \It{villges} ‘white’ in \REF{adjEx1}), which is often considered an ‘attribution’ marking suffix in the literature.\footnote{\citet[215-228]{Riessler2011} deals in detail with this final \It{-s}, which is common to all Saami languages. Rießler claims that it was grammaticalized from a 3\SGs\ possessive suffix, and originally only marked the attributive form of adjectives. However, note that Rießler also points out that “the system of attributive and predicative marking is highly irregular in the Saamic languages” \citep[215]{Riessler2011}.} 
However, as \It{njalga} ‘tasty’ in \REF{adjEx2} illustrates, not all adjectives are marked this way. Furthermore, corresponding predicate adjective forms (cf. §\,\ref{predADJ} below) often also have a final \It{-s}, sometimes even to the exclusion of the attributive adjective form. Because no consistent relationship between forms with and forms without a final \It{-s} exists, it is no longer a productive way to mark or derive either attributive or predicative adjectives, and is not considered to be morphologically meaningful in the present discussion. Nonetheless, it is worth nothing that adjectival forms (both attributive and predicative) ending in \It{-s} are common. 

%One difficulty encountered in researching adjectives was that s
Note also that some attributive adjectives appear to have two possible forms. For instance, \It{guhka} and \It{guhkes} ‘long’ were both encountered in elicitation sessions with a single speaker who insisted that both forms were equally valid (cf. recording \hyperlink{pit080819a}{pit080819a} starting at 33m14s). %In general there does not seem to be any reason to choose one form over another, and speakers used both forms seemingly interchangeably. 


\subsection{Attributive adjectives in elliptic constructions}\label{ADJinHeadlessNPs}
If the context of the wider discourse is sufficiently unambiguous, it is possible that the nominal head of an NP is not realized, but implied. %Because case and number marking occurs on the element of an NP which occurs at the right edge of the NP (cf. \Red{§\,\ref{}}), when such elliptic constructions (sometimes called ‘headless NPs’) feature an adjective, the adjective is then the right-most element and receives case and number marking. 
When such elliptic constructions (sometimes called ‘headless NPs’) feature an AP, the head adjective is then inflected for case and number. 

For instance, in the elliptic construction in \REF{adjNPheadEx1}, the adjective \It{ruopsis} ‘red’ %is the right-most element in the NP which 
is marked for singular number and as the object of the verb form \It{bårov} ‘I eat’ for accusative case. In \REF{adjNPheadEx2}, the subject NP also lacks an overt nominal head, but features the adjective \It{tjähppis} ‘black’ (cf. the example in \REF{adjEx1} above)%at the right edge
, which receives \NOMs.\PLs\ marking. 
\ea\label{adjNPheadEx1}
\glll	båråv ruopsisav\\
	bårå-v ruopsisa-v\\
	eat-\Sc{1sg.prs} red-\Sc{acc.sg}\\\nopagebreak
\Transl{I eat the red one}{}	\CorpusE{090930a}{119}
\z
\ea\label{adjNPheadEx2}
\glll	tjähppisa lä njallge\\
	tjähppis-a lä njallge\\
	black-\Sc{nom.pl} be\BS\Sc{3pl.prs} tasty\BS\Sc{nom.sg}\\\nopagebreak
\Transl{the black ones are tasty}{}	\CorpusE{090930a}{112}
\z

Adjectives in elliptic NPs can be preceded by a demonstrative, %emphasizing that a particular referent is intended, 
as in \REF{adjNPheadEx3}. 
\ea\label{adjNPheadEx3}
\glll	dat tjábba máhtta sáme gielav\\
	d-a-t tjábba máhtta sáme giela-v\\
	\Sc{dem}-\Sc{dist}-\Sc{nom.sg} beautiful\BS\Sc{nom.sg} can\BS\Sc{3sg.prs} Saami\BS\Sc{gen.sg} language-\Sc{acc.sg}\\\nopagebreak
\TranslLong{That beautiful one can (speak) the Saami language}{(referring to ‘beautiful girl’)}	\CorpusE{090930a}{148}
\z

As the host of case/number inflection, such adjectives look morphologically like nouns. %While future research is needed to supplement the data in the corpus, it seems that %UM: durchgestrichen
However, syntactically, these adjectives remain adjectives for two reasons. First, they can be modified by adverbs of grade, while nouns cannot be. Second, they generally have a referential antecedent that is the bearer of the property they denote.\footnote{The quantifier \It{aktak} ‘none, any’ can be used in an elliptic NP without a referential antecedent; cf. §\,\ref{quantifiers}, specifically example \REF{noneEx3}.} 
Semantically, they do not denote an entity (as nouns do), but a property, as with all adjectives. 
A further example is provided in \REF{adjNPheadEx4}. 
Here, the choice of the attributive adjective form \It{tjábba} ‘beautiful’ (as opposed to the corresponding predicate adjective form \It{tjábbe}) indicates that this is indeed an elliptical NP construction, and not predication. 
\ea\label{adjNPheadEx4}
\glll	lä huj tjábba, dat, dat lä jävvja\\
	lä huj tjábba d-a-t d-a-t lä jävvja\\
	be\BS\Sc{3sg.prs} quite beautiful\BS\Sc{nom.sg} \Sc{dem}-\Sc{dist}-\Sc{nom.sg} \Sc{dem}-\Sc{dist}-\Sc{nom.sg} be\BS\Sc{3sg.prs} white.reindeer\BS\Sc{nom.sg}\\\nopagebreak
\TranslLong{(it) is a quite beautiful one, it, it is a white reindeer}{(referring to a reindeer)}	\Corpus{100405b}{036-037}
%\glll	dat lä gal, lä huj tjábba, dat, dat lä jävvja\\
%	dat lä gal lä huj tjábba dat dat lä jävvja\\
%	\Sc{dem.dist}\BS\Sc{nom.sg} be\BS\Sc{3sg.prs} definitely be\BS\Sc{3sg.prs} quite beautiful\BS\Sc{nom.sg} \Sc{dem.dist}\BS\Sc{nom.sg} \Sc{dem.dist}\BS\Sc{nom.sg} be\BS\Sc{3sg.prs} white\_reindeer\BS\Sc{nom.sg}\\
%\trans	‘That is definitely, (it) is a quite beautiful (reindeer), it, it is a white reindeer’	\Corpus{100405b.036-037}
\z




\section{Predicative adjectives}\label{predADJ}%Modification of an NP using a predicative copula construction}
While attributive adjectives form the head of an attributive AP embedded in an NP, predicative adjectives form the head of an AP which is the complement of the copula verb \It{årrot} ‘be’ and ascribe a property to the subject referent. %which modify the NP which is the subject for the copula clause. %UM: durchgestrichen
%, but predicative APs modify the NP which is the subject of the copula predicate they are a constituent in. 
%Although the description of attributive adjectives above is quite straightforward, the situation becomes more complicated when considering predicative modification. This is because, in most cases, the predicative form is not identical to the attributive adjective form discussed above.  %UM: durchgestrichen
%Instead, most adjectives correspond to a form which functions as the complement in predicative copula constructions; such forms are hereafter referred to as ‘predicative adjectives’, in keeping with the tradition in Saami studies, and in emphasizing their semantic value as an property word. %However, morphologically speaking, such forms are more like nouns because they inflect for number. Whether these inflect for case is not able to be determined because they only occur in this syntactic position requiring \NOM\ case, which is not unambiguously marked for case. 
In \REF{adjPredEx1} and \REF{adjPredEx2}, for instance, the predicative adjective corresponding to the attributive adjective \It{tjähppis} ‘black’ (cf. the example in \REF{adjEx1} above) is \It{tjáhpat}. %, a class IVa noun. %, which inflects for number. %, as  in \NOMs.\PLs\ %using the \It{-a} suffix 
%in \REF{adjPredEx2}. 
\ea\label{adjPredEx1}
\glll	fáhttsa lä tjáhpat\\
	fáhttsa lä tjáhpat\\
	mitten\BS\Sc{nom.sg} be\BS\Sc{3sg.prs} black\BS\Sc{sg}\\\nopagebreak
\Transl{the mitten is black}{}	\CorpusE{090930a}{062}
\z
\ea\label{adjPredEx2}
\glll	fáhtsa lä tjáhpada\\
	fáhtsa lä tjáhpad-a\\
	mitten\BS\Sc{nom.pl} be\BS\Sc{3pl.prs} black-\Sc{pl}\\\nopagebreak
\Transl{the mittens are black}{}	\CorpusE{090930a}{063}
\z
Morphologically, predicative adjectives are much like nouns because they inflect for number. In fact, many predicative adjectives inflect for number in ways that clearly align with the \NOMs.\SGs\TILDE\NOMs.\PLs\ inflectional marking of certain noun classes. 
%Note that predicative adjectives are only attested in the corpus as a complement to a copula clause. 
The case could be made that they also inflect for case, although they are always in nominative case (cf. §\,\ref{copulaClauses} on copula clauses). %; with this in mind, the case could be made that predicative adjectives also inflect for case (even if that case is always \NOM). 
However, because no paradigmatic opposition to other case forms exists for predicative adjectives (they are only attested in the corpus as a nominative complement to a copula clause), I conclude that they only inflect for number. 
%, %\footnote{In non-predicative cases, the adjective (atributive) form is chosen.} %JW: not nec. because of next paragraph so claiming that case marking is present is purely based on analogous forms in other, full noun paradigms that clearly inflect for case.\footnote{For instance, the class IVa noun \It{vienak} ‘friend\Sc{nom.sg}’ is marked for \NOMs.\PLs\ by an \It{-a} suffix: \It{vienaga}.} 

Nonetheless, these are syntactically adjectives, as they head adjectival phrases and can be modified by adverbs of grade, such as \It{nav} ‘so’ as in \REF{adjPredEx3}.
\ea\label{adjPredEx3}
\glll	buhtsu lä nav buojde ja tjábbe\\
	buhtsu lä nav buojde ja tjábbe\\
	reindeer\BS\Sc{nom.pl} be\BS\Sc{3pl.prs} so fat\BS\Sc{pl} and beautiful\BS\Sc{pl}\\\nopagebreak
\Transl{the reindeer are so fat and beautiful}{}	\Corpus{080703}{014}
\z

Table \vref{typcialPredAdjTable} lists a number of attributive adjectives and the corresponding predicative adjectives; the latter clearly align with noun classes in their number marking. %, and the noun class the latter belong to. 
The table is divided into subgroups of word forms (indicated by small Roman numerals) that feature the same morphological relationship between attributive and predicative adjectives. 
%It is important to emphasize that these predicative adjectives are only attested in the corpus as complements to the copula verb, while the corresponding attributive adjectives are used in attribution. % and to head other NPs. 
%The example in \REF{adjNPheadEx2} above illustrates\marginpar{not sure \REF{adjNPheadEx2} really illustrates this well; maybe delete this sentence} this well: the subject is the adjective \It{tjähppis} ‘black’, and the copula’s complement is the predicative nominal \It{njallge} ‘tasty’ (and not the adjective \It{njalga} ‘tasty’, as in \REF{adjEx2}). 
\begin{table}\centering
\caption[Some attributive and predicative adjective sets]{Some attributive and predicative adjective sets, including the noun class corresponding to the number marking pattern exhibited by the predicate adjective}\label{typcialPredAdjTable}
\begin{tabular}{llllll}%\hline
	&		&\MC{3}{c}{\It{predicative adjectives}}			&\\
	&\It{attributive}&\MC{2}{c}{\It{number}}	&\It{corresp.}	&		\\%\hline
\It{no.}&\It{adjective}	&\SG	&\PL			&\It{N-class}	&\It{gloss}	\\\dline
%attr +s, pred +t
\MR{2}{*}{\It{i}}
	&tjähpis	&tjáhpat		&tjáhpada		&IIIa	& ‘black’	\\%\hline
	&rusjgis	&russjgat		&russjgada	&IIIa	& ‘red’	\\\hline%%\hline
%attr=pred.pl+s
\MR{4}{*}{\It{ii}}%\MR{4}{*}{\ATTRs=\PREDs.\PLs+s}
	&nievres	&nävvre		&nievre		&Ie	& ‘bad’	\\%\hline
	&vastes	&vasste		&vaste		&Ie	& ‘ugly’	\\%\hline
	&buosjes	&buossje		&buosje		&Ie	& ‘fearless’	\\%\hline
	&fávros	&fávvro		&fávro		&Ic	& ‘attractive’	\\\hline
%attr=pred.sg, both +s
\MR{6}{*}{\It{iii}}%\MR{6}{*}{\ATTRs=\PREDs.\SGs (-s)}
	&dájges	&dájges		&dájgesa		&IIIa	& ‘cowardly’	\\%\hline
	&åvros	&åvros		&åvrosa		&IIIa	& ‘nervous’	\\%\hline
	&vuoras	&vuoras		&vuorasa		&IIIa	& ‘old’	\\%\hline
	
	&gujos	&gujos		&gudjosa		&IIIa	& ‘frozen solid’	\\%\hline
	&luvas	&luvas		&luvvasa		&IIIa	& ‘wet’	\\%\hline
	&nanos	&nanos		&nannosa		&IIIa	& ‘sturdy’	\\\hline
%attr=pred.sg, pred.pl -s
\MR{1}{*}{\It{iv}}%\parbox{100pt}{\ATTRs=\PREDs.\SGs (-s); \PREDs.\PL\ w/o -s}
	&sádnes	&sádnes		&sádna		&II	& ‘true’	\\\hline%%\hline
%pred.sg+is
\MR{2}{*}{\It{v}}%\MR{2}{*}{\ATTRs=\PREDs.\SGs+s}
	&bivvalis	&bivval		&bivvala		&IIIa	& ‘warm’ (weather)	\\%\hline
	&buoragis	&buorak		&buoraga		&IIIa	& ‘good’	\\\hline%%\hline
%attr=pred.pl +g+s
\MR{1}{*}{\It{vi}}%
	&ånegis	&ådne		&åne			&Ie	& ‘short’	\\\hline%%\hline
%pred +t
\MR{4}{*}{\It{vii}}%
	&jallga	&jallgat		&jallgada		&IIIa	& ‘flat’	\\%\hline
	&njuallga	&njuallgat		&njuallgada	&IIIa	& ‘straight’	\\%\hline
%attr=pred.pl -s
%\MR{2}{*}{\It{viii}%
	&lägga	&lieggas		&läggasa		&IIIa	& ‘warm’	\\%\hline
	&galbma	&galmas		&galbmasa	&IIIa	& ‘cold’	\\\hline
\end{tabular}
\end{table}

\FloatBarrier

As is evident from the examples in Table \ref{typcialPredAdjTable}, the attributive forms and the predicative forms correspond in a variety of ways. 
These correspondence patterns are described here\marginpar{how to best format this list, and the one on the next page?}: 

\begin{tabular}{c p{313pt}}
%\BfIt{subgroup}	& description\\
\BfIt{i}& The attributive form differs from the predicative form in the choice of stem allomorph concerning V1, consonant center, V2 and the final consonant. %\footnotemark\ 
Number marking like class IIIa nouns.\\%\hline
\BfIt{ii}& The attributive form and the plural predicative form have the same V1 and consonant center, as opposed to the singular predicative adjective; the attributive form has a stem final \It{-s}, while the predicative forms have an open final syllable. Number marking like class I nouns.\\ 
\BfIt{iii}& The attributive form and the singular predicative form are syncretic and have a stem-final \It{-s}, while the plural predicative form’s stem is also identical, but marked for plural by a final \It{-a}. Number marking like class IIIa nouns.\\ 
\BfIt{iv}& The attributive form and the singular predicative form are syncretic and have a stem-final \It{-s}, while the plural predicative form is marked by a final \It{-a} instead of the final \It{es} in the other forms. Only one example in the corpus; number marking like a class II noun.\\ 
\BfIt{v}& The bare stem in the singular predicative form, an additional final \It{-is} for the attributive form and \It{-a} for the plural predicative form. Number marking like class IIIa nouns.\\ 
\BfIt{vi}& The singular and plural predicative forms differ only in the choice of stem allomorph (in the consonant center), while the attributive form is in the ‘weak’ grade (like predicative plural) but with a stem-final \It{gis}. %\footnote{Historically, the \It{-g-} could have been a nominalizer, which was then further derived to the attributive form by the (historical) \It{-s} adjectivizer.} 
Only one example in the corpus; number marking like a class Ie noun.\\ 
\BfIt{vii}& The predicative forms have a stem-final \It{-t} or \It{-s}, which is lacking in the attributive form. The plural predicative form is marked with a final \It{-a}. In the case of \It{galbma}, the attributive form and plural predicative form have the ‘strong’ stem form, as opposed to the singular predicative form, which is ‘weak’. Number marking like class IIIa nouns. 
%\BfIt{viii}& The predicative forms have a stem final \It{-s}, which is lacking in the attributive form. The plural predicative form is marked with a final \It{-a}. Only class IVa nouns. 
\end{tabular}%\end{longtable}
%\addtocounter{table}{-1}%added because the longtable above, which is not marked as a table in the output, increased the counter {table} by one, which throws off any following actual tables

Despite the similarities with nouns described above, it is important to point out that there are a number of predicative adjectives which do \It{not} inflect for number. Moreover, this lack of number marking cannot be assigned to any specific noun class, % described in §\,\ref{nounClasses} above because the predicative forms are not marked for number, 
particularly since noun classes with similar prosodic structures exhibit clear number marking strategies. Examples of such predicative adjectives are presented in Table \vref{noPredAdjTable}. 

\begin{table}\centering
\caption{Some attributive and predicative adjective sets for which the predicative adjective does not inflect for number}\label{noPredAdjTable}
\begin{tabular}{cccl}
	&\It{attributive}	&\It{predicative}	&\\
\It{no.}&\It{adjective}	&\It{adjective}				&\It{gloss}	\\\dline
\MR{4}{*}{\It{viii}}
	&tjábba	&tjábbe		& ‘beautiful’	\\%\hline
	&guhka	&guhke		& ‘long’		\\%\hline
	&unna	&unne 		& ‘small’ (only \SGs)	\\%\hline
	&smáva	&smáve		& ‘small’ (only \PLs)	\\\hline
\MR{5}{*}{\It{ix}}
	&tjåskes	&tjåskes		& ‘cold’ (weather)	\\%\hline
	&låjes	&låjes		& ‘tame’	\\%\hline
%\MR{3}{*}{\It{x}
	&gårå	&gårå		& ‘bad’	\\%\hline
	&räkta	&räkta		& ‘correct’	\\%\hline
	&buorre	&buorre		& ‘good’	\\\hline
\end{tabular}
\end{table}
\FloatBarrier

The paradigms in Table \ref{noPredAdjTable} are divided into two sub-groupings, again based on the relationship between the attributive form and the predicative forms. They are summarized here: 

%%\begin{tabular}{c p{310pt}}
%\begin{longtable}{c p{320pt}}
%\BfIt{viii}& The attributive form ends in \It{-a}, while the predicative form ends in \It{-e}.\footnote{Cf. §\,\ref{smallADJs} for more details on \It{unna}/\It{unna} and \It{smáve}/\It{smáve}, the words for ‘small’.} \\
%\BfIt{ix}& All forms are syncretic. The first two examples have a closed final syllable, while the last three examples have an open final syllable. \\
%\end{longtable}
%%\end{tabular}
%\addtocounter{table}{-1}%added because the longtable above, which is not marked as a table in the output, increased the counter {table} by one, which throws off any following actual tables

\begin{tabular}{c p{311pt}}
%\begin{longtable}{c p{320pt}}
\BfIt{viii}& The attributive form ends in \It{-a}, while the predicative form ends in \It{-e}.\footnotemark \\
\BfIt{ix}& All forms are syncretic. The first two examples have a closed final syllable, while the last three examples have an open final syllable. \\
%\end{longtable}
\end{tabular}\footnotetext{Cf. §\,\ref{smallADJs} for more details on \It{unna}/\It{unna} and \It{smáve}/\It{smáve}, the words for ‘small’.}%\addtocounter{table}{-1}%added because the longtable above, which is not marked as a table in the output, increased the counter {table} by one, which throws off any following actual tables

The variety evident in morphological correlations between the attributive and the predicative forms indicates that there is no regular form-to-function relationship between the attributive and predicative forms. Therefore, on formal grounds, the attributive and predicative forms of these property words are assigned to different, though formally and semantically related, adjectival lexemes, as argued for in the following section (§\,\ref{notePredNounsAdjs}). 
%In fact, such a variety exists that any sort of derivation from one form to another form can be excluded. %UM: suggested change
%\FloatBarrier
%\clearpage

\section{A note on attributive and predicative adjectives}\label{notePredNounsAdjs}
%\marginpar{what is wrong with this section? where is the pagebreak?!? something to do with longtable in previous section!}
In the literature on Saami languages, a convention prevails by which predicative adjectives are treated as having more or less derivable attributive forms.\footnote{Cf. \citet[71]{Sammallahti1998}, \citet[74-76;98]{Svonni2009} and \citet[179]{Feist2010}.} 
From a historical point of view, this may be reasonable, particularly if there was a point in the history of the Saami languages at which attributive forms were derived by adding \mbox{\It{-s}} and selecting the phonologically relevant stem allomorph, and thus the attributive forms were derivable from the predicate forms. For \PS, however, there is no clear or consistent morphological relationship synchronically between attributive adjectives and the corresponding predicative adjectives, as shown above. This is particularly exemplified by the existence of more than one acceptable attributive form, (as pointed out in §\,\ref{adjectivesATTR} for the attributive adjective forms \It{guhkes}\TILDE\It{guhka} ‘long’), as well as by the existence of a number of predicative forms ending in \It{-s}, but attributive forms lacking \It{-s} (cf. pattern \It{vii} above). 
Due to cases like those illustrated by subgrouping \It{ix} in Table \ref{noPredAdjTable}, it is not clear that it is sensible to claim that \It{all} adjectives have corresponding predicative adjectives that differ at all. 
Because of the wide variety of and the inconsistencies in morphological patterns between corresponding attributive and predicative adjectives, it is ultimately more elegant to analyze these two sets of adjectives simply as semantically and etymologically related – but not morphologically derivable – adjectives. 


\section{Comparatives and superlatives}\label{compSuperlADJs}
The comparative and superlative (abbreviated here as ‘C/S’) forms of attributive and predicative adjectives are derived using suffixes. It seems that, morphosyntactically speaking, comparative and superlative forms can be derived from all adjectives, even when a semantic restriction could lexically prevent such forms from occurring; cf., e.g., \It{guäktegierdakap} ‘more pregnant’ (\hyperlink{pit090927}{pit090927.07m01s}). 

The singular C/S predicative form is always identical %syncretic with 
to the C/S attributive form, and the plural C/S predicative form is always marked by a suffix consisting of a single vowel (mostly \It{-a}). In many cases, the stem to which a C/S suffix is attached is identical %syncretic with 
to the stem of the positive plural predicative form, but a number of exceptions exist. 

Table \vref{compSuperlADJsTable} provides some example paradigms. 
To help illustrate the morphophonemic relationship to positive forms, the singular predicative adjective form is also indicated. %as a comparison in the summary in Table \vref{compSuperlADJsTable}. 
Furthermore, the paradigms are divided into subgroupings (each marked with a Roman numeral) based on suffix allomorph patterns. The third and fifth columns in Table \ref{compSuperlADJsTable} provide the singular comparative and superlative adjectives (the attributive and predicative singular forms are syncretic), respectively, while the fourth and sixth columns only indicate the suffix used to mark the plural predicative comparative and superlative adjectives, respectively. Note that there are allomorphic alternations in the superlative suffix for subgroupings \It{iii} and \It{iv}; the relevant suffixes are indicated by italics. 

%The \ATTRs\ form and the \PREDs.\SGs\ form %of a comparative or a superlative adjective 
%are never marked as such (these are therefore always homophonous), while the \PREDs.\PLs\ form features the plural suffix \It{-a}. 
%The stem allomorph chosen for a comparative or superlative adjective is typically the same as the stem allomorph found in the positive \PREDs.\PLs\ adjective; however, exceptions exist. 
\begin{table}\centering
\caption{Some comparative and superlative adjective paradigms}\label{compSuperlADJsTable}
\resizebox{\columnwidth}{!}{
\begin{tabular}{ccccccl}
	&\It{positive}		&\MC{2}{c}{\It{comparative}}	&\MC{2}{c}{\It{superlative}}	&	\\
\It{no.}&\PREDs.\SGs	&\ATTRs /\PREDs.\SGs&\PREDs.\PLs\ &\ATTRs /\PREDs.\SGs&\PREDs.\PLs\ &\It{gloss}	\\\dline
\MR{3}{*}{\It{i}}
	&nävvre		&nievre-p		&-a			&nievre-mus		&-a	& ‘bad’	\\%\hline
	&ådne		&åne-p		&-a			&åne-mus			&-a	& ‘short’	\\%\hline
	&guhke		&guhke-p		&-a			&guhke-mus		&-a	& ‘long’	\\\hline%%\hline
\MR{4}{*}{\It{ii}}
	&tjábbe		&tjábba-p		&-a			&tjábba-jmus		&-a	& ‘beautiful’	\\%\hline
	&vasste		&vasste-p		&-a			&vasste-jmus		&-a	& ‘ugly’	\\%\hline
	&gårå		&gårå-p		&-a			&gårå-jmus		&-a	& ‘bad’	\\%\hline
	&fávvro		&fávro-p		&-a			&fávro-jmus		&-a	& ‘attractive’	\\\hline%%\hline
\MR{2}{*}{\It{iii}}
	&luvas		&luvasu-p		&-o			&luvasu-\It{mos}	&-\It{bmus}-a	& ‘wet’	\\%\hline
	&garras		&garrasu-p	&-o			&garrasu-\It{mos}	&-\It{bmus}-a	& ‘hard’	\\\hline%%\hline
%låjes			&låjes-u-p		&-o			&låjes-u-mos		&-musa	& ‘tame’	\\%\hline%JW: DS provides conflicting data!
\MR{6}{*}{\It{iv}}
	&nanos		&nanosu-p	&-a			&nanosu-\It{mos}	&-\It{bmus}-a	& ‘strong’	\\%\hline
	&bivval		&bivvalu-p		&-a			&bivvalu-\It{mos}	&-\It{bmus}-a	& ‘warm’	\\%\hline
	&tjáhpat		&tjáhpadu-p	&-a			&tjáhpadu-\It{mos}	&-\It{bmus}-a	& ‘black’	\\%\hline
	&galmas		&galbmasu-p	&-a			&galbmasu-\It{mos}	&-\It{bmus}-a	& ‘cold’	\\%\hline
	&vuoras		&vuorasu-p	&-a			&vuorasu-\It{mos}	&-\It{bmus}-a	& ‘old’	\\%\hline
	&njuallgat		&njuallgadu-p	&-a			&njuallgadu-\It{mos}	&-\It{bmus}-a	& ‘correct’	\\\hline
%nävvre		&nievre-p	&-a				&nievre-mus			&-musa	&\It{bad	\\\hline
%ådne			&åne-p	&-a					&åne-mus				&-musa	&\It{short	\\\hline
%gårå			&gårå-p	&-a				&gårå-jmus			&-jmusa	&\It{bad	\\\hline
%guhke		&guhke-p	&-a				&guhke-mus			&-musa	&\It{long	\\\hline
%bivval		&bivval-up	&-a				&bivval-umos			&-ubmusa	&\It{warm	\\\hline
%tjáhpat		&tjáhpad-up	&-a				&tjáhpad-umus			&-ubmusa&\It{black	\\\hline
%galmas		&galbmas-up	&-a			&galbmas-umos			&-ubmusa	&\It{cold	\\\hline
%vuoras		&vuoras-up	&-a				&vuoras-umos			&-ubmasa	&\It{old	\\\hline
%njuallgat		&njuallgad-up	&-a			&njuallgad-umus		&-ubmusa	&\It{correct	\\\hline
%låjes
%åvros		&åvrosup				&åvrosuba	&åvrojmus			&\Red{åvrojmusa!}	&\It{nervous	\\\hline
\end{tabular}}
\end{table}

Comparative adjectives are derived in a relatively straightforward way: the suffix \It{-p}\footnote{Note that, in the current working orthography, the comparative suffix \It{-p} is written \It{-b-} when intervocalic, such as in plural predicative forms.} 
is added to an adjective root. If the root has a closed final syllable, then an epenthetic vowel \It{-u-} is inserted between the root and the suffix. In predicative position, plural is always marked by a suffix consisting of a vowel; in most cases (groups \It{i}, \It{ii} and \It{iv}), the vowel is \It{-a}, but sometime it is \It{-o} (group \It{iii}). It is not clear what determines the choice of plural suffix for comparative forms. While all forms marked by \It{-o} in the corpus have a stem final \It{-s}, not all forms with a stem final \It{-s} are marked by \It{-o} (cf. \It{nanos} ‘strong’). 

The superlative suffix has four allomorphs. For the attributive and the singular predicative forms, the allomorph \It{-mos} is chosen when the root has a closed final syllable, as in groups \It{iii} and \It{iv}. Roots with an open final syllable have either the superlative suffix allomorph \It{-mus} or \It{-jmus}; however, it is not clear what drives the selection of these latter two allomorphs. 

The allomorph \It{-bmus-} occurs whenever the resulting form has an odd number of syllables, as is the case for roots with a final odd syllable in the plural predicative form. %\marginpar{UM asks ‘why in footnote?’ JW answers: ‘phonology not directly relevant here’}
Essentially, the superlative suffix always forms the final foot of a word, and thus is the location for consonant gradation alternations. If a final, odd syllable is present (e.g., for the plural predicative form), then the \mbox{\It{-bmus-}} allomorph is chosen.\footnote{Cf. §\,\ref{prosodicDomains} on prosodic domains and §\,\ref{Cgrad} on consonant gradation.} 

Examples for C/S adjectives in attributive position can be found in \REF{compATTRADJex1} and \REF{superlATTRADJex1}, respectively. Instances for predicative usage can be found in \REF{comparingNPsEx3} and \REF{comparingNPsEx5} in §\,\ref{comparingNPs}.
\ea\label{compATTRADJex1}
\glll	bivvalup dállke\\
	bivvalu-p dállke\\
	warm-\Sc{comp} weather\BS\Sc{nom.sg}\\
%	warm-\Sc{comp}\BS\Sc{attr} weather\BS\Sc{nom.sg}\\\nopagebreak
\Transl{warmer weather}{}	\CorpusE{090926}{23m22s}
\z
\ea\label{superlATTRADJex1}%så dä lä vuorasumos saddje, dát
\glll	så dä lä vuorasumos saddje\\
	så dä lä vuorasu-mos saddje\\
	so then be\BS\Sc{3sg.prs} old-\Sc{superl} place\BS\Sc{nom.sg}\\
%	so then be\BS\Sc{3sg.prs} old-\Sc{superl}\BS\Sc{attr} place\BS\Sc{nom.sg}\\\nopagebreak
\Transl{So then it’s the oldest place}{}	\CorpusLink{0906_Ahkajavvre_a}{0906\_Ahkajavvre\_a}{123}
\z

As with positive adjectives, C/S adjectives can occur in elliptic NPs, in which case they inflect for case and number (cf. §\,\ref{ADJinHeadlessNPs}). Examples are provided in \REF{compATTRADJex2} and \REF{superlATTRADJex2}. Note that in the second example, the superlative suffix allomorph is \It{-bmus} because the adjective has an odd number of syllables.
\ea\label{compATTRADJex2}%
\glll	mån uvadav tjábbabuv\\
	mån uvada-v tjábba-b-uv\\
	\Sc{1sg.nom} kiss-\Sc{1sg.prs} beautiful-\Sc{comp}-\Sc{acc.sg}\\\nopagebreak
\Transl{I kiss the more beautiful one}{(referring to ‘beautiful girl’)}	\CorpusE{090930a}{166}
%\glll	mån ådtjov girjijd tjábbajmusajst\\%JW: this example is just too artificial sounding, even for elicitation
%	mån ådtjo-v girji-jd tjábba-jmus-ajst\\
%	\Sc{1sg.nom} receive-\Sc{1sg.prs} book-\Sc{acc.pl} beautiful-\Sc{superl}-\Sc{elat.pl}\\
%\trans	‘I receive books from the most beautiful (girls)’	\CorpusE{090930a.189}
\z
\ea\label{superlATTRADJex2}
\glll	buhtsu lin mälgadubmusin\\
	buhtsu li-n mälgadu-bmus-in\\
	reindeer\BS\Sc{3pl.nom} be-\Sc{3pl.pst} far-\Sc{superl}-\Sc{iness.sg}\\\nopagebreak
\Transl{the reindeer were farthest away}{(lit.: in the farthest one (place))}	\CorpusE{090927}{88m34s}%JW: \parbox{width}{text} added to prevent pagebreak in the middle of text
\z


\section{Comparing NP referents}\label{comparingNPs}
%\marginpar{is this the best location for this section?}%UM: yes
Predicative adjectives can be used to compare the referents of nominal phrases. If both referents are considered equal concerning the characteristic of comparison, then the NP of comparison is the subject of a copula predicate which is complemented by a construction using the numeral \It{akta} ‘one’ and the relevant predicative adjective, while the NP of reference is in the comitative case. An example is provided in \REF{comparingNPsEx1}. In such constructions, \It{akta} can be shortened to \It{akt}. 
\ea\label{comparingNPsEx1}
\glll	Svenna lä akta vuoras Ingerijn\\
	Svenna lä akta vuoras Ingeri-jn\\
	Sven\BS\Sc{nom.sg} be\BS\Sc{3sg.prs} one old\BS\Sc{sg} Inger-\Sc{com.sg}\\\nopagebreak
\Transl{Sven is as old as Inger}{(lit.: Sven is one old with Inger)}	\CorpusE{110331b}{135}%JW: \mbox added b/c otherwise the automatic spacing was too spaced out
\z
Alternatively, both referents can be included in the subject NP, as in \REF{comparingNPsEx2}. %\marginpar{\REF{comparingNPsEx2} from an elicitation session, but uttered spontaneously and off-topic - still mark as elicitation?}%UM: no “overheard expl.”%JW: not sure what UM means by this.
\ea\label{comparingNPsEx2}
\glll	måj lin akta vuorasa\\
	måj li-n akta vuoras-a\\
	\Sc{1du.nom} be-\Sc{1du.prs} one old-\Sc{pl}\\\nopagebreak
\Transl{we two are the same age}{(lit.: we are one old)}	\Corpus{080621}{65m00s}
\z

When comparing two referents that are not considered equal, the NP of comparison is the subject of a copula predicate which is complemented by a comparative predicative adjective and the NP of reference in the elative case, as in \REF{comparingNPsEx3}. %and \REF{comparingNPsEx4}.
\ea\label{comparingNPsEx3}
\glll	Inger lä stuorap várest\\
	Inger lä stuora-p váre-st\\
	Inger\BS\Sc{nom.sg} be\BS\Sc{3sg.prs} big-\Sc{comp}\BS\Sc{sg} mountain-\Sc{elat.sg}\\\nopagebreak
\Transl{Inger is bigger than a mountain}{}	\CorpusE{110331b}{144}
%\trans	‘Inger is bigger than a mountain’ (lit.: Inger is bigger from a mountain)	\CorpusE{110331b.144}
%\ea\label{comparingNPsEx4}
%\glll	dat lij Svenna ja Inger majsste liv stuorab\\
%	dat li-j Svenna ja Inger maj-sste li-v stuora-b\\
%	\Sc{dem.dist}\BS\Sc{nom.sg} be-\Sc{3sg.pst} Sven\BS\Sc{nom.sg} and Inger\BS\Sc{nom.sg} who-\Sc{elat.pl} be-\Sc{1sg.prs} big-\Sc{comp}\\
%\trans	‘It was Sven and Inger who I am bigger than’	\CorpusE{110331b.093}
\z

To indicate that a referent is the most extreme concerning the characteristic of comparison (at least within the group being compared), the relevant NP is the subject of a copula predicate which is complemented by the superlative predicative adjective. The quantifier \It{gajk} ‘all’ can be added for emphasis,\footnote{The construction \It{gajk vuorasumos} ‘absolute oldest’ in \REF{comparingNPsEx5} is possibly a calque based on North Germanic; cf. Swedish \It{allra äldst} ‘absolute oldest. In both cases, the adverbial modifier is based on the word for ‘all’ and precedes the superlative adjective.} 
as in \REF{comparingNPsEx5}.
%så dä, dä lä vuorasamus saddje dát
%dát lä vanj dä gajk vuorasamos ääh dáhkaduvvum
\ea\label{comparingNPsEx5}
\glll	dát lä vanj dä gajk vuorasumos dágaduvvum\\
	d-á-t lä vanj dä gajk vuorasu-mos dága-duvvu-m\\
	\Sc{dem}-\Sc{prox}-\Sc{nom.sg} be\BS\Sc{3sg.prs} probably then all old-\Sc{superl}\BS\Sc{sg} make-\Sc{pass}-\Sc{prf}\\\nopagebreak
\Transl{This was probably the absolute oldest made}{}\\	\CorpusLink{0906_Ahkajavvre_a}{0906\_Ahkajavvre\_a}{120}
\z





\section{Restrictions on \It{smáva} and \It{unna} ‘small’}\label{smallADJs}
%\minisection[Restrictions on \It{smáv-} and  \It{unn-} ‘small’]{Restrictions on the adjective stems \It{smáv-} and  \It{unn-} ‘small’}\label{smallADJs}
%The adjective paradigms for the two \PS\ words for ‘small, little’ are in this table. %\ref{ADJclassVcTable}. 
The paradigms for the two \PS\ words for ‘small’ provided in Table \vref{noPredAdjTable} do not sufficiently indicate the restrictions placed on these specific adjectives. The root \It{smáv-}, a North Germanic loan word, only modifies plural nouns, while \mbox{\It{unn-},} %JW: \mbox to prevent line break before comma
the native word \citep[265]{Sammallahti1998}, usually only modifies singular nouns. 
Since no other \PS\ adjectives underlie such a restriction, while the Swedish adjective \It{små} (cognate with the source of \It{smáv-}) is also restricted to modifying plural nouns,\footnote{The Swedish adjective stem \It{lite-} is used for singular nouns.} 
it seems that this syntactic restriction was also borrowed. While a few examples exist in the corpus of \It{unn-} modifying a plural noun, \It{smáv-} is the preferred item and much more frequent in the corpus.%\marginpar{token count after UM’s suggestion}
\footnote{A corpus search (including elicitation sessions) resulted in 1 token of \It{unn-} and 10 tokens of \It{smáv-} modifying a plural noun (carried out on 12\superS{th} November 2012).} 
Examples are provided in \REF{smallADJex1} through \REF{smallADJex4}.
\ea\label{smallADJex1}
\glll	jå, månnå aj mujhtav gu liv unna mánátj\\
	jå månnå aj mujhta-v gu li-v unna máná-tj\\
	yes \Sc{1sg.nom} also remember-\Sc{1sg.prs} when be-\Sc{1sg.pst} small child-\Sc{dim}\BS\Sc{nom.sg}\\\nopagebreak
\Transl{yes, I also remember when I was a small child}{}	\Corpus{080924}{632}
%\glll	unna bena\\
%	unn-a bena\\
%	small-\Sc{attr} dog\BS\Sc{nom.sg}\\
%\trans	‘small dog’	\CorpusE{080819DSa.121}
\z
\ea\label{smallADJex2}
\glll	bena lä unne\\
	bena lä unn-e\\
	dog\BS\Sc{nom.sg} be\BS\Sc{3sg.prs} small\BS\Sc{sg}\\\nopagebreak
\Transl{the dog is small}{}	\CorpusE{080819a}{126}
\z
\ea\label{smallADJex3}
\glll	ber akta bällge, ja smáva gisstá, dá\\
	ber akta bällge ja smáva gisstá d-á\\
	only one thumb\BS\Sc{nom.sg} and small glove\BS\Sc{nom.pl} \Sc{dem}-\Sc{prox}\BS\Sc{nom.pl}\\\nopagebreak
\Transl{only one thumb, and small gloves, these here}{}	\CorpusLink{080708_Session08}{080708\_Session08}{031}
%\trans \parbox{\columnwidth}{‘only one thumb, and small gloves, these here’} \Corpus{080708DS\_Session08.031}
%\glll	smáva bednaga\\
%	smáv-a bednag-a\\
%	small-\Sc{attr} dog-\Sc{nom.pl}\\
%\trans	‘small dogs’	\CorpusE{080819DSa.130}
\z
\ea\label{smallADJex4}
\glll	bednaga lä smáve\\
	bednag-a lä smáve\\
	dog-\Sc{nom.pl} be\BS\Sc{3pl.prs} small\BS\Sc{pl}\\\nopagebreak
\Transl{dogs are small}{}	\CorpusE{080819a}{129}
\z

In non-elicited tokens from the corpus, nouns modified by the adjective \It{unna} are always diminutive nouns, as in \REF{smallADJex1} above and in \REF{smallADJex5} and \REF{smallADJex6} below.
\ea\label{smallADJex5}
\glll	dát lä dåpe sin, unna dåpátja sin\\
	d-á-t lä dåpe sin unna dåpá-tj-a sin\\
	\Sc{dem}-\Sc{prox}-\Sc{nom.sg} be\BS\Sc{3sg.prs} house\BS\Sc{gen.sg} in small house-\Sc{dim}-\Sc{gen.sg} in\\\nopagebreak
\Transl{this is in the house, in the little house}{}	\Corpus{100310b}{070}
\z
\ea\label{smallADJex6}
\glll	ja danne vuojdniv unna jåŋåtjav\\
	ja danne vuojdni-v unna jåŋå-tj-av\\
	and there see-\Sc{1sg.pst} small lingonberry-\Sc{dim}-\Sc{acc.sg}\\\nopagebreak
\Transl{and I saw a little lingonberry there}{}	\Corpus{100404}{353}
\z
%\footntoe{In North Saami, 


\section{Quantifiers}\label{quantifiers}\index{quantifiers}\index{parts of speech! quantifiers}
While quantifiers are semantically similar to numerals, formally they are adjectives. % because they inflect for case and number when functioning as the head of an NP, and have a nominalized form in predicative position. % (and not for case). %unlike numerals in that they refer to a general or approximate quantity. 
Quantifiers include %\footnote{Here, the attributive form is provided.} 
\It{edna} ‘many, much’, \It{gajk} ‘all’, \It{omasse} ‘all kinds of’, \It{färt} ‘every’, \It{nagin} ‘some’, \It{såmes} ‘some’, \It{suhta} ‘some, several’ and \It{binna} ‘a bit, a little’. %and \It{aktak} ‘no, none’.
Some examples of quantifiers in attributive APs are provided in \REF{quantEx1} %\footnote{In \REF{quantEx1}, the phrase \It{guhka juolgagijd} ‘long-leggers’ refers to moose.} 
through \REF{quantEx4}.
\ea\label{quantEx1}%edna
\glll	vuojdna edna guhkajuolgagijd?\\
	vuojdna edna guhka-juolga-gi-jd\\
	see\BS\Sc{2sg.pst} many long-leg-\Sc{nmlz}-\Sc{acc.pl}\\\nopagebreak
\Transl{did you see many long-leggers?}{(referring to moose)}	\Corpus{080924}{007}
\z
\ea\label{quantEx2a}%färrt
\glll	färt bäjjve mij bårojmä gulijd\\
	färt bäjjve mij båro-jmä guli-jd\\
	every day\BS\Sc{nom.sg} \Sc{1pl.nom} eat-\Sc{1pl.pst} fish-\Sc{acc.pl}\\\nopagebreak
\Transl{every day we ate fish}{}	\Corpus{100310b}{024}
\z
\ea\label{quantEx2b}%nagin
\glll	ja dä vállda nijbev ja tjuolast nagin rägijt\\
	ja dä vállda nijbe-v ja tjuolast nagin rägi-jt\\
	and then take\BS\Sc{3sg.prs} knife-\Sc{acc.sg} and cut\BS\Sc{3sg.prs} some hole-\Sc{acc.pl}\\\nopagebreak
\Transl{and then one takes a knife and cuts some holes}{}	\Corpus{100404}{098}
\z
\ea\label{quantEx3}%gajk
\glll	gajk almatja lä Árjepluoven\\
	gajk almatj-a lä Árjepluove-n\\
	all person-\Sc{nom.pl} be\BS\Sc{3pl.prs} Arjeplog-\Sc{iness.sg}\\\nopagebreak
\Transl{all people are in Arjeplog}{}	\Corpus{100310b}{132}
\z
\ea\label{quantEx4}%binna
\glll	muvne lä binna vuopta\\
	muvne lä binna vuopta\\
	\Sc{1sg.iness} be\BS\Sc{3sg.prs} little.bit hair\BS\Sc{nom.pl}\\\nopagebreak
\Transl{I have a little hair}{}	\CorpusE{080926}{02m05s}
\z

%\subsubsection{Case and number marking on quantifiers}\label{caseNumberOnQuant}
%As shown in the examples \REF{quantEx1} through \REF{noneEx2} above, when a quantifier modifies a noun in attributive position, it does not inflect for case or number. %, as illustrated by \REF{quantifierEx1} and \REF{quantifierEx2}.
%%\ea\label{quantifierEx1}
%\glll	ja mån lev bårråm edna gulijd\\
%	ja mån le-v bårrå-m edna guli-jd\\
%	and \Sc{1sg.nom} be-\Sc{1sg.prs} eat-\Sc{prf} much fish-\Sc{acc.pl}\\
%\trans	‘and I have eaten many fish’	\Corpus{100310b.023}
%\ea\label{quantifierEx2}
%\glll	gajk almatja lä Árjepluoven\\
%	gajk almatj-a lä Árjepluove-n\\
%	all person-\Sc{nom.pl} be\BS\Sc{3pl.prs} Arjeplog-\Sc{iness.sg}\\
%\trans	‘All people are in Arjeplog’	\Corpus{100310b.132}
%%\glll	muvne lä binna vuopta\\
%%	muvne lä binna vuopta\\
%%	\Sc{1sg.iness} be\BS\Sc{3pl.prs} a\_bit hair\BS\Sc{nom.pl}\\
%%\trans	‘I have a little bit of hair’	\CorpusE{080926.02m05s}
%\z
As with any attributive adjectives, quantifiers do not inflect for case or number, as evidenced by the examples above. 
Note, however, that \It{gajk} ‘all’ can optionally be marked for plural in attributive position by adding the suffix \It{-a}, as shown in \REF{quantifierEx3}.
\ea\label{quantifierEx3}
\glll	mån vaddav gajka buhtsujda biebmov\\
	mån vadda-v gajk-a buhtsu-jda biebmo-v\\
	\Sc{1sg.nom} give-\Sc{1sg.prs} all-\Sc{pl} reindeer-\Sc{ill.pl} food-\Sc{acc.sg}\\\nopagebreak
\Transl{I give all the reindeer food}{}	\CorpusE{110413b}{173}
\z

%\subsubsection{Quantifiers as the head of an NP}\label{quantNP}
However, when a quantifier is in an elliptic NP, %the head of an NP, 
it inflects for case and number (just as with other attributive adjectives). This is illustrated by \It{enabu} ‘more’ in \REF{quantifierEx6}, %which is inflected for \ACCs.\SGs\ case, and 
by \It{gajk} ‘all’ in \REF{quantifierEx7}, %in which \It{gajk} ‘all’ is inflected for \Sc{acc.pl}.
and by \It{nagin} ‘some’ in \REF{quantifierEx8}
%\ea\label{quantifierEx5}%JW: \marginpar{or is \Bf{-jd} \Sc{advz}, like viesov buoragijt}
%\glll	galgav enabujd?\\
%	galga-v ena-bu-jd\\
%	shall-\Sc{1sg.prs} some-\Sc{comp}-\Sc{acc.pl}\\
%\trans	‘shall I (say) more?’	\Corpus{100310b.128}
\ea\label{quantifierEx6}
\glll	galgav enabuv biejat?\\
	galga-v ena-b-uv bieja-t\\
	shall-\Sc{1sg.prs} much-\Sc{comp}-\Sc{acc.sg} put-\Sc{inf}\\\nopagebreak
\Transl{shall I put in more?}{}	\Corpus{090519}{156}
\z
\ea\label{quantifierEx7}
\glll	mana tjasskit dajd åjvijd ja gajkajd duhku\\
	mana tjasski-t d-a-jd åjvi-jd ja gajk-ajd duhku\\
	go\BS\Sc{2sg.imp} throw-\Sc{inf} \Sc{dem}-\Sc{dist}-\Sc{acc.pl} head-\Sc{acc.pl} and all-\Sc{acc.pl} over.there\\\nopagebreak
\Transl{go throw those heads and all that over there}{}	\Corpus{080909}{146}
\z
\ea\label{quantifierEx8}
\glll	hålå naginav mav galgav hållåt\\
	hålå nagina-v ma-v galga-v hållå-t\\
	say\BS\Sc{2sg.imp} some-\Sc{acc.sg} \Sc{rel}-\Sc{acc.sg} shall-\Sc{1sg.prs} say-\Sc{inf}\\\nopagebreak
\Transl{say something that I should say}{}	\Corpus{100304}{001}
\z

The quantifier \It{aktak} ‘none, any’ is used to emphasize a negated clause. 
It seems to be composed of the numeral \It{akta} ‘one’ and the suffix \It{-k}, which is a nominalizer in other cases; however, as illustrated by the examples in \REF{noneEx1} and \REF{noneEx2}, it heads an attributive AP and does not inflect for case and number, unless it is in an elliptic NP, %the head of an NP, 
as in \REF{noneEx3}. 
It is thus considered an adjective. 
%\footnote{Another possible analysis is that the final \It{-k} is an \Red{NMLZ!!}, similar to \It{ednak} ‘much’ the example in \REF{quantifierEx4}.} 
\ea\label{noneEx1}
\glll	muvne ij lä aktak vuopta\\
	muvne ij lä aktak vuopta\\
	\Sc{1sg.iness} \Sc{neg}\BS\Sc{3sg.prs} be\BS\Sc{conneg} none hair\BS\Sc{nom.pl}\\\nopagebreak
\Transl{I don’t have any hair}{(i.e.: I don’t have a single hair.)}	\CorpusE{080926}{02m02s}
\z
\ea\label{noneEx2}
\glll	gu itjij almatj dåbdå aktak almatjid\\
	gu itji-j almatj dåbdå aktak almatji-jd\\
	when \Sc{neg}-\Sc{3sg.pst} person\BS\Sc{nom.sg} know\BS\Sc{conneg} none person-\Sc{acc.pl}\\\nopagebreak
\Transl{if one didn’t know any people}{}	\Corpus{080924}{342}
\z
%However, it can be left out without substantially altering the meaning of such sentences. %\It{aktak} is based on the numeral \It{akta} ‘one’.\footnote{Another possible analysis is that the final \It{-k} is an adverbializer, similar to \It{ednak} ‘much’ the example in \REF{quantifierEx4}.}
%\It{aktak} also inflects for case and number when heading an NP, as shown in \REF{noneEx3}.
\ea\label{noneEx3}
\glll	itjij almatj åbbå hålå aktagav\\
	itji-j almatj åbbå hålå aktag-av\\
	\Sc{neg}-\Sc{3sg.pst} person\BS\Sc{nom.sg} at.all say\BS\Sc{conneg} none-\Sc{acc.sg}\\\nopagebreak
\Transl{one didn’t say anything at all}{}	\Corpus{080924}{354}
\z

Concerning the status of corresponding predicative quantifiers, there is not enough data in the corpus to come to a certain conclusion. However, at least the attribute adjective \It{edna} ‘many, much’ corresponds to the predicative adjective form \It{ednak}; 
this is illustrated by \REF{quantifierEx4a} and \REF{quantifierEx4b}. This indicates that attributive and predicative forms of quantifiers also differ in form, just as with other attributive and predicative adjective sets. 
\ea\label{quantifierEx4a}
\glll	bärrgo lä ednak.\\
	bärrgo lä ednak\\
	meat\BS\Sc{nom.sg} be\BS\Sc{3sg.prs} much\BS\Sc{sg}\\
%	meat\BS\Sc{nom.sg} be\BS\Sc{3sg.prs} much-\Sc{nmlz}\BS\Sc{nom.sg}\\\nopagebreak
\Transl{There is much meat}{(lit.: meat is much)}	\CorpusE{090926}{113}
\z
\ea\label{quantifierEx4b}
\glll	biergo bijta lä ednaga\\
	biergo bijta lä ednag-a\\
	meat\BS\Sc{gen.sg} piece\BS\Sc{nom.pl} be\BS\Sc{3pl.prs} much-\Sc{pl}\\
%	meat\BS\Sc{gen.sg} piece\BS\Sc{nom.pl} be\BS\Sc{3pl.prs} much-\Sc{nmlz}-\Sc{nom.pl}\\\nopagebreak
\Transl{There are many pieces of meat}{(lit.: meat pieces are many)}	\CorpusE{090926}{114}
\z
%\ea\label{quantifierEx5}%JW: this is not predicative, but NMLZed anyway! why? because V-final?
%\glll	ednagav lä dáhkam dan Áhkabákten\\
%	edna-g-av lä dáhka-m da-n Áhkabákte-n\\
%	much-\Sc{nmlz}-\Sc{acc.sg} be\BS\Sc{2sg.prs} do-\Sc{prf} \Sc{dem.dist}-\Sc{iness.sg} Áhkabákkte-\Sc{iness.sg}\\
%\trans	‘you have done much in Áhkabákkte.’	\Corpus{080924.402}
%\z
%Similarly, \It{binna} ‘a little’ is nominalized using the \DIM\ suffix \It{-tj} in predicative position in \REF{quantifierEx6}.
%%\ea\label{quantifierEx6}%not predicative, but ACC.PL!!
%\glll	ja dä bájjkam edne gábmagij sisa ja nubbe gábmaga sisa aj binnatjijd\\
%	ja dä bájjka-m edne gábmagi-j sisa ja nubbe gábmag-a sisa aj binna-tj-ijd\\
%	and then poop-\Sc{prf} mother\BS\Sc{gen.sg} into and other shoe-\Sc{gen.sg} into also a\_little-\Sc{dim}-\Sc{acc.pl}\\
%\trans	‘and then pooped into mother’s shoes, and also into the other shoe a little.’	\Corpus{080924.402}
%\z



\section{Demonstratives}\label{demonstratives}
%Note that demonstrative pronouns are identical to demonstrative adjectives modifying a noun; cf. §\,\ref{demonstrativeAdjectives}.
%\marginpar{should this perhaps be part of section on demonstrative pronouns?}
Demonstratives %described in §\,\ref{demonstrativePronouns} and inventoried in Table \vref{DemPronTable} on page \pageref{DemPronTable} 
modify a noun phrase by further specifying the head noun concerning the distance of the referent relative to the speaker. Just as with demonstrative pronouns, the corpus data indicates that there is a three-way distinction between referents close to the speaker (proximal), those away from the speaker (distal), and those particularly far away (remote). 
Indeed, they are identical in form with the demonstrative pronouns listed in Table \vref{DemProTable} in the section on demonstrative pronouns (§\,\ref{demonstrativePronouns}), and are therefore not listed separately here. 
Unlike adjectives, demonstratives always agree with the noun they modify in number (singular and plural, but not dual) and in case. %, and indicate distance of the referent relative to the speaker (just as when they are demonstrative pronouns). %The demonstrative adjectives are identical to the demonstrative pronouns described in §\,\ref{demonstrativePronouns} and inventoried in Table \vref{DemPronTable} on page \pageref{DemPronTable}. 
Examples of demonstratives % \It{dajd} 
are provided in \REF{demonstrativeAdjectivesEx1} through \REF{demonstrativeAdjectivesEx3}.
\ea\label{demonstrativeAdjectivesEx1}
\glll	dajd gulijd giesijmä tjielkajn dik\\
	d-a-jd guli-jd giesi-jmä tjielka-jn dik\\
	\Sc{dem}-\Sc{dist}-\Sc{acc.pl} fish-\Sc{acc.pl} pull-\Sc{1pl.pst} sled-\Sc{com.sg} to.here\\\nopagebreak
\Transl{we pulled those fish here with a sled}{}	\CorpusLink{0906_Ahkajavvre_a}{0906\_Ahkajavvre\_a}{043}
\z
\ea\label{demonstrativeAdjectivesEx2}
\glll	gu lijmä vuodjam dajna traktorijna ‘Grållåjn’\\
	gu li-jmä vuodja-m d-a-jna traktor-ijna Grållå-jn\\
	when be-\Sc{1pl.pst} drive-\Sc{prf} \Sc{dem}-\Sc{dist}-\Sc{com.sg} tractor-\Sc{com.sg} Grålle-\Sc{com.sg}\\\nopagebreak
\Transl{when we had driven that tractor ‘Grålle’}{}	\Corpus{090702}{287}
\z
\ea\label{demonstrativeAdjectivesEx3}
\glll	men dut biehtse, men ånekatj ja gassak\\
	men d-u-t biehtse men åneka-tj ja gassa-k\\
	but \Sc{dem}-\Sc{rmt}-\Sc{nom.sg} pine\BS\Sc{nom.sg} but short-\Sc{dim} and thick-\Sc{nmlz}\\\nopagebreak
\Transl{but yonder pine tree, how short and thick!}{}	\Corpus{090519}{284}
\z




\section{Numerals}\label{numerals}\index{numerals}\index{parts of speech!numerals}
Numerals in \PS\ form a closed class and a distinct closed sub-class of adjectives. Syntactically, they are adjectives because they head an adjectival phrase; however, morphologically, they differ from other adjectives by never inflecting (neither for number in predicative APs, nor for case and number in ellipsis constructions). 
%: in attributive position, they can modify a noun within a nominal phrase and do not inflect; in predicative position, they are found as the complement of a copula modifying the subject NP. 
Furthermore, numerals do not consist of attributive/predicative sets differing in form. %, nor do they ever inflect (not in predicative position for number, nor in elliptical constructions for case and number). %In other words, to the extent that, they do not feature corresponding sets of distinct attributive and predicative forms, and in predicative position, they do not inflect for number; 
Instead, numerals are consistent in form, regardless of being in attributive or predicative position. %\footnote{Quantifiers are similar to numerals semantically in that they also quantify the referent of the noun they modify; however, quantifiers behave morphosyntactically like adjectives, and are thus discussed in §\,\ref{quantifiers} in the chapter on adjectives.} 
%Finally, attributive numerals in elliptic constructions in which the head of an NP is not realized remain uninflected for case or number, unlike other attributive adjectives. %, while quantifiers inflect for case and number. % as nouns do. %
%\marginpar{Difference between Nums and other Quants (morphosyntactic), whether/how they form one or two closed class(es)}
%Numerals and quantifiers are described in the following two sections, respectively. %The final section (\ref{caseNumberOnNums}) discusses case and number marking on 
%In the following, §\,\ref{numerals} describes numerals, and §\,\ref{quantifiers} deals with quantifiers.

%\section{Numerals}\label{numerals}\index{numerals}\index{parts of speech!numerals}
\PS\ numerals form a decimal system consisting of the basic numerals for the numbers one through ten, hundred and thousand. All other numerals are compounds based on these basic terms, with the exception of \It{nolla} ‘zero’. Basic and complex numerals are dealt with in §\,\ref{basicNums} and §\,\ref{complexNums}, respectively; the derivation of ordinal numerals is described in §\,\ref{ordinalNums}. %Zero is also

\subsection{Basic numerals}\label{basicNums}
The basic numerals for the numbers one through ten in \PS\ are reconstructable native Saamic numerals, and \It{tjuohte} ‘hundred’ is at least from Proto-Saami.\footnote{\citet[234-235]{Sammallahti1998} indicates that \It{tjuohte} ‘hundred’ was originally a borrowing from Proto-Indo-European into Proto-Finno-Ugric.} 
The numerals \It{nolla} ‘zero’ and \It{tuvsan} ‘thousand’ are likely more recent borrowings, although it is not entirely clear whether they are from North Germanic or \marginpar{BibTeX needs work on website ‘Álgu’ which is output as \cite{alguWebsite} (see footnote)}Finnic.\footnote{The entries for the numerals ‘zero’ and ‘thousand’ in \citet{alguWebsite} only provide etymologies for North Saami and Inari Saami; however, while Finnic is unquestionably a contact language for North Saami and Inari Saami, North Germanic is a recent contact language for \PS, and therefore also a potential source for these two numerals; cf. Swedish \It{nolla} ‘zero’ and \It{tusan} ‘thousand’. } 
%Cf. Swedish \It{nolla} ‘zero’ and \It{tusan} ‘thousand’. To my knowledge, no etymology exists concerning } 
These basic cardinal numerals are listed on the left side of Table \vref{basicNumsTable}.
\begin{table}\centering
\caption{Cardinal and ordinal numerals}\label{basicNumsTable}
\begin{tabular}{ y{35pt}  x{70pt} c x{70pt}  p{35pt} }%\cline{2-2}\cline{4-4}%\hline
\MC{1}{c}{}&\It{cardinal}&&\It{ordinal}	&\MC{1}{c}{} \\\hline%\cline{1-2}\cline{4-5}%%\cline{1-2}\cline{4-5}%\dline
0	&nolla		&&-			&\\%\cline{1-2}\cline{4-5}%\hline
1	&akkta		&&vuostas	&1\superS{st}\\%%\cline{1-2}\cline{4-5}%\hline
	&			&&aktát		&{\footnotesize\It{n}}1\superS{st}\\%\cline{1-2}\cline{4-5}%\hline
2	&guäkte		&&mubbe		&2\superS{nd}	\\%%\cline{1-2}\cline{4-5}%\hline
	&			&&guoktát		&{\footnotesize\It{n}}2\superS{nd}\\%\cline{1-2}\cline{4-5}%\hline
3	&gålbmå		&&gålmát		&\Nth3\superS{rd}	\\%\cline{1-2}\cline{4-5}%\hline
4	&nällje		&&nielját		&\Nth4\superS{th}	\\%\cline{1-2}\cline{4-5}%\hline
5	&vihta		&&vidát		&\Nth5\superS{th}	\\%\cline{1-2}\cline{4-5}%\hline
6	&guhta		&&gudát		&\Nth6\superS{th}	\\%\cline{1-2}\cline{4-5}%\hline
7	&gietjav		&&giehtjet		&\Nth7\superS{th}	\\%\cline{1-2}\cline{4-5}%\hline
8	&gakktse		&&gáktsát		&\Nth8\superS{th} 	\\%\cline{1-2}\cline{4-5}%\hline
9	&åktse		&&åktsát		&\Nth9\superS{th}	\\%\cline{1-2}\cline{4-5}%\hline
10	&lågev		&&lågát		&\Nth10\superS{th}	\\%\cline{1-2}\cline{4-5}%\hline
100	&tjuohte		&&n.a.		&100\superS{th}	\\%\cline{1-2}\cline{4-5}%\hline
1000	&tuvsan		&&n.a.		&1000\superS{th}	\\%\cline{1-2}\cline{4-5}%\hline
\hline\end{tabular}
\end{table}

\subsubsection{Derivation and suppletion in ordinal numerals}\label{ordinalNums}
In general, ordinal numerals, which are listed on the right side of Table \vref{basicNumsTable}, can be derived from the corresponding cardinal numeral by replacing the vowel in V2 position and any final consonant with the suffix \It{-át} (and its allomorph \It{-et} in \It{giehtjet} ‘seventh’).  In addition, the weak stem allomorph is selected and umlaut of V1 occurs, if applicable (cf. §\,\ref{morphophonology} on stem allomorphy). %and subjecting the stem to the consonant gradation and/or umlaut, when relevant. 
The ordinal numerals corresponding to \It{tjuohte} ‘hundred’ and \It{tuvsan} ‘thousand’ are not attested in the corpus. 

However, there are exceptions. First, the ordinals \It{vuostas} ‘first’ and \It{mubbe} ‘second’ are suppletive forms compared to the corresponding cardinal numerals \It{akkta} ‘one’ and \It{guäkte} ‘two’.\footnote{Note that \It{vuostas} ‘first’ and \It{mubbe} ‘second’ are also reconstructable to at least Proto-Saami \citep[257;268]{Sammallahti1998}.} 
These two ordinals are used exclusively for the single-digit numbers ‘first’ and ‘second’; any ordinal numeral referring to a number of two or more digits uses a form derived from the cardinal numeral, as described above. This is illustrated in Table \vref{firstSecondOrdinals}.
\begin{table}\centering
\caption{Suppletive and derived ordinal numerals}\label{firstSecondOrdinals}
\begin{tabular}{c c cc c}%\hline
	&\It{cardinal}			&		&			&\It{ordinal}\\\hline
1	&\Bf{akkta}			&\ARROW&1\superS{st}	&\Bf{vuostas}		\\
11	&akta-låg-\Bf{akkta}		&\ARROW&11\superS{th}	&akta-låg-\Bf{aktát}	\\
21	&guäkte-låg-\Bf{akkta}	&\ARROW&21\superS{st}	&guäkte-låg-\Bf{aktát}\\\dline
2	&\Bf{guäkte}			&\ARROW&2\superS{nd}	&\Bf{mubbe}		\\
12	&akta-låg-\Bf{guäkte}	&\ARROW&12\superS{th}	&akta-låg-\Bf{guoktát}	\\
22	&guäkte-låg-\Bf{guäkte}	&\ARROW&22\superS{nd}	&guäkte-låg-\Bf{guoktát}\\\hline
%\begin{tabular}{c c | c c || c c | c c}\hline
%1	&akkta		&1\superS{st}	&vuostas		& 2	&guäkte			& 2\superS{nd}	& mubbe	\\\hline
%11	&akta-låg-akkta	&11\superS{th}	&akta-låg-aktát	& 12	&akta-låg-guäkte	& 12\superS{th}	& akta-låg-guoktát	\\\hline
\end{tabular}
\end{table}

Second, the cardinal numeral \It{giehtjav} ‘seven’ differs in the final two segments from the ordinal numeral \It{gietjet} ‘seventh’ (i.e., \It{-av} and \It{-et}). 

\subsection{Complex numerals}\label{complexNums}
Any numerals other than those listed in Table \vref{basicNumsTable} are complex numerals formed by combining the basic numerals. 
Multiples of ten are composed of the relevant cardinal numeral followed by \It{lågev} ‘ten’; examples are provided in Figure \vref{multiplesTen}. %\It{guäktelågev} ‘twenty’, \It{gålbmålågev} ‘thirty’, \It{nälljelågev} ‘forty’, etc. 
\begin{figure}\centering
\begin{tabular}{c| c| c c}
\It{guäkte-lågev}	&\It{gålbmå-lågev}	&\It{nällje-lågev} &\MR{2}{*}{\It{etc.}}\\
two-ten 		&three-ten			&four-ten\\
20			& 30				& 40\\
%’twenty’		& ‘thirty’			& ‘forty’\\
\end{tabular}
\caption{Multiples of ten}\label{multiplesTen}
\end{figure}

Note that \It{lågev} is often shortened to \It{låg} in fast speech, as in \REF{complexNumEx1}. 
\ea\label{complexNumEx1}
\glll	… gokt lij dánne giehtjavlåg jage maŋŋus\\
	{} gokt li-j dánne giehtjav-låg jage maŋŋus\\
	{} how be-\Sc{3sg.pst} here seven-ten year\BS\Sc{nom.pl} ago\\\nopagebreak
\Transl{…how it was here seventy years ago}{}	\CorpusLink{0906_Ahkajavvre_a}{0906\_Ahkajavvre\_a}{001}
\z

There are two ways to compose two-digit numerals that are not multiples of ten. 
One method appends the relevant numeral representing the ‘ones-digit’ to the multiple of ten, while \It{lågev} ‘ten’ is shortened to \It{låg}. 
This is illustrated in Figure \vref{twoDigitNumsA},  %\It{aktalågakta} ‘eleven’ (literally ‘one-ten-one’), \It{gålbmålågguhta} ‘thirtysix’ (literally ‘three-ten-six’), \It{åktselåkgakktse} ‘ninetyeight’. 
with examples of two-digit numerals from the corpus presented in \REF{complexNumEx2} and \REF{complexNumEx3}.
\begin{figure}\centering
\begin{tabular}{c| c| c c}
\It{akta-låg-akkta}	&\It{gålbmå-låg-guhta}&\It{åktse-låg-gakktse} &\MR{2}{*}{\It{etc.}}\\
one-ten-one	&three-ten-six		&nine-ten-eight\\
11			& 36				& 98\\
%’twenty’		& ‘thirty’			& ‘forty’\\
\end{tabular}
\caption{Two-digit numerals, method A}\label{twoDigitNumsA}
\end{figure}
\ea\label{complexNumEx3}
\glll	sån lä gakktselåggiehtjav jage\\
	sån lä gakktse-låg-giehtjav jage\\
	\Sc{3sg.nom} be\BS\Sc{3sg.prs} eight-ten-seven year\BS\Sc{nom.pl}\\\nopagebreak
\Transl{she is eighty-seven years old}{}	\Corpus{100310b}{146}
\z
\ea\label{complexNumEx2}
\glll	dä lij del tjuojgadam ja gåddam nälljalåkgakktse stalpe sájtejna\\
	dä li-j del tjuojgada-m ja gådda-m nällja-låk-gakktse stalpe sájte-jna\\
	then be-\Sc{3sg.pst} then ski-\Sc{prf} and slay-\Sc{prf} four-ten-eight wolf\BS\Sc{gen.pl} spear-\Sc{com.sg} \\\nopagebreak
\Transl{then he skied and slew forty-eight wolves with a spear}{}	\CorpusLink{0906_Ahkajavvre_a}{0906\_Ahkajavvre\_a}{088-089}
\z

Alternatively, complex numerals may be formed phrasally. According to this strategy, the ‘ones-digit’ precedes a postpositional phrase headed by the postposition \It{nanne}\footnote{Note that \It{nanne} is often shortened to \It{nan} in rapid speech.} 
‘on’ with the multiple of ten as the dependent \It{låge} (in \Sc{gen.sg} case), as illustrated in Figure \vref{twoDigitNumsB}. %: \It{guäkte låge nan} ‘twelve’ (literally ‘two on ten’), \It{guhta gålbmå låge nan} ‘thirtysix’ (literally ‘six on three-ten’), \It{gietjav gakktse låge nan} ‘eightyseven’ (literally ‘seven on eight-ten’). 
However, this latter method was only attested in elicitation sessions with one consultant, and is not found in non-elicited data from the corpus.
\begin{figure}\centering
\resizebox{\columnwidth}{!}{
\begin{tabular}{c| c| c c}
\It{guäkte låge nan}			&\It{guhta gålbmå-låge nan}		&\It{gakktse åktse-låge nan} &etc.\\
two ten\BS\GENs.\SGs\ on	&six three-ten\BS\GENs.\SGs\  on	&eight nine-ten\BS\GENs.\SGs\  on\\
%two ten on			&six three-ten on		&eight nine-ten on\\
12				& 36					& 98\\
%’twenty’		& ‘thirty’			& ‘forty’\\
\end{tabular}}
\caption{Two-digit numerals, method B}\label{twoDigitNumsB}
\end{figure}

Native ordinal numerals referring to numbers between ten and one hundred are only attested in the corpus in elicitation sessions, and speakers are quite inconsistent and unsure about them. 
The same is true for cardinal numerals larger than one hundred. %, which are only attested in the corpus in elicitation session, and speakers are unsure about them.
The only example in the corpus for a numeral larger than one thousand is not native, but a Swedish borrowing (in an NP with \PS\ case and number marking); this is provided in \REF{complexNumEx4}.\footnote{With the exception of the case/number suffix \It{-n}, the entire phrase \It{nittonhundratalan} ‘in the nineteen-hundreds’ in \REF{complexNumEx4} is borrowed from Swedish \It{nittonhundratalet} ‘the nineteen-hundreds’.}
\ea\label{complexNumEx4}
\glll	nittonhundratálan álgon ja dä viesoj\\
	nitton-hundra-tála-n álgo-n ja dä vieso-j\\
	nineteen-hundred-century-\Sc{iness.sg} beginning-\Sc{iness.sg} and then live-\Sc{3sg.pst}\\
\Transl{in the nineteen-hundreds, at the beginning, he lived}{}	\CorpusLink{0906_Ahkajavvre_a}{0906\_Ahkajavvre\_a}{070-072}
\z


\subsection{Numerals and morphosyntax}\label{caseNumberOnNum}%\subsection{Case and number marking on numerals}\label{caseNumberOnNum}
Numerals are generally not subject to inflectional morphology. %, i.e., unlike nouns, they do no inflect for case or number, and unlike adjectives, they do not inflect for \marginpar{update this comparison to ADJ if necessary!}attribution or predication. 
This is illustrated by the examples \REF{complexNumEx1} through \REF{complexNumEx3} above as well as in \REF{numeralEx3b} below. %through \REF{numeralEx4} below. 
%When a numeral modifies a noun in attributive position, the numeral does not inflect for case, number (unlike nouns) or attribution (unlike adjectives), as illustrated by \REF{complexNumEx1} above and \REF{numeralEx1}. 
%\ea\label{numeralEx1}
%\glll	dä guäkte dåpe lä danne\\
%	dä guäkte dåpe lä danne\\
%	then two house\BS\Sc{nom.pl} be\BS\Sc{3pl.prs} there\\
%\trans	‘two houses are there’	\Corpus{080924.385}
%\ea\label{numeralEx1}%cardnum+noun - plenty of these above!
%\glll	ja gu lip dä dajd gålbmå virbmijd jåddådam\\
%	ja gu li-p dä da-jd gålbmå virbmi-jd jåddåda-m\\
%	and when be-\Sc{1pl.prs} then \Sc{dem.dist}-\Sc{acc.pl} three fishing\_net-\Sc{acc.pl} set\_out\_net-\Sc{prf}\\
%\trans	‘and when we have set out those three fishing nets’	\Corpus{090702.042}
\ea\label{numeralEx3b}%PL-num + ILL
\glll	mån vaddav gålbmå buhtsujda biebmov\\
	mån vadda-v gålbmå buhtsu-jda biebmo-v\\
	\Sc{1sg.nom} give-\Sc{1sg.prs} three reindeer-\Sc{ill.pl} food-\Sc{acc.sg}\\\nopagebreak
\Transl{I give food to three reindeer}{}	\CorpusE{110413b}{156}
\z

Note that this is true even in predicative position and in elliptic constructions, as shown by the examples in \REF{numeralEx3} through \Ref{numeralEx4}.
\ea\label{numeralEx3}%cardnum headless
\glll	så dä lä guäkte\\
	så dä lä guäkte\\
	so then be\BS\Sc{3pl.prs} two\\\nopagebreak
\Transl{so then it’s two}{}	\Corpus{080924}{011}
\z
\ea\label{numeralEx2}%ordnum+noun
\glll	ja dä lä njelját aprilla uddne\\
	ja dä lä njelj-át aprilla uddne\\
	and then be\BS\Sc{3sg.prs} four-\Sc{ord} April today\\\nopagebreak
\Transl{and it is the fourth of April today}{}	\Corpus{100404}{018}
\z
\ea\label{numeralEx4}%ordnum headless
\glll	ja gålmát sjadda dä Stutjaj\\
	ja gålm-át sjadda dä Stutja-j\\
	and three-\Sc{ord} become\BS\Sc{3sg.prs} then Stutja-\Sc{ill.sg}\\\nopagebreak
\Transl{and the third one then goes to Stutja}{(referring to a fishing net)}	\Corpus{090702}{026-027}
\z

However, there are at least two exceptions. First, the numeral \It{akta} ‘one’ inflects for \ACCs.\SGs\ case when modifying a noun, as illustrated by \REF{numOneEx1}, as well as when it is in a headless elliptical construction, as in \REF{numOneEx2}.%\marginpar{footnote motivated by UM’s comment, not sure though if it’s a good idea}
\footnote{With this in mind, the word \It{akta} forms a word-class of its own, strictly speaking.}
\ea\label{numOneEx1}
\glll	åtjåjmen aktav guolev\\
	åtjå-jmen akta-v guole-v\\
	get-\Sc{1du.pst} one-\Sc{acc.sg} fish-\Sc{acc.sg}\\\nopagebreak
\Transl{we got one fish}{}	\CorpusLink{0906_Ahkajavvre_a}{0906\_Ahkajavvre\_a}{182}
\z
\ea\label{numOneEx2}
\glll	men vuotjiv mån aktav\\
	men vuotji-v mån akta-v\\
	but shoot-\Sc{1sg.pst} \Sc{1sg.nom} one-\Sc{acc.sg}\\\nopagebreak
\Transl{but I shot one}{}	\Corpus{080924}{008}
\z

Second, the example in \REF{veryFirstEx} indicates that at least the ordinal numeral \It{vuostas} ‘first’ can be inflected as a superlative as \It{vuostamos}, meaning ‘the very first’. 
\ea\label{veryFirstEx}%AdvP+ADJpred
\glll	dieda, mån vuotjev vuostamos guhkajuolgagav\\
	dieda mån vuotje-v vuosta-mos guhka-juolga-ga-v\\
	know\BS\Sc{2sg.prs} \Sc{1sg.nom} shoot-\Sc{1sg.pst} first-\Sc{superl} long-leg-\Sc{nmlz}-\Sc{acc.sg}\\\nopagebreak
\Transl{you know, I shot my very first long-legger}{(referring to a moose)}	\Corpus{080924}{079}
\z






%%%%%%% THIS IS NOT USED FOR THE ENTIRE COMPILATION, but only for individual chapters!!!!

\clearpage
\addcontentsline{toc}{chapter}{Bibliography}\label{Bibliography}
\bibliography{PiteGrammarBibSDL}%for bibtex
%\printbibliography%[title=Works Cited]%%for biber!






%%%NAME INDEX doesn’t work!?!? why???
\cleardoublepage\phantomsection%this allows hyperlink in ToC to work
\addcontentsline{toc}{chapter}{Name index}
\ohead{Name index}
\printindex[aut]

\cleardoublepage\phantomsection%this allows hyperlink in ToC to work
\addcontentsline{toc}{chapter}{Language index}
\ohead{Language index}
\printindex[lan]

\cleardoublepage\phantomsection%this allows hyperlink in ToC to work
\addcontentsline{toc}{chapter}{Subject index}
\ohead{Subject index}
\printindex


\end{document}