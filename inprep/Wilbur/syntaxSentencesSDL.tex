%\documentclass[ number=5
			   ,series=sidl
			   ,isbn=xxx-x-xxxxxx-xx-x
			   ,url=http://langsci-press.org/catalog/book/17
			   ,output=long   % long|short|inprep              
			   %,blackandwhite
			   %,smallfont
			   ,draftmode   
			  ]{LSP/langsci}                          

\usepackage{LSP/lsp-styles/lsp-gb4e}		% verhindert Komma bei mehrfachen Fußnoten?
                                                      
\usepackage{layout}
\usepackage{lipsum}

%%%% ABOVE FOR LangSciPress %%%%
%%%% ABOVE FOR LangSciPress %%%%
%%%% ABOVE FOR LangSciPress %%%%
\usepackage{libertine}%work-around solution for rendering problematic characters ʦ, ͡  (mostly in \textbf{})

\usepackage{longtable}%Double-lines (\hline\hline) aren’t typeset properly in ‘longtable’-environment across several pages! conflict with other package (maybe xcolor with option ‘tables’?)

\usepackage{multirow}

\usepackage{array} %allows, among other things, centering column content in a table while also specifying width, creates new column style "x" for center-alignment, "y" for right-alignment
\newcolumntype{x}[1]{>{\centering\hspace{0pt}}p{#1}}%
\newcolumntype{y}[1]{>{\raggedleft\hspace{0pt}}p{#1}}%

\usepackage[]{placeins}%using \FloatBarrier command, all floats still floating at that point will be typeset, and cannot cross that boundary. the option here \usepackage[section]{placeins} automatically adds \FloatBarrier to every \section command (only works for \section commands, nothing lower than that!)
%\usepackage{afterpage}%by using the command \afterpage{\clearpage}, all floats will appear, but no new page will be started, thus avoiding bad page breaks around floats

\usepackage{vowel} %for vowel space chart


%%%IS THIS NECESSARY??
%%%%following allows you to refer to footnotes (from http://anthony.liekens.net/index.php/LaTeX/MultipleFootnoteReferences)
%\newcommand{\footnoteremember}[2]{
%  \footnote{#2}
%  \newcounter{#1}
%  \setcounter{#1}{\value{footnote}}
%} \newcommand{\footnoterecall}[1]{
%  \footnotemark[\value{#1}]} 
%%%%previous allows you to refer to footnotes: use \footnoteremember{referenceText} in footnote, then \footnoterecall{referenceText} to refer.

\usepackage{tikz}%
\usetikzlibrary{plothandlers,matrix,decorations.text,shapes.arrows,shadows,chains,positioning,scopes}

\usepackage{synttree} %zeichnet linguistische Bäume
\branchheight{36pt}%sets height between rows in synttree

\usepackage{lscape}%used for landscape pages in index (list of recordings)

\usepackage{polyglossia}
\setmainlanguage{english}


%%%TAKE OUT FOR FINAL VERSION:
%%%TAKE OUT FOR FINAL VERSION:
%%%TAKE OUT FOR FINAL VERSION:

%%%%following readjusts margin text!
%\setlength{\marginparwidth}{20mm}
%\let\oldmarginpar\marginpar
%\renewcommand\marginpar[1]{\-\oldmarginpar[\raggedleft\footnotesize\vspace{-7pt}\color{red}\It{→ #1}]%
%{\raggedright\footnotesize\vspace{-7pt}\color{red}\It{→ #1}}}
%%%%previous readjusts margin text!

%%%The following lines set depth of ToC (LSP default is only 3 levels)!
%%%\renewcommand{\contentsname}{Table of Contents} % überschrift des inhaltsverzeichnisses
%\setcounter{secnumdepth}{5}%sets how deep section/subsection/subsubsections are numbered
%\setcounter{tocdepth}{5}%sets the depth of the ToC %but this doesn't seem to work!!!
%% new commands for LSP book (Grammar of Pite Saami, by J. Wilbur)

\newcommand{\PS}{Pite Saami}
\newcommand{\PSDP}{Pite Saami Documentation Project}
\newcommand{\WLP}{Wordlist Project}

\newcommand{\HANG}{\everypar{\hangindent15pt \hangafter1}}%also useful for table cells
\newcommand{\FB}{\FloatBarrier}%shortcut for this command to print all floats w/o pagebreak

\newcommand{\REF}[1]{(\ref{#1})}%adds parenthesis around the reference number, particularly useful for examples.%\Ref had clash with LSP!
\newcommand{\dline}{\hline\hline}%makes a double line in a table
\newcommand{\superS}[1]{\textsuperscript{#1}}%adds superscript element
\newcommand{\sub}[1]{$_{#1}$}%adds subscript element
\newcommand{\Sc}[1]{\textsc{#1}}%shortcut for small capitals (not to be confused with \sc, which changes the font from that point on)
\newcommand{\It}[1]{\textit{#1}}%shortcut for italics (not to be confused with \it, which changes the font from that point on)
\newcommand{\Bf}[1]{\textbf{#1}}%shortcut for bold (not to be confused with \bf, which changes the font from that point on)
\newcommand{\BfIt}[1]{\textbf{\textit{#1}}}
\newcommand{\BfSc}[1]{\textbf{\textsc{#1}}}
\newcommand{\Tn}[1]{\textnormal{#1}}%shortcut for normal text (undo italics, bolt, etc.)
\newcommand{\MC}{\multicolumn}%shortcut for multicolumn command in tabular environment - only replaces command, not variables!
\newcommand{\MR}{\multirow}%shortcut for multicolumn command in tabular environment - only replaces command, not variables!
\newcommand{\TILDE}{∼}%U+223C %OLD:~}%shortcut for tilde%command ‘\Tilde’ clashes with LSP!%
\newcommand{\BS}{\textbackslash}%backslash
\newcommand{\Red}[1]{{\color{red}{#1}}}%for red text
\newcommand{\Blue}[1]{{\color{blue}{#1}}}%for blue text
\newcommand{\PLUS}{+}%nicer looking plus symbol
\newcommand{\MINUS}{-}%nicer looking plus symbol
%    Was die Pfeile betrifft, kannst Du mal \Rightarrow \mapsto \textrightarrow probieren und dann \mathbf \boldsymbol oder \pbm dazutun.
\newcommand{\ARROW}{\textrightarrow}%→%dieser dicke Pfeil ➜ wird nicht von der LSP-Font unterstützt: %\newcommand{\ARROW}{{\fontspec{DejaVu Sans}➜}}
\newcommand{\DARROW}{\textleftrightarrow}%↔︎%DoubleARROW
\newcommand{\BULLET}{•}%
%%✓ does not exist in the default LSP font!
\newcommand{\CH}{\checkmark}%%\newcommand{\CH}{\fontspec{Arial Unicode MS}✓}%CH as in CHeck
%%following used to separate alternation forms for consonant gradation and umlaut patterns:
\newcommand{\Div}{‑}%↔︎⬌⟷⬄⟺⇔%non-breaking hyphen: ‑  
\newcommand{\QUES}{\textsuperscript{?}}%marks questionable/uncertain forms

\newcommand{\jvh}{\mbox{\It{j}-suffix} vowel harmony}%
%\newcommand{\Ptcl}{\Sc{ptcl} }%just shortcut for glossing ‘particle’
%\newcommand{\ATTR}{{\Sc{attributive}}}%shortcut for ATTRIBUTIVE in small caps
%\newcommand{\PRED}{{\Sc{predicative}}}%shortcut for PREDICATIVE in small caps
%\newcommand{\COMP}{{\Sc{comparative}}}%shortcut for COMPARATIVE in small caps
%\newcommand{\SUPERL}{{\Sc{superlative}}}%shortcut for SUPERLATIVE in small caps
\newcommand{\SG}{{\Sc{singular}}}%shortcut for SINGULAR in small caps
\newcommand{\DU}{{\Sc{dual}}}%shortcut for DUAL in small caps
\newcommand{\PL}{{\Sc{plural}}}%shortcut for PLURAL in small caps
%\newcommand{\NOM}{{\Sc{nominative}}}%shortcut for NOMINATIVE in small caps
%\newcommand{\ACC}{{\Sc{accusative}}}%shortcut for ACCUSATIVE in small caps
%\newcommand{\GEN}{{\Sc{genitive}}}%shortcut for GENITIVE in small caps
%\newcommand{\ILL}{{\Sc{illative}}}%shortcut for ILLATIVE in small caps
%\newcommand{\INESS}{{\Sc{inessive}}}%shortcut for INESSIVE in small caps
\newcommand{\ELAT}{{\Sc{elative}}}%shortcut for ELATIVE in small caps
%\newcommand{\COM}{{\Sc{comitative}}}%shortcut for COMITATIVE in small caps
%\newcommand{\ABESS}{{\Sc{abessive}}}%shortcut for ABESSIVE in small caps
%\newcommand{\ESS}{{\Sc{essive}}}%shortcut for ESSIVE in small caps
%\newcommand{\DIM}{{\Sc{diminutive}}}%shortcut for DIMINUTIVE in small caps
%\newcommand{\ORD}{{\Sc{ordinal}}}%shortcut for ORDINAL in small caps
%\newcommand{\CARD}{{\Sc{cardinal}}}%shortcut for CARDINAL in small caps
%\newcommand{\PROX}{{\Sc{proximal}}}%shortcut for PROXIMAL in small caps
%\newcommand{\DIST}{{\Sc{distal}}}%shortcut for DISTAL in small caps
%\newcommand{\RMT}{{\Sc{remote}}}%shortcut for REMOTE in small caps
%\newcommand{\REFL}{{\Sc{reflexive}}}%shortcut for REFLEXIVE in small caps
%\newcommand{\PRS}{{\Sc{present}}}%shortcut for PRESENT in small caps
%\newcommand{\PST}{{\Sc{past}}}%shortcut for PAST in small caps
%\newcommand{\IMP}{{\Sc{imperative}}}%shortcut for IMPERATIVE in small caps
%\newcommand{\POT}{{\Sc{potential}}}%shortcut for POTENTIAL in small caps
\newcommand{\PROG}{{\Sc{progressive}}}%shortcut for PROGRESSIVE in small caps
\newcommand{\PRF}{{\Sc{perfect}}}%shortcut for PERFECT in small caps
\newcommand{\INF}{{\Sc{infinitive}}}%shortcut for INFINITIVE in small caps
%\newcommand{\NEG}{{\Sc{negative}}}%shortcut for NEGATIVE in small caps
\newcommand{\CONNEG}{{\Sc{connegative}}}%shortcut for CONNEGATIVE in small caps
\newcommand{\ATTRs}{{\Sc{attr}}}%shortcut for ATTR in small caps
\newcommand{\PREDs}{{\Sc{pred}}}%shortcut for PRED in small caps
%\newcommand{\COMPs}{{\Sc{comp}}}%shortcut for COMP in small caps
%\newcommand{\SUPERLs}{{\Sc{superl}}}%shortcut for SUPERL in small caps
\newcommand{\SGs}{{\Sc{sg}}}%shortcut for SG in small caps
\newcommand{\DUs}{{\Sc{du}}}%shortcut for DU in small caps
\newcommand{\PLs}{{\Sc{pl}}}%shortcut for PL in small caps
\newcommand{\NOMs}{{\Sc{nom}}}%shortcut for NOM in small caps
\newcommand{\ACCs}{{\Sc{acc}}}%shortcut for ACC in small caps
\newcommand{\GENs}{{\Sc{gen}}}%shortcut for GEN in small caps
\newcommand{\ILLs}{{\Sc{ill}}}%shortcut for ILL in small caps
\newcommand{\INESSs}{{\Sc{iness}}}%shortcut for INESS in small caps
\newcommand{\ELATs}{{\Sc{elat}}}%shortcut for ELAT in small caps
\newcommand{\COMs}{{\Sc{com}}}%shortcut for COM in small caps
\newcommand{\ABESSs}{{\Sc{abess}}}%shortcut for ABESS in small caps
\newcommand{\ESSs}{{\Sc{ess}}}%shortcut for ESS in small caps
%\newcommand{\DIMs}{{\Sc{dim}}}%shortcut for DIM in small caps
%\newcommand{\ORDs}{{\Sc{ord}}}%shortcut for ORD in small caps
%\newcommand{\CARDs}{{\Sc{card}}}%shortcut for CARD in small caps
\newcommand{\PROXs}{{\Sc{prox}}}%shortcut for PROX in small caps
\newcommand{\DISTs}{{\Sc{dist}}}%shortcut for DIST in small caps
\newcommand{\RMTs}{{\Sc{rmt}}}%shortcut for RMT in small caps
\newcommand{\REFLs}{{\Sc{refl}}}%shortcut for REFL in small caps
\newcommand{\PRSs}{{\Sc{prs}}}%shortcut for PRS in small caps
\newcommand{\PSTs}{{\Sc{pst}}}%shortcut for PST in small caps
\newcommand{\IMPs}{{\Sc{imp}}}%shortcut for IMP in small caps
\newcommand{\POTs}{{\Sc{pot}}}%shortcut for POT in small caps
\newcommand{\PROGs}{{\Sc{prog}}}%shortcut for PROG in small caps
\newcommand{\PRFs}{{\Sc{prf}}}%shortcut for PRF in small caps
\newcommand{\INFs}{{\Sc{inf}}}%shortcut for INF in small caps
\newcommand{\NEGs}{{\Sc{neg}}}%shortcut for NEG in small caps
\newcommand{\CONNEGs}{{\Sc{conneg}}}%shortcut for CONNEG in small caps

\newcommand{\subNP}{{\footnotesize\sub{NP}}}%shortcut for NP (nominal phrase) in subscript
\newcommand{\subVC}{{\footnotesize\sub{VC}}}%shortcut for VC (verb complex) in subscript
\newcommand{\subAP}{{\footnotesize\sub{AP}}}%shortcut for NP (adjectival phrase) in subscript
\newcommand{\subAdvP}{{\footnotesize\sub{AdvP}}}%shortcut for AdvP (adverbial phrase) in subscript
\newcommand{\subPP}{{\footnotesize\sub{PP}}}%shortcut for NP (postpoistional phrase) in subscript

\newcommand{\ipa}[1]{{\fontspec{Linux Libertine}#1}}%specifying font for IPA characters

\newcommand{\SEC}{§}%standardize section symbol and spacing afterwards
%\newcommand{\SEC}{§\,}%

\newcommand{\Nth}{{\footnotesize(\It{n})}}%used in table of numerals in ADJ chapter

%%newcommands for tables in introductionSDL.tex:
\newcommand{\cliticExs}[3]{\Tn{\begin{tabular}{p{28mm} c p{28mm} p{35mm}}\It{#1}&\ARROW &\It{#2} & ‘#3’\\\end{tabular}}}%specifically for the two clitic examples
\newcommand{\Grapheme}[1]{\It{#1}}%formatting for graphemes in orthography tables
%%new command for the section on orthographic examples; syntax: #1=orthography, #2=phonology, #3=gloss
\newcommand{\SpellEx}[3]{\Tn{\begin{tabular}{p{70pt} p{70pt} l}\ipa{/#2/}&\It{#1}& ‘#3’ \\\end{tabular}}}%formatting for orthographic examples (intro-Chapter)


%%new transl tier in gb4e; syntax: #1=free translation (in single quotes), #2=additional comments, z.B. literal meaning:
\newcommand{\Transl}[2]{\trans\Tn{‘#1’ #2}}%new transl tier in gb4e;
\newcommand{\TranslMulti}[2]{\trans\hspace{12pt}\Tn{‘#1’ #2}}%new transl tier in gb4e for a dialog to be included under a single example number


%% used for examples in the Prosody and Segmental phonology chapters:
\newcommand{\PhonGloss}[7]{%PhonGloss = Phonology Gloss;
%pattern: \PhonGloss{label}{phonemic}{phonetic}{orthographic}{gloss}{recording}{utterance}
\ea\label{#1}
\Tn{\begin{tabular}[t]{p{30mm} l}
\ipa{/#2/}	& \It{#4} \\
\ipa{[#3]}	&\HANG ‘#5’\\%no table row can start with square brackets! thus the workaround with \MC
\end{tabular}\hfill\hyperlink{#6}{{\small\textnormal[pit#6#7]}}%\index{Z\Red{rec}!\Red{pit#6}}\index{Z\Red{utt}!\Red{pit#6#7} \Blue{Phon}}
}
\z}
\newcommand{\PhonGlossWL}[6]{%PhonGloss = Phonology Gloss for words from WORDLIST, not from corpus!;
%pattern: \PhonGloss{label}{phonemic}{phonetic}{orthographic}{gloss}{wordListNumber}
\ea\label{#1}
\Tn{\begin{tabular}[t]{p{30mm} l}
\ipa{/#2/}	& \It{#4} \\
\ipa{[#3]}	&\HANG ‘#5’\\%no table row can start with square brackets! thus the workaround with \MC
\end{tabular}\hfill\hyperlink{explExs}{{\small\textnormal[#6]}}%\index{Z\Red{wl}!\Red{#6}\Blue{Phon}}
}
\z}

%%for derivation examples in the derivational morphology chapter!
%syntax: \DerivExam{#1}{#2}{#3}{#4}{#5}{#6}
%#1: base, #2: base-gloss, #3: derived form, #4: derived form gloss, #5: derived form translation, #6: pit-recording, #7: utterance number
\newcommand{\DW}{28mm}%for following three commands, to align arrows throughout
%%%%OLD:
%%%\newcommand{\DerivExam}[7]{\Tn{\begin{tabular}[t]{p{\DW}cl}\It{#1}&\ARROW&\It{#3}\\#2&&#4\\\end{tabular}\hfill\pbox{.3\textwidth}{\hfill‘#5’\\\hbox{}\hfill\hyperlink{pit#6}{{\small\textnormal[pit#6.#7]}}}
%%%%\index{Z\Red{rec}!\Red{pit#6}}\index{Z\Red{utt}!\Red{pit#6.#7}}
%%%}}
%NEW:
\newcommand{\DerivExam}[7]{\Tn{
\begin{tabular}[t]{p{\DW}x{5mm}l}\It{#1}&\ARROW&\It{#3}\\\end{tabular}\hfill‘#5’\\
\hspace{1mm}\begin{tabular}[t]{p{\DW}x{5mm}l}#2&&#4\\\end{tabular}\hfill\hyperlink{pit#6}{{\small\textnormal[pit#6.#7]}}
%\index{Z\Red{rec}!\Red{pit#6}}\index{Z\Red{utt}!\Red{pit#6.#7}}
}}
%%same as above, but supress any reference to a specific utterance
\newcommand{\DerivExamX}[7]{\Tn{
\begin{tabular}[t]{p{\DW}x{5mm}l}\It{#1}&\ARROW&\It{#3}\\\end{tabular}\hfill‘#5’\\
\hspace{1mm}\begin{tabular}[t]{p{\DW}x{5mm}l}#2&&#4\\\end{tabular}\hfill\hyperlink{pit#6}{{\small\textnormal[pit#6]\It{e}}}
%\index{Z\Red{rec}!\Red{pit#6}}\index{Z\Red{utt}!\Red{pit#6.#7}}
}}
\newcommand{\DerivExamWL}[6]{\Tn{
\begin{tabular}[t]{p{\DW}x{5mm}l}\It{#1}&\ARROW&\It{#3}\\\end{tabular}\hfill‘#5’\\
\hspace{1mm}\begin{tabular}[t]{p{\DW}x{5mm}l}#2&&#4\\\end{tabular}\hfill\hyperlink{explExs}{{\small\textnormal[#6]}}
%\index{Z\Red{wl}!\Red{#6}}
}}


%formatting of corpus source information (after \transl in gb4e-environments):
\newcommand{\Corpus}[2]{\hspace*{1pt}\hfill{\small\mbox{\hyperlink{pit#1}{\Tn{[pit#1.#2]}}}}%\index{Z\Red{rec}!\Red{pit#1}}\index{Z\Red{utt}!\Red{pit#1.#2}}
}%
\newcommand{\CorpusE}[2]{\hspace*{1pt}\hfill{\small\mbox{\hyperlink{pit#1}{\Tn{[pit#1.#2]}}\It{e}}}%\index{Z\Red{rec}!\Red{pit#1}}\index{Z\Red{utt}!\Red{pit#1.#2}\Blue{-E}}
}%
%%as above, but necessary for recording names which include an underline because the first variable in \href understands _ but the second variable requires \_
\newcommand{\CorpusLink}[3]{\hspace*{1pt}\hfill{\small\mbox{\hyperlink{pit#1}{\Tn{[pit#2.#3]}}}}%\index{Z\Red{rec}!\Red{pit#2}}\index{Z\Red{utt}!\Red{pit#2.#3}}
}%
%%as above, but for newer recordings which begin with sje20 instead of pit
\newcommand{\CorpusSJE}[2]{\hspace*{1pt}\hfill{\small\mbox{\hyperlink{sje20#1}{\Tn{[sje20#1.#2]}}}}%\index{Z\Red{rec}!\Red{sje20#1}}\index{Z\Red{utt}!\Red{sje20#1.#2}}
}%
\newcommand{\CorpusSJEE}[2]{\hspace*{1pt}\hfill{\small\mbox{\hyperlink{sje20#1}{\Tn{[sje20#1.#2]}}\It{e}}}%\index{Z\Red{rec}!\Red{sje20#1}}\index{Z\Red{utt}!\Red{sje20#1.#2}\Blue{-E}}
}%











%%hyphenation points for line breaks
%%add to TeX file before \begin{document} with:
%%%%hyphenation points for line breaks
%%add to TeX file before \begin{document} with:
%%%%hyphenation points for line breaks
%%add to TeX file before \begin{document} with:
%%\include{hyphenationSDL}
\hyphenation{
ab-es-sive
affri-ca-te
affri-ca-tes
Ahka-javv-re
al-ve-o-lar
com-ple-ments
%check this:
de-cad-es
fri-ca-tive
fri-ca-tives
gemi-nate
gemi-nates
gra-pheme
gra-phemes
ho-mo-pho-nous
ho-mor-ga-nic
mor-pho-syn-tac-tic
or-tho-gra-phic
pho-neme
pho-ne-mes
phra-ses
post-po-si-tion
post-po-si-tion-al
pre-as-pi-ra-te
pre-as-pi-ra-ted
pre-as-pi-ra-tion
seg-ment
un-voiced
wor-king-ver-sion
}
\hyphenation{
ab-es-sive
affri-ca-te
affri-ca-tes
Ahka-javv-re
al-ve-o-lar
com-ple-ments
%check this:
de-cad-es
fri-ca-tive
fri-ca-tives
gemi-nate
gemi-nates
gra-pheme
gra-phemes
ho-mo-pho-nous
ho-mor-ga-nic
mor-pho-syn-tac-tic
or-tho-gra-phic
pho-neme
pho-ne-mes
phra-ses
post-po-si-tion
post-po-si-tion-al
pre-as-pi-ra-te
pre-as-pi-ra-ted
pre-as-pi-ra-tion
seg-ment
un-voiced
wor-king-ver-sion
}
\hyphenation{
ab-es-sive
affri-ca-te
affri-ca-tes
Ahka-javv-re
al-ve-o-lar
com-ple-ments
%check this:
de-cad-es
fri-ca-tive
fri-ca-tives
gemi-nate
gemi-nates
gra-pheme
gra-phemes
ho-mo-pho-nous
ho-mor-ga-nic
mor-pho-syn-tac-tic
or-tho-gra-phic
pho-neme
pho-ne-mes
phra-ses
post-po-si-tion
post-po-si-tion-al
pre-as-pi-ra-te
pre-as-pi-ra-ted
pre-as-pi-ra-tion
seg-ment
un-voiced
wor-king-ver-sion
}\begin{document}\tableofcontents\clearpage

%%%%%%%%%%%%%%%%%%%%%%%%%%%%%%%%% ALL THE ABOVE TO BE COMMENTED OUT FOR COMPLETE DOCUMENT! %%%%%%%%%%%


\chapter{Overview of the syntax of sentences}\label{overviewSyntax}
In describing \PS\ clauses, it is useful to begin with basic clauses that contain a full predicate and its arguments, complements and/or adjuncts, before moving on to describe complex clauses which consist of two or more clauses linked to one another. 
Therefore, basic clauses are described in Chapter \ref{basicClauses}, including declarative, interrogative and imperative clauses. Chapter \ref{complexClauses} then deals with complex clauses, covering coordination and subordination. 

However, in order to better understand the syntax of sentences, it is sensible to begin with two general discussions that provide a framework for understanding the syntactic descriptions that follow. The first of these, in \SEC\ref{grammaticalRelations} below, covers grammatical relations in Pite Saami. This leads to the second discussion in \SEC\ref{constituentOrderClauses}, which concerns clause-level constituent ordering, and the likely role that information structure plays in determining this.


\section{Grammatical relations}\label{grammaticalRelations}\is{grammatical relations}
\PS\ is an accusative language because %the subject and only nominal argument %delete; cf. Comrie 1989
the only argument of an intransitive\is{intransitive} verb (S) is marked in the same way as the most-agent-like argument of a transitive\is{transitive} verb (A): by the nominative\is{nominative} case. 
The most-patient-like argument of a transitive verb (P) is marked differently: by the accusative\is{accusative} case. This is illustrated by the following examples, with an intransitive verb in \REF{intrans1} and a (mono-)transitive verb in \REF{monotrans1}.
\ea\label{intrans1}
\glll	så mån tjielka sinne vällahiv\\
	så mån tjielka sinne vällahi-v\\
	so \Sc{1sg.nom} sled\BS\Sc{gen.sg} in lie-\Sc{1sg.pst}\\\nopagebreak
\Transl{so I lay in the sled}{} \Corpus{100404}{303}
\z
\ea\label{monotrans1}
\glll	dä almatj biejaj risev dále nåvte\\
	dä almatj bieja-j rise-v dále nåvte\\
	then person\BS\Sc{nom.sg} put-\Sc{3sg.prs} stick-\Sc{acc.sg} now so\\\nopagebreak
\Transl{then one places the stick like this}{} \Corpus{100404}{216}
\z

The direct object of a ditransitive\is{ditransitive} verb is also in the accusative case, while the indirect object %’goal’ is semantic role, not grammatical relation which is needed here
of a ditransitive verb is in an oblique case (usually in the illative\is{illative} case, which prototypically indicates that the noun refers to the goal of a movement), as illustrated in \REF{ditrans1}.% or even in a postpositional phrase\marginpar{find examples!}.
\ea\label{ditrans1}
\glll	mån vaddav suhta buhtsujda biebmov\\
	mån vadda-v suhta buhtsu-jda biebmo-v\\
	\Sc{1sg.nom} give-\Sc{1sg.prs} several reindeer-\Sc{ill.pl} food-\Sc{acc.sg}\\\nopagebreak
\Transl{I give food to several reindeer}{} \CorpusE{110413b}{157}
\z

Grammatical relations in \PS\ are thus marked by morphological means. Constituent ordering\is{constituent order} does not indicate grammatical relations in any way. For instance, in \REF{monotrans2} the object precedes the subject. % when the object is in focus, as illustrated by \REF{monotrans2}.%
\ea\label{monotrans2}
\glll	ja dáv aj mån vuojnav vinndegest muv dåbest\\
	ja d-á-v aj mån vuojna-v vindege-st muv dåbe-st\\
	and \Sc{dem}-\Sc{prox}-\Sc{acc.sg} also \Sc{1sg.nom} see-\Sc{1sg.prs} window-\Sc{elat.sg} \Sc{1sg.gen} house-\Sc{elat.sg}\\\nopagebreak
\Transl{and I also see this from the window from my house}{} \Corpus{100310b}{030}
%\glll	sågijd mån anav\\
%	sågi-jd mån ana-v\\
%	birch-\Sc{acc.pl} \Sc{1sg.nom} use-\Sc{1sg.prs}\\\nopagebreak
%\Transl{I use birch wood}{} \Corpus{090702}{149}
\z
The following section %\ref{constituentOrderClauses} 
provides more examples illustrating the flexibility of constituent ordering.


\section{Constituent order at clause level}\label{constituentOrderClauses}\is{constituent order}
Clause-level constituent ordering in \PS\ is not determined syntactically. 
That being said, in elicited clauses from the corpus, some ordering patterns do occur more frequently than others, and indicate that SVO ordering is preferred in context-free elicited clauses, everything else being equal. 
This is illustrated by the examples in \REF{standardConstOrder1} through \REF{standardConstOrder3}. 
\ea\label{standardConstOrder1}%
\glll	sån usjuda\\
	sån usjuda\\
	\Sc{3sg.nom} think\BS\Sc{3sg.prs} \\\nopagebreak
\Transl{he thinks}{} \CorpusE{081011}{154}
\z
\ea\label{standardConstOrder2}%
\glll	mån vuojnav bierdnav\\
	mån vuojna-v bierdna-v\\
	\Sc{1sg.nom} see-\Sc{1sg.prs} bear-\Sc{acc.sg} \\\nopagebreak
\Transl{I see a bear}{} \CorpusE{080926}{01m24s}
\z
\ea\label{standardConstOrder3}%
\glll	ålmaj vaddá blåmåv kuijdnaj\\
	ålmaj vaddá blåmå-v kuijdna-j\\
	man\BS\Sc{nom.sg} give\BS\Sc{3sg.prs} flower-\Sc{acc.sg} woman-\Sc{ill.sg}\\\nopagebreak
\Transl{the man gives the flower to the woman}{} \CorpusE{100324}{65m25s}
\z
It is possible that this ordering is triggered by typical Swedish constituent ordering, which is generally SVO, as Swedish was often used as the meta-language in elicitation sessions. 

More significantly, the part of the \PS\ corpus consisting of natural language situations confirms the lack of any set constituent ordering based on syntactic criteria.\footnote{Note that Sammallahti claims that at least for North Saami (although it is not entirely clear whether he means North Saami or is generalizing for all Saami languages here), the “order of the main constituents […] is largely free from formal restrictions and guided by pragmatic principles”, but then states that the “basic order is SVO” \citep[95]{Sammallahti1998}. This seems to reflect the data from the \PS\ corpus to the extent that context-free elicited clauses tend to be SVO, while in fact no syntactic criteria for constituent ordering can be ascertained in natural language. 
Lagercrantz takes several pages to describe a variety of tendencies in constituent ordering for \PS\ declarative clauses, even after describing ordering preferences concerning topic and focus within a discourse and summarizing the actual situation by stating quite vaguely that the position of the subject in a clause has a ‘certain stylistic effect’ (“Die Stellung des Satzgegenstandes hat eine gewisse stilistische Wirkung”) \citep[46]{Lagercrantz1926}. %(in pre-information-structure terms). 
Perhaps current descriptions of the syntax of the Saami languages would be better served if linguists %(including myself)
would cease trying to force these languages into an inaccurate (but typologically neat) label such as SVO.} 
To illustrate this syntactic flexibility, examples of SOV and OSV %and VS %removed: Adv+V+S - no ex. with V in first w/o dä or some other adv! (with overt subject and only one verb)
constituent ordering are provided in \REF{deviantConstituentOrder1} and \REF{deviantConstituentOrder2}, % and \REF{deviantConstituentOrder3}, 
respectively.
\ea\label{deviantConstituentOrder1}
\glll	mån vuostasj vierbmev biejav Áktjuotjålbmáj \\
	mån vuostasj vierbme-v bieja-v Áktjuotjålbmá-j \\
	\Sc{1sg.nom} first net-\Sc{acc.sg} put-\Sc{1sg.prs} Áktjuotjålbme-\Sc{ill.sg}\\\nopagebreak
\Transl{first I’ll put out the net in Áktjuotjålbme}{} \Corpus{090702}{024}
\z
\ea\label{deviantConstituentOrder2}%
\glll	sågijd mån anav\\
	sågi-jd mån ana-v\\
	birch-\Sc{acc.pl} \Sc{1sg.nom} use-\Sc{1sg.prs} \\\nopagebreak
\Transl{I use birchwood}{} \Corpus{090702}{149}
\z

The example in \REF{deviantConstituentOrder4} has VSO constituent order, and additionally has the non-finite verb complement (with OV ordering) in clause-final position.
\ea\label{deviantConstituentOrder4}%
\glll	dä galgav mån gåvåjd vuosedit\\
	dä galga-v mån gåvå-jd vuosedi-t\\
	then will-\Sc{1sg.prs} \Sc{1sg.nom} picture-\Sc{acc.pl}  show-\Sc{inf} \\\nopagebreak
\Transl{then I will show some pictures}{} \Corpus{080825}{036}
\z

Attempting to determine constituent order patterns is further complicated by the fact that it is sometimes impossible to tell what the constituent ordering is because NPs\is{nominal phrase} referring to %presupposed 
information provided by context alone are frequently not realized overtly, as in \REF{deviantConstituentOrder4} above and \REF{missingNP1} through \REF{missingNP4} below.
\ea\label{missingNP1}
\glll	ber aktak tjårvev adna \\
	ber aktak tjårve-v adna \\
	only one antler-\Sc{acc.sg} have\BS\Sc{3sg.prs}\\\nopagebreak
\Transl{(the reindeer) only has one antler}{} \Corpus{100405b}{019}
\z
\ea\label{missingNP2}%
\glll	gallga giesset ulgus\\
	gallga giesse-t ulgus\\
	will\BS\Sc{3sg.prs} pull-\Sc{inf} out\\\nopagebreak
\Transl{(the reindeer herder) will pull (the reindeer buck) out}{} \Corpus{080909}{017}
\z
\ea\label{missingNP3}%
\glll	mån biejav dut\\
	mån bieja-v dut\\
	\Sc{1sg.nom} put-\Sc{1sg.prs} there\\\nopagebreak
\Transl{I’ll put (the pole) there}{} \Corpus{100404}{218}
\z
\ea\label{missingNP4}%
\glll	ja vadde, Eva-Karin!\\
	ja vadde Eva-Karin\\
	and give\BS\Sc{2sg.imp} Eva-Karin\\\nopagebreak
\Transl{and give (me) (a sausage), Eva-Karin!}{} \Corpus{090519}{208}
\z
While person\is{person} and number\is{number} markers on the finite verb indicate grammatical information about the subject, there is no overt subject in \REF{missingNP1} or \REF{missingNP2}. The clauses in \REF{missingNP2} through \REF{missingNP4} are lacking overt objects. The final example is also missing the indirect object.

Indeed, a complete clause can consist of nothing more than an inflected verb, as in the response %\marginpar{the question is of course whether ‘beautiful weather’ is not also a complete clause.} 
in \REF{missingNP5}, which consists of nothing more than the copula verb\is{copular verb} inflected for \Sc{3sg.prs}.
\ea\label{missingNP5}
\glll	\Tn{A:} ja tjábba dállke!\\
	{} ja tjábba dállke\\
	{} and beautiful weather\BS\Sc{nom.sg} \\\nopagebreak
\TranslMulti{and such beautiful weather!}{}\\ %\Corpus{Hilpert}
\glll	\Tn{B:} lä.\\
	{} lä\\
	{} be\BS\Sc{3sg.prs}\\\nopagebreak
\TranslMulti{yes, it is}{(lit.: ‘is’)\footnotemark} %\Corpus{pit05HilpertDialog}{0m38s}%\footnotemark
%\footnotetext{The recording ‘\hyperlink{pit05HilpertDialog}{pit05HilpertDialog}’ was collected by linguist Martin Hilpert during a pilot project he completed on \PS\ in 2005. I am very grateful to him for providing me with his recordings and annotations. Note that Martin’s recordings are not included in the \PS\ documentation corpus.}
\z
\footnotetext{The source of the example in \REF{missingNP5} is a recording %‘\hyperlink{pit05HilpertDialog}{pit05HilpertDialog}’, which was 
collected by linguist Martin Hilpert\aimention{Hilpert, Martin} during a pilot project he completed on \PS\ in 2005. I am very grateful to him for providing me with his recordings and annotations. Note that Martin’s recordings are not included in the \PS\ documentation corpus.} 

However, there are no examples in the corpus of verb-initial clauses featuring both an overt subject and a VC with a single verb. While there are plenty of examples of the finite verb preceding the subject and most other clausal constituents, some constituent always precedes the verb. Frequently, it is the adverb \It{dä}, as in \REF{ConstOrderAdvVex1}. 
\ea\label{ConstOrderAdvVex1}%
\glll	ja dä båhta reksak\\
	ja dä båhta reksak\\
	and then come\BS\Sc{3sg.prs} ptarmigan\BS\Sc{nom.sg} \\\nopagebreak
\Transl{and then a ptarmigan comes}{} \Corpus{100404}{241}
\z
If the subject is not realized overtly, or if the VC contains more than one verb form, then the finite verb can be clause initial, as in \REF{ConstOrderAdvVex2} and \REF{ConstOrderAdvVex3}, respectively. 
\ea\label{ConstOrderAdvVex2}%
\glll	bisij dajd dä såbe nanne\\
	bisi-j d-a-jd dä såbe nanne\\
	grill-\Sc{3sg.pst} \Sc{dem}-\Sc{dist}-\Sc{acc.pl} then stick\BS\Sc{gen.sg} on\\\nopagebreak
\Transl{he grilled them then on a stick}{} \Corpus{100404}{125}
\z
\ea\label{ConstOrderAdvVex3}%
\glll	ittjiv mån mujte\\
	ittji-v mån mujte\\
	\Sc{neg}-\Sc{1sg.pst} \Sc{1sg.nom} remember\BS\Sc{conneg} \\\nopagebreak
\Transl{I didn’t remember}{} \Corpus{100404}{227}
\z


\subsection{Information structure}\label{infoStructure}\is{information structure}
Considering the syntactic flexibility described above, it is only reasonable to consider information structure as a constituent ordering\is{constituent order} strategy. 
While a thorough investigation of information structure in \PS\ is beyond the scope of the present study %, it seems that constituent order is likely determined by the pragmatic structure of a clause and the text it is part of. While the following is of a preliminary nature, 
and must be left for more thorough future research, some preliminary observations can be made. % and so the following is of a preliminary nature. 

Specifically, declarative\is{declarative} clauses typically begin with the topic (frequently the subject in the nominative\is{nominative} case) and end with a comment on that topic. If the comment involves a transitive\is{transitive} verb, the object or complement clause (the focus\is{focus}) normally follows the verb, as in \REF{standardConstOrder2} and \REF{standardConstOrder3} above. However, clausal elements in focus can be moved from their ‘default’ position, which results in significant deviations from the %basic/normal/unmarked/
preferred SVO constituent order. This is reflected in constituent interrogative\is{interrogative} clauses; here, the interrogative pronoun is in focus and always in clause-initial position %, as evidenced by the fact that interrogative pronouns and adverbs are always clause-initial 
(cf. \SEC\ref{constituentQs}).

The short example text presented in \REF{mouseText4} and \REF{mouseText5} below should serve to give an impression of how information structure may be the driving force behind constituent ordering at clause-level. Here, the speaker is talking about looking inside her mother’s shoes after discovering that a mouse had been in them.
\ea\label{mouseText4}%
\glll	ja danne vuojdniv unna jåŋåtjav.\\
	ja danne vuojdni-v unna jåŋå-tja-v\\
	and there see-\Sc{1sg.pst} small lingonberry-\Sc{dim}-\Sc{acc.sg}\\\nopagebreak
\Transl{and there I saw a little lingonberry}{}	\Corpus{100404}{353}
\ex\label{mouseText5}%
\glll	jahkav skafferijav lä danne adnam,\\
	jahka-v skafferija-v lä danne adna-m\\
	believe-\Sc{1sg.prs} pantry-\Sc{acc.sg} be\BS\Sc{3sg.prs} there have-\Sc{prf}\\\nopagebreak
\Transl{I think (the mouse) had a pantry there}{}	\Corpus{100404}{354}
\z

In the first clause \REF{mouseText4}, the topic is \It{danne} ‘there’, which refers to the shoes (the topic of the anecdote at this point) and is clause-initial. The constituent \It{jåŋåtjav} ‘lingonberry’ is the focus, but it is not particularly significant in the anecdote, and it follows the finite verb \It{vuojdniv} ‘I saw’. 
However, when particular emphasis is placed on the focus, as in the following clause in \REF{mouseText5}, the constituent in focus can be fronted. Here, \It{skafferijav} ‘pantry’ is in focus, and receives particular emphasis\footnote{The NP \It{skafferijav} ‘pantry’ probably receives special emphasis because it personifies the behavior of the mouse in a light-hearted way by claiming that a mouse can have a pantry.} by occurring before the verbal complex \It{lä danne adnam} ‘has had there’, while the topic (the mouse) is not realized overtly at all, but implied by the context and by the finite verb form inflected for \Sc{3sg}. 
This fronting of a constituent is often accompanied by higher acoustic intensity, as is the case here.%; such constituents are presented in \Bf{bold} face in the example text. 




%%%%%%%%%%%%%%%%%%%%%%%%%%%%%%%%%
%%%%%%%%%%%%%%%%%%%%%%%%%%%%%%%%%
%%%%%%%%%%%%%%%%%%%%%%%%%%%%%%%%%
%%%%%%%%%%%%%%%%%%%% B A S I C    C L A U S E S
%%%%%%%%%%%%%%%%%%%%%%%%%%%%%%%%%
%%%%%%%%%%%%%%%%%%%%%%%%%%%%%%%%%
%%%%%%%%%%%%%%%%%%%%%%%%%%%%%%%%%

\chapter{Basic clauses}\label{basicClauses}
A basic clause is a syntactic unit at text-level consisting minimally of a finite\is{finite verb} verb. In declarative clauses and interrogative clauses, this finite verb is marked morphologically for person\is{person}, number\is{number}, tense\is{tense} and/or mood\is{mood}. Aspect\is{aspect} can be expressed analytically at the clause level using an auxiliary verb and a non-finite verb form\is{non-finite verb form}. 
In all basic clauses, the finite verb agrees\is{agreement} in number and, with the exception of imperative mood, in person with the syntactic subject of the sentence, which is a nominal phrase in the nominative case. 
NPs referring to information provided by context alone are not necessarily realized overtly. As a result, the syntactic subject and other verbal arguments are often not overtly present. 

The following sections first present basic declarative clauses with intransitive and transitive verbs, existential clauses, copula clauses and complex verbal constructions consisting of more than one verb (\SEC\ref{declClauses}). Then, \SEC\ref{interrogClauses} deals with interrogative clauses, %highlighting the differences between interrogative and declarative clauses, %Negation, which is in fact a special case of complex verbal construction, is treated in \SEC\ref{},
before \SEC\ref{imperClauses} and \SEC\ref{potClauses} cover syntactic aspects of the imperative mood and the potential mood, respectively.


\section{Declarative clauses}\label{declClauses}\is{declarative}
Declarative clauses are the most common type of clause in the \PS\ corpus. In the following, declarative clauses with a single verb are dealt with first, covering intransitive and transitive verbs, and two special cases (existential clauses and copula clauses). Then, declarative clauses featuring a modal or auxiliary verb in addition to the lexical head verb are described; because negation\is{negation} is expressed by an auxiliary verb, it is covered in the same section. While constituent ordering is mentioned in the following sections, it mostly refers to tendencies only, and the flexible nature of \PS\ constituent ordering should always be kept in mind, as discussed in \SEC\ref{constituentOrderClauses}. 

\subsection{Basic intransitive declaratives}\label{basicIntransDeclaratives}
The subject of an intransitive declarative clause is in the nominative case, %. Such clauses typically have SV constituent ordering when the subject is overt, 
as in \REF{basicIntransDeclaratives1} through \REF{basicIntransDeclaratives3}. 
\ea\label{basicIntransDeclaratives1}
\glll	almatj usjut ja...\\
	almatj usjut ja\\
	person\BS\Sc{nom.sg} think\BS\Sc{3sg.prs} and \\%odd-syllable verbs for ER have no -a in 3sg.prs, but do have -a for DS, IF
\Transl{one thinks, and...}{} \Corpus{100404}{172}
\z
\ea\label{basicIntransDeclaratives2}
\glll	mån tjájmav\\
	mån tjájma-v\\
	\Sc{1sg.nom} laugh-\Sc{1sg.prs} \\\nopagebreak
\Transl{I laugh}{} \CorpusE{100323a}{005}
\z
\ea\label{basicIntransDeclaratives3}
\glll	dáj Skaile ello såkoj\\
	d-á-j Skaile ello såko-j\\
	\Sc{dem}-\Sc{prox}-\Sc{gen.pl} Skaile\BS\Sc{gen.pl} reindeer.herd\BS\Sc{nom.sg} drown-\Sc{3sg.pst} \\\nopagebreak
\Transl{The Skaile family’s reindeer herd drowned}{} \CorpusLink{0906_Ahkajavvre_b}{0906\_Ahkajavvre\_b}{002}
\z


\subsubsection{Clauses with a passive verb}\label{passiveVoice}\is{passive}
When a transitive verb is passivized,\footnote{Transitive verbs can be passivized using the derivational suffix\is{suffix!derivational} \It{-duvv}; cf. \SEC\ref{VdervPassives} on derivational morphology and \SEC\ref{passiveVinflection} on inflectional morphology for passives.} 
its valency is reduced, and it becomes intransitive. In this, the patient is the subject of the verbal complex\is{verbal complex}, and therefore marked by nominative\is{nominative} case; the agent may be left out. This is illustrated by the pair of elicited examples in \REF{passVoiceEx1} and \REF{passVoiceEx2}, in which the former is a transitive clause in active voice, and the latter is an intransitive clause in passive voice. %The first clause is a transitive verb in the active, while the second 
\ea\label{passVoiceEx1}%
\glll	máná lä tsiggim gådev\\
	máná lä tsiggi-m gåde-v\\
	child\BS\Sc{nom.pl} be\BS\Sc{3pl.prs} build-\Sc{prf} hut-\Sc{acc.sg}\\\nopagebreak
\Transl{children have built the hut}{}	\CorpusE{110518a}{28m14s}
\z
\ea\label{passVoiceEx2}%
\glll	dat lä tsiggiduvvum\\
	d-a-t lä tsiggi-duvvu-m\\
	\Sc{dem}-\Sc{dist}-\Sc{nom.sg} be\BS\Sc{3sg.prs} build-\Sc{pass}-\Sc{prf}\\\nopagebreak
\Transl{that (hut) was built}{}	\CorpusE{110518a}{27m41s}
\z
Similarly, the example from a narrative in \REF{passVoiceEx2b} presents an intransitive\is{intransitive} passive construction.
\ea\label{passVoiceEx2b}%
\glll	dát lä vanj dä gajk vuorasumos dágaduvvum\\
	d-á-t lä vanj dä gajk vuorasu-mos dága-duvvu-m\\
	\Sc{dem}-\Sc{prox}-\Sc{nom.sg} be\BS\Sc{3sg.prs} probably then all old-\Sc{superl}\BS\Sc{sg} make-\Sc{pass}-\Sc{prf}\\\nopagebreak
\Transl{this was probably the absolute oldest made}{}\\	\CorpusLink{0906_Ahkajavvre_a}{0906\_Ahkajavvre\_a}{120}
\z

The NP referring to the agent of an event can optionally occur obliquely in the elative case if the verb is passivized, as in \REF{passVoiceEx3}. \ea\label{passVoiceEx3}%
\glll	gåhte lä tsiggiduvvum mánájst\\
	gåhte lä tsiggi-duvvu-m máná-jst\\
	hut\BS\Sc{nom.sg} be\BS\Sc{3sg.prs} build-\Sc{pass}-\Sc{prf} child-\Sc{elat.pl}\\\nopagebreak
\Transl{the hut has been built by children}{}	\CorpusE{110518a}{28m41s}
\z


\subsection{Basic transitive declaratives}\label{basicMonotransDeclaratives}
In declarative clauses featuring a monotransitive\is{transitive} verb, %typically have SVO constituent ordering. The 
the subject is in nominative\is{nominative} and is typically the agent, while the object is in the accusative\is{accusative} case and is typically the patient of the predicate. Examples can be seen in \REF{basicMonotransDeclaratives1} through \REF{basicMonotransDeclaratives3}. 
\ea\label{basicMonotransDeclaratives1}
\glll	ja mån vuojnav muähtagav danne\\
	ja mån vuojna-v muähtaga-v danne\\
	and \Sc{1sg.nom} see-\Sc{1sg.prs} snow-\Sc{acc.sg} here\\\nopagebreak
\Transl{and I see snow here}{} \Corpus{100404}{020}
\z
\ea\label{basicMonotransDeclaratives2}%maybe not a good example here because ’danne’ is fronted due to focus, more like: “there is where the Saami have their reindeer in the summer”
\glll	danne sáme edne båhtsujd giesen\\
	danne sáme edne båhtsu-jd giese-n\\
	there Saami\BS\Sc{nom.pl} have\BS\Sc{3pl.prs} reindeer-\Sc{acc.pl} summer-\Sc{iness.sg} \\\nopagebreak
\Transl{the Saami keep the reindeer there in the summer}{} \Corpus{100404}{011}
\z
\ea\label{basicMonotransDeclaratives3}
\glll	almatj bedja virbmijd ehket\\
	almatj bedja virbmi-jd ehket\\
	person\BS\Sc{nom.sg} put\BS\Sc{3sg.prs} fishing.net-\Sc{acc.pl} evening\\%\BS\Sc{nom.sg}\\\nopagebreak
\Transl{one puts out fishing nets in the evening}{} \Corpus{100310b}{020}
\z

In declarative clauses with a ditransitive\is{ditransitive} verb, the direct object is also in the accusative case and is typically theme, while the indirect object, typically recipient, is normally in the illative\is{illative} case. Examples can be seen in \REF{basicDitransDeclaratives1} and \REF{basicDitransDeclaratives2}. 
\ea\label{basicDitransDeclaratives1}
\glll	mån vaddav gålbmå buhtsujda biebmov\\
	mån vadda-v gålbmå buhtsu-jda biebmo-v\\
	\Sc{1sg.nom} give-\Sc{1sg.prs} three reindeer-\Sc{ill.pl} food-\Sc{acc.sg}\\\nopagebreak
\Transl{I give food to three reindeer}{} \CorpusE{110413b}{156}
\z
\ea\label{basicDitransDeclaratives2}%
\glll	mån vaddav dunje fahtsajt\\
	mån vadda-v dunje fahtsa-jt\\
	\Sc{1sg.nom} give-\Sc{1sg.prs} \Sc{2sg.ill} glove-\Sc{acc.pl} \\\nopagebreak
\Transl{I give gloves to you}{} \CorpusE{080926}{01m01s}
\z


\subsection{Existential clauses}\label{existentialClauses}\is{existential}
The verb \It{gävdnut}\footnote{Much like its Swedish counterpart \It{finnas}%\marginpar{is the Swedish comparison relevant? especially without further comment?}
, which is a derivation of the verb \It{finna} ‘find’, the \PS\ verb \It{gävdnut} is derived from the verb \It{gávdnat} ‘find’.} 
is used as an existential verb.\footnote{Note that a copula clause containing an adjunct can also be used as an existential; cf. \SEC\ref{copulaClauses}.} 
The item whose existence is posited is the syntactic subject of the sentence and thus in the nominative\is{nominative} case, which \It{gävdnut} agrees\is{agreement} in person and number with. 
Examples can be found in \REF{existential1} and \REF{existential2}.%, as well as in \REF{existential3} below.
\ea\label{existential1}
\glll	váren gävdnu aj juomo\\
	váre-n gävdnu aj juomo\\
	mountain-\Sc{iness.sg} exist\BS\Sc{3pl.prs} also sorrel\BS\Sc{nom.pl}\\\nopagebreak
\Transl{there is sorrel in the mountains, too}{(lit.: there are sorrels…)} \Corpus{080924}{178}
\z
\ea\label{existential2}
\glll	dal itjij gävndoj aktak tjårvebielle\\
	dal itji-j gävndoj aktak tjårve-bielle\\
	now \Sc{neg}-\Sc{3sg.pst} exist\BS\Sc{conneg} any horn-half\BS\Sc{nom.sg}\\\nopagebreak
\Transl{there isn’t a single \It{tjårvebielle}\footnotemark\ now}{} \Corpus{100405b}{021}
\z
\footnotetext{A \It{tjårvebielle} (lit.: horn-half) is a term used to describe a reindeer with only one antler remaining after the other antler has broken off.} 

The subject frequently follows the verb because the subject is often the focus\is{focus} of the clause, but as the clause in \REF{existential3} shows, the subject may occur before the verb if it is the topic\is{topic} and/or is presupposed knowledge (cf. \SEC\ref{infoStructure} on information structure).
\ea\label{existential3}
\glll	motora vadnasij. “motora” ij gävdnu sáme gielan\\
	motora vadnasi-j motora ij gävdnu sáme giela-n\\
	motor\BS\Sc{gen.sg} boat-\Sc{com.pl} “motor” \Sc{neg}\BS\Sc{3sg.prs} exist\BS\Sc{conneg} Saami\BS\Sc{gen.sg} language-\Sc{iness.sg}\\\nopagebreak
\Transl{…with motor boats. There is no (word for) “motor” in the Saami language}{} \Corpus{080924}{482,484}
\z


\subsection{Copula clauses}\label{copulaClauses}
There are several types of copula clauses in Pite Saami. All of these feature the copula verb\is{copular verb} \It{årrot} ‘be’ %, which inflects for number and person of the subject and is marked for tense or mood when 
(see \SEC\ref{theCopulaVerb} for more details). 
Copula clauses can be used to express a variety of information about the subject referent, and these are discussed below. 

When the complement of the copula is an NP\is{nominal phrase} in nominative\is{nominative} case, it identifies or classifies the subject referent, as in \REF{copula1} and \REF{copula2}, respectively.
\ea\label{copula1}
\glll	Mattijá lij morbror munje\\
	Mattijá li-j {morbror\footnotemark} munje\\
	Matthias\BS\Sc{nom.sg} be-\Sc{3sg.pst} maternal.uncle\BS\Sc{nom.sg} \Sc{1sg.ill}\\\nopagebreak
\Transl{Matthias was my maternal uncle}{} \CorpusLink{0906_Ahkajavvre_a}{0906\_Ahkajavvre\_a}{007}%(lit.: uncle to me)
\z
\ea\label{copula2}
\glll	mån lev sábme\\
	mån le-v sábme\\
	\Sc{1sg.nom} be-\Sc{1sg.prs} Saami\BS\Sc{nom.sg}\\\nopagebreak
\Transl{I am a Saami}{} \Corpus{080813}{00m35s}
\z
\footnotetext{Note that \It{morbror} is a nonce borrowing from Swedish. Cf. Swedish \It{morbror} ‘maternal uncle’.}

The complement of a copula clause can be one or more adjectival phrases\is{adjectival phrase} headed by a predicative adjective which ascribes properties to the subject referent, as in \REF{copula3}.
\ea\label{copula3}
\glll	buhtsoj lä nav buojde ja tjábbe\\
	buhtsoj lä nav buojde ja tjábbe\\
	reindeer\BS\Sc{nom.pl} be\BS\Sc{3pl.prs} so fat\BS\Sc{pl} and beautiful\BS\Sc{pl}\\\nopagebreak
\Transl{the reindeer are so fat and beautiful}{} \Corpus{080703}{014}
\z

The complement of a copula clause can be an NP in the inessive\is{inessive} case in which case it describes the location of the subject referent, as in \REF{copula4}.
\ea\label{copula4}
\glll	måj lijmen Fuordnagin\\
	måj lij-men Fuordnagi-n\\
	\Sc{1du.nom} be-\Sc{1du.pst} Fuordnak-\Sc{iness.sg}\\\nopagebreak
\Transl{we two were in Fuordnak}{} \Corpus{080924}{590}
\z

The complement of a copula clause can be an NP in the elative\is{elative} case in which case it describes the material which the subject referent is made of, as in \REF{copula5}.
\ea\label{copula5}
\glll	ja dát lä aj struvdast\\
	ja d-á-t lä aj struvda-st\\
	and \Sc{dem}-\Sc{prox}-\Sc{nom.sg} be\BS\Sc{3sg.prs} also cloth-\Sc{elat.sg}\\\nopagebreak
\Transl{and this is also (made) of cloth}{} \CorpusLink{080708_Session08}{080708\_Session08}{015}
\z

A copula clause can also function as an existential clause when it includes a temporal adjunct. In such cases, the existence of the subject referent is posited at that particular time indicated by the adjunct. Pragmatically, this usually announces an event connected to the subject referent. Typically, the temporal referent occurs first\is{constituent order} in the sentence, then the copula verb, and the subject is last (just as with the existential verb \It{gädvnut}; cf. \SEC\ref{existentialClauses}), as it is usually the focus. 
This is illustrated by the example in \REF{copulaNr}.
\ea\label{copulaNr}
\glll	ja dále'l káffa\\
	ja dále=l káffa\\
	and now=be\BS\Sc{3sg.prs} coffee\BS\Sc{nom.sg}\\\nopagebreak
\Transl{and now it’s coffee (time)}{} \Corpus{090519}{313}
\z

Possession\is{possession} can also be expressed by a copula construction. In such a construction, the possessed NP is the subject of the clause in the nominative\is{nominative} case, which the finite verb agrees\is{agreement} with in person and number. The possessor NP is in the inessive\is{inessive} case. Such a construction is illustrated by the example in \REF{copula6}.
\ea\label{copula6}
\glll	muvne lä akta mánná\\
	muvne lä akta mánná\\
	\Sc{1sg.iness} be\BS\Sc{3sg.prs} one child\BS\Sc{nom.sg}\\\nopagebreak
\Transl{I have one child}{} \CorpusE{080621}{30m54s}
\z

In the corpus, such possessive constructions always have the constituent order 
\Sc{possessor}\PLUS\Sc{copula}\PLUS\Sc{possessed}. 
%\Sc{possessor} \PLUS\ \Sc{copula} \PLUS\ \Sc{possessed}. 
%\hfill\Sc{possessor} \PLUS\ \Sc{copula} \PLUS\ \Sc{possessed}\hfill\hbox{}
While this type of possessive construction is the native Saamic structure \citep[9]{Bergsland1977}, it is very uncommon in the \PS\ corpus, and almost exclusively limited to elicitation sessions. The elicitation scenario may have had an effect on the constituent order,\footnote{The equivalent Swedish structure is also normally \Sc{possessor \PLUS\ verb \PLUS\ possessed}.} 
but it is more likely the case that the constituent order reflects information structure preferences, specifically the tendency for the topic (more often the possessor, which is animate) to come before the focus\footnote{Cf. \SEC\ref{infoStructure} on information structure.} 
(more often the possessed, which is inanimate). 

In any case, a clause-level construction using the monotransitive\is{transitive} verb \It{adnet} ‘have’\footnote{Historically, \It{adnet} meant ‘use’ or ‘keep’ \citep[cf.][10]{Lehtiranta2001}, but synchronically it is most frequently used to indicate possession in a transitive verb construction which is essentially identical to the equivalent Swedish verb \It{ha} or the English verb \It{have}.} 
expressing possession, as in \REF{copula7}, is now the standard in Pite Saami.\footnote{While this construction using the transitive verb \It{adnet} ‘have’ is clearly not a copula clause, it is worth pointing this out here, particularly for any readers from Uralic studies, because the copular construction (as in the example in \REF{copula6}) is no longer the true standard for expressing possession at clause level in \PS.}
\ea\label{copula7}
\glll	ja dä inijmä gusajd \\
	ja dä ini-jmä gusa-jd \\
	and then have-\Sc{1pl.pst} cow-\Sc{acc.pl}\\\nopagebreak
\Transl{and then we had cows}{} \Corpus{080924}{091}
\z


\subsection{Multi-verb declarative clauses}\label{multiVdeclarativeClauses}
Verbs which govern non-finite\is{non-finite verb form} verbal complements can be classified into three groups based on the type of non-finite complement verb form they co-occur with, as illustrated in Table \vref{auxVerbTable}. % are three kinds of such verbs, with each category triggering a different kind of non-finite verb form:
\begin{table}[h]\centering
\caption{Verbs accompanied by a non-finite complement verb}\label{auxVerbTable}
\begin{tabular}{ll}\mytoprule
{}					&{non-finite form of complement} \\\hline
modals			& \INF \\
aspectual auxiliary	& \PRF\ / \PROG \\%\hline
negation verb		& \CONNEG \\\mybottomrule
\end{tabular}
\end{table}The finite verb occurs before the non-finite lexical complement verb, unless the complement is in focus, in which case it can occur before the finite verb. 
These verb types are dealt with in \SEC\ref{modalVs}, \SEC\ref{auxV} and \SEC\ref{negation}, respectively.


\subsubsection{Modal verbs}\label{modalVs}\is{modal}
Modal verbs are used to express modality for the event denoted by the verbal complement. The complementing verb is in the infinitive (marked by the \mbox{suffix \It{-t}).} Modal verbs include \It{máhttat} ‘can, be able to’, 
\It{ådtjot}\footnote{The modal verb \It{ådtjot} ‘be allowed to’ is homophonous with the full verb \It{ådtjot} ‘get, receive’. This pattern is found in Swedish, as well, with the verb \It{få} ‘be allowed to’ and ‘receive’, and in English, e.g.: ‘I get to go to the movies’.} 
‘may, be allowed to’, 
\It{virrtit} ‘must’, \It{hähttut}\footnote{The word \It{hähttut} ‘must’ is likely limited to northern dialects of Pite Saami.} 
‘must’, \It{sihtat} ‘want’ and \It{gallgat} ‘will/shall’. 
Some examples are provided in \REF{modalVerbs1} through \REF{modalVerbs3}. %\marginpar{what about bedjat ‘put’ used as let+INF (causative)?}
\ea\label{modalVerbs1}%
\glll	tjátsev ådtjobihtet juhgat dasste\\
	tjátse-v ådtjo-bihtet juhga-t d-a-sste\\
	water-\Sc{acc.sg} may-\Sc{2pl.prs} drink-\Sc{inf} \Sc{dem}-\Sc{dist}-\Sc{iness.sg}\\\nopagebreak
\Transl{you all may drink water from that}{} \Corpus{090519}{022}
\z
\ea\label{modalVerbs2a}%
\glll	ja dä del virrtin allget bäbbmat\\
	ja dä del virrti-n allge-t bäbbma-t\\
	and then well must-\Sc{3pl.pst} begin-\Sc{inf} feed-\Sc{inf}\\\nopagebreak
\Transl{and then they had to start to feed (the reindeer)}{} \Corpus{100405a}{029}
\z
\ea\label{modalVerbs3}%
\glll	mij máhttep ságastit Bidumsáme gielav\\
	mij máhtte-p ságasti-t Bidum-sáme giela-v\\
	\Sc{1pl.nom} can-\Sc{1pl.prs} speak-\Sc{inf} Pite-Saami\BS\Sc{gen.sg} language-\Sc{acc.sg}\\\nopagebreak
\Transl{we can speak the Pite Saami language}{} \Corpus{110517b2}{022}
\z

The modal verb \It{sihtat} ‘want’ behaves the same when the subject of the complementing verb complex is coreferential with the subject of the matrix clause, as in \REF{modalVerbs4}.
\ea\label{modalVerbs4}%
\glll	mån sidav gulijd adnet\\
	mån sida-v guli-jd adne-t\\
	\Sc{1sg.nom} want-\Sc{1sg.prs} fish-\Sc{acc.pl} have-\Sc{inf}\\\nopagebreak
\Transl{I want to have fish}{} \Corpus{090702}{012}
\z
However, when the subject of the modal verb \It{sihtat} is not coreferential with the subject of the complementing verbal complex\is{verbal complex}, then a finite verb clause %with a finite verb inflected for the second subject 
is the complement to the modal verb, as in the negated clause in \REF{modalVerbs5}. Note that, here, \It{sihtat} is in the connegative non-finite form \It{sida}. %Note that, Here the complement of the main verb \It{sida} (the connegative non-finite form) is the 
\ea\label{modalVerbs5}%
\glll	dä ij del almatj sida nagin sadjáj vuällget\\
	dä i-j del almatj sida nagin sadjá-j vuällge-t\\
	then \Sc{neg}-\Sc{3sg.prs} obviously person\BS\Sc{nom.sg} want\BS\Sc{conneg} some place-\Sc{ill.sg} go-\Sc{inf}\\\nopagebreak
\Transl{then one obviously doesn’t want to go anywhere}{} \Corpus{080924}{052}
%%following not good because ‘sida’ has an object-argument (mav) AND complement argument (galgav…)
%\glll	mav sida galgav enabujt ságastit\\
%	ma-v sida galga-v ena-bu-jt ságasti-t\\
%	what-\Sc{acc.sg} want\BS\Sc{2sg.prs} will-\Sc{1sg.prs} much-\Sc{comp-acc.pl} say-\Sc{inf}\\\nopagebreak
%\Transl{what more do you want me to say?}{} \Corpus{100404}{179}
\z

The modal verb \It{gallgat} ‘will’ can also be used to locate events in the future\is{future}, as in \REF{future1} through \REF{future3} below.
\ea\label{future1}%not really sure about the parsing of ‘såme-s’ as some-ATTR, don’t have PRED example!
\glll	dä galgav såmes mujjtemuv ságastit\\
	dä galga-v såmes mujjtemu-v ságasti-t\\
	then will-\Sc{1sg.prs} some memory-\Sc{acc.sg} say-\Sc{inf}\\\nopagebreak
\Transl{then I will tell (you) a memory}{} \Corpus{100703a}{001}
\z
\ea\label{future2}%
\glll	jo, da lä akta vuoberis, gallga giesset ulgus\\
	jo da lä akta {vuoberis\footnotemark} gallga giesse-t ulgus\\
	yes then be\BS\Sc{3sg.prs} one buck\BS\Sc{nom.sg} will\BS\Sc{3sg.prs} pull-\Sc{inf} out\\\nopagebreak
\Transl{yes, it’s a 3-year old reindeer buck, (he) will pull (it) out}{} \Corpus{080909}{016-017}
\footnotetext{Specifically, a \It{vuoberis} is a three-year-old reindeer buck, but the gloss has been shortened to save space.}
\z
\ea\label{future3}%
\glll	gallgap dav girjev ådtjot\\
	gallga-p d-a-v girje-v ådtjo-t\\
	will-\Sc{1pl.prs} \Sc{dem}-\Sc{dist}-\Sc{acc.sg} book-\Sc{acc.sg} get-\Sc{inf}\\\nopagebreak
\Transl{we will get that book}{} \Corpus{110517b2}{022}
\z

The modal verb \It{gallgat} ‘will’ is often used in conditional\is{conditional} clauses%\marginpar{does gallgat in conditionals need more explanation?}
, as in \REF{condClause1}.
\ea\label{condClause1}%
\glll	jus galga sáme viesov valldet, dä galga mielagav dal navt rutastit\\
	jus galga sáme vieso-v vallde-t dä galga mielaga-v dal navt rutasti-t\\
	if will\BS\Sc{3sg.prs} Saami\BS\Sc{gen.sg} life-\Sc{acc.sg} take-\Sc{inf} then will\BS\Sc{3sg.prs} sternum-\Sc{acc.sg} then so pull-\Sc{inf}\\\nopagebreak
\Transl{if I choose Saami style, then I will pull the sternum like this}{} \Corpus{080909}{097}
\z


\subsubsection{The aspectual auxiliary verb \It{årrot}}\label{auxV}
The auxiliary verb\is{auxiliary verb} \It{årrot} ‘be’ %\marginpar{check to see if årrot can really be claimed to be the infinitive form, see if “will have gone” is possible, for instance} 
together with a non-finite complement verb\is{non-finite verb form} is used to mark the perfective and progressive aspects\is{aspect}. This auxiliary verb is homophonous with the copula verb\is{copular verb}, and is also glossed as ‘be’.
In the perfective aspect, the complement verb is in a non-finite form marked by the suffix \It{-m} as in \REF{perfClause1} and \REF{perfClause2}, while the progressive non-finite verb is marked by the suffix \It{-min} %\marginpar{should progress be analyzed as -m-in ‘perf-prog’? if so, why?} 
as in \REF{progClause1} and \REF{progClause2}, respectively.
\ea\label{perfClause1}%
\glll	denne liv riegadam\\
	denne li-v riegada-m\\
	there be-\Sc{1sg.prs} be.born-\Sc{prf}\\\nopagebreak
\Transl{I was born there}{} \Corpus{090702}{008}
\z
\ea\label{perfClause2}%maybe move this NEG example to the NEG section?!
\glll	lä dån mannam nagin bále ja tjuvvum Vistegij?\\
	lä dån manna-m nagin bále ja tjuvvu-m Visteg-ij\\
	be\BS\Sc{2sg.prs} \Sc{2sg.nom} go-\Sc{prf} some time\BS\Sc{gen.sg} and accompany-\Sc{prf} Vistek-\Sc{ill.sg}\\\nopagebreak
\Transl{have you ever gone and accompanied (them) to Vistek?}{} \Corpus{080924}{630}
\z
\ea\label{progClause1}%not sure about spelling/translation for ‘rudastet’
\glll	men mån lev tjåjvev ruhtastemin ullgus\\
	men mån le-v tjåjve-v ruhtaste-min ullgus\\
	but \Sc{1sg.nom} be-\Sc{1sg.prs} stomach-\Sc{acc.sg} cut-\Sc{prog} out\\\nopagebreak
\Transl{but I’m cutting out the stomach}{} \Corpus{080909}{054}
\z
\ea\label{progClause2}
\glll	nå, mav lä låhkåmin?\\
	nå ma-v lä låhkå-min\\
	well what-\Sc{acc.sg} be\BS\Sc{3sg.prs} read-\Sc{prog}\\\nopagebreak
\Transl{well, what is he studying?}{(lit.: reading)} \Corpus{080924}{667}
\z


\subsubsection{The negation verb}\label{negation}
\PS\ clause negation is expressed by a special finite negation\is{negation} verb. Unlike other verbs, the negation verb does not have an infinitive or any other non-finite form\is{non-finite verb form}, but is always a finite verb (cf. \SEC\ref{theNegationVerb}). As such, it always agrees\is{agreement} in person and number with the subject of the clause, and inflects for tense and mood as well.\footnote{In this respect, \PS\ differs significantly from for instance North Saami negative clauses in which the main verb and not the finite negation verb inflects for tense \citep[cf.][92]{Svonni2009}.} 
The complement verb occurs in a special non-finite form called the connegative.
Examples for the negative verb can be found in \REF{negation1} through \REF{negation3}. 
\ea\label{negation1}%
\glll	iv jáhke\\
	i-v jáhke\\
	\Sc{neg-1sg.prs} believe\BS\Sc{conneg} \\\nopagebreak
\Transl{I don’t believe so}{} \Corpus{090702}{411}
\z
\ea\label{negation2}%
\glll	ittjij åbbå gävdno vuodja åsstet\\
	ittji-j åbbå gävdno vuodja åsste-t\\
	\Sc{neg-3sg.pst} at.all exist\BS\Sc{conneg} butter\BS\Sc{nom.sg} buy-\Sc{inf}\\\nopagebreak
\Transl{there wasn’t any butter to buy at all}{} \CorpusLink{080708_Session03}{080708\_Session03}{006}
\z
\ea\label{negation3}%
\glll	men ijtjin del bårå dan sisste \\
	men ijtji-n del bårå d-a-n sisste \\
	but \Sc{neg-3pl.pst} then eat\BS\Sc{conneg} \Sc{dem}-\Sc{dist}-\Sc{gen.sg} out\\\nopagebreak
\Transl{but they didn’t eat out of this}{} \CorpusLink{080708_Session03}{080708\_Session03}{019}
\z
In the examples above, the non-finite complement to the connegative\is{connegative} verb is a lexical verb. In the following examples in \REF{negation4} through \REF{negation6}, the complement connegative verb is a modal or auxiliary verb whose own complement then follows in the appropriate non-finite form.
\ea\label{negation4}%
\glll	ij vanj dä máhte ilá stuor dålåv adnet\\
	ij vanj dä máhte ilá stuor dålå-v adne-t\\
	\Sc{neg}\BS\Sc{3sg.prs} well then can\BS\Sc{conneg} too big fire-\Sc{acc.sg} have-\Sc{inf}\\\nopagebreak
\Transl{but you can’t have too big of a fire}{} \Corpus{090702}{176}
\z
\ea\label{negation5}%
\glll	dä iv lä åbbå gullam dav\\
	dä i-v lä åbbå gulla-m d-a-v\\
	then \Sc{neg}-\Sc{1sg.prs} be\BS\Sc{conneg} at.all hear-\Sc{prf} \Sc{dem}-\Sc{dist}-\Sc{acc.sg}\\\nopagebreak
\Transl{I haven’t heard that at all}{} \Corpus{090702}{203}
\z
\ea\label{negation6}%
\glll	nej, mån iv lä bårråm, men Jåssjå'l bårråm\\
	nej mån i-v lä bårrå-m men Jåssjå=l bårrå-m\\
	no \Sc{1sg.nom} \Sc{neg}-\Sc{1sg.prs} be\BS\Sc{conneg} eat-\Sc{prf} but Josh\BS\Sc{nom.sg}=be\BS\Sc{3sg.prs} eat-\Sc{prf}\\\nopagebreak
\Transl{no, I haven’t eaten (it), but Josh has eaten (it)}{} \Corpus{090519}{147}
\z

While \PS\ constituent order\is{constituent order} is generally flexible (cf. \SEC\ref{constituentOrderClauses}), there are no examples in the corpus of the verb of negation occurring after the negated complement, but instead the connegative complement verb always follows the finite negation verb in a clause. 


\section{Interrogative clauses}\label{interrogClauses}\is{interrogative}
Constituent interrogative clauses in \PS\ are consistently marked as such syntactically, and thus are distinct from declarative clauses. Polar interrogative clauses, on the other hand, do not differ significantly from declarative clauses, although some syntactic constructions are more common than others. The following sections deal first with constituent interrogative clauses, then polar interrogative clauses in more detail.

\subsection{Constituent interrogative clauses}\label{constituentQs}
Constituent interrogative clauses\is{constituent interrogative} are the only type of independent clause in \PS\ which is consistently marked syntactically as a clause type. Specifically, every constituent interrogative clause is marked as such by having an interrogative\is{interrogative} word or phrase in clause-initial position. When it is an interrogative pronoun, it inflects for case and number consistent with its grammatical role in the clause (as with any pronoun), while the humanness of its (expected) referent determines the choice of the root.\footnote{Interrogative pronouns referring to non-human NPs all feature a \It{ma-} root, while those referring to human NPs have a \It{ge-} root. Interrogative adverbs also begin with \It{g-}. Cf. \SEC\ref{interrogativePronouns}.} 
Some examples can be found in \REF{questionWordQ1} through \REF{questionWordQ4}. 
\ea\label{questionWordQ1}%
\glll	mav dån hålå?\\
	ma-v dån hålå\\
	what-\Sc{acc.sg} \Sc{2sg.nom} say\BS\Sc{2sg.prs}\\\nopagebreak
\Transl{what are you saying?}{} \Corpus{090519}{329}
\z
\ea\label{questionWordQ2}%
\glll	majd dä viehkedi?\\
	ma-jd dä viehkedi\\
	what-\Sc{acc.pl} then help\BS\Sc{2sg.pst}\\\nopagebreak
\Transl{what then did you help (with)?}{} \Corpus{080924}{615}
\z
\ea\label{questionWordQ3}%
\glll	gejna dä tjuovvo\\
	ge-jna dä tjuovvo\\
	who-\Sc{com.sg} then accompany\BS\Sc{2sg.pst}\\\nopagebreak
\Transl{who did you go with?}{} \Corpus{080924}{071}
\z
\ea\label{questionWordQ4}%
%\glll	nå, man ednak biejve galga danne årrot\\
%	nå man ednak biejve galga danne årro-t\\
%	well how many day\BS\Sc{nom.pl} will\BS\Sc{2sg.prs} there be-\Sc{2sg.pst}\\\nopagebreak
%\Transl{well, how many days will you be there?}{} \Corpus{080924}{658}
\glll	man mällgadav ana dajd riehpenen\\
	man mällgada-v ana d-a-jd riehpene-n\\
	how long-\Sc{acc.sg} have\BS\Sc{2sg.prs} \Sc{dem}-\Sc{dist}-\Sc{acc.pl} smoke.hole-\Sc{iness.sg}\\\nopagebreak
\Transl{for how long do you have those in the smoke hole?}{} \Corpus{090702}{168}
\z

Alternatively, the interrogative can be an adverb\is{adverb}, as in \REF{questionWordQ6} through \REF{questionWordQ8}. %\Red{intonation? nothing special? (not syntactic anyway).}
\ea\label{questionWordQ6}%
\glll	gukt lä dát\\
	gukte lä d-á-t\\
	how be\BS\Sc{3sg.prs} \Sc{dem}-\Sc{prox}-\Sc{nom.sg}\\\nopagebreak
\Transl{how is it?}{} \Corpus{080924}{130}
\z
\ea\label{questionWordQ7}%
\glll	guste dån bådá\\
	guste dån bådá\\
	from.where \Sc{2sg.nom} come\BS\Sc{2sg.prs}\\\nopagebreak
\Transl{where are you coming from?}{} \Corpus{080924}{003}
\z
\ea\label{questionWordQ8}%
\glll	gånne dajt tjogidä\\
	gånne d-a-jt tjogi-dä\\
	where \Sc{dem}-\Sc{dist}-\Sc{acc.pl} pick-\Sc{2pl.pst}\\\nopagebreak
\Transl{where did you pick those?}{} \Corpus{080924}{168}
\z

Assuming that any constituent which is the pragmatic focus\is{focus} can be marked by fronting, as preliminarily asserted in \SEC\ref{infoStructure}, then the fronting of the interrogative word is consistent with focus-marking. However, for constituent interrogative clauses, fronting is then obligatory. 
The rest of the clause is constructed syntactically just as freely as any declarative\is{declarative} clause would be. While subject-verb inversion can occur, the flexible nature of \PS\ constituent ordering prevents this from necessarily marking a clause as interrogative. 

It is worth noting that the discourse marker \It{nå}, which can be translated as ‘well’ or sometimes ‘yes’, frequently precedes constituent interrogative clauses, as in \REF{nåQ1}. However, it is not obligatory, nor is it restricted to interrogative clauses. It is likely a discourse marker, perhaps simply indicating the speaker’s active interest in the conversation.
\ea\label{nåQ1}%
\glll	nå gukte lij Áhkabákten gu dånnå lidje mánná?\\
	nå gukte li-j Áhkabákte-n gu dånnå lidje mánná\\
	well how be-\Sc{3sg.pst} Áhkkabákkte-\Sc{iness.sg} when \Sc{2sg.nom} be\BS\Sc{2sg.pst} child\BS\Sc{nom.sg}\\\nopagebreak
\Transl{well what was Áhkkabákkte like when you were a child?}{} \Corpus{080924}{063}
\z


\subsection{Polar interrogative clauses}\label{polarQs}\is{polar interrogative}
Because of flexible constituent ordering in Pite Saami, there is no reliable syntactic test for whether a clause is a polar interrogative. The intonation\is{intonation} of polar questions does not seem to differ significantly from any other types of clauses, either. 
However, polar interrogative clauses frequently have a constituent order\is{constituent order} in which the finite\is{finite verb} verb occurs before the subject. 
Furthermore, this finite verb is generally the first element in a clause. The examples in \REF{polarQinversion1} through \REF{polarQinversion3} illustrate this.
\ea\label{polarQinversion1}%enabu? enabuV?
\glll	galga dån ságastit enabuv?\\
	galga dån ságasti-t ena-bu-v\\
	will\BS\Sc{2sg.prs} \Sc{2sg.nom} say-\Sc{inf} much-\Sc{comp}-\Sc{acc.sg}\\\nopagebreak
\Transl{are you going to say more?}{} \CorpusLink{0906_Ahkajavvre_b}{0906\_Ahkajavvre\_b}{041}
\z
\ea\label{polarQinversion2}%from elicitation, but prompted by DS herself%not sure of the derivational morphology in suovadit, either
\glll	suovade dån?\\
	suovade dån\\
	smoke\BS\Sc{2sg.prs} \Sc{2sg.nom}\\\nopagebreak
\Transl{do you smoke?}{} \Corpus{080702b}{073}
\z
\ea\label{polarQinversion3}%
\glll	lij sån uktu jala lij Halvar aj maŋŋen\\
	li-j sån uktu jala li-j Halvar aj maŋŋen\\
	be-\Sc{3sg.pst} \Sc{3sg.nom} alone or be-\Sc{3sg.pst} Halvar\BS\Sc{nom.sg} also along\\\nopagebreak
\Transl{was he alone or was Halvar also along?}{} \Corpus{080924}{308}
\z

As with any \PS\ clause, the syntactic subject does not have to be realized overtly. In such cases, the finite verb is also usually word initial, as in \REF{polarQinversion4}. 
\ea\label{polarQinversion4}%
\glll	udtju sáme gielav danne sagastit?\\
	udtju sáme giela-v danne sagasti-t\\
	may\BS\Sc{2sg.pst} Saami\BS\Sc{gen.sg} language-\Sc{acc.sg} there speak-\Sc{inf}\\\nopagebreak
\Transl{were you allowed to speak Saami there?}{} \Corpus{080924}{351}
\z

However, it is also possible to front other elements which normally occur after the finite verb, as in \REF{polarQinversion5} and \REF{polarQinversion6}. Here, the non-finite\is{non-finite verb form} perfect form of the complement verb immediately precedes the aspect-marking\is{aspect} auxiliary verb.
\ea\label{polarQinversion5}%
\glll	juhkum lä gajtsa mielkev?\\
	juhku-m lä gajtsa mielke-v\\
	drink-\Sc{prf} be\BS\Sc{2sg.prs} goat\BS\Sc{gen.sg} milk-\Sc{acc.sg}\\\nopagebreak
\Transl{have you ever drunk goat’s milk?}{} \Corpus{080924}{128}
\z
\ea\label{polarQinversion6}
\glll	bårråm lä dån biergov danne?\\
	bårrå-m lä dån biergo-v danne\\
	eat-\Sc{prf} be\BS\Sc{2sg.prs} \Sc{2sg.nom} meat-\Sc{acc.sg} there\\\nopagebreak
\Transl{have you eaten meat there?}{} \Corpus{090519}{130}
\z


\subsubsection{Polar interrogatives and the question marker}\label{Qparticle}\is{question marker}
It is possible for polar interrogative clauses to be identified as such by a question marker \It{gu}\TILDE\It{gus} following the finite verb. However, the use of the question marker in polar interrogatives is exceptionally uncommon and can hardly be considered obligatory in current \PS\ usage; this is reflected in the data from the corpus, which contain only three tokens. See \SEC\ref{QpartWordform} for a preliminary discussion of the question marker, including the three tokens from the corpus. % in examples \REF{Qpart1} through \REF{Qpart3} on page \pageref{Qpart1}. 


\section{Clauses in the imperative mood}\label{imperClauses}\is{imperative}
Clauses in the imperative mood stand out syntactically by lacking an overt subject NP. Furthermore, they are marked by special portmanteau morphemes %\footnote{These imperative portmanteau morphemes indicate, in addition to imperative, the person and number of the syntactic subject, which is generally not overtly realized in imperative clauses.} 
on the finite verb which express imperative mood as well as the number of the implied subject of the clause, which is always 2\superS{nd} person. 
The finite verb tends to be in clause-initial position, as shown by the examples in \REF{impClause1} and \REF{impClause2}.
\ea\label{impClause1}%
\glll	giehto naginav dan Låddávre birra\\
	giehto nagina-v d-a-n {Låddávre\footnotemark} birra\\
	tell\BS\Sc{sg.imp} something-\Sc{acc.sg} \Sc{dem}-\Sc{dist}-\Sc{gen.sg} Låddávvre\BS\Sc{gen.sg} about\\\nopagebreak
\Transl{say something about this Låddávvre}{} \Corpus{080924}{314}
\z
\ea\label{impClause2}%from elicitation!
\glll	bieja pirunav bävvdaj\\
	bieja piruna-v bävvda-j\\
	put\BS\Sc{sg.imp} potato-\Sc{acc.sg} table-\Sc{ill.sg}\\\nopagebreak
\Transl{put the potato on the table}{} \CorpusE{101208}{478}
\z
\footnotetext{\It{Låddávvre} is the name of a lake.}

%However, other constituents may occur clause-initially as well, as in the standard phrase for ‘thank you’, shown in \REF{impClause3}.
Nonetheless, the standard phrase for ‘thank you’, shown in \REF{impClause3} in dual person, indicates that a constituent other than the finite verb may occur before a finite verb in imperative mood.
\ea\label{impClause3}
\glll	gijtov adnen\\
	gijto-v adne-n\\
	thank-\Sc{acc.sg} have-\Sc{du.imp}\\\nopagebreak
\Transl{thank you (two)}{(lit.: have thanks!)} \CorpusE{101208}{292}
\z
However, no other examples of such constructions are found in the corpus, and this constituent ordering may be due to this phrase being a common expression and non-productive lexicalized structure calqued from the Swedish expression \It{tack ska du ha!} (literally ‘thanks you shall have!’). 

The adverb \It{dále} ‘now’ is common in imperative clauses, and is frequently abbreviated to \It{dál}, as in \REF{impClause5}.
\ea\label{impClause5}
\glll	årren dál\\
	årre-n dál\\
	sleep-\Sc{du.imp} now\\\nopagebreak
\Transl{go to sleep now}{} \CorpusE{110518a}{06m55s}
\z


\section{Clauses in the potential mood}\label{potClauses}\is{potential}
Aside from featuring a finite verb inflected for the potential mood\footnote{Cf. \SEC\ref{POTmood} on the usage and the morphology of the potential mood.} by the \It{-tj} suffix, clauses in the potential mood generally lack an overt subject argument, as in %\marginpar{\REF{potSyntaxEx2} and \REF{potSyntaxEx3} already in other section on \POTs!} 
\REF{potSyntaxEx2} and \REF{potSyntaxEx3}. 
\ea\label{potSyntaxEx2}
\glll	nå hålåv, vuolgetjip del\\
	nå hålå-v vuolge-tji-p del\\
	well say-\Sc{1sg.prs} go-\Sc{pot}-\Sc{1pl} obviously\\\nopagebreak
\Transl{well then I say we really should probably go}{}	\Corpus{090702}{013}
\z
\ea\label{potSyntaxEx3}
\glll	nä, virtitjav nuollat\\
	nä virti-tja-v nuolla-t\\
	no must-\Sc{pot}-\Sc{1sg} undress-\Sc{inf}\\\nopagebreak
\Transl{oh no, I’ll probably have to take off some clothes}{}	\Corpus{090519}{029}
\z
However, as the clause in \REF{potSyntaxEx1} makes clear, it is possible to have an overt subject argument. 
\ea\label{potSyntaxEx1}
\glll	jus sån vuosjatja káfav\\
	jus sån vuosja-tj-a káfa-v\\
	if \Sc{3sg.nom} prepare.coffee-\Sc{pot}-\Sc{3sg} coffee-\Sc{acc.sg}\\\nopagebreak
\Transl{if he will perhaps make coffee}{}	\CorpusE{110404}{269}
\z
With this in mind, clauses in the potential mood do not differ syntactically from declarative\is{declarative} clauses. %, although the frequency of lacking overt subject arguments seems to be higher for potential mood than declaratives. 

As mentioned in \SEC\ref{POTmood}, the potential mood can also be used as a less severe command. This resembles clauses in the imperative mood by also never occurring with an overt subject, as shown %\marginpar{\REF{potSyntaxEx4} already in \SEC on \POTs!} 
in example \REF{potSyntaxEx4}. 
\ea\label{potSyntaxEx4}%
\glll	vuosjatja káfav\\
	vuosja-tj-a káfa-v\\
	prepare.coffee-\Sc{pot}-\Sc{2sg} coffee-\Sc{acc.sg}\\\nopagebreak
\Transl{perhaps you could make some coffee}{}	\CorpusE{110404}{267}
\z

%Future research could be used to shed more light on the potential syntactic constraints on clauses in the potential mood.



%%%%%%%%%%%%%%%%%%%%%%%%%%%%%%%%%
%%%%%%%%%%%%%%%%%%%%%%%%%%%%%%%%%
%%%%%%%%%%%%%%%%%%%%%%%%%%%%%%%%%
%%%%%%%%%%%%%%%%% C O M P L E X    C L A U S E S
%%%%%%%%%%%%%%%%%%%%%%%%%%%%%%%%%
%%%%%%%%%%%%%%%%%%%%%%%%%%%%%%%%%
%%%%%%%%%%%%%%%%%%%%%%%%%%%%%%%%%


\chapter{Complex clauses}\label{complexClauses}%
Two or more clauses can be conjoined by coordination or subordination. %In coordination, clauses are connected at the discourse level, while in subordination, one clause - as a unit - fulfills a constituent role in the matrix clause. 
After coordination is covered in \SEC\ref{clausalCoordination}, complement clauses are presented in \SEC\ref{complementClauses} and adverbial clauses are dealt with in \SEC\ref{adverbialClauses}. 
Finally, relative clauses which do not form a constituent of a matrix clause, but are instead part of a nominal phrase, are described in detail in this chapter as well (in \SEC\ref{relativeClauses}). 

\section{Clausal coordination}\label{clausalCoordination}\is{coordination}
There are several coordinating conjunctions that are used to syntactically join the basic clauses described in Chapter \ref{basicClauses}. In such cases, a coordinating conjunction occurs between the two clauses it connects. The clauses themselves are otherwise not marked in any way for coordination. The coordinating conjunctions are \It{ja} ‘and’, \It{vala} ‘but’, \It{men}\footnote{Note that \It{men} is a borrowing from Swedish (< \It{men} ‘but’) and is used almost exclusively in the corpus, while the native Saamic word \It{vala} is only found in a \PS\ reading based on a Lule Saami translation of the New Testament (recording \hyperlink{pit100403}{pit100403}). Several examples in \citet{Lagercrantz1926} include \It{men} (e.g. on p. 20), so it has been part of \PS\ for at least a century.} 
‘but’, \It{jala} ‘or’ and \It{eller}\footnote{Just as with \It{men}, \It{eller} is a borrowing from Swedish (< \It{eller} ‘or’); however, the native Saamic word \It{jala} is rather common as well in the corpus. Furthermore, unlike \It{men}, \It{eller} is not mentioned in \citet{Lagercrantz1926}, so it seems that \It{eller} is likely a more recent word-choice development, perhaps due to increased dominance of the Swedish language over the last century.} 
‘or’. The examples in \REF{coordination1} and \REF{coordination2} illustrate clausal coordination using the coordinators \It{ja} and \It{men}, respectively.
\ea\label{coordination1}
\glll	mån anav Árjepluove gaptev nanne, ja Ivan adna Arrvehavre gaptev\\
	mån ana-v Árjepluove gapte-v nanne ja Ivan adna Arrvehavre gapte-v\\
	\Sc{1sg.nom} have-\Sc{1sg.prs} Arjeplog\BS\Sc{gen.sg} frock-\Sc{acc.sg} on and Ivan have-\Sc{3sg.prs} Arvidsjaur\BS\Sc{gen.sg} frock-\Sc{acc.sg}\\\nopagebreak
\Transl{I have an Arjeplog frock on, and Ivan has an Arvidsjaur frock}{} \Corpus{080825}{047}
\z
\ea\label{coordination2}
\glll	men ijtjin del bårå dan siste, men ednen biebmojd biergojd ja dále návte deggara sinne\\
	men ijtji-n del bårå d-a-n siste men edne-n biebmo-jd biergo-jd ja dále návte deggara sinne\\
	but \Sc{neg}-\Sc{3pl.pst} obviously eat\BS\Sc{conneg} \Sc{dem}-\Sc{dist}-\Sc{gen.sg} out but have-\Sc{3pl.pst} food-\Sc{acc.pl} meat-\Sc{acc.pl} and now like.this such\BS\Sc{gen.pl} in\\\nopagebreak
\Transl{but they obviously didn't eat out of that, but they had food, meat and so on in such things}{} \CorpusLink{080708_Session03}{080708\_Session03}{019}
\z

When \It{jala} and \It{eller} function as clausal coordinators in the corpus, 
they are mostly used to indicate meta-language commentary showing that the second clause is an alternate or amended version of the first clause, as in \REF{ORcoordination1}, rather than to provide clause-level alternatives.
\ea\label{ORcoordination1}%
\glll	dále’l gámbal dåhpe, jala almatj hållå “unna dåbátj”\\
	dále=l gámbal dåhpe jala almatj hållå unna dåbá-tj\\
	now=be\BS\Sc{3sg.prs} old house\BS\Sc{nom.sg} or person\BS\Sc{nom.sg} say\BS\Sc{3sg.prs} small house-\Sc{dim}\BS\Sc{nom.sg}\\\nopagebreak
\Transl{now this is the old house, or one says “the little house”}{} \Corpus{100310b}{047-049}%
\z


\section{Clausal subordination}\label{clausalSubordination}\is{subordination}
Certain types of \PS\ clauses can be subordinate to another clause or to a nominal phrase\is{nominal phrase}. %are normally marked as being subordinate to a matrix phrase. 
When embedded at clause-level, a subordinate clause can be either a complement clause or an adverbial clause, depending on whether it fills an argument or an adverbial role. 
These two types of subordinate clause are described in \SEC\ref{complementClauses} and \SEC\ref{adverbialClauses}, respectively. 
Subordinate clauses featuring non-finite verb forms\is{non-finite verb form} are likely also found in \PS, a possibility which is dealt with briefly in \SEC\ref{otherSubclauses}. 
Finally, relative clauses are covered in \SEC\ref{relativeClauses}.
%A final type of clausal subordination, when embedded in an NP, a subordinate clause is a relative clause, and covered in \SEC\ref{relativeClauses}.

\subsection{Complement clauses}\label{complementClauses}\is{complement clause}
A complement clause fills an argument slot of the verbal predicate in the matrix clause it belongs to. 
There are a variety of complement clause constructions, and both finite and infinitive predicates are possible. Complement\is{complement} clauses can be marked by a complementizer %, a question word or question pronoun, 
or can stand in juxtaposition to the matrix clause. %In most cases, the complement clause contains a fully inflected finite verb; however, an infinite verb form can also be part of the juxtaposition strategy. 
The different complement clause marking strategies are summarized in Table \vref{complementClauseSummary} and described in the following sections.

\begin{table}[h]\centering
\caption{Types of complement clause marking}\label{complementClauseSummary}
\begin{tabular}{ll}\mytoprule
{predicate type}	&{subordination strategy}	\\\hline%&\it  typical verbs \\\mybottomrule
{finite}	& complementizer \It{att}	\\%\cline{2-2}%\hline%& hållåt, diehtet \\\hline
%finite				& fronted question word/pronoun	\\\cline{2-2}%\hline%& diehtet \\\hline
				& juxtaposition	\\%question/relative pronoun	& finite		\\\hline%& skenet \\\hline
infinitive			& juxtaposition	\\\mybottomrule
\end{tabular}
\end{table}


\subsubsection{Complement clauses with a finite predicate}\label{finiteComplementClauses}
Complement clauses with a fully inflected finite\is{finite verb} predicate are attested using one of two strategies. 
First, the borrowed complementizer\is{complementizer} \It{att}\footnote{Cf. the Swedish marker \It{att}, which is also a complementizer.} 
can mark a complement clause. In such cases, the complement clause typically follows the matrix clause. The complementizer is in clause-initial position in the complement clause. 
Examples can be found in \REF{complementizer1} and \REF{complementizer2}.
\ea\label{complementizer1}
\glll	ja dä mån hålåv att sidav bajket\\
	ja dä mån hålå-v att sida-v bajke-t\\
	and then \Sc{1sg.nom} say-\Sc{1sg.prs} \Sc{subord} want-\Sc{1sg.prs} poop-\Sc{inf}\\\nopagebreak
\Transl{and then I say that I want to poop}{} \Corpus{080924}{591}
\z
\ea\label{complementizer2}
\glll	men mån diedav att háre lä jávren\\
	men mån dieda-v att háre lä jávre-n\\
	but \Sc{1sg.nom} know-\Sc{1sg.prs} \Sc{subord} greyling\BS\Sc{nom.pl} be\BS\Sc{3pl.prs} lake-\Sc{iness.sg}\\\nopagebreak
\Transl{but I know that there are greyling in the lake}{} \Corpus{100404}{052}
\z

Secondly, complement clauses with a finite predicate may be juxtaposed to the matrix clause they belong to. The complement clause typically follows the matrix clause. 
Verbs hosting such complements include \It{jáhkket} ‘believe’, \It{diehtet} ‘know’, \It{hållåt} ‘say’ and \It{tuhtjet} ‘like’. %, \It{} ‘’, \It{} ‘’
Examples can be found in \REF{complClauseJuxFin1} through \REF{complClauseJuxFin3}.%\marginpar{other verbs for this? skenet(understand)?}
\ea\label{complClauseJuxFin1}
\glll	mån jáhkav stuor tjuovtja lä danne\\
	mån jáhka-v stuor tjuovtj-a lä danne\\
	\Sc{1sg.nom} believe-\Sc{1sg.prs} big whitefish-\Sc{nom.pl} be\BS\Sc{3pl.prs} there\\\nopagebreak
\Transl{I believe there are big whitefish there}{} \Corpus{090702}{123}
\z
\ea\label{complClauseJuxFin2}%
\glll	men mån tuhtjiv dat lij nav suohtas tieltajn viessot\\
	men mån tuhtji-v d-a-t lij nav suohtas tielta-jn viesso-t\\
	but \Sc{1sg.nom} think-\Sc{1sg.prs} \Sc{dem}-\Sc{dist}-\Sc{nom.sg} be\BS\Sc{3sg.pst} so nice tent-\Sc{iness.pl} live-\Sc{inf}\\\nopagebreak
\Transl{but I think it was so nice to stay in tents}{} \Corpus{080924}{644}
\z
\ea\label{complClauseJuxFin3}%
\glll	men hålåv, vuhtjijmä mija sárvav\\
	men hålå-v vuhtji-jmä mija sárva-v\\
	but say-\Sc{1sg.prs} shoot-\Sc{1pl.pst} \Sc{1pl.nom} moose-\Sc{acc.sg}\\\nopagebreak
\Transl{but then I say we shot a moose}{} \Corpus{090702}{404}
\z

Constituent interrogative clauses\is{constituent interrogative} can also be juxtaposed complement clauses. As with any such interrogative clause, an interrogative pronoun\is{pronoun!interrogative} or other question word occurs as the initial element of the complement clause. %The question word or interrogative pronoun is in clause-initial position in the complement clause. 
This strategy typically coincides with complements of epistemic verbs such as \It{diehtet} ‘know’ or \It{skenit} ‘understand’. %, \It{} ‘’, \It{} ‘’, \It{} ‘’, \It{} ‘’
Some examples are provided in \REF{Qsubordination1} through \REF{Qsubordination3}. 
\ea\label{Qsubordination1}
\glll	mån iv diede gåsse gillgin gávnadit maŋep bále\\
	mån i-v diede gåsse gillgi-n gávnadi-t maŋe-p bále\\
	\Sc{1sg.nom} \Sc{neg-1sg.prs} know\BS\Sc{conneg} when will-\Sc{1du.prs} meet-\Sc{inf} after-\Sc{comp} time\BS\Sc{gen.sg}\\\nopagebreak
\Transl{I don’t know when we’ll meet next time}{} \Corpus{081011}{183}
\z
\ea\label{Qsubordination2}
\glll	mån iv skene mav dån hålå\\
	mån i-v skene ma-v dån hålå\\
	\Sc{1sg.nom} \Sc{neg-1sg.prs} understand\BS\Sc{conneg} what-\Sc{acc.sg} \Sc{2sg.nom} say\BS\Sc{2sg.prs}\\\nopagebreak
\Transl{I don’t understand what you’re saying}{} \CorpusE{080926}{05m14s}
\z
\ea\label{Qsubordination3}
\glll	mån diedav gie lä\\
	mån dieda-v gie lä\\
	\Sc{1sg.nom} know-\Sc{1sg.prs} who\BS\Sc{nom.sg} be\BS\Sc{3sg.prs}\\\nopagebreak
\Transl{I know who she is}{} \Corpus{090702}{460}
\z

While the complement clause typically follows the matrix clause, this does not necessarily have to be the case, as illustrated by \REF{Qsubordination4}.% and \REF{Qsubordination5}.
\ea\label{Qsubordination4}
\glll	man mälgat lij gu lij hiejman, iv mån diede\\
	man mälgat li-j gu li-j hiejma-n i-v mån diede\\
	how far be-\Sc{3sg.pst} when be-\Sc{3sg.pst} home-\Sc{iness.sg} \Sc{neg-1sg.prs} \Sc{1sg.nom} know\BS\Sc{conneg}\\\nopagebreak
\Transl{I don’t know how far it was to get home}{} \Corpus{100404}{317}
\z



\subsubsection{Complement clauses with an infinitive predicate}\label{infinitiveComplementClauses}
Complement clauses with an infinitive\is{non-finite verb form} predicate can be juxtaposed to the matrix clause they belong to. The complement clause typically follows the matrix clause. 
While not particularly common in the corpus, verbs such as \It{állget} ‘begin’ and \It{vajáldahtet}/\It{åjaldahtet}\footnote{The word \It{vajáldahtet} ‘forget’ is likely limited to the northern dialects of \PS, while \It{åjaldahtet} is preferred in the south.} 
‘forget’ %\marginpar{other verbs for this?} %\It{} ‘’, \It{} ‘’, \It{} ‘’
are accompanied by complement clauses headed by an infinitive verb, as in \REF{complClauseJuxInf1} through \REF{complClauseJuxInf4}.
\ea\label{complClauseJuxInf1}
\glll	nå gosse dijá älgijdä Örnvikast vuodjet vadnásav?\\
	nå gosse dijá älgi-jdä Örnvika-st vuodje-t vadnása-v\\
	well when \Sc{2pl.nom} begin-\Sc{2pl.pst} Örnvik-\Sc{elat.sg} drive-\Sc{inf} boat-\Sc{acc.sg}\\\nopagebreak
\Transl{well when did you start to take the boat from Örnvik?}{} \Corpus{080924}{563}
\z
\ea\label{complClauseJuxInf2}
\glll	ja dä del virrtin allget biebmat, fodderijd vuodjet\\
	ja dä del virrti-n allge-t biebma-t fodderi-jd vuodje-t\\
	and then obviously must-\Sc{3pl.pst} begin-\Sc{inf} feed-\Sc{inf} feed-\Sc{acc.pl} drive-\Sc{inf}\\\nopagebreak
\Transl{and so they obviously had to start to feed, to transport the feed}{} \Corpus{100405a}{029}
\z
\ea\label{complClauseJuxInf3}%
\glll	nä, mån liv åjaldahtam valldet maŋen\\
	nä mån li-v åjaldahta-m vallde-t maŋen\\
	no \Sc{1sg.nom} be-\Sc{1sg.prs} forget-\Sc{prf} take-\Sc{inf} with\\\nopagebreak
\Transl{no, I forgot to take it along}{} \Corpus{090519}{322}
\z
\ea\label{complClauseJuxInf4}%
\glll	vajálduhtiv hållåt, gu vusjkonijd dihkiv...\\
	vajálduhti-v hållå-t gu vusjkoni-jd dihki-v\\
	forget-\Sc{1sg.prs} say-\Sc{inf} when perch-\Sc{acc.pl} do-\Sc{1sg.pst}\\\nopagebreak
\Transl{I am forgetting to say, when I did the perch...}{} \Corpus{090702}{079}
\z



\subsection{Adverbial clauses}\label{adverbialClauses}\is{adverbial clause}
An adverbial clause is a subordinate clause that fills an adverbial function in the matrix clause. %, e.g., indicating when, how or why something happens. 
Adverbial clauses begin with a subordinating element such as \It{gu} ‘when, once’, \It{jus} ‘if’, \It{maŋŋel} ‘after’, \It{åvdål} ‘before’, \It{innan}\footnote{Note that the conjunction \It{innan} is a borrowing from Swedish and is only attested once in the corpus.} ‘before’ %\marginpar{add to list of subordinators!}
or \It{gukte} ‘how’, %\marginpar{\It{danen} ‘because’ requires following \It{gu}, so danen is not really a subordinator! cf. \ref{adverbialClause2}} 
but otherwise are not marked syntactically as subordinate clauses. The adverbial clause itself is headed by a fully inflected finite verb\is{finite verb}.% agreeing in person and number with the syntactic subject and inflecting for tense\marginpar{are other moods possible?}.

For instance, the example in \REF{adverbialClause1} shows that the adverbial clause can follow the matrix clause.%, while the clause in \REF{adverbialClause3} provides an example for an adverbial with 
\ea\label{adverbialClause1}%
\glll	hihtu vanj dä baktjat innan mån stärtiv motorav\\
	hihtu vanj dä baktja-t innan mån stärti-v motora-v\\
	must\BS\Sc{2sg.prs} well then back-\Sc{inf} before \Sc{1sg.nom} start-\Sc{1sg.prs} motor-\Sc{acc.sg}\\\nopagebreak
\Transl{well you have to back up then before I start the motor}{} \Corpus{090702}{018-019}
\z

In the example in \REF{adverbialClause2}, the dependent complement clause \It{gu lidjin sladjim} ‘once they had harvested’ precedes the matrix clause \It{dä båhtin da bajás} ‘then they came up’. %could stand alone syntactically, but is supplemented by the preceding .
\ea\label{adverbialClause2}%
\glll	gu lidjin sladjim, dä båhtin da bajás\\
	gu lidji-n sladji-m dä båhti-n d-a bajás\\
	when be-\Sc{3pl.pst} harvest-\Sc{prf} then come-\Sc{3pl.pst} \Sc{dem}-\Sc{dist}\BS\Sc{sg.nom} up\\\nopagebreak
\Transl{once (the farmers) had harvested, then they (the plants) came up}{} \Corpus{080924}{173}
\z



Adverbial clauses introduced by the subordinator \It{jus} ‘if’ set a condition for the matrix sentence. Other than this conjunction, there is no special marking for the conditional. 
%Conditional clauses are adverbial clauses introduced by the subordinator \It{jus} ‘if, whether’. The finite verb in both the subordinate clause and the superordinate clause is not marked in any way for conditionality, as in \Ref {conditionalClause1} and \REF{conditionalClause2}. 
The conditional clause can occur before or after the matrix clause, as shown in \REF{conditionalClause1} through \REF{conditionalClause3}.%
%Examples for conditional clauses in \ref{conditionalClause1} through \ref{conditionalClause2}.
%jus gussa dajd olli dä borre dajd dija gabmasuijnid%if the cows reach up to it, then they eat your shoe hay%pit080924.221
\ea\label{conditionalClause1}%
\glll	jus gussa dajd ulli, dä bårre dajd, dija gamasuijnijd\\
	jus gussa d-a-jd ulli dä bårre d-a-jd dija gama-suijni-jd\\
	if cow\BS\Sc{nom.pl} \Sc{dem}-\Sc{dist}-\Sc{acc.pl} reach-\Sc{3pl.prs} then eat\BS\Sc{3pl.prs} \Sc{dem}-\Sc{dist}-\Sc{acc.pl} \Sc{2pl.gen} shoe-hay-\Sc{acc.pl}\\\nopagebreak
\Transl{if the cows reach up to it, then they eat it, your shoe hay\footnotemark}{} \Corpus{080924}{221}\footnotetext{Note that in example \REF{conditionalClause1}, ‘hay’ and both pronouns referring to ‘hay’ are plural; however, for ease of reading, these are singular in the English translation. \It{Gamasuäjdne} ‘shoe-hay’ refers to a special type of grass placed inside shoes to insulate one’s feet from cold and moisture.}
\z
\ea\label{conditionalClause2}%
\glll	ja dat lij samma, jus del lij guallbana vaj bijjadaga vaj smav biehtsasdaga\\
	ja d-a-t li-j samma jus del lij guallban-a vaj bijjadag-a vaj smav biehtsasdag-a\\
	and \Sc{dem}-\Sc{dist}-\Sc{nom.sg} be-\Sc{3sg.pst} same if then be-\Sc{3sg.pst} flat.pine.heath-\Sc{nom.pl} or high.ground-\Sc{nom.pl} or small pine.forest-\Sc{nom.pl}\\\nopagebreak
\Transl{and that was the same whether it was flat-pine-heath or higher-ground or small pine-forests}{} \Corpus{100405a}{009}
\z
\ea\label{conditionalClause3}%
\glll	jus galga njuallga dajd njuovvat dä galga dajd valdet olgus åvdål gádsastam\\
	jus galga njuallga d-a-jd njuovva-t dä galga d-a-jd valde-t olgus åvdål gádsasta-m\\
	if will\BS\Sc{2sg.prs} correct \Sc{dem}-\Sc{dist}-\Sc{acc.pl} slaughter-\Sc{inf} then will-\Sc{2sg.prs} \Sc{dem}-\Sc{dist}-\Sc{acc.pl} take-\Sc{inf} out before hang-\Sc{prf}\\\nopagebreak
\Transl{if you slaughter them correctly, then you take them out before hanging (them) up}{} \Corpus{080909}{105}
\z


\subsection{Other subordinate clauses with non-finite verb forms}\label{otherSubclauses}\is{non-finite verb form}
The literature on Saami languages often mentions other non-finite verb forms in addition to those mentioned above, which can be considered part of non-finite subordinate clauses often in adverbial function. These include the verb genitive, verb abessive or gerunds, for instance (cf. \citet[103-104]{Sammallahti1998} and \citet[67-73]{Svonni2009} for North Saami, or \citet[104-111]{Spiik1989} for Lule Saami). 
For \PS, \citet[95-106]{Lehtiranta1992} describes the morphological form for a number of such non-finite forms,\footnote{The non-finite forms are also included in the verb paradigms in \citet[150-155]{Lehtiranta1992}.} 
but does not go into how these are used syntactically, and only one or two example clauses are provided at all. \citet{Lagercrantz1926} does not describe such verb forms. 

With this in mind, it is certainly plausible that \PS\ can make use of non-finite verb forms other than those mentioned above. However, there is little evidence of such forms in the corpus, and even this is limited to progressive forms in elicitation sessions. For instance, as mentioned in \SEC\ref{ADVverbs}, the progressive non-finite verb form can be used adverbially. One example featuring the progressive form \It{gullamin} ‘listening’ is repeated here in \REF{ADVverbsEx2repeat}. Even here, it is not clear whether such non-finite forms can include arguments or adjuncts. 
\ea\label{ADVverbsEx2repeat}%
\glll	gullamin mån tjálav\\
	gulla-min mån tjála-v\\
	listen-\Sc{prog} \Sc{1sg.nom} write-\Sc{1sg.prs}\\\nopagebreak
\Transl{I write while listening}{}	\CorpusE{110404}{089}
\z
Ultimately, the syntactic behavior of such non-finite verbs, and whether these can be part of subordinate clauses, must be left for future study.  



\subsection{Relative clauses}\label{relativeClauses}\is{relative clause}
\PS\ relative clauses are marked by a clause-initial relative pronoun\is{pronoun!relative}. The fact that this relative pronoun  
%pivot NP within the relative clause itself is never a full NP, but is instead reduced to a relative pronoun. This relative pronoun 
is always the initial constituent\is{constituent order} in the relative clause is the only internal syntactic marking for relative clauses; otherwise, relative clauses are ordinary clauses with a fully inflected finite verb\is{finite verb}. %The relative clause is in most respects a regular finite clause containing a fully inflected finite verb, however it is syntactically marked by always having a relative pronoun in clause-initial position. 
The relative pronoun %links the relativized noun phrase to the the relative clause, and 
inflects for case according to the syntactic function it fills within the relative clause, and for the number of the NP\is{nominal phrase} that it modifies, as illustrated by \REF{relClause1} through \REF{relClause3}.
\ea\label{relClause1}%
\glll	dä inijmä aktav vuoksav majna vuojadijmä muorajd\\
	dä ini-jmä akta-v vuoksa-v ma-jna vuojadi-jmä muora-jd\\
	then have-\Sc{1pl.pst} one-\Sc{acc.sg} bull-\Sc{acc.sg} \Sc{rel}-\Sc{com.sg} drive-\Sc{1pl.pst} wood-\Sc{acc.pl}\\\nopagebreak
\Transl{we had one bull with which we transported firewood}{} \CorpusLink{0906_Ahkajavvre_a}{0906\_Ahkajavvre\_a}{020}
\z
\ea\label{relClause2}%
\glll	ja dä maŋŋemus skoterijd majd iniga\\
	ja dä maŋŋe-mus skoteri-jd ma-jd ini-ga\\
	and then after-\Sc{superl} snowmobile-\Sc{acc.pl} \Sc{rel}-\Sc{acc.pl} have-\Sc{3du.pst}\\\nopagebreak
\Transl{…and the last snowmobiles which they had}{} \Corpus{100404}{281}
\z
\ea\label{relClause3}%
\glll	dä lä ájge ma lä urrum\\
	dä lä ájge ma lä urru-m \\
	then be\BS\Sc{3pl.prs} time\BS\Sc{nom.pl} \Sc{rel}\BS\Sc{nom.pl} be\BS\Sc{3pl.prs} be-\Sc{prf}\\\nopagebreak
\Transl{those are times which have been}{(i.e.: ‘those were the good old days’)} \Corpus{090702}{409}
\z

Just as demonstrative\is{pronoun!demonstrative} and interrogative\is{pronoun!interrogative} pronouns, relative pronouns only inflect for singular and plural, but not for dual\is{dual} number, as illustrated by the example in \REF{relClause4}. 
\ea\label{relClause4}%
\glll	måj ma lin båhtam\\
	måj ma li-n båhta-m\\
	\Sc{1du.nom} \Sc{rel}\BS\Sc{nom.pl} be-\Sc{3pl.pst} come-\Sc{prf}\\\nopagebreak
\Transl{we two who had come}{} \CorpusE{110329}{32m45s}
\z

Note that the relative pronouns are homophonous with the %\marginpar{is ‘impersonal’ the right term for (-hum)?} 
set of interrogative pronouns referring to non-human NPs.\footnote{Cf. \SEC\ref{interrogativePronouns} and \SEC\ref{relativePronouns} for more on interrogative and relative pronouns, respectively.} 
However, not only do relative pronouns have a different syntactic function than interrogative pronouns in general, they are not sensitive to the humanness of the referent, unlike interrogative pronouns. For instance, the relative pronoun \It{ma} is the same in both \REF{relClause3} above and \REF{relClause5} below, although the former has ‘times’ as an antecedent and the latter refers to ‘young people’.
\ea\label{relClause5}%
\glll	dä lin nuora álmatja ma lin riejdnohimen\\
	dä li-n nuora álmatj-a ma li-n riejdnohi-men\\
	then be-\Sc{3pl.pst} young\BS\Sc{pred.pl} people-\Sc{nom.pl} \Sc{rel}\BS\Sc{nom.pl} be-\Sc{3pl.pst} herd-\Sc{prog}\\\nopagebreak
\Transl{they were young people who were herding}{} \CorpusLink{0906_Ahkajavvre_b}{0906\_Ahkajavvre\_b}{017}
\z

In the previous examples, relative clauses immediately follow the head of the noun phrase they modify. However, it is possible for a postposition\is{postposition} to occur between the modified NP and the relative clause, as illustrated by \REF{relClause6}. 
\ea\label{relClause6}%
\glll	dat lij duv gugu masa båhten\\
	d-a-t li-j duv gugu ma-sa båhte-n\\
	\Sc{dem}-\Sc{dist}-\Sc{nom.sg} be-\Sc{3sg.pst} \Sc{2sg.gen} to \Sc{rel}-\Sc{ill.sg} come-\Sc{3pl.pst}\\\nopagebreak
\Transl{it was to you they came}{} \CorpusE{110329}{37m04s}
\z
This shows that it is possible, in this case, for a relative clause to not be embedded syntactically in the modified NP, as the relative clause can occur outside the postpositional phrase\is{postpositional phrase} which the modified NP is a constituent of. It should be emphasized that only a postposition can split a relative clause from the matrix NP it modifies.

There does not appear to be any restriction on the syntactic function that a relative pronoun can fill within the relative clause. With the exception of abessive\is{abessive} and essive\is{essive}, which are rare in the corpus for all nominals, relative pronouns are attested for all grammatical cases, as well as being the dependent NP in a postpositional phrase (in genitive\is{genitive} case), as in \REF{relClause7}, or similarly as the possessor NP (also in genitive case) modifying a noun, as in \REF{relClause8}. 
\ea\label{relClause7}%
\glll	dat lä náhppe man sisa båhtjen buhtsujd\\
	d-a-t lä náhppe ma-n sisa båhtje-n buhtsu-jd\\
	\Sc{dem}-\Sc{dist}-\Sc{nom.sg} be\BS\Sc{3sg.prs} milking.bowl\BS\Sc{nom.sg} \Sc{rel}-\Sc{gen.sg} into milk-\Sc{3pl.pst} reindeer-\Sc{acc.pl}\\\nopagebreak
\Transl{this is a milking bowl into which they milked reindeer}{} \CorpusLink{080708_Session02}{080708\_Session02}{003}
\z
\ea\label{relClause8}%
\glll	men dä lä danne urrum dat pluovve man namma lä, mij lä namma dan, dáv mijá Árjepluovev\\
	men dä lä danne urru-m d-a-t pluovve ma-n namma lä mij lä namma d-a-n d-á-v mijá Árjepluove-v\\
	but then be\BS\Sc{3sg.prs} there be-\Sc{prf} \Sc{dem}-\Sc{dist}-\Sc{nom.sg} pond\BS\Sc{nom.sg} \Sc{rel}-\Sc{gen.sg} name\BS\Sc{nom.sg} be\BS\Sc{3sg.prs} \Sc{rel}\BS\Sc{nom.sg} be\BS\Sc{3sg.prs} name\BS\Sc{nom.sg} \Sc{dem}-\Sc{dist}-\Sc{gen.sg} \Sc{dem}-\Sc{prox}-\Sc{acc.sg} \Sc{1pl.gen} Arjeplog-\Sc{acc.sg}\\\nopagebreak
\Transl{but here was the pond whose name was, which was the name of that, this, our Arjeplog}{} \Corpus{090915}{013}
\z
This latter example is clear evidence for such a structure, but it is part of a false-start, as it also contains a semantically-driven self-correction just after the targeted example; however, as the single instance in the corpus for a relative pronoun modifying a noun, it must suffice as evidence at this point. % until further research is done. 

%A summary of the syntactic functions that relative pronouns can fulfill within a relative clause is provided in Table \vref{syntacticFunctionsRelPronouns}. %, including the numbers of examples which provide evidence for each function. 
%\begin{table}[h]\centering
%\caption[Possible syntactic functions of relative pronouns]{Possible syntactic functions of relative pronouns}\label{syntacticFunctionsRelPronouns}
%\begin{tabular}{lc}\mytoprule
%				&{possible for}	\\%\dline
%{syntactic function}	&{relative pronoun}	\\\hline
%argument NP		&\CH			\\
%adjunct NP		&\CH			\\
%dependent of PP	&\CH			\\
%possessor of NP	&\CH			\\\mybottomrule
%\end{tabular}
%\end{table}

In summary, relative pronouns can fulfill a variety of syntactic functions in a relative clause. Specifically, relative pronouns can be  
an argument NP, 
an adjunct NP, 
a dependent of a PP, or 
a possessor of an NP.




%%%%%%% THIS IS NOT USED FOR THE ENTIRE COMPILATION, but only for individual chapters!!!!

\clearpage
\addcontentsline{toc}{chapter}{Bibliography}\label{Bibliography}
\bibliography{PiteGrammarBibSDL}%for bibtex
%\printbibliography%[title=Works Cited]%%for biber!






%%%NAME INDEX doesn’t work!?!? why???
\cleardoublepage\phantomsection%this allows hyperlink in ToC to work
\addcontentsline{toc}{chapter}{Name index}
\ohead{Name index}
\printindex[aut]

\cleardoublepage\phantomsection%this allows hyperlink in ToC to work
\addcontentsline{toc}{chapter}{Language index}
\ohead{Language index}
\printindex[lan]

\cleardoublepage\phantomsection%this allows hyperlink in ToC to work
\addcontentsline{toc}{chapter}{Subject index}
\ohead{Subject index}
\printindex


\end{document}