%\documentclass[ number=5
			   ,series=sidl
			   ,isbn=xxx-x-xxxxxx-xx-x
			   ,url=http://langsci-press.org/catalog/book/17
			   ,output=long   % long|short|inprep              
			   %,blackandwhite
			   %,smallfont
			   ,draftmode   
			  ]{LSP/langsci}                          

\usepackage{LSP/lsp-styles/lsp-gb4e}		% verhindert Komma bei mehrfachen Fußnoten?
                                                      
\usepackage{layout}
\usepackage{lipsum}

%%%% ABOVE FOR LangSciPress %%%%
%%%% ABOVE FOR LangSciPress %%%%
%%%% ABOVE FOR LangSciPress %%%%
\usepackage{libertine}%work-around solution for rendering problematic characters ʦ, ͡  (mostly in \textbf{})

\usepackage{longtable}%Double-lines (\hline\hline) aren’t typeset properly in ‘longtable’-environment across several pages! conflict with other package (maybe xcolor with option ‘tables’?)

\usepackage{multirow}

\usepackage{array} %allows, among other things, centering column content in a table while also specifying width, creates new column style "x" for center-alignment, "y" for right-alignment
\newcolumntype{x}[1]{>{\centering\hspace{0pt}}p{#1}}%
\newcolumntype{y}[1]{>{\raggedleft\hspace{0pt}}p{#1}}%

\usepackage[]{placeins}%using \FloatBarrier command, all floats still floating at that point will be typeset, and cannot cross that boundary. the option here \usepackage[section]{placeins} automatically adds \FloatBarrier to every \section command (only works for \section commands, nothing lower than that!)
%\usepackage{afterpage}%by using the command \afterpage{\clearpage}, all floats will appear, but no new page will be started, thus avoiding bad page breaks around floats

\usepackage{vowel} %for vowel space chart


%%%IS THIS NECESSARY??
%%%%following allows you to refer to footnotes (from http://anthony.liekens.net/index.php/LaTeX/MultipleFootnoteReferences)
%\newcommand{\footnoteremember}[2]{
%  \footnote{#2}
%  \newcounter{#1}
%  \setcounter{#1}{\value{footnote}}
%} \newcommand{\footnoterecall}[1]{
%  \footnotemark[\value{#1}]} 
%%%%previous allows you to refer to footnotes: use \footnoteremember{referenceText} in footnote, then \footnoterecall{referenceText} to refer.

\usepackage{tikz}%
\usetikzlibrary{plothandlers,matrix,decorations.text,shapes.arrows,shadows,chains,positioning,scopes}

\usepackage{synttree} %zeichnet linguistische Bäume
\branchheight{36pt}%sets height between rows in synttree

\usepackage{lscape}%used for landscape pages in index (list of recordings)

\usepackage{polyglossia}
\setmainlanguage{english}


%%%TAKE OUT FOR FINAL VERSION:
%%%TAKE OUT FOR FINAL VERSION:
%%%TAKE OUT FOR FINAL VERSION:

%%%%following readjusts margin text!
%\setlength{\marginparwidth}{20mm}
%\let\oldmarginpar\marginpar
%\renewcommand\marginpar[1]{\-\oldmarginpar[\raggedleft\footnotesize\vspace{-7pt}\color{red}\It{→ #1}]%
%{\raggedright\footnotesize\vspace{-7pt}\color{red}\It{→ #1}}}
%%%%previous readjusts margin text!

%%%The following lines set depth of ToC (LSP default is only 3 levels)!
%%%\renewcommand{\contentsname}{Table of Contents} % überschrift des inhaltsverzeichnisses
%\setcounter{secnumdepth}{5}%sets how deep section/subsection/subsubsections are numbered
%\setcounter{tocdepth}{5}%sets the depth of the ToC %but this doesn't seem to work!!!
%% new commands for LSP book (Grammar of Pite Saami, by J. Wilbur)

\newcommand{\PS}{Pite Saami}
\newcommand{\PSDP}{Pite Saami Documentation Project}
\newcommand{\WLP}{Wordlist Project}

\newcommand{\HANG}{\everypar{\hangindent15pt \hangafter1}}%also useful for table cells
\newcommand{\FB}{\FloatBarrier}%shortcut for this command to print all floats w/o pagebreak

\newcommand{\REF}[1]{(\ref{#1})}%adds parenthesis around the reference number, particularly useful for examples.%\Ref had clash with LSP!
\newcommand{\dline}{\hline\hline}%makes a double line in a table
\newcommand{\superS}[1]{\textsuperscript{#1}}%adds superscript element
\newcommand{\sub}[1]{$_{#1}$}%adds subscript element
\newcommand{\Sc}[1]{\textsc{#1}}%shortcut for small capitals (not to be confused with \sc, which changes the font from that point on)
\newcommand{\It}[1]{\textit{#1}}%shortcut for italics (not to be confused with \it, which changes the font from that point on)
\newcommand{\Bf}[1]{\textbf{#1}}%shortcut for bold (not to be confused with \bf, which changes the font from that point on)
\newcommand{\BfIt}[1]{\textbf{\textit{#1}}}
\newcommand{\BfSc}[1]{\textbf{\textsc{#1}}}
\newcommand{\Tn}[1]{\textnormal{#1}}%shortcut for normal text (undo italics, bolt, etc.)
\newcommand{\MC}{\multicolumn}%shortcut for multicolumn command in tabular environment - only replaces command, not variables!
\newcommand{\MR}{\multirow}%shortcut for multicolumn command in tabular environment - only replaces command, not variables!
\newcommand{\TILDE}{∼}%U+223C %OLD:~}%shortcut for tilde%command ‘\Tilde’ clashes with LSP!%
\newcommand{\BS}{\textbackslash}%backslash
\newcommand{\Red}[1]{{\color{red}{#1}}}%for red text
\newcommand{\Blue}[1]{{\color{blue}{#1}}}%for blue text
\newcommand{\PLUS}{+}%nicer looking plus symbol
\newcommand{\MINUS}{-}%nicer looking plus symbol
%    Was die Pfeile betrifft, kannst Du mal \Rightarrow \mapsto \textrightarrow probieren und dann \mathbf \boldsymbol oder \pbm dazutun.
\newcommand{\ARROW}{\textrightarrow}%→%dieser dicke Pfeil ➜ wird nicht von der LSP-Font unterstützt: %\newcommand{\ARROW}{{\fontspec{DejaVu Sans}➜}}
\newcommand{\DARROW}{\textleftrightarrow}%↔︎%DoubleARROW
\newcommand{\BULLET}{•}%
%%✓ does not exist in the default LSP font!
\newcommand{\CH}{\checkmark}%%\newcommand{\CH}{\fontspec{Arial Unicode MS}✓}%CH as in CHeck
%%following used to separate alternation forms for consonant gradation and umlaut patterns:
\newcommand{\Div}{‑}%↔︎⬌⟷⬄⟺⇔%non-breaking hyphen: ‑  
\newcommand{\QUES}{\textsuperscript{?}}%marks questionable/uncertain forms

\newcommand{\jvh}{\mbox{\It{j}-suffix} vowel harmony}%
%\newcommand{\Ptcl}{\Sc{ptcl} }%just shortcut for glossing ‘particle’
%\newcommand{\ATTR}{{\Sc{attributive}}}%shortcut for ATTRIBUTIVE in small caps
%\newcommand{\PRED}{{\Sc{predicative}}}%shortcut for PREDICATIVE in small caps
%\newcommand{\COMP}{{\Sc{comparative}}}%shortcut for COMPARATIVE in small caps
%\newcommand{\SUPERL}{{\Sc{superlative}}}%shortcut for SUPERLATIVE in small caps
\newcommand{\SG}{{\Sc{singular}}}%shortcut for SINGULAR in small caps
\newcommand{\DU}{{\Sc{dual}}}%shortcut for DUAL in small caps
\newcommand{\PL}{{\Sc{plural}}}%shortcut for PLURAL in small caps
%\newcommand{\NOM}{{\Sc{nominative}}}%shortcut for NOMINATIVE in small caps
%\newcommand{\ACC}{{\Sc{accusative}}}%shortcut for ACCUSATIVE in small caps
%\newcommand{\GEN}{{\Sc{genitive}}}%shortcut for GENITIVE in small caps
%\newcommand{\ILL}{{\Sc{illative}}}%shortcut for ILLATIVE in small caps
%\newcommand{\INESS}{{\Sc{inessive}}}%shortcut for INESSIVE in small caps
\newcommand{\ELAT}{{\Sc{elative}}}%shortcut for ELATIVE in small caps
%\newcommand{\COM}{{\Sc{comitative}}}%shortcut for COMITATIVE in small caps
%\newcommand{\ABESS}{{\Sc{abessive}}}%shortcut for ABESSIVE in small caps
%\newcommand{\ESS}{{\Sc{essive}}}%shortcut for ESSIVE in small caps
%\newcommand{\DIM}{{\Sc{diminutive}}}%shortcut for DIMINUTIVE in small caps
%\newcommand{\ORD}{{\Sc{ordinal}}}%shortcut for ORDINAL in small caps
%\newcommand{\CARD}{{\Sc{cardinal}}}%shortcut for CARDINAL in small caps
%\newcommand{\PROX}{{\Sc{proximal}}}%shortcut for PROXIMAL in small caps
%\newcommand{\DIST}{{\Sc{distal}}}%shortcut for DISTAL in small caps
%\newcommand{\RMT}{{\Sc{remote}}}%shortcut for REMOTE in small caps
%\newcommand{\REFL}{{\Sc{reflexive}}}%shortcut for REFLEXIVE in small caps
%\newcommand{\PRS}{{\Sc{present}}}%shortcut for PRESENT in small caps
%\newcommand{\PST}{{\Sc{past}}}%shortcut for PAST in small caps
%\newcommand{\IMP}{{\Sc{imperative}}}%shortcut for IMPERATIVE in small caps
%\newcommand{\POT}{{\Sc{potential}}}%shortcut for POTENTIAL in small caps
\newcommand{\PROG}{{\Sc{progressive}}}%shortcut for PROGRESSIVE in small caps
\newcommand{\PRF}{{\Sc{perfect}}}%shortcut for PERFECT in small caps
\newcommand{\INF}{{\Sc{infinitive}}}%shortcut for INFINITIVE in small caps
%\newcommand{\NEG}{{\Sc{negative}}}%shortcut for NEGATIVE in small caps
\newcommand{\CONNEG}{{\Sc{connegative}}}%shortcut for CONNEGATIVE in small caps
\newcommand{\ATTRs}{{\Sc{attr}}}%shortcut for ATTR in small caps
\newcommand{\PREDs}{{\Sc{pred}}}%shortcut for PRED in small caps
%\newcommand{\COMPs}{{\Sc{comp}}}%shortcut for COMP in small caps
%\newcommand{\SUPERLs}{{\Sc{superl}}}%shortcut for SUPERL in small caps
\newcommand{\SGs}{{\Sc{sg}}}%shortcut for SG in small caps
\newcommand{\DUs}{{\Sc{du}}}%shortcut for DU in small caps
\newcommand{\PLs}{{\Sc{pl}}}%shortcut for PL in small caps
\newcommand{\NOMs}{{\Sc{nom}}}%shortcut for NOM in small caps
\newcommand{\ACCs}{{\Sc{acc}}}%shortcut for ACC in small caps
\newcommand{\GENs}{{\Sc{gen}}}%shortcut for GEN in small caps
\newcommand{\ILLs}{{\Sc{ill}}}%shortcut for ILL in small caps
\newcommand{\INESSs}{{\Sc{iness}}}%shortcut for INESS in small caps
\newcommand{\ELATs}{{\Sc{elat}}}%shortcut for ELAT in small caps
\newcommand{\COMs}{{\Sc{com}}}%shortcut for COM in small caps
\newcommand{\ABESSs}{{\Sc{abess}}}%shortcut for ABESS in small caps
\newcommand{\ESSs}{{\Sc{ess}}}%shortcut for ESS in small caps
%\newcommand{\DIMs}{{\Sc{dim}}}%shortcut for DIM in small caps
%\newcommand{\ORDs}{{\Sc{ord}}}%shortcut for ORD in small caps
%\newcommand{\CARDs}{{\Sc{card}}}%shortcut for CARD in small caps
\newcommand{\PROXs}{{\Sc{prox}}}%shortcut for PROX in small caps
\newcommand{\DISTs}{{\Sc{dist}}}%shortcut for DIST in small caps
\newcommand{\RMTs}{{\Sc{rmt}}}%shortcut for RMT in small caps
\newcommand{\REFLs}{{\Sc{refl}}}%shortcut for REFL in small caps
\newcommand{\PRSs}{{\Sc{prs}}}%shortcut for PRS in small caps
\newcommand{\PSTs}{{\Sc{pst}}}%shortcut for PST in small caps
\newcommand{\IMPs}{{\Sc{imp}}}%shortcut for IMP in small caps
\newcommand{\POTs}{{\Sc{pot}}}%shortcut for POT in small caps
\newcommand{\PROGs}{{\Sc{prog}}}%shortcut for PROG in small caps
\newcommand{\PRFs}{{\Sc{prf}}}%shortcut for PRF in small caps
\newcommand{\INFs}{{\Sc{inf}}}%shortcut for INF in small caps
\newcommand{\NEGs}{{\Sc{neg}}}%shortcut for NEG in small caps
\newcommand{\CONNEGs}{{\Sc{conneg}}}%shortcut for CONNEG in small caps

\newcommand{\subNP}{{\footnotesize\sub{NP}}}%shortcut for NP (nominal phrase) in subscript
\newcommand{\subVC}{{\footnotesize\sub{VC}}}%shortcut for VC (verb complex) in subscript
\newcommand{\subAP}{{\footnotesize\sub{AP}}}%shortcut for NP (adjectival phrase) in subscript
\newcommand{\subAdvP}{{\footnotesize\sub{AdvP}}}%shortcut for AdvP (adverbial phrase) in subscript
\newcommand{\subPP}{{\footnotesize\sub{PP}}}%shortcut for NP (postpoistional phrase) in subscript

\newcommand{\ipa}[1]{{\fontspec{Linux Libertine}#1}}%specifying font for IPA characters

\newcommand{\SEC}{§}%standardize section symbol and spacing afterwards
%\newcommand{\SEC}{§\,}%

\newcommand{\Nth}{{\footnotesize(\It{n})}}%used in table of numerals in ADJ chapter

%%newcommands for tables in introductionSDL.tex:
\newcommand{\cliticExs}[3]{\Tn{\begin{tabular}{p{28mm} c p{28mm} p{35mm}}\It{#1}&\ARROW &\It{#2} & ‘#3’\\\end{tabular}}}%specifically for the two clitic examples
\newcommand{\Grapheme}[1]{\It{#1}}%formatting for graphemes in orthography tables
%%new command for the section on orthographic examples; syntax: #1=orthography, #2=phonology, #3=gloss
\newcommand{\SpellEx}[3]{\Tn{\begin{tabular}{p{70pt} p{70pt} l}\ipa{/#2/}&\It{#1}& ‘#3’ \\\end{tabular}}}%formatting for orthographic examples (intro-Chapter)


%%new transl tier in gb4e; syntax: #1=free translation (in single quotes), #2=additional comments, z.B. literal meaning:
\newcommand{\Transl}[2]{\trans\Tn{‘#1’ #2}}%new transl tier in gb4e;
\newcommand{\TranslMulti}[2]{\trans\hspace{12pt}\Tn{‘#1’ #2}}%new transl tier in gb4e for a dialog to be included under a single example number


%% used for examples in the Prosody and Segmental phonology chapters:
\newcommand{\PhonGloss}[7]{%PhonGloss = Phonology Gloss;
%pattern: \PhonGloss{label}{phonemic}{phonetic}{orthographic}{gloss}{recording}{utterance}
\ea\label{#1}
\Tn{\begin{tabular}[t]{p{30mm} l}
\ipa{/#2/}	& \It{#4} \\
\ipa{[#3]}	&\HANG ‘#5’\\%no table row can start with square brackets! thus the workaround with \MC
\end{tabular}\hfill\hyperlink{#6}{{\small\textnormal[pit#6#7]}}%\index{Z\Red{rec}!\Red{pit#6}}\index{Z\Red{utt}!\Red{pit#6#7} \Blue{Phon}}
}
\z}
\newcommand{\PhonGlossWL}[6]{%PhonGloss = Phonology Gloss for words from WORDLIST, not from corpus!;
%pattern: \PhonGloss{label}{phonemic}{phonetic}{orthographic}{gloss}{wordListNumber}
\ea\label{#1}
\Tn{\begin{tabular}[t]{p{30mm} l}
\ipa{/#2/}	& \It{#4} \\
\ipa{[#3]}	&\HANG ‘#5’\\%no table row can start with square brackets! thus the workaround with \MC
\end{tabular}\hfill\hyperlink{explExs}{{\small\textnormal[#6]}}%\index{Z\Red{wl}!\Red{#6}\Blue{Phon}}
}
\z}

%%for derivation examples in the derivational morphology chapter!
%syntax: \DerivExam{#1}{#2}{#3}{#4}{#5}{#6}
%#1: base, #2: base-gloss, #3: derived form, #4: derived form gloss, #5: derived form translation, #6: pit-recording, #7: utterance number
\newcommand{\DW}{28mm}%for following three commands, to align arrows throughout
%%%%OLD:
%%%\newcommand{\DerivExam}[7]{\Tn{\begin{tabular}[t]{p{\DW}cl}\It{#1}&\ARROW&\It{#3}\\#2&&#4\\\end{tabular}\hfill\pbox{.3\textwidth}{\hfill‘#5’\\\hbox{}\hfill\hyperlink{pit#6}{{\small\textnormal[pit#6.#7]}}}
%%%%\index{Z\Red{rec}!\Red{pit#6}}\index{Z\Red{utt}!\Red{pit#6.#7}}
%%%}}
%NEW:
\newcommand{\DerivExam}[7]{\Tn{
\begin{tabular}[t]{p{\DW}x{5mm}l}\It{#1}&\ARROW&\It{#3}\\\end{tabular}\hfill‘#5’\\
\hspace{1mm}\begin{tabular}[t]{p{\DW}x{5mm}l}#2&&#4\\\end{tabular}\hfill\hyperlink{pit#6}{{\small\textnormal[pit#6.#7]}}
%\index{Z\Red{rec}!\Red{pit#6}}\index{Z\Red{utt}!\Red{pit#6.#7}}
}}
%%same as above, but supress any reference to a specific utterance
\newcommand{\DerivExamX}[7]{\Tn{
\begin{tabular}[t]{p{\DW}x{5mm}l}\It{#1}&\ARROW&\It{#3}\\\end{tabular}\hfill‘#5’\\
\hspace{1mm}\begin{tabular}[t]{p{\DW}x{5mm}l}#2&&#4\\\end{tabular}\hfill\hyperlink{pit#6}{{\small\textnormal[pit#6]\It{e}}}
%\index{Z\Red{rec}!\Red{pit#6}}\index{Z\Red{utt}!\Red{pit#6.#7}}
}}
\newcommand{\DerivExamWL}[6]{\Tn{
\begin{tabular}[t]{p{\DW}x{5mm}l}\It{#1}&\ARROW&\It{#3}\\\end{tabular}\hfill‘#5’\\
\hspace{1mm}\begin{tabular}[t]{p{\DW}x{5mm}l}#2&&#4\\\end{tabular}\hfill\hyperlink{explExs}{{\small\textnormal[#6]}}
%\index{Z\Red{wl}!\Red{#6}}
}}


%formatting of corpus source information (after \transl in gb4e-environments):
\newcommand{\Corpus}[2]{\hspace*{1pt}\hfill{\small\mbox{\hyperlink{pit#1}{\Tn{[pit#1.#2]}}}}%\index{Z\Red{rec}!\Red{pit#1}}\index{Z\Red{utt}!\Red{pit#1.#2}}
}%
\newcommand{\CorpusE}[2]{\hspace*{1pt}\hfill{\small\mbox{\hyperlink{pit#1}{\Tn{[pit#1.#2]}}\It{e}}}%\index{Z\Red{rec}!\Red{pit#1}}\index{Z\Red{utt}!\Red{pit#1.#2}\Blue{-E}}
}%
%%as above, but necessary for recording names which include an underline because the first variable in \href understands _ but the second variable requires \_
\newcommand{\CorpusLink}[3]{\hspace*{1pt}\hfill{\small\mbox{\hyperlink{pit#1}{\Tn{[pit#2.#3]}}}}%\index{Z\Red{rec}!\Red{pit#2}}\index{Z\Red{utt}!\Red{pit#2.#3}}
}%
%%as above, but for newer recordings which begin with sje20 instead of pit
\newcommand{\CorpusSJE}[2]{\hspace*{1pt}\hfill{\small\mbox{\hyperlink{sje20#1}{\Tn{[sje20#1.#2]}}}}%\index{Z\Red{rec}!\Red{sje20#1}}\index{Z\Red{utt}!\Red{sje20#1.#2}}
}%
\newcommand{\CorpusSJEE}[2]{\hspace*{1pt}\hfill{\small\mbox{\hyperlink{sje20#1}{\Tn{[sje20#1.#2]}}\It{e}}}%\index{Z\Red{rec}!\Red{sje20#1}}\index{Z\Red{utt}!\Red{sje20#1.#2}\Blue{-E}}
}%











%%hyphenation points for line breaks
%%add to TeX file before \begin{document} with:
%%%%hyphenation points for line breaks
%%add to TeX file before \begin{document} with:
%%%%hyphenation points for line breaks
%%add to TeX file before \begin{document} with:
%%\include{hyphenationSDL}
\hyphenation{
ab-es-sive
affri-ca-te
affri-ca-tes
Ahka-javv-re
al-ve-o-lar
com-ple-ments
%check this:
de-cad-es
fri-ca-tive
fri-ca-tives
gemi-nate
gemi-nates
gra-pheme
gra-phemes
ho-mo-pho-nous
ho-mor-ga-nic
mor-pho-syn-tac-tic
or-tho-gra-phic
pho-neme
pho-ne-mes
phra-ses
post-po-si-tion
post-po-si-tion-al
pre-as-pi-ra-te
pre-as-pi-ra-ted
pre-as-pi-ra-tion
seg-ment
un-voiced
wor-king-ver-sion
}
\hyphenation{
ab-es-sive
affri-ca-te
affri-ca-tes
Ahka-javv-re
al-ve-o-lar
com-ple-ments
%check this:
de-cad-es
fri-ca-tive
fri-ca-tives
gemi-nate
gemi-nates
gra-pheme
gra-phemes
ho-mo-pho-nous
ho-mor-ga-nic
mor-pho-syn-tac-tic
or-tho-gra-phic
pho-neme
pho-ne-mes
phra-ses
post-po-si-tion
post-po-si-tion-al
pre-as-pi-ra-te
pre-as-pi-ra-ted
pre-as-pi-ra-tion
seg-ment
un-voiced
wor-king-ver-sion
}
\hyphenation{
ab-es-sive
affri-ca-te
affri-ca-tes
Ahka-javv-re
al-ve-o-lar
com-ple-ments
%check this:
de-cad-es
fri-ca-tive
fri-ca-tives
gemi-nate
gemi-nates
gra-pheme
gra-phemes
ho-mo-pho-nous
ho-mor-ga-nic
mor-pho-syn-tac-tic
or-tho-gra-phic
pho-neme
pho-ne-mes
phra-ses
post-po-si-tion
post-po-si-tion-al
pre-as-pi-ra-te
pre-as-pi-ra-ted
pre-as-pi-ra-tion
seg-ment
un-voiced
wor-king-ver-sion
}\begin{document}\tableofcontents\clearpage

%%%%%%%%%%%%%%%%%%%%%%%%%%%%%%%% ALL THE ABOVE TO BE COMMENTED OUT FOR COMPLETE DOCUMENT! %%%%%%%%%%%%%%%%%%%%%%%%


\chapter{Phrase types}\label{phraseTypesCh}\index{phrase types}
There are five types of phrases in \PS\ which form syntactic constituents of other phrases or of clauses: %, and may consist of one word or a number of words:
\begin{itemize}
\item{verb complex (VC)}
\item{nominal phrase (NP)}
\item{adjectival phrase (AP)}
\item{adverbial phrase (AdvP)}
\item{postpositional phrase (PP)}
\end{itemize}
Table \vref{phraseTypeSummary} summarizes the main syntactic functions of the various phrase types, and the sections of this chapter that deal with them. 
\begin{table}\centering
\caption{Summary of phrase types and their syntactic functions}\label{phraseTypeSummary}
%\resizebox{\columnwidth}{!}{
\begin{tabular}{|c|c|c|c|c|}\hline
		&\MC{4}{c|}{\It{syntactic function}}\\
\It{phrase}	&			&\It{argument/adjunct/}	&			&			\\%\dline
\It{type}	&\It{predicate}	&\It{complement}	&\It{modifier in NP}&\It{modifier in AP}\\\dline%&\It{section}
VC	&\CH		&				&			&			\\\hline%predicate%&\ref{verbComplex}	
NP	&\CH		&\CH			&\CH		&			\\\hline%argument, adjunct, modify NP (gen)%&\ref{nominalPhrases}
AP	&			&\CH			&\CH		&			\\\hline%modify NP-head, modify NP%&\ref{adjectivalPhrases}
AdvP&			&\CH			&			&\CH		\\\hline%adjunct, modify AP-head%&\ref{adverbialPhrases}
PP	&			&\CH			&			&			\\\hline%adjunct%&\ref{postpositionalPhrases}	
\end{tabular}%}
\end{table}


\section{Verb complex}\label{verbComplex}\index{verb complex}\index{phrase types!verb complex}
%\Blue{Here comes the verbal complex!!!!!!!!!!!!!!!} oder auch nicht.
The \PS\ verb complex (abbreviated ‘VC’) consists minimally of a finite verb, and maximally of a finite verb and one or two non-finite verb forms. 
With the exception of the imperative, the finite verb inflects for tense or mood, number and person, and agrees with the subject. 
The imperative only inflects for number. %, which cannot be regarded as an agreement suffix in this case because imperatives do not have subject arguments. %NOT RELEVANT
%
In combination with non-finite verb forms, the verbal categories negation, mood, and aspect can also be expressed. 

To better describe the distribution of finite and non-finite verbs forms in VCs, verbs are divided into two groupings: % based on when they can be a finite verb: %in VCs with more than one verb: 
\begin{itemize}
\item{lexical verbs and the copula verb \It{årrot} ‘be’}
\item{grammatical verbs (the negation verb, the aspectual auxiliary verb \It{årrot} ‘be’, and the modal verbs; cf. §\,\ref{multiVdeclarativeClauses}).}
\end{itemize} 

In VCs featuring only one verb form, the finite verb is a lexical verb or the copula verb. In VCs with two or three verb forms, the finite verb is a grammatical verb, while 
the selection of each non-finite form is determined by the type of verb governing it: the verb of negation triggers the connegative form, the aspectual auxiliary verb triggers either the perfect or the progressive form, and the modal verbs trigger the infinitive form.
This is summarized in Table \vref{VCstructures}. % on page \pageref{VCstructures}. 
\begin{table}\centering
\caption{Verb complex structures with one, two or three verbs}\label{VCstructures}
%used so that both tabulars are same width:
\resizebox{\textwidth}{!}{
\begin{tabular}{|x{.05\textwidth}|p{.25\textwidth} x{.025\textwidth}p{.295\textwidth} x{.025\textwidth}p{.275\textwidth}|}\hline%.025+.025+.025+.3+.3+.3=.975
%\begin{tabular}{|x{.05\textwidth}|p{.45\textwidth} x{.025\textwidth}p{.45\textwidth}|}\hline%.025+.025+.45+.45=.95
\It{qty.}&\MC{5}{l|}{\It{verb form or forms}}		\\\dline
%\MC{1}{c}{\Bf{qty.}}&\MC{3}{l}{\Bf{verb forms}}		\\\hline
\MR{2}{*}{1}&\MC{5}{l|}{\Bf{finite verb}} \\%\hline
&\It{(lexical/copula)}	& & &&	\\\hline%\dline%\cline{2-4}
%&copula verb	& & 	\\\hline%\dline
%&&&\\\hline
%\MC{6}{l}{}\\\hline
\MR{2}{*}{2}&\Bf{finite verb}		&\MR{1}{*}{\Bf{\PLUS}} &\MC{3}{l|}{\Bf{non-finite form}} \\%\hline
&\It{(grammatical)}		& &\MC{3}{l|}{\It{(lexical/copula)}} \\\hline%\cline{2-4}
\MC{1}{c|}{}&modal verb	&\PLUS &\MC{3}{l|}{\INFs}	\\\cline{2-6}
\MC{1}{c|}{}&aspectual auxiliary&\PLUS &\MC{3}{l|}{\PRFs{\tiny\ or }\PROGs}	\\\cline{2-6}%\hline
\MC{1}{c|}{}&negation verb	&\PLUS &\MC{3}{l|}{\CONNEGs}	\\\hline
%\end{tabular}}
%\resizebox{\textwidth}{!}{
%\begin{tabular}{|x{.05\textwidth}|p{.25\textwidth} x{.025\textwidth}p{.295\textwidth} x{.025\textwidth}p{.275\textwidth}|}\hline%.025+.025+.025+.3+.3+.3=.975
%\MC{6}{l}{}\\\hline
\MR{2}{*}{3}&\Bf{finite verb}		&\MR{1}{*}{\Bf{\PLUS}} &\Bf{non-finite form}	&\MR{1}{*}{\Bf{\PLUS}} &\Bf{non-finite form} \\%\hline
&\It{(grammatical)}		& &\It{(grammatical)}		& &\It{(lexical/copula)} \\\hline
\MC{1}{c|}{}&aspectual auxiliary		&\PLUS & modal\BS\PRFs{\tiny\ or }\PROGs			&\PLUS & \INFs	\\\cline{2-6}%\hline
\MC{1}{c|}{}&negation verb			&\PLUS & modal\BS\CONNEGs		&\PLUS & \INFs	\\\cline{3-6}
\MC{1}{c|}{}&			&\PLUS & asp. auxiliary\BS\CONNEGs		&\PLUS & \PRFs{\tiny\ or }\PROGs	\\\cline{2-6}
%&modal verb			&\PLUS & asp. auxiliary/\INFs		&\PLUS & \PRFs{\tiny\ or }\PROGs	\\\hline%%%JW: commented out because not found in corpus! mentioned in last paragraph of section
%&aspectual aux. \It{årrot}	&\PLUS & aspectual (\PRFs\ or \PROGs)	&\PLUS &	\\%\hline
%&modal verb			&\PLUS & \INF						&\PLUS &	\\\hline
\end{tabular}}
\end{table}

\FloatBarrier
The constituent order of the individual verbal components is not strictly set, although the ordering indicated in Table \vref{VCstructures} is most common. Furthermore, other clause-level components may occur between these verb forms (cf. §\,\ref{constituentOrderClauses} and §\,\ref{multiVdeclarativeClauses}). 

For instance, the examples in \REF{VCex1} and \REF{VCex2} each feature a VC consisting solely of a finite verb. In \REF{VCex1} it is the singular imperative form of the lexical verb \It{vädtjat} ‘fetch’, while in \REF{VCex2} it is the encliticized \Sc{3sg.prs} form of the copula verb. %, which consists exclusively of the \Sc{3sg.prs} encliticized form of the verb \It{årrot} ‘be’. 
\ea\label{VCex1}
\glll	vietja pahparav!\\
	{[vietja]\subVC} pahpara-v\\
	fetch\BS\Sc{sg.imp} paper-\Sc{acc.sg}\\\nopagebreak
\Transl{get some paper!}{}	\Corpus{090519}{316}
\z
\ea\label{VCex2}
\glll	dun váre namma'l Sállvo\\
	d-u-n váre {namma=[l]\subVC} Sállvo\\
	\Sc{dem}-\Sc{rmt}-\Sc{gen.sg} mountain\BS\Sc{gen.sg} name\BS\Sc{nom.sg}=be\BS\Sc{3sg.prs} Sállvo\BS\Sc{nom.sg}\\\nopagebreak
\Transl{the name of that mountain is Sállvo}{}	\Corpus{100404}{005}
\z

In \REF{VCex3}, there are two VCs. The first VC is \It{ij…dága}%, literally ‘does not make’,
\footnote{The complete phrase \It{ij aktagav dága} is likely a calque of the Swedish phrase \It{det gör ingenting} ‘that doesn’t matter’ (lit.: ‘that does nothing’).} %; cf.\ German \It{das macht nichts}).} 
and consists of the finite negation verb and the lexical verb \It{dáhkat} ‘make, do’ in its connegative form, but is split by the particle \It{dä} and the NP argument \It{aktagav}. The second VC is the verb \It{vähtjat} ‘fetch’, here in its \Sc{1sg.prs} finite form. 
\ea\label{VCex3}
\glll	ij dä aktagav dága, mån viehtjav dav maŋŋel\\
	{[i-j]\subVC\sub{\It{1}}} dä aktaga-v {[dága]\subVC\sub{\It{1}}} mån {[viehtja-v]\subVC\sub{\It{2}}} d-a-v maŋŋel\\
	\Sc{neg}-\Sc{3sg.prs} then nothing-\Sc{acc.sg} make\BS\Sc{conneg} \Sc{1sg.nom} fetch-\Sc{1sg.prs} \Sc{dem}-\Sc{dist}-\Sc{acc.sg} after\\\nopagebreak
\Transl{it doesn’t matter, I’ll get that later}{}	\Corpus{100404}{157}
\z

The example in \REF{VCex4} consists of three VCs. The first, \It{lin båhtam} ‘had come’, is headed by the \Sc{3pl.pst} form of the auxiliary verb \It{årrot} ‘be’ and combines with the perfect form of the main lexical verb \It{båhtet} ‘come’. The second and third VCs are both simple VCs consisting only of a finite verb form.
\ea\label{VCex4}
\glll	jus stalpe lin båhtam elo sissa, dä vuolgin ja vitjin davva\\
	jus stalpe [li-n {båhta-m]\subVC\sub{\It{1}}} elo sissa dä {[vuolgi-n]\subVC\sub{\It{2}}} ja {[vitji-n]\subVC\sub{\It{3}}} d-a-vva\\
	if wolf\BS\Sc{nom.pl} be-\Sc{3pl.pst} come-\Sc{prf} reindeer.herd\BS\Sc{gen.sg} into then drive-\Sc{3pl.pst} and fetch-\Sc{3pl.pst} \Sc{dem}-\Sc{dist}-\Sc{acc.sg}\\\nopagebreak
\Transl{if wolves had entered the reindeer herd, they went and got him}{}	\CorpusLink{0906_Ahkajavvre_a}{0906\_Ahkajavvre\_a}{091}
%%\glll	jus del nagin ájtsaj, jus stalpe lin båhtam elo sissa, dä vuolgin ja vidtjin davva\\
%%	jus del nagin ájtsa-j jus stalpe li-n båhta-m elo sissa dä vuolgi-n ja vidtji-n davva\\
%%	if evidently some\BS\Sc{nom.sg} observe-\Sc{3sg.pst} if wolf\BS\Sc{nom.pl} be-\Sc{3pl.pst} come-\Sc{prf} reindeer\_herd\BS\Sc{gen.sg} into then drive-\Sc{3pl.pst} and fetch-\Sc{3pl.pst} \Sc{dem.dist}-\Sc{acc.sg}\\
%%\Transl{if supposedly someone saw, if wolves had entered the reindeer herd, they went and got him}{}	\Corpus{0906\_Ahkajavvre\_a.091}
%%\ex\label{VCex1}%JW: complicated, not sure why ‘tjuajjgat’ is INF
%%\glll	ja muhtin del sjaddaj, vahkov tjuajjgat daj urudasaj maŋŋen, inan fáhtadij\\
%%	ja muhtin del {[sjadda-j]\subVC} vahko-v {[tjuajjga-t]\subVC} da-j urudasa-j maŋŋen inan {[fáhtadi-j]\subVC}\\
%%	and sometimes evidently become-\Sc{3sg.pst} week-\Sc{acc.sg} ski-\Sc{inf} \Sc{dem.dist}-\Sc{gen.pl} predator-\Sc{gen.pl} with before catch-\Sc{3sg.pst}\\
%%\Transl{and sometimes it supposedly turned out that he skied for a week with those predators before he caught them}{}	\Corpus{0906\_Ahkajavvre\_a.094}
\z

A modal verb %(cf. §\,\ref{modalVs}) 
and a non-finite verb form are illustrated by the VC \It{máhtta… båhtet} ‘can come’ in \REF{VCex5}. Here, the modal \It{máhtta} ‘can’ is the finite verb, and \It{båhtet} ‘come’ is in the infinitive form. 
\ea\label{VCex5}
\glll	båtsoj máhtta duv nala båhtet\\
	båtsoj {[máhtta]\subVC} duv nala {[båhte-t]\subVC}\\
	reindeer\BS\Sc{nom.sg} can\BS\Sc{3sg.prs} \Sc{2sg.gen} upon come-\Sc{inf}\\\nopagebreak
\Transl{the reindeer can attack you}{(lit: ‘come upon you’)}	\Corpus{080909}{048}
\z

Similarly, in \REF{VCex5a}, the VC \It{virrten märrket} ‘have to mark’ contains the finite verb \It{virrten} ‘must’ and the infinitive verb form \It{märrket} ‘mark’. 
\ea\label{VCex5a}
\glll	dä virrten märrket dajt miesijd dále tjaktjan\\
	dä {[virrte-n} {märrke-t]\subVC} d-a-jt miesi-jd dále tjaktja-n\\
	then must-\Sc{1du.prs} mark-\Sc{inf} \Sc{dem}-\Sc{dist}-\Sc{acc.pl} calf-\Sc{acc.pl} now autumn-\Sc{iness.sg}\\\nopagebreak
\Transl{then we have to mark those calves now in the autumn}{}	\Corpus{080909}{008}
\z
Note that, formally, clauses featuring a modal verb are identical to complement clauses featuring an infinite predicate; cf. §\,\ref{infinitiveComplementClauses}. 

Finally, the clauses in \REF{VCex6} through \REF{VCex8} provide examples of VCs with three verb forms. In \REF{VCex6}, the VC consists of the finite negation verb in \Sc{1sg.prs} (\It{iv}), the aspectual auxiliary \It{årrot} in connegative form (\It{lä}), and the lexical verb \It{gullat} ‘hear’ in its perfect form (\It{gullam}). The VC in example \REF{VCex7} contains the finite negation verb \Sc{3sg.prs} (\It{ij}), the modal verb \It{máhttet} in connegative form (\It{máhte}), and the lexical verb \It{adnet} ‘have’, in its infinitive form. Finally, in \REF{VCex8}, the finite aspectual auxiliary \It{lev} combined with the perfect form of the modal verb \It{máhttet} ‘can’ and the infinite complement \It{ságastit} ‘speak’ constitute the VC.
\ea\label{VCex6}
\glll	dä iv lä åbå gullam dav\\
	dä [i-v {lä]\subVC} åbå {[gulla-m]\subVC} d-a-v\\
	then \Sc{neg}-\Sc{1sg.prs} be\BS\Sc{conneg} at.all hear-\Sc{prf} \Sc{dem}-\Sc{dist}-\Sc{acc.sg}\\\nopagebreak
\Transl{I haven’t heard that at all}{}	\Corpus{090702}{203}
%%\glll	mån iv dede åbå, dä iv lä åbå gullam dav\\
%%	mån [i-v {dede]\subVC\sub{\it 1}} åbå dä [i-v {lä]\subVC\sub{\it 2}} åbå {[gulla-m]\subVC\sub{\it 2}} da-v\\
%%	\Sc{1sg.nom} \Sc{neg}-\Sc{1sg.prs} know\BS\Sc{conneg} at\_all then \Sc{neg}-\Sc{1sg.prs} be\BS\Sc{conneg} at\_all hear-\Sc{prf} \Sc{dem.dist}-\Sc{acc.sg}\\
%%\Transl{I don’t know at all, I haven’t heard that at all}{}	\Corpus{090702.203}
\z
\ea\label{VCex7}
\glll	ij vanj dä máhte ilá stuor dålåv adnet\\
	{[i-j]\subVC} vanj dä {[máhte]\subVC} ilá stuor dålå-v {[adne-t]\subVC}\\
	\Sc{neg}-\Sc{3sg.prs} really then can\BS\Sc{conneg} too big fire-\Sc{acc.sg} have-\Sc{inf}\\\nopagebreak
\Transl{one cannot really have too big of a fire}{}	\Corpus{090702}{176}
\z
\ea\label{VCex8}
\glll	mån lev máhttam sámev ságastit\\
	mån {[le-v} {máhtta-m]\subVC} sáme-v {[ságasti-t]\subVC}\\
	\Sc{1sg.nom} be-\Sc{1sg.prs} can-\Sc{prf} Saami-\Sc{acc.sg} speak-\Sc{inf}\\\nopagebreak
\Transl{I have been able to speak Saami}{}	\CorpusSJEE{121009}{46m27s}
\z


The corpus does not provide any evidence for three-verb VCs with a modal verb as the finite verb, so whether this logically possible structures is acceptable 
must be left to future research. However, it is clear that the negation verb is only attested as a finite verb, and can never occur as the second or third verb in a multi-verb VC. 




\section{Nominal phrases}\label{nominalPhrases}\index{nominal phrases}\index{phrase types!nominal phrases}
Nominal phrases (abbreviated ‘NP’) in \PS\ are divided into two groups:
\begin{itemize}
\item{full NPs}
\item{pronouns}
%\item{predicate APs}
\end{itemize}
NPs can function as arguments, adjuncts, predicates, adverbials, dependents of postpositions and possessors or modifiers of other NPs. 
They consist of at least one nominal component that inflects for case and number. 
Note that NPs can also be modified by postpositional phrases and non-finite verb forms, but due to a lack of sufficient data, a description of these NP modifiers must be left for future study. 
Finally, relative clauses also modify an NP; these are covered in §\,\ref{relativeClauses}. 


NPs have the structure illustrated in Figure \vref{NPstructures}, with optional components in parentheses. 
\begin{figure}\centering
\resizebox{1\linewidth}{!} {
\begin{tabular}{c}
[(demonstrative) \PLUS\ (other modifier{\scriptsize(s)}) \PLUS\ nominal \PLUS\ (refl-intensifier) \PLUS\ (rel-clause)]\subNP \\
%full NPs:		&[(determiner) \PLUS\ (other modifier{\scriptsize(s)}) \PLUS\ noun]\subNP \\
%pronouns:		&\verb|[[|pronoun]\subNP\\%\[ added so that the second [ is typeset at beginning of line
%predicate APs:	&\verb|[[|predicate AP]\subNP\\%\[ added so that the second [ is typeset at beginning of line
\end{tabular}}
\caption{The structure of nominal phrases}\label{NPstructures}
\end{figure}
Either a noun or a pronoun forms the head of an NP. The demonstrative, the nominal and any attributive reflexive pronoun always inflect for case and number, while generally the other modifiers do not. Other modifiers may be an adjectival phrase, a numeral or an NP in genitive case. An intensifier in the form of a reflexive pronoun (cf. §\,\ref{reflexivePronouns}) 
can modify a noun phrase as well, and occurs after the head. Finally, the head can be modified by a relative clause, which also occurs after the head. 
Some examples for possible NP structures are found below.%, and in §\,\ref{} for relative clauses. 

The only NP in \REF{NPstructureEx0} consists solely of the noun \It{Tjeggelvasav} ‘Lake Tjeggelvas’. 
\ea\label{NPstructureEx0}%only N
\glll	ja dä vuojnav Tjeggelvasav\\
	ja dä vuojna-v [Tjeggelvasa-v]\subNP\\
	and then see-\Sc{1sg.prs} Lake.Tjeggelvas-\Sc{acc.sg}\\\nopagebreak
\Transl{and then I see Lake Tjeggelvas}{}	\Corpus{100404}{013}
\z

In the example in \REF{NPstructureEx1}, the NP \It{dat ello} ‘that reindeer herd’ consists of a demonstrative and the head noun, and is the subject of the clause. 
In \REF{NPstructureEx2}, the subject NP consists of the head noun \It{ello} ‘reindeer herd’ and the genitive NP \It{dáj Skailej} ‘of these Skailes’\footnote{\It{Skaile} is a family name.} 
which modifies the head noun. 
\ea\label{NPstructureEx1}%det+N
\glll	ja dä såkoj dat ello\\
	ja dä såko-j [d-a-t ello]\subNP \\
	and then drown-\Sc{3sg.pst} \Sc{dem}-\Sc{dist}-\Sc{nom.sg} reindeer.herd\BS\Sc{nom.sg}\\\nopagebreak
\Transl{and then that reindeer herd drowned}{}	\CorpusLink{0906_Ahkajavvre_b}{0906\_Ahkajavvre\_b}{010}
\z
\ea\label{NPstructureEx2}%gen+N, det+N
\glll	dáj Skailej ello såkoj\\
	[[d-á-j {Skaile-j]\subNP\sub{\It{1}}}	{ello]\subNP\sub{\It{2}}} såko-j\\
	\Sc{dem}-\Sc{prox}-\Sc{gen.pl} Skaile-\Sc{gen.pl} reindeer.herd\BS\Sc{nom.sg} drown-\Sc{3sg.pst}\\\nopagebreak
\Transl{these Skailes’ reindeer herd drowned}{}	\CorpusLink{0906_Ahkajavvre_b}{0906\_Ahkajavvre\_b}{002}
\z

An NP marked for genitive case can also function as a modifier that narrows the reference of the head noun by signifying some characteristic of the head noun’s referent, as in \REF{NPstructureEx2} above. Similarly, in \REF{NPstructureGenNPex1} the genitive NP \It{mále} ‘blood’ modifies the head noun \It{gamsajd} ‘dumplings’. 
\ea\label{NPstructureGenNPex1}%
\glll	ja dágaj mále gamsajd\\
	ja dága-j {[[mále]\subNP\sub{\It{1}}} {gamsa-jd]\subNP\sub{\It{2}}} \\
	and make-\Sc{3sg.pst} blood\BS\Sc{gen.sg} dumpling-\Sc{acc.pl}\\\nopagebreak
\Transl{and one made blood dumplings}{}	\Corpus{080924}{253}
\z

In the example in \REF{NPstructureEx3}, the NP \It{nåv edna båtsoj} ‘so many reindeer’\footnote{Note that the word \It{båtsoj} ‘reindeer’ is often marked for singular, even when referring to more than one reindeer.} 
consists of the AP \It{nåv edna} ‘so many’ and the head noun \It{båtsoj} ‘reindeer’. 
\ea\label{NPstructureEx3}%AP+N
\glll	dä lij nåv edna båtsoj\\
	dä li-j [nåv edna {båtsoj]\subNP}\\
	then be-\Sc{3sg.pst} so much reindeer\BS\Sc{nom.sg}\\\nopagebreak
\Transl{there were so many reindeer}{(lit.: ‘so much reindeer’)}	\CorpusLink{0906_Ahkajavvre_b}{0906\_Ahkajavvre\_b}{013}
\z

As the example in \REF{NPstructureEx4} illustrates, it is possible for more than one modifier to be included in an NP. Here, both APs \It{guäkte} ‘two’ and \It{stuor} ‘big’ modify the noun head, which is the derived compound \It{guhkajuolgagijd} ‘long-leggers’ (referring to moose). 
\ea\label{NPstructureEx4}%AP+N
\glll	dä inijmä guäkte stuora guhkajuolgagijd\\
	dä ini-jmä [guäkte stuora {guhka-juolga-gi-jd]\subNP}\\
	then have-\Sc{1pl.pst} two big long-leg-\Sc{nmlz}-\Sc{acc.pl}\\\nopagebreak
\Transl{then we had two big moose}{(lit.: ‘long-leggers’)}	\Corpus{090702}{331}
\z

If the context is sufficiently clear, it is possible that the head noun is not realized when a demonstrative and/or modifier is present; such cases are referred to as elliptical constructions. 
Numerals, for instance, can be in elliptical constructions, %, both with or without a demonstrative, %JW: not sure about this! no data for it.
as in \REF{NPstructureEx6}, in which the noun referring to nets is not realized, so \It{gålmát} ‘third’ composes the entire NP. %In \REF{NPstructureEx7}, . 
\ea\label{NPstructureEx6}
\glll	ja gålmát sjadda dä Stutjaj\\
	ja {[gålm-át]\subNP\sub{\It{1}}} sjadda dä {[Stutja-j]\subNP\sub{\It{2}}}\\
	and three-\Sc{ord} become\BS\Sc{3sg.prs} then Stutja-\Sc{ill.sg}\\\nopagebreak
\Transl{and the third one is then placed at Stutja}{(referring to ‘fishing net’)}	\Corpus{090702}{026-027}
\z
With the exception of the \ACCs.\SGs\ form of \It{akta} ‘one’, numerals do not inflect for case and number (cf. §\,\ref{numerals}). 

Less commonly, an adjectival phrase can be in an elliptical construction, either with or without a demonstrative. In the absence of a head noun, the adjective in the AP is the host for case and number, and is morphologically a nominal. In the example in \REF{NPstructureEx8}, the head noun referring to a piece of candy is not realized, and the NP consists only of the adjective \It{rupsisav} ‘red’, which inflects for case and number.\footnote{Cf. §\,\ref{ADJinHeadlessNPs} for more details about adjectives in elliptical structures.} 
The sentence in \REF{adjNPheadEx3repeat} illustrates a similar construction including a demonstrative; here, the head noun referring to a girl is not overt, and the NP only contains the demonstrative \It{dat} ‘that’ and the adjective \It{tjábba} ‘beautiful’. 
\ea\label{NPstructureEx8}
\glll	bårov rupsisav\\
	båro-v {[rupsisa-v]\subNP}\\
	eat-\Sc{1sg.prs} red-\Sc{acc.sg}\\\nopagebreak
\Transl{I eat the red one}{}	\CorpusE{090930a}{119}
\z
\ea\label{adjNPheadEx3repeat}%this example copied from the ADJ chapter
\glll	dat tjábba máhtta sáme gielav\\
	{[d-a-t} {tjábba]\subNP\sub{\It{1}}} máhtta {[sáme} {giela-v]\subNP\sub{\It{2}}}\\
	\Sc{dem}-\Sc{dist}-\Sc{nom.sg} beautiful\BS\Sc{nom.sg} can\BS\Sc{3sg.prs} Saami\BS\Sc{gen.sg} language-\Sc{acc.sg}\\\nopagebreak
\TranslLong{That beautiful one can (speak) the Saami language}{}	\CorpusE{090930a}{148}%(referring to ‘beautiful girl’)
\z

There are many examples in the corpus of NPs consisting of a pronoun, such as the demonstrative pronoun \It{dajd} in \REF{NPstructureProEx1}, or the personal pronoun \It{mån} in \REF{NPstructureProEx2}. %Here, the first NP \It{dajd} ‘those’ consists only of the determiner. %, while the head noun referring to fish is not realized. 
%The other NP in the clause is \It{såbe} ‘stick’, the dependent of the postposition \It{nanne} ‘on’.
\ea\label{NPstructureProEx1}
\glll	bisij dajd såbe nanne\\
	bisi-j {[d-a-jd]\subNP\sub{\It{1}}} {[såbe]\subNP\sub{\It{2}}} nanne\\
	fry-\Sc{3sg.pst} \Sc{dem}-\Sc{dist}-\Sc{acc.pl} stick\BS\Sc{gen.sg} on\\\nopagebreak
\Transl{he fried them on a stick}{}	\Corpus{100404}{125}
\z
\ea\label{NPstructureProEx2}
\glll	mån mähttiv ráhpat uksav\\
	{[mån]\subNP\sub{\It{1}}} mähtti-v ráhpa-t {[uksa-v]\subNP\sub{\It{2}}}\\
	\Sc{1sg.nom} can-\Sc{1sg.pst} open-\Sc{inf} door-\Sc{acc.sg}\\\nopagebreak
\Transl{I could open the door}{}	\Corpus{100404}{347}
\z
The second NP in \REF{NPstructureProEx1}, \It{såbe} ‘stick’, also illustrates a genitive NP functioning as the complement of a PP. 

When the head of an NP is a personal pronoun, it is also possible to modify it. For instance, in \REF{NPstructureProEx3}, the \Sc{1sg.nom} reflexive pronoun \It{etj} modifies the \Sc{1sg.nom} personal pronoun \It{mån} as an intensifier. Note that this ordering (modifier following the head) deviates from the general pattern, in which the modifier occurs before the head of the NP. 
\ea\label{NPstructureProEx3}%
\glll	mån etj hålåv dále navte\\
	{[mån} {etj]\subNP} hålå-v dále navte\\
	\Sc{1sg.nom} \Sc{refl}\BS\Sc{1sg.nom} say-\Sc{1sg.prs} now like.that\\\nopagebreak
\Transl{I myself say it like that}{}	\CorpusE{090910}{29m05s}
\z

Finally, a relative clause can also occur after the head as part of the NP. An example is provided in \REF{NPstructureRelClEx1}; see §\,\ref{relativeClauses} for more on relative clauses. 
\ea\label{NPstructureRelClEx1}%
\glll	dá lä jisse gistá mejd dálven gallga adnet\\
	d-á lä jisse {[gistá} me-jd dálve-n gallga {adne-t]\subNP}\\
	\Sc{dem}\BS\Sc{prox}-\Sc{nom.pl} be\BS\Sc{3pl.prs} precisely glove\BS\Sc{nom.pl} \Sc{rel}-\Sc{acc.pl} winter-\Sc{iness.sg} shall\BS\Sc{3sg.prs} have-\Sc{inf}\\\nopagebreak
\Transl{these are gloves which one has in winter}{}	\Corpus{080708\_Session08}{006}
\z


\subsection{NPs in adverbial function}\label{ADVnouns}
Nominal phrases, %\footnote{Cf. §\,\ref{nominalPhrases} for more details about nominal phrases.} 
particularly when referring to time or place, and thus inflected for one of the locative cases (illative, inessive or elative), are often used as temporal or locational adverbials, as in \It{giesen} ‘in summer’ in \REF{ADVnounsEx1}, and in \It{hiejman} ‘at home’ in \REF{ADVnounsEx2}, respectively. Furthermore, the word \It{vahkov} ‘week’ in \REF{ADVnounsEx2} shows that an accusative NP can function as a temporal adverbial indicating a period of time. 
\ea\label{ADVnounsEx1}%
\glll	men jáhkav minniv giesen\\
	men jáhka-v minni-v {[giese-n]\subNP}\\
	but believe-\Sc{1sg.prs} go-\Sc{1sg.pst} summer-\Sc{iness.sg}\\\nopagebreak
\Transl{I believe I went in the summer}{}	\Corpus{100404}{163}
\z
\ea\label{ADVnounsEx2}
\glll	ja Henning lij jou hiejman urrum vahkov\\
	ja {[Henning]\subNP\sub{\It{1}}} li-j jou {[hiejma-n]\subNP\sub{\It{2}}} urru-m {[vahko-v]\subNP\sub{\It{3}}}\\
	and Henning\BS\Sc{nom.sg} be-\Sc{3sg.pst} well home-\Sc{iness.sg} be-\Sc{prf} week-\Sc{acc.sg}\\\nopagebreak
\Transl{and Henning had been home for a week}{}	\Corpus{090702}{285}
\z

Finally, the deverbal noun \It{diedon} literally translates as ‘in knowledge’, and can be used as a modal adverbial meaning ‘of course’, as in \REF{ADVnounsEx3}. 
\ea\label{ADVnounsEx3}%
\glll	etja diedon ednen mielkev ja vuostav\\
	{[etj-a]\subNP\sub{\It{1}}} {[died-o-n]\subNP\sub{\It{2}}} edne-n {[mielke-v]\subNP\sub{\It{3}}} ja {[vuosta-v]\subNP\sub{\It{4}}}\\
	self-\Sc{nom.pl} know-\Sc{nmlz1}-\Sc{iness.sg} have-\Sc{3pl.pst} milk-\Sc{acc.sg} and cheese-\Sc{acc.sg}\\\nopagebreak
\Transl{they themselves of course had milk and cheese}{}	\CorpusLink{080708_Session02}{080708\_Session02}{034}
\z




\section{Adjectival phrases}\label{adjectivalPhrases}\index{adjectival phrases}\index{phrase types!adjectival phrases}
Adjectival phrases (abbreviated ‘AP’) in \PS\ are divided into three groups based on the type of adjective required as head: 
\begin{itemize}
\item{attributive APs (headed by attributive adjectives)}% in an NP}
\item{predicate APs (headed by predicative adjectives)}% as the complement of the copula verb \It{årrot} ‘be’}
\item{numeral APs (headed by numerals)}%
\end{itemize}
%In all cases, APs are used as modifiers. %the head noun of an NP. 
In the first case, the attributive AP is syntactically embedded in an NP whose head noun it modifies. In the second case, the predicative AP ascribes the property it concerns to the entire NP which is the subject of the copula clause that the AP is embedded in. Numeral APs can occur both attributively and predicatively. 

The possible constituent structures of APs are presented in Figure \vref{APstructures}. 
The only difference in internal structure between the two adjective APs is the choice of adjective: attributive vs. predicative. % APs only allow attributive adjectives, while predicative APs only allow predicative adjectives to head them. 
Numeral APs normally consist only of a numeral, but can be further modified by an adverbial. 
\begin{figure}\centering
\begin{tabular}{l l}
attributive AP:	&[(AdvP) \PLUS\ attributive adjective]\subAP \\%\[ added so that the second [ is typeset at beginning of line
predicative AP:	&[(AdvP) \PLUS\ predicative adjective]\subAP \\%\[ added so that the second [ is typeset at beginning of line
numeral AP:	&[(AdvP) \PLUS\ numeral]\subAP \\%\[ added so that the second [ is typeset at beginning of line
\end{tabular}
\caption{The three possible structures of adjectival phrases}\label{APstructures}
\end{figure}

Adverbial phrases are not common in APs. %, and only certain adverbial phrases can be a constituent in an AP. 
The data is quite limited concerning which type of AdvPs are acceptable, and future study is needed in this respect. Examples from the corpus include %\marginpar{more AdvP in APs??} 
\It{ilá} ‘too’, \It{nåv} ‘so’, \It{huj} ‘really’, \It{åbbå} ‘completely’ and \It{gajk} ‘all’. 

Several examples of APs are provided here. In \REF{APstructureEx1}, the AP consists only of the attributive adjective \It{njalga} ‘tasty’ and modifies \It{biebmov} ‘food’, the head of the NP. 
%\Red{need example for \REF{APstructureEx2} with AdvP\PLUS\ Adj!}
\ea\label{APstructureEx1}%ADJattr
\glll	ja danna lip bårråm njalga biebmov\\
	ja danna li-p bårrå-m {[njalga]\subAP} biebmo-v\\
	and there be-\Sc{1pl.prs} eat-\Sc{prf} tasty food-\Sc{acc.sg}\\\nopagebreak
\Transl{and we ate tasty food there}{}	\Corpus{110517b2}{005}
\z

The example in \REF{APstructureEx2} illustrates an attributive adjective modified by an AdvP. 
\ea\label{APstructureEx2}%AdvP+ADJattr ??
\glll	ja dat lä urrum huj buorak giesse\\
	ja d-a-t lä urru-m [huj {buorak]\subAP} giesse\\
	and \Sc{dem}-\Sc{dist}-\Sc{nom.sg} be\BS\Sc{3sg.prs} be-\Sc{prf} really good summer\BS\Sc{nom.sg}\\\nopagebreak
\Transl{and it has been a really good summer}{}	\Corpus{080909}{009}
\z

In \REF{APstructureEx3}, the predicative adjective \It{nuorra} ‘young’ agrees with the subject NP \It{mánná} ‘child’ in number. 
\ea\label{APstructureEx3}%ADJpred
\glll	mánná lä nuorra\\
	mánná lä {[nuorra]\subAP}\\
	child\BS\Sc{nom.sg} be\BS\Sc{3sg.prs} young\BS\Sc{sg}\\\nopagebreak
\Transl{the child is young}{}	\CorpusE{090930a}{310}
\z

As the example in \REF{APstructureEx4} shows, predicative adjectives are syntactically adjectives, as the predicate adjective \It{buojde} ‘fat’ is the head of an AP modified by an AdvP (\It{nåv} ‘so’). %, but also inflect for number (here, non-concatenatively). 
\ea\label{APstructureEx4}%AdvP+ADJpred
\glll	dä lä vuosjkuna nåv buojde\\
	dä lä vuosjkun-a [nåv {buojde]\subAP}\\
	then be\BS\Sc{3pl.prs} perch-\Sc{nom.pl} so fat\BS\Sc{pl}\\\nopagebreak
\Transl{then the perch are so fat}{}	\Corpus{090702}{080}
\z


Finally, examples of numeral APs are provided in \REF{APstructureEx5} through \REF{APstructureEx7}. In the example in \REF{APstructureEx5}, \It{guäkte} ‘two’ modifies the NP \It{dåpe} ‘houses’. %is provided in \REF{APstructureEx5}. 
\ea\label{APstructureEx5}%AdvP+ADJpred
\glll	dä guäkte dåpe lä danne\\
	dä {[guäkte]\subAP} dåpe lä danne \\
	then two house\BS\Sc{nom.pl} be\BS\Sc{3pl.prs} there\\\nopagebreak
\Transl{then two houses are there}{}	\Corpus{080924}{385}
\z
The example in \REF{APstructureEx6} %\footnote{This example is also provided in example \REF{veryFirstEx} in §\,\ref{caseNumberOnNum}, but is repeated here for convenience.} 
is noteworthy because it shows that the modifying numeral can be inflected as a superlative; here the ordinal numeral \It{vuostas} is in the superlative form \It{vuostamos} meaning ‘very first’, and modifies the NP \It{guhkajuolgak} ‘moose’. 
\ea\label{APstructureEx6}%AdvP+ADJpred
\glll	dieda, mån vuotjev vuostamos guhkajuolgagav\\
	dieda mån vuotje-v {[vuosta-mos]\subAP} guhka-juolga-ga-v\\
	know\BS\Sc{2sg.prs} \Sc{1sg.nom} shoot-\Sc{1sg.pst} first-\Sc{superl} long-leg-\Sc{nmlz}-\Sc{acc.sg}\\\nopagebreak
\Transl{you know, I shot my very first long-legger}{}	\Corpus{080924}{079}%(referring to a moose)
\z

Numeral APs can include an adverbial modifier. The adverb \It{ber} ‘only’ modifies the numeral \It{akkta} ‘one’ in \REF{APstructureEx7}.
\ea\label{APstructureEx7}%
\glll	vuostak lij ber akkta rommå\\
	vuostak li-j {[ber} {akkta]\subAP} råmmå\\
	first be-\Sc{3sg.pst} only one room\BS\Sc{nom.sg}\\\nopagebreak
\Transl{initially there was only one room}{}	\Corpus{100310b}{051}
\z


\subsection{APs with an adverbial function}\label{ADVadjectives}%
While the data in the corpus is quite limited, it appears that APs headed by an attributive adjective can be used adverbially. This is illustrated by \It{njuallga} ‘correct’ in \REF{ADVadjectivesEx1}. %and \It{spájta} ‘quick’ in \REF{ADVadjectivesEx2}.%quick-ATTR = spájtas, quick-ADV = spájta
\ea\label{ADVadjectivesEx1}%njuallga
\glll	jus galga njuallga dajd njuovvat dä galga dajd valldet ulgus åvdål gádtsastij\\
	jus galga {[njuallga]\subAP} d-a-jd njuovva-t dä galga d-a-jd vallde-t ulgus åvdål gádtsasti-j\\
	if will\BS\Sc{3sg.prs} correct \Sc{dem}-\Sc{dist}-\Sc{acc.pl} slaughter-\Sc{inf} then will\BS\Sc{3sg.prs} \Sc{dem}-\Sc{dist}-\Sc{acc.pl} take-\Sc{inf} out before hang.up-\Sc{3sg.pst}\\
%	if will.3SG.PRS correct DEM slaughter-INF then will.3SG.PRS DEM take-INF ut before?(förre) hang-3SG.PST? \\\nopagebreak
\TranslLong{If one slaughtered them correctly, then one would take them out before one hung them up}{}	\Corpus{080909}{105}
%%\ex\label{ADVadjectivesEx2}
%%\glll	mån vádtsav \Bf{spájta}\\
%%	mån vádtsa-v spájta\\
%%	\Sc{1sg.nom} go-\Sc{1sg.prs} quick\\
%%\Transl{I go quickly}{}	\CorpusSJEE{121011.30m22s}
\z





\section{Adverbial phrases}\label{adverbialPhrases}\index{adverbial phrases}\index{phrase types!adverbial phrases}
An adverbial phrase (abbreviated ‘AdvP’) has an adverb as its head, and can potentially be modified by a further AdvP, as illustrated in Figure \vref{AdvPstructures}
\begin{figure}\centering
\begin{tabular}{l }
[(AdvP) \PLUS\ adverb]\subAdvP \\%\[ added so that the second [ is typeset at beginning of line
%AdvP:	&\verb|[[|(AdvP?) \PLUS\ adverb]\subAdvP \\%\[ added so that the second [ is typeset at beginning of line
\end{tabular}
\caption{The structure of adverbial phrases}\label{AdvPstructures}
\end{figure}

The AdvP in example \REF{AdvPex1} consists only of the adverb \It{buoragit} ‘well’ (derived from \It{buorre} ‘good’).
\ea\label{AdvPex1}
\glll	dalloj dä lij, manaj buoragit\\
	dalloj dä li-j mana-j {[buoragi-t]\subAdvP}\\
	at.that.time then be-\Sc{3sg.pst} go-\Sc{3sg.pst} good-\Sc{advz}\\\nopagebreak
\Transl{at that time it was, it went well}{}	\CorpusLink{0906_Ahkajavvre_a}{0906\_Ahkajavvre\_a}{023}
\z

The sentence in \REF{derivedADVsEx2repeat} shows that an adverb can be further modified by another, preceding adverb. Here, the AP head \It{buoragit} ‘well’ is further modified by the adverb \It{ganska} ‘quite’.\footnote{The adverb \It{ganska} is a nonce borrowing from Swedish (cf. Swedish \It{ganska} ‘quite’).}
\ea\label{derivedADVsEx2repeat}%copied from Adverb chapter
\glll	viesojmä vanj ganska buoragit dajna guollemijn aj\\
	vieso-jmä vanj {[[ganska]\subAdvP\sub{\It{1}}} {buoragi-t]\subAdvP\sub{\It{2}}} d-a-jna guollemi-jn aj\\
	live-\Sc{1pl.pst} definitely quite good-\Sc{advz} \Sc{dem}-\Sc{dist}-\Sc{com.sg} fishing-\Sc{com.sg} also\\\nopagebreak
\Transl{we definitely lived quite well with the fishing, too}{}	\CorpusLink{0906_Ahkajavvre_a}{0906\_Ahkajavvre\_a}{164}
\z

%While the adverb \It{ganska} in \REF{derivedADVsEx2repeat} is a nonce borrowing from Swedish,\footnote{Cf. Swedish \It{ganska} ‘quite’.} 
The example in \REF{AdvPex2} illustrates a AdvP (\It{åbbå} ‘quite’) which modifies the head of an AP (\It{vuoras} ‘old’).
\ea\label{AdvPex2}
\glll	men åbbå vuoras lä del dát\\
	men {[åbbå]\subAdvP} vuoras lä del d-á-t\\
	but quite old\BS\Sc{sg} be\BS\Sc{3sg.prs} obviously \Sc{dem}-\Sc{prox}-\Sc{nom.sg}\\\nopagebreak
\Transl{but this one is obviously quite old}{}	\CorpusLink{080708_Session07}{080708\_Session07}{006}
\z

Other forms also frequently fulfill an adverbial function; cf. 
§\,\ref{ADVnouns} for nominal phrases, 
§\,\ref{ADVadjectives} for adjectival phrases, 
§\,\ref{postpositionalPhrases} for postpositional phrases 
and §\,\ref{ADVverbs} for non-finite (progressive) verb forms. 





\section{Postpositional phrases}\label{postpositionalPhrases}\index{postpositional phrases}\index{phrase types!postpositional phrases}
A postpositional phrase (abbreviated ‘PP’) is headed by a postposition, which is always preceded by an NP complement. Any components in this complementing NP which are subject to case inflection inflect for genitive (as well as number). This structure is illustrated in Figure \vref{PPstructure}. 
%The choice of postposition determines the nature of this relationship; 
\begin{figure}\centering
[NP\sub{[\Sc{gen}]} \PLUS\ post-position]\sub{\It{PP}}
%\begin{tabular}{}
%\end{tabular}
\caption{Syntactic structure of postpositional phrases}\label{PPstructure}
\end{figure}
See §\,\vref{postpositions} for a list of postpositions. 

The complement in a PP can be any valid nominal phrase. A number of examples for various NPs complementing the head of a PP are provided below: %and marked in \Bf{bold} face: 
a noun with a demonstrative in \REF{PPex1}, a single noun in \REF{PPex2}, a demonstrative pronoun in \REF{PPex3}, a personal pronoun in \REF{PPex4}, an interrogative pronoun in \REF{PPex5}, and an interrogative NP \REF{PPex6}.
\ea\label{PPex1}%DET+N/pl+PostP
\glll	mån gillgiv daj gusaj birra ságastit\\
	mån gillgi-v [d-a-j gusa-j {birra]\subPP} ságastit\\
	\Sc{1sg.nom} will-\Sc{1sg.pst} \Sc{dem}-\Sc{dist}-\Sc{gen.pl} cow-\Sc{gen.pl} about say-\Sc{inf}\\\nopagebreak
\Transl{I was going to talk about those cows}{}	\Corpus{080924}{089}
\z
\ea\label{PPex2}%N/sg+PostP
\glll	dä mån tjuotjuv Stuornjárga nanne Álesgiehtjen\\
	dä mån tjuotju-v [Stuor-njárga {nanne]\subPP} Álesgiehtje-n\\
	then \Sc{1sg.nom} stand-\Sc{1sg.prs} big-point\BS\Sc{gen.sg} on Västerfjäll-\Sc{iness.sg}\\\nopagebreak
\Transl{I am standing on ‘Big Point’ in Västerfjäll}{}	\Corpus{100404}{012}
\z
\ea\label{PPex3}%DET/sg+PostP
\glll	mån virtev tjátsev bejat dan sisa\\
	mån virte-v tjátse-v beja-t [d-a-n {sisa]\subPP}\\
	\Sc{1sg.nom} must-\Sc{1sg.prs} water-\Sc{acc.sg} put-\Sc{inf} \Sc{dem}-\Sc{dist}-\Sc{gen.sg} into\\\nopagebreak
\Transl{I have to put water into that}{}	\Corpus{080909}{164}
\z
\ea\label{PPex4}%prn\sg+PostP
\glll	da lin duv gugu båhtam\\
	d-a li-n [du-v {gugu]\subPP} båhta-m\\
	\Sc{dem}-\Sc{dist}\BS\Sc{nom.pl} be\BS\Sc{3pl.prs} \Sc{2sg.gen} to come-\Sc{prf}\\\nopagebreak
\Transl{they have come to you}{}	\CorpusE{110329}{35m03s}
\z
\ea\label{PPex5}%Q/pl+PostP
\glll	mej nanne lä da?\\
	[me-j {nanne]\subPP} lä d-a\\
	what-\Sc{gen.pl} on be\BS\Sc{3pl.prs} \Sc{dem}-\Sc{dist}\BS\Sc{nom.pl}\\
%	what-\Sc{gen.pl} on be\BS\Sc{3pl.prs} \Sc{dem.\Red{YONDER}}\BS\Sc{nom.pl}\\\nopagebreak
\Transl{what are those on?}{}	\CorpusE{110331a}{27m28s}
%\Transl{what are those\Red{-yonder} on?}{}	\CorpusE{110331a.27m28s}
\z
\ea\label{PPex6}%Q/sg-N/sg+PostP
\glll	mikkir gierge nanne?\\
	[mikkir gierge {nanne]\subPP}\\
	which rock\BS\Sc{gen.sg} on\\\nopagebreak
\Transl{on which rock?}{}	\CorpusE{110331a}{110}
\z

Postpositional phrases %\footnote{Cf. §\,\ref{postpositionalPhrases} for more details about postpositional phrases.} 
can function as clause-level adverbials, often indicating the location of an action, as in \It{dan giedge nanne} ‘on that stone’ in \REF{ADVppEx1}. 
\ea\label{ADVppEx1}%
\glll	dä Kataridna ja månnå dan gierge nanne pruvkojin ståhkåt\\
	dä Kataridna ja månnå {[d-a-n} gierge {nanne]\subPP} pruvko-jin ståhkå-t\\
	then Katarina\BS\Sc{nom.sg} and \Sc{1sg.nom} \Sc{dem}-\Sc{dist}-\Sc{gen.sg} rock\BS\Sc{gen.sg} on used.to-\Sc{3pl.pst} play-\Sc{inf}\\\nopagebreak
%	then Katarina\BS\Sc{nom.sg} and \Sc{1sg.nom} \Sc{dem.dist}-\Sc{gen.sg} rock\BS\Sc{gen.sg} on usually\_do-\Sc{3pl.pst} play-\Sc{inf}\\\nopagebreak
\Transl{Katarina and I used to play on that rock}{}	\Corpus{100404}{159}
%\ex\label{ADVnounsEx2}
%\glll	\\
%	\\
%	\\
%\Transl{}{}	\Corpus{}
\z


Note that, to a very limited extent, some postpositions can be used as prepositions, in which case they head a prepositional phrase. However, the data in the corpus concerning prepositional phrases is so limited that no conclusions can be made at this point. See §\,\ref{prepositions} for the two examples from the corpus. 





%%%%%%%% THIS IS NOT USED FOR THE ENTIRE COMPILATION, but only for individual chapters!!!!

\clearpage
\addcontentsline{toc}{chapter}{Bibliography}\label{Bibliography}
\bibliography{PiteGrammarBibSDL}%for bibtex
%\printbibliography%[title=Works Cited]%%for biber!






%%%NAME INDEX doesn’t work!?!? why???
\cleardoublepage\phantomsection%this allows hyperlink in ToC to work
\addcontentsline{toc}{chapter}{Name index}
\ohead{Name index}
\printindex[aut]

\cleardoublepage\phantomsection%this allows hyperlink in ToC to work
\addcontentsline{toc}{chapter}{Language index}
\ohead{Language index}
\printindex[lan]

\cleardoublepage\phantomsection%this allows hyperlink in ToC to work
\addcontentsline{toc}{chapter}{Subject index}
\ohead{Subject index}
\printindex


\end{document}%no citation in this chapter
%\end{document}