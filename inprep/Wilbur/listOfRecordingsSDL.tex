%\documentclass[ number=5
			   ,series=sidl
			   ,isbn=xxx-x-xxxxxx-xx-x
			   ,url=http://langsci-press.org/catalog/book/17
			   ,output=long   % long|short|inprep              
			   %,blackandwhite
			   %,smallfont
			   ,draftmode   
			  ]{LSP/langsci}                          

\usepackage{LSP/lsp-styles/lsp-gb4e}		% verhindert Komma bei mehrfachen Fußnoten?
\usepackage{LSP/lsp-styles/avm}
\avmfont{\sc} 
\avmvalfont{\it}
                                                      
\usepackage{layout}
\usepackage{lipsum}

%%for LSP-lines in tables: %%doesn’t work for some reason. Plus, not all my tables have a single-line header. Double-lines aren’t typeset properly in ‘longtable’-environment across several pages.
%\usepackage{booktabs}
%\newcommand{\mytoprule}{\midrule\toprule}
%\newcommand{\mybottomrule}{\bottomrule\midrule}

%%%following now in main document (XeTeX_pitePhDSDL.tex)
%\title{A corpus-based grammar \\ of spoken Pite Saami}  
%%\subtitle{2000+ Years of Language Science and no End in Sight}  
%\author{Joshua Wilbur}
%\dedication{Gijtov adnet!}
%\renewcommand{\lsBackBody}{This grammar of Pite Saami (Uralic; Sweden) is simply bad-ass.}%for back cover text
%\renewcommand{\lsBackTitle}{Biddumsáme giella}%for back title

%%%% ABOVE FOR LangSciPress %%%%
%%%% ABOVE FOR LangSciPress %%%%
%%%% ABOVE FOR LangSciPress %%%%

\usepackage{longtable}

\usepackage{multirow}
\usepackage{array} %allows, among other things, centering column content in a table while also specifying width, creates new column style "x" for center-alignment, "y" for right-alignment
\newcolumntype{x}[1]{%
>{\centering\hspace{0pt}}p{#1}}%
\newcolumntype{y}[1]{%
>{\raggedleft\hspace{0pt}}p{#1}}%

\usepackage[]{placeins}%using \FloatBarrier command, all floats still floating at that point will be typeset, and cannot cross that boundary. the option here \usepackage[section]{placeins} automatically adds \FloatBarrier to every \section command (only works for \section commands, nothing lower than that!)
\usepackage{afterpage}%by using the command \afterpage{\clearpage}, all floats will appear, but no new page will be started, thus avoiding bad page breaks around floats

\usepackage{vowel} %for vowel space chart


%%IS THIS NECESSARY??
%%%following allows you to refer to footnotes (from http://anthony.liekens.net/index.php/LaTeX/MultipleFootnoteReferences)
\newcommand{\footnoteremember}[2]{
  \footnote{#2}
  \newcounter{#1}
  \setcounter{#1}{\value{footnote}}
} \newcommand{\footnoterecall}[1]{
  \footnotemark[\value{#1}]} 
%%%previous allows you to refer to footnotes: use \footnoteremember{referenceText} in footnote, then \footnoterecall{referenceText} to refer.

\usepackage{tikz}
\usetikzlibrary{plothandlers,matrix,decorations.text,shapes.arrows,shadows,chains,positioning,scopes}

\usepackage{synttree} %zeichnet linguistische Bäume
\branchheight{36pt}%sets height between rows in synttree

\usepackage{lscape}%used for landscape pages in index (list of recordings)

\usepackage{polyglossia}
\setmainlanguage{english}



%%%TAKE OUT FOR FINAL VERSION:
%%%TAKE OUT FOR FINAL VERSION:
%%%TAKE OUT FOR FINAL VERSION:

%%%following readjusts margin text!
\setlength{\marginparwidth}{20mm}
\let\oldmarginpar\marginpar
\renewcommand\marginpar[1]{\-\oldmarginpar[\raggedleft\footnotesize\vspace{-7pt}\color{red}\It{→ #1}]%
{\raggedright\footnotesize\vspace{-7pt}\color{red}\It{→ #1}}}
%%%previous readjusts margin text!

%%%The following lines set depth of ToC (LSP default is only 3 levels)!
%%%\renewcommand{\contentsname}{Table of Contents} % überschrift des inhaltsverzeichnisses
%\setcounter{secnumdepth}{5}%sets how deep section/subsection/subsubsections are numbered
%\setcounter{tocdepth}{5}%sets the depth of the ToC %but this doesn't seem to work!!!
\include{newcommandsSDL}\include{hyphenationSDL}\begin{document}

%%%%%%%%%%%%%%%%%%%%%%%%%%%%%%%% ALL THE ABOVE TO BE COMMENTED OUT FOR COMPLETE DOCUMENT! %%%%%%%%%%%


\appendix%\chapter{First Appendix}
\chapter*{Appendix: \\Inventory of \PS\ Recordings}
\label{inventory}\hypertarget{inventoryRef}{}
The inventory on pages \pageref{inventoryBegin} through \pageref{inventoryEnd} 
lists recordings collected for the \PSDP\ by Joshua Wilbur. Not all recordings have provided examples which are included in this grammatical description, but they were all nonetheless essential in the process of coming to terms with the \PS\ language, and thus relevant for the present study. 
The information provided reflects the state of the corpus in early March 2014. 

These 123 recordings are listed in the order they were created. In the vast majority of cases, the name of each recording indicates the date of recording as well (cf. Section \ref{collectionMethods}); when this is not the case, the date is indicated in the brief description. To keep the inventory simple, %initials have been used for participant names, while 
abbreviations are used for the columns ‘genre’ and ‘medium’; %and the archiving institutions preserving copies of the digital recordings and relevant supplemental materials, such as transcriptions and annotations. %as of 4\textsuperscript{th} OCTOBER 2011. 
the explanations for these abbreviations are listed in the table below. %\ref{inventoryAbbs} below. %on page \pageref{inventoryAbbs}. %
The column ‘words’ indicates the current number of transcribed and translated \PS\ words for each recording; some have been only partly transcribed or not at all. 
%\footnote{Note that some archived recordings in this inventory have been only partly transcribed or not at all.} 
The column ‘brief description’ provides a concise summary of the content of each recording; more detailed descriptions are provided in the metadata available from the archives hosting the materials. 

\begin{table}\centering
\begin{tabular}{c}
Abbreviations used in the inventory of recordings\\
\end{tabular}
\begin{tabular}{|l c l|}\hline
\It{column} &\It{abbreviation}	&\It{explanation} \\\dline
\multirow{2}{*}{\textit{media}}
& A				& audio only\\
& A/V			& audio and video \\\hline
\multirow{7}{*}{\textit{genre}}
& conv.			& conversation \\
& elic.			& elicitation \\
& expl.			& explanation \\
& narr.			& narration \\
& perf.			& performance \\
& read.			& reading \\
& writ.			& written text \\
\hline
\end{tabular}
\end{table}


\begin{landscape}
\begin{longtable}[c]{| l | x{15mm} | c | c | p{95mm} |}
\hline
\It{name}	&\It{genre}	&\It{medium}	&\It{words}	&\It{brief description}\\\dline
\endhead\endfoot
%\multicolumn{5}{l}{here end\label{inventoryEnd}}\endlastfoot
\label{inventoryBegin}


%%%%%%%%%%%%%%%%%%%%%%%%%%%%%%%%%%%%%%%%%%%%%%%%%%%%%%%%%%%%
%%%%%%%%%%%%%%%%%%%%%%%%%%%%%%%%%%%%%%%%%%%%%%%%%%%%%%%%%%%%
%%%%%%%%%%%%%%%%%%%%BEGIN DATA IMPORT%%%%%%%%%%%%%%%%%%%%%%%%%%%%%
%%%%%%%%%%%%%%%%%%%%%%%%%%%%%%%%%%%%%%%%%%%%%%%%%%%%%%%%%%%%
%%%%%%%%%%%%%%%%%%%%%%%%%%%%%%%%%%%%%%%%%%%%%%%%%%%%%%%%%%%%

\input{RecordingsInventorySDL}

%%%%%%%%%%%%%%%%%%%%%%%%%%%%%%%%%%%%%%%%%%%%%%%%%%%%%%%%%%%%
%%%%%%%%%%%%%%%%%%%%%%%%%%%%%%%%%%%%%%%%%%%%%%%%%%%%%%%%%%%%
%%%%%%%%%%%%%%%%%%%%END DATA IMPORT%%%%%%%%%%%%%%%%%%%%%%%%%%%%%
%%%%%%%%%%%%%%%%%%%%%%%%%%%%%%%%%%%%%%%%%%%%%%%%%%%%%%%%%%%%
%%%%%%%%%%%%%%%%%%%%%%%%%%%%%%%%%%%%%%%%%%%%%%%%%%%%%%%%%%%%


\end{longtable}
\phantomsection\label{inventoryEnd}%%allows end of table to be referred to by \pageref{}
\end{landscape}


%\end{document}