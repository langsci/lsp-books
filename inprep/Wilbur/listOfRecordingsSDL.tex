%\documentclass[ number=5
			   ,series=sidl
			   ,isbn=xxx-x-xxxxxx-xx-x
			   ,url=http://langsci-press.org/catalog/book/17
			   ,output=long   % long|short|inprep              
			   %,blackandwhite
			   %,smallfont
			   ,draftmode   
			  ]{LSP/langsci}                          

\usepackage{LSP/lsp-styles/lsp-gb4e}		% verhindert Komma bei mehrfachen Fußnoten?
                                                      
\usepackage{layout}
\usepackage{lipsum}

%%%% ABOVE FOR LangSciPress %%%%
%%%% ABOVE FOR LangSciPress %%%%
%%%% ABOVE FOR LangSciPress %%%%
\usepackage{libertine}%work-around solution for rendering problematic characters ʦ, ͡  (mostly in \textbf{})

\usepackage{longtable}%Double-lines (\hline\hline) aren’t typeset properly in ‘longtable’-environment across several pages! conflict with other package (maybe xcolor with option ‘tables’?)

\usepackage{multirow}

\usepackage{array} %allows, among other things, centering column content in a table while also specifying width, creates new column style "x" for center-alignment, "y" for right-alignment
\newcolumntype{x}[1]{>{\centering\hspace{0pt}}p{#1}}%
\newcolumntype{y}[1]{>{\raggedleft\hspace{0pt}}p{#1}}%

\usepackage[]{placeins}%using \FloatBarrier command, all floats still floating at that point will be typeset, and cannot cross that boundary. the option here \usepackage[section]{placeins} automatically adds \FloatBarrier to every \section command (only works for \section commands, nothing lower than that!)
%\usepackage{afterpage}%by using the command \afterpage{\clearpage}, all floats will appear, but no new page will be started, thus avoiding bad page breaks around floats

\usepackage{vowel} %for vowel space chart


%%%IS THIS NECESSARY??
%%%%following allows you to refer to footnotes (from http://anthony.liekens.net/index.php/LaTeX/MultipleFootnoteReferences)
%\newcommand{\footnoteremember}[2]{
%  \footnote{#2}
%  \newcounter{#1}
%  \setcounter{#1}{\value{footnote}}
%} \newcommand{\footnoterecall}[1]{
%  \footnotemark[\value{#1}]} 
%%%%previous allows you to refer to footnotes: use \footnoteremember{referenceText} in footnote, then \footnoterecall{referenceText} to refer.

\usepackage{tikz}%
\usetikzlibrary{plothandlers,matrix,decorations.text,shapes.arrows,shadows,chains,positioning,scopes}

\usepackage{synttree} %zeichnet linguistische Bäume
\branchheight{36pt}%sets height between rows in synttree

\usepackage{lscape}%used for landscape pages in index (list of recordings)

\usepackage{polyglossia}
\setmainlanguage{english}


%%%TAKE OUT FOR FINAL VERSION:
%%%TAKE OUT FOR FINAL VERSION:
%%%TAKE OUT FOR FINAL VERSION:

%%%%following readjusts margin text!
%\setlength{\marginparwidth}{20mm}
%\let\oldmarginpar\marginpar
%\renewcommand\marginpar[1]{\-\oldmarginpar[\raggedleft\footnotesize\vspace{-7pt}\color{red}\It{→ #1}]%
%{\raggedright\footnotesize\vspace{-7pt}\color{red}\It{→ #1}}}
%%%%previous readjusts margin text!

%%%The following lines set depth of ToC (LSP default is only 3 levels)!
%%%\renewcommand{\contentsname}{Table of Contents} % überschrift des inhaltsverzeichnisses
%\setcounter{secnumdepth}{5}%sets how deep section/subsection/subsubsections are numbered
%\setcounter{tocdepth}{5}%sets the depth of the ToC %but this doesn't seem to work!!!
%% new commands for LSP book (Grammar of Pite Saami, by J. Wilbur)

\newcommand{\PS}{Pite Saami}
\newcommand{\PSDP}{Pite Saami Documentation Project}
\newcommand{\WLP}{Wordlist Project}

\newcommand{\HANG}{\everypar{\hangindent15pt \hangafter1}}%also useful for table cells
\newcommand{\FB}{\FloatBarrier}%shortcut for this command to print all floats w/o pagebreak

\newcommand{\REF}[1]{(\ref{#1})}%adds parenthesis around the reference number, particularly useful for examples.%\Ref had clash with LSP!
\newcommand{\dline}{\hline\hline}%makes a double line in a table
\newcommand{\superS}[1]{\textsuperscript{#1}}%adds superscript element
\newcommand{\sub}[1]{$_{#1}$}%adds subscript element
\newcommand{\Sc}[1]{\textsc{#1}}%shortcut for small capitals (not to be confused with \sc, which changes the font from that point on)
\newcommand{\It}[1]{\textit{#1}}%shortcut for italics (not to be confused with \it, which changes the font from that point on)
\newcommand{\Bf}[1]{\textbf{#1}}%shortcut for bold (not to be confused with \bf, which changes the font from that point on)
\newcommand{\BfIt}[1]{\textbf{\textit{#1}}}
\newcommand{\BfSc}[1]{\textbf{\textsc{#1}}}
\newcommand{\Tn}[1]{\textnormal{#1}}%shortcut for normal text (undo italics, bolt, etc.)
\newcommand{\MC}{\multicolumn}%shortcut for multicolumn command in tabular environment - only replaces command, not variables!
\newcommand{\MR}{\multirow}%shortcut for multicolumn command in tabular environment - only replaces command, not variables!
\newcommand{\TILDE}{∼}%U+223C %OLD:~}%shortcut for tilde%command ‘\Tilde’ clashes with LSP!%
\newcommand{\BS}{\textbackslash}%backslash
\newcommand{\Red}[1]{{\color{red}{#1}}}%for red text
\newcommand{\Blue}[1]{{\color{blue}{#1}}}%for blue text
\newcommand{\PLUS}{+}%nicer looking plus symbol
\newcommand{\MINUS}{-}%nicer looking plus symbol
%    Was die Pfeile betrifft, kannst Du mal \Rightarrow \mapsto \textrightarrow probieren und dann \mathbf \boldsymbol oder \pbm dazutun.
\newcommand{\ARROW}{\textrightarrow}%→%dieser dicke Pfeil ➜ wird nicht von der LSP-Font unterstützt: %\newcommand{\ARROW}{{\fontspec{DejaVu Sans}➜}}
\newcommand{\DARROW}{\textleftrightarrow}%↔︎%DoubleARROW
\newcommand{\BULLET}{•}%
%%✓ does not exist in the default LSP font!
\newcommand{\CH}{\checkmark}%%\newcommand{\CH}{\fontspec{Arial Unicode MS}✓}%CH as in CHeck
%%following used to separate alternation forms for consonant gradation and umlaut patterns:
\newcommand{\Div}{‑}%↔︎⬌⟷⬄⟺⇔%non-breaking hyphen: ‑  
\newcommand{\QUES}{\textsuperscript{?}}%marks questionable/uncertain forms

\newcommand{\jvh}{\mbox{\It{j}-suffix} vowel harmony}%
%\newcommand{\Ptcl}{\Sc{ptcl} }%just shortcut for glossing ‘particle’
%\newcommand{\ATTR}{{\Sc{attributive}}}%shortcut for ATTRIBUTIVE in small caps
%\newcommand{\PRED}{{\Sc{predicative}}}%shortcut for PREDICATIVE in small caps
%\newcommand{\COMP}{{\Sc{comparative}}}%shortcut for COMPARATIVE in small caps
%\newcommand{\SUPERL}{{\Sc{superlative}}}%shortcut for SUPERLATIVE in small caps
\newcommand{\SG}{{\Sc{singular}}}%shortcut for SINGULAR in small caps
\newcommand{\DU}{{\Sc{dual}}}%shortcut for DUAL in small caps
\newcommand{\PL}{{\Sc{plural}}}%shortcut for PLURAL in small caps
%\newcommand{\NOM}{{\Sc{nominative}}}%shortcut for NOMINATIVE in small caps
%\newcommand{\ACC}{{\Sc{accusative}}}%shortcut for ACCUSATIVE in small caps
%\newcommand{\GEN}{{\Sc{genitive}}}%shortcut for GENITIVE in small caps
%\newcommand{\ILL}{{\Sc{illative}}}%shortcut for ILLATIVE in small caps
%\newcommand{\INESS}{{\Sc{inessive}}}%shortcut for INESSIVE in small caps
\newcommand{\ELAT}{{\Sc{elative}}}%shortcut for ELATIVE in small caps
%\newcommand{\COM}{{\Sc{comitative}}}%shortcut for COMITATIVE in small caps
%\newcommand{\ABESS}{{\Sc{abessive}}}%shortcut for ABESSIVE in small caps
%\newcommand{\ESS}{{\Sc{essive}}}%shortcut for ESSIVE in small caps
%\newcommand{\DIM}{{\Sc{diminutive}}}%shortcut for DIMINUTIVE in small caps
%\newcommand{\ORD}{{\Sc{ordinal}}}%shortcut for ORDINAL in small caps
%\newcommand{\CARD}{{\Sc{cardinal}}}%shortcut for CARDINAL in small caps
%\newcommand{\PROX}{{\Sc{proximal}}}%shortcut for PROXIMAL in small caps
%\newcommand{\DIST}{{\Sc{distal}}}%shortcut for DISTAL in small caps
%\newcommand{\RMT}{{\Sc{remote}}}%shortcut for REMOTE in small caps
%\newcommand{\REFL}{{\Sc{reflexive}}}%shortcut for REFLEXIVE in small caps
%\newcommand{\PRS}{{\Sc{present}}}%shortcut for PRESENT in small caps
%\newcommand{\PST}{{\Sc{past}}}%shortcut for PAST in small caps
%\newcommand{\IMP}{{\Sc{imperative}}}%shortcut for IMPERATIVE in small caps
%\newcommand{\POT}{{\Sc{potential}}}%shortcut for POTENTIAL in small caps
\newcommand{\PROG}{{\Sc{progressive}}}%shortcut for PROGRESSIVE in small caps
\newcommand{\PRF}{{\Sc{perfect}}}%shortcut for PERFECT in small caps
\newcommand{\INF}{{\Sc{infinitive}}}%shortcut for INFINITIVE in small caps
%\newcommand{\NEG}{{\Sc{negative}}}%shortcut for NEGATIVE in small caps
\newcommand{\CONNEG}{{\Sc{connegative}}}%shortcut for CONNEGATIVE in small caps
\newcommand{\ATTRs}{{\Sc{attr}}}%shortcut for ATTR in small caps
\newcommand{\PREDs}{{\Sc{pred}}}%shortcut for PRED in small caps
%\newcommand{\COMPs}{{\Sc{comp}}}%shortcut for COMP in small caps
%\newcommand{\SUPERLs}{{\Sc{superl}}}%shortcut for SUPERL in small caps
\newcommand{\SGs}{{\Sc{sg}}}%shortcut for SG in small caps
\newcommand{\DUs}{{\Sc{du}}}%shortcut for DU in small caps
\newcommand{\PLs}{{\Sc{pl}}}%shortcut for PL in small caps
\newcommand{\NOMs}{{\Sc{nom}}}%shortcut for NOM in small caps
\newcommand{\ACCs}{{\Sc{acc}}}%shortcut for ACC in small caps
\newcommand{\GENs}{{\Sc{gen}}}%shortcut for GEN in small caps
\newcommand{\ILLs}{{\Sc{ill}}}%shortcut for ILL in small caps
\newcommand{\INESSs}{{\Sc{iness}}}%shortcut for INESS in small caps
\newcommand{\ELATs}{{\Sc{elat}}}%shortcut for ELAT in small caps
\newcommand{\COMs}{{\Sc{com}}}%shortcut for COM in small caps
\newcommand{\ABESSs}{{\Sc{abess}}}%shortcut for ABESS in small caps
\newcommand{\ESSs}{{\Sc{ess}}}%shortcut for ESS in small caps
%\newcommand{\DIMs}{{\Sc{dim}}}%shortcut for DIM in small caps
%\newcommand{\ORDs}{{\Sc{ord}}}%shortcut for ORD in small caps
%\newcommand{\CARDs}{{\Sc{card}}}%shortcut for CARD in small caps
\newcommand{\PROXs}{{\Sc{prox}}}%shortcut for PROX in small caps
\newcommand{\DISTs}{{\Sc{dist}}}%shortcut for DIST in small caps
\newcommand{\RMTs}{{\Sc{rmt}}}%shortcut for RMT in small caps
\newcommand{\REFLs}{{\Sc{refl}}}%shortcut for REFL in small caps
\newcommand{\PRSs}{{\Sc{prs}}}%shortcut for PRS in small caps
\newcommand{\PSTs}{{\Sc{pst}}}%shortcut for PST in small caps
\newcommand{\IMPs}{{\Sc{imp}}}%shortcut for IMP in small caps
\newcommand{\POTs}{{\Sc{pot}}}%shortcut for POT in small caps
\newcommand{\PROGs}{{\Sc{prog}}}%shortcut for PROG in small caps
\newcommand{\PRFs}{{\Sc{prf}}}%shortcut for PRF in small caps
\newcommand{\INFs}{{\Sc{inf}}}%shortcut for INF in small caps
\newcommand{\NEGs}{{\Sc{neg}}}%shortcut for NEG in small caps
\newcommand{\CONNEGs}{{\Sc{conneg}}}%shortcut for CONNEG in small caps

\newcommand{\subNP}{{\footnotesize\sub{NP}}}%shortcut for NP (nominal phrase) in subscript
\newcommand{\subVC}{{\footnotesize\sub{VC}}}%shortcut for VC (verb complex) in subscript
\newcommand{\subAP}{{\footnotesize\sub{AP}}}%shortcut for NP (adjectival phrase) in subscript
\newcommand{\subAdvP}{{\footnotesize\sub{AdvP}}}%shortcut for AdvP (adverbial phrase) in subscript
\newcommand{\subPP}{{\footnotesize\sub{PP}}}%shortcut for NP (postpoistional phrase) in subscript

\newcommand{\ipa}[1]{{\fontspec{Linux Libertine}#1}}%specifying font for IPA characters

\newcommand{\SEC}{§}%standardize section symbol and spacing afterwards
%\newcommand{\SEC}{§\,}%

\newcommand{\Nth}{{\footnotesize(\It{n})}}%used in table of numerals in ADJ chapter

%%newcommands for tables in introductionSDL.tex:
\newcommand{\cliticExs}[3]{\Tn{\begin{tabular}{p{28mm} c p{28mm} p{35mm}}\It{#1}&\ARROW &\It{#2} & ‘#3’\\\end{tabular}}}%specifically for the two clitic examples
\newcommand{\Grapheme}[1]{\It{#1}}%formatting for graphemes in orthography tables
%%new command for the section on orthographic examples; syntax: #1=orthography, #2=phonology, #3=gloss
\newcommand{\SpellEx}[3]{\Tn{\begin{tabular}{p{70pt} p{70pt} l}\ipa{/#2/}&\It{#1}& ‘#3’ \\\end{tabular}}}%formatting for orthographic examples (intro-Chapter)


%%new transl tier in gb4e; syntax: #1=free translation (in single quotes), #2=additional comments, z.B. literal meaning:
\newcommand{\Transl}[2]{\trans\Tn{‘#1’ #2}}%new transl tier in gb4e;
\newcommand{\TranslMulti}[2]{\trans\hspace{12pt}\Tn{‘#1’ #2}}%new transl tier in gb4e for a dialog to be included under a single example number


%% used for examples in the Prosody and Segmental phonology chapters:
\newcommand{\PhonGloss}[7]{%PhonGloss = Phonology Gloss;
%pattern: \PhonGloss{label}{phonemic}{phonetic}{orthographic}{gloss}{recording}{utterance}
\ea\label{#1}
\Tn{\begin{tabular}[t]{p{30mm} l}
\ipa{/#2/}	& \It{#4} \\
\ipa{[#3]}	&\HANG ‘#5’\\%no table row can start with square brackets! thus the workaround with \MC
\end{tabular}\hfill\hyperlink{#6}{{\small\textnormal[pit#6#7]}}%\index{Z\Red{rec}!\Red{pit#6}}\index{Z\Red{utt}!\Red{pit#6#7} \Blue{Phon}}
}
\z}
\newcommand{\PhonGlossWL}[6]{%PhonGloss = Phonology Gloss for words from WORDLIST, not from corpus!;
%pattern: \PhonGloss{label}{phonemic}{phonetic}{orthographic}{gloss}{wordListNumber}
\ea\label{#1}
\Tn{\begin{tabular}[t]{p{30mm} l}
\ipa{/#2/}	& \It{#4} \\
\ipa{[#3]}	&\HANG ‘#5’\\%no table row can start with square brackets! thus the workaround with \MC
\end{tabular}\hfill\hyperlink{explExs}{{\small\textnormal[#6]}}%\index{Z\Red{wl}!\Red{#6}\Blue{Phon}}
}
\z}

%%for derivation examples in the derivational morphology chapter!
%syntax: \DerivExam{#1}{#2}{#3}{#4}{#5}{#6}
%#1: base, #2: base-gloss, #3: derived form, #4: derived form gloss, #5: derived form translation, #6: pit-recording, #7: utterance number
\newcommand{\DW}{28mm}%for following three commands, to align arrows throughout
%%%%OLD:
%%%\newcommand{\DerivExam}[7]{\Tn{\begin{tabular}[t]{p{\DW}cl}\It{#1}&\ARROW&\It{#3}\\#2&&#4\\\end{tabular}\hfill\pbox{.3\textwidth}{\hfill‘#5’\\\hbox{}\hfill\hyperlink{pit#6}{{\small\textnormal[pit#6.#7]}}}
%%%%\index{Z\Red{rec}!\Red{pit#6}}\index{Z\Red{utt}!\Red{pit#6.#7}}
%%%}}
%NEW:
\newcommand{\DerivExam}[7]{\Tn{
\begin{tabular}[t]{p{\DW}x{5mm}l}\It{#1}&\ARROW&\It{#3}\\\end{tabular}\hfill‘#5’\\
\hspace{1mm}\begin{tabular}[t]{p{\DW}x{5mm}l}#2&&#4\\\end{tabular}\hfill\hyperlink{pit#6}{{\small\textnormal[pit#6.#7]}}
%\index{Z\Red{rec}!\Red{pit#6}}\index{Z\Red{utt}!\Red{pit#6.#7}}
}}
%%same as above, but supress any reference to a specific utterance
\newcommand{\DerivExamX}[7]{\Tn{
\begin{tabular}[t]{p{\DW}x{5mm}l}\It{#1}&\ARROW&\It{#3}\\\end{tabular}\hfill‘#5’\\
\hspace{1mm}\begin{tabular}[t]{p{\DW}x{5mm}l}#2&&#4\\\end{tabular}\hfill\hyperlink{pit#6}{{\small\textnormal[pit#6]\It{e}}}
%\index{Z\Red{rec}!\Red{pit#6}}\index{Z\Red{utt}!\Red{pit#6.#7}}
}}
\newcommand{\DerivExamWL}[6]{\Tn{
\begin{tabular}[t]{p{\DW}x{5mm}l}\It{#1}&\ARROW&\It{#3}\\\end{tabular}\hfill‘#5’\\
\hspace{1mm}\begin{tabular}[t]{p{\DW}x{5mm}l}#2&&#4\\\end{tabular}\hfill\hyperlink{explExs}{{\small\textnormal[#6]}}
%\index{Z\Red{wl}!\Red{#6}}
}}


%formatting of corpus source information (after \transl in gb4e-environments):
\newcommand{\Corpus}[2]{\hspace*{1pt}\hfill{\small\mbox{\hyperlink{pit#1}{\Tn{[pit#1.#2]}}}}%\index{Z\Red{rec}!\Red{pit#1}}\index{Z\Red{utt}!\Red{pit#1.#2}}
}%
\newcommand{\CorpusE}[2]{\hspace*{1pt}\hfill{\small\mbox{\hyperlink{pit#1}{\Tn{[pit#1.#2]}}\It{e}}}%\index{Z\Red{rec}!\Red{pit#1}}\index{Z\Red{utt}!\Red{pit#1.#2}\Blue{-E}}
}%
%%as above, but necessary for recording names which include an underline because the first variable in \href understands _ but the second variable requires \_
\newcommand{\CorpusLink}[3]{\hspace*{1pt}\hfill{\small\mbox{\hyperlink{pit#1}{\Tn{[pit#2.#3]}}}}%\index{Z\Red{rec}!\Red{pit#2}}\index{Z\Red{utt}!\Red{pit#2.#3}}
}%
%%as above, but for newer recordings which begin with sje20 instead of pit
\newcommand{\CorpusSJE}[2]{\hspace*{1pt}\hfill{\small\mbox{\hyperlink{sje20#1}{\Tn{[sje20#1.#2]}}}}%\index{Z\Red{rec}!\Red{sje20#1}}\index{Z\Red{utt}!\Red{sje20#1.#2}}
}%
\newcommand{\CorpusSJEE}[2]{\hspace*{1pt}\hfill{\small\mbox{\hyperlink{sje20#1}{\Tn{[sje20#1.#2]}}\It{e}}}%\index{Z\Red{rec}!\Red{sje20#1}}\index{Z\Red{utt}!\Red{sje20#1.#2}\Blue{-E}}
}%











%%hyphenation points for line breaks
%%add to TeX file before \begin{document} with:
%%%%hyphenation points for line breaks
%%add to TeX file before \begin{document} with:
%%%%hyphenation points for line breaks
%%add to TeX file before \begin{document} with:
%%\include{hyphenationSDL}
\hyphenation{
ab-es-sive
affri-ca-te
affri-ca-tes
Ahka-javv-re
al-ve-o-lar
com-ple-ments
%check this:
de-cad-es
fri-ca-tive
fri-ca-tives
gemi-nate
gemi-nates
gra-pheme
gra-phemes
ho-mo-pho-nous
ho-mor-ga-nic
mor-pho-syn-tac-tic
or-tho-gra-phic
pho-neme
pho-ne-mes
phra-ses
post-po-si-tion
post-po-si-tion-al
pre-as-pi-ra-te
pre-as-pi-ra-ted
pre-as-pi-ra-tion
seg-ment
un-voiced
wor-king-ver-sion
}
\hyphenation{
ab-es-sive
affri-ca-te
affri-ca-tes
Ahka-javv-re
al-ve-o-lar
com-ple-ments
%check this:
de-cad-es
fri-ca-tive
fri-ca-tives
gemi-nate
gemi-nates
gra-pheme
gra-phemes
ho-mo-pho-nous
ho-mor-ga-nic
mor-pho-syn-tac-tic
or-tho-gra-phic
pho-neme
pho-ne-mes
phra-ses
post-po-si-tion
post-po-si-tion-al
pre-as-pi-ra-te
pre-as-pi-ra-ted
pre-as-pi-ra-tion
seg-ment
un-voiced
wor-king-ver-sion
}
\hyphenation{
ab-es-sive
affri-ca-te
affri-ca-tes
Ahka-javv-re
al-ve-o-lar
com-ple-ments
%check this:
de-cad-es
fri-ca-tive
fri-ca-tives
gemi-nate
gemi-nates
gra-pheme
gra-phemes
ho-mo-pho-nous
ho-mor-ga-nic
mor-pho-syn-tac-tic
or-tho-gra-phic
pho-neme
pho-ne-mes
phra-ses
post-po-si-tion
post-po-si-tion-al
pre-as-pi-ra-te
pre-as-pi-ra-ted
pre-as-pi-ra-tion
seg-ment
un-voiced
wor-king-ver-sion
}\begin{document}

%%%%%%%%%%%%%%%%%%%%%%%%%%%%%%%% ALL THE ABOVE TO BE COMMENTED OUT FOR COMPLETE DOCUMENT! %%%%%%%%%%%
\addchap{Appendix: Inventory of Recordings}
%\rohead{Appendix: Inventory of Recordings}%added b/c it’s missing otherwise, then removed b/c it destroys ro-headers for following chapters
\label{inventory}\hypertarget{inventoryRef}{}
The inventory on pages \pageref{inventoryBegin} through \pageref{inventoryEnd} 
lists recordings collected for the \PSDP\ by Joshua Wilbur. Not all recordings have provided examples which are included in this grammatical description, but they were all nonetheless essential in the process of coming to terms with the \PS\ language, and thus relevant for the present study. 
The information provided reflects the state of the corpus in June 2014. 

These 123 recordings are listed in the order they were created. In the vast majority of cases, the name of each recording indicates the date of recording as well (cf. \SEC\ref{collectionMethods}); when this is not the case, the date is indicated in the brief description. To keep the inventory simple, %initials have been used for participant names, while 
abbreviations are often used for the columns ‘genre’ and ‘medium’; %and the archiving institutions preserving copies of the digital recordings and relevant supplemental materials, such as transcriptions and annotations. %as of 4\textsuperscript{th} OCTOBER 2011. 
the explanations for these abbreviations are listed below. %\ref{inventoryAbbs} below. %on page \pageref{inventoryAbbs}. %
The column ‘words’ indicates the current number of transcribed and translated \PS\ words for each recording; currently, some have only been partly transcribed, and some not at all, particularly newer materials. 
%\footnote{Note that some archived recordings in this inventory have been only partly transcribed or not at all.} 
The column ‘brief description’ provides a summary of the content of each recording; more detailed descriptions are provided in the metadata available at the archives hosting the materials. 

\vfill

%\begin{table}
{
\begin{center}
%\begin{tabular}{c}
Abbreviations used in the inventory of recordings\\\vspace{10pt}
%\end{tabular}
\begin{tabular}{l l l}\mytoprule
{column} &{abbreviation}	&{explanation} \\\hline
{media}
& A				& audio only\\
& A/V			& audio and video \\%\hline
{genre}
& conv.			& conversation \\
& elic.			& elicitation \\
& explan.			& explanation \\
& narr.			& narration \\
& perf.			& performance \\
& read.			& reading \\
& writ.			& written text \\
\dline
\end{tabular}
%\end{table}
\end{center}}
\vfill

\begin{landscape}
%\begin{longtable}[c]{| l | x{15mm} | c | c | p{95mm} |}
\begin{longtable}[c]{ l  x{15mm}  c  c  p{95mm} }
\mytoprule%\hline\hline
{name}	&{genre}	&{medium}	&{words}	&{brief description}\\\hline\endhead
\mybottomrule\endfoot
%\multicolumn{5}{l}{here end\label{inventoryEnd}}\endlastfoot
\label{inventoryBegin}\hypertarget{pit080621}{pit080621} & elic. & A & 119 & Basic phrases, wordlist \\\hline %
\hypertarget{pit080622a}{pit080622a} & elic. & A & 168 & Verb paradigms (\Sc{prs}); numbers 1–1000 \\\hline %
\hypertarget{pit080622b}{pit080622b} & elic. & A & 6 & Phrase: “thank you” \\\hline %
\hypertarget{pit080627}{pit080627} & elic. & A & 39 & More exact numbers \\\hline %
\hypertarget{pit080701a}{pit080701a} & elic. & A & 4 & Phrase: "how do you say \_\_\_ in Saami" \\\hline %
\hypertarget{pit080701b}{pit080701b} & elic. & A & 116 & Swadesh list - words: 7–15, 17–21, 27–90, 91–114 \\\hline %
\hypertarget{pit080702a}{pit080702a} & elic. & A & 25 & A few short words/phrases (landscape words etc.) \\\hline %
\hypertarget{pit080702b}{pit080702b} & elic. & A & 226 & Swadesh list - words: 115–207; a few short phrases \\\hline %
\hypertarget{pit080703}{pit080703} & explan. & A/V & 334 & Descriptions of two pictures: 1: Saami camp 2: Reindeer in the tundra \\\hline %
\hypertarget{pit080708_Session01}{pit080708\_Session01} & explan. & A/V & 0 & Description of reindeer saddle carriers \\\hline %
\hypertarget{pit080708_Session02}{pit080708\_Session02} & explan. & A/V & 202 & Description of reindeer milking \\\hline %
\hypertarget{pit080708_Session03}{pit080708\_Session03} & explan. & A/V & 136 & Description of making butter, butter dishes and other dishes \\\hline %
\hypertarget{pit080708_Session04}{pit080708\_Session04} & explan. & A/V & 151 & Description of a Saami chest \\\hline %
\hypertarget{pit080708_Session05}{pit080708\_Session05} & explan. & A/V & 14 & Description of some traditional Pite Saami objects \\\hline %
\hypertarget{pit080708_Session06}{pit080708\_Session06} & explan. & A/V & 0 & Description of a Saami shirt design, reindeer-skin shoes \\\hline %
\hypertarget{pit080708_Session07}{pit080708\_Session07} & explan. & A/V & 74 & Description of some traditional Saami tools:  a weaving reed (a sort of mini-loom), a lasso ring and an unfinished sheath \\\hline %
\hypertarget{pit080708_Session08}{pit080708\_Session08} & explan. & A/V & 164 & Description of Saami kids’ shoes, two hats, reindeer-fur gloves \\\hline %
\hypertarget{pit080708_Session09}{pit080708\_Session09} & explan. & A/V & 124 & Description of animal traps \\\hline %
\hypertarget{pit080708_Session10}{pit080708\_Session10} & explan. & A/V & 0 & Description of how animal traps work \\\hline %
\hypertarget{pit080803a}{pit080803a} & elic. & A & 11 & Reindeer antler terms \\\hline %
\hypertarget{pit080803b}{pit080803b} & elic. & A & 118 & Ordinal numbers 1–10, >10 \\\hline %
\hypertarget{pit080811b1}{pit080811b1} & elic. & A & 4 & Numerals (\Sc{card}\PLUS\Sc{ord}), some verb paradigms \\\hline %
\hypertarget{pit080811b2}{pit080811b2} & elic. & A & 9 & Pronouns (\Sc{nom}), and other random words \\\hline %
\hypertarget{pit080813}{pit080813} & elic. & A & 163 & Verb paradigms \\\hline %
\hypertarget{pit080818}{pit080818} & elic., explan. & A/V & 0 & Reindeer-related words; numbers 1–10+ (\Sc{card}), 1–9 (\Sc{ord}) \\\hline %
\hypertarget{pit080819a}{pit080819a} & elic. & A & 448 & Adjective paradimgs \\\hline %
\hypertarget{pit080819b}{pit080819b} & elic. & A & 1 & Word: “sárrge” \\\hline %
\hypertarget{pit080819c}{pit080819c} & elic. & A & 17 & Phrase: “thank you for today” \\\hline %
\hypertarget{pit080825}{pit080825} & narr., song & A/V & 270 & Description of speaker’s family and her life in her childhood home; Singing of two hymns \\\hline %
\hypertarget{pit080909}{pit080909} & explan. & A/V & 736 & Film of reindeer roundup/slaughter, including footage of reindeer being selected, caught, slaughtered, and commentary on butchering a reindeer \\\hline %
\hypertarget{pit080917a}{pit080917a} & elic. & A & 201 & Some question words; some noun paradigms \\\hline %
\hypertarget{pit080917b}{pit080917b} & elic. & A & 8 & Numerals 20–30, 40, 50, 60, 1000, 2000 \\\hline %
\hypertarget{pit080917c}{pit080917c} & elic. & A & 36 & Some noun paradigms; some of question word paradigms “what” and “who” \\\hline %
\hypertarget{pit080924}{pit080924} & conv. & A/V & 2440 & Conversation about old times in Ákkapakte \\\hline %
\hypertarget{pit080926}{pit080926} & elic. & A & 107 & Word list from Pite-saami lessons from 25/26 september 2008 \\\hline %
\hypertarget{pit081011}{pit081011} & elic. & A & 355 & Random words collected during a previous Pite Saami lesson \\\hline %
\hypertarget{pit081012a}{pit081012a} & narr. & A & 0 & Descriptions of pictures from photo album, mostly of reindeer and calf marking \\\hline %
\hypertarget{pit081012b}{pit081012b} & elic. & A & 0 & Random words, mostly resulting from pit081012a \\\hline %
\hypertarget{pit081017}{pit081017} & elic. & A & 8 & Days of the week; months; seasons \\\hline %
\hypertarget{pit081021a_Story}{pit081021a\_Story} & read. & A/V & 0 & Reading of a story by Lars Rensund \\\hline %
\hypertarget{pit081021b}{pit081021b} & elic. & A & 0 & Demonstratives; some vocab from pit080708\_Session08 \\\hline %
\hypertarget{pit081028}{pit081028} & elic. & A & 29 & Words beginning with “sjnj-” (/ʃɲ/) \\\hline %
\hypertarget{pit081106}{pit081106} & explan. & A/V & 0 & Description of objects from Saami exhibit at Silvermuseet \\\hline %
\hypertarget{pit081111}{pit081111} & elic. & A & 55 & Adjective paradigms; some lexical items \\\hline %
\hypertarget{pit090411}{pit090411} & song, read., perf., writ. & A/V & 0 & Reading of scripture, singing of hymn \\\hline %
\hypertarget{pit090513}{pit090513} & elic. & A & 18 & Paradigm for noun \It{sábme} ‘Saami’ \\\hline %
\hypertarget{pit090519}{pit090519} & conv. & A/V & 1247 & A group of language activists have a picnic around a campfire, sometimes discussing words for a word list, but also just chatting \\\hline %
\hypertarget{pit090525a}{pit090525a} & elic. & A & 24 & Noun paradigms for: sábme (Saami), bena (dog) \\\hline %
\hypertarget{pit090525b}{pit090525b} & elic. & A & 77 & Six noun paradigms; short discussion of (near) minimal pairs \\\hline %
\hypertarget{pit090525c}{pit090525c} & conv. & A & 1 & Word \It{buris(t)} \\\hline %
\hypertarget{pit0906_Ahkajavvre_a}{pit0906\_Ahkajavvre\_a} & explan., narr. & A/V & 1105 & Description of the history and buildings at Ahkajavvre; performance of how to retrieve fishing nets and fish, and how to gut and wash fish; recorded on 9/10 June 2009 \\\hline %
\hypertarget{pit0906_Ahkajavvre_b}{pit0906\_Ahkajavvre\_b} & explan. & A/V & 301 & Description of the history of Ahkajavvre; recorded on 9/10 June 2009 \\\hline %
\hypertarget{pit090625}{pit090625} & elic. & A & 0 & A few words from the loanword typology list, mostly about geographic features \\\hline %
\hypertarget{pit090630}{pit090630} & conv., narr. & A/V & 103 & Conversation about a trip to Västerfjäll, driving across Tjeggelvas, going to school in Arjeplog; telling of a ghost story \\\hline %
\hypertarget{pit090702}{pit090702} & conv., narr. & A/V & 2245 & Conversation about fishing, hunting moose and preparing food in Västerfjäll/Álesgiehtje and Áhkkabakkte \\\hline %
\hypertarget{pit090705}{pit090705} & explan., narr. & A/V & 0 & On the way to and at reindeer calf marking the night of 5-6 July 2009 \\\hline %
\hypertarget{pit090821}{pit090821} & elic., explan. & A & 0 & A variety of words relating to berries, insects, house, etc. \\\hline %
\hypertarget{pit090822}{pit090822} & explan., narr. & A/V & 0 & Description of a variety of places around the speaker’s family homestead \\\hline %
\hypertarget{pit090823}{pit090823} & explan. & A/V & 0 & Description of the old house at the speaker’s family homestead \\\hline %
\hypertarget{pit090826}{pit090826} & explan. & A & 301 & Description of how reindeer herders look for unmarked calves \\\hline %
\hypertarget{pit090910}{pit090910} & elic. & A & 101 & Reflexive pronouns; some verb paradigms \\\hline %
\hypertarget{pit090912}{pit090912} & explan., conv., narr. & A/V & 0 & Video of reindeer slaughter, including first stages of butchering a reindeer \\\hline %
\hypertarget{pit090915a}{pit090915a} & narr. & A/V & 131 & Speaker talks about the hill on which Samegården stands \\\hline %
\hypertarget{pit090915b}{pit090915b} & narr. & A/V & 115 & Speaker talks about a pond that used to be in central Arjeplog, and how the Pite Saami name ‘Árjepluovve’ got its name from that pond \\\hline %
\hypertarget{pit090915c}{pit090915c} & narr. & A/V & 312 & Speaker talks about ‘Saami hill’ in central Arjeplog \\\hline %
\hypertarget{pit090915d}{pit090915d} & narr. & A/V & 103 & Speaker talks about ‘Knabben’-hill in central Arjeplog \\\hline %
\hypertarget{pit090915e}{pit090915e} & narr. & A/V & 116 & Speaker talks about the lake Hornavan/Tjårvek \\\hline %
\hypertarget{pit090915f}{pit090915f} & narr. & A/V & 225 & Speaker talks about the Skeppsviken/Hakksaluakkta in Arjeplog \\\hline %
\hypertarget{pit090915g}{pit090915g} & narr. & A/V & 201 & Speaker talks about the Skeppsholmen/Hakksasuolo in Arjeplog \\\hline %
\hypertarget{pit090915h}{pit090915h} & narr. & A/V & 87 & Speaker talks about a ‘njalla’ (raised storage shed) on Skeppsholmen in Arjeplog \\\hline %
\hypertarget{pit090915i}{pit090915i} & narr. & A/V & 123 & Speaker talks about a ‘luäppte’-storage shed on Skeppsholmen in Arjeplog \\\hline %
\hypertarget{pit090915j}{pit090915j} & narr. & A/V & 195 & Speaker talks about how reindeer and calves used to swim across the bay to Kraja \\\hline %
\hypertarget{pit090915k}{pit090915k} & narr. & A/V & 207 & Speaker talks about the location of the original marketplace in Arjeplog \\\hline %
\hypertarget{pit090926}{pit090926} & elic. & A & 494 & Adjective paradigms \\\hline %
\hypertarget{pit090927}{pit090927} & elic. & A & 445 & Adjective paradigms \\\hline %
\hypertarget{pit090930a}{pit090930a} & elic. & A & 829 & Adjectives in elliptical NPs; some color adjectives; more adjective paradigms \\\hline %
\hypertarget{pit090930b}{pit090930b} & elic. & A & 120 & Adjective paradigms \\\hline %
\hypertarget{pit091001}{pit091001} & elic. & A & 281 & Adjective paradigms \\\hline %
\hypertarget{pit100304}{pit100304} & elic. & A & 62 & Basic random wordlist (from the Leipzig-Jakarta list of basic vocabulary) \\\hline %
\hypertarget{pit100308a}{pit100308a} & elic. & A & 10 & Some basic elicitation forms, \Sc{nom.sg} and \Sc{acc.sg} noun paradigms \\\hline %
\hypertarget{pit100310b}{pit100310b} & narr., explan. & A/V & 0 & Description of a slide show concerning life in Álesgiehtje/Västerfjäll \\\hline %
\hypertarget{pit100323a}{pit100323a} & elic. & A & 481 & Verb paradigms \\\hline %
\hypertarget{pit100323b}{pit100323b} & song & A & 28 & Singing of a lullaby \\\hline %
\hypertarget{pit100324}{pit100324} & elic. & A/V & 291 & Expressions for spatial relations (mostly postpositions) \\\hline %
\hypertarget{pit100326}{pit100326} & elic. & A & 0 & Postpositions; some basic elicitation of existentials and demonstratives \\\hline %
\hypertarget{pit100403}{pit100403} & perf., read., writ. & A/V & 227 & Reading of scripture for Easter church service \\\hline %
\hypertarget{pit100404}{pit100404} & explan. & A/V & 1704 & Description of the winter landscape around Västerfjäll/Álesgiehtje, skiing, snowmobiles, playing there as a child, trapping ptarmigan, etc. \\\hline %
\hypertarget{pit100405a}{pit100405a} & explan., narr. & A & 758 & Description of the current winter from a reindeer herder’s perspective and of the activities that went on at the recording location near Blavvtajåhkå \\\hline %
\hypertarget{pit100405b}{pit100405b} & explan. & A/V & 461 & Description of different kinds of reindeer \\\hline %
\hypertarget{pit100703a}{pit100703a} & narr. & A & 312 & Story about waiting for the bus with the narrator’s aunt \\\hline %
\hypertarget{pit101208}{pit101208} & elic. & A & 674 & Verb paradigms \\\hline %
\hypertarget{pit110329}{pit110329} & elic. & A & 112 & Pronoun paradigms (personal, demonstrative, relative, interrogative); a few Saami village lexical items \\\hline %
\hypertarget{pit110331a}{pit110331a} & elic. & A & 405 & Pronoun paradigms (personal, demonstrative, interrogative, reflexive) for \Sc{nom},\Sc{acc}, \Sc{gen},\Sc{ill} \\\hline %
\hypertarget{pit110331b}{pit110331b} & elic. & A & 371 & Pronoun paradigms (personal, demonstrative, interrogative, reflexive) for \Sc{iness}, \Sc{elat}, \Sc{com} \\\hline %
\hypertarget{pit110404}{pit110404} & elic. & A & 530 & Verb paradigms \\\hline %
\hypertarget{pit110413a}{pit110413a} & elic. & A & 394 & Some noun paradigms \\\hline %
\hypertarget{pit110413b}{pit110413b} & elic. & A & 382 & Noun paradigms; ‘båtsoj’ includes numerals and quantifiers \\\hline %
\hypertarget{pit110415}{pit110415} & elic. & A & 158 & Kinship vocabulary; paradigm for ålmaj (man) \\\hline %
\hypertarget{pit110421}{pit110421} & elic. & A & 89 & Noun paradigms for juällge (leg/foot) and rejjdo (tool) \\\hline %
\hypertarget{pit110509a}{pit110509a} & elic. & A & 239 & Random verbs; random questions about subordinate clause linking; some noun paradigms \\\hline %
\hypertarget{pit110509b}{pit110509b} & elic. & A & 111 & Noun paradigms \\\hline %
\hypertarget{pit110517a}{pit110517a} & elic. & A & 870 & Some verb paradigms; includes some potential forms of verbs \\\hline %
\hypertarget{pit110517b1}{pit110517b1} & elic. & A & 0 & Some verb paradigms \\\hline %
\hypertarget{pit110517b2}{pit110517b2} & narr. & A & 380 & Short narrative about the orthography workshop on the previous weekend \\\hline %
\hypertarget{pit110518a}{pit110518a} & elic. & A & 187 & Verb paradigms; discussion of passives \\\hline %
\hypertarget{pit110518b}{pit110518b} & elic. & A & 0 & Some aspects in verbs; verbs for scratch/dig \\\hline %
\hypertarget{pit110519a}{pit110519a} & elic. & A & 0 & Some partial verb paradigms; some verbal derivations \\\hline %
\hypertarget{pit110519b}{pit110519b} & elic. & A & 47 & Some conjunction/subordinators;  question particle discussion;  some partial noun paradigms \\\hline %
\hypertarget{pit110521a}{pit110521a} & elic. & A & 150 & Pronoun paradigms: \Sc{nom}, \Sc{acc} for personal, most reflexive, demonstrative, question, selection, relative pronouns \\\hline %
\hypertarget{pit110521b1}{pit110521b1} & elic. & A & 529 & Some pronoun paradigms; a short narrative about what speaker did yesterday in Piteå \\\hline %
\hypertarget{pit110521b2}{pit110521b2} & narr. & A & 8 & A short narrative about what speaker did yesterday in Piteå \\\hline %
\hypertarget{pit110522}{pit110522} & elic. & A & 213 & Some verb and noun paradigms \\\hline %
\hypertarget{sje20121009}{sje20121009} & elic. & A & 252 & random grammatical topics: - DIM allomorphy - contracted verbs (gullut) - NEG.IMP (SG/DU/PL) - DEM - 3-way distinction - ADJ vs. ADV - passives - ADJ as head of NP - possessive suffixes \\\hline %
\hypertarget{sje20121014a}{sje20121014a} & elic. & A & 1 & Questions meant to elicit suspected differences between Pite Saami dialects \\\hline %
\hypertarget{sje20121014b}{sje20121014b} & conv. & A/V & 0 & Conversation about the old days, topics such as reindeer calves, coffee cheese, eating bear meat, seeing a bear, etc. \\\hline %
\hypertarget{sje20121011}{sje20121011} & elic. & A & 178 & Questions about \Sc{dim} suffix allomorphy in nouns; ‘contracted’ verb paradigm; imperative of negation verb; possessive suffixes, etc. \\\hline %
\hypertarget{sje20121014d1}{sje20121014d1} & elic. & A & 3 & NP-syntax, gapping; \Sc{comp} for Adj; coordination, complementizers \\\hline %
\hypertarget{sje20130523}{sje20130523} & narr. & A/V & 763 & A narrative about going to church and praying \\\hline %
\hypertarget{sje20130530b}{sje20130530b} & conv. & A & 1067 & a discussion about words, coffee and other topics while preparing and drinking coffee in the kitchen \\\hline %
\hypertarget{sje20131006}{sje20131006} & conv., explan., narr. & A/V & 181 & Discussion of the homestead; Speaker plays with his granddaughte; describes various traditional sheds on his property. \\\hline %
\hypertarget{sje20131012a}{sje20131012a} & conv. & A/V & 0 & Short conversation about Trollforsen and some of the old Saami huts there \\\hline %
\hypertarget{sje20131012b}{sje20131012b} & conv., narr., explan. & A/V & 0 & Tour of the homestead there, gathering reindeer moss, playing with a speaker’s granddaughter, tour of pictures and artefacts in the living room \\\hline %
\hypertarget{sje20131031}{sje20131031} & explan., narr. & A/V & 0 & Description of various things related to driving to check on the reindeer and then a potential moose hunt \\\hline %
\end{longtable}
\phantomsection\label{inventoryEnd}%%allows end of table to be referred to by \pageref{}
\end{landscape}


%\end{document}