%\documentclass[ number=5
			   ,series=sidl
			   ,isbn=xxx-x-xxxxxx-xx-x
			   ,url=http://langsci-press.org/catalog/book/17
			   ,output=long   % long|short|inprep              
			   %,blackandwhite
			   %,smallfont
			   ,draftmode   
			  ]{LSP/langsci}                          

\usepackage{LSP/lsp-styles/lsp-gb4e}		% verhindert Komma bei mehrfachen Fußnoten?
                                                      
\usepackage{layout}
\usepackage{lipsum}

%%%% ABOVE FOR LangSciPress %%%%
%%%% ABOVE FOR LangSciPress %%%%
%%%% ABOVE FOR LangSciPress %%%%
\usepackage{libertine}%work-around solution for rendering problematic characters ʦ, ͡  (mostly in \textbf{})

\usepackage{longtable}%Double-lines (\hline\hline) aren’t typeset properly in ‘longtable’-environment across several pages! conflict with other package (maybe xcolor with option ‘tables’?)

\usepackage{multirow}

\usepackage{array} %allows, among other things, centering column content in a table while also specifying width, creates new column style "x" for center-alignment, "y" for right-alignment
\newcolumntype{x}[1]{>{\centering\hspace{0pt}}p{#1}}%
\newcolumntype{y}[1]{>{\raggedleft\hspace{0pt}}p{#1}}%

\usepackage[]{placeins}%using \FloatBarrier command, all floats still floating at that point will be typeset, and cannot cross that boundary. the option here \usepackage[section]{placeins} automatically adds \FloatBarrier to every \section command (only works for \section commands, nothing lower than that!)
%\usepackage{afterpage}%by using the command \afterpage{\clearpage}, all floats will appear, but no new page will be started, thus avoiding bad page breaks around floats

\usepackage{vowel} %for vowel space chart


%%%IS THIS NECESSARY??
%%%%following allows you to refer to footnotes (from http://anthony.liekens.net/index.php/LaTeX/MultipleFootnoteReferences)
%\newcommand{\footnoteremember}[2]{
%  \footnote{#2}
%  \newcounter{#1}
%  \setcounter{#1}{\value{footnote}}
%} \newcommand{\footnoterecall}[1]{
%  \footnotemark[\value{#1}]} 
%%%%previous allows you to refer to footnotes: use \footnoteremember{referenceText} in footnote, then \footnoterecall{referenceText} to refer.

\usepackage{tikz}%
\usetikzlibrary{plothandlers,matrix,decorations.text,shapes.arrows,shadows,chains,positioning,scopes}

\usepackage{synttree} %zeichnet linguistische Bäume
\branchheight{36pt}%sets height between rows in synttree

\usepackage{lscape}%used for landscape pages in index (list of recordings)

\usepackage{polyglossia}
\setmainlanguage{english}


%%%TAKE OUT FOR FINAL VERSION:
%%%TAKE OUT FOR FINAL VERSION:
%%%TAKE OUT FOR FINAL VERSION:

%%%%following readjusts margin text!
%\setlength{\marginparwidth}{20mm}
%\let\oldmarginpar\marginpar
%\renewcommand\marginpar[1]{\-\oldmarginpar[\raggedleft\footnotesize\vspace{-7pt}\color{red}\It{→ #1}]%
%{\raggedright\footnotesize\vspace{-7pt}\color{red}\It{→ #1}}}
%%%%previous readjusts margin text!

%%%The following lines set depth of ToC (LSP default is only 3 levels)!
%%%\renewcommand{\contentsname}{Table of Contents} % überschrift des inhaltsverzeichnisses
%\setcounter{secnumdepth}{5}%sets how deep section/subsection/subsubsections are numbered
%\setcounter{tocdepth}{5}%sets the depth of the ToC %but this doesn't seem to work!!!
%% new commands for LSP book (Grammar of Pite Saami, by J. Wilbur)

\newcommand{\PS}{Pite Saami}
\newcommand{\PSDP}{Pite Saami Documentation Project}
\newcommand{\WLP}{Wordlist Project}

\newcommand{\HANG}{\everypar{\hangindent15pt \hangafter1}}%also useful for table cells
\newcommand{\FB}{\FloatBarrier}%shortcut for this command to print all floats w/o pagebreak

\newcommand{\REF}[1]{(\ref{#1})}%adds parenthesis around the reference number, particularly useful for examples.%\Ref had clash with LSP!
\newcommand{\dline}{\hline\hline}%makes a double line in a table
\newcommand{\superS}[1]{\textsuperscript{#1}}%adds superscript element
\newcommand{\sub}[1]{$_{#1}$}%adds subscript element
\newcommand{\Sc}[1]{\textsc{#1}}%shortcut for small capitals (not to be confused with \sc, which changes the font from that point on)
\newcommand{\It}[1]{\textit{#1}}%shortcut for italics (not to be confused with \it, which changes the font from that point on)
\newcommand{\Bf}[1]{\textbf{#1}}%shortcut for bold (not to be confused with \bf, which changes the font from that point on)
\newcommand{\BfIt}[1]{\textbf{\textit{#1}}}
\newcommand{\BfSc}[1]{\textbf{\textsc{#1}}}
\newcommand{\Tn}[1]{\textnormal{#1}}%shortcut for normal text (undo italics, bolt, etc.)
\newcommand{\MC}{\multicolumn}%shortcut for multicolumn command in tabular environment - only replaces command, not variables!
\newcommand{\MR}{\multirow}%shortcut for multicolumn command in tabular environment - only replaces command, not variables!
\newcommand{\TILDE}{∼}%U+223C %OLD:~}%shortcut for tilde%command ‘\Tilde’ clashes with LSP!%
\newcommand{\BS}{\textbackslash}%backslash
\newcommand{\Red}[1]{{\color{red}{#1}}}%for red text
\newcommand{\Blue}[1]{{\color{blue}{#1}}}%for blue text
\newcommand{\PLUS}{+}%nicer looking plus symbol
\newcommand{\MINUS}{-}%nicer looking plus symbol
%    Was die Pfeile betrifft, kannst Du mal \Rightarrow \mapsto \textrightarrow probieren und dann \mathbf \boldsymbol oder \pbm dazutun.
\newcommand{\ARROW}{\textrightarrow}%→%dieser dicke Pfeil ➜ wird nicht von der LSP-Font unterstützt: %\newcommand{\ARROW}{{\fontspec{DejaVu Sans}➜}}
\newcommand{\DARROW}{\textleftrightarrow}%↔︎%DoubleARROW
\newcommand{\BULLET}{•}%
%%✓ does not exist in the default LSP font!
\newcommand{\CH}{\checkmark}%%\newcommand{\CH}{\fontspec{Arial Unicode MS}✓}%CH as in CHeck
%%following used to separate alternation forms for consonant gradation and umlaut patterns:
\newcommand{\Div}{‑}%↔︎⬌⟷⬄⟺⇔%non-breaking hyphen: ‑  
\newcommand{\QUES}{\textsuperscript{?}}%marks questionable/uncertain forms

\newcommand{\jvh}{\mbox{\It{j}-suffix} vowel harmony}%
%\newcommand{\Ptcl}{\Sc{ptcl} }%just shortcut for glossing ‘particle’
%\newcommand{\ATTR}{{\Sc{attributive}}}%shortcut for ATTRIBUTIVE in small caps
%\newcommand{\PRED}{{\Sc{predicative}}}%shortcut for PREDICATIVE in small caps
%\newcommand{\COMP}{{\Sc{comparative}}}%shortcut for COMPARATIVE in small caps
%\newcommand{\SUPERL}{{\Sc{superlative}}}%shortcut for SUPERLATIVE in small caps
\newcommand{\SG}{{\Sc{singular}}}%shortcut for SINGULAR in small caps
\newcommand{\DU}{{\Sc{dual}}}%shortcut for DUAL in small caps
\newcommand{\PL}{{\Sc{plural}}}%shortcut for PLURAL in small caps
%\newcommand{\NOM}{{\Sc{nominative}}}%shortcut for NOMINATIVE in small caps
%\newcommand{\ACC}{{\Sc{accusative}}}%shortcut for ACCUSATIVE in small caps
%\newcommand{\GEN}{{\Sc{genitive}}}%shortcut for GENITIVE in small caps
%\newcommand{\ILL}{{\Sc{illative}}}%shortcut for ILLATIVE in small caps
%\newcommand{\INESS}{{\Sc{inessive}}}%shortcut for INESSIVE in small caps
\newcommand{\ELAT}{{\Sc{elative}}}%shortcut for ELATIVE in small caps
%\newcommand{\COM}{{\Sc{comitative}}}%shortcut for COMITATIVE in small caps
%\newcommand{\ABESS}{{\Sc{abessive}}}%shortcut for ABESSIVE in small caps
%\newcommand{\ESS}{{\Sc{essive}}}%shortcut for ESSIVE in small caps
%\newcommand{\DIM}{{\Sc{diminutive}}}%shortcut for DIMINUTIVE in small caps
%\newcommand{\ORD}{{\Sc{ordinal}}}%shortcut for ORDINAL in small caps
%\newcommand{\CARD}{{\Sc{cardinal}}}%shortcut for CARDINAL in small caps
%\newcommand{\PROX}{{\Sc{proximal}}}%shortcut for PROXIMAL in small caps
%\newcommand{\DIST}{{\Sc{distal}}}%shortcut for DISTAL in small caps
%\newcommand{\RMT}{{\Sc{remote}}}%shortcut for REMOTE in small caps
%\newcommand{\REFL}{{\Sc{reflexive}}}%shortcut for REFLEXIVE in small caps
%\newcommand{\PRS}{{\Sc{present}}}%shortcut for PRESENT in small caps
%\newcommand{\PST}{{\Sc{past}}}%shortcut for PAST in small caps
%\newcommand{\IMP}{{\Sc{imperative}}}%shortcut for IMPERATIVE in small caps
%\newcommand{\POT}{{\Sc{potential}}}%shortcut for POTENTIAL in small caps
\newcommand{\PROG}{{\Sc{progressive}}}%shortcut for PROGRESSIVE in small caps
\newcommand{\PRF}{{\Sc{perfect}}}%shortcut for PERFECT in small caps
\newcommand{\INF}{{\Sc{infinitive}}}%shortcut for INFINITIVE in small caps
%\newcommand{\NEG}{{\Sc{negative}}}%shortcut for NEGATIVE in small caps
\newcommand{\CONNEG}{{\Sc{connegative}}}%shortcut for CONNEGATIVE in small caps
\newcommand{\ATTRs}{{\Sc{attr}}}%shortcut for ATTR in small caps
\newcommand{\PREDs}{{\Sc{pred}}}%shortcut for PRED in small caps
%\newcommand{\COMPs}{{\Sc{comp}}}%shortcut for COMP in small caps
%\newcommand{\SUPERLs}{{\Sc{superl}}}%shortcut for SUPERL in small caps
\newcommand{\SGs}{{\Sc{sg}}}%shortcut for SG in small caps
\newcommand{\DUs}{{\Sc{du}}}%shortcut for DU in small caps
\newcommand{\PLs}{{\Sc{pl}}}%shortcut for PL in small caps
\newcommand{\NOMs}{{\Sc{nom}}}%shortcut for NOM in small caps
\newcommand{\ACCs}{{\Sc{acc}}}%shortcut for ACC in small caps
\newcommand{\GENs}{{\Sc{gen}}}%shortcut for GEN in small caps
\newcommand{\ILLs}{{\Sc{ill}}}%shortcut for ILL in small caps
\newcommand{\INESSs}{{\Sc{iness}}}%shortcut for INESS in small caps
\newcommand{\ELATs}{{\Sc{elat}}}%shortcut for ELAT in small caps
\newcommand{\COMs}{{\Sc{com}}}%shortcut for COM in small caps
\newcommand{\ABESSs}{{\Sc{abess}}}%shortcut for ABESS in small caps
\newcommand{\ESSs}{{\Sc{ess}}}%shortcut for ESS in small caps
%\newcommand{\DIMs}{{\Sc{dim}}}%shortcut for DIM in small caps
%\newcommand{\ORDs}{{\Sc{ord}}}%shortcut for ORD in small caps
%\newcommand{\CARDs}{{\Sc{card}}}%shortcut for CARD in small caps
\newcommand{\PROXs}{{\Sc{prox}}}%shortcut for PROX in small caps
\newcommand{\DISTs}{{\Sc{dist}}}%shortcut for DIST in small caps
\newcommand{\RMTs}{{\Sc{rmt}}}%shortcut for RMT in small caps
\newcommand{\REFLs}{{\Sc{refl}}}%shortcut for REFL in small caps
\newcommand{\PRSs}{{\Sc{prs}}}%shortcut for PRS in small caps
\newcommand{\PSTs}{{\Sc{pst}}}%shortcut for PST in small caps
\newcommand{\IMPs}{{\Sc{imp}}}%shortcut for IMP in small caps
\newcommand{\POTs}{{\Sc{pot}}}%shortcut for POT in small caps
\newcommand{\PROGs}{{\Sc{prog}}}%shortcut for PROG in small caps
\newcommand{\PRFs}{{\Sc{prf}}}%shortcut for PRF in small caps
\newcommand{\INFs}{{\Sc{inf}}}%shortcut for INF in small caps
\newcommand{\NEGs}{{\Sc{neg}}}%shortcut for NEG in small caps
\newcommand{\CONNEGs}{{\Sc{conneg}}}%shortcut for CONNEG in small caps

\newcommand{\subNP}{{\footnotesize\sub{NP}}}%shortcut for NP (nominal phrase) in subscript
\newcommand{\subVC}{{\footnotesize\sub{VC}}}%shortcut for VC (verb complex) in subscript
\newcommand{\subAP}{{\footnotesize\sub{AP}}}%shortcut for NP (adjectival phrase) in subscript
\newcommand{\subAdvP}{{\footnotesize\sub{AdvP}}}%shortcut for AdvP (adverbial phrase) in subscript
\newcommand{\subPP}{{\footnotesize\sub{PP}}}%shortcut for NP (postpoistional phrase) in subscript

\newcommand{\ipa}[1]{{\fontspec{Linux Libertine}#1}}%specifying font for IPA characters

\newcommand{\SEC}{§}%standardize section symbol and spacing afterwards
%\newcommand{\SEC}{§\,}%

\newcommand{\Nth}{{\footnotesize(\It{n})}}%used in table of numerals in ADJ chapter

%%newcommands for tables in introductionSDL.tex:
\newcommand{\cliticExs}[3]{\Tn{\begin{tabular}{p{28mm} c p{28mm} p{35mm}}\It{#1}&\ARROW &\It{#2} & ‘#3’\\\end{tabular}}}%specifically for the two clitic examples
\newcommand{\Grapheme}[1]{\It{#1}}%formatting for graphemes in orthography tables
%%new command for the section on orthographic examples; syntax: #1=orthography, #2=phonology, #3=gloss
\newcommand{\SpellEx}[3]{\Tn{\begin{tabular}{p{70pt} p{70pt} l}\ipa{/#2/}&\It{#1}& ‘#3’ \\\end{tabular}}}%formatting for orthographic examples (intro-Chapter)


%%new transl tier in gb4e; syntax: #1=free translation (in single quotes), #2=additional comments, z.B. literal meaning:
\newcommand{\Transl}[2]{\trans\Tn{‘#1’ #2}}%new transl tier in gb4e;
\newcommand{\TranslMulti}[2]{\trans\hspace{12pt}\Tn{‘#1’ #2}}%new transl tier in gb4e for a dialog to be included under a single example number


%% used for examples in the Prosody and Segmental phonology chapters:
\newcommand{\PhonGloss}[7]{%PhonGloss = Phonology Gloss;
%pattern: \PhonGloss{label}{phonemic}{phonetic}{orthographic}{gloss}{recording}{utterance}
\ea\label{#1}
\Tn{\begin{tabular}[t]{p{30mm} l}
\ipa{/#2/}	& \It{#4} \\
\ipa{[#3]}	&\HANG ‘#5’\\%no table row can start with square brackets! thus the workaround with \MC
\end{tabular}\hfill\hyperlink{#6}{{\small\textnormal[pit#6#7]}}%\index{Z\Red{rec}!\Red{pit#6}}\index{Z\Red{utt}!\Red{pit#6#7} \Blue{Phon}}
}
\z}
\newcommand{\PhonGlossWL}[6]{%PhonGloss = Phonology Gloss for words from WORDLIST, not from corpus!;
%pattern: \PhonGloss{label}{phonemic}{phonetic}{orthographic}{gloss}{wordListNumber}
\ea\label{#1}
\Tn{\begin{tabular}[t]{p{30mm} l}
\ipa{/#2/}	& \It{#4} \\
\ipa{[#3]}	&\HANG ‘#5’\\%no table row can start with square brackets! thus the workaround with \MC
\end{tabular}\hfill\hyperlink{explExs}{{\small\textnormal[#6]}}%\index{Z\Red{wl}!\Red{#6}\Blue{Phon}}
}
\z}

%%for derivation examples in the derivational morphology chapter!
%syntax: \DerivExam{#1}{#2}{#3}{#4}{#5}{#6}
%#1: base, #2: base-gloss, #3: derived form, #4: derived form gloss, #5: derived form translation, #6: pit-recording, #7: utterance number
\newcommand{\DW}{28mm}%for following three commands, to align arrows throughout
%%%%OLD:
%%%\newcommand{\DerivExam}[7]{\Tn{\begin{tabular}[t]{p{\DW}cl}\It{#1}&\ARROW&\It{#3}\\#2&&#4\\\end{tabular}\hfill\pbox{.3\textwidth}{\hfill‘#5’\\\hbox{}\hfill\hyperlink{pit#6}{{\small\textnormal[pit#6.#7]}}}
%%%%\index{Z\Red{rec}!\Red{pit#6}}\index{Z\Red{utt}!\Red{pit#6.#7}}
%%%}}
%NEW:
\newcommand{\DerivExam}[7]{\Tn{
\begin{tabular}[t]{p{\DW}x{5mm}l}\It{#1}&\ARROW&\It{#3}\\\end{tabular}\hfill‘#5’\\
\hspace{1mm}\begin{tabular}[t]{p{\DW}x{5mm}l}#2&&#4\\\end{tabular}\hfill\hyperlink{pit#6}{{\small\textnormal[pit#6.#7]}}
%\index{Z\Red{rec}!\Red{pit#6}}\index{Z\Red{utt}!\Red{pit#6.#7}}
}}
%%same as above, but supress any reference to a specific utterance
\newcommand{\DerivExamX}[7]{\Tn{
\begin{tabular}[t]{p{\DW}x{5mm}l}\It{#1}&\ARROW&\It{#3}\\\end{tabular}\hfill‘#5’\\
\hspace{1mm}\begin{tabular}[t]{p{\DW}x{5mm}l}#2&&#4\\\end{tabular}\hfill\hyperlink{pit#6}{{\small\textnormal[pit#6]\It{e}}}
%\index{Z\Red{rec}!\Red{pit#6}}\index{Z\Red{utt}!\Red{pit#6.#7}}
}}
\newcommand{\DerivExamWL}[6]{\Tn{
\begin{tabular}[t]{p{\DW}x{5mm}l}\It{#1}&\ARROW&\It{#3}\\\end{tabular}\hfill‘#5’\\
\hspace{1mm}\begin{tabular}[t]{p{\DW}x{5mm}l}#2&&#4\\\end{tabular}\hfill\hyperlink{explExs}{{\small\textnormal[#6]}}
%\index{Z\Red{wl}!\Red{#6}}
}}


%formatting of corpus source information (after \transl in gb4e-environments):
\newcommand{\Corpus}[2]{\hspace*{1pt}\hfill{\small\mbox{\hyperlink{pit#1}{\Tn{[pit#1.#2]}}}}%\index{Z\Red{rec}!\Red{pit#1}}\index{Z\Red{utt}!\Red{pit#1.#2}}
}%
\newcommand{\CorpusE}[2]{\hspace*{1pt}\hfill{\small\mbox{\hyperlink{pit#1}{\Tn{[pit#1.#2]}}\It{e}}}%\index{Z\Red{rec}!\Red{pit#1}}\index{Z\Red{utt}!\Red{pit#1.#2}\Blue{-E}}
}%
%%as above, but necessary for recording names which include an underline because the first variable in \href understands _ but the second variable requires \_
\newcommand{\CorpusLink}[3]{\hspace*{1pt}\hfill{\small\mbox{\hyperlink{pit#1}{\Tn{[pit#2.#3]}}}}%\index{Z\Red{rec}!\Red{pit#2}}\index{Z\Red{utt}!\Red{pit#2.#3}}
}%
%%as above, but for newer recordings which begin with sje20 instead of pit
\newcommand{\CorpusSJE}[2]{\hspace*{1pt}\hfill{\small\mbox{\hyperlink{sje20#1}{\Tn{[sje20#1.#2]}}}}%\index{Z\Red{rec}!\Red{sje20#1}}\index{Z\Red{utt}!\Red{sje20#1.#2}}
}%
\newcommand{\CorpusSJEE}[2]{\hspace*{1pt}\hfill{\small\mbox{\hyperlink{sje20#1}{\Tn{[sje20#1.#2]}}\It{e}}}%\index{Z\Red{rec}!\Red{sje20#1}}\index{Z\Red{utt}!\Red{sje20#1.#2}\Blue{-E}}
}%











%%hyphenation points for line breaks
%%add to TeX file before \begin{document} with:
%%%%hyphenation points for line breaks
%%add to TeX file before \begin{document} with:
%%%%hyphenation points for line breaks
%%add to TeX file before \begin{document} with:
%%\include{hyphenationSDL}
\hyphenation{
ab-es-sive
affri-ca-te
affri-ca-tes
Ahka-javv-re
al-ve-o-lar
com-ple-ments
%check this:
de-cad-es
fri-ca-tive
fri-ca-tives
gemi-nate
gemi-nates
gra-pheme
gra-phemes
ho-mo-pho-nous
ho-mor-ga-nic
mor-pho-syn-tac-tic
or-tho-gra-phic
pho-neme
pho-ne-mes
phra-ses
post-po-si-tion
post-po-si-tion-al
pre-as-pi-ra-te
pre-as-pi-ra-ted
pre-as-pi-ra-tion
seg-ment
un-voiced
wor-king-ver-sion
}
\hyphenation{
ab-es-sive
affri-ca-te
affri-ca-tes
Ahka-javv-re
al-ve-o-lar
com-ple-ments
%check this:
de-cad-es
fri-ca-tive
fri-ca-tives
gemi-nate
gemi-nates
gra-pheme
gra-phemes
ho-mo-pho-nous
ho-mor-ga-nic
mor-pho-syn-tac-tic
or-tho-gra-phic
pho-neme
pho-ne-mes
phra-ses
post-po-si-tion
post-po-si-tion-al
pre-as-pi-ra-te
pre-as-pi-ra-ted
pre-as-pi-ra-tion
seg-ment
un-voiced
wor-king-ver-sion
}
\hyphenation{
ab-es-sive
affri-ca-te
affri-ca-tes
Ahka-javv-re
al-ve-o-lar
com-ple-ments
%check this:
de-cad-es
fri-ca-tive
fri-ca-tives
gemi-nate
gemi-nates
gra-pheme
gra-phemes
ho-mo-pho-nous
ho-mor-ga-nic
mor-pho-syn-tac-tic
or-tho-gra-phic
pho-neme
pho-ne-mes
phra-ses
post-po-si-tion
post-po-si-tion-al
pre-as-pi-ra-te
pre-as-pi-ra-ted
pre-as-pi-ra-tion
seg-ment
un-voiced
wor-king-ver-sion
}\begin{document}

%%%%%%%%%%%%%%%%%%%%%%%%%%%%%%%% ALL THE ABOVE TO BE COMMENTED OUT FOR COMPLETE DOCUMENT! %%%%%%%%%%%


\noindent \Bf{Inventory of revisions to the manuscript \\‘A corpus-based grammar of spoken \PS’} \\by Joshua Wilbur \\for publication with Language Science Press.

\vspace{10pt}
\begin{tabular}{lc}
\It{motivated by:}		&\It{starts on page:}	\\\hline
\It{Reviewer A - review text}	&\korrPG{revA1}\\%\hline
\It{Reviewer A - comments in pdf}	&\korrPG{revA2}\\%\hline
\It{Reviewer B - review text}	&\korrPG{revB}\\%\hline
\It{Reviewer C - review text}	&\korrPG{revC1}\\%\hline
\It{Reviewer C - comments in pdf}	&\korrPG{revC2}\\%\hline
\It{Consulting editors}	&\korrPG{revCEs}\\%\hline
\It{Joshua Wilbur}	&\korrPG{jkw}\\%\hline
\end{tabular}

\vspace{12pt}\noindent 
The suggested revisions are listed on the following pages. Each entry lists the section the revision has been suggested for in the current version (not the original reviewed version) and can be navigated to directly by clicking on the section number (hyperlinks). For suggestions which cannot be directly assigned to a single specific section, the word ‘general’ is written instead. The abbreviation in parentheses indicates the source of the suggestion.\\\noindent
The row ‘\Bf{Suggestion}’ includes the original comment or summary of the comment. Page and section numbers here refer to the version submitted to LSP in 2013 and may deviate from the current version!\\\noindent
The row ‘\Bf{Reaction}’ indicates what revisions have been made, or at least a response to the original suggestion. Page and section numbers here function as hyperlinks to the relevant section in the current version.

\vspace{12pt}\noindent
\It{Note also that revisions based on the \href{http://saami.uni-freiburg.de/psdp/components/documents/PiteSaamiGrammar_Wilbur2013_errata.pdf}{list of errata (click here to download)} %\footnote{Downloadable at \href{http://saami.uni-freiburg.de/psdp/components/documents/PiteSaamiGrammar_Wilbur2013_errata.pdf}{http://saami.uni-freiburg.de/psdp/components/documents/\\PiteSaamiGrammar\_Wilbur2013\_errata.pdf}.} 
have also been made since the manuscript was sent to the reviewers!}

\vfill
\pagebreak

\vspace{4pt}
\newcounter{korrNo}%adds counter
%\addtocounter{korrNo}{1}%sets counter
\newcommand{\KORRno}{\stepcounter{korrNo}\thekorrNo}%

\newcommand{\KORRgen}[2]{\Bf{\KORRno.} \It{general} (\spring): \\*\Bf{Suggestion:} #1\\*\Bf{Reaction:} #2 \\\hline}%command for rows in table below only; syntax: #1=comment, #2=response
\newcommand{\KORR}[3]{\Bf{\KORRno.} Sec. \ref{#1} (\spring): \\*\Bf{Suggestion:}  #2\\*\Bf{Reaction:} #3 \\\hline}%command for rows in table below only; syntax: #1=section ref, #2=comment, #3=response %%
\newcommand{\KORRspec}[4]{\Bf{\KORRno.} #1 (#2): \\*\Bf{Suggestion:}  #3\\*\Bf{Reaction:} #4 \\\hline}%command for rows in table below only; syntax: #1=section ref, #2=source of suggestion, #3=comment, #4=response %%
\newcommand{\noKORR}[3]{\Red{\ARROW\Bf{\KORRno.} Sec. \ref{#1} (\spring):} \\*\Bf{Suggestion:}  #2\\*\Bf{Reaction:} #3 \\\hline}%command for rows in table below only; syntax: #1=section ref, #2=comment, #3=response %%

%%%%%%% von review A:
\newcommand{\spring}{review A}%

\begin{longtable}{p{.97\textwidth}}\hline\endhead\korr{revA1}
\BfIt{\Blue{Suggestions from Reviewer A from review text}}\\*\hline
%%%%%%%
\KORRgen{a point which I would like to see described and explained in more detail…is the question of the degree of language competence of the speakers}{More details added on page \korrPG{034} in Section \ref{sociolinguistics}.}
\KORR{wordStress}{more about ‘the acoustic correlates for stress are intensity and pitch’ (p. 27)}{added comment that more research necessary to really describe this thoroughly (p. \korrPG{001})}
\KORR{syllabification}{not convinced by claim that geminates and preaspirated consonants are split up at syllable boundaries (p. 32-33)}{Relevant paragraphs (p. \korrPG{002}, \korrPG{003}) revised to indicate that division of geminate/preaspirated phonemes into different syllables is \It{symbolic} in nature; the main motivation is to adhere to a syllable template with a singleton in the onset. The fact that syllable boundaries are not relevant in \PS\ prosody is emphasized more.}
\KORR{utteranceIntonation}{concerning utterance intonation downstep: ‘I’m not sure this is a necessary component of a grammatical description of a language’}{this may not be particularly interesting or unique (and perhaps even a ‘universal phenomenon’ as the reviewer suggests), but it is still worth mentioning since it is one aspect of \PS\ prosody. No revision made.}
\KORRgen{concerning structure of the first three sections in chapter \ref{nouns}: first morphologically relevant categories (number, case) are discussed, then the forms (morphology via suffixes, umlaut, stem alternations); this order is not traditional, should be other way around}{This is a tempting revision, however, I have decided not to incorporate it because the current structure first provides a general description of the number/case categories (Sections \ref{numberNouns} and \ref{case}) as a background to ease understanding the complex morphological marking strategies described in Section \ref{NumCaseNouns}. The second paragraph of the chapter’s introduction/outline (p. \korrPG{004}) was revised slightly to make this intention more clear. Chapter \ref{verbs} on verbs has a similar structure; this was not restructured for the same reason, but I made a small revision to the introduction/outline (p. \korrPG{005}).}
\KORRgen{not clear whether consonant gradation and umlaut is dependent on noun class or otherwise; needs clarification}
{significant revisions done to Section \ref{Cgrad} to include when c-grad occurs, including adding attested consonant-gradations to Table \ref{CgradPatternSummary}. Because it is now mentioned in three different places that C-grad and inflectional class are independent variables, this is not mentioned again in the descriptions of the individual inflectional classes (for verbs and nouns). I also revised the intro to Section \ref{nounClasses} (p. \korrPG{006}), but this needs to be reconsidered!! A small revision was made to clarify the situation on verb classes, but this was not as relevant for verb classes since gradation does depend on verb class to some extent (p. \korrPG{007}).}
\KORRgen{problem with including demonstratives as pronouns and as determiners in the same section (in word class ‘pronouns’, p.116 and passim)}{renamed and moved section on dem-as-det to subsection of adjective-chapter (cf. p. \korrPG{009}); updated section on NPs (\ref{nominalPhrases}) to reflect this (i.e., cf. p. \korrPG{008}).}
\KORR{reflexivePronouns}{usage of reflexive pronouns as intensifier, not true reflexive pronouns, gloss as ‘\Sc{refl}’ should be reconsidered (cf. examples (268) and (269).}{Formally, these pronouns are the same as the ‘true’ reflexive pronouns, so I choose to keep the gloss ‘\Sc{refl}’. I now mention that the reflx-pronouns in these examples (p. \korrPG{011}, also examples \Ref{reflPron3} and \Ref{reflPron2}) are used as intensifiers.}
\KORR{verbs}{‘It would be useful to introduce all non-finite verb forms in a separate section, not, as the author does, little by little and by the way, while introducing periphrastic categories. My suggestion would be to first have a section on finite verb forms, then one on non-finite forms, and then a section on periphrastic forms. It’d be much easier to follow that way.’}{Restructuring/revision to the relevant section, which is now Section \ref{nonFiniteVerbforms}. Now the focus is initially on non-finite forms, then goes on to describe periphrastically marked verbal categories.}
\KORR{ADVverbs}{‘The form in –mnin is used as a complement of the copula to express the progressive aspect and as an adjunct. Perhaps the author could consider classifying this form as a converb, as its functions roughly correspond to what converbs do in many Eurasian languages.’}{Certainly a possibility, but the current description (functions as an adverbial) is sufficient for this description, particularly since robust data is lacking. This potential conclusion can be drawn by any typologists reading the material as well.}
\KORR{passiveVinflection}{(p. 157: The argumentation to the effect that the passive is a derivational rather than an inflectional category is not quite clear. …}{Removed ‘main’ from phrase ‘main lexical verbs’ as this was indeed confusing, as well as superfluous. Removed entire clause ‘[the passive marker] does not necessarily occur on the finite verb of a clause (as the examples with perfect passive participles attest)’ because stating that only lexical verbs can be passivized is in fact sufficient to make clear that auxiliary verbs cannot be passivized (p. \korrPG{012}).}
\KORR{POTinflection}{pp. 164-5: I don’t find the argumentation about the status of the potential mood (derivation or inflection) convincing.… …If such evidence is not available in the corpus, then the author should say so, that’s fine.}{The entire section (Section \ref{POTinflection}) points out more than once that there is not sufficient data, but instead attempts to say that it is worth considering that potential mood could be a derivational category synchronically, unlike the traditional analysis. I think this section is written to be obviously preliminary, and the comparison between typically inflectional vs. typically derivational properties quite useful. No revisions made.}
\KORR{grammaticalRelations}{Section 13.1 on Grammatical relations is rather terse. I would suggest the author to discuss other criteria for grammatical relations (agreement, control, control of reflexives, etc.) and introduce other constituents apart from subjects, objects and indirect objects (adjunct and similar).}{I would love to do this, but current time restraints prevent this from happening, must be left for future work.}
\KORR{infoStructure}{pp. 225-6: It would be interesting to know whether there is a fixed order of fronted topics and foci, so that we have to assume two different positions in the sentence template, or their ordering is free}{As with previous comment, I would love to go into detail on this, but current time restraints prevent this from happening, must be left for future work.}
\KORRgen{The format of the references in the text...}{Due to a LaTeX problem, not all citations were typeset correctly in the version sent to the reviewers. This has been/will be corrected for the final version!}
\end{longtable}
%\pagebreak

\begin{longtable}{p{.97\textwidth}}\hline\endhead\korr{revA2}
\BfIt{\Blue{Suggestions from Reviewer A from notes in pdf}}\\*\hline
\KORR{PSandUralic}{p. 1: The figure actually does not show the position of the Saami subbranch within Uralic - it shows the internal structure of the Saami subbranch.}{Changed wording on p. \korrPG{013} to better describe Figure \ref{familyTree}.}
\KORR{onomastics}{p. 2: I don't understand how the above two alternative forms of the autonym show that speakers simply call their language Saami.}{revision made to clarify this on p. \korrPG{014}.}
\KORR{previousWork}{p. 9: change ‘language sociology point of view’ to ‘sociolinguistic point of view’}{revised to ‘point of view of the sociology of language’ on p. \korrPG{015}.}
\KORR{orthography}{p. 17: This is rather clumsly - how about smth along these lines "The affricate phonemes are represented with digraphs <ts> and <tj>”?}{revised as suggested, p. \korrPG{016}. Similar revision made on p. \korrPG{017} concerning spelling geminates.}
\KORR{c3}{p. 29: Is it really 'finUs'? The Latin word for 'end' is 'finis', that'd sound more logical to me.}{revision made accordingly, p. \korrPG{018}.}
\KORR{Plosives}{p. 41, Figure 3.1: This is graphically unfortunate, as the additional sign on p, t, k is covered by the square brackets. Would it be possible to add a space between the character and the closing bracket?}{The font used by the LSP template does a better job the the font used in the pdf sent to the reviewers (Charis SIL), but I have also added a small space (\TeX\ command: \BS,) after the unreleased symbol; \ref{PlosivePhonemes} on p. \pageref{PlosivePhonemes}.}
\KORR{morphophonology}{p. 70: This is so vague that it makes no sense. What do you mean by "a behaviour of a derivational process”?}{changed to ‘in derivation’ (p. \korrPG{019}).}
\KORR{Cgrad}{p. 72, Table 4.2: this row [3rd row] is superfluous (already contained in the previous row)}{Indeed, it seems superfluous, but patterns in the table represent the entire consonant center, and thus there is a difference between the pattern in the 2nd row and the 3rd row. This table (Table \ref{CgradPatternSummary}) has been significantly revised anyway due to other suggestions, but an explanatory note has been added to help clarify this on page \korrPG{020}, and the caption for Tabel \ref{CgradPatternSummary} has been revised to ‘Consonant center gradation patterns’.}
\KORR{introWordForms}{p. 78, Table 5.2: What is VC? Verbal Clause? Please resolve the abbreviation, it is not self-explanatory.}{All abbreviations in Table \ref{wordClassSummary1} now written out.}
\KORR{numberNouns}{p. 80: I think the correct term here would be 'generic’}{revised as recommended, p. \korrPG{021}.}
\KORR{nominative}{p. 81: Isn't this just a special case of the subject of an (intransitive) existential clause? Why emphasise this subtype in particular?}{This is worth mentioning: 1) particularly for Saamic studies, because this is the expected type, but contrasts with the more common type (due to language contact with North Germanic) using a ‘have’-construction discussed in the referenced section on copular clauses; 2) any reader expecting a ‘have’-construction should be aware of this type of structure, too. No revisions made.}
\KORR{nonlinearNounMorph}{p. 94: As above - this row is basically the same as the previous, perhaps it's not necessary to have it at all.}{Table \ref{CGpatterns} has been altered to reflect changes made to Table \ref{CgradPatternSummary}, and, as above, the caption has been altered to stress the fact that patterns in the entire consonant center are relevant here.}
\KORR{possSuffixes}{p. 111: A predicate missing here}{Revised on p. \korrPG{022}.}
\KORR{personalPronouns}{p. 114: This sounds cumbersome - how about 'roots’?}{Point taken, but I prefer this more cumbersome phrase. no revision made.}
\KORR{personalPronouns}{p. 114: It would be good to provide a short overview of these regularities.}{Similarities now listed starting p. \korrPG{023}.}
\KORR{demonstratives}{p. 116-117: Wouldn't it be more elegant (and easier to parse) if you'd say 'agree with the noun they modify in number and case’?}{Revised accordingly (p. \korrPG{024}).}
\KORR{compSuperlADJs}{p. 135: I am not sure I would use the term 'syncretic' here, even though the intended meaning is clear, since this word implies the existence of a previous diachronic state in which two forms were different, and we do not see that this was the case in Pite Saami. Why not simply say 'identical’?}{revised accordingly (p. \korrPG{025}).}
\KORR{compSuperlADJs}{p. 136, Table 8.6: why is this in italics?}{italics removed in Table \ref{compSuperlADJsTable}.}
\KORR{caseNumberOnNum}{p. 147: A thoght: Is it possible that 'one' is sometimes used as a kind of specific indefinite article, as in many languages of the world, and therefore has some properties of a determiner (i.e. inflection in attributive position, like demonstratives)?}{absolutely, but strange that it only happens in accusative case, not in other cases. No revision made.}
\KORR{inflectionalCatsVerbs}{p. 149: Predicate missing here - "and are inflected for tense..." is what is meant, right?}{revised accordingly (p. \korrPG{026}).}
\KORR{tense}{p. 151: 'interpretation' would be a more correct term here.}{revised accordingly (p. \korrPG{027}).}
\KORR{IMPmood}{p. 152: revise sentence}{revised (p. \korrPG{028}).}
\KORR{inflectionalSuffVerbs}{p. 158: I would suggest giving the full list of all suffixes here, regardless of their frequency. Full lists of forms are extremely useful for readers.}{Revised section \ref{inflectionalSuffVerbs} on page \korrPG{076}, and added all suffixes to Table \ref{verbSuffixes}. Revisions to following section on syncretism \ref{verbalSuffixesSyncretism} made, too.}
\KORR{verbSuffixes}{p. 158, Table 9.1: Delete 'other' (it wrongly implies that the previous forms are also non-finite).}{revised accordingly here (Table \ref{verbSuffixes}) and in other similar tables throughout Chapter \ref{verbs}.}
\KORR{verbComplex}{p. 208: I am not sure this is correct: imperatives do not have expressed subjects, but there is an agent for each imperative (if there were not potential agents, it wouldn't make much sense to express a request), and this agent has different subject properties, the agreement being one of them (see Keenan's 1976 paper on prototypical properties of subjects).}{Point taken. To avoid implying such theoretical assumptions, the theoretically controversial (and for this context irrelevant) assertion in this sentence has been deleted. (p. \korrPG{029})}
\KORR{constituentOrderClauses}{p. 224: 'Presupposed' is not the correct term here (only propositional entities can be presupposed, not referents) - I'd suggest 'contextually given’.}{revised to ‘information provided by context alone’ (p. \korrPG{030})}
\KORR{basicClauses}{p. 227: contextually given}{revised to ‘information provided by context alone’ (p. \korrPG{031})}
\KORR{existentialClauses}{p. 230: This is in contradiction to what is said on p. 225 (there are no clauses with postverbal subjects).}{I can’t find anything around p. 225 (Section 13.2 (currently \ref{constituentOrderClauses}), I assume?), but maybe the reviewer is referring to the claim on p. 225: ‘However, there are no examples in the corpus of verb-initial clauses featuring an overt subject and a VC with one verb.’ This has been revised to clarify this: ‘…featuring BOTH an overt subject and a VC with A SINGLE verb.’ (p. \korrPG{032})}
\KORR{copulaClauses}{p. 232: I am not sure that it is syntactically correct to label the temporal expression 'the complement of the copula" - according to most accounts, this is rather a (temporal) adjunct. The clause is, as the author says, an existential clause, and it is introduced by a temporal adjunct.}{revision made accordingly (p. \korrPG{033}). Footnote also added in Section \ref{existentialClauses} on existential clauses to refer to this construction.}
\KORR{potClauses}{p. 241: I am not sure that this is relevant for the description of the syntax of potential clauses. The impression is that potentials usually have 1st or 2nd person subjects, which is why they can often be left unexpressed. This is a fact about the pragmatics of these sentences and is not their inherent syntactic feature.}{It may be pragmatically motivated, but still relevant for a syntactic description of potential clauses. No revision made.}
\KORR{clausalSubordination}{p. 245, Table 15.1: either "non-finite" or "infinitive" (the latter if only infinitives occur as complements, not other non-finite forms)}{revised to ‘infinitive’ (Table \ref{complementClauseSummary})}
\KORR{infiniteComplementClauses}{p. 247, title of Section 15.2.1.2: "non-finite" or "infinitive”}{revised to ‘infintive’ (Section \ref{infiniteComplementClauses}).}
\end{longtable}
%%%%
%%%%
\pagebreak



%%%%%%%% von review B:
\renewcommand{\spring}{reviewer B}%

\begin{longtable}{p{.97\textwidth}}\hline\endhead\korr{revB}
\BfIt{\Blue{Suggestions from Reviewer B from review text}}\\*\hline
%%%%%%%
\noKORR{}{…only a sketchy description of syntax and only some details of the semantics of grams. But this is not, of course, and obstacle for publishing the book; maybe the author may think of making the title more precise.}{yes, thinking about a different title… … perhaps simply ‘A grammar of \PS’? (even if that isn’t really a solution to this comment)}
\KORRgen{indicate explicitly that the relevant forms are attested only through elicitation}{in fact, this is usually indicated in the text, and examples from elicitation are marked as such, as well.}
\KORRgen{It would be also fine to distinguish throughout the description between phenomena specific for Pite Saami, phenomena that Pite Saami shares with other Saami languages, and phenomena that Pite Saami shares with all the Uralic languages.}{this could quickly get out of hand, and is only mentioned when exceptionally relevant in the text.}
\KORRgen{Maybe it would be good to structure them [chapters] into more big units, e.g. parts consisting of several chapters.}{I considered this at the beginning of the project long ago, but was just never completely satisfied with any of the different ways of combining chapters.}
\KORRgen{Chapters of just 3-5 pages (chapters 5 and 13) are not a perfect compositional solution, maybe these chapters are to be included somewhere as sections.}{Indeed, particularly Ch. 5 (now Ch. \ref{introWordForms}). I’m satisfied with Ch. 13 (now Ch. \ref{overviewSyntax}) as its own chapter.}
\KORR{morphology}{Titles of some chapters are not very successful: ch. 4 is called ‘Morphology’, but all the chapters 4-11 are on morphology}{Changed chapter title to ‘Overview of \PS\ Morphology’; also changed section \ref{linearMorphology} from ‘Overview of linear morphology’ to ‘Linear morphology’ and \ref{morphophonology} from ‘Overview of non-linear morphology (morphophonology)’ to ‘Non-linear morphology (morphophonology)’.}
\KORR{introWordForms}{the word ‘overview’ is redundant in the title of the ch. 5}{Former ch. 5 has now been merged with former ch. 4 on morphology, and is now Section \ref{introWordForms} in new Chapter \ref{morphWordClassCh} ‘Morphological patterns and word classes’, but the title has not been changed because this really is just an overview, as the individual word classes comprise the following five chapters.}
\KORRgen{ch. 6 and 7 would look better if called simply ‘Nouns’ and ‘Pronouns’}{Yes, but I classify both nouns and pronouns as ‘nominals’ because of their shared syntactic and inflectional criteria (cf. overviews of word classes in Tables \ref{wordClassList} and \ref{wordClassSummary1})}
\KORR{basicClauses}{ch. 14 could be called ‘Clause structure’}{yes, but it does not deal with complex clauses; no revision made.}
\KORR{complexClauses}{ch. 15 cannot be called ‘Complex clauses’, since it is devoted to clause combining – i.e. structures consisting of several clauses}{but that is exactly why they are ‘complex’. no revision made}
\KORRgen{there is no section describing systematically the morphology of non- finite forms and overviewing their use, although the author mentions non-finite forms speaking of periphrastically marked categories (9.2) and multiverb clauses (14.1.5).}{Section \ref{nonFiniteVerbforms} revised, restructured to focus on non-finite forms, including summary/example tables for non-finite forms}
\KORR{complexClauses}{in ch.15 it is necessary to emphasize explicitly that non-finite forms cannot be used in any structures of clausal subordination (or, if this is not true, it is necessary to provide this information).}{The use of infinitive forms in subordinate clauses is already discussed in Section \ref{complementClauses} on complement clauses. Subsection \ref{otherSubclauses} on other non-finite forms which could be expected based on neighboring Saami languages, but not attested at all in the corpus, has been added to section (\ref{clausalSubordination}) on page \pageref{otherSubclauses}.} %\korrPG{073}.}
\KORR{verbs}{Especially, the morphology of Connegative is to be discussed – why is it glossed with a slash, is it the same as the stem of the verb or not? }{Due to restructuring of Section \ref{nonFiniteVerbforms} to focus on non-finite forms, this becomes more clear. Additionally, a note explaining its morphological form was added to section \ref{negationVerb} on page \korrPG{075}.}
\KORR{inflectionalSuffVerbs}{On p. 158 the author says that “a number of other non-finite forms also exist...” – it is necessary to list them completely and say what is known on their formation.}{restructuring of Section \ref{nonFiniteVerbforms} to focus on non-finite forms, and address other non-finite forms which could be expected but are not attested in the corpus. This somewhat mis-leading sentence has now been removed, as well, from Section \ref{inflectionalSuffVerbs}, as it is no longer relevant, due to the revisions/restructuring mentioned.}
\KORRgen{It is also important that, in contrast to the separate chapter (14) devoted to the clause structure, there is no chapter devoted to NP structure. This information is provided in different chapters on nominal morphology and in the chapter 12, but is to be assembled in one place.}{Not sure I understand this point. The book is structured to deal with morphology of the various word classes first, then the phrase types, thus there is a section (\ref{nominalPhrases} – but not a chapter) on NP structure. No revision made.}
\KORRgen{Sections 6.1 [Number in nouns] and 6.3 [Number and case marking on nouns] are to be unified (I am not sure if they should precede 6.2 [The nominal case system] or follow it).}{I disagree. The point of this structure is to first posit the relevant inflectional categories (number in \ref{numberNouns}, case in \ref{case}), then discuss how these are marked (Section \ref{NumCaseNouns}), particularly since they cannot always be linearly separated from each other. No revision made.}
\KORR{}{GENERAL: Ordinal numerals syntactically are attributive adjectives (this is stated explicitly in 11.3.2), and therefore, their description should completely go to 11.3.2 from 8.8.1.}{This actually is not stated explicitly in 11.3.2 (now Section \ref{ordNUMderiv}), and even if it were, I don’t understand the argument for moving it to the derivation section for that reason. However, I do understand that it would be more consistent to have ordinal numerals discussed in the chapter on derivation, since they are derived, after all. Despite that, I choose to leave them in the numeral section (Section \ref{numerals}) in the Adjectivals chapter so that a comparison between cardinal and ordinal numerals is easier, particularly using Table \ref{basicNumsTable}. Ordinals now have their own subsection here (section \ref{ordinalNums}), to set them apart from cardinals and emphasize that they are derived. The short subsection (\ref{ordNUMderiv}) in the chapter on derivation is still there, with a summary of ordinal derivation and a link to section \ref{ordinalNums} and Table \ref{basicNumsTable}. Furthermore, small revisions made in writing the derivational suffix consistently in sections \ref{basicNums} on page \korrPG{076} and \ref{ordNUMderiv}.}
\KORR{inflectionalCatsVerbs}{In the ch. 9 one would expect first the discussion of specifically verbal (TAM) categories and only then the discussion of person and number.}{That is certainly an alternative, but I’ve structured Ch. \ref{verbs} on verbs to present inflectional categories first, and I think it is just as reasonable to present person/number first. Aspect, which is marked periphrastically, is discussed later in the relevant section (\ref{aspect}). No revision made.}
\KORRgen{The section 9.3 [passive voice] should be joint with the section 11.2.5 [Passivization with the derivational suffix \It{-duvv}] and excluded from the chapter 9.}{I agree that this structure is a bit awkward because is splits various aspects of passives across several chapters. But the idea is that inflectional morphological aspects are discussed in Section \ref{passiveVinflection} in the chapter on verbal morphology, and how to derive passives using \It{-duvv} as one example of derivation in section \ref{VdervPassives} in the chapter derivational morphology. Syntactic aspects are discussed in section \ref{passiveVoice} in the chapter on basic clauses. Each section includes references to the other sections, and the index in the published version will also make finding information on passives easier. No revisions made.}
\KORRgen{Section 9.2.4 to a significant extent repeats what has been discussed in ch.4 [Morphology].}{There is no Section 9.2.4! If the reviewer means 9.4.2 (now Section \ref{nonLinearMorphVerbs} on non-linear morphology in verbs), then, s/he is somewhat correct, but the point of this section is to elaborate specifically on verbal non-linear morphology. No revision made.}
\KORRgen{Maybe the chapter 13 (general overview of the syntax) should precede the chapter 12 (phrase types).}{Certainly possible, but the idea is to present the building blocks first (here, phrase types), then the structures these fit into (clauses). No revisions made.}
\KORRgen{General word order should be described before the constituents.}{It’s not entirely clear which sections/chapters the reviewer is referring to. In general, there is no syntactically determined ‘general’ word order. This is all pointed out in Section \ref{constituentOrderClauses} on constituent ordering. No revisions made.}
\KORRgen{The sections 14.1.1 and 14.1.2 significantly repeat the section 13.1.}{Yes, but only partially and from different aspects: section 13.1 (now \ref{grammaticalRelations}) discusses grammatical relations in general, while sections 14.1.1 (now \ref{basicIntransDeclaratives}) and 14.1.2 (now \ref{basicMonotransDeclaratives}) present very basic intransitive and transitive clauses; these latter examples, in their simplicity, are also good examples for grammatical relations, thus the seeming doubled information. No revision made.}
\KORRgen{Presumably, there should be a Conclusion in the book.}{Not really. The book presents \PS\ structures, that’s the point, which is made in the introduction. There is no theoretical point that can/should be summarized. No revision made.}
\KORRgen{The manuscript currently lacks Bibliography! Of course this is to be fixed before publishing it.}{Of course. Unfortunately, there was a LaTeX problem in the version sent to the reviewers that deleted the bibliography.}
\KORRgen{Maybe, the section 6.4 [Inflectional classes for nouns] should be the first in the ch. 6.}{This section (now \ref{nounClasses}) builds on the three sections before it. No revision made.}
\KORR{PSandUralic}{p.1 Say a couple of words re traditional classification of Saami as a single language.}{Not sure what the point of such a comment would be. The current/modern classification is clear and more relevant. I also point out that it is a dialect continuum. No revision made.}
\KORR{geography}{p.3 Re Pite Saami linguistic area it would be possible rely not only on the places where the speakers com from, but also on the traditional ethnic knowledge? Did you ask the speakers about the place their ancestors lived?}{Not sure what the suggestion is. The families and their ancestors are generally from that area. No revisions made.}
\KORR{sociolinguistics}{p.6-7 It is necessary to mention if nowadays the speakers of Pite Saami form a community (or are isolated from one another). Also, the author is expected to provide the information of the use of the language nowadays – is it spoken and in what situations.}{Added some content to the relevant section on page \korrPG{034}.}
\KORR{PSDPcorpus}{p.9 It would be nice to mention another members of the documentation project if any.}{just me.}
\KORR{collectionMethods}{p.11 It would be necessary to indicate explicitly how many hours of recordings (among the mentioned 36 hours) are naturalistic speech of different genres, and how many elicitation is.}{Added to page \korrPG{035}.}
\KORR{collectionMethods}{p.12 ‘Transcriptions are written in various versions of the Pite Saami orthography’. The question arises if the orthography is unified in the book (and if not, it is necessary to unify).}{Standardized when cited in the book based on orthography described in Section \ref{orthography}. This is now mentioned on p. \korrPG{035}.}
\KORR{ProsodicStructure}{p.25 Excuses for the composition of the paper are not necessary :)}{Ok, slight rewording of very beginning of Ch. \ref{ProsodicStructure}.}
\KORRgen{p. 25-26 sect. 2.1-2.2. I am not sure that the word structure in terms of foots and syllables is a part of the domain of prosody.}{I disagree. For instance, feet and syllables are parts of the prosodic hierarchy. No revision made.}
\KORR{multisyllabicWords}{p. 26 et passim, sect. 2.2 Defining the notion of foot and describing its structure should be a separate section.}{I disagree. no revision made.}
\KORR{multisyllabicWords}{p.26, sect. 2.2 Examples of words of the discussed structures are necessary.}{These come later in Section \ref{exampleFootedness} ‘Discussion and examples’.}
\KORR{multisyllabicWords}{p.26 ‘Evidence for the foot ... can be found ...’ – please provide the sections where this is discussed in detail.}{Section references added on page \korrPG{037}.}
\KORR{wordStress}{p. 27 Examples of words with different stress patterns are necessary.}{Added, see table \ref{syllTempExs} and text on p. \korrPG{038}.}
\KORR{wordStress}{p. 27, para ‘The acoustic correlates’ – how do acoustic correlates differ for primary and secondary stress.}{not sure, short note added on p. \korrPG{039}.}
\KORR{prosodicDomains}{p. 28, p.30 – how do Figure 2.2 and Table 2.2 correlate? What is the difference between core and foot?}{I think this is clear from the discussion, particularly in Section \ref{footedness}. no revision made.}
\KORR{syllabification}{p. 31, para ‘The possible syllabification patterns’ – logically possible or possible in Pite Saami?}{Changed caption to start ‘Logically possible…’ in Table \ref{syllabificationPatternsInCCent}.}
\KORR{syllabification}{p. 33 The sentence ‘’Nonetheless, the actual position of syllable boundaries...’ should go the beginning of 2.2.3}{It is sufficient to have this at the end. Syllabification is still important, and this is not the most important thing. no revision made.}
\KORR{footedness}{p.33, sect. 2.2.4 Is there then any reason at all to use the notion of syllable in the description?}{Yes, they also bear word stress! Note added to Section \ref{footedness} on page \korrPG{040}.}
\KORR{utteranceIntonation}{p.34 sect. 2.3.1 Now this section looks like a discussion of a single accidental example, maybe it would be good to rewrite it showing that you speak of the general patterns.}{Small revision made on page \korrPG{041}.}
\KORR{utteranceFinalDevoicing}{p. 35 ‘can be weakened to mark the end of an utterance’ It would be nice to say smth, on conditions of such weakening or on the place of such utterances in the discourse structure. Or, if so, say explicitly that this is not clear.}{added note to end of section \ref{utteranceFinalDevoicing} on page \korrPG{042}.}
\KORR{csANDvs}{p. 37 Speaking of ‘native’ phonemes please say at once what are non-native phonemes and how many of them are there.}{A number of new loans, perhaps nonce loans, from Swedish have potentially introduced new phonemes, but there isn’t enough data, plus their existence is potentially very interim. Solution: ‘native’ dropped from this context. (p. \korrPG{043}); also in Section \ref{typologicalProfile} on page \korrPG{079} (typological profile).}
\KORR{geminateCs}{p. 39, sect. 3.1.1.2 It would be fine to provide any arguments why the geminates are separate phonemes and not clusters of two identical phonemes.}{I think this section does a good job arguing this as it is. No revision made.}
\KORR{geminateCs}{p. 39 et passim Since geminates are restricted to the consonant center, it is necessary to provide minimal pairs showing non-geminates in the same context.}{There are plenty of examples throughout the book, particularly when consonant gradation is discussed. A reference to the section on consonant gradation was added here on p. \korrPG{044}.}
\KORR{vowelPhonemes}{p.60 It would be fine to provide any arguments why the diphthong is a separate phoneme and not a cluster of two vowel phonemes.}{I don’t think this is necessary. no revision made.}
\KORR{Vallophones}{p. 61, Table 3.9 Is it possible to analyze this distribution as vowel reduction, V1 being the most distinctive position and V3 being the least distinctive one?}{Probably. No revision made.}
\KORR{epentheticSchwa}{p.66, sect. 3.2.2 Are there (lexical or phonological) contexts where the epenthetic schwa is obligatory? (If so, maybe it is not so clear that it has no phonemic status).}{no general lexical or phonological contexts where it is required, plus it varies between dialects, but more data is needed. I think this is clear in the text as is. no revision made.}
\KORR{Cgrad}{p. 72 The Table 4.2 is to be exemplified with specific consonants.}{Done, plus more revisions to this entire section (Table \ref{CgradPatternSummary}, page \pageref{CgradPatternSummary}).}
\KORR{introWordForms}{p. 78 et passim It is not convincing why adpositions form a common word class. Based on syntactic criteria, prepositions and postpositions are separate word classes.}{Table \ref{wordClassSummary1} changed to indicate that adpositions head an adpositional phrase (not a postpositional phrase). Small change on page \korrPG{045} in Section \ref{adpositions} on adpositions to emphasize that prepositions are infrequent and can also be used as postpositions. They are similar enough to warrant considering them one wordclass with two subtypes.}
\KORR{nouns}{p. 79 et passim In the context of general linguistics the term ‘class marker’ is misleading, since it can be associated with nominal classification systems.}{True, but the relevant section makes it clear that these are inflectional class markers. ‘inflection’ added to a few places to make this more clear, but not in Figure \ref{nounStructure} for space reasons.}
\KORR{case}{p. 82-87 The labels used for the cases are not self-evident. Presumably, the author follows the terms established in Finnic linguistics, but maybe it would be better to use smth. more common. Illative – why not Allative (or even Dative)? Illative – why not Locative? Elative – why not Ablative? Abessive – why not Caritive? Essive is a very infelicitous label in contexts of the semantics of this case form; maybe, Translative?}{Footnote added stating that the Finno-Ugristic standard names have been chosen. (p. \korrPG{046})}
\KORR{personalPronouns}{p. 113 It is necessary to say more explicitly about the contexts where the two nominative forms, monosyllabic and bisyllabic, of personal pronouns are used. Are these interchangeable in all contexts alternates or there are contexts where only one form is possible?}{The text is explicit enough as it is, I think. No revision made.}
\KORR{adjectivesIntro}{p. 127 The information provided in the Table 8.1 poses the question if it is appropriate to consider attributive and predicative adjectives the same part of speech, since their syntactic contexts are so different.}{True, but they both head APs, that’s the main criterium, as pointed out in the first sentence. They are considered sub-categories because of these differences. No revisions made.}
\KORR{predADJ}{p.131 It would seem reasonable to me to test the ‘predicative adjectives’ on their (in)ability to be a head of an NP (e.g. to have case forms).}{The predicative adjectives only occur in what might resemble nominative case, so such a test would not be very meaningful. No revision made.}
\KORR{compSuperlADJs}{p.136 It would be better to provide the full sentence in the example (294).}{Example (now \ref{compATTRADJex1} on page \pageref{compATTRADJex1}) is from an elicitation session, as marked in the text, and wasn’t in a full sentence. No revision made.}
\KORR{quantifiers}{p. 140 The footnote says that the attributive form of the quantifiers is provided, but it would be better to provide both forms for a so important class of adjectives.}{Removed this footnote because it was misleading as not all examples listed have predicative forms, nor is there sufficient data in the corpus to elaborate.}
\KORR{caseNumberOnNum}{p. 146 The example (326) seems to be unrelevant, since the ordinal numeral is an attributive adjective.}{The ordinal numeral is much like an attributive adjective, even in this respect, but this example (\ref{numeralEx2} on page \pageref{numeralEx2}) still illustrates the behavior of numerals concerning inflectional morphology. No revision made.}
\KORR{caseNumberOnNum}{p. 146-147 It would be fine to provide some evidence that the morphosyntax of the attributive adjectives is (or is not) different.}{The first sentence in this section (\ref{caseNumberOnNum}) clearly describes the morphosyntax of numerals, even contrasting them to adjectives in elliptic constructions and predicative position. This is sufficient for anyone interesting in making the comparison. No revision made.}
\KORR{personNumberVerbs}{p. 150 Is it possible to use finite non-imperative verbal forms with no overt subject? In other words: are verbal person-number markers anaphoric markers on their own or they are only agreement markers that follow an overt pronominal subject? What about 1-2 pers.?}{Sentence added on p. \korrPG{047} indicating that inflection is present even if the subject is not overt.}
\KORR{tense}{p. 151 The fact that the present tense can express planned actions in future does not make it ‘non-past’, since it is very typical for present tense forms.
Anyway, how are another future meanings (e.g. predictive) expressed in Pite Saami?}{Not sure I follow the argument in the first sentence. As for other future meanings, such as predictive, more study would be required. I added a section on using \It{gallgat} ‘will’ for periphrastic future tense (Section \ref{futureTense}).}
\KORR{tense}{p. 151 It would be good to provide at least one example on the Past tense form.}{There is already one provided (in the previous section, ex \ref{inflectionalCatsVerbsEx2} on page \pageref{inflectionalCatsVerbsEx2}). No revision made.}
\KORR{aspect}{p. 153-154 The section 9.2.1 is to be divided into separte subsections on Perfect and on Progressive.}{Done. Now sections \ref{perfectAspect} and \ref{progressiveAspect}.}
\KORR{aspect}{p. 154 It would be good to give an example on present Progressive.}{There is already an example provided and discussed (ex. \ref{perfectEx1}).}
\KORR{inflectionalSuffVerbs}{p. 158, Table 9.1 ‘other non-finite forms’ – why other?}{‘other’ removed here (Table \ref{verbSuffixes}) and in similar tables in this chapter.}
\KORR{inflectionalSuffVerbs}{p. 158, Table 9.1 – how the 2nd person affixes listed with the slash are distributed?}{This is already discussed in the text above the table. no revision made.}
\KORR{POTinflection}{p. 163-164, sect. 9.4.3 The discussion seems very strange to me. The fact of expression the Potential with an affix, not with a portemanteau is not an argument for classifying it as a derivation. On contrary, the complementary distribution with other TAM forms is enough for classifying it as inflection. Additionally, in Uralic context I would also look on the behavior of this form with the negation: inflectional morphemes are usually added to the negative verb, and derivational ones are added to the lexical verbs.}{The discussion in Section \ref{POTinflection} on the possibility of rethinking POT as derivation is just conjecture; at the beginning of the section, it is clearly stated that it is considered inflection in this book. Nonetheless, I think it is worth keeping this to illustrate that another analysis may be more appropriate. Indeed, the test to see if it is possible to have on the negation verb would be a good one; unfortunately, the corpus data doesn’t have this. No revisions made.}
\KORR{verbInflectionalClasses}{p. 166, sect. 9.5, the bottom It would be fine to provide examples for every criterion of the four listed.}{These examples are provided in the relevant sections on the inflectional classes. This is just the introduction. No revisions made.}
\KORR{QpartWordform}{p. 184, sect. 10.1.2.1 It seems strange that the question marker is the only attested particle. There are no other particles? Is it sure that syntactically particles are a subclass of the adverbs?}{The term ‘particle’ is quoted from the literature, but I avoid this term because I don’t think it is clear. Yes, it is unique. No, it is not clear what its status is, as is already quite evident in the text as it is. No revisions made.}
\KORR{postpositions}{p. 186-187, sect. 10.2 It is necessary to say smth. about how the adpositions are combined with pronouns. e.g. are the case forms the same?}{Not necessary. the formulation ‘postpositions are complemented by NPs in the genitive case’ makes this clear. But I’ve added an example with a pronoun anyway (ex. \Ref{postpositionEx2} on page \pageref{postpositionEx2}).}
\KORR{derivMorph}{p. 193 et passim, p. 206 The author says he provides “an overview of some of the more common derivational morphemes”. Presumably, it would be more useful to list all the attested affixes, not only the most common, leaving, of course, a more detailed description for the most common ones.}{Good idea, but unfortunately there is not even close enough data in the corpus to do much beyond what I have already done as it is, plus time restraints prevent this. Furthermore, \cite{Ruong1943} goes into much detail on verbal derivations; this is mentioned several times in the text. No revisions made.}
\KORR{nmlz1}{p. 197, sect. 11.1.3 For action nominalizations it would be good to show that morphologically and syntactically they are full-fledged nouns, since typologically it is not self-evident.}{Short explanation and example added on page \korrPG{048}.}
\KORR{verbDIM}{p. 200-201 Presumably, the verbal diminutive suffix is the same suffix as the verbal one. Therefore it is necessary to make it clear explicitly.}{Not sure I understand, the reviewer probably means ‘as the nominal one’; historically, this is certainly the case, and the similarity is pointed out in the footnote at the beginning of Section \ref{verbDIM}. I think that is sufficient. No revision made.}
\KORR{verbComplex}{p. 207, 208 et passim It is at least not self-evident that VC (‘Verb Complex’) is a constituent. Does this mean that there is no Verb Phrase in Pite Saami? Otherwise, an auxiliary should form a constituent not only with the non-finite form, but also with the direct and indirect objects (and some adverbials).}{The point of the VC is to encompass all the verbs but not any arguments. The idea of a VP including arguments is not necessary to sufficiently describe \PS\ syntax. The verbs together form a syntactic constituent, even if it is not a single linear constituent. I think the description of VC at the beginning of Section \ref{verbComplex} makes it sufficiently clear that a VC is enough for this description. No revision made.}
\KORR{nominalPhrases}{p. 211, Figure 12.1 First, it would be good not to say ‘other modifiers’, but to distinguish at least genitives, numerals, quantifiers, other attributive adjectives. Second, it is not clear what ‘reflexive’ is here. Third, it is necessary to show in this figure the position of the relative clauses.}{1. the point of Figure \ref{NPstructures} is to provide a summary, thus I will stick to ‘other modifiers’ here; which modifiers are intended is listed in the text concerning this figure. 2. In fact, it’s not a true reflexive, but an intensifier in the form of a reflexive pronoun. In Figure \ref{NPstructures}, this has been changed to ‘refl-intensifier’ and a sentence added to the initial paragraph on p. \korrPG{049}; one example is already included at the end of the section. 3. Relative clauses now included in Figure \ref{NPstructures}, plus an example (no. \ref{NPstructureRelClEx1}) was added to the end of the section on p. \korrPG{050}, with a reference to the section on relative clauses has been added on p. \korrPG{049}.}
\KORR{adjectivalPhrases}{p. 215, the bottom It is not clear if the distinction of the three types of adjectives heading the adjectival phrases is relevant for the structure of the adjectival phrase.}{But the following paragraph makes this clear, along with Figure \ref{APstructures}. No revision made.}
\KORR{copulaClauses}{p. 232-233The information on the verb ‘have’ should not be in the section ‘Copula clauses’ anyway.}{This short paragraph plus example are simply an aside, intended mostly for Uralists who may not expect this. Footnote added on page \korrPG{051}.}
\KORR{modalVs}{p. 233 et passim, sect. 14.1.5.1 It is necessary to provide any arguments that the modal verbs and their complements form a single clause, not a multi-clausal structure.}{The modal verbs require a lexical infinitive verb, not a subordinate/complement clause with a finite verb. Furthermore, it is beyond the scope of this book to provide a theoretical framework about these structures. No revisions made.}
\KORR{auxV}{p. 235 The title of the section 14.1.5.2 is not correct, since it is not the auxiliary that marks the aspect, it is the non-finite form.}{In fact, it is this auxiliary PLUS the non-finite that mark aspect, as pointed out at the beginning of this section (\ref{auxV}). This is clear from this section, and from the introduction to Section \ref{multiVdeclarativeClauses}. I have altered the title slightly to ‘The aspectual auxiliary verb \It{årrot}’.}
\KORRgen{The way the author mentions the references in bold and without spaces (e.g. ‘ValijarviWilbur2011’ instead of standard ‘(Valijarvi \& Wilbur2011)’) is not neither usual nor felicitous. Additionally, it is especially strange that some references are cited with initial lowercase letters (e.g. ‘sjaggo2010a’). I would strongly recommend to switch to the standard way.}{\LaTeX\ problem in the version sent to reviewers; this has been fixed.}
\KORR{preaspiration}{p. 38 Footnote 1 on the term should go to the title of 3.1.1.1, not to the word on the first line of the section.}{\LaTeX\ doesn’t like this, not worth trying to figure it out, not sure it’s really necessary anyway.}
\KORR{Cgrad}{p. 71 ‘The term consonant gradation is commonly used’ – please provide references.}{Revisions made to main text and footnote on page \korrPG{052}.}
\KORR{introWordForms}{p. 77 It is necessary to explain what ‘Ch./Sec.’ is.}{This is already mentioned in the caption for Table \ref{wordClassList}. No revision made.}
\KORR{NclassIIIsummary}{p. 107 The sentence ‘One language consultant ...’ would be more appropriate in a footnote.}{Incorporated into the footnote already there on page \korrPG{053}.}
\KORR{adjectivesIntro}{The Table 8.4 follows the Table 8.2, there is no Table 8.3.}{\LaTeX\ problem using ‘longtable’-package for tables not marked as tables in text, but that affect the numbering. Fixed by decreasing counter ‘table’ by one after each longtable-environment like this: \BS addtocounter\{table\}\{-1\}}
\KORR{agentNMLZ}{p. 198, example (410): if the verb means ‘learn’, presumably the agent noun should be ‘learner’, and if the agent non means ‘teacher’, presumably the verb should mean ‘teach’.}{This is explained in the text in this section, specifically on page \korrPG{054}. Small revision made to this text anyway to make it clearer.}
\KORR{vDerivation}{p. 199, the bottom The full title of the reference is not necessary in the text.}{I disagree. For readers who do not understand German, it is useful to know the title of this very relevant work. No revision made.}

\end{longtable}
%%%%
%%%%
\pagebreak

%%%%%%%% von review C:
\renewcommand{\spring}{reviewer C}%

\begin{longtable}{p{.97\textwidth}}\hline\endhead\korr{revC1}
\BfIt{\Blue{Suggestions from Reviewer C from review text}}\\*\hline
\KORR{introChapter}{Intro – background on language, somewhat condescending treatment of previous literature, but no discussion of what a corpus-based grammar is, or why it is important to do research in that way.}{Revisions made to section on previous literature to be hopefully less condescending (Section \ref{previousWork}); Revisions made to first paragraph in Section \ref{lingDoc} on page \korrPG{077} to present this description vis-a-vis older studies, and what it means to be ‘corpus-based’.}
\KORRgen{Next section 2 is Prosody. Why? First two paragraphs do give a justification for beginning this way, but only relate to why the author is beginning with prosody and not segmental phonology. How do other grammars of Saami dialects begin? Do we seriously expect phonology before we have discussed syntax or morphology? What is the expected structure of a corpus-based grammar? Too many questions.}{I’m pleased with the structure as it is. How grammars of other Saami languages are structured is not really relevant. That this is a corpus-based grammar does not affect its structure, just the method of collecting the data for it. No revisions made}
\KORRgen{[from email to M. Haspelmath] Reading this grammar as an outsider-nonspecialist, I find my major problem to be one of organization. What the author presents as "phonology" is so tightly involved with the morphology of Saami languages that until one gets into the well-presented morphological examples, the prosodic (and some segmental) phonological observations make very little sense. I think some of this could be overcome with a little reorganization: some morphology could come first. However, I could also use much more detailed citation on the origin and nature of conventional descriptions of Saami prosody, and a clear explanation of which parts of the analysis have been accepted in the Saamic field, and which are the author's own. Throughout, I could use much more recognition by the author of the contributions of other linguists working on Saami languages.}{The structure stands as it is, as pointed out in the previous reaction. I have already marked things from Saami studies, any other analyses and even organizational structures in the book are mine. While much literature on Saami languages exists, it is almost exclusively embedded in the Finno-Ugric philological tradition, while the point of this book is to describe \PS\ from the point of general linguistics; as a result, the amount of references to other works from Saami studies is rather limited.}
%\KORR{}{}{}
%\KORR{}{}{}
%\KORR{}{}{}
%\KORR{}{}{}
%\KORR{}{}{}
%\KORR{}{}{}
\end{longtable}
%%%%
%%%%
\pagebreak
%%%%%%%

\begin{longtable}{p{.97\textwidth}}\hline\endhead\korr{revC2}
\BfIt{\Blue{Suggestions from Reviewer C from comments in pdf}}\\*\hline
\KORRgen{[comment on title] I do not know what it means for this grammar to be corpus based.  We really need some definitions of what the author considers sufficient evidence for a form or phenomenon. And does he really derive it all from the corpus?  If so, why is it so much like the other grammars of Saami languages? }{This is all discussed in the introduction (Chapter \ref{introChapter}). The point of the title is to emphasize that it is based on a specific corpus, and not on previous work or historic reconstructions (unlike other works in Saami studies). Why should a corpus-based grammar not be like other grammars of Saami languages?}
\KORR{morphology}{This chapter would better be called "morphological processes" or something, since it just describes types of morphological marking in general, which are later exemplified much better in context.  It is possible that parts of the chapter on prosody could mix with this information, and the rest of that chapter be discussed in relation to segmental phonology? I know it's difficult to know where to start, but I really found it hard to read the phonological material without knowing what kinds of forms I would be seeing in morphology and in connected phrases.}{No more major changes to the structure at this point. Anyway, except for reviews like this, most readers will probably not read it from beginning to end, but will look for information relevant for their needs only.}
\KORR{introChapter}{I think that I could have used a brief section on the goals of this study, right up at the top. Everything that is in the introduction needs to be there, but I also need a sense of where I will be going if I read this book.}{Mini summary added to the beginning of Chapter \ref{introChapter} on page \korrPG{055}, but I’m not so sure this is necessary.}
\KORR{geography}{use of 1sS pronoun, dispreferred}{Sorry, it was me who met these people, me who is writing this book. And I see no point in avoiding that fact at all costs.}
\noKORR{geography}{the map is nice, but how will it show up in the book?  Will it be in color? BW?  Should it be reviewed in BW?}{good question, ask about this (map in Figure \ref{piteTerritory})}
\KORR{previousWork}{evaluative words like "useful" sound creep in here and there, sounding arrogant and condescending}{Such words have been removed.}
\KORR{previousWork}{It may be that the corpus alone could magically provide all insights needed to write a grammar; however, I dont this, and think "indirect role" is a little too slight an acknowledgement of previous scholarship.  I think this is a matter of polishing the wording a little, not redoing the whole intro}{Slight modification on page \korrPG{056}.}
\KORR{archiveAccess}{This note, conventional as it is, makes me wonder how different potential users of the grammar would react.  Most of us are "anyone."  When the book has been on the shelf in the library for 20 years, will the author still have the same email address?  I think this link is appropriate in the diss but not the book.  I don't know how to think about the longevity of audio archives, relative to a published volume.  Maybe the data is eternally in place, and undeletable, but the links can be dreadfully ephemeral.}{Very good point. I have deleted the paragraph suggesting contacting me. Who knows what will be in the future, but listing this archive at least by name should make it possible to track down the data somehow, or at least find a contact person who can help.}
\KORR{orthography}{I'm conflicted about this presentation of the orthography and segmental inventories.   It's appropriate to present the two together, certainly, but since the inventories are simply presented as lists, the featural relationships and inventory structure are not highlighted. I feel as if the reader is already expected to know quite a bit about Saami sounds.    }{This is provided purely as a reference to help understand examples written in orthography. Perhaps it should be included in the Appendix instead? Again, most readers will not read this linearly from beginning to end, so it should be included somewhere. A note to refer to Chapter \ref{csANDvs} on segmental phonology has been added on page \korrPG{057}.}
\KORR{ProsodicStructure}{It does seem unusual to start with prosody, but my problem with this section has less to do with this choice (which the author does argue for) than with the omission of systematic citation for ideas about Saami languages that he is adopting as assumptions.  At the very lest, Sammalahti 1998 could be appealed to (a general overview, cited in another grammar I looked at.)}{My ideas are all my own, thus there aren’t many works cited here, except where something is not my own. No revisions made.}
\KORR{ProsodicStructure}{There is another issue, which is the following.  Prosodic domains in Saami languages are important not only because they relate to phonological processes, but because they are connected to morphology in a complex way.  For the outsider beginning to read the grammar, it is very unclear why a disyllabic (by the way, I HATE the etymological mongrel "bisyllabic") domain should be appealed to as something basic - especially when it turns out that describing it as a "foot" doesn't quite work.   Once one has a look at the morphology, all this makes more sense.  The author is trying to organize the grammar agnostically, I think, but still feels the pull to answer difficult questions about morphology by appealing to prosody.  This will make sense if we understand what the problems are, but for a non-initiate, the prosodic introduction is not helpful.}{Starting with morphology would lead to similar difficulties. You have to start somewhere. This structure is still satisfactory to me. No revisions made.}
\KORR{multisyllabicWords}{p. 26: this requires a citation}{It’s not clear what this refers to, but citations have been added to ‘basic phonological theory’ and ‘prosodic hierarchy’ on page \korrPG{058}.}
\KORR{wordStress}{p. 27: does he mean 'word’?}{no, i mean foot, as stated. No revision made.}
\KORR{wordStress}{p. 27: this suggests the feet are quantity-insensitive, but he didn't say that in defining them.  should be referring to metrical theory}{This was mentioned in the previous section. No revision made.}
\KORR{wordStress}{p. 27: it would be good to say how this conclusion was arrived at.  If it is impressionistic, fine, but he should say.}{Revised on page \korrPG{001} accordingly.}
\KORR{foot}{again, references to the concept of 'foot' should be cited}{These are mentioned in the beginning of this section (Section \ref{multisyllabicWords}). I think this is sufficient, and that the average reader will not have a problem with this term, especially since I definite it as well for \PS\ in that same section. No revision made.}
\KORR{footOnset}{does this concept [???] exist in the description of other languages? If so, it is worth discussing, if not, it is probably important to say why it is important for Saamic linguistics.  I've never heard of it.}{These (see also following comments) are all my terms, my approach to describing the phonology of \PS; I have included similar terms used in ‘Saamic linguistics’ for reference/comparison only; these have been cited now on pages \korrPG{059} through \korrPG{060}.}
\KORR{exampleFootedness}{p. 29: I think I would find the description of the salient prosodic positions easier to swallow if I could see examples along with the description. Separated this way, I have to keep looking back.}{I have added a reference on page \korrPG{061} to Table \ref{domainExamples}, although I’m not sure this is really necessary. Otherwise I stick to my structure of presenting these prosodic positions in general, then fleshing them out in Section \ref{exampleFootedness}.}
\KORR{exampleFootedness}{p. 30: I’d be happier if I knew what consonant gradation meant.  I kind of want the moprhological questions first, followed by the prosodic explanation.}{Not changing structure of the whole thing. There is already a reference in the text to Section \ref{Cgrad} on consonant gradation. No revisions made.}
\KORR{syllabification}{p. 31: citation [probably referring to sonority sequencing principle]}{Citation added on page \korrPG{062}.}
\KORR{syllabification}{p. 32: how do you know this?}{This is the only possibility assuming the syllabification process I argue for at the beginning of the section. Ultimately, syllabification is not relevant for much in \PS, so it’s not terribly important anyway. No revision made.}
\KORR{syllabification}{p. 33: how do you know this?}{This is the only possibility assuming the syllabification process I argue for at the beginning of the section. Ultimately, syllabification is not relevant for much in \PS, so it’s not terribly important anyway (explained in the following section). No revision made.}
\KORR{footedness}{p. 33: doesn't this suggest that the relevant domain is not actually phonological, but morphological?  The idea of a foot as being built on syllables (not syllabic diphones, like the minimal core) is not specific to Saami linguistics, and the reinterpretation of this core as a foot doesn't really seem to help the description much, at least not yet.}{This is an essential domain in \PS\ morphology, but it is most elegantly described referring to these prosodic domains, and not morphologically. No revision made.}
\KORR{csANDvs}{p. 37: The traditionalist in me would prefer to see this chapter first, before the prosodic phonology}{No revision made.}
\KORR{preaspiration}{p. 38, footnote 1: citation please!}{Citations added to \PS\ and other Saami literature on p. \korrPG{063}.}
\KORR{geminateCs}{p. 40: These are the places where I think, oh wouldn't it be cool to hear the word.  So I try to imagine what it will be like in the future of this book to access these data.  Suppose I am "anyone," as the outsider reader of the grammar, and since it's 2043 and the grammar has been in the library since 2013; will the email address still work?  What other mechanisms are there to make the connection to the archive?  Can I get more than a disappointing scrap of metadata?  }{Data in the IMDI archive in Nijmegen, hopefully still accessible in the future – definitely something vital to think about!}
\KORR{consecutiveCs}{p. 57: Is this the first time that consonant gradation has actually been mentioned in text, beyond acknowledging its existence?   Whatever, it still doesn't follow that a list of possible consonant clusters is the answer.  }{Consonant gradation is morphophonology, thus it doesn’t get much treatment here, yet. Continuing to the next comment at the end of the same section…}
\KORR{consecutiveCs}{OK, so we have a list of possible consonant clusters in the consonant center.  this does not feel explanatory to me.  I think the author really needs to acknowledge the role of morphology in the shape of words.  It seems as though he wants prosody to take on a large explanatory role, but without knowing the morphological questions to be answered, we can't appreciate that approach.  In any case, I can't see how this list helps. }{A small addition was made to the very end of Section \ref{consecutiveCs} on page \korrPG{064} to point the reader to realize that the existence of such extensive (and potentially useless) inventories can best be explained in light of the morphology that takes place here.}
\KORR{Vallophones}{p. 61: the terminology "closed" is unfamiliar, as opposed to IPA "close."   The sentence about the closed oral cavity doesn't make sense for a vowel.   I see a few other nonstandard labels in the phonological description, and recommend the author look carefully at usage and make it consistent.  }{changes made, p. \korrPG{065} and elsewhere from ‘closed’ to ‘close’.}
\KORR{Vallophones}{p. 62: Same thing here.  Does somebody say "closed" for vowels?  }{see response to previous comment.}
\KORR{Vallophones}{p. 65: bucket vowel indicates a more open vowel than u.  I think the author needs to take a close look at the vowel descriptions.}{Revised mistakes here on page \korrPG{066}.}
\KORR{Vallophones}{p. 66: In these descriptions, some of the most interesting stuff seems to get thrown into the footnotes.  Why can't we have a sentence that explains that this vowel harmony appears to operate phonologically, but that there is another harmony pattern that can only be explained by reference to morphology?  }{Ok, took this out of the footnote on page \korrPG{067}.}
\KORR{epentheticSchwa}{p. 66: the word "epenthesis" triggers skepticism in Athabaskanists.   How do we know the vowel-zero alternation represents epenthesis?  There should be an argument behind the term, or at least a reference to somebody else's argument.}{I try to avoid postulating zeros, and to avoid frameworks that do this; I think ‘epenthetic’ is sufficiently clear, without being too framework-driven, either. No revision made. But I have added a few words about how this phenomenon is quite rare on page \korrPG{068}. Not to mention that the final paragraph in this section goes into why it’s likely not even phonemic.}
\KORR{epentheticSchwa}{p. 67: all the more reason not to label as "epenthesis" until these issues are resolved.}{See response to previous comment.}
\KORR{VH}{p. 73: might seem like a funny place to put this usually phonological phenomenon, but it seems in Saami languages that vowel harmony operates more within morphology than phonology.}{Indeed. No revision made.}
\KORR{introWordForms}{p. 77: I wonder whether it would be possible to incorporate the material in these very short overview chapters into the chapters that follow.  I like their clarity and the way they set up their section, but I don't know why they have to be chapters. }{This short chapter is now section \ref{introWordForms}; merged with former chapter on morphology (Ch. \ref{morphWordClassCh}).}
\KORR{nouns}{p. 79: I think editors often prefer "that" to "which" in this context.  The highlighted sentence has a number agreement problem. }{Slight revisions made on page \korrPG{069}.}
\KORR{essive}{p. 88: This type of comment seems overly personal and informal, and perhaps the author could try to make them more neutral, while still communicating the type of distinction he wants to make.}{I disagree with trying to avoid the fact that I interacted with other people in gathering the data. This particular sentence even provides additional commentary on the data which is not usually provided when just looking at the ‘raw’ data.}
\KORR{possSuffixes}{p. 111: Here and in some other places I'm confused as to where the generalizations come from.  Most of the examples are presented without argumentation, though with a corpus example; i'm not sure what kind of evidence the author is relying on for the acceptance or rejection of a structural description.   Here the word "enough" is used, and I don't have any definition of "enough" in the context of corpus grammar.}{My approach to this description is precisely that: to describe what I find in the corpus. A single example is thus potentially ‘enough’ to warrant incorporation into the description as a valid structure for the language. Thus, particularly in places like this, where the data is insufficient, it is beyond my abilities to say how ‘general’ such structures are. No revisions made.}
\KORR{pronouns}{p. 113: this is what is bothering me.  I think it would be nice if the criteria for consideration of a structure were clearer.  I don't mind what they are.  }{Not sure what exactly this comment refers to, but I have added a few lines to the very beginning of chapter \ref{pronouns} on page \korrPG{070} concerning the syntactic criteria for pronouns.}
\KORR{demonstrativePronouns}{p. 115: I like the question mark, but I would like to know more about how forms are chosen  - the criteria for choosing or questioning forms.}{Note added on page \korrPG{071}.}
\KORR{POTinflection}{p. 164: I appreciate the footnotes on these remarks, but for some especially, it would be nice to have page numbers (esp since we have so many references to some of the same grammatical references) }{Page numbers added to text and footnote on page \korrPG{072}.}
%\KORR{}{p. 165: yellow highlighting without comment}{}
\KORR{verbInflectionalClasses}{p. 166: in some passages like this one, it seems as if the author is commenting on the whole enterprise of "corpus" grammar, concerned about its observational adequacy.  Yet elicitation was used, as well as reference to grammars of closely related languages.  Why do the limitations of corpus work matter so much here? }{This part of the text is important in pointing out that much of the information is \It{not} from ‘natural’ language of the corpus, but from elicitation sessions. This is important for transparency as it could potentially affect the way the reader interprets the data. No revisions made.}
\KORR{otherVerbClasses}{p. 177: it seems like the process of creating this grammar involved the analysis of the corpus and the matching of its contents to structures described for other Saami languages.   One of the odd things about it is that the author keeps apologizing for absence of data- but who write a a grammar based on a really statistically sufficient pile of data?  What is his benchmark for the enterprise? }{The point of such comments is to be transparent (honest!) about the data and its limitations, rather than just positing something even when the data is insufficient. Because \PS\ to some extent and other Saami languages are well described, certain expectations exist (e.g., what structures to expect), and it is vital, and ultimately scientific, to point out when the data is ultimately insufficient in reliably fulfilling such expectations. No revisions made.}
\KORR{QpartWordform}{p. 185: a lot of extra dithering about inadequacy of data.  I think sections like this could be pruned a lot.  }{I disagree. See reactions to previous comments. No revision made.}
%\noKORR{}{p. 199: yellow highlighting without comment)}{}
%\KORR{}{}{}
%\KORR{}{}{}
%\KORR{}{}{}
%\KORR{}{}{}
%\KORR{}{}{}
%\KORR{}{}{}
%\KORR{}{}{}
%\KORR{}{}{}
%\KORR{}{}{}
%\KORR{}{}{}
%\KORR{}{}{}
%\KORR{}{}{}
%\KORR{}{}{}
%\KORR{}{}{}
%\KORR{}{}{}
%\KORR{}{}{}
%\KORR{}{}{}
%\KORR{}{}{}
\end{longtable}
%%%%
%%%%
\pagebreak

%%%%%%%% von consulting editors:
\renewcommand{\spring}{Consulting Editor 1}%

\begin{longtable}{p{.97\textwidth}}\hline\endhead\korr{revCEs}
\BfIt{\Blue{Suggestions from Consulting Editor 1}}\\*\hline
%%%%%%%
\KORRgen{I agree with pretty much everything Reports A and B said as to the necessary addenda and changes that will no doubt greatly improve the work -- particularly the remarks on nonfinite verb forms and syntax}{Revisions done to both non-finite forms and syntax as inventoried above.}
\KORRgen{Report C I found somewhat less helpful, but especially the point about not being condescending to other people's work is worth taking into account, as is the importance of clearly delimiting the author's contribution from what other scholars have said.}{Revisions to section \ref{lingDoc} do exactly this.}
\KORRgen{I was not as unhappy as the reviewer with the prosody-before-morphology thing, so I would rather not encourage the author to turn the phonology section into a blur.
}{No revisions made.}
\end{longtable}

\renewcommand{\spring}{Consulting Editor 2}%
\begin{longtable}{p{.97\textwidth}}\hline\endhead%\korr{revCEs}
\BfIt{\Blue{Suggestions from Consulting Editor 2}}\\*\hline
\KORRgen{I also think that we should ask the author to provide us with a report on how they decided to address the reviewers' suggestions.}{Here you go.}
\end{longtable}

\renewcommand{\spring}{Consulting Editor 3}%
\begin{longtable}{p{.97\textwidth}}\hline\endhead%\korr{revCEs}
\BfIt{\Blue{Suggestions from Consulting Editor 3}}\\*\hline
\KORRgen{1. Need to balance the focus on form with some discussion of function}{I’ll keep this in mind when re-reading the manuscript, but without specific sections, I’m not sure what to add or where to do it. Overall, I try to focus on forms and present the most usual functions, and leave out other functions without enough data.}
\KORRgen{2. Clarify the 'grey areas' in speakers competencies}{Revisions made to section \ref{sociolinguistics} on page \korrPG{034} should make it more clear that swedish domination of everyday life for most \PS\ speakers could be an important factor in grey areas in the data. But no chance to do more field work at this point to fill holes in the data.}
\KORRgen{3. More comparison with related languages}{I don’t see this as a main point of this presentation of \PS\ at this stage. There are enough references out there for scholars to do this on a topic-by-topic basis.}
\KORRgen{4. Improvements in organising the sections (although the suggestions conflict)}{Lots of reorganization done; see inventory above.}
\KORRgen{5. Making 'live' links between examples and the corpus if this is possible - if it is seen as technically 'too difficult' I think that what we need to do at Language Sciences Press is make this possible, not just for this author but for the many future authors who will want to make to make the most of the electronic medium (such as myself!).}{This would be very cool! Not sure how to implement it now, especially guarantee long-term functionality.}
\KORRgen{a section on non-finite verb forms for example.}{Done. Cf. Section \ref{nonFiniteVerbforms}.}
\end{longtable}

\renewcommand{\spring}{Consulting Editor 4}%
\begin{longtable}{p{.97\textwidth}}\hline\endhead%\korr{revCEs}
\BfIt{\Blue{Suggestions from Consulting Editor 4}}\\*\hline
\KORRgen{I have a minor question regarding the author's analysis that geminates split up at syllable boundaries. Reviewer A pointed out that if these were phonemes they shouldn't be further divisible. However, I have a similar analysis for a language I am studying.}{Revisions made to section \ref{syllabification} on syllabification on page \korrPG{020}.}
\end{longtable}

\pagebreak
\renewcommand{\spring}{Martin Haspelmath}%
\begin{longtable}{p{.97\textwidth}}\hline\endhead%\korr{revCEs}
\BfIt{\Blue{Suggestions from Martin Haspelmath (email )}}\\*\hline
\KORRgen{Vielleicht ist es nicht sehr schwierig, z.B. etwas mehr über die Morphologie der nichtfiniten Formen zu sagen.}{}
\KORRgen{Und ich würde mich dem Gutachter anschließen, der vorschlägt, die Mini-Kapitel zu überdenken (also z.B. Kap. 4-5 zusammenfassen und "Morphological patterns and word classes" nennen).
}{}
\end{longtable}

%%%%
%%%%
%\pagebreak



%%%%%%% von JKW:
\renewcommand{\spring}{JW}%

\begin{longtable}{p{.97\textwidth}}\hline\endhead\korr{jkw}
\BfIt{\Blue{Revisions done by Joshua Wilbur on his own}}\\*\hline
%%%%%%%
\KORRgen{For Section \ref{nounClasses}, including class II nouns as part of class I (but as a separate subclass Ie, with a discussion of \jvh) would be more elegant}{the section \ref{nounClasses} on inflectional classes in nouns has been updated accordingly, as well as references to inflectional classes for nouns throughout. Now there are three noun classes I, II, III.}
\KORRgen{potential problem with including non-nominal pro-form interrogatives in the pronoun chapter}{small revision to explain why there are here nonetheless (syntactically still pro-forms) on p. \korrPG{010} in the section on non-nominal pro-forms}
\KORR{wordStress}{no mention of deviant word stress in some recent Swedish loans}{added on page \korrPG{039}.}
\KORRgen{orthography}{General revisions, updates to the section on orthographic conventions in Section \ref{orthography}.}
\KORR{PSDPcorpus}{unclear, needs revision (last sentence in Section 1.2.2 on p. 11).}{Revised to be more clear concerning why access to archived data is necessary as scientific undertaking (Section \ref{PSDPcorpus} on page \korrPG{078}).}
\KORR{Plosives}{inconsistent description of plosive+plosive/affricate CCs in consonant center in Section 3.1.1.3}{Revisions made throughout Section \ref{Plosives} to clarify that aspiration occurs here.}
\KORRgen{not enough mention of allophony of preaspiration in Section 3.1.1.3 (plosives) and 3.1.1.4 (affricates)}{Revisions made throughout Sections \ref{Plosives} and \ref{Affricates} to refer to section \ref{preaspiration} on allophony in preaspirated Cs.}
%\KORR{}{}{}
%\KORR{}{}{}
%\KORR{}{}{}
%\KORR{}{}{}
%\KORR{}{}{}
%\KORR{}{}{}
%\KORR{}{}{}

\end{longtable}
%%%%
%%%%
%\pagebreak









%\end{document}