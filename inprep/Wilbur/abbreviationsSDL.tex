%\documentclass[ number=5
			   ,series=sidl
			   ,isbn=xxx-x-xxxxxx-xx-x
			   ,url=http://langsci-press.org/catalog/book/17
			   ,output=long   % long|short|inprep              
			   %,blackandwhite
			   %,smallfont
			   ,draftmode   
			  ]{LSP/langsci}                          

\usepackage{LSP/lsp-styles/lsp-gb4e}		% verhindert Komma bei mehrfachen Fußnoten?
\usepackage{LSP/lsp-styles/avm}
\avmfont{\sc} 
\avmvalfont{\it}
                                                      
\usepackage{layout}
\usepackage{lipsum}

%%for LSP-lines in tables: %%doesn’t work for some reason. Plus, not all my tables have a single-line header. Double-lines aren’t typeset properly in ‘longtable’-environment across several pages.
%\usepackage{booktabs}
%\newcommand{\mytoprule}{\midrule\toprule}
%\newcommand{\mybottomrule}{\bottomrule\midrule}

%%%following now in main document (XeTeX_pitePhDSDL.tex)
%\title{A corpus-based grammar \\ of spoken Pite Saami}  
%%\subtitle{2000+ Years of Language Science and no End in Sight}  
%\author{Joshua Wilbur}
%\dedication{Gijtov adnet!}
%\renewcommand{\lsBackBody}{This grammar of Pite Saami (Uralic; Sweden) is simply bad-ass.}%for back cover text
%\renewcommand{\lsBackTitle}{Biddumsáme giella}%for back title

%%%% ABOVE FOR LangSciPress %%%%
%%%% ABOVE FOR LangSciPress %%%%
%%%% ABOVE FOR LangSciPress %%%%

\usepackage{longtable}

\usepackage{multirow}
\usepackage{array} %allows, among other things, centering column content in a table while also specifying width, creates new column style "x" for center-alignment, "y" for right-alignment
\newcolumntype{x}[1]{%
>{\centering\hspace{0pt}}p{#1}}%
\newcolumntype{y}[1]{%
>{\raggedleft\hspace{0pt}}p{#1}}%

\usepackage[]{placeins}%using \FloatBarrier command, all floats still floating at that point will be typeset, and cannot cross that boundary. the option here \usepackage[section]{placeins} automatically adds \FloatBarrier to every \section command (only works for \section commands, nothing lower than that!)
\usepackage{afterpage}%by using the command \afterpage{\clearpage}, all floats will appear, but no new page will be started, thus avoiding bad page breaks around floats

\usepackage{vowel} %for vowel space chart


%%IS THIS NECESSARY??
%%%following allows you to refer to footnotes (from http://anthony.liekens.net/index.php/LaTeX/MultipleFootnoteReferences)
\newcommand{\footnoteremember}[2]{
  \footnote{#2}
  \newcounter{#1}
  \setcounter{#1}{\value{footnote}}
} \newcommand{\footnoterecall}[1]{
  \footnotemark[\value{#1}]} 
%%%previous allows you to refer to footnotes: use \footnoteremember{referenceText} in footnote, then \footnoterecall{referenceText} to refer.

\usepackage{tikz}
\usetikzlibrary{plothandlers,matrix,decorations.text,shapes.arrows,shadows,chains,positioning,scopes}

\usepackage{synttree} %zeichnet linguistische Bäume
\branchheight{36pt}%sets height between rows in synttree

\usepackage{lscape}%used for landscape pages in index (list of recordings)

\usepackage{polyglossia}
\setmainlanguage{english}



%%%TAKE OUT FOR FINAL VERSION:
%%%TAKE OUT FOR FINAL VERSION:
%%%TAKE OUT FOR FINAL VERSION:

%%%following readjusts margin text!
\setlength{\marginparwidth}{20mm}
\let\oldmarginpar\marginpar
\renewcommand\marginpar[1]{\-\oldmarginpar[\raggedleft\footnotesize\vspace{-7pt}\color{red}\It{→ #1}]%
{\raggedright\footnotesize\vspace{-7pt}\color{red}\It{→ #1}}}
%%%previous readjusts margin text!

%%%The following lines set depth of ToC (LSP default is only 3 levels)!
%%%\renewcommand{\contentsname}{Table of Contents} % überschrift des inhaltsverzeichnisses
%\setcounter{secnumdepth}{5}%sets how deep section/subsection/subsubsections are numbered
%\setcounter{tocdepth}{5}%sets the depth of the ToC %but this doesn't seem to work!!!
\include{newcommandsSDL}\include{hyphenationSDL}\begin{document}

%%%%%%%%%%%%%%%%%%%%%%%%%%%%%%%% ALL THE ABOVE TO BE COMMENTED OUT FOR COMPLETE DOCUMENT! %%%%%%%%%%%

\chapter*{Abbreviations and symbols}\label{abbreviations}
%\Red{\Large Move to front material!!!} \\
%Particularly when glossing \PS\ text, abbreviations are used to save space. 
Table \vref{abbrevList} provides a list of abbreviations used in the present study and their full forms; grammatical categories are represented by text in \Sc{small capitals}. 
Table \ref{symbolList} lists symbols used and what they indicate. 

\begin{longtable}{ll}%\centering
\caption{List of abbreviations used in this study\label{abbrevList}}\\%longtable-caption a bit special: needs \label{} inside caption, needs \\ at end of line, comes after \begin{longtable}{cclr}...
\hline\It{abbreviation} & \It{full form} \\\hline
\endfirsthead
%\begin{tabular}{ l  l }%\hline
\hline\It{abbreviation} & \It{full form} \\\hline\endhead%\hline
\hline
\endfoot
%\hline\\
%\endlastfoot
\Sc{abess}	& abessive \\
\Sc{acc}	& accusative \\
%adj & adjective\\
\Sc{adjz}	& adjectivalizer \\
AdvP	&adverbial phrase\\
\Sc{advz}	& adverbializer \\
AP		&adjectival phrase\\
\Sc{attr} 	& attributive\\
%adv & adverb \Red{nec.?}\\
C		&consonant segment\\
CC		&two-segment consonant cluster\\
CCC		&three-segment consonant cluster\\
\Sc{card}	& cardinal number \\
\Sc{com}	& comitative \\
\Sc{comp}	& comparative \\
\Sc{conneg}& connegative \\
\Sc{dem}	& demonstrative\\
\Sc{dim}	& diminutive \\
\Sc{dist}	& distal \\
\Sc{elat}	& elative \\
\Sc{ess}	& essive \\
\Sc{gen}	& genitive \\
\Sc{ill}	& illative \\
\Sc{imp}	& imperative \\
\Sc{iness}	& inessive \\
\Sc{inf}	& infinitive \\
%\Sc{med}	& medial \\%decided not to use this anymore
\Sc{nmlz}	& nominalizer \\
\Sc{nom}	& nominative \\
%n&noun \Red{nec.?}\\
NP&nominal phrase\\
%num& numeral \Red{nec.?}\\
\Sc{ord}	& ordinal number \\
%part&particle \Red{nec.?}\\
\Sc{pl}	& plural \\
%pn&pronoun \Red{nec.?}\\
PP&postpositional phrase\\
%postp&postposition \Red{nec.?}\\
\Sc{pot}	& potential \\
\Sc{pred} & predicative\\
\Sc{prf}	& perfective aspect \\
\Sc{prog}	& progressive aspect \\
\Sc{prox}	& proximal \\
\Sc{prs}	& present tense \\
%PSDP	& the \PSDP\ \Red{nec.?} \\
\Sc{pst}	& past tense \\
%\Sc{purp}	& purposive  \Red{nec.?}\\
\Sc{Q}	& question marker \\
\Sc{refl}	& reflexive\\
\Sc{rel}	& relative pronoun \\
\Sc{rmt}	& remote \\
%sf&suffix \Red{nec.?}\\
\Sc{sg}	& singular \\
str		&strong grade (consonant gradation) \\
\Sc{subord}& subordinator \\
\Sc{superl}& superlative \\
Sw.		& Swedish \\
%interj&interjection \Red{nec.?}\\
%v&verb \Red{nec.?}\\
V		&vowel segment\\
\Sc{vblz}	& verbalizer \\
VC		&verbal complex\\
VH		&vowel harmony\\
wk		&weak grade (consonant gradation) \\
%?&unsure \Red{nec.?}\\%\hline
%\hline
%\\%\end{tabular}
%\caption{List of abbreviations used in this study.}\label{abbrevList}
\end{longtable}

\begin{table}\centering
\caption{List of symbols used in this study}\label{symbolList}
\begin{tabular}{c l }\hline
\It{symbol	}&\It{indicates} 	\\\hline
-		& segmentable morpheme boundary \\
\BS		& morpheme via stem alternation (non-segmentable) \\
\PLUS	& compound boundary \\%\Red{nec.?} \\
=		& clitic boundary \\
.		& syllable boundary\\
σ		& syllable \\
∑		& morphological stem \\
*		& ungrammatical form \\%\Red{nec.?}\\
<		& source language \\
/ /		& phonological representation\\
\verb|[ ]|		& phonetic form\\%%\verb => for short verbatim text. the delimiters can be anything except <*>; here I use |
< >		& orthographic representation\\\hline
\end{tabular}
\end{table}



%\end{document}