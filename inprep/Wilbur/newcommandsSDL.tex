%% new commands for LSP book (Grammar of Pite Saami, by J. Wilbur)

\newcommand{\PS}{Pite Saami}
\newcommand{\PSDP}{Pite Saami Documentation Project}
\newcommand{\WLP}{Wordlist Project}

\newcommand{\HANG}{\everypar{\hangindent15pt \hangafter1}}%also useful for table cells

\newcommand{\REF}[1]{(\ref{#1})}%adds parenthesis around the reference number, particularly useful for examples.%\Ref had clash with LSP!
\newcommand{\dline}{\hline\hline}%makes a double line in a table
\newcommand{\superS}[1]{\textsuperscript{#1}}%adds superscript element
\newcommand{\sub}[1]{$_{#1}$}%adds subscript element
\newcommand{\Sc}[1]{\textsc{#1}}%shortcut for small capitals (not to be confused with \sc, which changes the font from that point on)
\newcommand{\It}[1]{\textit{#1}}%shortcut for italics (not to be confused with \it, which changes the font from that point on)
\newcommand{\Bf}[1]{\textbf{#1}}%shortcut for bold (not to be confused with \bf, which changes the font from that point on)
\newcommand{\BfIt}[1]{\textbf{\textit{#1}}}
\newcommand{\BfSc}[1]{\textbf{\textsc{#1}}}
\newcommand{\Tn}[1]{\textnormal{#1}}%shortcut for normal text (undo italics, bolt, etc.)
\newcommand{\MC}{\multicolumn}%shortcut for multicolumn command in tabular environment - only replaces command, not variables!
\newcommand{\MR}{\multirow}%shortcut for multicolumn command in tabular environment - only replaces command, not variables!
\newcommand{\TILDE}{{\fontspec{Arial Unicode MS}~}}%shortcut for tilde%\Tilde clash with LSP!
\newcommand{\BS}{\textbackslash}%backslash
\newcommand{\Red}[1]{{\color{red}{#1}}}%for red text
\newcommand{\Blue}[1]{{\color{blue}{#1}}}%for blue text
\newcommand{\PLUS}{{\fontspec{Arial}+}}%nicer looking plus symbol
\newcommand{\MINUS}{{\fontspec{Arial}-}}%nicer looking plus symbol
%\newcommand{\PLUSMINUS}{{\fontspec{Arial}±}}%nicer looking plus-minus symbol
\newcommand{\ARROW}{{\fontspec{DejaVu Sans}➜}}%nicer looking arrow
\newcommand{\BULLET}{{\fontspec{Arial Unicode MS}•}}
\newcommand{\CH}{\fontspec{Arial Unicode MS}✓}%CH as in CHeck
%\newcommand{\Marker}{{\fontspec{Arial Unicode MS}☀}}
%\newcommand{\X}{\Marker}%used to indicated redundant combination in CC table
%\newcommand{\minisection}[1]{\paragraph{#1}}%makes a mini section with no indent and bold mini-title

\newcommand{\jvh}{\It{j}-suffix vowel harmony}%
%\newcommand{\Ptcl}{\Sc{ptcl} }%just shortcut for glossing ‘particle’
%\newcommand{\ATTR}{{\Sc{attributive}}}%shortcut for ATTRIBUTIVE in small caps
%\newcommand{\PRED}{{\Sc{predicative}}}%shortcut for PREDICATIVE in small caps
%\newcommand{\COMP}{{\Sc{comparative}}}%shortcut for COMPARATIVE in small caps
%\newcommand{\SUPERL}{{\Sc{superlative}}}%shortcut for SUPERLATIVE in small caps
\newcommand{\SG}{{\Sc{singular}}}%shortcut for SINGULAR in small caps
%\newcommand{\DU}{{\Sc{dual}}}%shortcut for DUAL in small caps
\newcommand{\PL}{{\Sc{plural}}}%shortcut for PLURAL in small caps
%\newcommand{\NOM}{{\Sc{nominative}}}%shortcut for NOMINATIVE in small caps
%\newcommand{\ACC}{{\Sc{accusative}}}%shortcut for ACCUSATIVE in small caps
%\newcommand{\GEN}{{\Sc{genitive}}}%shortcut for GENITIVE in small caps
%\newcommand{\ILL}{{\Sc{illative}}}%shortcut for ILLATIVE in small caps
%\newcommand{\INESS}{{\Sc{inessive}}}%shortcut for INESSIVE in small caps
\newcommand{\ELAT}{{\Sc{elative}}}%shortcut for ELATIVE in small caps
%\newcommand{\COM}{{\Sc{comitative}}}%shortcut for COMITATIVE in small caps
%\newcommand{\ABESS}{{\Sc{abessive}}}%shortcut for ABESSIVE in small caps
%\newcommand{\ESS}{{\Sc{essive}}}%shortcut for ESSIVE in small caps
%\newcommand{\DIM}{{\Sc{diminutive}}}%shortcut for DIMINUTIVE in small caps
%\newcommand{\ORD}{{\Sc{ordinal}}}%shortcut for ORDINAL in small caps
%\newcommand{\CARD}{{\Sc{cardinal}}}%shortcut for CARDINAL in small caps
%\newcommand{\PROX}{{\Sc{proximal}}}%shortcut for PROXIMAL in small caps
%\newcommand{\DIST}{{\Sc{distal}}}%shortcut for DISTAL in small caps
%\newcommand{\RMT}{{\Sc{remote}}}%shortcut for REMOTE in small caps
%\newcommand{\REFL}{{\Sc{reflexive}}}%shortcut for REFLEXIVE in small caps
%\newcommand{\PRS}{{\Sc{present}}}%shortcut for PRESENT in small caps
%\newcommand{\PST}{{\Sc{past}}}%shortcut for PAST in small caps
%\newcommand{\IMP}{{\Sc{imperative}}}%shortcut for IMPERATIVE in small caps
%\newcommand{\POT}{{\Sc{potential}}}%shortcut for POTENTIAL in small caps
\newcommand{\PROG}{{\Sc{progressive}}}%shortcut for PROGRESSIVE in small caps
\newcommand{\PRF}{{\Sc{perfect}}}%shortcut for PERFECT in small caps
\newcommand{\INF}{{\Sc{infinitive}}}%shortcut for INFINITIVE in small caps
%\newcommand{\NEG}{{\Sc{negative}}}%shortcut for NEGATIVE in small caps
\newcommand{\CONNEG}{{\Sc{connegative}}}%shortcut for CONNEGATIVE in small caps
\newcommand{\ATTRs}{{\Sc{attr}}}%shortcut for ATTR in small caps
\newcommand{\PREDs}{{\Sc{pred}}}%shortcut for PRED in small caps
%\newcommand{\COMPs}{{\Sc{comp}}}%shortcut for COMP in small caps
%\newcommand{\SUPERLs}{{\Sc{superl}}}%shortcut for SUPERL in small caps
\newcommand{\SGs}{{\Sc{sg}}}%shortcut for SG in small caps
\newcommand{\DUs}{{\Sc{du}}}%shortcut for DU in small caps
\newcommand{\PLs}{{\Sc{pl}}}%shortcut for PL in small caps
\newcommand{\NOMs}{{\Sc{nom}}}%shortcut for NOM in small caps
\newcommand{\ACCs}{{\Sc{acc}}}%shortcut for ACC in small caps
\newcommand{\GENs}{{\Sc{gen}}}%shortcut for GEN in small caps
\newcommand{\ILLs}{{\Sc{ill}}}%shortcut for ILL in small caps
\newcommand{\INESSs}{{\Sc{iness}}}%shortcut for INESS in small caps
\newcommand{\ELATs}{{\Sc{elat}}}%shortcut for ELAT in small caps
\newcommand{\COMs}{{\Sc{com}}}%shortcut for COM in small caps
\newcommand{\ABESSs}{{\Sc{abess}}}%shortcut for ABESS in small caps
\newcommand{\ESSs}{{\Sc{ess}}}%shortcut for ESS in small caps
%\newcommand{\DIMs}{{\Sc{dim}}}%shortcut for DIM in small caps
%\newcommand{\ORDs}{{\Sc{ord}}}%shortcut for ORD in small caps
%\newcommand{\CARDs}{{\Sc{card}}}%shortcut for CARD in small caps
\newcommand{\PROXs}{{\Sc{prox}}}%shortcut for PROX in small caps
\newcommand{\DISTs}{{\Sc{dist}}}%shortcut for DIST in small caps
\newcommand{\RMTs}{{\Sc{rmt}}}%shortcut for RMT in small caps
\newcommand{\REFLs}{{\Sc{refl}}}%shortcut for REFL in small caps
\newcommand{\PRSs}{{\Sc{prs}}}%shortcut for PRS in small caps
\newcommand{\PSTs}{{\Sc{pst}}}%shortcut for PST in small caps
\newcommand{\IMPs}{{\Sc{imp}}}%shortcut for IMP in small caps
\newcommand{\POTs}{{\Sc{pot}}}%shortcut for POT in small caps
\newcommand{\PROGs}{{\Sc{prog}}}%shortcut for PROG in small caps
\newcommand{\PRFs}{{\Sc{prf}}}%shortcut for PRF in small caps
\newcommand{\INFs}{{\Sc{inf}}}%shortcut for INF in small caps
\newcommand{\NEGs}{{\Sc{neg}}}%shortcut for NEG in small caps
\newcommand{\CONNEGs}{{\Sc{conneg}}}%shortcut for CONNEG in small caps

\newcommand{\subNP}{\sub{NP}}%shortcut for NP (nominal phrase) in subscript
\newcommand{\subVC}{\sub{VC}}%shortcut for VC (verb complex) in subscript
\newcommand{\subAP}{\sub{AP}}%shortcut for NP (adjectival phrase) in subscript
\newcommand{\subAdvP}{\sub{AdvP}}%shortcut for AdvP (adverbial phrase) in subscript
\newcommand{\subPP}{\sub{PP}}%shortcut for NP (postpoistional phrase) in subscript

\newcommand{\Nth}{{\footnotesize(\It{n})}}%used in table of numerals in ADJ chapter


%%new transl tier in gb4e; syntax: #1=free translation (in single quotes), #2=additional comments, z.B. literal meaning:
\newcommand{\Transl}[2]{\trans \mbox{‘#1’ #2}}%new transl tier in gb4e;
\newcommand{\TranslLong}[2]{\trans ‘#1’ #2}%new transl tier in gb4e when free translation particularly long
\newcommand{\TranslMulti}[2]{\trans \mbox{\hspace{10pt}‘#1’ #2}}%new transl tier in gb4e for a dialog to be included under a single example number
\newcommand{\TranslLongMulti}[2]{\trans \hspace{10pt}‘#1’ #2}%new transl tier in gb4e for a dialog when free translation is particularly long


%% used for examples in the Prosody and Segmental phonology chapters:
\newcommand{\PhonGloss}[7]{%PhonGloss = Phonology Gloss;
%pattern: \PhonGloss{label}{phonemic}{phonetic}{orthographic}{gloss}{recording}{utterance}
\ea\label{#1}
\Tn{\begin{tabular}{p{30mm} l}
/#2/	& \It{#4} \\
\MC{1}{l}{[#3]}	&\HANG ‘#5’\\%no table row can start with square brackets! thus the workaround with \MC
\end{tabular}\hfill\hyperlink{#6}{{\small\textnormal[pit#6#7]}}\index{Z\Red{rec}!\Red{pit#6}}\index{Z\Red{utt}!\Red{pit#6#7} \Blue{Phon}}}
\z}
\newcommand{\PhonGlossWL}[6]{%PhonGloss = Phonology Gloss for words from WORDLIST, not from corpus!;
%pattern: \PhonGloss{label}{phonemic}{phonetic}{orthographic}{gloss}{wordListNumber}
\ea\label{#1}
\Tn{\begin{tabular}{p{30mm} l}
/#2/	& \It{#4} \\
\MC{1}{l}{[#3]}	&\HANG ‘#5’\\%no table row can start with square brackets! thus the workaround with \MC
\end{tabular}\hfill\hyperlink{explExs}{{\small\textnormal[#6]}}\index{Z\Red{wl}!\Red{#6}\Blue{Phon}}}
\z}


%%for derivation examples in the derivational morphology chapter!
%syntax: \DerivExam{#1}{#2}{#3}{#4}{#5}{#6}
%#1: base, #2: base-gloss, #3: derived form, #4: derived form gloss, #5: derived form translation, #6: pit-recording, #7: utterance number
\newcommand{\DerivExam}[7]{\Tn{\begin{tabular}{lcl}\It{#1}&\ARROW&\It{#3}\\#2&&#4\\\end{tabular}\hfill\pbox{.3\textwidth}{\hfill‘#5’\\\hbox{}\hfill\hyperlink{pit#6}{{\small\textnormal[pit#6.#7]}}}
\index{Z\Red{rec}!\Red{pit#6}}\index{Z\Red{utt}!\Red{pit#6.#7}}}}
\newcommand{\DerivExamWL}[6]{\Tn{\begin{tabular}{lcl}\It{#1}&\ARROW&\It{#3}\\#2&&#4\\\end{tabular}\hfill\pbox{.3\textwidth}{\hfill‘#5’\\\hbox{}\hfill\hyperlink{explExs}{{\small\textnormal[#6]}}}
\index{Z\Red{wl}!\Red{#6}}}}



%INCLUDE RECORDINGS IN INDEX:
\newcommand{\Corpus}[2]{\hspace*{1pt}\hfill{\small\mbox{\hyperlink{pit#1}{\Tn{[pit#1.#2]}}}}\index{Z\Red{rec}!\Red{pit#1}}\index{Z\Red{utt}!\Red{pit#1.#2}}}%
\newcommand{\CorpusE}[2]{\hspace*{1pt}\hfill{\small\mbox{\hyperlink{pit#1}{\Tn{[pit#1.#2 (\It{elic.})]}}}}\index{Z\Red{rec}!\Red{pit#1}}\index{Z\Red{utt}!\Red{pit#1.#2}\Blue{-E}}}%
%%as above, but necessary for recording names which include an underline because the first variable in \href understands _ but the second variable requires \_
\newcommand{\CorpusLink}[3]{\hspace*{1pt}\hfill{\small\mbox{\hyperlink{pit#1}{\Tn{[pit#2.#3]}}}}\index{Z\Red{rec}!\Red{pit#2}}\index{Z\Red{utt}!\Red{pit#2.#3}}}%
%%as above, but for newer recordings which begin with sje20 instead of pit
\newcommand{\CorpusSJE}[2]{\hspace*{1pt}\hfill{\small\mbox{\hyperlink{sje20#1}{\Tn{[sje20#1.#2]}}}}\index{Z\Red{rec}!\Red{sje20#1}}\index{Z\Red{utt}!\Red{sje20#1.#2}}}%
\newcommand{\CorpusSJEE}[2]{\hspace*{1pt}\hfill{\small\mbox{\hyperlink{sje20#1}{\Tn{[sje20#1.#2 (\It{elic.})]}}}}\index{Z\Red{rec}!\Red{sje20#1}}\index{Z\Red{utt}!\Red{sje20#1.#2}\Blue{-E}}}%











