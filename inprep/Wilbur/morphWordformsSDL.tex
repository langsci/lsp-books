%\documentclass[ number=5
			   ,series=sidl
			   ,isbn=xxx-x-xxxxxx-xx-x
			   ,url=http://langsci-press.org/catalog/book/17
			   ,output=long   % long|short|inprep              
			   %,blackandwhite
			   %,smallfont
			   ,draftmode   
			  ]{LSP/langsci}                          

\usepackage{LSP/lsp-styles/lsp-gb4e}		% verhindert Komma bei mehrfachen Fußnoten?
                                                      
\usepackage{layout}
\usepackage{lipsum}

%%%% ABOVE FOR LangSciPress %%%%
%%%% ABOVE FOR LangSciPress %%%%
%%%% ABOVE FOR LangSciPress %%%%
\usepackage{libertine}%work-around solution for rendering problematic characters ʦ, ͡  (mostly in \textbf{})

\usepackage{longtable}%Double-lines (\hline\hline) aren’t typeset properly in ‘longtable’-environment across several pages! conflict with other package (maybe xcolor with option ‘tables’?)

\usepackage{multirow}

\usepackage{array} %allows, among other things, centering column content in a table while also specifying width, creates new column style "x" for center-alignment, "y" for right-alignment
\newcolumntype{x}[1]{>{\centering\hspace{0pt}}p{#1}}%
\newcolumntype{y}[1]{>{\raggedleft\hspace{0pt}}p{#1}}%

\usepackage[]{placeins}%using \FloatBarrier command, all floats still floating at that point will be typeset, and cannot cross that boundary. the option here \usepackage[section]{placeins} automatically adds \FloatBarrier to every \section command (only works for \section commands, nothing lower than that!)
%\usepackage{afterpage}%by using the command \afterpage{\clearpage}, all floats will appear, but no new page will be started, thus avoiding bad page breaks around floats

\usepackage{vowel} %for vowel space chart


%%%IS THIS NECESSARY??
%%%%following allows you to refer to footnotes (from http://anthony.liekens.net/index.php/LaTeX/MultipleFootnoteReferences)
%\newcommand{\footnoteremember}[2]{
%  \footnote{#2}
%  \newcounter{#1}
%  \setcounter{#1}{\value{footnote}}
%} \newcommand{\footnoterecall}[1]{
%  \footnotemark[\value{#1}]} 
%%%%previous allows you to refer to footnotes: use \footnoteremember{referenceText} in footnote, then \footnoterecall{referenceText} to refer.

\usepackage{tikz}%
\usetikzlibrary{plothandlers,matrix,decorations.text,shapes.arrows,shadows,chains,positioning,scopes}

\usepackage{synttree} %zeichnet linguistische Bäume
\branchheight{36pt}%sets height between rows in synttree

\usepackage{lscape}%used for landscape pages in index (list of recordings)

\usepackage{polyglossia}
\setmainlanguage{english}


%%%TAKE OUT FOR FINAL VERSION:
%%%TAKE OUT FOR FINAL VERSION:
%%%TAKE OUT FOR FINAL VERSION:

%%%%following readjusts margin text!
%\setlength{\marginparwidth}{20mm}
%\let\oldmarginpar\marginpar
%\renewcommand\marginpar[1]{\-\oldmarginpar[\raggedleft\footnotesize\vspace{-7pt}\color{red}\It{→ #1}]%
%{\raggedright\footnotesize\vspace{-7pt}\color{red}\It{→ #1}}}
%%%%previous readjusts margin text!

%%%The following lines set depth of ToC (LSP default is only 3 levels)!
%%%\renewcommand{\contentsname}{Table of Contents} % überschrift des inhaltsverzeichnisses
%\setcounter{secnumdepth}{5}%sets how deep section/subsection/subsubsections are numbered
%\setcounter{tocdepth}{5}%sets the depth of the ToC %but this doesn't seem to work!!!
%% new commands for LSP book (Grammar of Pite Saami, by J. Wilbur)

\newcommand{\PS}{Pite Saami}
\newcommand{\PSDP}{Pite Saami Documentation Project}
\newcommand{\WLP}{Wordlist Project}

\newcommand{\HANG}{\everypar{\hangindent15pt \hangafter1}}%also useful for table cells
\newcommand{\FB}{\FloatBarrier}%shortcut for this command to print all floats w/o pagebreak

\newcommand{\REF}[1]{(\ref{#1})}%adds parenthesis around the reference number, particularly useful for examples.%\Ref had clash with LSP!
\newcommand{\dline}{\hline\hline}%makes a double line in a table
\newcommand{\superS}[1]{\textsuperscript{#1}}%adds superscript element
\newcommand{\sub}[1]{$_{#1}$}%adds subscript element
\newcommand{\Sc}[1]{\textsc{#1}}%shortcut for small capitals (not to be confused with \sc, which changes the font from that point on)
\newcommand{\It}[1]{\textit{#1}}%shortcut for italics (not to be confused with \it, which changes the font from that point on)
\newcommand{\Bf}[1]{\textbf{#1}}%shortcut for bold (not to be confused with \bf, which changes the font from that point on)
\newcommand{\BfIt}[1]{\textbf{\textit{#1}}}
\newcommand{\BfSc}[1]{\textbf{\textsc{#1}}}
\newcommand{\Tn}[1]{\textnormal{#1}}%shortcut for normal text (undo italics, bolt, etc.)
\newcommand{\MC}{\multicolumn}%shortcut for multicolumn command in tabular environment - only replaces command, not variables!
\newcommand{\MR}{\multirow}%shortcut for multicolumn command in tabular environment - only replaces command, not variables!
\newcommand{\TILDE}{∼}%U+223C %OLD:~}%shortcut for tilde%command ‘\Tilde’ clashes with LSP!%
\newcommand{\BS}{\textbackslash}%backslash
\newcommand{\Red}[1]{{\color{red}{#1}}}%for red text
\newcommand{\Blue}[1]{{\color{blue}{#1}}}%for blue text
\newcommand{\PLUS}{+}%nicer looking plus symbol
\newcommand{\MINUS}{-}%nicer looking plus symbol
%    Was die Pfeile betrifft, kannst Du mal \Rightarrow \mapsto \textrightarrow probieren und dann \mathbf \boldsymbol oder \pbm dazutun.
\newcommand{\ARROW}{\textrightarrow}%→%dieser dicke Pfeil ➜ wird nicht von der LSP-Font unterstützt: %\newcommand{\ARROW}{{\fontspec{DejaVu Sans}➜}}
\newcommand{\DARROW}{\textleftrightarrow}%↔︎%DoubleARROW
\newcommand{\BULLET}{•}%
%%✓ does not exist in the default LSP font!
\newcommand{\CH}{\checkmark}%%\newcommand{\CH}{\fontspec{Arial Unicode MS}✓}%CH as in CHeck
%%following used to separate alternation forms for consonant gradation and umlaut patterns:
\newcommand{\Div}{‑}%↔︎⬌⟷⬄⟺⇔%non-breaking hyphen: ‑  
\newcommand{\QUES}{\textsuperscript{?}}%marks questionable/uncertain forms

\newcommand{\jvh}{\mbox{\It{j}-suffix} vowel harmony}%
%\newcommand{\Ptcl}{\Sc{ptcl} }%just shortcut for glossing ‘particle’
%\newcommand{\ATTR}{{\Sc{attributive}}}%shortcut for ATTRIBUTIVE in small caps
%\newcommand{\PRED}{{\Sc{predicative}}}%shortcut for PREDICATIVE in small caps
%\newcommand{\COMP}{{\Sc{comparative}}}%shortcut for COMPARATIVE in small caps
%\newcommand{\SUPERL}{{\Sc{superlative}}}%shortcut for SUPERLATIVE in small caps
\newcommand{\SG}{{\Sc{singular}}}%shortcut for SINGULAR in small caps
\newcommand{\DU}{{\Sc{dual}}}%shortcut for DUAL in small caps
\newcommand{\PL}{{\Sc{plural}}}%shortcut for PLURAL in small caps
%\newcommand{\NOM}{{\Sc{nominative}}}%shortcut for NOMINATIVE in small caps
%\newcommand{\ACC}{{\Sc{accusative}}}%shortcut for ACCUSATIVE in small caps
%\newcommand{\GEN}{{\Sc{genitive}}}%shortcut for GENITIVE in small caps
%\newcommand{\ILL}{{\Sc{illative}}}%shortcut for ILLATIVE in small caps
%\newcommand{\INESS}{{\Sc{inessive}}}%shortcut for INESSIVE in small caps
\newcommand{\ELAT}{{\Sc{elative}}}%shortcut for ELATIVE in small caps
%\newcommand{\COM}{{\Sc{comitative}}}%shortcut for COMITATIVE in small caps
%\newcommand{\ABESS}{{\Sc{abessive}}}%shortcut for ABESSIVE in small caps
%\newcommand{\ESS}{{\Sc{essive}}}%shortcut for ESSIVE in small caps
%\newcommand{\DIM}{{\Sc{diminutive}}}%shortcut for DIMINUTIVE in small caps
%\newcommand{\ORD}{{\Sc{ordinal}}}%shortcut for ORDINAL in small caps
%\newcommand{\CARD}{{\Sc{cardinal}}}%shortcut for CARDINAL in small caps
%\newcommand{\PROX}{{\Sc{proximal}}}%shortcut for PROXIMAL in small caps
%\newcommand{\DIST}{{\Sc{distal}}}%shortcut for DISTAL in small caps
%\newcommand{\RMT}{{\Sc{remote}}}%shortcut for REMOTE in small caps
%\newcommand{\REFL}{{\Sc{reflexive}}}%shortcut for REFLEXIVE in small caps
%\newcommand{\PRS}{{\Sc{present}}}%shortcut for PRESENT in small caps
%\newcommand{\PST}{{\Sc{past}}}%shortcut for PAST in small caps
%\newcommand{\IMP}{{\Sc{imperative}}}%shortcut for IMPERATIVE in small caps
%\newcommand{\POT}{{\Sc{potential}}}%shortcut for POTENTIAL in small caps
\newcommand{\PROG}{{\Sc{progressive}}}%shortcut for PROGRESSIVE in small caps
\newcommand{\PRF}{{\Sc{perfect}}}%shortcut for PERFECT in small caps
\newcommand{\INF}{{\Sc{infinitive}}}%shortcut for INFINITIVE in small caps
%\newcommand{\NEG}{{\Sc{negative}}}%shortcut for NEGATIVE in small caps
\newcommand{\CONNEG}{{\Sc{connegative}}}%shortcut for CONNEGATIVE in small caps
\newcommand{\ATTRs}{{\Sc{attr}}}%shortcut for ATTR in small caps
\newcommand{\PREDs}{{\Sc{pred}}}%shortcut for PRED in small caps
%\newcommand{\COMPs}{{\Sc{comp}}}%shortcut for COMP in small caps
%\newcommand{\SUPERLs}{{\Sc{superl}}}%shortcut for SUPERL in small caps
\newcommand{\SGs}{{\Sc{sg}}}%shortcut for SG in small caps
\newcommand{\DUs}{{\Sc{du}}}%shortcut for DU in small caps
\newcommand{\PLs}{{\Sc{pl}}}%shortcut for PL in small caps
\newcommand{\NOMs}{{\Sc{nom}}}%shortcut for NOM in small caps
\newcommand{\ACCs}{{\Sc{acc}}}%shortcut for ACC in small caps
\newcommand{\GENs}{{\Sc{gen}}}%shortcut for GEN in small caps
\newcommand{\ILLs}{{\Sc{ill}}}%shortcut for ILL in small caps
\newcommand{\INESSs}{{\Sc{iness}}}%shortcut for INESS in small caps
\newcommand{\ELATs}{{\Sc{elat}}}%shortcut for ELAT in small caps
\newcommand{\COMs}{{\Sc{com}}}%shortcut for COM in small caps
\newcommand{\ABESSs}{{\Sc{abess}}}%shortcut for ABESS in small caps
\newcommand{\ESSs}{{\Sc{ess}}}%shortcut for ESS in small caps
%\newcommand{\DIMs}{{\Sc{dim}}}%shortcut for DIM in small caps
%\newcommand{\ORDs}{{\Sc{ord}}}%shortcut for ORD in small caps
%\newcommand{\CARDs}{{\Sc{card}}}%shortcut for CARD in small caps
\newcommand{\PROXs}{{\Sc{prox}}}%shortcut for PROX in small caps
\newcommand{\DISTs}{{\Sc{dist}}}%shortcut for DIST in small caps
\newcommand{\RMTs}{{\Sc{rmt}}}%shortcut for RMT in small caps
\newcommand{\REFLs}{{\Sc{refl}}}%shortcut for REFL in small caps
\newcommand{\PRSs}{{\Sc{prs}}}%shortcut for PRS in small caps
\newcommand{\PSTs}{{\Sc{pst}}}%shortcut for PST in small caps
\newcommand{\IMPs}{{\Sc{imp}}}%shortcut for IMP in small caps
\newcommand{\POTs}{{\Sc{pot}}}%shortcut for POT in small caps
\newcommand{\PROGs}{{\Sc{prog}}}%shortcut for PROG in small caps
\newcommand{\PRFs}{{\Sc{prf}}}%shortcut for PRF in small caps
\newcommand{\INFs}{{\Sc{inf}}}%shortcut for INF in small caps
\newcommand{\NEGs}{{\Sc{neg}}}%shortcut for NEG in small caps
\newcommand{\CONNEGs}{{\Sc{conneg}}}%shortcut for CONNEG in small caps

\newcommand{\subNP}{{\footnotesize\sub{NP}}}%shortcut for NP (nominal phrase) in subscript
\newcommand{\subVC}{{\footnotesize\sub{VC}}}%shortcut for VC (verb complex) in subscript
\newcommand{\subAP}{{\footnotesize\sub{AP}}}%shortcut for NP (adjectival phrase) in subscript
\newcommand{\subAdvP}{{\footnotesize\sub{AdvP}}}%shortcut for AdvP (adverbial phrase) in subscript
\newcommand{\subPP}{{\footnotesize\sub{PP}}}%shortcut for NP (postpoistional phrase) in subscript

\newcommand{\ipa}[1]{{\fontspec{Linux Libertine}#1}}%specifying font for IPA characters

\newcommand{\SEC}{§}%standardize section symbol and spacing afterwards
%\newcommand{\SEC}{§\,}%

\newcommand{\Nth}{{\footnotesize(\It{n})}}%used in table of numerals in ADJ chapter

%%newcommands for tables in introductionSDL.tex:
\newcommand{\cliticExs}[3]{\Tn{\begin{tabular}{p{28mm} c p{28mm} p{35mm}}\It{#1}&\ARROW &\It{#2} & ‘#3’\\\end{tabular}}}%specifically for the two clitic examples
\newcommand{\Grapheme}[1]{\It{#1}}%formatting for graphemes in orthography tables
%%new command for the section on orthographic examples; syntax: #1=orthography, #2=phonology, #3=gloss
\newcommand{\SpellEx}[3]{\Tn{\begin{tabular}{p{70pt} p{70pt} l}\ipa{/#2/}&\It{#1}& ‘#3’ \\\end{tabular}}}%formatting for orthographic examples (intro-Chapter)


%%new transl tier in gb4e; syntax: #1=free translation (in single quotes), #2=additional comments, z.B. literal meaning:
\newcommand{\Transl}[2]{\trans\Tn{‘#1’ #2}}%new transl tier in gb4e;
\newcommand{\TranslMulti}[2]{\trans\hspace{12pt}\Tn{‘#1’ #2}}%new transl tier in gb4e for a dialog to be included under a single example number


%% used for examples in the Prosody and Segmental phonology chapters:
\newcommand{\PhonGloss}[7]{%PhonGloss = Phonology Gloss;
%pattern: \PhonGloss{label}{phonemic}{phonetic}{orthographic}{gloss}{recording}{utterance}
\ea\label{#1}
\Tn{\begin{tabular}[t]{p{30mm} l}
\ipa{/#2/}	& \It{#4} \\
\ipa{[#3]}	&\HANG ‘#5’\\%no table row can start with square brackets! thus the workaround with \MC
\end{tabular}\hfill\hyperlink{#6}{{\small\textnormal[pit#6#7]}}%\index{Z\Red{rec}!\Red{pit#6}}\index{Z\Red{utt}!\Red{pit#6#7} \Blue{Phon}}
}
\z}
\newcommand{\PhonGlossWL}[6]{%PhonGloss = Phonology Gloss for words from WORDLIST, not from corpus!;
%pattern: \PhonGloss{label}{phonemic}{phonetic}{orthographic}{gloss}{wordListNumber}
\ea\label{#1}
\Tn{\begin{tabular}[t]{p{30mm} l}
\ipa{/#2/}	& \It{#4} \\
\ipa{[#3]}	&\HANG ‘#5’\\%no table row can start with square brackets! thus the workaround with \MC
\end{tabular}\hfill\hyperlink{explExs}{{\small\textnormal[#6]}}%\index{Z\Red{wl}!\Red{#6}\Blue{Phon}}
}
\z}

%%for derivation examples in the derivational morphology chapter!
%syntax: \DerivExam{#1}{#2}{#3}{#4}{#5}{#6}
%#1: base, #2: base-gloss, #3: derived form, #4: derived form gloss, #5: derived form translation, #6: pit-recording, #7: utterance number
\newcommand{\DW}{28mm}%for following three commands, to align arrows throughout
%%%%OLD:
%%%\newcommand{\DerivExam}[7]{\Tn{\begin{tabular}[t]{p{\DW}cl}\It{#1}&\ARROW&\It{#3}\\#2&&#4\\\end{tabular}\hfill\pbox{.3\textwidth}{\hfill‘#5’\\\hbox{}\hfill\hyperlink{pit#6}{{\small\textnormal[pit#6.#7]}}}
%%%%\index{Z\Red{rec}!\Red{pit#6}}\index{Z\Red{utt}!\Red{pit#6.#7}}
%%%}}
%NEW:
\newcommand{\DerivExam}[7]{\Tn{
\begin{tabular}[t]{p{\DW}x{5mm}l}\It{#1}&\ARROW&\It{#3}\\\end{tabular}\hfill‘#5’\\
\hspace{1mm}\begin{tabular}[t]{p{\DW}x{5mm}l}#2&&#4\\\end{tabular}\hfill\hyperlink{pit#6}{{\small\textnormal[pit#6.#7]}}
%\index{Z\Red{rec}!\Red{pit#6}}\index{Z\Red{utt}!\Red{pit#6.#7}}
}}
%%same as above, but supress any reference to a specific utterance
\newcommand{\DerivExamX}[7]{\Tn{
\begin{tabular}[t]{p{\DW}x{5mm}l}\It{#1}&\ARROW&\It{#3}\\\end{tabular}\hfill‘#5’\\
\hspace{1mm}\begin{tabular}[t]{p{\DW}x{5mm}l}#2&&#4\\\end{tabular}\hfill\hyperlink{pit#6}{{\small\textnormal[pit#6]\It{e}}}
%\index{Z\Red{rec}!\Red{pit#6}}\index{Z\Red{utt}!\Red{pit#6.#7}}
}}
\newcommand{\DerivExamWL}[6]{\Tn{
\begin{tabular}[t]{p{\DW}x{5mm}l}\It{#1}&\ARROW&\It{#3}\\\end{tabular}\hfill‘#5’\\
\hspace{1mm}\begin{tabular}[t]{p{\DW}x{5mm}l}#2&&#4\\\end{tabular}\hfill\hyperlink{explExs}{{\small\textnormal[#6]}}
%\index{Z\Red{wl}!\Red{#6}}
}}


%formatting of corpus source information (after \transl in gb4e-environments):
\newcommand{\Corpus}[2]{\hspace*{1pt}\hfill{\small\mbox{\hyperlink{pit#1}{\Tn{[pit#1.#2]}}}}%\index{Z\Red{rec}!\Red{pit#1}}\index{Z\Red{utt}!\Red{pit#1.#2}}
}%
\newcommand{\CorpusE}[2]{\hspace*{1pt}\hfill{\small\mbox{\hyperlink{pit#1}{\Tn{[pit#1.#2]}}\It{e}}}%\index{Z\Red{rec}!\Red{pit#1}}\index{Z\Red{utt}!\Red{pit#1.#2}\Blue{-E}}
}%
%%as above, but necessary for recording names which include an underline because the first variable in \href understands _ but the second variable requires \_
\newcommand{\CorpusLink}[3]{\hspace*{1pt}\hfill{\small\mbox{\hyperlink{pit#1}{\Tn{[pit#2.#3]}}}}%\index{Z\Red{rec}!\Red{pit#2}}\index{Z\Red{utt}!\Red{pit#2.#3}}
}%
%%as above, but for newer recordings which begin with sje20 instead of pit
\newcommand{\CorpusSJE}[2]{\hspace*{1pt}\hfill{\small\mbox{\hyperlink{sje20#1}{\Tn{[sje20#1.#2]}}}}%\index{Z\Red{rec}!\Red{sje20#1}}\index{Z\Red{utt}!\Red{sje20#1.#2}}
}%
\newcommand{\CorpusSJEE}[2]{\hspace*{1pt}\hfill{\small\mbox{\hyperlink{sje20#1}{\Tn{[sje20#1.#2]}}\It{e}}}%\index{Z\Red{rec}!\Red{sje20#1}}\index{Z\Red{utt}!\Red{sje20#1.#2}\Blue{-E}}
}%











%%hyphenation points for line breaks
%%add to TeX file before \begin{document} with:
%%%%hyphenation points for line breaks
%%add to TeX file before \begin{document} with:
%%%%hyphenation points for line breaks
%%add to TeX file before \begin{document} with:
%%\include{hyphenationSDL}
\hyphenation{
ab-es-sive
affri-ca-te
affri-ca-tes
Ahka-javv-re
al-ve-o-lar
com-ple-ments
%check this:
de-cad-es
fri-ca-tive
fri-ca-tives
gemi-nate
gemi-nates
gra-pheme
gra-phemes
ho-mo-pho-nous
ho-mor-ga-nic
mor-pho-syn-tac-tic
or-tho-gra-phic
pho-neme
pho-ne-mes
phra-ses
post-po-si-tion
post-po-si-tion-al
pre-as-pi-ra-te
pre-as-pi-ra-ted
pre-as-pi-ra-tion
seg-ment
un-voiced
wor-king-ver-sion
}
\hyphenation{
ab-es-sive
affri-ca-te
affri-ca-tes
Ahka-javv-re
al-ve-o-lar
com-ple-ments
%check this:
de-cad-es
fri-ca-tive
fri-ca-tives
gemi-nate
gemi-nates
gra-pheme
gra-phemes
ho-mo-pho-nous
ho-mor-ga-nic
mor-pho-syn-tac-tic
or-tho-gra-phic
pho-neme
pho-ne-mes
phra-ses
post-po-si-tion
post-po-si-tion-al
pre-as-pi-ra-te
pre-as-pi-ra-ted
pre-as-pi-ra-tion
seg-ment
un-voiced
wor-king-ver-sion
}
\hyphenation{
ab-es-sive
affri-ca-te
affri-ca-tes
Ahka-javv-re
al-ve-o-lar
com-ple-ments
%check this:
de-cad-es
fri-ca-tive
fri-ca-tives
gemi-nate
gemi-nates
gra-pheme
gra-phemes
ho-mo-pho-nous
ho-mor-ga-nic
mor-pho-syn-tac-tic
or-tho-gra-phic
pho-neme
pho-ne-mes
phra-ses
post-po-si-tion
post-po-si-tion-al
pre-as-pi-ra-te
pre-as-pi-ra-ted
pre-as-pi-ra-tion
seg-ment
un-voiced
wor-king-ver-sion
}\begin{document}\tableofcontents\clearpage

%%%%%%%%%%%%%%%%%%%%%%%%%%%%%%%%% ALL THE ABOVE TO BE COMMENTED OUT FOR COMPLETE DOCUMENT! %%%%%%%%%%%


%%%%%%%%%%%%%%%%%%%%%%%%%%%%%% C H A P T E R %%%%%%%%%%%%%%%%%%%%%%%%%%%%%
%%%%%%%%%%%%%%%%%%%%%%%%%%%%%% C H A P T E R %%%%%%%%%%%%%%%%%%%%%%%%%%%%%
\chapter{Morphological patterns and word classes}\label{morphWordClassCh}
Morphology plays an essential role in \PS, a highly synthetic language. Based on morphological patterning and in addition to syntactic criteria, seven word classes can be posited. \SEC\ref{morphology} first provides an introduction to morphological phenomena in \PS, before \SEC\ref{introWordForms} summarizes the word classes.  


\section{Overview of morphology}\label{morphology}
A number of inflectional categories exist in \PS; Table \vref{summaryMorphCats} provides a summary of inflectional categories relevant for each word class or sub-category. 
Derivational morphology is commonly used to create nouns, verbs, and, to a lesser extent, adjectives and adverbs. %\footnote{\Red{Historically, at least attributive adjectives were probably derived as well, but synchronically, this is no longer the case; cf. \SEC\ref{notePredNounsAdjs}.}} 
Both derivational\is{derivation} and inflectional\is{inflection} morphology manifest themselves linearly (by suffixing) or non-linearly, via consonant gradation, umlaut and/or vowel harmony. More often than not, linear and non-linear morphological phenomena are combined. 
\begin{table}[h]\centering
\caption{Inflectional categories for pertinent word classes and sub-categories of word classes}\label{summaryMorphCats}
\begin{tabular}{ll}\mytoprule
{word class/sub-category}	&{inflectional categories}	\\\hline
{verbs}	&	\\%\hline
\BULLET\ finite forms	&person, tense, mood	\\%\hline
\BULLET\ non-finite forms	&aspect, connegation, etc.	\\%\hline
{nominals}	&	\\%\hline
\BULLET\ nouns			&case, number	\\%\hline
\BULLET\ dem., indef., inter. pronouns	&case, number	\\%\hline
\BULLET\ pers., refl., rel. pronouns			&case, number, person	\\%\hline
{adjectivals}	&	\\%\hline
\BULLET\ attributive adjectives			&comparative, superlative	\\%\hline
\BULLET\ predicative adjectives		&comparative, superlative, number	\\\mybottomrule
\end{tabular}
\end{table}

The present section only provides an overview of these various morphological phenomena, and is divided into \SEC\ref{linearMorphology} on linear morphology and \SEC\ref{morphophonology} on non-linear morphological processes. %(consonant gradation, umlaut and morphologically-triggered vowel harmony).
Because morphological behavior varies between the word classes, it is described in more detail individually in the relevant word class chapters.  


\subsection{Linear morphology}\label{linearMorphology}\is{linear morphology|(}
Concerning linearly separable morphology, \PS\ is an exclusively suffixing language. Both inflectional and derivational suffixes exist. %, and can be ascertained on nouns, pronouns, adjectives, adverbs, verbs. 
The general linear morphological structure of \PS\ words has derivational suffixes attaching to a root before inflectional suffixes occur on the resulting stem, as illustrated by %\marginpar{make sure Figure \ref{linearMorphStructure} and Table \ref{summaryMorphCats} aren’t directly above one another}
Figure \vref{linearMorphStructure}.
\begin{figure}[h]\centering
[lexical root \PLUS\ derivational morphemes \PLUS\ inflectional morphemes]\sub{word}
\caption{General structure of linear morphology composing \PS\ words}\label{linearMorphStructure}
\end{figure}
\is{linear morphology|)}
%\subsection{Suffixes}\label{suffixes}
%\subsubsection{inflectional}\label{inflectionalSuffixes}
%\subsubsection{derivational}\label{derivationalSuffixes}
%


\subsection{Non-linear morphology (morphophonology)}\is{non-linear morphology|(}
There are three ways %\marginpar{mention stem extension for verb morphology?} %eigentlich ja.
in which non-linear morphology can be expressed in \PS:
\begin{itemize}
\item{stem consonant alternations (consonant gradation)}
\item{stem vowel alternations in V1 position (umlaut)}
\item{regressive vowel harmony in V1 and V2 vowels}
\end{itemize}
These are triggered by a word’s position %paradigmatic relationship 
within an inflectional paradigm, or in derivation. All inflectional non-linear morphology is restricted to the final foot\is{foot} of a given word, while derivational non-linear morphology can also occur in a non-ultimate foot. Non-linear processes may apply simultaneously. The following sections describe these phenomena in more detail: \SEC\ref{Cgrad} for consonant gradation, \SEC\ref{umlaut} for umlaut and \SEC\ref{VH} for vowel harmony.\footnote{Cf. \citet{KorhonenM1969} for historical explanations of these morphophonological processes in Saami.} 


\subsubsection{Consonant gradation}\label{Cgrad}\is{consonant gradation|(}
The term \It{consonant gradation}\footnote{This term is quite common in Saami linguistics, e.g.,~in \citet{Feist2010}, although it is also known as ‘grade alternation’, e.g.,~in \citet{Sammallahti1998}. As much of the literature on Saami linguistics is in languages other than English, it may be useful to provide some translations of the term ‘consonant gradation’: German: \It{Stufenwechsel}, Swedish: \It{stadieväxling}, Finnish: \It{astevaihtelu}, Hungarian: \It{fokváltakozás} and Russian: \It{чередование ступеней}.} 
refers to regular alternations of the consonant phonemes in the consonant center of the final foot of a word.\footnote{Cf. \SEC\ref{prosodicDomains} on prosodic positions, including the foot and the consonant center.} 
%Many, but not all, lexical items are subject to consonant gradation. 

All consonant phonemes in the consonant center are included in the present classification of such alternations.
These alternations come in pairs of stem allomorphs that differ quantitatively and/or qualitatively. 
Alternations can be between a preaspirated\is{aspiration!preaspiration} and the corresponding non-aspirated consonant \mbox{(ʰx\Div x)},  
 a geminate\is{geminate} consonant and the corresponding singleton consonant \mbox{(xː\Div x)}, 
 a geminate\PLUS singleton and the corresponding singleton\PLUS singleton \mbox{(xːy\Div xy)}, 
 two singletons and only the latter singleton \mbox{(xy\Div y)}, 
and  three singletons and only the initial and final singleton \mbox{(xyz\Div xz)}.\footnote{There are certainly other ways to classify these alternations in the consonant center as well. For instance, one could disregard any consonant phonemes that are present in both alternations (then \mbox{xː\Div x} and \mbox{xːy\Div xy} would be the same type), or consider the xː\Div x and ʰxː\Div x patterns in Table \ref{CgradPatternSummary} to be the same type as simply a case of alternating consonant phonemes. However, the patterns in the present classification demonstrate the regularity of patterning between phonological features (such as geminate vs. singleton).}  

These patterns and the attested alternations are provided in Table \vref{CgradPatternSummary}. 
\begin{table}[h]\centering
\caption{Consonant center gradation patterns}\label{CgradPatternSummary}
\begin{tabular}{c c c p{240pt}}\mytoprule
%\MC{3}{c}{{pattern}}&\\
strong&\DARROW &weak	&{attested alternations}\\\hline
%\MC{4}{l}{\It{qualitative differences}}\\
ʰx	&\DARROW &x		
	& ʰp\Div p, ʰt\Div t, ʰk\Div k, ʰʦ\Div ʦ, ʰʧ\Div ʧ \\
%\MC{4}{l}{\It{quantitative differences}}\\
%\MC{4}{l}{geminate\Div singleton}\\
xː	&\DARROW &x
	& fː\Div f, vː\Div v, sː\Div s, ʃː\Div ʃ, mː\Div m, nː\Div n, ɲː\Div ɲ, rː\Div r, lː\Div l, jː\Div j \\
&&%ʰxː	&\Div &ʰx
	& ʰpː\Div ʰp, ʰtː\Div ʰt, ʰkː\Div ʰk, ʰʧː\Div ʰʧ, ʰʦː\Div ʰʦ \\
%	& !pː\Div p, ʰpː\Div ʰp, !tː\Div t, ʰtː\Div ʰt, !kː\Div k, ʰkː\Div ʰk, !ʧː\Div ʧ, ʰʧː\Div ʰʧ, !ʦː\Div ʦ, ʰʦː\Div ʰʦ, fː\Div f, vː\Div v, sː\Div s, ʃː\Div ʃ, mː\Div m, nː\Div n, ɲː\Div ɲ, rː\Div r, lː\Div l, jː\Div j \\
%\MC{4}{l}{preaspirated singleton\Div singleton}\\
%\MC{4}{l}{geminate+singleton\Div singleton+singleton}\\
xːy	&\DARROW & xy
	& pːt\Div pt, pːk\Div pk, pːʦ\Div pʦ, pːʧ\Div pʧ, pːs\Div ps, pːm\Div pm, pːn\Div pn, pːɲ\Div pɲ, pːr\Div pr, pːl\Div pl, pːj\Div pj, tːk\Div tk, tːm\Div tm, tːn\Div tn, tːɲ\Div tɲ, kːt\Div kt, kːʧ\Div kʧ, kːʦ\Div kʦ, kːs\Div ks, kːʃ\Div kʃ, kːŋ\Div kŋ, kːl\Div kl, \\
	&&& fːt\Div ft, fːn\Div n, vːt\Div vt, vːk\Div vk, vːʦ\Div vʦ, vːʧ\Div vʧ, vːs\Div vs, vːʃ\Div vʃ, vːr\Div vr, vːl\Div vl, vːj\Div vj, sːp\Div sp, sːt\Div st, sːk\Div sk, sːm\Div sm, sːn\Div sn, ʃːk\Div ʃk, \\
	&&& mːs\Div ms, mːʃ\Div mʃ, nːt\Div nt, ŋːk\Div ŋk \\%, mːʰp\Div mʰp, mːʰk\Div mʰk, nːʰt\Div nʰt, nːʰʦ\Div nʰʦ, ŋːʰk\Div ŋʰk, \\
	&&&rːp\Div rp, rːt\Div rt, rːk\Div rk, rːʦ\Div rʦ, rːf\Div rf, rːs\Div rs, rːʃ\Div rʃ, rːf\Div rf, rːv\Div rv, rːj\Div rj, lːp\Div lp, lːt\Div lt, lːk\Div lk, lːf\Div lf, lːv\Div lv, lːs\Div ls, lːj\Div lj, jːp\Div jp, jːt\Div jt, jːk\Div jk, jːs\Div js, jːf\Div jf, jːv\Div jv, jːr\Div jr, jːl\Div jl \\%, rːʰp\Div rʰp, rːʰt\Div rʰt, rːʰk\Div rʰk, rːʰʧ\Div rʰʧ, rːʰʦ\Div rʰʦ, lːʰp\Div lʰp, lːʰt\Div lʰt, lːʰk\Div lʰk, lːʰʦ\Div lʰʦ, lːʰʧ\Div lʰʧ, jːʰt\Div jʰt, jːʰk\Div jʰk, jːʰʧ\Div jʰʧ, jːʰʦ\Div jʰʦ\\%
%\MC{4}{l}{geminate+preaspirated singleton\Div preaspirated singleton+singleton}\\
&&%xːʰy	&\Div & xʰy	
	& mːʰp\Div mʰp, mːʰk\Div mʰk, nːʰt\Div nʰt, nːʰʦ\Div nʰʦ, ŋːʰk\Div ŋʰk, \\
	&&&rːʰp\Div rʰp, rːʰt\Div rʰt, rːʰk\Div rʰk, rːʰʧ\Div rʰʧ, rːʰʦ\Div rʰʦ, lːʰp\Div lʰp, lːʰt\Div lʰt, lːʰk\Div lʰk, lːʰʦ\Div lʰʦ, lːʰʧ\Div lʰʧ, jːʰt\Div jʰt, jːʰk\Div jʰk, jːʰʧ\Div jʰʧ, jːʰʦ\Div jʰʦ \\%
%xy	&\Div &x		&  \\%ʰpp-ʰp (assumes that there is no geminate ʰpː but instead ʰp+p)%ADAPTED ANALYSISː ʰpː-ʰp
%\MC{4}{l}{singleton12\Div singleton2}\\
xy	&\DARROW &y		
	& pm\Div m, pɲ\Div ɲ, tn\Div n, tɲ\Div ɲ, tj\Div j, kŋ\Div ŋ \\
%\MC{4}{l}{singleton123\Div singleton13}\\
xyz	&\DARROW & xz	
	& vtn\Div vn, vtɲ\Div vɲ, rpm\Div rm, rtn\Div rn, rtj\Div rj, lpm\Div lm, ltn\Div ln, ltɲ\Div lɲ, jpm\Div jm, jtn\Div jn \\\mybottomrule
\end{tabular}
\end{table}
The term \It{strong grade}\is{consonant gradation!strong grade} (abbreviated ‘str’) is used to refer to the form with preaspiration\is{aspiration!preaspiration}, a geminate\is{geminate} or more consonant segments than the corresponding form. Likewise, the term \It{weak grade}\is{consonant gradation!weak grade} (abbreviated ‘wk’) refers to the form lacking a preaspirated or geminate consonant, or having fewer consonant segments, respectively. 
To facilitate reading Table \ref{CgradPatternSummary}, the attested alternations for each pattern are organized by alternations with a non-aspirated element first, followed by patterns with at least one preaspirated element. Furthermore, the individual lists of alternations are organized by mode of articulation and by the order set forth in the consonant phoneme inventory in Table \vref{Cphonemes}. 
Some examples illustrating this can be found below. 


The minimal pair in \REF{CgradEx3} shows a consonant gradation alternation \mbox{/ʰp\Div p/}, which corresponds to the pattern \mbox{ʰx\Div x}. %in Table \ref{CgradPatternSummary}. 
\ea\label{CgradEx3}%várre mountain\BS\Sc{nom.sg} / váre mountain\BS\Sc{nom.pl} 
\Tn{\begin{tabular}{c c}
/dɔ\Bf{ʰp}e/		&/dɔ\Bf{p}e/\\
\It{dåhpe}		&\It{dåbe}\\
house\BS\Sc{nom.sg}	&house\BS\Sc{gen.sg}	\\
\end{tabular}
\hfill\hyperlink{pit100324}{{\small [pit100324]}}}
\z

The minimal pairs in \REF{CgradEx1a} and \REF{CgradEx1b} are examples of consonant gradation patterns which differ in a geminate\Div singleton alternation. The consonant gradation alternations illustrated are \mbox{/vː\Div v/} and \mbox{/rːk\Div rk/}, respectively, and correspond to the patterns \mbox{xː\Div x} and \mbox{xːy\Div xy}. % in Table \ref{CgradPatternSummary}. 
%In the first example, a geminate /vː/ in the consonant center of the \Sc{3sg.prs} form alternates with a singleton /v/ in the \Sc{2sg.prs} form. In the latter example, geminate\PLUS singleton /rːk/ in \Sc{nom.sg} alternates with two singeltons /rk/ in \Sc{nom.pl}. 
\ea\label{CgradEx1a}%várre mountain\BS\Sc{nom.sg} / váre mountain\BS\Sc{nom.pl} 
\Tn{\begin{tabular}{c c}
/saː\Bf{vː}a/		&/saː\Bf{v}a/\\
\It{sávva}		&\It{sáva}\\
wish\BS\Sc{3sg.prs}	&wish\BS\Sc{2sg.prs}\\
\end{tabular}
\hfill\hyperlink{pit100323a}{{\small [pit100323a]}}}
\z
\ea\label{CgradEx1b}%várre mountain\BS\Sc{nom.sg} / váre mountain\BS\Sc{nom.pl} 
\Tn{\begin{tabular}{c c}
/pɛ\Bf{rːk}o/	&/pe\Bf{rk}o/\\
\It{bärrgo}		&\It{biergo}\\
meat\BS\Sc{nom.sg}	&meat\BS\Sc{nom.pl}\\
\end{tabular}
\hfill\hyperlink{pit090926}{{\small [pit090926]}}}
\z

The minimal pairs in \REF{CgradEx2a} and \REF{CgradEx2b} are examples of consonant gradation patterns in which a phoneme present in the first form is absent in the second form. The consonant gradation alternations illustrated here are \mbox{/tn\Div n/} and \mbox{/jpm\Div jm/}, respectively, and correspond to the patterns \mbox{xy\Div y} and \mbox{xyz\Div xz}. %in Table \ref{CgradPatternSummary}. 
%Finally, the minimal pair in \REF{CgradEx3} is an example of a consonant gradation pattern that alternates in both quantity and quality, as it alternates between a consonant center consisting of the three segments /jpm/ in the \Sc{nom.sg} form, and a consonant center consisting of the two segments /jm/ in the \Sc{nom.pl} form. 
%, as it alternates between the segment /ʰp/ in the consonant center of the \Sc{nom.sg} form, and the segment /p/ in the \Sc{gen.sg} form.
\ea\label{CgradEx2a}%várre mountain\BS\Sc{nom.sg} / váre mountain\BS\Sc{nom.pl} 
\Tn{\begin{tabular}{c c}
/a\Bf{tn}a/		&/a\Bf{n}a/\\
\It{adna}		&\It{ana}\\
have\BS\Sc{3sg.prs}	&have\BS\Sc{2sg.prs}	\\
\end{tabular}
\hfill\hyperlink{pit101208}{{\small [pit101208]}}}
\z
\ea\label{CgradEx2b}%várre mountain\BS\Sc{nom.sg} / váre mountain\BS\Sc{nom.pl} 
\Tn{\begin{tabular}{c c}
/vaː\Bf{jpm}o/		&/vaː\Bf{jm}o/\\
\It{vájbmo}		&\It{vájmo}\\
heart\BS\Sc{nom.sg}	&heart\BS\Sc{nom.pl}	\\
\end{tabular}
\hfill\hyperlink{pit110413a}{{\small [pit110413a]}}}
\z

As may be inferred from the examples above, paradigmatic alternations between \mbox{\Sc{nom.sg}} and \mbox{\Sc{nom.pl}} forms for nouns, or between \Sc{2sg.prs} and \Sc{3sg.prs} forms for verbs are often a good source of minimal pairs concerning consonant gradation alternations, and are a useful way to determine consonant gradation patterns. 

Note that the geminate\is{geminate} plosives\is{plosive} and affricates\is{affricate} \mbox{/pː\,tː\,kː\,ʦː\,ʧː/} are lacking in Table \vref{CgradPatternSummary} for the pattern \mbox{xː\Div x}, although alternations such as \mbox{pː\Div p} could be expected. However, due to a lack of sufficient data and some conflicting data in the corpus, it is not entirely clear what the current status is for consonant gradation in words with a consonant center consisting solely of a geminate plosive or affricate. The fact that \PS\ lacks consonant gradation in a limited number of contexts is one of the main differences to Lule Saami to the north, which does not lack consonant gradation, and Ume Saami to the south, which features consonant gradation even less frequently \citep[cf.][21-23]{Sammallahti1998}. 
The example in \REF{CgradEx4a} illustrates a word clearly lacking consonant gradation in the corpus data, here with the geminate velar plosive /kː/. 
\ea\label{CgradEx4a}%
\Tn{\begin{tabular}{c c c}
/vaː\Bf{kː}e/	&/vaː\Bf{kː}e/	&(*/vaː\Bf{k}e/)\\
\It{vágge}		&\It{vágge}&\\
valley\BS\Sc{nom.sg}	& valley\BS\Sc{nom.pl}	&	\\%
\end{tabular}
\hfill\hyperlink{pit110522}{{\small [pit110522]}}}
\z

Corpus data also indicate that variation within the \PS\ area complicates things. For instance, the adjective \It{tjábbe} ‘beautiful’ undergoes consonant gradation in the speech\is{dialect variation} of speakers from the northern parts of Arjeplog, but does not for southern speakers, as illustrated in \REF{CgradEx4b}. For northern speakers, the gradation is realized as an alternation in voicing, and not length. 
\ea\label{CgradEx4b}%
\Tn{\begin{tabular}{c c c}
			&southern	&northern	\\
/ʧaː\Bf{pː}a/	&/ʧaː\Bf{pː}e/	&/ʧaː\Bf{bː}e/\\
\It{tjábba}		&\MC{2}{c}{\It{tjábbe}}\\
beautiful\BS\Sc{attr}	&\MC{2}{c}{beautiful\BS\Sc{pred}}\\
\end{tabular}
\hfill\hyperlink{pit110522}{{\small [pit110522}}, \hyperlink{sje20131017}{{\small sje20131017]}
}}
\z
This being the case, some speakers from farther north may have further voiced plosive phonemes \mbox{/bː dː gː/} that only occur in consonant gradation alternations with the corresponding unvoiced phonemes \mbox{/pː tː kː/}. However, it is not clear based on the corpus data how widespread this feature is, or if it also affects geminate affricates. 

Finally, it is not clear from the corpus data what the status of the phonological contexts lacking consonant gradation mentioned in \citet[21]{Sammallahti1998} (working from a historical perspective and with older data) is. Further research is needed to complete the picture, and %, although due to the current state of the \PS\ language \citep[cf.][]{ValijarviWilbur2011}, gathering such data is becoming more and more difficult; 
variation within \PS\ and possible effects of language attrition should also be taken into consideration.\is{consonant gradation|)}



\subsubsection{Umlaut}\label{umlaut}\is{umlaut|(}
The term \It{umlaut} refers to regular allomorphic alternations of the vowels in the V1 position of a stem.\footnote{Cf. \SEC\ref{prosodicDomains} on prosodic positions, including V1 position.} 
The two umlaut patterns attested in the corpus are listed in Table \vref{umlautPatternSummary}. %\footnote{Due to the orthography being more phonetic than phonemic, particularly concerning umlaut, phonemic representations in IPA are used in Table \vref{umlautPatternSummary}. However, these umlaut alternations are typically represented orthographically by the graphemes \It{ä}\TILDE\It{ie} and \It{ua}\TILDE\It{uo}, respectively. Cf. \SEC\ref{orthography} on the orthographic representation of \PS.} %Generally speaking, in V1 position, /u͡a/ is spelled \It{u͡a}, and its allophone [uɛ] is spelled \It{uä}, while /o/ is spelled \It{uo}, but /e/ is spelled \It{e} or \It{ie}.} 
%Some examples to help illustrate this can be found below. 
\begin{table}[h]\centering
\caption{The two attested umlaut patterns}\label{umlautPatternSummary}
\begin{tabular}{ccccccc}\mytoprule
\MC{3}{c}{{IPA}}&	&\MC{3}{c}{{orthography}}\\\hline
ɛ &\DARROW& e &&\It{ä} &\DARROW& \It{ie} \\
u͡a &\DARROW& o &&\It{ua}/\It{uä} &\DARROW& \It{uo} \\\mybottomrule\end{tabular}
%\newcommand{\wz}{28pt}%only for spacing in the following table
%\begin{tabular}{|ccc|ccc|}\hline
%\MC{3}{|c}{\It{IPA}}&\MC{3}{|c|}{\It{orthographic}}\\\mybottomrule
%ɛ &\ARROW& e	&ä &\ARROW& ie\\
%u͡a &\ARROW& o	&ua/uä &\ARROW& uo	\\\hline
%\begin{tabular}{rx{\wz}p{12pt}x{\wz}x{\wz}p{12pt}x{\wz}p{30pt}}%\cline{2-7}
%		&\MC{6}{c}{\It{pattern}}&\\
%		&\MC{3}{c}{\It{A}}		&\MC{3}{c}{\It{B}}	&\\\cline{2-7}
%\It{phonemic}	&	ɛ &\MR{2}{*}{\DARROW}& e	& u͡a &\MR{2}{*}{\DARROW}& o	&\\%\cline{2-7}
%\It{orthographic}&	ä &			& ie	& ua/uä && uo &\\\cline{2-7}
%\end{tabular}
%\begin{tabular}{|c| ccc |}\hline
%A	&ɛ&:&e\\\hline
%B	&u͡a&:&o	\\\hline
%%C-grad&str&:&wk\\\hline
%\end{tabular}
\end{table}

These umlaut alternations are qualitative and not quantitative. %The first one consists of a pair, and the second a triplet. 
These alternations are not triggered by the phonological environment, but instead morphologically. 
The allomorph /ɛ/ in the first pattern is found in the same paradigmatic slots for each inflectional class as /u͡a/\footnote{Note that /u͡a/ has an allomorph [u͡ɛ] triggered by purely phonological vowel harmony; cf. \SEC\ref{Vua}.} 
in the second pattern, just as the allomorphs /e/ and /o/ also correspond to the same paradigmatic slots. 
%Paradigm slots featuring Inflectional forms with /ɛ/ (in umlaut pattern A) correspond to inflectional forms with both /uɛ/ and /ua/ in umlaut pattern B, while forms with /e/ from pattern A and forms with /uo/ from pattern B always correspond across paradigms.
Word forms for \It{bägge} ‘wind’ in \REF{umlautEx1} and for \It{buälldet} ‘burn’ in \REF{umlautEx2} provide examples of the two umlaut patterns.
%In \REF{umlautEx1} an example of umlaut pattern A is given. It alternates between /ɛ/ in the V1 vowel of the \Sc{nom.sg} form, and /e/ in the \Sc{nom.pl} form.
\ea\label{umlautEx1}%
\Tn{\begin{tabular}{c c}
/b\Bf{ɛ}gːa/		&/b\Bf{e}gːa/\\
\It{bägga}		&\It{biegga}\\
wind\BS\Sc{nom.sg}	&wind\BS\Sc{nom.pl}\\
\end{tabular}
\hfill\hyperlink{pit080621}{{\small [pit080621]}}}
\z
%
%%, as it alternates between /u͡a/ in the V1 vowel of the \Sc{nom.sg} form, and /o/ in the \Sc{nom.pl} form. %Note that the example in \REF{umlautEx2} reflects a more southern \PS\ dialect, and that the more northern dialects usually have the allophone [u͡ɛ] \It{uä} here. %, although the exact relationship between the dialects must be left to future study. 
\ea\label{umlautEx2}%
\Tn{\begin{tabular}{c c}
/p\Bf{u͡a}lːta/		&/p\Bf{o}lta/\\
\It{buallda}		&\It{buolda}\\
ignite\BS\Sc{3sg.prs}	&ignite\BS\Sc{2sg.prs}\\
%/lu͡akːta/		&/lokta/\\
%\It{luakkta		&\It{luokta\\
%bay\BS\Sc{nom.sg}	&bay\BS\Sc{nom.pl}\\
%\begin{tabular}{c c c}
%/lu͡ɛkːta/		&/lu͡ak:taj/ &/lokːta/\\
%\It{luäkkta		&\It{luakkta-j	&\It{luokta\\
%bay\BS\Sc{nom.sg}	&bay-\Sc{ill.sg}&bay\BS\Sc{nom.pl}\\
\end{tabular}
\hfill\hyperlink{pit101208}{{\small [pit101208]}}}
\z
For lexemes subject to consonant gradation, forms featuring /ɛ/ or /u͡a/ are typically in the strong grade\is{consonant gradation!strong grade}, while forms with /e/ or /o/ are normally in the weak grade\is{consonant gradation!weak grade}; cf. the word forms in example \REF{umlautEx2}. \is{umlaut|)}


\subsubsection{Vowel harmony}\label{VH}\is{vowel harmony|(}
The term \It{vowel harmony} (abbrviated ‘VH’) here refers to non-adjacent regressive phonological assimilation\is{assimilation} concerning the place of articulation of the V1 vowel of a stem in the context of certain V2 vowels.\footnote{Cf. \SEC\ref{prosodicDomains} on prosodic positions, including V1 and V2 positions.} 
Specifically, mid-high or high front vowels in V2 position in specific paradigmatic slots trigger raising of the vowel in V1 position. Because the paradigmatic slots that trigger vowel harmony differ between word classes and inflectional classes, and do not apply across the board, vowel harmony is not a purely phonological process, but morphophonological. Furthermore, the results of harmony on the same underlying vowel are inconsistent, and may be due to a word’s membership is certain morphological classes concerning vowel harmony. However, future research must be conducted to come to a more thorough conclusion on this. 

Verbs\is{verb} and nouns\is{noun} can be subject to vowel harmony, but the assimilation patterns vary both between these word classes and within them. 
Table \vref{VHsummaryTable} summarizes the various patterns and the word classes that they are attested in based on the current corpus. 
It is possible that, as a result of more documentation and study, this table may need to be updated. 
\begin{table}[h]\centering
\caption{Vowel harmony assimilation patterns and the word classes these are found in}\label{VHsummaryTable}
\begin{tabular}{x{30pt}cx{30pt}cc}\mytoprule
\MC{3}{c}{{VH pattern}}	&{nouns}	&{verbs}		\\\hline	
%\It{ &&\It{VH		&\It{nouns	&\It{verbs	\\\mybottomrule	
%				&&&\It{nouns	&\It{verbs	\\\mybottomrule
ɛ/e&\ARROW&i			&\CH	&\CH	\\
u͡a/o&\ARROW&u		&\CH	&\CH	\\
aː&\ARROW&ɛ			&\CH	&\CH	\\
ɔ&\ARROW&u			&\CH	&\CH	\\
a&\ARROW&ɛ			&\CH	&	\\
a&\ARROW&i			&		&\CH	\\
aː&\ARROW&i			&		&\CH	\\
a&\ARROW&e			&		&\CH	\\\mybottomrule
\end{tabular}
\end{table}

The morphological categories that trigger vowel harmony also vary. This is the case not only between nouns and verbs (as these have different inflectional categories), but also between inflectional classes for verbs. These categories are presented in Table \vref{VHtriggerSummary}. %on page \pageref{VHtriggerSummary}. 
\begin{table}[h]\centering
\caption{Paradigm slots that trigger vowel harmony}\label{VHtriggerSummary}
\begin{tabular}{ccp{178pt}}\mytoprule
{word class}&{inflectional class}&{forms triggering VH}	\\\hline
nouns	&class Ie			&\Sc{gen.pl, acc.pl, ill.pl, iness.pl,}	\\%\hline
		&				&\Sc{elat.pl, com.sg, com.pl} \\%\hline
verbs	&class II			&\Sc{1du.prs, 3pl.prs, 1sg.pst, 2sg.pst,} 		\\%\cline{2-3}
		&				&\Sc{3pl.pst, pl.imp} \\%\cline{2-3}
		&class III			&\Sc{1du.prs, 3pl.prs, 1sg.pst, 2sg.pst,}	\\
		&				&\Sc{3sg.pst, 1du.pst, 2du.pst, 3du.pst,} \\
		&				&\Sc{1pl.pst, 2pl.pst, 3pl.pst, pl.imp} 	\\\mybottomrule
%\MR{2}{*}{nouns}&\MR{2}{*}{class Ie}&\Sc{gen.pl, acc.pl, ill.pl, iness.pl,}	\\%\hline
%		&				&\Sc{elat.pl, com.sg, com.pl} \\\hline
%\MR{5}{*}{verbs}&\MR{2}{*}{class II}&\Sc{1du.prs, 3pl.prs, 1sg.pst, 2sg.pst,} 		\\%\cline{2-3}
%		&				&\Sc{3pl.pst, pl.imp} \\\cline{2-3}
%		&\MR{3}{*}{class III}	&\Sc{1du.prs, 3pl.prs, 1sg.pst, 2sg.pst,}	\\
%		&				&\Sc{3sg.pst, 1du.pst, 2du.pst, 3du.pst,} \\
%		&				&\Sc{1pl.pst, 2pl.pst, 3pl.pst, pl.imp} 	\\\hline
\end{tabular}
\end{table}

%\FB

Some examples for vowel harmony are provided here. 
In \REF{vhEx1}, an example is shown of vowel harmony in the Class Ie noun \It{guolle} ‘fish’, as it alternates between /o/ in the V1 vowel of the \Sc{nom.sg} form, and /u/ in the \Sc{nom.pl} form.
\ea\label{vhEx1}%
\Tn{\begin{tabular}{c c}
/k\Bf{o}le/		&/k\Bf{u}l\Bf{i}j/\\
\It{guole}		&\It{guli-j}\\
fish\BS\Sc{nom.pl}	&fish-\Sc{gen.pl}\\
\end{tabular}
\hfill\hyperlink{pit110413a}{{\small [pit110413a]}}}
\z

In \REF{vhEx2}, an example of vowel harmony in the class II verb \It{bassat} ‘wash’ is provided. Here, a vowel harmony alternation between /a/ in the V1 vowel of the \Sc{2sg.prs} form, and /i/ in the \Sc{2sg.pst} form is evident (in addition to a consonant gradation alternation). 
\ea\label{vhEx2}%
\Tn{\begin{tabular}{c c}
/p\Bf{a}sa/		&/b\Bf{i}sː\Bf{e}/\\
\It{basa}		&\It{bisse}\\
wash\BS\Sc{2sg.prs}	&wash\BS\Sc{2sg.pst}\\
\end{tabular}
\hfill\hyperlink{pit101208}{{\small [pit101208]}}}
\z

Finally, \REF{vhEx3} shows an example of vowel harmony in the class III verb \It{buälldet} ‘burn’. Here, a vowel harmony alternation between /o/ in the V1 vowel of the \Sc{2sg.prs} form, and /u/ in the \Sc{2sg.pst} form is evident (in addition to a consonant gradation alternation). 
\ea\label{vhEx3}%
\Tn{\begin{tabular}{c c}
/p\Bf{o}lta/		&/p\Bf{u}lːt\Bf{e}/\\
\It{buolda}		&\It{bullde}\\
ignite\BS\Sc{2sg.prs}	&ignite\BS\Sc{2sg.pst}\\
\end{tabular}
\hfill\hyperlink{pit101208}{{\small [pit101208]}}}
\z


See \SEC\ref{JcomponentNounSuffixes} for more details on vowel harmony in nouns, and \SEC\ref{VHPatternSectionVerbs} in verbs. Note that, for nouns, vowel harmony is also referred to as ‘\jvh’.
\is{vowel harmony|)}
\is{non-linear morphology|)}
\FB


%%%%%%% WORD CLASS OVERVIEW
%%%%%%% WORD CLASS OVERVIEW
%%%%%%% WORD CLASS OVERVIEW
%%%%%%% WORD CLASS OVERVIEW

\section{Overview of word classes}\label{introWordForms}\is{word class (overview)}
By characterizing the morphological and syntactic behavior of words in \PS, and grouping such words based on that behavior, a total of seven word classes can be distinguished. These can be divided into two general categories containing generally \It{open} word classes and \It{closed} word classes, and are listed in Table \vref{wordClassList}. 
The specific syntactic criteria and inflectional\is{inflection} categories defining these are summarized in Table \ref{wordClassSummary1} on the same page. %\footnote{The abbreviation} 
% as follows:
%\begin{itemize*}\item{open word classes:\begin{itemize*}\item{verbs}\item{nouns}\item{adjectives}\item{adverbs}\end{itemize*}}\item{closed word classes:\begin{itemize*}\item{pronouns}\item{demonstratives}\item{numerals}\item{verbal particles}\item{postpositions}\item{conjunctions}\item{interjections}%\item{}\end{itemize*}}\end{itemize*}

Some word classes consist of two or more subclasses: 
\It{nominals} refer to \It{nouns} %(including a subcategory for predicative adjectives) 
and \It{pronouns} (personal, demonstrative, reflexive, interrogative and relative), and \It{adjectivals} include both \It{adjectives} and \It{numerals}. %Sets of finite and non-finite \It{verbs} belong to the same lexeme. 
Note that pronouns and numerals are closed subclasses belonging to open classes.

This categorization is intended to provide a broad starting point for classifying \PS\ words; details for each word class can be found in the relevant chapters below. 
%Although nouns and pronouns are both considered nominals, they are dealt with in separate chapters for the sake of clarity. 
Chapter \ref{nouns} concerns the nominal subclass \It{nouns}, which provide fairly straightforward examples of the morphophonological complexities involved in \PS\ inflection and derivation, while 
the nominal subclass \It{pronouns} is dealt with in Chapter \ref{pronouns}. % for the sake of clarity. 
Chapter \ref{adjectivesIntro} then covers the adjectival subclasses \It{adjectives} and \It{numerals}. Following this, Chapter \ref{verbs} deals with \It{verbs}. Finally, the remaining small classes (\It{adverbials}, \It{adpositions}, \It{conjunctions} %, \It{particles}, 
and \It{interjections}) are covered in Chapter \ref{otherWordClasses}. 
% are sufficiently complex or not nece and traditionally considered unique word classes in other languages, they are given

\begin{table}\centering
\caption[\PS\ word classes]{\PS\ word classes and the corresponding chapter/section}\label{wordClassList}
\begin{tabular}{l l c  l c}\mytoprule
\MC{2}{l}{{open word classes}}&{Ch./Sec.}	&{closed word classes}&{Sec.}	\\\hline
\MC{2}{l}{{nominals}}&				&{adpositions} & \SEC\ref{adpositions}		\\
	&nouns	& Ch.\,\ref{nouns}			&{conjunctions} & \SEC\ref{conjunctions}\\
	&pronouns& Ch.\,\ref{pronouns}		&{interjections} 	& \SEC\ref{interjections}\\
\MC{2}{l}{{adjectivals}}&				&&\\
	&adjectives & Ch.\,\ref{adjectivesIntro}	&&\\
	&numerals & \SEC\ref{numerals}		&&\\
\MC{2}{l}{{verbs}}& Ch.\,\ref{verbs}			&&\\
\MC{2}{l}{{adverbials}}& \SEC\ref{adverbs}	&&\\\mybottomrule
%\MC{2}{l}{\Bf{particles}}& \SEC\ref{particles}		&&\\
\end{tabular}
\end{table}


\begin{table}\centering
\caption[Morphological and syntactic criteria for word classes]{Morphological and syntactic criteria for word classes}\label{wordClassSummary1}
%\begin{tabular}[l]{l l l}
\begin{tabular}[l]{l p{130pt} p{130pt}}\mytoprule
{word class}	&{inflectional categories}		&{syntactic criteria}		\\\hline
nominals		&case/number					&head of a nominal phrase	\\
adjectivals		&number (for predicate adjectives)	&head of an adjectival phrase	\\%, but can head an NP in elliptic constructions\\
verbs		&tense/mood/person/number,		&head of a verb complex	\\%
			&non-finite forms (connegation, aspect)&					\\
adverbials		& none							&head of an adverbial phrase	\\
%pronouns		&substitutes an NP							&case/number\\%part of NOUNS
%demonstratives	&specifies deictic relationship of an NP			&case/number\\%part of NOUNS
%numerals		&specifies count of an NP						& none\\%part of ADJ
%verbal particles	&complement? to VC head					& none\\%part of ADV
adpositions	& none							&head of an adpositional phrase	\\
conjunctions	& none							&connect words, phrases, clauses, texts	\\
%particles		& none							&independent words within clauses	\\
interjections	& none							&independent words at clause-level	\\\mybottomrule
\end{tabular}
%%%Other way around:
%%\begin{tabular}[l]{l p{130pt} p{130pt}}
%%\It{word class}	&\It{syntactic criteria}						&\It{inflectional categories}\\\mybottomrule
%%nominals		&head of a nominal phrase					&case/number\\\hline
%%verbs		&head of a verb complex						&tense/mood/person/number,\\%\hline
%%			&										&non-finite forms (negation, aspect)\\\hline
%%adjectivals		&head of an adjectival phrase					&number for predicate adjectives\\\hline%, but can head an NP in elliptic constructions\\\hline
%%adverbials		&head of an adverbial phrase					&-\\\hline
%%%pronouns		&substitutes an NP							&case/number\\\hline%part of NOUNS
%%%demonstratives	&specifies deictic relationship of an NP			&case/number\\\hline%part of NOUNS
%%%numerals		&specifies count of an NP						&-\\\hline%part of ADJ
%%%verbal particles	&complement? to VC head					&-\\\hline%part of ADV
%%adpositions	&head of an adpositional phrase					&-\\\hline
%%conjunctions	&connect words, phrases, clauses, texts			&-\\\hline
%%particles		&independent words within clauses				&-\\\hline
%%interjections	&independent words at clause-level				&-\\\hline
%%\end{tabular}
\end{table}








%%%%%%% THIS IS NOT USED FOR THE ENTIRE COMPILATION, but only for individual chapters!!!!

\clearpage
\addcontentsline{toc}{chapter}{Bibliography}\label{Bibliography}
\bibliography{PiteGrammarBibSDL}%for bibtex
%\printbibliography%[title=Works Cited]%%for biber!






%%%NAME INDEX doesn’t work!?!? why???
\cleardoublepage\phantomsection%this allows hyperlink in ToC to work
\addcontentsline{toc}{chapter}{Name index}
\ohead{Name index}
\printindex[aut]

\cleardoublepage\phantomsection%this allows hyperlink in ToC to work
\addcontentsline{toc}{chapter}{Language index}
\ohead{Language index}
\printindex[lan]

\cleardoublepage\phantomsection%this allows hyperlink in ToC to work
\addcontentsline{toc}{chapter}{Subject index}
\ohead{Subject index}
\printindex


\end{document}
