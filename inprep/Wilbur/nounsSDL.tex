%\documentclass[ number=5
			   ,series=sidl
			   ,isbn=xxx-x-xxxxxx-xx-x
			   ,url=http://langsci-press.org/catalog/book/17
			   ,output=long   % long|short|inprep              
			   %,blackandwhite
			   %,smallfont
			   ,draftmode   
			  ]{LSP/langsci}                          

\usepackage{LSP/lsp-styles/lsp-gb4e}		% verhindert Komma bei mehrfachen Fußnoten?
\usepackage{LSP/lsp-styles/avm}
\avmfont{\sc} 
\avmvalfont{\it}
                                                      
\usepackage{layout}
\usepackage{lipsum}

%%for LSP-lines in tables: %%doesn’t work for some reason. Plus, not all my tables have a single-line header. Double-lines aren’t typeset properly in ‘longtable’-environment across several pages.
%\usepackage{booktabs}
%\newcommand{\mytoprule}{\midrule\toprule}
%\newcommand{\mybottomrule}{\bottomrule\midrule}

%%%following now in main document (XeTeX_pitePhDSDL.tex)
%\title{A corpus-based grammar \\ of spoken Pite Saami}  
%%\subtitle{2000+ Years of Language Science and no End in Sight}  
%\author{Joshua Wilbur}
%\dedication{Gijtov adnet!}
%\renewcommand{\lsBackBody}{This grammar of Pite Saami (Uralic; Sweden) is simply bad-ass.}%for back cover text
%\renewcommand{\lsBackTitle}{Biddumsáme giella}%for back title

%%%% ABOVE FOR LangSciPress %%%%
%%%% ABOVE FOR LangSciPress %%%%
%%%% ABOVE FOR LangSciPress %%%%

\usepackage{longtable}

\usepackage{multirow}
\usepackage{array} %allows, among other things, centering column content in a table while also specifying width, creates new column style "x" for center-alignment, "y" for right-alignment
\newcolumntype{x}[1]{%
>{\centering\hspace{0pt}}p{#1}}%
\newcolumntype{y}[1]{%
>{\raggedleft\hspace{0pt}}p{#1}}%

\usepackage[]{placeins}%using \FloatBarrier command, all floats still floating at that point will be typeset, and cannot cross that boundary. the option here \usepackage[section]{placeins} automatically adds \FloatBarrier to every \section command (only works for \section commands, nothing lower than that!)
\usepackage{afterpage}%by using the command \afterpage{\clearpage}, all floats will appear, but no new page will be started, thus avoiding bad page breaks around floats

\usepackage{vowel} %for vowel space chart


%%IS THIS NECESSARY??
%%%following allows you to refer to footnotes (from http://anthony.liekens.net/index.php/LaTeX/MultipleFootnoteReferences)
\newcommand{\footnoteremember}[2]{
  \footnote{#2}
  \newcounter{#1}
  \setcounter{#1}{\value{footnote}}
} \newcommand{\footnoterecall}[1]{
  \footnotemark[\value{#1}]} 
%%%previous allows you to refer to footnotes: use \footnoteremember{referenceText} in footnote, then \footnoterecall{referenceText} to refer.

\usepackage{tikz}
\usetikzlibrary{plothandlers,matrix,decorations.text,shapes.arrows,shadows,chains,positioning,scopes}

\usepackage{synttree} %zeichnet linguistische Bäume
\branchheight{36pt}%sets height between rows in synttree

\usepackage{lscape}%used for landscape pages in index (list of recordings)

\usepackage{polyglossia}
\setmainlanguage{english}



%%%TAKE OUT FOR FINAL VERSION:
%%%TAKE OUT FOR FINAL VERSION:
%%%TAKE OUT FOR FINAL VERSION:

%%%following readjusts margin text!
\setlength{\marginparwidth}{20mm}
\let\oldmarginpar\marginpar
\renewcommand\marginpar[1]{\-\oldmarginpar[\raggedleft\footnotesize\vspace{-7pt}\color{red}\It{→ #1}]%
{\raggedright\footnotesize\vspace{-7pt}\color{red}\It{→ #1}}}
%%%previous readjusts margin text!

%%%The following lines set depth of ToC (LSP default is only 3 levels)!
%%%\renewcommand{\contentsname}{Table of Contents} % überschrift des inhaltsverzeichnisses
%\setcounter{secnumdepth}{5}%sets how deep section/subsection/subsubsections are numbered
%\setcounter{tocdepth}{5}%sets the depth of the ToC %but this doesn't seem to work!!!
\include{newcommandsSDL}\include{hyphenationSDL}\begin{document}\tableofcontents

%%%%%%%%%%%%%%%%%% ALL THE ABOVE TO BE COMMENTED OUT FOR COMPLETE DOCUMENT! %%%%%%%%%%%


\chapter{Nominals I: Nouns}\label{nouns}\index{nouns}\index{parts of speech!nouns}
\Bf{Nouns} in \PS\ form an open class of words that are formally defined %, as a morphological class, by their ability to inflect for case and number, and, as a syntactic class, 
by their ability to head a nominal phrase. \korr{069}As the head of an NP, a noun inflects for case and number. 
%which are formally defined by their ability to inflect for case and number. %the head of a nominal phrase\marginpar{can GEN ever be head? but it’s definitely a form of noun. why?}. 
Each nouns consists of a lexical stem (∑) followed by an inflectional class marker and a portmanteau suffix indicating case and number, as illustrated in Figure \vref{nounStructure}.
\begin{figure}\centering
\fbox{∑ \PLUS\ class-marker \PLUS\ case/number}
\caption{The morphological structure of \PS\ nouns}\label{nounStructure}
\end{figure}

\PS\ noun stems can have up to three allomorphic forms throughout the nominal paradigm due to a complex combination of morphophonological processes. 
\korr{004}The current chapter first describes the morphological categories number (Section \ref{numberNouns}) and case (Section \ref{case}) in a general way in order to provide a background for the variables discussed in Section \ref{NumCaseNouns} on morphological case and number marking. % depending on the nominal categories case and number. %and phonologically motivated vowel harmony. 
%Different nouns behave in different ways and can therefore be divided into
A description of the inflectional class markers and the resulting inflectional classes for nouns is given in Section \ref{nounClasses}.\footnote{Cf. Section \ref{nDerivation} for derivational morphology creating nouns.} % of Chapter \ref{derivMorph}.
The final section (\ref{possSuffixes}) deals briefly with the possessive suffixes, an infrequent set of archaic suffixes that indicate number and case and signify the possessor of the head noun’s referent. % possession of the noun’s referent, are dealt with in the last section (\ref{possSuffixes}).

\section{Number in nouns}\label{numberNouns}\index{number}\index{nouns!number}
\PS\ nouns inflect for singular and plural in all grammatical cases except the essive and possibly the abessive case. Dual is not a relevant category for nouns, despite being an integral category in verb morphology %, cf. Chapter \ref{verbs}
and for some pronoun classes. Number is expressed along with case by portmanteau suffixes, stem alternations, or a combination of both. %, depending on the inflectional class a noun belongs to and the case it is in for a particular utterance. 
Section \ref{NumCaseNouns} on number and case marking %and Section \ref{nounInflectionClasses} on inflectional classes 
treats this in more detail.

There is no formal distinction between countable and mass nouns in \PS, as illustrated by the example in \REF{uncountableEx1}, in which the words for ‘flour’, ‘sugar’ and ‘food’ are all inflected for plural. %\It{biergo} ‘meat’ is \PL\ and refers to the meat from two moose. 
\ea\label{uncountableEx1}
\glll	dán ájten inimä jáfojd ja suhkurijd ja gárvojd ja iehtjá biebmojd\\
	d-á-n ájte-n ini-mä jáfo-jd ja suhkuri-jd ja gárvo-jd ja iehtjá biebmo-jd\\
	\Sc{dem}-\Sc{prox}-\Sc{iness.sg} shed-\Sc{iness.sg} have-\Sc{1pl.pst} flour-\Sc{acc.pl} and sugar-\Sc{acc.pl} and clothing-\Sc{acc.pl} and other food-\Sc{acc.pl}\\\nopagebreak
\Transl{in this shed we had flour and sugar and clothing and other food}{}	\Corpus{100310b}{100-104}
%\glll	udjun biergo tjåsskot\\
%	udju-n biergo tjåssko-t\\
%	be\_allowed-\Sc{3pl.pst} meat\BS\Sc{nom.pl} cool-\Sc{inf}\\
%\Transl{the meat was allowed to cool down}{}	\Corpus{090702.338-339}
%\Transl{the meat was allowed to cool down}{} (lit.: the meats were allowed to cool down)	\Corpus{090702.338-339}
\z
When the singular form is used, a noun’s referent is either \korr{021}generic, as illustrated by both nouns in \REF{uncountableEx2}, or it refers to a single unit, %\PL\ always refers to several units.
as the noun \It{ájten} in example \REF{uncountableEx1} above. 
\ea\label{uncountableEx2}
\glll	men vuästa, del káfan njallge\\
	men vuästa del káfa-n njallge\\
	but cheese\BS\Sc{nom.sg} definitely coffee-\Sc{iness.sg} tasty\\\nopagebreak
\Transl{but cheese, (it’s) definitely tasty in coffee}{}	\Corpus{080924}{139}
\z


\section{The nominal case system}\label{case}\index{case system}\index{nouns!case system}
\PS\ has nine cases%\marginpar{\Bf{grammatical} cases? not right term? just cases?}
%: \NOM, \GEN, \ACC, \ILL, \INESS, \ELAT, \COM, \ABESS, and \ESS. 
: nominative, genitive, accusative, illative, inessive, elative, comitative, abessive, and essive.\korr{046}\footnote{The terminology chosen here for the nine cases reflects the names used traditionally in Uralic studies.} 
Nouns inflect for these cases, in addition to number, via portmanteau suffixes, stem alternations, or a combination of both. %,  depending on the inflectional class a noun belongs to and the case it is in for a particular utterance. 
%in a variety of ways which are covered in Section \ref{NumCaseNouns}; this allows at least ?how?many? inflectional classes to be posited (see Section \ref{nounInflectionClasses}). Pronouns also inflect for case (cf. Chapter \ref{pronouns}). 
A general description of the cases is provided here. %before describing the inflectional classes that result from case and number inflectional patterning in nouns.
Note that the case system is valid for pronouns (also a subclass of nominals) as well, but not for adjectives and numerals. %(cf. Chapter \ref{adjectivesIntro}). %To a very limited extent, adjectives 

\subsection{Nominative case}\label{nominative}\index{nominative}\index{case system!nominative}
In addition to being used as the citation form, most commonly in singular, nominative case (glossed as \NOMs) marks the grammatical subject of a verbal clause (typically the most agent-like argument for transitive verbs) 
%is typically used for the sole argument (subject) of an intransitive verb 
as in \REF{nom1} and %for the more agent-like argument of a transitive verb as in 
\REF{nom2}. %The \Sc{nom.sg} form is generally used for entries in a dictionary or wordlist.
\ea\label{nom1}
\glll dä stuor \Bf{sarves} båhta\\ %\Corpus{090702.319} \\ %
	dä stuor sarves båhta\\
	then big moose\BS\Sc{nom.sg} come\BS\Sc{3sg.prs}\\\nopagebreak
\Transl{then a big moose arrives}{} \Corpus{090702}{319}
\z
\ea\label{nom2}
\glll	ja dä dáhka \Bf{almatj} dålåv\\ %
	ja dä dáhka almatj dålå-v\\
	and then make\BS\Sc{3sg.prs} person\BS\Sc{nom.sg} fire-\Sc{acc.sg}\\\nopagebreak
\Transl{and then one makes a fire}{} \Corpus{100404}{102}
%\glll nå mav dä duv \Bf{äddne} dakaj tjuolejst\\ %\Corpus{080924.229} \\ %
%	nå ma-v dä duv äddne dahka(wk)-j tjuolle(wk)-jst\\
%	well what-\Sc{acc.sg} then 2.\Sc{gen.sg} mother.\Sc{nom.sg} make-\Sc{3sg.pst} intestine-\Sc{elat.pl}\\
%\Transl{and what did your mother make out of the intestines?’ \Corpus{080924.229}%\marginpar{check intestines SG or PL??, if PL change to \It{-jd}}
\z

The possessed noun in a possessive copular clause (cf. \ref{copulaClauses}) is also in the nominative case, as in \REF{nom3}.
\ea\label{nom3}
\glll muvne lä \Bf{bijjla}\\ %
	muvne lä bijjla\\
	\Sc{1sg.iness} be\BS\Sc{3sg.prs} car\BS\Sc{nom.sg}\\\nopagebreak
\Transl{I have a car}{(lit.: ‘at-me is car’)} \CorpusE{080926}{01m44s}
\z


\subsection{Genitive case}\label{genitive}\index{genitive}\index{case system!genitive}
The genitive case (glossed as \GENs), the only adnominal case in \PS, %and never a direct argument or adjunct to a verbal clause, but 
marks the possessor modifying the head of a noun phrase (the possessed noun), as in \REF{gen1}. 
\ea\label{gen1}
\glll	gokt lä \Bf{dan} \Bf{almatja} namma majna ságasta\\ %
	gokt lä d-a-n almatj-a namma ma-jna ságasta\\
	how be\BS\Sc{3sg.prs} \Sc{dem}-\Sc{dist}-\Sc{gen.sg} person-\Sc{gen.sg} name\BS\Sc{nom.sg} \Sc{rel}-\Sc{com.sg} talk\BS\Sc{2sg.prs}\\\nopagebreak
\Transl{what is the name of that person who you are talking with?}{} \CorpusE{110521b1}{040}
%\glll jus gallga \Bf{sáme} viesov válldet\\ %JW: this is maybe not the best example because the genitive ’sáme’ doesn’t translate well in SG
%	jus gallga sáme vieso-v vállde-t\\
%	if shall\BS\Sc{3sg.prs} Saami\BS\Sc{gen.sg} way-\Sc{acc.sg} take-\Sc{inf}\\
%\Transl{if one shall use the Saami way’ \Corpus{080909.097}
\z

Furthermore, the nominal complement in a postpositional phrase occurs in the genitive case, %UM: revise, add sources, or DELETE footnote: \footnote{Historically, postpositional phrases stem from a figurative use of possession relationships concerning spatial relations; see Section \ref{postpositions} for more details.} 
as the noun \It{gåde} in \REF{gen2}.
\ea\label{gen2}
\glll \Bf{gåde} sinne suovastit\\ %
	gåde sinne suovasti-t\\
	hut\BS\Sc{gen.sg} in smoke-\Sc{inf}\\\nopagebreak
\Transl{to smoke (something) inside a hut}{} \Corpus{100405a}{157}
\z


\subsection{Accusative case}\label{accusative}\index{accusative}\index{case system!accusative}
The accusative case (glossed as \ACCs) marks the object of a transitive verb, as illustrated by the monotransitive clause in \REF{acc1}.  
\ea\label{acc1}
\glll	dä virtiv válldet \Bf{giehpajd} ja \Bf{ribrev} ja \Bf{dagarijd} ulgos\\ %
	dä virti-v vállde-t giehpa-jd ja ribbre-v ja dagari-jd ulgos\\
	then must-\Sc{1sg.prs} take-\Sc{inf} lung-\Sc{acc.pl} and liver-\Sc{acc.sg} and such-\Sc{acc.pl} out\\\nopagebreak
\Transl{then I have to take out the lungs, the liver and such things}{} \Corpus{080909}{103}
\z

In ditransitive clauses, the accusative marks the object referring to the theme while the recipient is marked by the illative, as in \REF{acc2}.
\ea\label{acc2}
\glll	mån vaddav gajka buhtsujda \Bf{biebmov}\\ %
	mån vadda-v gajka buhtsu-jda biebmo-v\\
	\Sc{1sg.nom} give-\Sc{1sg.prs} all\BS\Sc{ill} reindeer-\Sc{ill.pl} food-\Sc{acc.sg}\\\nopagebreak
\Transl{I give food to all the reindeer}{} \CorpusE{110413b}{137}
\z
The accusative can also mark nouns functioning as a clause-level temporal adverbial phrase denoting a period of time, %are normally also inflected for the \ACC\ case, 
as in \REF{acc3}.%JW: another example found in 080702b.092: ‘mån lev ijav årram’ (I have slept there one night)
\ea\label{acc3}
\glll	jo dan vuolen udemä \Bf{ijav}\\ %deleten ‘vaj’ - check original!
	jo d-a-n vuolen ude-mä ija-v\\
	yes \Sc{dem}-\Sc{dist}-\Sc{gen.sg} under sleep-\Sc{1pl.pst} night-\Sc{acc.sg}\\\nopagebreak
\Transl{yes and we slept under that for a night}{} \Corpus{090702}{305}
\z


\subsection{Illative case}\label{illative}\index{illative}\index{case system!illative}
The illative case (glossed as \ILLs) marks nouns that are the goal of the action expressed by a verb of motion, as in \REF{ill1}. %, the addressee of communication as in \REF{ill2} or the beneficiary of an action as in \REF{ill3}. The recipient of ‘giving’ actions is also in \Sc{illative} as in \REF{ill4}.
%Both literal and figurative motion towards a noun normally triggers \Sc{illative} case on that noun, as in \REF{someReindeerDontComeIntoCorral}. This includes the recipient affected by the action expressed by a verb, as in \REF{myFatherSaidToFriend} and \REF{bootsMadeLeatherForKidsNaturally}. \marginpar{check this about comparisons! and find example!}In comparisons, if a noun is a standard for comparison, then it is in this \Sc{illative}, too???.
\ea\label{ill1}
\glll	muhten båtsoj ij både \Bf{gärrdáj}\\ %
	muhten båtsoj ij både gärrdá-j\\
	some reindeer\BS\Sc{nom.pl} \Sc{neg\BS3pl.prs} come\BS\Sc{conneg} corral-\Sc{ill.sg}\\%yes dem.iness.sg under eller? sleep-1pl.pst night-acc.sg
\Transl{some reindeer don’t come into the corral}{} \Corpus{080909}{007}
\z
In addition, the illative case marks nouns that refer to the addressee of communication, as in \REF{ill2}, and the recipient of ‘giving’ actions, as in \REF{ill4}.
\ea\label{ill2}
\glll	muv áhttje hålloj såmes \Bf{raddnáj}\\
	muv áhttje hållo-j såmes raddná-j\\
	\Sc{1sg.gen} father\BS\Sc{nom.sg} say-\Sc{3sg.pst} some friend-\Sc{ill.sg}\\\nopagebreak
\Transl{my father said to some friend}{} \Corpus{090702}{505}
\z
%
%Nouns that refer to %the beneficiary of an action, as in \REF{ill3}, including %not sure about this.
%the recipient of ‘giving’ actions are in the \ILL\ case, as in \REF{ill4}. 
%%\ea\label{ill3}
%\glll	dá lä sasnest gårroduvum, tjatsega \Bf{mánnáj} diedon\\
%	dá lä sasne-st gårro-duvu-m	tjatseg-a	mánná-j diedo-n\\
%	\Sc{dem.prox}\BS\Sc{nom.pl} be\BS\Sc{3pl.prs} furless\_leather-\Sc{elat.sg} sew-\Sc{pass-prf} waterproof\_boot-\Sc{nom.pl} child-\Sc{ill.sg} of\_course \\ %JW: of course = knowledge-\Sc{iness.sg}\\%
%\Transl{these are sewn out of furless leather, waterproof boots for a child, of course’ \Corpus{080708\_Session08.001}
%\z
%
\ea\label{ill4}
\glll	vadde \Bf{Jåssjåj} aj \\
	vadde Jåssjå-j also \\
	give\BS\Sc{sg.imp} Josh-\Sc{ill.sg} also \\%
\Transl{give (one) to Josh, too}{} \Corpus{090519}{033}%\marginpar{check the transcription of \REF{giveJoshToo}}
\z

Finally, familial relations can also be expressed using an illative construction. In such cases, the ‘ego’ of the family relation is in the illative, as in \REF{ill5}.
\ea\label{ill5}
\glll	dån lä eddno \Bf{munje} \\
	dån lä eddno munje\\
	\Sc{2sg.nom} be\BS\Sc{2sg.prs} maternal\_uncle\BS\Sc{nom.sg} \Sc{1sg.ill} \\%
\Transl{you are my maternal uncle}{(lit.: you are maternal uncle to me)} \CorpusE{110413b}{035}%
\z
In this example, the illative nominal is a pronoun, but it is plausible that full nouns are possible in this function as well, although there are no such tokens in the corpus.


\subsection{Inessive case}\label{inessive}\index{inessive}\index{case system!inessive}
The inessive case (glossed as \INESSs) marks nouns which function as adjuncts to verbal clauses indicating the location of the event or action, as in \REF{iness1}. 
%The \Sc{inessive} case can be the adjunct of a predicate, including the copula, and signifies the general location of an event or an object, as in \REF{whatMoreDidYouDoAhkkabakkte} and \REF{SalvoCreekInValley}. 
% is used in a general stative locative meaning, as in \REF{whatMoreDidYouDoAhkkabakkte} and \REF{SalvoCreekInValley}. To indicate more detailed information on stative position, post positions are normally used (see Section \ref{?}). 
%When combined with the copula \It{lä} and a following noun, \Sc{inessive} case can also be used to indicate the possessor (with the noun following the copula as the possession), as in \REF{SaamiHasDog} and \REF{youAlsoHaveFjällrävenPants}.
\ea\label{iness1}
\glll	nå, mav enabov dihki \Bf{Áhkkabakten} \\
	nå ma-v enabo-v dihki Ahkkabakte-n \\
	well what-\Sc{acc.sg} more-\Sc{acc.sg} do\BS\Sc{2sg.pst} Ahkkabakte-\Sc{iness.sg} \\\nopagebreak
\Transl{well, what more did you do in Áhkkabakkte?}{} \Corpus{080924}{021}
\z

Similarly, as the complement of the copular verb, an inessive noun indicates the location of the subject referent, as in \REF{iness2}.
\ea\label{iness2}
\glll	\Bf{vággen} Sálvojåhkå'l\\
	vágge-n Sálvo-jåhkå=l\\
	valley-\Sc{iness.sg} Sálvo-creek\BS\Sc{nom.sg}=be\BS\Sc{3sg.prs}\\\nopagebreak
\Transl{Sálvo Creek is in the valley}{} \Corpus{100404}{007}
\z

The possessor noun in a possessive copular clause (cf. \ref{copulaClauses}) is also in the inessive case, as in \REF{iness3}.
%In possessive copula clauses, the possessor is in this case as in \REF{SaamiHasDog}.\footnote{Cf. Section \ref{possCopClause} on the possessive copula clause}
\ea\label{iness3}
\glll	\Bf{sámen} lä bena\\
	sáme-n lä bena\\
	Saami-\Sc{iness.sg} be\BS\Sc{3sg.prs} dog\BS\Sc{nom.sg}\\\nopagebreak
\Transl{The Saami has a dog}{(lit.: at Saami is dog)} \CorpusE{080917a}{068}%JW: Is this really an elicitation session?
%%%%here a very nice example for possessive copula clause, but with a pronoun possessor, so not the best example here in this section on nouns.
%%%\ea\label{youAlsoHaveFjällrävenPants}
%%%\glll	\Bf{duvne}’l aj svála båkså\\
%%%	duvne=l aj sválla(wk) båkkså(wk)\\
%%%	\Sc{2sg.iness}=be.\Sc{3pl.prs} also mountain\_fox.\Sc{gen.sg} trouser.\Sc{nom.pl}\\
%%%\Transl{You also have Fjällräven\footnote{Fjällräven is a Swedish outdoor clothing company named for the mountain fox (\It{fjällräven} in Swedish); here the company name has been understood literally and translated into its \PS\ equivalent: \It{sválla} ‘mountain fox’. Nonetheless, it refers here to the company and not the animal.} trousers’ \Corpus{090519.073}
\z


\subsection{Elative case}\label{elative}\index{elative}\index{case system!elative}
The elative case (glossed as \ELATs) marks nouns as the source of an action of transfer, as in \REF{elat1}, as well as the origin, as in \REF{elat2a} and \REF{elat2b}. 
\ea\label{elat1}
\glll	váldav tjåjvev \Bf{ribrist} luovas\\
	válda-v tjåjve-v ribri-st luovas\\
	take-\Sc{1sg.prs} stomach-\Sc{acc.sg} liver-\Sc{elat.sg} loose\\\nopagebreak
\Transl{I loosen the stomach from the liver}{} \Corpus{080909}{079}
\z
\ea\label{elat2a}
\glll	nå gåsse dija älgijdä \Bf{Örnvikast} vuodjet vadnásav\\
	nå gåsse dija älgi-jdä Örnvika-st vuodje-t vadnása-v\\ %vánás(str)??
	well when \Sc{2pl.nom} begin-\Sc{2pl.pst} Örnvik-\Sc{elat.sg} drive-\Sc{inf} boat-\Sc{acc.sg}\\\nopagebreak
\Transl{well when did you start taking the boat from Örnvik?}{} \Corpus{080924}{563}%UM: ‘driving’ statt ‘taking’? aber hier geht es darum, wie man sich fortbewegt, also ‘take the bus’ (mit dem bus fahren) statt den Bus bzw. das Boot selber fahren. ist auch nicht so relevant für dieses beispiel, da es um Örnvik-ELAT geht.
\z
\ea\label{elat2b}
\glll	dån båda \Bf{Amerigist}\\
	dån båda Amerig-ist\\
	\Sc{2sg.nom} come\BS\Sc{2sg.prs} America-\Sc{elat.sg}\\\nopagebreak
\Transl{you come from America}{} \CorpusE{080621}{28m02s}
\z

The elative also marks the addressee of a question (the source of information), as in \REF{elat5}.
\ea\label{elat5}
\glll	\Bf{Eddest} galgav gatjadit\\
	Eddest galga-v gatjadi-t\\
	Edgar-\Sc{elat.sg} will-\Sc{1sg.prs} ask-\Sc{inf}\\\nopagebreak
\Transl{I will ask Edgar}{} \Corpus{090519}{357}
%\ea\label{}
%\glll	\\
%	\\
%	\\
%\Transl{’ \Corpus{}
\z

Similarly, in a copula clause, the elative case can mark a noun whose referent is a source of pain, as in \REF{elat6}.
\ea\label{elat6}
\glll	mån lev \Bf{åjvest}\\
	mån le-v åjve-st\\
	\Sc{1sg.nom} be-\Sc{1sg.prs} head-\Sc{elat.sg}\\\nopagebreak
\Transl{I have a headache}{(lit.: ‘I am from head’)} \CorpusE{110331b}{079}
%\ea\label{}
%\glll	\\
%	\\
%	\\
%\Transl{’ \Corpus{}
\z

The noun referring to the material that something consists of or is made of is in the elative case, as in \REF{elat3} and \REF{elat4}. %\marginpar{\REF{elat4} ist das gleiche Beispiel wie \REF{ill3}; soll ich das erwähnen?} 
\ea\label{elat3}
\glll	mån iv tuhtje dav färska \Bf{málest}\\ %
	mån i-v tuhtje d-a-v färska mále-st\\ 
	\Sc{1sg.nom} \Sc{neg-1sg.prs} like\BS\Sc{conneg} \Sc{dem}-\Sc{dist}-\Sc{acc.sg} fresh blood-\Sc{elat.sg}\\\nopagebreak
\Transl{I don’t like that (made) of fresh blood}{} \Corpus{080924}{271}%
%\glll mav dä duv äddne dakaj \Bf{tjuolejst}\\ %\Corpus{080924.229} \\ %
%	ma-v dä duv äddne dahka(wk)-j tjuolle(wk)-jst\\
%	what-\Sc{acc.sg} then \Sc{2sg.gen} mother\BS\Sc{nom.sg} make-\Sc{3sg.pst} intestine-\Sc{elat.pl}\\
%\Transl{what did your mother make out of the intestines?’ \Corpus{080924.229}%\marginpar{check intestines SG or PL??, if PL change to \It{-jd}}
\z
\ea\label{elat4}
%\glll	ja dát lä aj \Bf{struvdast}\\
%	ja dát lä aj struvda-st\\
%	and \Sc{dem.prox.sg} be\BS\Sc{3sg.prs} also cloth-\Sc{elat.sg}\\ %
%\Transl{and this is also (made) of cloth’ \Corpus{080708\_Session08.015}
\glll	dá lä \Bf{sasnest} gårroduvum\\
	d-á lä sasne-st gårro-duvu-m\\
	\Sc{dem}-\Sc{prox}\BS\Sc{nom.pl} be\BS\Sc{3pl.prs} furless\_leather-\Sc{elat.sg} sew-\Sc{pass-prf}\\\nopagebreak
\Transl{these are sewn out of furless leather}{} \CorpusLink{080708_Session08}{080708\_Session08}{001}
\z

The elative case can be used to mark the agent which carries out the action referred to by a passivized verb, as in \REF{elat5b}.
\ea\label{elat5b}%
\glll	gåhte lä tsiggijduvvum \Bf{mánájst}\\
	gåhte lä tsiggij-duvvu-m máná-jst\\
	hut\BS\Sc{nom.sg} be\BS\Sc{3sg.prs} build-\Sc{pass}-\Sc{prf} child-\Sc{elat.pl}\\\nopagebreak
\Transl{the hut has been built by children}{}	\CorpusE{110518a}{28m41s}
\z

In comparative constructions, elative marks a noun whose referent is the standard in the comparison, as in \REF{elat7}.
\ea\label{elat7}
\glll	mån lev stuorab \Bf{Svienast}\\
	mån le-v stuora-b Sviena-st\\
	\Sc{1sg.nom} be-\Sc{1sg.prs} big-\Sc{comp} Sven-\Sc{elat.sg}\\ %
\Transl{I am bigger than Sven}{} \CorpusE{110331b}{087}
\z


\subsection{Comitative case}\label{comitative}\index{comitative}\index{case system!comitative}
The comitative case (glossed as \COMs) marks nouns referring to someone or something participating in an action together with the agent as in \REF{com1}, or some other participant, as in \REF{com1b}.
\ea\label{com1}
\glll	men ådtjo sáme gielav ságastit duv \Bf{årbenij}\\
	men ådtjo sáme giela-v ságasti-t duv årbeni-j\\
	but may\BS\Sc{2sg.pst} Saami\BS\Sc{gen.sg} language-\Sc{acc.sg} speak-\Sc{inf} \Sc{2sg.gen} sibling-\Sc{com.pl}\\\nopagebreak
\TranslLong{but were you allowed to speak the Saami language with your siblings?}{} \Corpus{080924}{366}
\z
\ea\label{com1b}
\glll	válda káfav \Bf{suhkorijn} jala suhkorahta\\
	válda káfa-v suhkori-jn jala suhkor-ahta\\
	take\BS\Sc{2sg.prs} coffee-\Sc{acc.sg} sugar-\Sc{com.sg} or sugar-\Sc{abess}\\\nopagebreak
\Transl{Do you take your coffee with sugar or without sugar?}{} \CorpusE{110509b}{11m41s}
\z

The comitative also marks nouns referring to an instrument used to carry out an action, as in \REF{com2}.
\ea\label{com2}
\glll	del vuodja \Bf{bijlajn} Örnvikaj ja dä \Bf{vádnasijn} Tjeggelvasa badjel\\
	del vuodja bijla-jn Örnvika-j ja dä vádnasi-jn Tjeggelvas-a badjel\\
	now drive\BS\Sc{3sg.prs} car-\Sc{com.sg} Örnvik-\Sc{ill.sg} and then boat-\Sc{com.sg} Tjeggelvas-\Sc{gen.sg} over\\\nopagebreak
\Transl{now one drives to Örnvik by car, then by boat over Lake Tjeggelvas}{} \Corpus{080924}{471}
%\glll	ja dä del \Bf{giedij} dun hämpej\\ %JW: not so sure about the glossing of this, better not use it.
%	ja dä del giedi-j dun hämpe-j\\
%	and then obviously hand-\Sc{com.pl} \Sc{3sg.iness??} tickle-\Sc{inf}\\
%\Transl{and then obviously she tickled her with her hands’ \Corpus{100703a.036}
%\ea\label{}
%\glll	\\
%	\\
%	\\
%\Transl{’ \Corpus{}
\z

When two persons or things are equated with respect to a certain characteristic, the comitative marks the noun whose referent is the standard of comparison, as in \REF{com3}.
\ea\label{com3}
\glll	Svenna lä akta vuoras \Bf{Ingerijn}\\
	Svenna lä akta vuoras Inger-ijn\\
	Sven\BS\Sc{nom.sg} be\BS\Sc{3sg.prs} one old Inger-\Sc{com.sg}\\\nopagebreak
\Transl{Sven is as old as Inger}{(lit.: Sven is one age with Inger)} \CorpusE{110331b}{135}
\z


\subsection{Abessive case}\label{abessive}\index{abessive}\index{case system!abessive}%
The referent of a noun marked by the abessive case (glossed as \ABESSs) is lacking or missing, as illustrated by \REF{abess1}\footnote{This example is also found in \REF{com1b} above but is repeated here for convenience, as well as to focus on the abessive noun.} 
and \REF{abess2}. 
\ea\label{abess1}
\glll	válda káfav suhkorijn jala \Bf{suhkorahta}\\
	válda káfa-v suhkori-jn jala suhkor-ahta\\
	take\BS\Sc{2sg.prs} coffee-\Sc{acc.sg} sugar-\Sc{com.sg} or sugar-\Sc{abess}\\\nopagebreak
\Transl{Do you take your coffee with sugar or without sugar?}{} \CorpusE{110509b}{11m41s}
\z
%
\ea\label{abess2}
\glll	dån lä \Bf{vájmodak} dal\\
	dån lä vájmo-dak dal\\
	\Sc{2sg.nom} be\BS\Sc{2sg.prs} heart-\Sc{abess} now\\\nopagebreak
\Transl{You are heartless now}{} \CorpusE{110413a}{226}
%\ea\label{}
%\glll	\\
%	\\
%	\\
%\Transl{’ \Corpus{}
\z
Note that nouns in abessive are rare in natural speech, and limited to elicitation sessions in the corpus.%\footnote{The example in \REF{abess2} was produced spontaneously in the context of an elicitation session.} %JW: I didn’t elicit this clause, but the session was about ABESS, so it was definitely triggered by elicitation
\footnote{The \WLP’s wordlist indicates that the word \It{ájnát} can also be used to express ‘without’, but the corpus does not provide any tokens of this.} 
While the meaning of nouns in the abessive case is quite clear, their morphophonological behavior is problematic; see Section \ref{abessiveProblematic} for more details.
%\Red{Maybe best to not analyze \ABESS\ as part of the case system, but instead a derivational morpheme whose realized form is driven by the syllable length of the stem (\It{-dak} for even-σ, \It{-ahta} for odd-σ), which creates a new NOUN, something like English ‘-lessness’; it patterns like other \It{-ag} words in any case; thus \It{vájmodak} is really ‘heartlessness’ and \It{suhkorahta} ‘sugarlessness’; but why would these be nouns here?}


\subsection{Essive case}\label{essive}\index{essive}\index{case system!essive}
The essive case (glossed as \ESSs) generally %from UM:
marks predicative nouns functioning as complements of certain verbs such as \It{sjaddat} ‘become’, as in \REF{ess1a} and \REF{ess1b}, and \It{gähtjoduvvat} ‘be called’, as in \REF{ess2}. Note that nouns in the essive case do not inflect for number. 
%indicates being in the state of the noun’s referent. %\marginpar{this definition may need work?} 
%For instance, the object of the verb \It{sjaddat} ‘become’ is in the essive case, as in \REF{ess1a} and \REF{ess1b}. 
\ea\label{ess1a}
\glll	\Bf{bednan} sjaddav\\
	bedna-n sjadda-v\\
	dog-\Sc{ess} become-\Sc{1sg.prs}\\\nopagebreak
\Transl{I become a dog}{} \CorpusE{110509b}{05m49s}
\z
\ea\label{ess1b}
\glll	jegŋa sjaddá \Bf{tjáhtsen}\\
	jegŋa sjaddá tjáhtse-n\\
	ice\BS\Sc{nom.sg} become\BS\Sc{3sg.prs} water-\Sc{ess}\\\nopagebreak
\Transl{ice becomes water}{} \CorpusE{110331b}{160}
\z
%
%The essive also occurs in other functions which are not easily summarized by the ‘state-of-being’ description, such as in \REF{ess2}, where the subject of the verb \It{gähtjoduvvat} ‘be called’ is in essive. 
%%When a noun is in the \ESS\ case, this generally indicates a\marginpar{this definition may need work?} state of being the referent of the noun. 
%%A small number of verbs such as \It{sjaddat} ‘become’ and \It{gähtjoduvvat} ‘be called’ govern objects in this case, as in \REF{ess1} and \REF{ess2}. 
%%The state something/someone is in or into which something/someone changes is marked by the \Sc{essive} case as in \REF{iBecomeDog}. 
\ea\label{ess2}
\glll	dut såhke \Bf{vadnásan} gåhtjoduvva\\
	d-u-t såhke vadnása-n gåhtjo-duvva \\
	\Sc{dem}-\Sc{rmt}-\Sc{nom.sg} birch\BS\Sc{nom.sg} boat-\Sc{ess} call-\Sc{pass\BS3sg.prs}\\\nopagebreak
\Transl{yonder birch is called a boat}{} \CorpusE{110509b}{14m02s}
\z%\ea\label{}
%\glll	\\
%	\\
%	\\
%\Transl{’ \Corpus{}
%\z

Furthermore, the complement of the particle \It{dugu} ‘like’ can be in the essive case, as illustrated by the example in \REF{ess3}. 
\ea\label{ess3}
\glll	dat vuodja dugu \Bf{goullen}\\
	d-a-t vuodja dugu goulle-n\\
	\Sc{dem}-\Sc{dist}-\Sc{nom.sg} swim\BS\Sc{3sg.prs} like fish-\Sc{ess}\\\nopagebreak
\Transl{he swims like a fish}{} \CorpusE{110413a}{059}
\z
However, while my main consultant accepted constructions like in \REF{ess3}, her initial response normally consisted of nearly the same construction, only with the noun in nominative case, as in \REF{ess4}. 
\ea\label{ess4}
\glll	vuodja dugu \Bf{goulle}\\
	vuodja dugu goulle\\
	swim\BS\Sc{3sg.prs} like fish\BS\Sc{nom.sg}\\\nopagebreak
\Transl{(he) swims like a fish}{} \CorpusE{110413a}{052}
\z
Finally, it should be pointed out that essive is not particularly common in the corpus; tokens for this case are only found in elicitation sessions. In summary, there is not enough data to come to any definitive conclusions concerning the status of essive in current \PS\ usage. %UM: deleted:, and further study is necessary.


\section{Number and case marking on nouns}\label{NumCaseNouns}
%\section{Noun paradigms}\label{caseParadigm}
As indicated in the previous sections, \PS\ nouns inflect for nine cases and two number categories (only the essive and possibly the abessive cases do not inflect for number). While case and number are generally marked by nominal suffixes, they are often supplemented by other morphophonological marking strategies, or even expressed solely by non-linear morphology. These other strategies are: 
\begin{itemize}
\item{consonant alternations in the stem (also known as \It{consonant gradation})}
\item{stem-vowel alternations (umlaut)}
\item{vowel harmony}
\end{itemize} %and trisyllabicity\marginpar{Decide on and check consistency of this term ‘trisyllabicity’!}, 
Concerning nouns, the segmental alternations are discussed in detail in section \ref{nonlinearNounMorph}, while vowel harmony is presented in \ref{NclassIe}. First, a short discussion of the nominal suffixes follows here.


\subsection{Nominal suffixes}\label{nominalSuffixes}\index{nominal suffixes}\index{suffixes!nominal}\index{nouns!suffixes}
\PS\ has a number of portmanteau suffixes expressing case and number. %that a noun is in. %UM: ‘that a noun is in’ streichen
Only \Sc{nom.sg}, \Sc{nom.pl} and \Sc{gen.sg} are generally %\marginpar{what about “stem extensions”?}
not marked by any linear morphology (although even this has exceptions). The nominal suffixes marking case and number are listed in Table \vref{nounSuffixes}. Note that the status of the abessive suffixes is unclear, including whether they inflect for number (cf. Section \ref{abessiveProblematic}).
%Most case-number combinations are additionally marked by an obligatory suffix, and these case suffixes can found in Table \vref{nounSuffixes}. Only \Sc{nom.sg}, \Sc{nom.pl} and \Sc{gen.sg} are not indicated by a suffix. 
%The \Sc{nom.pl} and \Sc{gen.sg} forms of a noun are always homophonous, and are optionally marked by a word-final \It{-h} suffix, which has only become optional over the course of the last few generations of speakers\footnote{cf. \cite{Lehtiranta1992}, \cite{Lagercrantz1926}, \cite{Halasz1896} for \PS\ before 1950.}. Without the presence of the final \It{-h}, is it even possible, in certain inflectional classes, for the \Sc{nom.sg} form to be homophonous as well.
%\abbrevsc{nom}{nominative case}\abbrevsc{gen}{genitive case}\abbrevsc{acc}{accusative case}\abbrevsc{ill}{illative case}\abbrevsc{iness}{inessive case}\abbrevsc{elat}{elative case}\abbrevsc{com}{comitative case}\abbrevsc{abess}{abessive case}\abbrevsc{ess}{essive case}
%\marginpar{\ABESS\ removed from table!}
\begin{table}\centering%\scriptsize
\caption{Nominal case and number suffixes}\label{nounSuffixes}
\begin{tabular}{ r | x{35mm} | x{35mm} | l }
			&\MC{2}{c|}{\It{number}}			&	\\
\It{case}	& \Sc{singular} 	& \Sc{plural}		& \It{case} \\\hline\hline
\Sc{nom}	&  - 				&  - ($\sim$ -h)		& \Sc{nom}\\\cline{2-3}
\Sc{gen}	&  - ($\sim$ -h)		&  -j				& \Sc{gen}\\\cline{2-3}
\Sc{acc}	&  -v				&  -jt				& \Sc{acc}\\\cline{2-3}
\Sc{ill}		&  -j				&  -jda			& \Sc{ill}\\\cline{2-3}
\Sc{iness}	&  -n				&  -jn				& \Sc{iness}\\\cline{2-3}
\Sc{elat}	&  -st				&  -jst			& \Sc{elat}\\\cline{2-3}
\Sc{com}	&  -jn(a)			&  -j				& \Sc{com}\\\cline{2-3}
\Sc{abess}	& \MC{2}{c |}{-dak, -daga, -gat, -gahta, -ahta}& \Sc{abess}\\\cline{2-3}% \It{-dak}, \It{-daga}, \It{-gat}, \It{-gahta} and \It{-ahta}. 
\Sc{ess}	& \MC{2}{c |}{-n}				& \Sc{ess}\\\hline%\hline
%case 		& \Sc{singular} 	& \Sc{plural}		& case \\
\end{tabular}
%\end{tabular*}
\end{table}

In \Sc{nom.pl} and \Sc{gen.sg}, the \It{-h} suffix is optional in \PS\ (and therefore appears in parentheses in Table \ref{nounSuffixes}).%, but has only become so over the course of the last few generations of speakers.
\footnote{The paradigms in \citet[156-157]{Lehtiranta1992} also indicate an optional \It{-h}, while \citet[104-105]{Lagercrantz1926} does not indicate any \It{-h} at all.} 
The \Sc{com.sg} suffix has two allomorphs: \It{-jn} and \It{-jna}, which seem to be in free variation in the corpus, and not determined phonologically. 


\subsubsection{Nominal suffixes and syncretism}\label{nominalSuffixesSyncretism}\index{syncretism}\index{suffixes!nominal}\index{nouns!suffixes}
Several of the nominal inflectional suffixes, considered by themselves, %\marginpar{ist dieser Abschnitt überhaupt relevant? vielleicht besser weglassen?}%UM: drin lassen! %in Table \vref{nounSuffixes} 
are homophonous: %constitute synchretic\index{homophony} sets:%; these are listed in Figure \vref{homoNounSuffixes}.
%\begin{figure}
\begin{itemize}
\item{\It{-j} for \Sc{ill.sg}, \Sc{gen.pl}, and \Sc{com.pl}}
\item{\It{-jn} for \Sc{iness.pl} and \Sc{com.sg}}
\item{\It{-n} for \Sc{ess} and \Sc{iness.sg}}
%\item{\It{-Ca} for \Sc{nom.pl} and \Sc{gen.sg} in Class III nouns\footnote{Cf. Section \ref{nounClasses} for inflectional classes for nouns; note that \It{C} stands for the final consonant of a Class III stem.}}
\item{(optional) \It{-h} for \Sc{nom.pl} and \Sc{gen.sg}\footnote{The alternative to the optional \It{-h} suffix for \Sc{nom.pl} and \Sc{gen.sg} forms is no suffix (except for Class III nouns, which are marked by \It{-Ca}).}}
%\item{\It{no suffix} for \Sc{nom.sg}, \Sc{nom.pl} and \Sc{gen.sg}\footnote{\Sc{Nom.pl} and \Sc{gen.sg} forms can be marked by an \It{-h} suffix, in which case the resulting forms are not syncretic with the \Sc{nom.sg} form; however, \It{-h} is optional and not particularly common, so it is not considered further in this discussion.}}
%\item{\It{-h} (optional) for \Sc{nom.pl} and \Sc{gen.sg} \Bf{OR} \It{no suffix} for \Sc{nom.sg}, \Sc{nom.pl} and \Sc{gen.sg}}
\end{itemize}
%\caption{Homophonous noun case-number suffixes}\label{homoNounSuffixes}\end{figure}
%Concerning the \Sc{nom.pl}-\Sc{gen.sg} suffix pair, these two suffixes are homophonous if the optional \It{-h} suffix variant is selected, but without this optional suffix, nouns in \Sc{nom.sg}, \Sc{nom.pl} and \Sc{gen.sg} completely lack a suffix, and thus they could also be considered identical as unmarked, just as \Sc{nom.sg} is unmarked. %a null-suffix \It{-ø}.

For Class III nouns which do not exhibit any stem allomorphy, %\marginpar{but there is only one w/o any allomorphy, right?} (including umlaut), 
the corresponding inflected noun forms within a paradigm are therefore syncretic. %Disregarding the optional \It{-h} suffix in \Sc{nom.pl} and \Sc{gen.sg}, then the \Sc{nom.sg}, \Sc{nom.pl} and \Sc{gen.sg} forms are also all homophonous (and lacking a suffix). 
Two examples are listed in Table \vref{syncreticInflFormPairsNoStemAllomorphy}.
\begin{table}\centering
\caption{Examples of syncretic inflectional form sets for Class III nouns without stem allomorphy}\label{syncreticInflFormPairsNoStemAllomorphy}
\resizebox{\columnwidth}{!}{
\begin{tabular}{| c | c | c | c | l |}\hline
\Sc{ill.sg/gen.pl/com.pl}	&\Sc{iness.pl/com.sg}&\Sc{ess/iness.sg}	&\Sc{nom.pl/gen.sg}	&\It{gloss}	\\\dline
almatjij				& almatjijn			& almatjin			& almatja(h)		& ‘person’	\\\hline
ålolij					& ålolijn			& ålolin			& ålola(h)			& ‘tool’ 	\\\hline
\end{tabular}
}
\end{table}

However, for nouns which have stem allomorphy (consonant gradation, umlaut and/or \jvh), different stem allomorphs are chosen for \Sc{ill.sg} than for \Sc{gen.pl} and \Sc{com.pl}, for \Sc{ess} than for \Sc{iness.sg}, and for \Sc{nom.sg} than for \Sc{nom.pl} and \Sc{gen.sg}.
As a result, only the inflected forms for
\begin{itemize}
\item{\Sc{gen.pl} and \Sc{com.pl}}
\item{\Sc{iness.pl} and \Sc{com.sg} (the \It{-jn} variant of the latter)}
\item{\Sc{nom.pl} and \Sc{gen.sg}}
\end{itemize}
%\caption{Homophonous inflected noun forms in all classes}\label{homoInflNounPairs}\end{figure}% form homophonous inflected noun forms in all inflectional classes,
are syncretic in all noun paradigms. Some examples are provided in Table \vref{syncreticInflFormPairExamples}.
%these homophonous suffixes only correlate with homophonous inflected noun forms in certain cases, depending on the extent of stem allomorphy accompanying these suffixes. Nonetheless, at the very least, the inflected noun forms listed in Figure \vref{homoInflNounPairs} are always homophonous pairs for all noun classes, due in part to the relevant homophonous suffix pairs.
%\begin{figure}\begin{enumerate}\item\Sc{nom.pl} and \Sc{gen.sg}\item\Sc{gen.pl} and \Sc{com.pl}\item\Sc{iness.pl} and \Sc{com.sg} (the \It{-jn} variant of the latter)\end{enumerate}\caption{Homophonous inflected noun forms in all classes}\label{homoInflNounPairs}\end{figure}% form homophonous inflected noun forms in all inflectional classes,
%Some examples of homophonous noun pairs are listed in Table \vref{syncreticInflFormPairExamples}. See Section \ref{nounStemAllomorphy} and \ref{nounUmlaut} as well as Table \vref{nounSuffixesWithStemAlts} below for a more detailed description of such syncretism.
\begin{table}\centering
\caption{Examples for syncretic inflectional form pairs valid for all noun classes}\label{syncreticInflFormPairExamples}
\resizebox{1\linewidth}{!} {
\begin{tabular}{| y{30mm} | y{30mm} | y{30mm} |c| p{20mm} |}\hline
\Sc{nom.pl/gen.sg}	&\Sc{gen.pl/com.pl}	&\Sc{iness.pl/com.sg}&\It{class}&\It{gloss} \\\dline
\It{luokta(-h)}		& \It{luokta-j}		& \It{luokta-jn}		&Ia & ‘bay’ \\
\It{vajmå(-h)}		& \It{vajmå-j}		& \It{vajmå-jn}		&Id & ‘heart’ \\
\It{guole(-h)}		& \It{guli-j}			& \It{guli-jn}		&Ie & ‘fish’ \\
\It{vágge(-h)}		& \It{väggi-j}		& \It{väggi-jn}		&Ie & ‘valley’ \\
\It{ålma(-h)}		& \It{ålma-j}		& \It{ålma-jn}		&II & ‘man’ \\
\It{almatja(-h)}		& \It{almatji-j}		& \It{almatji-jn}		&IIIa & ‘person’ \\%\hline
\It{bednaga(-h)}		& \It{bednagi-j}		& \It{bednagi-jn}	&IIIb & ‘dog’ \\\hline
\end{tabular}}
\end{table}

%However, homophonous noun suffixes do not necessarily correspond to homophonous inflectional noun forms, particularly concerning three sets of homophony in the noun suffix table in \ref{nounSuffixes}: 
%\begin{enumerate}\item the \It{null} suffix for \Sc{nom.sg} and \Sc{nom.pl}/\Sc{gen.sg}\item the \It{-j} suffix for \Sc{ill.sg} and \Sc{gen.pl}/\Sc{com.pl}\item the \It{-n} suffix for \Sc{ess} and \Sc{iness.sg}\end{enumerate}
%In other words, the \Sc{nom.sg} and \Sc{nom.pl} realizations of a noun normally differ in the quantity and/or quality of their respective stem consonants, as in the examples in Table \vref{homoSuffixHeteroNouns}. 
%\begin{table}\centering
%\begin{tabular}{| c | x{31mm} | x{31mm} | p{25mm} |}\hline
%%\MC{3}{|l|}{quantitative/qualitative differences:}\\\hline
%\It{suffix			&\MC{2}{c|}{\It{pattern 1 vs. pattern 2}	&\It{gloss \\\dline
%				&\BfSc{nom.sg}	&\BfSc{nom.pl/gen.sg}	& \\\cline{2-3}%\hline
%\multirow{3}{*}{\It{-ø}	&\It{vajbmå		& \It{vajmå			& ‘heart’ \\
%				&\It{guolle			& \It{guole			& ‘fish’ \\
%				&\It{luakkta		& \It{luokta			& ‘bay’ \\\hline
%				&\BfSc{ill.sg}	&\BfSc{gen.pl}	& \\\cline{2-3}%\hline
%\multirow{3}{*}{\It{-j}	&\It{vajbmåj?		& \It{vajmåj		& ‘heart’ \\
%				&\It{guolláj			& \It{gulij			& ‘fish’ \\
%				&\It{luäkktaj		& \It{luoktaj		& ‘bay’ \\\hline
%				&\BfSc{ess}		&\BfSc{iness.sg}	& \\\cline{2-3}%\hline
%\multirow{3}{*}{\It{-n}	&\It{vajbmån?		& \It{vajmån		& ‘heart’ \\
%				&\It{guollen		& \It{guolen		& ‘fish’ \\
%				&\It{luäkktan		& \It{luoktan		& ‘bay’ \\\hline
%%\MC{3}{|l|}{homophonous:}\\\hline
%%\BfSc{nom.sg}	&\BfSc{nom.pl}	&\bf gloss \\\hline
%%\It{vágge			& \It{vágge		& ‘valley’ \\\hline
%\end{tabular}
%\caption{Comparison of homophonous suffixes and corresponding heterophonous inflectional noun forms}\label{homoSuffixHeteroNouns}
%\end{table}
%The same holds for  \Sc{ill.sg} vs. \Sc{gen.pl}/\Sc{com.pl} and \Sc{ess} vs \Sc{iness.sg}. %Due to the lack of consonant gradation and initial vowel umlaut, inflectional noun class 6 ‘\It{vágge}’\marginpar{check this for class 6!} does have homophonous forms in \Sc{nom.sg} and \Sc{nom.pl} and \Sc{ess} and \Sc{iness.sg}.
%eg. \It{vajbmå} ‘heart.\Sc{nom.sg}’ $\sim$ \It{vajmå} ‘heart.\Sc{nom.pl}’, \It{guolle} ‘fish.\Sc{nom.sg}’ $\sim$ \It{guole} ‘fish.\Sc{nom.pl}’. 
%Only very few noun classes, such as \It{vágge} ‘valley’, have homophonous \Sc{nom.sg} and \Sc{nom.pl} forms, cf. the bottom half of Table \vref{NomSgVsNomPl}. 


\subsubsection{Nominal suffixes with a \It{-j} component}\label{JcomponentNounSuffixes}\index{suffixes!nominal}
When looking at the inflectional suffixes, it is noticeable that a number of suffixes contain a \It{-j} component, as highlighted by Table \vref{nounSuffixesWithJ}. 

It is tempting to posit a plural marking suffix \It{-j} because it occurs in \Sc{gen.pl},  \Sc{acc.pl}, \Sc{ill.pl},  \Sc{iness.pl}, \Sc{elat.pl} and \Sc{com.pl}; however, %UM comments ‘no, because case+number always go together and NOM is unmarked’:, it is not part of the \Sc{nominative} case marking for plural, which is problematic for such an analysis. Otherwise, the remaining plural cases do not %seem to 
%have any common feature which could be marked by this potential \It{-j} suffix%, with the exception of not being in \Sc{nom} case; but this negative definition is not particularly elegant
%. Furthermore, 
the \Sc{ill.sg} suffix \It{-j} and the \Sc{com.sg} suffix \It{-jn(a)} both have a similar \It{-j} element, but are clearly not plural. %UM delete:, at least in a grammatical sense. 
%UM points out that this footnote argument not valid because of e.g. ‘sugarijn’ with sugar: \footnote{The case could perhaps be made, that \Sc{com.sg} implies a semantically plural agent, at least in its true comitative sense, but it remains grammatically singular.} 
As illustrated in Table \vref{nounSuffixesWithJandUmlaut}, the plural cases with a \It{-j} component in the suffix trigger vowel harmony in stem consonants in Class Ie nouns,\footnote{Cf. Section \ref{NclassIe}.} but so does the \COMs.\SGs\ suffix, while the \Sc{ill.sg} suffix does not trigger \jvh\ (despite being segmentally identical to \GENs.\PLs\ and \COMs.\PLs\ suffixes). Thus, \It{-j} suffixes that trigger vowel harmony also fail to align with number marking. As a result, I do not analyze any \It{-j} suffix as a plural marker, but do point out this nearly pervasive plural pattern.\footnote{Cf. ‘eidemic resonance’ in \citet[209-210]{BickelNichols2007}.}
\begin{table}\centering%\scriptsize
\caption{Nominal case and number suffixes with a \It{-j-} segment}\label{nounSuffixesWithJ}
\begin{tabular}{ r | x{35mm} | x{35mm} | l }
%			&\MC{2}{c|}{\It{number}}			&	\\
\It{case}	& \Sc{singular} 	& \Sc{plural}		& \It{case} \\\hline\hline
\Sc{nom}	&   				&  				& \Sc{nom}\\\cline{2-3}
\Sc{gen}	&   				&  -j				& \Sc{gen}\\\cline{2-3}
\Sc{acc}	&  				&  -jt				& \Sc{acc}\\\cline{2-3}
\Sc{ill}	&  -j				&  -jda			& \Sc{ill}\\\cline{2-3}
\Sc{iness}	&  				&  -jn				& \Sc{iness}\\\cline{2-3}
\Sc{elat}	&  				&  -jst			& \Sc{elat}\\\cline{2-3}
\Sc{com}	&  -jn(a)			&  -j				& \Sc{com}\\\cline{2-3}
\Sc{abess}	& &		  				& \Sc{abess}\\\cline{2-3}
%\Sc{abess}	& \MC{2}{c |}{}		  				& \Sc{abess}\\\cline{2-3}
\Sc{ess}	& \MC{2}{c |}{}						& \Sc{ess}\\\hline%\hline
\end{tabular}
\end{table}
\begin{table}\centering%\scriptsize
\caption{Nominal case/number suffixes with a \It{-j-} segment and \It{j-suffix harmony}}\label{nounSuffixesWithJandUmlaut}
\begin{tabular}{ r | x{20mm} x{15mm} | x{20mm} x{15mm} | l }
%			&\MC{2}{c|}{\It{number}}			&	\\
\It{case}	&\It{j}-VH	& \Sc{sg}	&\It{j}-VH	& \Sc{pl}& \It{case} \\\hline\hline
\Sc{nom}	&		&		&  	&		& \Sc{nom}\\\cline{2-5}
\Sc{gen}	&		&		&  \CH	& -j		& \Sc{gen}\\\cline{2-5}
\Sc{acc}	&  		&		&  \CH	& -jt		& \Sc{acc}\\\cline{2-5}
\Sc{ill}	&\Bf{X}	& -j		&  \CH	& -jda	& \Sc{ill}\\\cline{2-5}
\Sc{iness}	&  		&		&  \CH	& -jn		& \Sc{iness}\\\cline{2-5}
\Sc{elat}	&  		&		&  \CH	& -jst		& \Sc{elat}\\\cline{2-5}
\Sc{com}	&  \CH		& -jn(a)	&  \CH	& -j		& \Sc{com}\\\cline{2-5}
\Sc{abess}	&  \MC{2}{c |}{}	&  \MC{2}{c |}{}		& \Sc{abess}\\\cline{2-5}
%\Sc{abess}	&  \MC{4}{c |}{}					& \Sc{abess}\\\cline{2-5}
\Sc{ess}	& \MC{4}{c |}{}					& \Sc{ess}\\\hline%\hline
\end{tabular}
\end{table}



\subsection{Non-linear noun morphology}\label{nonlinearNounMorph}\index{nouns!non-linear morphology}
In addition to the suffixes described above, most nouns are also marked for case and number by non-linear stem allomorphy (cf. Section \ref{morphophonology}). % alternations of stem consonants and vowels. 
%Concerning stem consonant alternations, this non-linear morphological marking is referred to as \It{consonant gradation}, and when stem vowels alternate, it is called \It{umlaut}. 
Because \Sc{nom.sg}, \Sc{nom.pl} and \Sc{gen.sg} lack suffixes completely,\footnote{As mentioned in Section \ref{nominalSuffixes} above, there is an optional \It{-h} suffix marking \Sc{nom.pl} and  \Sc{gen.sg}.} 
nouns in these three case/number categories can only be marked by non-linear morphology.

To illustrate this, the inflectional paradigm for the noun \It{bärrgo} ‘meat’ is provided in Table \vref{meatParadigm} and described here. Note that, due to the \It{-o-} vowel in V2 position in all forms being the inflectional class marker, the stem has two allomorphs: \It{bärrg-} and \It{bierg-}.%\footnote{Note that the examples used in this description of non-linear noun morphology is based on the current \PS\ orthography, which is still a work in progress. Particularly concerning the allomorphs of \It{bärrgo}, it should be pointed out that the grapheme <ä> refers to the phoneme /ɛ/, while <ie> refers to /e/ and <g> to /k/. However, because the orthography is to a great extent phonemic, the orthographic representations are sufficient for the current discussion.} 
\begin{table}\centering
\caption{The inflectional paradigm for the noun \It{bärrgo} ‘meat’}\label{meatParadigm}
\begin{tabular}{ |r | c | c | }\hline
			&\MC{2}{c|}{\It{number}}\\
\It{case}	& \Sc{singular}	& \Sc{plural}	 \\\dline
\Sc{nom}	& bärrgo			& biergo		\\\hline%\cline{2-3}
\Sc{gen}	& biergo			& biergoj		\\\hline%\cline{2-3}
\Sc{acc}	& biergov			& biergojd		\\\hline%\cline{2-3}
\Sc{ill}	& bärrgoj			& biergojda	\\\hline%\cline{2-3}
\Sc{iness}	& biergon			& biergojn		\\\hline%\cline{2-3}
\Sc{elat}	& biergost			& biergojst	\\\hline%\cline{2-3}
\Sc{com}	& biergojn			& biergo		\\\hline%\cline{2-3}
\Sc{abess}& biergodak		& biergodahta	\\\hline%\cline{2-3}
\Sc{ess}	&\MC{2}{c|}{bärrgon}\\\hline%\hline
\end{tabular}
\end{table}

%Traditionally in Saami linguistics, the stem allomorph with more segmental material in the consonant center\footnote{Cf. Section \ref{CCent} about the consonant center.} 
%(i.e., with a geminate consonant or with three consecutive consonants) is called the strong grade, while the shorter allomorph (i.e., with a singleton or with only two consecutive consonants) is termed the weak grade; this terminology is adopted here, as well.

In summary, the inflectional paradigm for \It{bärrgo} is characterized by both consonant gradation and umlaut in the stem, %\footnote{As described in Section \ref{}, umlaut only occurs in stems with consonant gradation, but consonant gradation can occur without umlaut.} %JW: NOT true, cf. bägga-biegga (wind)
and the morphological environment determines which of these allomorphs is selected. %\footnote{Historically, the choice of stem allomorphs was determined phonologically, depending on whether the second syllable was open or closed (\cite{Sammallahti1998}:191,etc.). However, this process is no longer a productive, and these stem alternations are now morphologically determined.} 
As a result, the \Sc{acc.pl} form \It{biergojd} is marked for case/number by the weak \It{bierg-} stem and the \mbox{\It{-jd}} suffix simultaneously, and the \Sc{ill.sg} form \It{bärrgoj} is marked by the strong \It{bärrg-} stem and the \It{-j} suffix. The most obvious evidence that the choice of stem allophone is morphologically meaningful can be found in a comparison of the \Sc{nom.sg} form \It{bärrgo} and the \Sc{nom.pl}\footnote{As the \Sc{nom.pl} form is always syncretic with the \Sc{gen.sg} form, a comparison of the latter with the \Sc{nom.sg} form would be equally insightful.} 
form \It{biergo}. These forms differ exclusively in the choice of the strong versus the weak stem allomorph and in the choice of umlaut. Thus, the \Sc{nom.sg} form \It{bärrgo} is marked for case/number by the fact that the stem is in the strong grade and features the vowel \It{ä}, while the \Sc{nom.pl} stem is in the weak grade and features the vowel \It{ie}. 

This pattern of non-linear case/number marking throughout the paradigm for \It{bärrgo} is illustrated in Table \vref{meatParadigmPattern}.
\begin{table}\centering
\caption{Non-linear morphological case/number marking in the paradigm for the noun \It{bärrgo} ‘meat’}\label{meatParadigmPattern}
\begin{tabular}{ |r || c  c | }\hline
			&\MC{2}{c|}{\It{number}}\\
\It{case}	& \MC{1}{c|}{\Sc{singular}}			& \Sc{plural}	 \\\dline
\Sc{nom}	& \MC{1}{c|}{ ä\PLUS str}	& \MR{8}{*}{ie\PLUS wk}\\\cline{1-2}
\Sc{gen}	&  \MR{2}{*}{}						& \\\cline{1-1}
\Sc{acc}	& 								& \\\cline{1-2}
\Sc{ill}		& \MC{1}{c|}{ ä\PLUS str}	& \\\cline{1-2}
\Sc{iness}	&  \MR{4}{*}{}						& \\\cline{1-1}
\Sc{elat}	& 								& \\\cline{1-1}
\Sc{com}	& 								& \\\cline{1-1}
\Sc{abess}	& 								& \\\hline%\cline{1-2}
%\Sc{nom}	& ä\PLUS strong	& \MR{8}{*}{ie\PLUS weak}\\\hline%\cline{2-3}
%\Sc{gen}	& ie\PLUS weak	& ie\PLUS weak\\\hline%\cline{2-3}
%\Sc{acc}	& ie\PLUS weak	& ie\PLUS weak\\\hline%\cline{2-3}
%\Sc{ill}		& ä\PLUS strong	& ie\PLUS weak\\\hline%\cline{2-3}
%\Sc{iness}	& ie\PLUS weak	& ie\PLUS weak\\\hline%\cline{2-3}
%\Sc{elat}	& ie\PLUS weak	& ie\PLUS weak\\\hline%\cline{2-3}
%\Sc{com}	& ie\PLUS weak	& ie\PLUS weak\\\hline%\cline{2-3}
%\Sc{abess}	& ie\PLUS weak	& ie\PLUS weak\\\hline%\cline{2-3}
\Sc{ess}	&\MC{2}{c|}{ ä\PLUS str}\\\hline%\hline
\end{tabular}
\end{table}
Here, two patterns are manifest: the forms for \Sc{nom.sg}, \Sc{ill.sg} and \Sc{ess} show one pattern, while all other case/number combinations exhibit the other pattern. This alignment of the choice of stem allomorphs is more or less prevalent throughout \PS\ noun paradigms whenever stem allomorphy is a part of a noun’s inflectional paradigm. 
Note, however, that not every noun undergoes consonant gradation and/or umlaut; instead, their presence are determined by the phonological form of a noun. Consonant gradation is described in detail in Section \ref{Cgrad}, and umlaut in Section \ref{umlaut}; some examples of nouns with consonant gradation and umlaut alternations are shown in Table \vref{CGpatterns} and Table \vref{umlautPatterns}, respectively. 
%Furthermore, the specific consonant gradation and umlaut alternations that occur vary, and are described briefly here. %in the following two sections. 
%%%%UM schlägt folgende Bemerkung vor: ‘ Note that some singular forms are more marked than the corresponding plural forms and thus do not show the iconic markedness pattern of number marking that is prevailing in the World's languages in which the plural is morphophonologically more marked than the singular.’%JW: aber ist SG wirklich markierter als PL?

%Note that there are numerous nouns lacking umlaut. Furthermore, a smaller number of nouns exist which lack consonant gradation. A few nouns lack both consonant gradation and umlaut. 
%At least one noun (\It{eddno} ‘maternal uncle’) lacking both consonant gradation and umlaut is attested as well, but the data at this point is non sufficiently extensive to identify any more such examples.

%\subsubsection{Consonant gradation patterns}\label{CGpatternsection}
%While consonant gradation is described in detail in Section \ref{Cgrad}, Table \vref{CGpatterns} provides some examples of such stem consonant alternations in nouns. The examples provided show minimal pairs differing only in the choice of stem allomorph (here, the \Sc{nom.sg} form vs. the \Sc{nom.pl} form for nouns in class I are presented). 
%Note that the presence or absence of consonant gradation for a given noun is not dependent on the inflection class it is in. 
\begin{table}\centering
\caption{Examples of consonant center gradation patterns for nouns}\label{CGpatterns}
\begin{tabular}{|c c c || c c c | l|}\hline
\MC{3}{|c||}{\It{pattern}}	&\MC{3}{c}{\It{examples}}&	\\
strong&\Div &weak	& strong	&\Div &weak	&\It{gloss}\\\dline
ʰx	&\Div &x		&/tɔʰpe/	&\Div &/tɔpe/	& ‘house’\\%bm-m
	&&		&\It{dåhpe	}&&\It{dåbe}&\\\hline
xː	&\Div &x		&/kolːe/	&\Div &/kole/	& ‘fish’\\%ll-l
	&&		&\It{guolle}&&\It{guole}&\\\cline{4-7}
	&&		&/naːʰpːe/	&\Div &/naːʰpe/	& ‘milking bowl’\\%bm-m
	&&		&\It{náhppe}&&\It{náhpe}&\\\hline
xːy	&\Div & xy	&/ɲaːrːka/	&\Div &/ɲaːrka	/& ‘cape’ (geog.)\\%rrg-rg
	&&		&\It{njárrga}&&\It{njárga}&\\\hline
xy	&\Div &y		&/etno/	&\Div &/eno/	& ‘river’\\%bm-m
	&&		&\It{edno}	&&\It{eno}&\\\hline
xyz	&\Div & xz	&/vaːjpmo/&\Div &/vaːjmo/	& ‘heart’\\%\hline%jbm-jm
	&&		&\It{vájbmo}&&\It{vájmo}&\\\hline
\MC{1}{c}{}	&&\MC{1}{c|}{}	&\NOMs.\SGs&\Div &\NOMs.\PLs&\MC{1}{c}{}\\\cline{4-6}
%\begin{tabular}{c c c}
%strong&\Div &weak\\\dline
%xx	&\Div &x	\\%ll-l
%xy	&\Div &y	\\%bm-m
%xxy	&\Div & xy	\\%rrg-rg
%xyz	&\Div & xz	\\\hline%jbm-jm
\end{tabular}
\end{table}
%Here, \It{x}, \It{y} and \It{z} stand for different consonant segments. The examples provided show minimal pairs differing only in the choice of stem allomorph (the \Sc{nom.sg} form vs. the \Sc{nom.pl} form). 


%\subsubsection{Umlaut patterns}\label{umlautPatternsection}
%Umlaut in \PS\ is described in more detail in Section \ref{}. 
%The two attested umlaut patterns in the V1 vowel of a noun stem are illustrated by examples in Table \vref{umlautPatterns}. % on page \pageref{umlautPatterns}.
\begin{table}\centering
\caption{Umlaut alternation patterns for nouns}\label{umlautPatterns}
\begin{tabular}{|c c c || c c c | l|}\hline
\MC{3}{|c||}{\It{pattern}}	&\MC{3}{c}{\It{examples}}&	\\
A&\Div &B		&A	&\Div &B	&\It{gloss}\\\dline
ɛ	&\Div &e		&/pɛkːa/	&\Div &/pekːa/	& ‘wind’\\%l
	&&		&\It{bägga}	&&\It{biegga}	& \\\hline%l
u͡a	&\Div &o		&/lu͡akːta/	&\Div &/lokta/	& ‘bay’\\%
	&&		&\It{luakkta}&&\It{luokta}	& \\\hline%
\MC{1}{c}{}&&\MC{1}{c|}{}&\NOMs.\SGs&\Div &\NOMs.\PLs&\MC{1}{c}{}\\\cline{4-6}
\end{tabular}
\end{table}
%There are only a few examples of noun paradigms that have umlaut and lack consonant gradation; in most cases of umlaut, consonant gradation is also present. %Particularly concerning the \It{ua-uo} pattern, the data in the corpus is to limited to be conclusive, and must be left to future study. However, it seems likely that umlaut and consonant gradation occur independently of one another, and should be treated separately. 


\subsection{Problematic case/number marking in abessive case}\label{abessiveProblematic}\index{cases!abessive}
%1. problematic
% because: rare, thus inconsistent results
% mention old lit in footnote
%adding to confusion:
%2. allomorphy in suffixes (some of which are bisyllabic, thus a foot)
%3. question of number inflection
Unlike the other cases, the behavior of nouns in the abessive case is a bit of an enigma, even if its meaning, which typically translates as ‘without’, is quite clear.\footnote{Cf. Section \ref{abessive}.} 
Indeed, it is difficult to come to any certain conclusions about the relationship between abessive as a case \It{per se} and the morphophonological marking of nouns in the abessive case. It seems to be rarely used in natural speech, and is only attested in the corpus in elicitation sessions. Even in elicitation sessions, language consultants were often hesitant or uncertain of the word forms they produced, and often produced conflicting forms for a single item. Indeed, the slipperiness of the abessive case is nothing new, as both \citet{Lagercrantz1926} and \citet{Lehtiranta1992} only provide incomplete treatments of abessive. 

One potential source of the confusion (even on speakers’ behalf) is the fact that abessive suffixes are unique in two ways. 
First, there is significant allomorphy, and, secondly, some of the allomorphs are the only bisyllabic nominal inflection suffixes in \PS. %\footnote{Note that the \Sc{comitative singular} suffix has two forms \It{-jn} and \It{-jna} (the latter being bisyllabic), but these seem to be in free variation with one another.} %JW: nonsense!
The attested forms are \It{-dak}, \It{-daga}, \It{-gat}, \It{-gahta} and \It{-ahta} (cf. examples \REF{abess1} and \REF{abess2} on page \pageref{abess1}). 
Furthermore, the weak grade usually accompanies abessive, but sometimes the strong grade does. In some cases of Class Ie nouns, \jvh\ is triggered, in others it is not. In some cases, number is clearly marked, in other cases, there is no distinction between singular and plural.

As a result, the following sections on inflectional noun classes are only able to provide a limited and preliminary description concerning abessive. 
%UM: delete, bei einer aussterbenden sprache hat man für solche obsoleten formen kaum chancen: %Future research must be done to attain more certainty about this case.
%poor inventory of ABESS forms from corpus:
%DS:
%1. sámedak, sämijdak (SG/PL) várijdak, värijdak, jávredak, jävrijdak, náhpedak, nähpijdak, vággedak, vággijdak
%2. luoktodak, luoktodaga (SG/PL), guoladak, guoladaga, guolastak, vájmodak, vájmodaga, sáltedak, sáltedaga, juolgedak, juolgedaga, rejdodak, rejdodaga
%3. vuostadak, vuostajdak (SG/PL), but uncertain
%4. buhtsudak, båtsuojdak (SG/PL) ??, buhtsudaga
%5. guoledak
%6. bårodak (w/o eating)
%7. ednojdak, ednojdaga
%8. enodak
%9. ålmagat, ålmagahta, ålmadak, ålmadaga
%ER:
%1. sämijdak, sämijdaga (SG/PL)
%2. bednadaga, bednagahta (SG/PL) (same for other Class III)
%3. skåvlåjdaga, skåvlåjdahta (SG/PL) biergodaga, biergodahta, luoktadak, luoktajdag
%4. guoledaga (SG)
%5. suhkorahta
%ME:
%1. biergodak

%The status of the \Sc{abessive} case, meaning essentially ‘without’, is uncertain\marginpar{this discussion on ABESS is very rough!}. It is often included as a case in descriptions of other Saamic languages\footnote{cf. \cite{Sammallahti1998} for North Saami dialects, some of which still have an abessive case, others only have an abessive postposition; \cite{Spiik1989} for Lule Saami (including a very brief discussion of the semantically (and cognate!) postposition \It{dagi} ‘without’)} and even for Pite\footnote{cf. \cite{Lehtiranta1992}, \cite{Lagercrantz1926}, \cite{Halasz1896}.}, but speakers are not consistent in their usage, partly because it seems to be a rather infrequent construction. %Possibly a better analysis would be that it is a postposition \It{dahk} or \It{daka} governing \Sc{gen.sg} or \Sc{gen.pl}, but it also inflects for number itself (-dak/-dahk SG, -daga PL), which no other postpositions do. %Use of a \Sc{plural} \It{-j} very inconsistent.




%There are several non-linear aspects of noun morphology: stem vowel alternations, stem consonant alternations, \It{j-}suffix harmony and trisyllabicity. Because these are intricately involved in defining inflectional noun classes, they are described in more detail in the relevant parts of Section \ref{nounInflectionClasses} below.


\section{Inflectional classes for nouns}\label{nounClasses}\index{inflectional classes!nouns}
Nouns in \PS\ can be grouped into three main inflectional classes, with several subclasses, based on recurring patterns across case/number inflectional paradigms. Each noun is marked by a class suffix\footnote{I am indebted to phonologist and Lule Saami scholar Bruce Morén-Duolljá for inspiring me to consider an approach to the data involving post-stem class marking morphology.} 
which is attached directly after the noun stem and precedes case/number suffixes (cf. Figure \vref{nounStructure}). For the majority of nouns, this suffix consists only of a vowel (in V2 position); however, the class marking suffixes in the less frequent classes II and III deviate from this pattern. 
\korr{006}The presence of umlaut alternations and/or consonant gradation for a given noun is not dependent on the noun’s membership in a specific class, but is determined by whether the phonemes occupying the V1 position and the consonant center of the final foot, respectively, are susceptible to umlaut and/or consonant gradation. Furthermore, some derivational suffixes (such as the diminutive suffix \It{-tj}) can block consonant gradation and umlaut from happening in the new derived form. 
%\Red{but derivational suffixes block umlaug/c-grad, too! e.g.: guolátj -- not -ll-!!} 
Note that membership in a specific noun class does not seem to be semantically motivated. 

The following sections present the four inflectional noun classes based on a preliminary analysis of the corpus; it is possible that, with more research, more noun classes may result, or that the present classes may need revision. Because each noun paradigm consists of seventeen inflectional forms, most of the data on which these classes are based comes from elicitation sessions, as it is far beyond realistic for a single, non-native-speaker linguist to collect a sufficiently large natural (i.e., un-elicited, spontaneous) spoken language corpus which includes all inflectional forms for a large variety of nouns.

There are two main criteria for positing the different noun classesː
\begin{itemize}
\item{the allomorphy of the \Sc{nom.sg} form of a noun stem in relation to the rest of the inflectional paradigm (i.e., consonant gradation, umlaut)}
\item{the regularity of the pattern of vowels occurring between the stem and case/number suffixes (i.e., the class marking suffix)}
%\item{whether a noun is subject to vowel harmony triggered by certain case/number suffixes (i.e., \jvh)}
\end{itemize} 
To illustrate these differences, it is sufficient to look at the class suffix in \Sc{nom.sg} and the alignment of consonant gradation allomorphs, as summarized in  
% is the pattern of the segmental phonology occurring between the noun stem and the case/number suffixes, i.e., the class marking suffix. For some noun classes, this suffix is consistent in all paradigm slots; for others, this suffix is determined by the case/number of the noun form. Furthermore, supplementary criteria differentiating nouns from one another includes the pattern of consonant gradation alignment and the presence of \It{-j}-suffix vowel harmony. 
Table \vref{nounClassSummary}. %summarizes the noun inflectional classes. 
The header \It{grade alignment} refers to the choice of stem allomorph in \Sc{nom.sg} versus \Sc{nom.pl} whenever consonant gradation is relevant for a specific noun paradigm: the value ‘str-wk’ indicates that \Sc{nom.sg} is marked by the strong grade and \Sc{nom.pl} by the weak grade, while ‘wk-str’ is the opposite, inverted pattern. 
%The feature ‘\jvh’ (abbreviated \It{j-VH}) indicates whether vowel harmony in V1 and V2 vowels exists in the presence of certain case/number suffixes with a \It{-j-} element (cf. Section \ref{jSuffixHarmony} below). 
%Here, \Bf{V} stands for a vowel phoneme. %, and \Bf{ø} for an unmarked form. 
\begin{table}\centering
\caption{Summary of noun classes and their defining features}\label{nounClassSummary}
\begin{tabular}{c c c}
		&\It{grade} 	&\It{class suffix}		\\
\It{class}	&\It{alignment}	&\It{in} \Sc{nom.sg}\\\dline
I		&str-wk		&-a/á/o/å/e		\\\hline
II		&wk-str		&-Vj	\\\hline
III		&wk-str		&-	\\\hline
\end{tabular}
%\begin{tabular}{c c c c}
%		&\Sc{nom.sg	&\It{grade		&	\\
%\It{class	&\It{class suffix	&\It{alignment	&\It{j-VH	\\\dline
%I		&-a/á/o/å		& str-wk		& -		\\\hline
%%II		&-o			& str-wk		& -		\\\hline
%%III		&-å			& str-wk		& -		\\\hline
%II		&-e			& str-wk		& \PLUS		\\\hline
%III		&-Vj			& wk-str		& -		\\\hline
%III		&-			& wk-str		& -		\\\hline
%\end{tabular}
\end{table}

%Furthermore, the existence of \jvh\ ...

Class I is a sort of default class %in which no morphophonology beyond consonant gradation and umlaut occurs within the noun paradigm, 
and is therefore dealt with first in Section \ref{NclassI}, %Because Class I are very similar, they are dealt with together in the first section below (Section \ref{NclassI2III}), 
while classes II and III are described in sections \ref{NclassII} and \ref{NclassIII}. The final section (\ref{summaryNounClasses}) provides a brief summary of the noun classes, including a table listing examples from each class. 

%\Red{It may be best to combine classes I-II-III into one class - they’re identical except for the class-vowel. Maybe like this: other criteria missing, this is the default pattern.}
%Because of their strikingly different morphophonological behavior, it is useful to divide noun classes into two large groups based on the number of syllables in the \Sc{nominative plural} inflectional form, and this is reflected in the structure of the following description. Section \ref{} describes bisyllabic/even-syllabled noun classes, and Section \ref{} describes trisyllabic/odd-syllabled noun classes. 


%\subsection{Bisyllabic noun classes}\label{bisyllabicNounClasses}
%All \Red{four} bisyllabic noun classes

\subsection{Class I}\label{NclassI}
Nouns in Class I are characterized by:
\begin{itemize}
\item{the default \It{str-wk} gradation pattern}
%\item{lacking class suffix allomorphy within a paradigm}
%\item{lacking vowel harmony}
\end{itemize}
%The class-marking suffixes for nouns in \Bf{Class I} do not exhibit allomorphy. Because all nouns which do not exhibit any morphophonology in the noun paradigm beyond consonant gradation and vowel harmony are in this class, it is considered a kind of ‘default’ class. It is a very common class. 
Class I nouns can be divided into five subclasses, depending on the class-marking vowel they have, as illustrated in Table \vref{Nclass1summary}. The first four subclasses behave very similarly, while subclass Ie is unique due to the presence of \jvh, as discussed in detail in Section \ref{NclassIe}. 
For subclasses Ia, Ib, Ic and Id, the class marking suffix is invariable throughout the paradigm, while the class marking suffix for subclass Ie varies due to \jvh. %\marginpar{eine kurze liste von anderen Nomen für jede Klasse nötig/nützlich?}%UM: ja; JW: ist jetzt auch dabei, im letzten abschnitt
\begin{table}\centering
\caption{Subclasses of Class I nouns and their class marking suffixes}\label{Nclass1summary}
\begin{tabular}{c | c}
\It{subclass}	&\It{class marking suffix} \\\dline
Ia		&-a-	\\
Ib		&-á-	\\
Ic		&-o-	\\
Id		&-å-	\\
Ie		&-e,i,á-	\\
\end{tabular}
\end{table}


For nouns with consonant gradation and/or umlaut in Class I, \Sc{nom.sg}, \Sc{ill.sg} and \Sc{ess} are in the strong grade and have \It{ä} or \It{ua/uä} in V1 position, while other case/number slots in a paradigm have the weak grade and \It{ie} or \It{uo} in V1 position. 
The gradation pattern and class marking suffixes for Class I are summarized in Table \vref{NclassIsuffixes}. Here, V stands for the vowel which comprises the suffix for each class (i.e., \It{a} for Class Ia, \It{á} for Class Ib, \It{o} for Class Ic, \It{å} for Class Id and \It{e/i/á} for Class Ie).
\begin{table}\centering
\caption{The consonant gradation pattern and inflectional noun class suffixes for Class I}\label{NclassIsuffixes}
\begin{tabular}{ |r || c | c || c | c | }\hline
			&\MC{4}{c|}{\It{number}}\\
			&\MC{2}{c||}{\Sc{singular}}&\MC{2}{c|}{\Sc{plural}}	 \\%\dline
\It{case}	&\It{C-grad}&\It{class suffix}&\It{C-grad}&\It{class suffix}	 \\\dline
\Sc{nom}	&str		& -V			&wk		& -V		\\\hline%\cline{2-3}
\Sc{gen}	&wk		& -V			&wk		& -V-		\\\hline%\cline{2-3}
\Sc{acc}	&wk		& -V-			&wk		& -V-		\\\hline%\cline{2-3}
\Sc{ill}		&str		& -V-			&wk		& -V-		\\\hline%\cline{2-3}
\Sc{iness}	&wk		& -V-			&wk		& -V-		\\\hline%\cline{2-3}
\Sc{elat}	&wk		& -V-			&wk		& -V-		\\\hline%\cline{2-3}
\Sc{com}	&wk		& -V-			&wk		& -V-		\\\hline%\cline{2-3}
\Sc{abess}	&wk		& -V-			&wk		& -V-		\\\hline%\cline{2-3}
\MR{2}{*}{\Sc{ess}}&\MC{2}{c|}{\It{C-grad}}	&\MC{2}{c|}{\It{class suffix}}	\\\cline{2-5}
		&\MC{2}{c|}{str}	&\MC{2}{c|}{-V-}\\\hline%\hline
\end{tabular}
\end{table}

% The Class Ia suffix is \It{-a}, the Class IIb suffix is \It{-o}, and the Class III suffix is \It{-å}. In all other respects, nouns in these classes behave the same paradigmatically, and are therefore treated together here. % are consistently marked by a the \It{-a} class marking suffix in all case/number slots of the paradigm. 
Four inflectional paradigms can be found in the tables on \marginpar{check pagerefs here to tables make sense}pages \pageref{bayParadigm} and \pageref{schoolParadigm} to illustrate subclasses Ia, Ib, Ic and Id. 
The paradigm for the word \It{luakkta} ‘bay’ is provided in Table \ref{bayParadigm} %on page \pageref{bayParadigm} 
as an example for a Class Ia noun, while the Class Ib noun \It{mánná} ‘child’ is found in Table \ref{childParadigm}, %on page \pageref{childParadigm}, 
Table \ref{foodParadigm} %on page \pageref{foodParadigm} 
shows the paradigm for \It{bäbbmo} ‘food’, a Class Ic noun, and the Class Id noun \It{skåvvlå} ‘school’ is illustrated in Table \ref{schoolParadigm}. %on page \pageref{schoolParadigm}.
Because subclass Ie is more complex due to \jvh, it is discussed separately in Section \ref{NclassIe}. 

\begin{table}\centering
\caption{The inflectional paradigm for the Class Ia noun \It{luakkta} ‘bay’}\label{bayParadigm}
\begin{tabular}{ |r | c | c | }\hline
			&\MC{2}{c|}{\It{number}}\\
\It{case}	& \Sc{singular}	& \Sc{plural}	 \\\dline
\Sc{nom}	& luakkt-a			& luokt-a		\\\hline%\cline{2-3}
\Sc{gen}	& luokt-a			& luokt-a-j		\\\hline%\cline{2-3}
\Sc{acc}	& luokt-a-v		& luokt-a-jd	\\\hline%\cline{2-3}
\Sc{ill}		& luakkt-a-j		& luokt-a-jda	\\\hline%\cline{2-3}
\Sc{iness}	& luokt-a-n		& luokt-a-jn	\\\hline%\cline{2-3}
\Sc{elat}	& luokt-a-st		& luokt-a-jst	\\\hline%\cline{2-3}
\Sc{com}	& luokt-a-jn		& luokt-a-j		\\\hline%\cline{2-3}
\Sc{abess}	& luokt-a-dak		& luokt-a-daga	\\\hline%\cline{2-3}
\Sc{ess}	&\MC{2}{c|}{luakkt-a-n}\\\hline%\hline
\end{tabular}
\end{table}

%%CLASS Ib
\begin{table}\centering
\caption{The inflectional paradigm for the Class Ib noun \It{mánná} ‘child’}\label{childParadigm}
\begin{tabular}{ |r | c | c | }\hline
			&\MC{2}{c|}{\It{number}}\\
\It{case}	& \Sc{singular}	& \Sc{plural}	 \\\dline
\Sc{nom}	& mánn-á		& mán-á		\\\hline%\cline{2-3}
\Sc{gen}	& mán-á		& mán-á-j		\\\hline%\cline{2-3}
\Sc{acc}	& mán-á-v		& mán-á-jd	\\\hline%\cline{2-3}
\Sc{ill}	& mánn-á-j	& mán-á-jda	\\\hline%\cline{2-3}
\Sc{iness}	& mán-á-n		& mán-á-jn	\\\hline%\cline{2-3}
\Sc{elat}	& mán-á-st	& mán-á-jst	\\\hline%\cline{2-3}
\Sc{com}	& mán-á-jn	& mán-á-j		\\\hline%\cline{2-3}
\Sc{abess}	& ?		& ?	\\\hline%\cline{2-3}
\Sc{ess}	&\MC{2}{c|}{?}\\\hline%\hline
\end{tabular}
\end{table}



%%CLASS Ic
\begin{table}\centering
\caption{The inflectional paradigm for the Class Ic noun \It{bäbbmo} ‘food’}\label{foodParadigm}
\begin{tabular}{ |r | c | c | }\hline
			&\MC{2}{c|}{\It{number}}\\
\It{case}	& \Sc{singular}	& \Sc{plural}	 \\\dline
\Sc{nom}	& bäbbm-o		& biebm-o		\\\hline%\cline{2-3}
\Sc{gen}	& biebm-o			& biebm-o-j		\\\hline%\cline{2-3}
\Sc{acc}	& biebm-o-v		& biebm-o-jd	\\\hline%\cline{2-3}
\Sc{ill}		& bäbbm-o-j		& biebm-o-jda	\\\hline%\cline{2-3}
\Sc{iness}	& biebm-o-n		& biebm-o-jn	\\\hline%\cline{2-3}
\Sc{elat}	& biebm-o-st		& biebm-o-jst	\\\hline%\cline{2-3}
\Sc{com}	& biebm-o-jn		& biebm-o-j		\\\hline%\cline{2-3}
\Sc{abess}	& biebm-o-dak		& biebm-o-daga	\\\hline%\cline{2-3}
\Sc{ess}	&\MC{2}{c|}{bäbbm-o-n}\\\hline%\hline
\end{tabular}
\end{table}

%%CLASS Id
\begin{table}\centering
\caption{The inflectional paradigm for the Class Id noun \It{skåvvlå} ‘school’}\label{schoolParadigm}
\begin{tabular}{ |r | c | c | }\hline
			&\MC{2}{c|}{\It{number}}\\
\It{case}	& \Sc{singular}	& \Sc{plural}	 \\\dline
\Sc{nom}	& skåvvl-å			& skåvl-å		\\\hline%\cline{2-3}
\Sc{gen}	& skåvl-å			& skåvl-å-j		\\\hline%\cline{2-3}
\Sc{acc}	& skåvl-å-v		& skåvl-å-jd	\\\hline%\cline{2-3}
\Sc{ill}		& skåvvl-å-j		& skåvl-å-jda	\\\hline%\cline{2-3}
\Sc{iness}	& skåvl-å-n		& skåvl-å-jn	\\\hline%\cline{2-3}
\Sc{elat}	& skåvl-å-st		& skåvl-å-jst	\\\hline%\cline{2-3}
\Sc{com}	& skåvl-å-jn		& skåvl-å-j		\\\hline%\cline{2-3}
\Sc{abess}	& skåvl-å-dak		& skåvl-å-daga	\\\hline%\cline{2-3}
\Sc{ess}	&\MC{2}{c|}{skåvvl-å-n}\\\hline%\hline
\end{tabular}
\end{table}


%\subsection{Class II}\label{NclassIe}
%Nouns in \Bf{Class II} are consistently marked by the \It{-o} class marking suffix in all case/number slots of the paradigm. The inflectional paradigm for the word \It{bäbbmo} ‘food’ is provided as an example in Table \vref{foodParadigm}.
%\begin{table}\centering
%\begin{tabular}{ |r | c | c | }\hline
%			&\MC{2}{c|}{\It{number}\\
%\It{case}	& \Sc{singular}	& \Sc{plural}	 \\\dline
%\Sc{nom}	& bäbbm-o		& biebm-o		\\\hline%\cline{2-3}
%\Sc{gen}	& biebm-o			& biebm-o-j		\\\hline%\cline{2-3}
%\Sc{acc}	& biebm-o-v		& biebm-o-jd	\\\hline%\cline{2-3}
%\Sc{ill}		& bäbbm-o-j		& biebm-o-jda	\\\hline%\cline{2-3}
%\Sc{iness}	& biebm-o-n		& biebm-o-jn	\\\hline%\cline{2-3}
%\Sc{elat}	& biebm-o-st		& biebm-o-jst	\\\hline%\cline{2-3}
%\Sc{com}	& biebm-o-jn		& biebm-o-j		\\\hline%\cline{2-3}
%\Sc{abess}	& biebm-o-dak		& biebm-o-daga	\\\hline%\cline{2-3}
%\Sc{ess}	&\MC{2}{c|}{bäbbm-o-n}\\\hline%\hline
%\end{tabular}
%\caption{The class-marker suffix and case/number paradigm for the Class II noun \It{bäbbmo} ‘food’}\label{foodParadigm}
%\end{table}
%
%For nouns with consonant gradation and/or umlaut in Class I, \Sc{acc.sg}, \Sc{ill.sg} and \Sc{ess} are in the strong grade and have \It{ä} or \It{ua/uä} in V1 position, while other case/number slots have the weak grade and \It{ie} or \It{uo} in V1 position, respectively.
%
%
%\subsection{Class III}\label{NclassIII}
%Nouns in \Bf{Class III} are consistently marked by the \It{-å} class marking suffix in all case/number slots of the paradigm. The inflectional paradigm for the word \It{skåvvlå} ‘school’ is provided as an example in Table \vref{schoolParadigm}.
%\begin{table}[H]\centering
%\begin{tabular}{ |r | c | c | }\hline
%			&\MC{2}{c|}{\It{number}\\
%\It{case}	& \Sc{singular}	& \Sc{plural}	 \\\dline
%\Sc{nom}	& skåvvl-å			& skåvl-å		\\\hline%\cline{2-3}
%\Sc{gen}	& skåvl-å			& skåvl-å-j		\\\hline%\cline{2-3}
%\Sc{acc}	& skåvl-å-v		& skåvl-å-jd	\\\hline%\cline{2-3}
%\Sc{ill}		& skåvvl-å-j		& skåvl-å-jda	\\\hline%\cline{2-3}
%\Sc{iness}	& skåvl-å-n		& skåvl-å-jn	\\\hline%\cline{2-3}
%\Sc{elat}	& skåvl-å-st		& skåvl-å-jst	\\\hline%\cline{2-3}
%\Sc{com}	& skåvl-å-jn		& skåvl-å-j		\\\hline%\cline{2-3}
%\Sc{abess}	& skåvl-å-dak		& skåvl-å-daga	\\\hline%\cline{2-3}
%\Sc{ess}	&\MC{2}{c|}{skåvvl-å-n}\\\hline%\hline
%\end{tabular}
%\caption{The class-marker suffix and case/number paradigm for the Class III noun \It{skåvvlå} ‘school’}\label{schoolParadigm}
%\end{table}
%
%For nouns with consonant gradation and/or umlaut in Class I, \Sc{acc.sg}, \Sc{ill.sg} and \Sc{ess} are in the strong grade and have \It{ä} or \It{ua/uä} in V1 position, while other case/number slots have the weak grade and \It{ie} or \It{uo} in V1 position, respectively.


\FloatBarrier

\subsubsection{Class Ie}\label{NclassIe}
Class Ie nouns are a special subset of class I nouns due to two features:
\begin{itemize}
\item{\jvh}
\item{allophony in the class markers (\It{e}, \It{i}, \It{á})}
\end{itemize}
%The choice of class suffix allomorphs (e, i, á) depends on the case/number slot of the paradigm. Furthermore, nouns in this class are subject to \jvh\ (described below in \ref{jSuffixHarmony}). 
The inflectional paradigm for the words \It{guolle} ‘fish’ and \It{vágge} ‘valley’ are provided as examples for Ie nouns in Table \vref{fishValleyParadigm}; note that \It{vágge} is not subject to consonant gradation. 
For nouns with consonant gradation and/or umlaut in Class Ie, \Sc{nom.sg}, \Sc{ill.sg} and \Sc{ess} are in the strong grade and have \It{ä} or \It{ua/uä} in V1 position, while other case/number slots have the weak grade and \It{ie} or \It{uo} in V1 position, just as with all Class I nouns. 
\begin{table}\centering
\caption{The inflectional paradigms for the Class Ie nouns \It{guolle} ‘fish’ and \It{vágge} ‘valley’}\label{fishValleyParadigm}
\begin{tabular}{ |r | c | c | }\hline
			&\MC{2}{c|}{\It{number}}\\
\It{case}	& \Sc{singular}	& \Sc{plural}	 \\\dline
\Sc{nom}	& guoll-e			& guol-e		\\\hline%\cline{2-3}
\Sc{gen}	& guol-e			& gul-i-j		\\\hline%\cline{2-3}
\Sc{acc}	& guol-e-v			& gul-i-jd		\\\hline%\cline{2-3}
\Sc{ill}		& guoll-á-j			& gul-i-jda		\\\hline%\cline{2-3}
\Sc{iness}	& guol-e-n			& gul-i-jn		\\\hline%\cline{2-3}
\Sc{elat}	& guol-e-st		& gul-i-jst		\\\hline%\cline{2-3}
\Sc{com}	& gul-i-jn			& gul-i-j		\\\hline%\cline{2-3}
\Sc{abess}	& guol-e-dak		& guol-e-daga	\\\hline%\cline{2-3}
\Sc{ess}	&\MC{2}{c|}{guoll-e-n}\\\hline%\hline
\end{tabular}
\begin{tabular}{c}
\\
\end{tabular}
\begin{tabular}{ |r | c | c | }\hline
			&\MC{2}{c|}{\It{number}}\\
\It{case}	& \Sc{singular}	& \Sc{plural}	 \\\dline
\Sc{nom}	& vágg-e			& vágg-e		\\\hline%\cline{2-3}
\Sc{gen}	& vágg-e			& vägg-i-j		\\\hline%\cline{2-3}
\Sc{acc}	& vágg-e-v		& vägg-i-jd		\\\hline%\cline{2-3}
\Sc{ill}		& vágg-á-j			& vägg-i-jda		\\\hline%\cline{2-3}
\Sc{iness}	& vágg-e-n		& vägg-i-jn		\\\hline%\cline{2-3}
\Sc{elat}	& vágg-e-st		& vägg-i-jst		\\\hline%\cline{2-3}
\Sc{com}	& vägg-i-jn		& vägg-i-j		\\\hline%\cline{2-3}
\Sc{abess}	& vágg-e-dak		& vágg-e-daga	\\\hline%\cline{2-3}
\Sc{ess}	&\MC{2}{c|}{vágg-e-n}\\\hline%\hline
\end{tabular}
\end{table}



%\subsubsubsection{\jvh}\label{jSuffixHarmony} 
As mentioned above, nouns in Class Ie are subject to \jvh. This refers to regressive vowel harmony triggered by the presence of \It{j} in certain case/number suffixes. In this, certain V1 vowels and the V2 vowel are raised in accommodating the palatal position of the \It{j} in the suffix. The vowel in V2, which is \It{e} in Class Ie nouns, is raised to \It{i}, while the vowel in V1 is raised depending on its initial value: ä\ARROW i%\marginpar{need more evidence for ä\ARROW i !! (jägge-jiggijn, värrbme-virbmijd}
, uo/uä\ARROW u, %uä\ARROW u\marginpar{uä is an allophone of ua!}, 
a\ARROW ä,  á\ARROW ä and å\ARROW u. Other V1 vowels in Class Ie are not affected, but the V2 vowel is always raised. %(but the class marker is still \It{i} for all Class II nouns in the relevant paradigm slots.). 


The class marking suffixes and consonant gradation pattern for Class Ie are summarized in Table \vref{NclassIeSuffixes}. % \pageref{NclassIIsuffixes}.


\begin{table}\centering
\caption{The Class Ie consonant gradation pattern and inflectional noun class suffixes}\label{NclassIeSuffixes}
\begin{tabular}{ |r || c | c || c | c | }\hline
			&\MC{4}{c|}{\It{number}}\\
			&\MC{2}{c||}{\Sc{singular}}&\MC{2}{c|}{\Sc{plural}}	 \\%\dline
\It{case}	&\It{C-grad}&\It{class suffix}&\It{C-grad}&\It{class suffix}	 \\\dline
\Sc{nom}	&str		& -e			&wk		& -e		\\\hline%\cline{2-3}
\Sc{gen}	&wk		& -e			&wk		& -i-		\\\hline%\cline{2-3}
\Sc{acc}	&wk		& -e-			&wk		& -i-		\\\hline%\cline{2-3}
\Sc{ill}		&str		& -á-			&wk		& -i-		\\\hline%\cline{2-3}
\Sc{iness}	&wk		& -e-			&wk		& -i-		\\\hline%\cline{2-3}
\Sc{elat}	&wk		& -e-			&wk		& -i-		\\\hline%\cline{2-3}
\Sc{com}	&wk		& -i-			&wk		& -i-		\\\hline%\cline{2-3}
\Sc{abess}	&wk		& -e-			&wk		& -e-	\\\hline%\cline{2-3}
\MR{2}{*}{\Sc{ess}}	&\MC{2}{c|}{\It{C-grad}}	&\MC{2}{c|}{\It{class suffix}}	\\\cline{2-5}
	&\MC{2}{c|}{str}	&\MC{2}{c|}{-e-}\\\hline%\hline
%\begin{tabular}{ |r | c | c | }\hline
%			&\MC{2}{c|}{\It{number}\\
%\It{case}	& \Sc{singular}	& \Sc{plural}	 \\\dline
%\Sc{nom}	& -e			& -e		\\\hline%\cline{2-3}
%\Sc{gen}	& -e			& -i-		\\\hline%\cline{2-3}
%\Sc{acc}	& -e-			& -i-		\\\hline%\cline{2-3}
%\Sc{ill}		& -á-			& -i-		\\\hline%\cline{2-3}
%\Sc{iness}	& -e-			& -i-		\\\hline%\cline{2-3}
%\Sc{elat}	& -e-			& -i-		\\\hline%\cline{2-3}
%\Sc{com}	& -i-			& -i-		\\\hline%\cline{2-3}
%\Sc{abess}	& -e-			& -e-	\\\hline%\cline{2-3}
%\Sc{ess}	&\MC{2}{c|}{-e-}\\\hline%\hline
\end{tabular}
\end{table}

%Note that the plural suffixes \It{-j} \Sc{gen.pl}, \It{-jd} \Sc{acc.pl}, \It{-jda} \Sc{ill.pl}, \It{-jn} \Sc{iness.pl}, \It{-jst} \Sc{elat.pl} and \It{-j} \Sc{com.pl}, as well as the singular suffix \It{-jn(a)} \Sc{com.sg} all trigger this vowel harmony. It is therefore not possible to claim it is a \PL\ marker, since \Sc{com.sg} also triggers it; 
%However, the \Sc{ill.sg} suffix \It{-j} does \It{not} trigger j-suffix harmony, despite having a /j/ segment and even being phonemically identical with the suffixes for both \Sc{gen.pl} and \Sc{com.pl}, so it cannot be a purely phonological process because \Sc{ill.sg} does not trigger it. 
%It must be concluded that this phonological  process is morphologically conditioned. %, and not a true (post-lexical?) phonological process%\marginpar{is this really relevant?}
\FloatBarrier




\subsection{Class II}\label{NclassII}
Two features mark nouns in Class II: 
\begin{itemize}
\item{the class marking suffix for \Sc{nom.sg} is \It{-Vj}, while for the other case/number slots, the class marker is \It{-V}}
\item{the gradation pattern (when relevant) is inverted.}
\end{itemize}
%consistently marked by the \It{-å} class marking suffix in all case/number slots of the paradigm. 
The inflectional paradigm for the words \It{båtsoj} ‘reindeer’ and \It{ålmaj} ‘man’ are provided as examples in Table \vref{reindeerParadigm}.
\begin{table}\centering
\caption{The class-marker suffix and case/number paradigms for the Class II nouns  \It{båtsoj} ‘reindeer’ and \It{ålmaj} ‘man’}\label{reindeerParadigm}
\begin{tabular}{ |r | c | c | }\hline
			&\MC{2}{c|}{\It{number}}\\
\It{case}	& \Sc{singular}	& \Sc{plural}	 \\\dline
\Sc{nom}	& båts-oj			& buhts-u		\\\hline%\cline{2-3}
\Sc{gen}	& buhts-u			& buhts-u-j		\\\hline%\cline{2-3}
\Sc{acc}	& buhts-u-v		& buhts-u-jd	\\\hline%\cline{2-3}
\Sc{ill}		& buhts-u-j		& buhts-u-jda	\\\hline%\cline{2-3}
\Sc{iness}	& buhts-u-n		& buhts-u-jn	\\\hline%\cline{2-3}
\Sc{elat}	& buhts-u-st		& buhts-u-jst	\\\hline%\cline{2-3}
\Sc{com}	& buhts-u-jn		& buhts-u-j		\\\hline%\cline{2-3}
\Sc{abess}	& buhts-u-dak		& buhts-u-daga	\\\hline%\cline{2-3}
\Sc{ess}	&\MC{2}{c|}{båts-o-n}\\\hline%\hline
\end{tabular}%\caption{The case/number paradigm for the Class V noun \It{båtsoj} ‘reindeer’}\label{reindeerParadigm}
\begin{tabular}{c}
\\
\end{tabular}
\begin{tabular}{ |r | c | c | }\hline
			&\MC{2}{c|}{\It{number}}\\
\It{case}	& \Sc{singular}	& \Sc{plural}	 \\\dline
\Sc{nom}	& ålm-aj			& ålm-a		\\\hline%\cline{2-3}
\Sc{gen}	& ålm-a			& ålm-a-j		\\\hline%\cline{2-3}
\Sc{acc}	& ålm-a-v			& ålm-a-jd	\\\hline%\cline{2-3}
\Sc{ill}		& ålm-a-j			& ålm-a-jda	\\\hline%\cline{2-3}
\Sc{iness}	& ålm-a-n			& ålm-a-jn	\\\hline%\cline{2-3}
\Sc{elat}	& ålm-a-st			& ålm-a-jst	\\\hline%\cline{2-3}
\Sc{com}	& ålm-a-jn			& ålm-a-j		\\\hline%\cline{2-3}
\Sc{abess}	& ?				& ?	\\\hline%\cline{2-3}
\Sc{ess}	&\MC{2}{c|}{ålm-a-n}\\\hline%\hline
\end{tabular}
\end{table}

Note that the consonant gradation pattern in the \It{båtsoj} paradigm is for the most part inverted, i.e., the weak grade is found in \Sc{nom.sg} and \Sc{ess}, and the strong grade elsewhere. The lexical item \It{ålmaj} does not feature consonant gradation, and attempts to elicit the abessive forms resulted in three inconsistent forms.
The gradation pattern and class marking suffixes for Class II are summarized in Table \vref{NclassIIsuffixes}. % on page \pageref{NclassIIIsuffixes}.
\begin{table}\centering
\caption{The Class II consonant gradation pattern and inflectional noun class suffixes}\label{NclassIIsuffixes}
\begin{tabular}{ |r || c | c || c | c | }\hline
			&\MC{4}{c|}{\It{number}}\\
			&\MC{2}{c||}{\Sc{singular}}	&\MC{2}{c|}{\Sc{plural}}	 \\%\dline
\It{case}	&\It{C-grad}&\It{class suffix}	&\It{C-grad}&\It{class suffix	} \\\dline
\Sc{nom}	&wk		& -Vj			&str		& -V		\\\hline%\cline{2-3}
\Sc{gen}	&str		& -V			&str		& -V-		\\\hline%\cline{2-3}
\Sc{acc}	&str		& -V-			&str		& -V-		\\\hline%\cline{2-3}
\Sc{ill}		&str		& -V-			&str		& -V-		\\\hline%\cline{2-3}
\Sc{iness}	&str		& -V-			&str		& -V-		\\\hline%\cline{2-3}
\Sc{elat}	&str		& -V-			&str		& -V-		\\\hline%\cline{2-3}
\Sc{com}	&str		& -V-			&str		& -V-		\\\hline%\cline{2-3}
\Sc{abess}	&str		& -V-			&str		& -V-	\\\hline%\cline{2-3}
\MR{2}{*}{\Sc{ess}}	&\MC{2}{c|}{\It{C-grad}}	&\MC{2}{c|}{\It{class suffix}}	\\\cline{2-5}
	&\MC{2}{c|}{wk}	&\MC{2}{c|}{-V-}\\\hline%\hline
%\begin{tabular}{ |r | c | c | }\hline
%			&\MC{2}{c|}{\It{number}\\
%\It{case}	& \Sc{singular}	& \Sc{plural}	 \\\dline
%\Sc{nom}	& -Vj			& -V		\\\hline%\cline{2-3}
%\Sc{gen}	& -V			& -V-		\\\hline%\cline{2-3}
%\Sc{acc}	& -V-			& -V-		\\\hline%\cline{2-3}
%\Sc{ill}		& -V-			& -V-		\\\hline%\cline{2-3}
%\Sc{iness}	& -V-			& -V-		\\\hline%\cline{2-3}
%\Sc{elat}	& -V-			& -V-		\\\hline%\cline{2-3}
%\Sc{com}	& -V-			& -V-		\\\hline%\cline{2-3}
%\Sc{abess}	& -V-			& -V-	\\\hline%\cline{2-3}
%\Sc{ess}	&\MC{2}{c|}{-V-}\\\hline%\hline
\end{tabular}
\end{table}

While there do not appear to be many words in Class II, the data in the corpus is ultimately inconclusive. %More study is required to determine more about this class, %UM’s suggestion:
There are some irregularities which cannot be explained, as for instance 
why the vowel in V2 position in \It{båtsoj} is only \It{o} in \Sc{nom.sg} and \Sc{ess}, but otherwise \It{u}, while the V2 vowel in \It{ålmaj} is consistently \It{a}. Furthermore, it is unclear why consonant gradation in the \Sc{ill.sg} form in the \It{båtsoj} paradigm does not align with \Sc{nom.sg} and \Sc{ess}. %The \ABESS\ forms for \It{ålmaj} are also inconclusive in the data. %Future study must be done to better understand this class.


\FloatBarrier


\subsection{Class III}\label{NclassIII}
Three features mark nouns in Class III: 
\begin{itemize}
\item{the stem is consonant-final}
\item{the \Sc{nom.sg} form lacks a class suffix}
\item{the gradation pattern (when relevant) is inverted}
\end{itemize}
This class consists of two subclasses (IIIa and IIIb). 


\subsubsection{Class IIIa}\label{NclassIIIa}
Class IIIa is the more common Class III subclass. It exhibits a \Sc{nom.sg} form which lacks a class marker and ends in a closed syllable; in this case, the stem-final consonant is thus the word-final consonant. 
The paradigms for the nouns \It{sabek}\footnote{In adhering to \PS\ orthographic conventions, word-final /k/ is spelled with <k>, while intervocalic /k/ is spelled <g>.} 
‘ski’ and \It{vanás} ‘boat’ are provided in Table \vref{skiBoatParadigm} %on page \pageref{skiParadigm} 
as examples for this subclass. %, in which the class suffix for \Sc{nom.sg} forms a closed syllable. 
The word \It{vanás} ‘boat’ is similar to \It{sabek} ‘ski’, but is subject to consonant gradation. %The \It{vanás}-paradigm is shown in Table \vref{boatParadigm}. 
Finally, denominal nouns derived by the diminutive suffix \It{-tj} are all in Class IIIa. Table \vref{fishDIMparadigm} provides a nearly complete paradigm for \It{guolátj} ‘little fish’. 

\begin{table}\centering
\caption{The inflectional paradigms for the Class IIIa nouns \It{sabek} ‘ski’ and \It{vanás} ‘boat’}\label{skiBoatParadigm}
\begin{tabular}{ |r | c | c | }\hline
			&\MC{2}{c|}{\It{number}}\\
\It{case}	& \Sc{singular}	& \Sc{plural}	 \\\dline
\Sc{nom}	& sabek			& sabeg-a		\\\hline%\cline{2-3}%JW: other suggestion: sabek-ø
\Sc{gen}	& sabeg-a			& sabeg-i-j		\\\hline%\cline{2-3}
\Sc{acc}	& sabeg-a-v		& sabeg-i-jd	\\\hline%\cline{2-3}
\Sc{ill}		& sabeg-i-j		& sabeg-i-jda	\\\hline%\cline{2-3}
\Sc{iness}	& sabeg-i-n		& sabeg-i-jn	\\\hline%\cline{2-3}
\Sc{elat}	& sabeg-i-st		& sabeg-i-jst	\\\hline%\cline{2-3}
\Sc{com}	& sabeg-i-jn		& sabeg-i-j		\\\hline%\cline{2-3}
\Sc{abess}	& ?				& ?	\\\hline%\cline{2-3}
\Sc{ess}	&\MC{2}{c|}{?}\\\hline%\hline
\end{tabular}
%\end{table}
\begin{tabular}{c}
\\
\end{tabular}
%\begin{table}\centering
%\caption{The class-marker suffix and case/number paradigm for the Class IIIa noun \It{vanás} ‘boat’}\label{boatParadigm}
\begin{tabular}{ |r | c | c | }\hline
			&\MC{2}{c|}{\It{number}}\\
\It{case}	& \Sc{singular}	& \Sc{plural}	 \\\dline
\Sc{nom}	& vanás			& vadnás-a		\\\hline%\cline{2-3}%JW: other suggestion: sabek-ø
\Sc{gen}	& vadnás-a		& vadnás-i-j		\\\hline%\cline{2-3}
\Sc{acc}	& vadnás-a-v		& vadnás-i-jd	\\\hline%\cline{2-3}
\Sc{ill}		& vadnás-i-j		& vadnás-i-jda	\\\hline%\cline{2-3}
\Sc{iness}	& vadnás-i-n		& vadnás-i-jn	\\\hline%\cline{2-3}
\Sc{elat}	& vadnás-i-st		& vadnás-i-jst	\\\hline%\cline{2-3}
\Sc{com}	& vadnás-i-jn		& vadnás-i-j		\\\hline%\cline{2-3}
\Sc{abess}	& ?				& ?	\\\hline%\cline{2-3}
\Sc{ess}	&\MC{2}{c|}{?}\\\hline%\hline
\end{tabular}
\end{table}

\begin{table}\centering
\caption{The inflectional paradigm for the Class IIIa denominal noun \It{guolátj} ‘little fish’}\label{fishDIMparadigm}
\begin{tabular}{ |r | c | c | }\hline
			&\MC{2}{c|}{\It{number}}\\
\It{case}	& \Sc{singular}	& \Sc{plural}	 \\\dline
\Sc{nom}	& guolátj			& guolátj-a		\\\hline%\cline{2-3}%JW: other suggestion: sabek-ø
\Sc{gen}	& guolátj-a		& guolátj-i-j		\\\hline%\cline{2-3}
\Sc{acc}	& guolátj-a-v		& guolátj-i-jd	\\\hline%\cline{2-3}
\Sc{ill}		& guolátj-i-j		& guolátj-i-jda	\\\hline%\cline{2-3}
\Sc{iness}	& guolátj-i-n		& guolátj-i-jn	\\\hline%\cline{2-3}
\Sc{elat}	& guolátj-i-st		& guolátj-i-jst	\\\hline%\cline{2-3}
\Sc{com}	& guolátj-i-jn		& guolátj-i-j		\\\hline%\cline{2-3}
\Sc{abess}	& ?				& ?	\\\hline%\cline{2-3}
\Sc{ess}	&\MC{2}{c|}{?}\\\hline%\hline
\end{tabular}
\end{table}


\subsubsection{Class IIIb}\label{NclassIIIb}
The less common subclass of Class III nouns exhibits a \Sc{nom.sg} form which also lacks a class marker, but ends in an open syllable; thus, the stem-final consonant, which is present in all other slots in the paradigm, is lacking. 
The word \It{bena} ‘dog’ is provided in Table \vref{dogParadigm} as a example for this second subclass. %
\begin{table}\centering
\caption{The inflectional paradigm for the Class IIIb noun \It{bena} ‘dog’}\label{dogParadigm}
\begin{tabular}{ |r | c | c | }\hline
			&\MC{2}{c|}{\It{number}}\\
\It{case}	& \Sc{singular}	& \Sc{plural}	 \\\dline
\Sc{nom}	& bena			& bednag-a		\\\hline%\cline{2-3}
\Sc{gen}	& bednag-a		& bednag-i-j		\\\hline%\cline{2-3}
\Sc{acc}	& bednag-a-v		& bednag-i-jd	\\\hline%\cline{2-3}
\Sc{ill}		& bednag-i-j		& bednag-i-jda	\\\hline%\cline{2-3}
\Sc{iness}	& bednag-i-n		& bednag-i-jn	\\\hline%\cline{2-3}
\Sc{elat}	& bednag-i-st		& bednag-i-jst	\\\hline%\cline{2-3}
\Sc{com}	& bednag-i-jn		& bednag-i-j		\\\hline%\cline{2-3}
\Sc{abess}	& ?				& ?	\\\hline%\cline{2-3}
\Sc{ess}	&\MC{2}{c|}{?}\\\hline%\hline
\end{tabular}
\end{table}

%The class marking suffix consists of a V2 vowel, a single consonant, and a predicable final vowel, and is represented by the structure \It{-VCv}, where \It{V} and \It{C} stand for a vowel and consonant specific to the particular lexical item, and \It{v} is a vowel occurring regularly throughout Class VI. 


\subsubsection{Class III summary}\label{NclassIIIsummary}
The gradation pattern and class marking suffixes for Class III are summarized in Table \vref{NclassIIIsuffixes}. 
%Note that the consonant gradation pattern for Class III nouns is inverted, i.e., the weak grade is found in \Sc{nom.sg}, and the strong grade elsewhere. 
\begin{table}\centering
\caption{The Class III consonant gradation pattern and inflectional noun class suffixes}\label{NclassIIIsuffixes}
\begin{tabular}{ |r || c | c || c | c | }\hline
			&\MC{4}{c|}{\It{number}}\\
			&\MC{2}{c||}{\Sc{singular}}	&\MC{2}{c|}{\Sc{plural}}	 \\%\dline
\It{case}	&\It{C-grad}&\It{class suffix}	&\It{C-grad}&\It{class suffix	} \\\dline
\Sc{nom}	&wk		& -			&str		& -a		\\\hline%\cline{2-3}
\Sc{gen}	&str		& -a			&str		& -i-		\\\hline%\cline{2-3}
\Sc{acc}	&str		& -a-			&str		& -i-		\\\hline%\cline{2-3}
\Sc{ill}		&str		& -i-			&str		& -i-		\\\hline%\cline{2-3}
\Sc{iness}	&str		& -i-			&str		& -i-		\\\hline%\cline{2-3}
\Sc{elat}	&str		& -i-			&str		& -i-		\\\hline%\cline{2-3}
\Sc{com}	&str		& -i-			&str		& -i-		\\\hline%\cline{2-3}
\Sc{abess}	&str		& ?			&str		& ?	\\\hline%\cline{2-3}
\MR{2}{*}{\Sc{ess}}	&\MC{2}{c|}{\It{C-grad}}	&\MC{2}{c|}{\It{class suffix}}	\\\cline{2-5}
	&\MC{2}{c|}{?}	&\MC{2}{c|}{?}\\\hline%\hline
%\begin{tabular}{ |r | c | c | }\hline
%			&\MC{2}{c|}{\It{number}\\
%\It{case}	& \Sc{singular}	& \Sc{plural}	 \\\dline
%\Sc{nom}	& -Vj			& -V		\\\hline%\cline{2-3}
%\Sc{gen}	& -V			& -V-		\\\hline%\cline{2-3}
%\Sc{acc}	& -V-			& -V-		\\\hline%\cline{2-3}
%\Sc{ill}		& -V-			& -V-		\\\hline%\cline{2-3}
%\Sc{iness}	& -V-			& -V-		\\\hline%\cline{2-3}
%\Sc{elat}	& -V-			& -V-		\\\hline%\cline{2-3}
%\Sc{com}	& -V-			& -V-		\\\hline%\cline{2-3}
%\Sc{abess}	& -V-			& -V-	\\\hline%\cline{2-3}
%\Sc{ess}	&\MC{2}{c|}{-V-}\\\hline%\hline
\end{tabular}
\end{table}
As with Class II words, the corpus only provides limited data on Class III words, and attempts to elicit abessive and essive forms led to inconsistent results, partly due to uncertain native speaker intuition for these rare forms. However, elicited abessive forms were consistently in the strong grade, while elicited essive forms were sometimes in the strong grade, sometimes in the weak grade, without any seemingly consistent patterns.\korr{053}\footnote{One language consultant fairly consistently produced \Sc{abess.sg} forms without the stem final consonant for some nouns in this class, but still felt uncertain about this. Specifically, this individual produced the forms \It{vadnádaga} ‘boat-\Sc{abess.sg}’, \It{bednadaga} ‘dog-\Sc{abess.sg}’ and \It{sabedaga} ‘ski-\Sc{abess.sg}’.} 
%but still felt uncertain about this. 
%UM: delete: Ultimately, a more thorough description of this class must be left to future research.


\subsection{Summary of noun classes}\label{summaryNounClasses}
Table \vref{nounClassExamples} is provided to facilitate cross-class comparison of paradigms for examples from the various noun classes. While the whole paradigm for each word is not listed due to a lack of space, the forms for \Sc{nom.sg}, \Sc{nom.pl}, \Sc{acc.sg}, \Sc{gen.pl}, \Sc{ill.sg} and \Sc{elat.sg} are more than sufficient to convey the relevant morphological differences between the classes.
%\begin{table}\centering
\begin{sidewaystable}\centering
%\hspace{-15mm}
\caption{Comparison of noun class examples}\label{nounClassExamples}
\begin{tabular}{| cc | c | c | c | c | c | c || l |}\hline
\MC{2}{|c|}{\It{class}}&\Sc{nom.sg}&\Sc{nom.pl}	&\Sc{acc.sg}	&\Sc{gen.pl}	&\Sc{ill.sg}	&\Sc{elat.sg}	&\It{gloss}	\\\dline
I	&a		&luakkt-a		&luokt-a		&luokt-a-v		&luokt-a-j		&luakkt-a-j	&luokt-a-st	& ‘bay’		\\\cline{2-9}%\hline
	&b		&mánn-á		&mán-á		&mán-á-v		&mán-á-j		&mánn-á-j		&mán-á-st		& ‘child’	\\\cline{2-9}%\hline
	&c		&bäbbm-o		&biebm-o		&biebm-o-v	&biebm-o-j	&bäbbm-o-j	&biebm-o-st	& ‘food’	\\\cline{2-9}%\hline
	&d		&skåvvl-å		&skåvl-å		&skåvl-å-v		&skåvl-å-j		&skåvvl-å-j	&skåvl-å-st	& ‘school’	\\\cline{2-9}

	&e		&guoll-e		&guol-e		&guol-e-v		&gul-i-j		&guoll-á-j		&guol-e-st		& ‘fish’		\\\cline{3-9}%\hline
	&		&vágg-e		&vágg-e		&vágg-e-v		&vägg-i-j		&vágg-á-j		&vágg-e-st	& ‘valley’	\\\cline{3-9}%\hline
%	&		&áhttj-e		&áhtj-e		&áhtj-e-v		&ähtj-i-j?		&áhttj-á-j?		&áhtj-e-st		& ‘father’	\\\cline{3-9}%\hline%JW: NOT enough data in corpus
	&		&sábm-e		&sám-e		&sám-e-v		&säm-i-j		&sábm-á-j		&sám-e-st		& ‘Saami’	\\\dline

II	&		&båts-oj		&buhts-u		&buhts-u-v	&buhts-u-j		&buhts-u-j		&buhts-u-st	& ‘reindeer’	\\\cline{3-9}%\hline
	&		&ålm-aj		&ålm-a		&ålm-a-v		&ålm-a-j		&ålm-a-j		&ålm-a-st		& ‘man’	\\\dline

III	&a		&sabek		&sabeg-a		&sabeg-a-v	&sabeg-i-j		&sabeg-i-j		&sabeg-i-st	& ‘ski’		\\\cline{3-9}%\hline
	&		&vanás		&vadnás-a	&vadnás-a-v	&vadnás-i-j	&vadnás-i-j	&vadnás-i-st	& ‘boat’	\\\cline{2-9}%\hline
	&b		&bena		&bednag-a	&bednag-a-v	&bednag-i-j	&bednag-i-j	&bednag-i-st	& ‘dog’		\\\cline{3-9}%\hline
	&		&gáma		&gábmag-a	&gábmag-a-v	&gábmag-i-j	&gábmag-i-j	&gábmag-i-st	& ‘shoe’	\\\hline
\end{tabular}
\end{sidewaystable}
%\end{table}
\FloatBarrier



\section{Possessive suffixes}\label{possSuffixes}\index{possessive suffixes}
A special set of possessive suffixes exists in \PS\ which indicate, in addition to case and number for the host noun, the person and number of the possessor of the referent of the host noun. %UM commented on “by the referent of an external NP.” aber dafür habe ich keine Beispiele. war wohl ungünstig formuliert, also gelöscht.
While the possessive suffixes go back to Proto-Saami \citep[73]{Sammallahti1998}, they seem to have nearly fallen out of use in contemporary \PS, and are only attested in three recordings from the corpus. These examples from the corpus are presented first, and a discussion follows. 

While there are technically nine tokens of possessive pronouns in the corpus, these nine tokens can be grouped into two identical sets, so that effectively only two examples are available. Specifically, there are three tokens of \It{áhttjes} ‘my father’ in nominative case by one speaker in two different recordings, and three tokens of the parallel construction \It{mammaset ja pappaset} ‘your mother and your father’ by another speaker in one recording. An example from the first speaker is provided in \REF{possSuffix1}, and an example\footnote{The example in \REF{possSuffix5} is essentially identical to the other utterance with four tokens of these same noun stems with possessive suffixes (pit100703a.034).} 
from the second speaker in \REF{possSuffix5}.
\ea\label{possSuffix1}%JW: where’s the finite verb here?!?
\glll	áhttjes dá lä gähtjamin jus gävdnij aktak, nag getjokmiesse\\
	áhttje-s dá lä gähtja-min jus gävdni-j aktak nagin getjok-miesse\\
	father-\Sc{1sg.poss\BS nom.sg} then be\BS\Sc{3sg.prs} look-\Sc{prog} if exist-\Sc{3sg.pst} any some unmarked-calf\BS\Sc{nom.sg}\\%pappa-1sg.poss nu titta-3sg.prog om finns nån någan omärk-kalv
\Transl{my father is looking if there is an unmarked calf}{}	\Corpus{080909}{004}
\z
%\ea\label{possSuffix2}%JW: listen to this again!
%\glll	jus áhttjes gilgij tjuorvot dan nala\\
%	jus áhttje-s gilgi-j tjuorvo-t da-n nala\\
%	if father-\Sc{1sg.poss\BS nom.sg} shall-\Sc{3sg.pst} call-\Sc{inf} \Sc{dem-gen.sg} to\\
%\Transl{if my father called it’	\Corpus{100405b.087-088}
%\ea\label{possSuffix3}
%\glll	áhtjes, da gal juksa\\
%	áhtjes, da gal juksa\\
%	\\
%\Transl{father, he would catch it’	\Corpus{100405b.095}
%\ea\label{possSuffix4}
%\glll	nå hälset del mammaset ja pappaset, hälset del edna..., ja mammaset ja pappaset, edna\\
%	nå hälset del mammaset ja pappaset, hälset del edna..., ja mammaset ja pappaset, edna\\
%	\\
%\Transl{’	\Corpus{100703a.034}
\ea\label{possSuffix5}
\glll	nå dä hulij, nå hälset del mammaset ja pahppaset\\
	nå dä huli-j nå hälse-t del mamma-set ja pahppa-set\\
	well then say-\Sc{3sg.pst} well greet-\Sc{pl.imp} then mother-\Sc{2pl.poss\BS ill.sg} and father-\Sc{2pl.poss\BS ill.sg} \\\nopagebreak
\Transl{well then she said “well, say hello to your mother and your father”}{}	\Corpus{100703a}{038}
\z

Just as with the other case/number suffixes, the possessive case/number suffixes follow the inflectional class marker, as illustrated in Figure \vref{nounStructurePossSuff}. %on page \pageref{nounStructurePossSuff}.
\begin{figure}\centering
\fbox{∑ \PLUS\ class-marker \PLUS\ possessive-case/number}
\caption{The morphological structure of \PS\ nouns with possessive suffixes}\label{nounStructurePossSuff}
\end{figure}

The three examples from the corpus can thus be parsed morphologically as in \REF{possSuffixParse1} through \REF{possSuffixParse3}:
\ea\label{possSuffixParse1}
\gll	áhttj-e-s\\
	father-Ie-\Sc{1sg.poss\BS nom.sg}\\\nopagebreak
\Transl{my father}{}
\z
\ea\label{possSuffixParse2}
\gll	mamm-a-set\\
	mother-Ia-\Sc{2pl.poss\BS ill.sg}\\\nopagebreak
\Transl{to your (\Sc{pl}) mother}{}
\z
\ea\label{possSuffixParse3}
\gll	pahpp-a-set\\
	father-Ia-\Sc{2pl.poss\BS ill.sg}\\\nopagebreak
\Transl{to your (\Sc{pl}) father}{}
\z

While these three examples do not provide enough evidence for case and number marking in addition to possession, the thorough paradigm for \It{åbba} ‘sister’ and a very partial paradigm for \It{áhttje} ‘father’ in \citet[158-159]{Lehtiranta1992}\footnote{Note that \citet{Lehtiranta1992} uses a different orthography: \It{ååp'paa} for ‘sister’ and \It{aah'tjie} for ‘father’.} indicate that the possessive suffixes are best described as portmanteau suffixes which indicate the number and case of the host noun as well as the person and number of the external possessor.\footnote{\citet[110]{Lagercrantz1926} only lists six possessive suffixes (\Sc{1/2/3sg.poss}, \Sc{3du.poss} and \Sc{3pl.poss}), but he does not provide any further information concerning possessive suffixes.}
As substitutes for the external possessor NP, they fill a pronomial function, as well. %as indicating purely inflectional. 

It should be pointed out that the possessive suffixes above do not correspond to the equivalent examples in the Lehtiranta paradigms: Lehtiranta has \It{aah'tjaam} for ‘father-\Sc{1sg.poss\BS nom.sg}’, while \It{aah'tjies} is listed as ‘father-\Sc{3sg.poss\BS nom.sg}’, a form which is much closer to the form in \REF{possSuffixParse1}, but means ‘his/her father’.\footnote{\citet[110]{Lagercrantz1926} also indicates that the \Sc{3sg.poss} suffix is \It{-s}.} 
Furthermore, Lehtiranta indicates that \It{ååp'paasetieh} is ‘sister-\Sc{2pl.poss\BS ill.sg}, which has an additional \It{-ieh} word-finally not found in \REF{possSuffixParse2} or \REF{possSuffixParse3}. 

In all other cases in the corpus, NP-internal possession is expressed using a noun or pronoun in the genitive case, as in \REF{noPossSuffix1}, which is from the same speaker and recording as in \REF{possSuffix5} above.
\ea\label{noPossSuffix1}
\glll	ja dä lij mijan sessa Kärin\\
	ja dä li-j mijan sessa Kärin\\
	and then be-\Sc{3sg.pst} \Sc{1pl.gen} paternal\_aunt\BS\Sc{nom.sg} Karin \\\nopagebreak
\Transl{and then it was our paternal aunt Karin}{}	\Corpus{100703a}{014}
\z

It is likely the case that alienability plays (or played) a role in which nouns can be marked with possessive suffixes. It is also possible that certain nouns with possessive suffixes have been lexicalized in current usage. %, such as in \REF{possSuffixParse1}. 
While the lack of possessive suffixes in the corpus seems to indicate that they are no longer used regularly, the fact that the two obvious loan words in \REF{possSuffixParse2} and \REF{possSuffixParse3}\footnote{From Sw. \It{mamma} ‘mother’ and \It{pappa} ‘father’.} 
\korr{022}have possessive suffixes, indicates that they may still be productive somehow, or at least retrievable via analogy. At any rate, the corpus does not provide nearly enough data on the possessive suffixes and any conclusions on their current state this topic must be left to future research.



%%%%%%% THIS IS NOT USED FOR THE ENTIRE COMPILATION, but only for individual chapters!!!!

\clearpage
\addcontentsline{toc}{chapter}{Bibliography}\label{Bibliography}
\bibliography{PiteGrammarBibSDL}%for bibtex
%\printbibliography%[title=Works Cited]%%for biber!






%%%NAME INDEX doesn’t work!?!? why???
\cleardoublepage\phantomsection%this allows hyperlink in ToC to work
\addcontentsline{toc}{chapter}{Name index}
\ohead{Name index}
\printindex[aut]

\cleardoublepage\phantomsection%this allows hyperlink in ToC to work
\addcontentsline{toc}{chapter}{Language index}
\ohead{Language index}
\printindex[lan]

\cleardoublepage\phantomsection%this allows hyperlink in ToC to work
\addcontentsline{toc}{chapter}{Subject index}
\ohead{Subject index}
\printindex


\end{document}