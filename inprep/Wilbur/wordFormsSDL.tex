%\documentclass[ number=5
			   ,series=sidl
			   ,isbn=xxx-x-xxxxxx-xx-x
			   ,url=http://langsci-press.org/catalog/book/17
			   ,output=long   % long|short|inprep              
			   %,blackandwhite
			   %,smallfont
			   ,draftmode   
			  ]{LSP/langsci}                          

\usepackage{LSP/lsp-styles/lsp-gb4e}		% verhindert Komma bei mehrfachen Fußnoten?
                                                      
\usepackage{layout}
\usepackage{lipsum}

%%%% ABOVE FOR LangSciPress %%%%
%%%% ABOVE FOR LangSciPress %%%%
%%%% ABOVE FOR LangSciPress %%%%
\usepackage{libertine}%work-around solution for rendering problematic characters ʦ, ͡  (mostly in \textbf{})

\usepackage{longtable}%Double-lines (\hline\hline) aren’t typeset properly in ‘longtable’-environment across several pages! conflict with other package (maybe xcolor with option ‘tables’?)

\usepackage{multirow}

\usepackage{array} %allows, among other things, centering column content in a table while also specifying width, creates new column style "x" for center-alignment, "y" for right-alignment
\newcolumntype{x}[1]{>{\centering\hspace{0pt}}p{#1}}%
\newcolumntype{y}[1]{>{\raggedleft\hspace{0pt}}p{#1}}%

\usepackage[]{placeins}%using \FloatBarrier command, all floats still floating at that point will be typeset, and cannot cross that boundary. the option here \usepackage[section]{placeins} automatically adds \FloatBarrier to every \section command (only works for \section commands, nothing lower than that!)
%\usepackage{afterpage}%by using the command \afterpage{\clearpage}, all floats will appear, but no new page will be started, thus avoiding bad page breaks around floats

\usepackage{vowel} %for vowel space chart


%%%IS THIS NECESSARY??
%%%%following allows you to refer to footnotes (from http://anthony.liekens.net/index.php/LaTeX/MultipleFootnoteReferences)
%\newcommand{\footnoteremember}[2]{
%  \footnote{#2}
%  \newcounter{#1}
%  \setcounter{#1}{\value{footnote}}
%} \newcommand{\footnoterecall}[1]{
%  \footnotemark[\value{#1}]} 
%%%%previous allows you to refer to footnotes: use \footnoteremember{referenceText} in footnote, then \footnoterecall{referenceText} to refer.

\usepackage{tikz}%
\usetikzlibrary{plothandlers,matrix,decorations.text,shapes.arrows,shadows,chains,positioning,scopes}

\usepackage{synttree} %zeichnet linguistische Bäume
\branchheight{36pt}%sets height between rows in synttree

\usepackage{lscape}%used for landscape pages in index (list of recordings)

\usepackage{polyglossia}
\setmainlanguage{english}


%%%TAKE OUT FOR FINAL VERSION:
%%%TAKE OUT FOR FINAL VERSION:
%%%TAKE OUT FOR FINAL VERSION:

%%%%following readjusts margin text!
%\setlength{\marginparwidth}{20mm}
%\let\oldmarginpar\marginpar
%\renewcommand\marginpar[1]{\-\oldmarginpar[\raggedleft\footnotesize\vspace{-7pt}\color{red}\It{→ #1}]%
%{\raggedright\footnotesize\vspace{-7pt}\color{red}\It{→ #1}}}
%%%%previous readjusts margin text!

%%%The following lines set depth of ToC (LSP default is only 3 levels)!
%%%\renewcommand{\contentsname}{Table of Contents} % überschrift des inhaltsverzeichnisses
%\setcounter{secnumdepth}{5}%sets how deep section/subsection/subsubsections are numbered
%\setcounter{tocdepth}{5}%sets the depth of the ToC %but this doesn't seem to work!!!
%% new commands for LSP book (Grammar of Pite Saami, by J. Wilbur)

\newcommand{\PS}{Pite Saami}
\newcommand{\PSDP}{Pite Saami Documentation Project}
\newcommand{\WLP}{Wordlist Project}

\newcommand{\HANG}{\everypar{\hangindent15pt \hangafter1}}%also useful for table cells
\newcommand{\FB}{\FloatBarrier}%shortcut for this command to print all floats w/o pagebreak

\newcommand{\REF}[1]{(\ref{#1})}%adds parenthesis around the reference number, particularly useful for examples.%\Ref had clash with LSP!
\newcommand{\dline}{\hline\hline}%makes a double line in a table
\newcommand{\superS}[1]{\textsuperscript{#1}}%adds superscript element
\newcommand{\sub}[1]{$_{#1}$}%adds subscript element
\newcommand{\Sc}[1]{\textsc{#1}}%shortcut for small capitals (not to be confused with \sc, which changes the font from that point on)
\newcommand{\It}[1]{\textit{#1}}%shortcut for italics (not to be confused with \it, which changes the font from that point on)
\newcommand{\Bf}[1]{\textbf{#1}}%shortcut for bold (not to be confused with \bf, which changes the font from that point on)
\newcommand{\BfIt}[1]{\textbf{\textit{#1}}}
\newcommand{\BfSc}[1]{\textbf{\textsc{#1}}}
\newcommand{\Tn}[1]{\textnormal{#1}}%shortcut for normal text (undo italics, bolt, etc.)
\newcommand{\MC}{\multicolumn}%shortcut for multicolumn command in tabular environment - only replaces command, not variables!
\newcommand{\MR}{\multirow}%shortcut for multicolumn command in tabular environment - only replaces command, not variables!
\newcommand{\TILDE}{∼}%U+223C %OLD:~}%shortcut for tilde%command ‘\Tilde’ clashes with LSP!%
\newcommand{\BS}{\textbackslash}%backslash
\newcommand{\Red}[1]{{\color{red}{#1}}}%for red text
\newcommand{\Blue}[1]{{\color{blue}{#1}}}%for blue text
\newcommand{\PLUS}{+}%nicer looking plus symbol
\newcommand{\MINUS}{-}%nicer looking plus symbol
%    Was die Pfeile betrifft, kannst Du mal \Rightarrow \mapsto \textrightarrow probieren und dann \mathbf \boldsymbol oder \pbm dazutun.
\newcommand{\ARROW}{\textrightarrow}%→%dieser dicke Pfeil ➜ wird nicht von der LSP-Font unterstützt: %\newcommand{\ARROW}{{\fontspec{DejaVu Sans}➜}}
\newcommand{\DARROW}{\textleftrightarrow}%↔︎%DoubleARROW
\newcommand{\BULLET}{•}%
%%✓ does not exist in the default LSP font!
\newcommand{\CH}{\checkmark}%%\newcommand{\CH}{\fontspec{Arial Unicode MS}✓}%CH as in CHeck
%%following used to separate alternation forms for consonant gradation and umlaut patterns:
\newcommand{\Div}{‑}%↔︎⬌⟷⬄⟺⇔%non-breaking hyphen: ‑  
\newcommand{\QUES}{\textsuperscript{?}}%marks questionable/uncertain forms

\newcommand{\jvh}{\mbox{\It{j}-suffix} vowel harmony}%
%\newcommand{\Ptcl}{\Sc{ptcl} }%just shortcut for glossing ‘particle’
%\newcommand{\ATTR}{{\Sc{attributive}}}%shortcut for ATTRIBUTIVE in small caps
%\newcommand{\PRED}{{\Sc{predicative}}}%shortcut for PREDICATIVE in small caps
%\newcommand{\COMP}{{\Sc{comparative}}}%shortcut for COMPARATIVE in small caps
%\newcommand{\SUPERL}{{\Sc{superlative}}}%shortcut for SUPERLATIVE in small caps
\newcommand{\SG}{{\Sc{singular}}}%shortcut for SINGULAR in small caps
\newcommand{\DU}{{\Sc{dual}}}%shortcut for DUAL in small caps
\newcommand{\PL}{{\Sc{plural}}}%shortcut for PLURAL in small caps
%\newcommand{\NOM}{{\Sc{nominative}}}%shortcut for NOMINATIVE in small caps
%\newcommand{\ACC}{{\Sc{accusative}}}%shortcut for ACCUSATIVE in small caps
%\newcommand{\GEN}{{\Sc{genitive}}}%shortcut for GENITIVE in small caps
%\newcommand{\ILL}{{\Sc{illative}}}%shortcut for ILLATIVE in small caps
%\newcommand{\INESS}{{\Sc{inessive}}}%shortcut for INESSIVE in small caps
\newcommand{\ELAT}{{\Sc{elative}}}%shortcut for ELATIVE in small caps
%\newcommand{\COM}{{\Sc{comitative}}}%shortcut for COMITATIVE in small caps
%\newcommand{\ABESS}{{\Sc{abessive}}}%shortcut for ABESSIVE in small caps
%\newcommand{\ESS}{{\Sc{essive}}}%shortcut for ESSIVE in small caps
%\newcommand{\DIM}{{\Sc{diminutive}}}%shortcut for DIMINUTIVE in small caps
%\newcommand{\ORD}{{\Sc{ordinal}}}%shortcut for ORDINAL in small caps
%\newcommand{\CARD}{{\Sc{cardinal}}}%shortcut for CARDINAL in small caps
%\newcommand{\PROX}{{\Sc{proximal}}}%shortcut for PROXIMAL in small caps
%\newcommand{\DIST}{{\Sc{distal}}}%shortcut for DISTAL in small caps
%\newcommand{\RMT}{{\Sc{remote}}}%shortcut for REMOTE in small caps
%\newcommand{\REFL}{{\Sc{reflexive}}}%shortcut for REFLEXIVE in small caps
%\newcommand{\PRS}{{\Sc{present}}}%shortcut for PRESENT in small caps
%\newcommand{\PST}{{\Sc{past}}}%shortcut for PAST in small caps
%\newcommand{\IMP}{{\Sc{imperative}}}%shortcut for IMPERATIVE in small caps
%\newcommand{\POT}{{\Sc{potential}}}%shortcut for POTENTIAL in small caps
\newcommand{\PROG}{{\Sc{progressive}}}%shortcut for PROGRESSIVE in small caps
\newcommand{\PRF}{{\Sc{perfect}}}%shortcut for PERFECT in small caps
\newcommand{\INF}{{\Sc{infinitive}}}%shortcut for INFINITIVE in small caps
%\newcommand{\NEG}{{\Sc{negative}}}%shortcut for NEGATIVE in small caps
\newcommand{\CONNEG}{{\Sc{connegative}}}%shortcut for CONNEGATIVE in small caps
\newcommand{\ATTRs}{{\Sc{attr}}}%shortcut for ATTR in small caps
\newcommand{\PREDs}{{\Sc{pred}}}%shortcut for PRED in small caps
%\newcommand{\COMPs}{{\Sc{comp}}}%shortcut for COMP in small caps
%\newcommand{\SUPERLs}{{\Sc{superl}}}%shortcut for SUPERL in small caps
\newcommand{\SGs}{{\Sc{sg}}}%shortcut for SG in small caps
\newcommand{\DUs}{{\Sc{du}}}%shortcut for DU in small caps
\newcommand{\PLs}{{\Sc{pl}}}%shortcut for PL in small caps
\newcommand{\NOMs}{{\Sc{nom}}}%shortcut for NOM in small caps
\newcommand{\ACCs}{{\Sc{acc}}}%shortcut for ACC in small caps
\newcommand{\GENs}{{\Sc{gen}}}%shortcut for GEN in small caps
\newcommand{\ILLs}{{\Sc{ill}}}%shortcut for ILL in small caps
\newcommand{\INESSs}{{\Sc{iness}}}%shortcut for INESS in small caps
\newcommand{\ELATs}{{\Sc{elat}}}%shortcut for ELAT in small caps
\newcommand{\COMs}{{\Sc{com}}}%shortcut for COM in small caps
\newcommand{\ABESSs}{{\Sc{abess}}}%shortcut for ABESS in small caps
\newcommand{\ESSs}{{\Sc{ess}}}%shortcut for ESS in small caps
%\newcommand{\DIMs}{{\Sc{dim}}}%shortcut for DIM in small caps
%\newcommand{\ORDs}{{\Sc{ord}}}%shortcut for ORD in small caps
%\newcommand{\CARDs}{{\Sc{card}}}%shortcut for CARD in small caps
\newcommand{\PROXs}{{\Sc{prox}}}%shortcut for PROX in small caps
\newcommand{\DISTs}{{\Sc{dist}}}%shortcut for DIST in small caps
\newcommand{\RMTs}{{\Sc{rmt}}}%shortcut for RMT in small caps
\newcommand{\REFLs}{{\Sc{refl}}}%shortcut for REFL in small caps
\newcommand{\PRSs}{{\Sc{prs}}}%shortcut for PRS in small caps
\newcommand{\PSTs}{{\Sc{pst}}}%shortcut for PST in small caps
\newcommand{\IMPs}{{\Sc{imp}}}%shortcut for IMP in small caps
\newcommand{\POTs}{{\Sc{pot}}}%shortcut for POT in small caps
\newcommand{\PROGs}{{\Sc{prog}}}%shortcut for PROG in small caps
\newcommand{\PRFs}{{\Sc{prf}}}%shortcut for PRF in small caps
\newcommand{\INFs}{{\Sc{inf}}}%shortcut for INF in small caps
\newcommand{\NEGs}{{\Sc{neg}}}%shortcut for NEG in small caps
\newcommand{\CONNEGs}{{\Sc{conneg}}}%shortcut for CONNEG in small caps

\newcommand{\subNP}{{\footnotesize\sub{NP}}}%shortcut for NP (nominal phrase) in subscript
\newcommand{\subVC}{{\footnotesize\sub{VC}}}%shortcut for VC (verb complex) in subscript
\newcommand{\subAP}{{\footnotesize\sub{AP}}}%shortcut for NP (adjectival phrase) in subscript
\newcommand{\subAdvP}{{\footnotesize\sub{AdvP}}}%shortcut for AdvP (adverbial phrase) in subscript
\newcommand{\subPP}{{\footnotesize\sub{PP}}}%shortcut for NP (postpoistional phrase) in subscript

\newcommand{\ipa}[1]{{\fontspec{Linux Libertine}#1}}%specifying font for IPA characters

\newcommand{\SEC}{§}%standardize section symbol and spacing afterwards
%\newcommand{\SEC}{§\,}%

\newcommand{\Nth}{{\footnotesize(\It{n})}}%used in table of numerals in ADJ chapter

%%newcommands for tables in introductionSDL.tex:
\newcommand{\cliticExs}[3]{\Tn{\begin{tabular}{p{28mm} c p{28mm} p{35mm}}\It{#1}&\ARROW &\It{#2} & ‘#3’\\\end{tabular}}}%specifically for the two clitic examples
\newcommand{\Grapheme}[1]{\It{#1}}%formatting for graphemes in orthography tables
%%new command for the section on orthographic examples; syntax: #1=orthography, #2=phonology, #3=gloss
\newcommand{\SpellEx}[3]{\Tn{\begin{tabular}{p{70pt} p{70pt} l}\ipa{/#2/}&\It{#1}& ‘#3’ \\\end{tabular}}}%formatting for orthographic examples (intro-Chapter)


%%new transl tier in gb4e; syntax: #1=free translation (in single quotes), #2=additional comments, z.B. literal meaning:
\newcommand{\Transl}[2]{\trans\Tn{‘#1’ #2}}%new transl tier in gb4e;
\newcommand{\TranslMulti}[2]{\trans\hspace{12pt}\Tn{‘#1’ #2}}%new transl tier in gb4e for a dialog to be included under a single example number


%% used for examples in the Prosody and Segmental phonology chapters:
\newcommand{\PhonGloss}[7]{%PhonGloss = Phonology Gloss;
%pattern: \PhonGloss{label}{phonemic}{phonetic}{orthographic}{gloss}{recording}{utterance}
\ea\label{#1}
\Tn{\begin{tabular}[t]{p{30mm} l}
\ipa{/#2/}	& \It{#4} \\
\ipa{[#3]}	&\HANG ‘#5’\\%no table row can start with square brackets! thus the workaround with \MC
\end{tabular}\hfill\hyperlink{#6}{{\small\textnormal[pit#6#7]}}%\index{Z\Red{rec}!\Red{pit#6}}\index{Z\Red{utt}!\Red{pit#6#7} \Blue{Phon}}
}
\z}
\newcommand{\PhonGlossWL}[6]{%PhonGloss = Phonology Gloss for words from WORDLIST, not from corpus!;
%pattern: \PhonGloss{label}{phonemic}{phonetic}{orthographic}{gloss}{wordListNumber}
\ea\label{#1}
\Tn{\begin{tabular}[t]{p{30mm} l}
\ipa{/#2/}	& \It{#4} \\
\ipa{[#3]}	&\HANG ‘#5’\\%no table row can start with square brackets! thus the workaround with \MC
\end{tabular}\hfill\hyperlink{explExs}{{\small\textnormal[#6]}}%\index{Z\Red{wl}!\Red{#6}\Blue{Phon}}
}
\z}

%%for derivation examples in the derivational morphology chapter!
%syntax: \DerivExam{#1}{#2}{#3}{#4}{#5}{#6}
%#1: base, #2: base-gloss, #3: derived form, #4: derived form gloss, #5: derived form translation, #6: pit-recording, #7: utterance number
\newcommand{\DW}{28mm}%for following three commands, to align arrows throughout
%%%%OLD:
%%%\newcommand{\DerivExam}[7]{\Tn{\begin{tabular}[t]{p{\DW}cl}\It{#1}&\ARROW&\It{#3}\\#2&&#4\\\end{tabular}\hfill\pbox{.3\textwidth}{\hfill‘#5’\\\hbox{}\hfill\hyperlink{pit#6}{{\small\textnormal[pit#6.#7]}}}
%%%%\index{Z\Red{rec}!\Red{pit#6}}\index{Z\Red{utt}!\Red{pit#6.#7}}
%%%}}
%NEW:
\newcommand{\DerivExam}[7]{\Tn{
\begin{tabular}[t]{p{\DW}x{5mm}l}\It{#1}&\ARROW&\It{#3}\\\end{tabular}\hfill‘#5’\\
\hspace{1mm}\begin{tabular}[t]{p{\DW}x{5mm}l}#2&&#4\\\end{tabular}\hfill\hyperlink{pit#6}{{\small\textnormal[pit#6.#7]}}
%\index{Z\Red{rec}!\Red{pit#6}}\index{Z\Red{utt}!\Red{pit#6.#7}}
}}
%%same as above, but supress any reference to a specific utterance
\newcommand{\DerivExamX}[7]{\Tn{
\begin{tabular}[t]{p{\DW}x{5mm}l}\It{#1}&\ARROW&\It{#3}\\\end{tabular}\hfill‘#5’\\
\hspace{1mm}\begin{tabular}[t]{p{\DW}x{5mm}l}#2&&#4\\\end{tabular}\hfill\hyperlink{pit#6}{{\small\textnormal[pit#6]\It{e}}}
%\index{Z\Red{rec}!\Red{pit#6}}\index{Z\Red{utt}!\Red{pit#6.#7}}
}}
\newcommand{\DerivExamWL}[6]{\Tn{
\begin{tabular}[t]{p{\DW}x{5mm}l}\It{#1}&\ARROW&\It{#3}\\\end{tabular}\hfill‘#5’\\
\hspace{1mm}\begin{tabular}[t]{p{\DW}x{5mm}l}#2&&#4\\\end{tabular}\hfill\hyperlink{explExs}{{\small\textnormal[#6]}}
%\index{Z\Red{wl}!\Red{#6}}
}}


%formatting of corpus source information (after \transl in gb4e-environments):
\newcommand{\Corpus}[2]{\hspace*{1pt}\hfill{\small\mbox{\hyperlink{pit#1}{\Tn{[pit#1.#2]}}}}%\index{Z\Red{rec}!\Red{pit#1}}\index{Z\Red{utt}!\Red{pit#1.#2}}
}%
\newcommand{\CorpusE}[2]{\hspace*{1pt}\hfill{\small\mbox{\hyperlink{pit#1}{\Tn{[pit#1.#2]}}\It{e}}}%\index{Z\Red{rec}!\Red{pit#1}}\index{Z\Red{utt}!\Red{pit#1.#2}\Blue{-E}}
}%
%%as above, but necessary for recording names which include an underline because the first variable in \href understands _ but the second variable requires \_
\newcommand{\CorpusLink}[3]{\hspace*{1pt}\hfill{\small\mbox{\hyperlink{pit#1}{\Tn{[pit#2.#3]}}}}%\index{Z\Red{rec}!\Red{pit#2}}\index{Z\Red{utt}!\Red{pit#2.#3}}
}%
%%as above, but for newer recordings which begin with sje20 instead of pit
\newcommand{\CorpusSJE}[2]{\hspace*{1pt}\hfill{\small\mbox{\hyperlink{sje20#1}{\Tn{[sje20#1.#2]}}}}%\index{Z\Red{rec}!\Red{sje20#1}}\index{Z\Red{utt}!\Red{sje20#1.#2}}
}%
\newcommand{\CorpusSJEE}[2]{\hspace*{1pt}\hfill{\small\mbox{\hyperlink{sje20#1}{\Tn{[sje20#1.#2]}}\It{e}}}%\index{Z\Red{rec}!\Red{sje20#1}}\index{Z\Red{utt}!\Red{sje20#1.#2}\Blue{-E}}
}%











%%hyphenation points for line breaks
%%add to TeX file before \begin{document} with:
%%%%hyphenation points for line breaks
%%add to TeX file before \begin{document} with:
%%%%hyphenation points for line breaks
%%add to TeX file before \begin{document} with:
%%\include{hyphenationSDL}
\hyphenation{
ab-es-sive
affri-ca-te
affri-ca-tes
Ahka-javv-re
al-ve-o-lar
com-ple-ments
%check this:
de-cad-es
fri-ca-tive
fri-ca-tives
gemi-nate
gemi-nates
gra-pheme
gra-phemes
ho-mo-pho-nous
ho-mor-ga-nic
mor-pho-syn-tac-tic
or-tho-gra-phic
pho-neme
pho-ne-mes
phra-ses
post-po-si-tion
post-po-si-tion-al
pre-as-pi-ra-te
pre-as-pi-ra-ted
pre-as-pi-ra-tion
seg-ment
un-voiced
wor-king-ver-sion
}
\hyphenation{
ab-es-sive
affri-ca-te
affri-ca-tes
Ahka-javv-re
al-ve-o-lar
com-ple-ments
%check this:
de-cad-es
fri-ca-tive
fri-ca-tives
gemi-nate
gemi-nates
gra-pheme
gra-phemes
ho-mo-pho-nous
ho-mor-ga-nic
mor-pho-syn-tac-tic
or-tho-gra-phic
pho-neme
pho-ne-mes
phra-ses
post-po-si-tion
post-po-si-tion-al
pre-as-pi-ra-te
pre-as-pi-ra-ted
pre-as-pi-ra-tion
seg-ment
un-voiced
wor-king-ver-sion
}
\hyphenation{
ab-es-sive
affri-ca-te
affri-ca-tes
Ahka-javv-re
al-ve-o-lar
com-ple-ments
%check this:
de-cad-es
fri-ca-tive
fri-ca-tives
gemi-nate
gemi-nates
gra-pheme
gra-phemes
ho-mo-pho-nous
ho-mor-ga-nic
mor-pho-syn-tac-tic
or-tho-gra-phic
pho-neme
pho-ne-mes
phra-ses
post-po-si-tion
post-po-si-tion-al
pre-as-pi-ra-te
pre-as-pi-ra-ted
pre-as-pi-ra-tion
seg-ment
un-voiced
wor-king-ver-sion
}\begin{document}

%%%%%%%%%%%%%%%%%%%%%%%%%%%%%%%% ALL THE ABOVE TO BE COMMENTED OUT FOR COMPLETE DOCUMENT! %%%%%%%%%%%




\chapter{Overview of Word Classes}\label{introWordForms}\index{parts of speech}\index{word classes}
By characterizing the syntactic and morphological behavior of %syntactically independent %UM: ??
words in \PS, and grouping such words based on that behavior, a total of seven word classes can be distinguished. These can be divided into two general categories containing generally \It{open} word classes and \It{closed} word classes, and are listed in Table \ref{wordClassList}. 
The specific syntactic criteria and inflectional categories defining these are summarized in Table \ref{wordClassSummary1}\footnote{The abbreviation} 
on page \pageref{wordClassSummary1}. % as follows:
%\begin{itemize*}\item{open word classes:\begin{itemize*}\item{verbs}\item{nouns}\item{adjectives}\item{adverbs}\end{itemize*}}\item{closed word classes:\begin{itemize*}\item{pronouns}\item{demonstratives}\item{numerals}\item{verbal particles}\item{postpositions}\item{conjunctions}\item{interjections}%\item{}\end{itemize*}}\end{itemize*}
\begin{table}\centering
\caption[\PS\ word classes]{\PS\ word classes and the relevant chapter/section}\label{wordClassList}
\begin{tabular}{l l c | l c}
\MC{2}{l}{\it open word classes}&\it Ch./Sec.	&\it closed word classes&\it Ch./Sec.	\\\dline
\MC{2}{l}{\bf nominals}&				&\bf adpositions &\ref{adpositions}		\\
	&nouns	&\ref{nouns}			&\bf conjunctions &\ref{conjunctions}\\
	&pronouns&\ref{pronouns}		&\bf interjections 	&\ref{interjections}\\
\MC{2}{l}{\bf adjectivals}&				&&\\
	&adjectives &\ref{adjectivesIntro}	&&\\
	&numerals &\ref{numerals}		&&\\
\MC{2}{l}{\bf verbs}&\ref{verbs}			&&\\
\MC{2}{l}{\bf adverbs}&\ref{adverbs}	&&\\
%\MC{2}{l}{\bf particles}&\ref{particles}		&&\\
\end{tabular}
%\begin{tabular}{c c c}
%open word classes	&&closed word classes	\\\dline
%nominals
%				&&postpositions		\\
%%				&&verbal particles	\\
%%				&&pronouns		\\
%adjectivals
%				&&conjunctions		\\
%%				&&demonstratives	\\
%verbs			
%				&&particles		\\
%%				&&numerals		\\
%adverbs			
%				&&interjections		\\\hline
%\end{tabular}
\end{table}
%The division of \PS\ words into these eight word classes is based %first and foremost 
%on recurring syntactic patterns as well as by morphological behavior concerning which inflectional categories are relevant. This is 
%The specific syntactic criteria and inflectional categories defining these are summarized in Table \ref{wordClassSummary1} on page \pageref{wordClassSummary1}. % as follows:

Some word classes consist of two or more subclasses: %, each of which corresponds to a single lexeme type: %UM: delete this!
\It{nominals} refer to \It{nouns} %(including a subcategory for predicative adjectives) 
and \It{pronouns} (personal, reflexive, relative, demonstrative, indefinite, relative and interrogative), and \It{adjectivals} include both \It{adjectives} and \It{numerals}. %Sets of finite and non-finite \It{verbs} belong to the same lexeme. 
Note that pronouns and numerals are closed subclasses belonging to open classes.

This categorization is intended to provide a broad starting point for classifying \PS\ words; details for each word class can be found in the relevant chapters below. 
%Although nouns and pronouns are both considered nominals, they are dealt with in separate chapters for the sake of clarity. 
Chapter \ref{nouns} concerns the nominal subclass \It{nouns}, which provide a fairly straightforward example of the morphophonological complexities involved in inflectional paradigms, while 
the nominal subclass \It{pronouns} is dealt with in the following chapter (Ch. \ref{pronouns}). % for the sake of clarity. 
Chapter \ref{adjectivesIntro} then covers the adjectival subclasses \It{adjectives} and \It{numerals}. Following this, Chapter \ref{verbs} deals with \It{verbs}. Finally, the remaining small classes (\It{adverbials}, \It{adpositions}, \It{conjunctions} %, \It{particles}, 
and \It{interjections}) are covered in Chapter \ref{otherWordClasses}. 
% are sufficiently complex or not nece and traditionally considered unique word classes in other languages, they are given

\vfill
\begin{table}\centering
\caption[Syntactic and morphological criteria for word classes]{Summary of syntactic and morphological criteria for word classes}\label{wordClassSummary1}
\begin{tabular}[l]{l p{140pt} p{170pt}}
%\begin{longtable}[l]{l l p{100pt}}
\Bf{word class}	&\Bf{syntactic criteria}						&\Bf{inflectional categories}\\\dline
nominals		&head of a nominal phrase					&case/number\\\hline
verbs		&head of a verb complex						&tense/mood/person/number,\\%\hline
			&										&non-finite forms (negation, aspect)\\\hline
adjectivals		&head of an adjectival phrase					&number for predicate adjectives\\\hline%, but can head an NP in elliptic constructions\\\hline
adverbials		&head of an adverbial phrase					&-\\\hline
%pronouns		&substitutes an NP							&case/number\\\hline%part of NOUNS
%demonstratives	&specifies deictic relationship of an NP			&case/number\\\hline%part of NOUNS
%numerals		&specifies count of an NP						&-\\\hline%part of ADJ
%verbal particles	&complement? to VC head					&-\\\hline%part of ADV
adpositions	&head of an adpositional phrase					&-\\\hline
conjunctions	&connect words, phrases, clauses, texts			&-\\\hline
particles		&independent words within clauses				&-\\\hline
interjections	&independent words at clause-level				&-\\\hline
%\end{longtable}
\end{tabular}
\end{table}

\vfill




%Further support for some of these categories can be found in their morphological behavior. Initially, these can be divided into inflectional and non-inflectional word classes, as in Figure \ref{wordClassesInflectionally}: %was also considered for word classes subject to inflectional morphology. 
%\begin{figure}\centering
%\begin{tabular}{c|c|c}
%				&					&adverbs	\\
%verbs			&					&verbal particles	\\
%nouns			&					&postpositions	\\
%pronouns			&adjectives			&conjunctions	\\
%demonstratives		&numerals			&interjections	\\\hline
%%\bf always inflectional&\bf occasionally inflectional&\bf never inflectional	\\\hline
%\bf always			&\bf as NP-head		&\bf never	\\%\hline
%%\MC{3}{l}{inflect}\\
%\end{tabular}
%\caption{\PS\ word classes grouped by ability to inflect}\label{wordClassesInflectionally}
%\end{figure}
%
%\begin{tabular}{l p{100pt} l}
%\bf clearly inflectional	&\MC{2}{l}{\bf inflectional categories}	\\\dline
%verb				&\MC{2}{l}{verb class, tense/mood/person/number}\\%\hline
%noun				&\MC{2}{l}{noun class, case/number}\\
%pronoun			&\MC{2}{l}{case/number}\\
%demonstrative		&\MC{2}{l}{case/number}\\
%\bf partially inflectional	&\MC{2}{l}{\bf }	\\\dline
%\MR{2}{*}{adjective}	&as head of NP in elliptical construction	&case/number	\\
%				&normally none\\
%\end{tabular}




%\newcommand{\PBa}[1]{\parbox{140pt}{\vspace{2pt}#1}\vspace{2pt}}%shortcut for \parbox to allow line breaks within table cells
%\newcommand{\PBb}[1]{\parbox{100pt}{\vspace{2pt}#1}\vspace{2pt}}%shortcut for \parbox to allow line breaks within table cells
%%\renewcommand{\X}{\PLUS}
%
%\begin{sidewaystable}\centering
%\begin{tabular}{|c|c||c|p{140pt}||c|p{140pt}||p{100pt}|}\hline
%\MC{2}{|c||}{wordform}		&\MC{2}{c||}{morphological}				&\MC{2}{c||}{syntactic}	& \\%\dline
%\MC{1}{|c|}{category}	&\MC{1}{c||}{type}&type &\MC{1}{c||}{definition}			&type&\MC{1}{c||}{definition}	&\MC{1}{c|}{other} \\\dline
%%%%NOUNS
%\bf noun			&open	&nominal	&\X\ inflects for class, case/number	&\MR{12}{*}{\begin{sideways}\parbox{8mm}{argument}\end{sideways}}
%																&\X\ head of NP
%																				& \\\cline{1-4}\cline{6-7}%\hline
%%%%PRNOUNS
%\bf pronoun		&closed	&nominal	&\X\ inflects for case/number		&&\X\ substitutes for NP
%																				& \\\cline{1-4}\cline{6-7}%\hline
%%%%DEMONSTRATIVES
%\bf demonstrative	&closed	&nominal&\X\ inflects for case/number		&&\PBa{\X\ modifies head of NP \\
%																\X\ can occur alone (ellipsis)}		
%																				& \\\cline{1-4}\cline{6-7}%\hline
%%\bf dem-ADJ		&closed	&nominal&\X\ inflects for case/number		&&\X\ dep of NP			& \\\hline
%%%%ADJECTIVES
%\bf adjective		&open	&nominal&\PBa{\X\ no inflection in \ATTRs \\
%									\X\ base for \COMPs, \SUPERLs\\
%									\X\ inflects for case/number in ellipsis}
%												 			&&\PBa{\X\ head of AP \\
%																\X\ modifies a noun \\
%																\X\ can occur alone (ellipsis)}
%																	&\PBb{\X\ has corresp.\\ \PREDs\ form}
%																				 \\\cline{1-4}\cline{6-7}%\hline
%%%%NUMERALS
%\bf numeral		&closed	&non-infl.&-							&&\PBa{\X\ modifies a noun \\
%																\X\ can occur alone (ellipsis)}	
%																				& \\\hline
%%%%VERBS
%\bf verb			&open	&verbal	&\X\ inflects for class, person/number/tense/mood&\MR{5}{*}{\begin{sideways}\parbox{8mm}{verb}\end{sideways}}
%																		&\X\ head of VP
%																				& \\\cline{1-4}\cline{6-7}%\hline
%%%%VERBAL PARTICLES
%\bf verbal particle	&closed	&non-infl.	&-						&&\X\ co-occur with (certain) verbs	
%																				& \\\hline
%%%%ADVERBS
%\bf adverb			&open?	&non-infl.	&-						&\MR{3}{*}{\begin{sideways}\parbox{8mm}{adverb}\end{sideways}}
%															&\PBa{\X\ head of AdvP? \\
%																\X\ modifies clauses, verbs, adjs, advs}
%																				& \\\cline{1-4}\cline{6-7}%\hline
%%%%POSTPOSITIONS
%\bf postposition		&closed	&non-infl.&-						&&\X\ head of PP
%																				& \\\hline
%%%%%DISCOURSE PARTICLES
%%\bf discourse particle	&closed	&non-infl.&-						&&\X\ not part of any phrase?
%%																				& actually sentence adverb?\\\hline
%%%%CONJUNCTIONS
%\bf conjunction		&closed	&non-infl.&-						&\MR{5}{*}{\begin{sideways}\parbox{8mm}{free}\end{sideways}}
%															&\X\ connects words, phrases, clauses
%																				& \\\cline{1-4}\cline{6-7}%\hline
%%%%INTERJECTIONS
%\bf interjection		&closed	&non-infl.&-						&&\X\ stands alone, not connected to neighboring clauses	
%																				& \\\hline
%
%
%
%
%%\dline&&&&&\\\dline
%%\bf noun			&open	&nominal&\X\ inflects for case/number	&\parbox{100pt}{\X\ head of NP\\\X\  pig}		& \\%\hline
%%				&		&&								&\X\ can be modified by AP or DEM-ADJ		& \\\hline
%%\bf pronoun		&closed	&nominal&\X\ inflects for case/number	&\X\ head of NP?		& \\%\hline
%%				&		&		&						&\X\ substitutes for NP?	&  \\%\hline
%%				&		&&								&\X\ no modification		& \\\hline
%%\bf demonstrative	&closed	&nominal&\X\ inflects for case/number	&\X\ head of NP		& \\%\hline
%%				&		&&								&\X\ no modification		& \\\hline
%%\bf dem-ADJ		&closed	&nominal&\X\ inflects for case/number	&\X\ dep of NP			& \\\hline
%%\bf adjective		&open	&nominal&\X\ no inflection in \ATTRs 			&\X\ head of AP		&\X\ has corresp. \\%\hline
%%				&		&		&\X\ base for \COMPs, \SUPERLs\	&	& \PREDs\ form \\%\hline
%%				&		&		&\X\ inflects for case/number in ellipsis constructions	&	& \\\hline
%%\bf verb			&open	&verbal	&\X\ inflects for person/number/tense/mood	&\X\ head of V-complex/VP?	& \\\hline
%%\bf adverb			&open?	&non-infl.&\X\ ?						&\X\ head of AdvP?		& \\%\hline
%%				&		&&								&\X\ modifies any non-nouns?		& \\\hline
%%\bf verbal particle	&closed	&non-infl.	&						&\X\ co-occur with (certain) verbs	& \\\hline
%%\bf postposition		&closed	&non-infl.&						&\X\ head of PP		& \\\hline
%%\bf discourse particle	&closed	&non-infl.&						&\X\ not part of any phrase?& \\\hline
%%\bf conjunction		&closed	&non-infl.&						&\X\ connects words, phrases, clauses		& \\\hline
%%\bf numeral		&closed	&various?&\MC{2}{c|}{various categories, not its own though!}		& \\\hline
%%\bf interjection		&closed	&non-infl.&						&\X\ stands alone, not connected to neighboring clauses		& \\\hline
%\end{tabular}
%\caption{Summary of \PS\ word forms}\label{wordFormSummary}
%\end{sidewaystable}















%%%%%%%% THIS IS NOT USED FOR THE ENTIRE COMPILATION, but only for individual chapters!!!!

\clearpage
\addcontentsline{toc}{chapter}{Bibliography}\label{Bibliography}
\bibliography{PiteGrammarBibSDL}%for bibtex
%\printbibliography%[title=Works Cited]%%for biber!






%%%NAME INDEX doesn’t work!?!? why???
\cleardoublepage\phantomsection%this allows hyperlink in ToC to work
\addcontentsline{toc}{chapter}{Name index}
\ohead{Name index}
\printindex[aut]

\cleardoublepage\phantomsection%this allows hyperlink in ToC to work
\addcontentsline{toc}{chapter}{Language index}
\ohead{Language index}
\printindex[lan]

\cleardoublepage\phantomsection%this allows hyperlink in ToC to work
\addcontentsline{toc}{chapter}{Subject index}
\ohead{Subject index}
\printindex


\end{document}
%\end{document}