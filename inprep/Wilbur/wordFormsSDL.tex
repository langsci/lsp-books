%\documentclass[ number=5
			   ,series=sidl
			   ,isbn=xxx-x-xxxxxx-xx-x
			   ,url=http://langsci-press.org/catalog/book/17
			   ,output=long   % long|short|inprep              
			   %,blackandwhite
			   %,smallfont
			   ,draftmode   
			  ]{LSP/langsci}                          

\usepackage{LSP/lsp-styles/lsp-gb4e}		% verhindert Komma bei mehrfachen Fußnoten?
\usepackage{LSP/lsp-styles/avm}
\avmfont{\sc} 
\avmvalfont{\it}
                                                      
\usepackage{layout}
\usepackage{lipsum}

%%for LSP-lines in tables: %%doesn’t work for some reason. Plus, not all my tables have a single-line header. Double-lines aren’t typeset properly in ‘longtable’-environment across several pages.
%\usepackage{booktabs}
%\newcommand{\mytoprule}{\midrule\toprule}
%\newcommand{\mybottomrule}{\bottomrule\midrule}

%%%following now in main document (XeTeX_pitePhDSDL.tex)
%\title{A corpus-based grammar \\ of spoken Pite Saami}  
%%\subtitle{2000+ Years of Language Science and no End in Sight}  
%\author{Joshua Wilbur}
%\dedication{Gijtov adnet!}
%\renewcommand{\lsBackBody}{This grammar of Pite Saami (Uralic; Sweden) is simply bad-ass.}%for back cover text
%\renewcommand{\lsBackTitle}{Biddumsáme giella}%for back title

%%%% ABOVE FOR LangSciPress %%%%
%%%% ABOVE FOR LangSciPress %%%%
%%%% ABOVE FOR LangSciPress %%%%

\usepackage{longtable}

\usepackage{multirow}
\usepackage{array} %allows, among other things, centering column content in a table while also specifying width, creates new column style "x" for center-alignment, "y" for right-alignment
\newcolumntype{x}[1]{%
>{\centering\hspace{0pt}}p{#1}}%
\newcolumntype{y}[1]{%
>{\raggedleft\hspace{0pt}}p{#1}}%

\usepackage[]{placeins}%using \FloatBarrier command, all floats still floating at that point will be typeset, and cannot cross that boundary. the option here \usepackage[section]{placeins} automatically adds \FloatBarrier to every \section command (only works for \section commands, nothing lower than that!)
\usepackage{afterpage}%by using the command \afterpage{\clearpage}, all floats will appear, but no new page will be started, thus avoiding bad page breaks around floats

\usepackage{vowel} %for vowel space chart


%%IS THIS NECESSARY??
%%%following allows you to refer to footnotes (from http://anthony.liekens.net/index.php/LaTeX/MultipleFootnoteReferences)
\newcommand{\footnoteremember}[2]{
  \footnote{#2}
  \newcounter{#1}
  \setcounter{#1}{\value{footnote}}
} \newcommand{\footnoterecall}[1]{
  \footnotemark[\value{#1}]} 
%%%previous allows you to refer to footnotes: use \footnoteremember{referenceText} in footnote, then \footnoterecall{referenceText} to refer.

\usepackage{tikz}
\usetikzlibrary{plothandlers,matrix,decorations.text,shapes.arrows,shadows,chains,positioning,scopes}

\usepackage{synttree} %zeichnet linguistische Bäume
\branchheight{36pt}%sets height between rows in synttree

\usepackage{lscape}%used for landscape pages in index (list of recordings)

\usepackage{polyglossia}
\setmainlanguage{english}



%%%TAKE OUT FOR FINAL VERSION:
%%%TAKE OUT FOR FINAL VERSION:
%%%TAKE OUT FOR FINAL VERSION:

%%%following readjusts margin text!
\setlength{\marginparwidth}{20mm}
\let\oldmarginpar\marginpar
\renewcommand\marginpar[1]{\-\oldmarginpar[\raggedleft\footnotesize\vspace{-7pt}\color{red}\It{→ #1}]%
{\raggedright\footnotesize\vspace{-7pt}\color{red}\It{→ #1}}}
%%%previous readjusts margin text!

%%%The following lines set depth of ToC (LSP default is only 3 levels)!
%%%\renewcommand{\contentsname}{Table of Contents} % überschrift des inhaltsverzeichnisses
%\setcounter{secnumdepth}{5}%sets how deep section/subsection/subsubsections are numbered
%\setcounter{tocdepth}{5}%sets the depth of the ToC %but this doesn't seem to work!!!
\include{newcommandsSDL}\include{hyphenationSDL}\begin{document}

%%%%%%%%%%%%%%%%%%%%%%%%%%%%%%%% ALL THE ABOVE TO BE COMMENTED OUT FOR COMPLETE DOCUMENT! %%%%%%%%%%%




\chapter{Overview of Word Classes}\label{introWordForms}\index{parts of speech}\index{word classes}
By characterizing the syntactic and morphological behavior of %syntactically independent %UM: ??
words in \PS, and grouping such words based on that behavior, a total of seven word classes can be distinguished. These can be divided into two general categories containing generally \It{open} word classes and \It{closed} word classes, and are listed in Table \ref{wordClassList}. 
The specific syntactic criteria and inflectional categories defining these are summarized in Table \ref{wordClassSummary1}\footnote{The abbreviation} 
on page \pageref{wordClassSummary1}. % as follows:
%\begin{itemize*}\item{open word classes:\begin{itemize*}\item{verbs}\item{nouns}\item{adjectives}\item{adverbs}\end{itemize*}}\item{closed word classes:\begin{itemize*}\item{pronouns}\item{demonstratives}\item{numerals}\item{verbal particles}\item{postpositions}\item{conjunctions}\item{interjections}%\item{}\end{itemize*}}\end{itemize*}
\begin{table}\centering
\caption[\PS\ word classes]{\PS\ word classes and the relevant chapter/section}\label{wordClassList}
\begin{tabular}{l l c | l c}
\MC{2}{l}{\it open word classes}&\it Ch./Sec.	&\it closed word classes&\it Ch./Sec.	\\\dline
\MC{2}{l}{\bf nominals}&				&\bf adpositions &\ref{adpositions}		\\
	&nouns	&\ref{nouns}			&\bf conjunctions &\ref{conjunctions}\\
	&pronouns&\ref{pronouns}		&\bf interjections 	&\ref{interjections}\\
\MC{2}{l}{\bf adjectivals}&				&&\\
	&adjectives &\ref{adjectivesIntro}	&&\\
	&numerals &\ref{numerals}		&&\\
\MC{2}{l}{\bf verbs}&\ref{verbs}			&&\\
\MC{2}{l}{\bf adverbs}&\ref{adverbs}	&&\\
%\MC{2}{l}{\bf particles}&\ref{particles}		&&\\
\end{tabular}
%\begin{tabular}{c c c}
%open word classes	&&closed word classes	\\\dline
%nominals
%				&&postpositions		\\
%%				&&verbal particles	\\
%%				&&pronouns		\\
%adjectivals
%				&&conjunctions		\\
%%				&&demonstratives	\\
%verbs			
%				&&particles		\\
%%				&&numerals		\\
%adverbs			
%				&&interjections		\\\hline
%\end{tabular}
\end{table}
%The division of \PS\ words into these eight word classes is based %first and foremost 
%on recurring syntactic patterns as well as by morphological behavior concerning which inflectional categories are relevant. This is 
%The specific syntactic criteria and inflectional categories defining these are summarized in Table \ref{wordClassSummary1} on page \pageref{wordClassSummary1}. % as follows:

Some word classes consist of two or more subclasses: %, each of which corresponds to a single lexeme type: %UM: delete this!
\It{nominals} refer to \It{nouns} %(including a subcategory for predicative adjectives) 
and \It{pronouns} (personal, reflexive, relative, demonstrative, indefinite, relative and interrogative), and \It{adjectivals} include both \It{adjectives} and \It{numerals}. %Sets of finite and non-finite \It{verbs} belong to the same lexeme. 
Note that pronouns and numerals are closed subclasses belonging to open classes.

This categorization is intended to provide a broad starting point for classifying \PS\ words; details for each word class can be found in the relevant chapters below. 
%Although nouns and pronouns are both considered nominals, they are dealt with in separate chapters for the sake of clarity. 
Chapter \ref{nouns} concerns the nominal subclass \It{nouns}, which provide a fairly straightforward example of the morphophonological complexities involved in inflectional paradigms, while 
the nominal subclass \It{pronouns} is dealt with in the following chapter (Ch. \ref{pronouns}). % for the sake of clarity. 
Chapter \ref{adjectivesIntro} then covers the adjectival subclasses \It{adjectives} and \It{numerals}. Following this, Chapter \ref{verbs} deals with \It{verbs}. Finally, the remaining small classes (\It{adverbials}, \It{adpositions}, \It{conjunctions} %, \It{particles}, 
and \It{interjections}) are covered in Chapter \ref{otherWordClasses}. 
% are sufficiently complex or not nece and traditionally considered unique word classes in other languages, they are given

\vfill
\begin{table}\centering
\caption[Syntactic and morphological criteria for word classes]{Summary of syntactic and morphological criteria for word classes}\label{wordClassSummary1}
\begin{tabular}[l]{l p{140pt} p{170pt}}
%\begin{longtable}[l]{l l p{100pt}}
\Bf{word class}	&\Bf{syntactic criteria}						&\Bf{inflectional categories}\\\dline
nominals		&head of a nominal phrase					&case/number\\\hline
verbs		&head of a verb complex						&tense/mood/person/number,\\%\hline
			&										&non-finite forms (negation, aspect)\\\hline
adjectivals		&head of an adjectival phrase					&number for predicate adjectives\\\hline%, but can head an NP in elliptic constructions\\\hline
adverbials		&head of an adverbial phrase					&-\\\hline
%pronouns		&substitutes an NP							&case/number\\\hline%part of NOUNS
%demonstratives	&specifies deictic relationship of an NP			&case/number\\\hline%part of NOUNS
%numerals		&specifies count of an NP						&-\\\hline%part of ADJ
%verbal particles	&complement? to VC head					&-\\\hline%part of ADV
adpositions	&head of an adpositional phrase					&-\\\hline
conjunctions	&connect words, phrases, clauses, texts			&-\\\hline
particles		&independent words within clauses				&-\\\hline
interjections	&independent words at clause-level				&-\\\hline
%\end{longtable}
\end{tabular}
\end{table}

\vfill




%Further support for some of these categories can be found in their morphological behavior. Initially, these can be divided into inflectional and non-inflectional word classes, as in Figure \ref{wordClassesInflectionally}: %was also considered for word classes subject to inflectional morphology. 
%\begin{figure}\centering
%\begin{tabular}{c|c|c}
%				&					&adverbs	\\
%verbs			&					&verbal particles	\\
%nouns			&					&postpositions	\\
%pronouns			&adjectives			&conjunctions	\\
%demonstratives		&numerals			&interjections	\\\hline
%%\bf always inflectional&\bf occasionally inflectional&\bf never inflectional	\\\hline
%\bf always			&\bf as NP-head		&\bf never	\\%\hline
%%\MC{3}{l}{inflect}\\
%\end{tabular}
%\caption{\PS\ word classes grouped by ability to inflect}\label{wordClassesInflectionally}
%\end{figure}
%
%\begin{tabular}{l p{100pt} l}
%\bf clearly inflectional	&\MC{2}{l}{\bf inflectional categories}	\\\dline
%verb				&\MC{2}{l}{verb class, tense/mood/person/number}\\%\hline
%noun				&\MC{2}{l}{noun class, case/number}\\
%pronoun			&\MC{2}{l}{case/number}\\
%demonstrative		&\MC{2}{l}{case/number}\\
%\bf partially inflectional	&\MC{2}{l}{\bf }	\\\dline
%\MR{2}{*}{adjective}	&as head of NP in elliptical construction	&case/number	\\
%				&normally none\\
%\end{tabular}




%\newcommand{\PBa}[1]{\parbox{140pt}{\vspace{2pt}#1}\vspace{2pt}}%shortcut for \parbox to allow line breaks within table cells
%\newcommand{\PBb}[1]{\parbox{100pt}{\vspace{2pt}#1}\vspace{2pt}}%shortcut for \parbox to allow line breaks within table cells
%%\renewcommand{\X}{\PLUS}
%
%\begin{sidewaystable}\centering
%\begin{tabular}{|c|c||c|p{140pt}||c|p{140pt}||p{100pt}|}\hline
%\MC{2}{|c||}{wordform}		&\MC{2}{c||}{morphological}				&\MC{2}{c||}{syntactic}	& \\%\dline
%\MC{1}{|c|}{category}	&\MC{1}{c||}{type}&type &\MC{1}{c||}{definition}			&type&\MC{1}{c||}{definition}	&\MC{1}{c|}{other} \\\dline
%%%%NOUNS
%\bf noun			&open	&nominal	&\X\ inflects for class, case/number	&\MR{12}{*}{\begin{sideways}\parbox{8mm}{argument}\end{sideways}}
%																&\X\ head of NP
%																				& \\\cline{1-4}\cline{6-7}%\hline
%%%%PRNOUNS
%\bf pronoun		&closed	&nominal	&\X\ inflects for case/number		&&\X\ substitutes for NP
%																				& \\\cline{1-4}\cline{6-7}%\hline
%%%%DEMONSTRATIVES
%\bf demonstrative	&closed	&nominal&\X\ inflects for case/number		&&\PBa{\X\ modifies head of NP \\
%																\X\ can occur alone (ellipsis)}		
%																				& \\\cline{1-4}\cline{6-7}%\hline
%%\bf dem-ADJ		&closed	&nominal&\X\ inflects for case/number		&&\X\ dep of NP			& \\\hline
%%%%ADJECTIVES
%\bf adjective		&open	&nominal&\PBa{\X\ no inflection in \ATTRs \\
%									\X\ base for \COMPs, \SUPERLs\\
%									\X\ inflects for case/number in ellipsis}
%												 			&&\PBa{\X\ head of AP \\
%																\X\ modifies a noun \\
%																\X\ can occur alone (ellipsis)}
%																	&\PBb{\X\ has corresp.\\ \PREDs\ form}
%																				 \\\cline{1-4}\cline{6-7}%\hline
%%%%NUMERALS
%\bf numeral		&closed	&non-infl.&-							&&\PBa{\X\ modifies a noun \\
%																\X\ can occur alone (ellipsis)}	
%																				& \\\hline
%%%%VERBS
%\bf verb			&open	&verbal	&\X\ inflects for class, person/number/tense/mood&\MR{5}{*}{\begin{sideways}\parbox{8mm}{verb}\end{sideways}}
%																		&\X\ head of VP
%																				& \\\cline{1-4}\cline{6-7}%\hline
%%%%VERBAL PARTICLES
%\bf verbal particle	&closed	&non-infl.	&-						&&\X\ co-occur with (certain) verbs	
%																				& \\\hline
%%%%ADVERBS
%\bf adverb			&open?	&non-infl.	&-						&\MR{3}{*}{\begin{sideways}\parbox{8mm}{adverb}\end{sideways}}
%															&\PBa{\X\ head of AdvP? \\
%																\X\ modifies clauses, verbs, adjs, advs}
%																				& \\\cline{1-4}\cline{6-7}%\hline
%%%%POSTPOSITIONS
%\bf postposition		&closed	&non-infl.&-						&&\X\ head of PP
%																				& \\\hline
%%%%%DISCOURSE PARTICLES
%%\bf discourse particle	&closed	&non-infl.&-						&&\X\ not part of any phrase?
%%																				& actually sentence adverb?\\\hline
%%%%CONJUNCTIONS
%\bf conjunction		&closed	&non-infl.&-						&\MR{5}{*}{\begin{sideways}\parbox{8mm}{free}\end{sideways}}
%															&\X\ connects words, phrases, clauses
%																				& \\\cline{1-4}\cline{6-7}%\hline
%%%%INTERJECTIONS
%\bf interjection		&closed	&non-infl.&-						&&\X\ stands alone, not connected to neighboring clauses	
%																				& \\\hline
%
%
%
%
%%\dline&&&&&\\\dline
%%\bf noun			&open	&nominal&\X\ inflects for case/number	&\parbox{100pt}{\X\ head of NP\\\X\  pig}		& \\%\hline
%%				&		&&								&\X\ can be modified by AP or DEM-ADJ		& \\\hline
%%\bf pronoun		&closed	&nominal&\X\ inflects for case/number	&\X\ head of NP?		& \\%\hline
%%				&		&		&						&\X\ substitutes for NP?	&  \\%\hline
%%				&		&&								&\X\ no modification		& \\\hline
%%\bf demonstrative	&closed	&nominal&\X\ inflects for case/number	&\X\ head of NP		& \\%\hline
%%				&		&&								&\X\ no modification		& \\\hline
%%\bf dem-ADJ		&closed	&nominal&\X\ inflects for case/number	&\X\ dep of NP			& \\\hline
%%\bf adjective		&open	&nominal&\X\ no inflection in \ATTRs 			&\X\ head of AP		&\X\ has corresp. \\%\hline
%%				&		&		&\X\ base for \COMPs, \SUPERLs\	&	& \PREDs\ form \\%\hline
%%				&		&		&\X\ inflects for case/number in ellipsis constructions	&	& \\\hline
%%\bf verb			&open	&verbal	&\X\ inflects for person/number/tense/mood	&\X\ head of V-complex/VP?	& \\\hline
%%\bf adverb			&open?	&non-infl.&\X\ ?						&\X\ head of AdvP?		& \\%\hline
%%				&		&&								&\X\ modifies any non-nouns?		& \\\hline
%%\bf verbal particle	&closed	&non-infl.	&						&\X\ co-occur with (certain) verbs	& \\\hline
%%\bf postposition		&closed	&non-infl.&						&\X\ head of PP		& \\\hline
%%\bf discourse particle	&closed	&non-infl.&						&\X\ not part of any phrase?& \\\hline
%%\bf conjunction		&closed	&non-infl.&						&\X\ connects words, phrases, clauses		& \\\hline
%%\bf numeral		&closed	&various?&\MC{2}{c|}{various categories, not its own though!}		& \\\hline
%%\bf interjection		&closed	&non-infl.&						&\X\ stands alone, not connected to neighboring clauses		& \\\hline
%\end{tabular}
%\caption{Summary of \PS\ word forms}\label{wordFormSummary}
%\end{sidewaystable}















%%%%%%%% THIS IS NOT USED FOR THE ENTIRE COMPILATION, but only for individual chapters!!!!

\clearpage
\addcontentsline{toc}{chapter}{Bibliography}\label{Bibliography}
\bibliography{PiteGrammarBibSDL}%for bibtex
%\printbibliography%[title=Works Cited]%%for biber!






%%%NAME INDEX doesn’t work!?!? why???
\cleardoublepage\phantomsection%this allows hyperlink in ToC to work
\addcontentsline{toc}{chapter}{Name index}
\ohead{Name index}
\printindex[aut]

\cleardoublepage\phantomsection%this allows hyperlink in ToC to work
\addcontentsline{toc}{chapter}{Language index}
\ohead{Language index}
\printindex[lan]

\cleardoublepage\phantomsection%this allows hyperlink in ToC to work
\addcontentsline{toc}{chapter}{Subject index}
\ohead{Subject index}
\printindex


\end{document}
%\end{document}