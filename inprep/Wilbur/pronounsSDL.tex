%a\documentclass[ number=5
			   ,series=sidl
			   ,isbn=xxx-x-xxxxxx-xx-x
			   ,url=http://langsci-press.org/catalog/book/17
			   ,output=long   % long|short|inprep              
			   %,blackandwhite
			   %,smallfont
			   ,draftmode   
			  ]{LSP/langsci}                          

\usepackage{LSP/lsp-styles/lsp-gb4e}		% verhindert Komma bei mehrfachen Fußnoten?
                                                      
\usepackage{layout}
\usepackage{lipsum}

%%%% ABOVE FOR LangSciPress %%%%
%%%% ABOVE FOR LangSciPress %%%%
%%%% ABOVE FOR LangSciPress %%%%
\usepackage{libertine}%work-around solution for rendering problematic characters ʦ, ͡  (mostly in \textbf{})

\usepackage{longtable}%Double-lines (\hline\hline) aren’t typeset properly in ‘longtable’-environment across several pages! conflict with other package (maybe xcolor with option ‘tables’?)

\usepackage{multirow}

\usepackage{array} %allows, among other things, centering column content in a table while also specifying width, creates new column style "x" for center-alignment, "y" for right-alignment
\newcolumntype{x}[1]{>{\centering\hspace{0pt}}p{#1}}%
\newcolumntype{y}[1]{>{\raggedleft\hspace{0pt}}p{#1}}%

\usepackage[]{placeins}%using \FloatBarrier command, all floats still floating at that point will be typeset, and cannot cross that boundary. the option here \usepackage[section]{placeins} automatically adds \FloatBarrier to every \section command (only works for \section commands, nothing lower than that!)
%\usepackage{afterpage}%by using the command \afterpage{\clearpage}, all floats will appear, but no new page will be started, thus avoiding bad page breaks around floats

\usepackage{vowel} %for vowel space chart


%%%IS THIS NECESSARY??
%%%%following allows you to refer to footnotes (from http://anthony.liekens.net/index.php/LaTeX/MultipleFootnoteReferences)
%\newcommand{\footnoteremember}[2]{
%  \footnote{#2}
%  \newcounter{#1}
%  \setcounter{#1}{\value{footnote}}
%} \newcommand{\footnoterecall}[1]{
%  \footnotemark[\value{#1}]} 
%%%%previous allows you to refer to footnotes: use \footnoteremember{referenceText} in footnote, then \footnoterecall{referenceText} to refer.

\usepackage{tikz}%
\usetikzlibrary{plothandlers,matrix,decorations.text,shapes.arrows,shadows,chains,positioning,scopes}

\usepackage{synttree} %zeichnet linguistische Bäume
\branchheight{36pt}%sets height between rows in synttree

\usepackage{lscape}%used for landscape pages in index (list of recordings)

\usepackage{polyglossia}
\setmainlanguage{english}


%%%TAKE OUT FOR FINAL VERSION:
%%%TAKE OUT FOR FINAL VERSION:
%%%TAKE OUT FOR FINAL VERSION:

%%%%following readjusts margin text!
%\setlength{\marginparwidth}{20mm}
%\let\oldmarginpar\marginpar
%\renewcommand\marginpar[1]{\-\oldmarginpar[\raggedleft\footnotesize\vspace{-7pt}\color{red}\It{→ #1}]%
%{\raggedright\footnotesize\vspace{-7pt}\color{red}\It{→ #1}}}
%%%%previous readjusts margin text!

%%%The following lines set depth of ToC (LSP default is only 3 levels)!
%%%\renewcommand{\contentsname}{Table of Contents} % überschrift des inhaltsverzeichnisses
%\setcounter{secnumdepth}{5}%sets how deep section/subsection/subsubsections are numbered
%\setcounter{tocdepth}{5}%sets the depth of the ToC %but this doesn't seem to work!!!
%% new commands for LSP book (Grammar of Pite Saami, by J. Wilbur)

\newcommand{\PS}{Pite Saami}
\newcommand{\PSDP}{Pite Saami Documentation Project}
\newcommand{\WLP}{Wordlist Project}

\newcommand{\HANG}{\everypar{\hangindent15pt \hangafter1}}%also useful for table cells
\newcommand{\FB}{\FloatBarrier}%shortcut for this command to print all floats w/o pagebreak

\newcommand{\REF}[1]{(\ref{#1})}%adds parenthesis around the reference number, particularly useful for examples.%\Ref had clash with LSP!
\newcommand{\dline}{\hline\hline}%makes a double line in a table
\newcommand{\superS}[1]{\textsuperscript{#1}}%adds superscript element
\newcommand{\sub}[1]{$_{#1}$}%adds subscript element
\newcommand{\Sc}[1]{\textsc{#1}}%shortcut for small capitals (not to be confused with \sc, which changes the font from that point on)
\newcommand{\It}[1]{\textit{#1}}%shortcut for italics (not to be confused with \it, which changes the font from that point on)
\newcommand{\Bf}[1]{\textbf{#1}}%shortcut for bold (not to be confused with \bf, which changes the font from that point on)
\newcommand{\BfIt}[1]{\textbf{\textit{#1}}}
\newcommand{\BfSc}[1]{\textbf{\textsc{#1}}}
\newcommand{\Tn}[1]{\textnormal{#1}}%shortcut for normal text (undo italics, bolt, etc.)
\newcommand{\MC}{\multicolumn}%shortcut for multicolumn command in tabular environment - only replaces command, not variables!
\newcommand{\MR}{\multirow}%shortcut for multicolumn command in tabular environment - only replaces command, not variables!
\newcommand{\TILDE}{∼}%U+223C %OLD:~}%shortcut for tilde%command ‘\Tilde’ clashes with LSP!%
\newcommand{\BS}{\textbackslash}%backslash
\newcommand{\Red}[1]{{\color{red}{#1}}}%for red text
\newcommand{\Blue}[1]{{\color{blue}{#1}}}%for blue text
\newcommand{\PLUS}{+}%nicer looking plus symbol
\newcommand{\MINUS}{-}%nicer looking plus symbol
%    Was die Pfeile betrifft, kannst Du mal \Rightarrow \mapsto \textrightarrow probieren und dann \mathbf \boldsymbol oder \pbm dazutun.
\newcommand{\ARROW}{\textrightarrow}%→%dieser dicke Pfeil ➜ wird nicht von der LSP-Font unterstützt: %\newcommand{\ARROW}{{\fontspec{DejaVu Sans}➜}}
\newcommand{\DARROW}{\textleftrightarrow}%↔︎%DoubleARROW
\newcommand{\BULLET}{•}%
%%✓ does not exist in the default LSP font!
\newcommand{\CH}{\checkmark}%%\newcommand{\CH}{\fontspec{Arial Unicode MS}✓}%CH as in CHeck
%%following used to separate alternation forms for consonant gradation and umlaut patterns:
\newcommand{\Div}{‑}%↔︎⬌⟷⬄⟺⇔%non-breaking hyphen: ‑  
\newcommand{\QUES}{\textsuperscript{?}}%marks questionable/uncertain forms

\newcommand{\jvh}{\mbox{\It{j}-suffix} vowel harmony}%
%\newcommand{\Ptcl}{\Sc{ptcl} }%just shortcut for glossing ‘particle’
%\newcommand{\ATTR}{{\Sc{attributive}}}%shortcut for ATTRIBUTIVE in small caps
%\newcommand{\PRED}{{\Sc{predicative}}}%shortcut for PREDICATIVE in small caps
%\newcommand{\COMP}{{\Sc{comparative}}}%shortcut for COMPARATIVE in small caps
%\newcommand{\SUPERL}{{\Sc{superlative}}}%shortcut for SUPERLATIVE in small caps
\newcommand{\SG}{{\Sc{singular}}}%shortcut for SINGULAR in small caps
\newcommand{\DU}{{\Sc{dual}}}%shortcut for DUAL in small caps
\newcommand{\PL}{{\Sc{plural}}}%shortcut for PLURAL in small caps
%\newcommand{\NOM}{{\Sc{nominative}}}%shortcut for NOMINATIVE in small caps
%\newcommand{\ACC}{{\Sc{accusative}}}%shortcut for ACCUSATIVE in small caps
%\newcommand{\GEN}{{\Sc{genitive}}}%shortcut for GENITIVE in small caps
%\newcommand{\ILL}{{\Sc{illative}}}%shortcut for ILLATIVE in small caps
%\newcommand{\INESS}{{\Sc{inessive}}}%shortcut for INESSIVE in small caps
\newcommand{\ELAT}{{\Sc{elative}}}%shortcut for ELATIVE in small caps
%\newcommand{\COM}{{\Sc{comitative}}}%shortcut for COMITATIVE in small caps
%\newcommand{\ABESS}{{\Sc{abessive}}}%shortcut for ABESSIVE in small caps
%\newcommand{\ESS}{{\Sc{essive}}}%shortcut for ESSIVE in small caps
%\newcommand{\DIM}{{\Sc{diminutive}}}%shortcut for DIMINUTIVE in small caps
%\newcommand{\ORD}{{\Sc{ordinal}}}%shortcut for ORDINAL in small caps
%\newcommand{\CARD}{{\Sc{cardinal}}}%shortcut for CARDINAL in small caps
%\newcommand{\PROX}{{\Sc{proximal}}}%shortcut for PROXIMAL in small caps
%\newcommand{\DIST}{{\Sc{distal}}}%shortcut for DISTAL in small caps
%\newcommand{\RMT}{{\Sc{remote}}}%shortcut for REMOTE in small caps
%\newcommand{\REFL}{{\Sc{reflexive}}}%shortcut for REFLEXIVE in small caps
%\newcommand{\PRS}{{\Sc{present}}}%shortcut for PRESENT in small caps
%\newcommand{\PST}{{\Sc{past}}}%shortcut for PAST in small caps
%\newcommand{\IMP}{{\Sc{imperative}}}%shortcut for IMPERATIVE in small caps
%\newcommand{\POT}{{\Sc{potential}}}%shortcut for POTENTIAL in small caps
\newcommand{\PROG}{{\Sc{progressive}}}%shortcut for PROGRESSIVE in small caps
\newcommand{\PRF}{{\Sc{perfect}}}%shortcut for PERFECT in small caps
\newcommand{\INF}{{\Sc{infinitive}}}%shortcut for INFINITIVE in small caps
%\newcommand{\NEG}{{\Sc{negative}}}%shortcut for NEGATIVE in small caps
\newcommand{\CONNEG}{{\Sc{connegative}}}%shortcut for CONNEGATIVE in small caps
\newcommand{\ATTRs}{{\Sc{attr}}}%shortcut for ATTR in small caps
\newcommand{\PREDs}{{\Sc{pred}}}%shortcut for PRED in small caps
%\newcommand{\COMPs}{{\Sc{comp}}}%shortcut for COMP in small caps
%\newcommand{\SUPERLs}{{\Sc{superl}}}%shortcut for SUPERL in small caps
\newcommand{\SGs}{{\Sc{sg}}}%shortcut for SG in small caps
\newcommand{\DUs}{{\Sc{du}}}%shortcut for DU in small caps
\newcommand{\PLs}{{\Sc{pl}}}%shortcut for PL in small caps
\newcommand{\NOMs}{{\Sc{nom}}}%shortcut for NOM in small caps
\newcommand{\ACCs}{{\Sc{acc}}}%shortcut for ACC in small caps
\newcommand{\GENs}{{\Sc{gen}}}%shortcut for GEN in small caps
\newcommand{\ILLs}{{\Sc{ill}}}%shortcut for ILL in small caps
\newcommand{\INESSs}{{\Sc{iness}}}%shortcut for INESS in small caps
\newcommand{\ELATs}{{\Sc{elat}}}%shortcut for ELAT in small caps
\newcommand{\COMs}{{\Sc{com}}}%shortcut for COM in small caps
\newcommand{\ABESSs}{{\Sc{abess}}}%shortcut for ABESS in small caps
\newcommand{\ESSs}{{\Sc{ess}}}%shortcut for ESS in small caps
%\newcommand{\DIMs}{{\Sc{dim}}}%shortcut for DIM in small caps
%\newcommand{\ORDs}{{\Sc{ord}}}%shortcut for ORD in small caps
%\newcommand{\CARDs}{{\Sc{card}}}%shortcut for CARD in small caps
\newcommand{\PROXs}{{\Sc{prox}}}%shortcut for PROX in small caps
\newcommand{\DISTs}{{\Sc{dist}}}%shortcut for DIST in small caps
\newcommand{\RMTs}{{\Sc{rmt}}}%shortcut for RMT in small caps
\newcommand{\REFLs}{{\Sc{refl}}}%shortcut for REFL in small caps
\newcommand{\PRSs}{{\Sc{prs}}}%shortcut for PRS in small caps
\newcommand{\PSTs}{{\Sc{pst}}}%shortcut for PST in small caps
\newcommand{\IMPs}{{\Sc{imp}}}%shortcut for IMP in small caps
\newcommand{\POTs}{{\Sc{pot}}}%shortcut for POT in small caps
\newcommand{\PROGs}{{\Sc{prog}}}%shortcut for PROG in small caps
\newcommand{\PRFs}{{\Sc{prf}}}%shortcut for PRF in small caps
\newcommand{\INFs}{{\Sc{inf}}}%shortcut for INF in small caps
\newcommand{\NEGs}{{\Sc{neg}}}%shortcut for NEG in small caps
\newcommand{\CONNEGs}{{\Sc{conneg}}}%shortcut for CONNEG in small caps

\newcommand{\subNP}{{\footnotesize\sub{NP}}}%shortcut for NP (nominal phrase) in subscript
\newcommand{\subVC}{{\footnotesize\sub{VC}}}%shortcut for VC (verb complex) in subscript
\newcommand{\subAP}{{\footnotesize\sub{AP}}}%shortcut for NP (adjectival phrase) in subscript
\newcommand{\subAdvP}{{\footnotesize\sub{AdvP}}}%shortcut for AdvP (adverbial phrase) in subscript
\newcommand{\subPP}{{\footnotesize\sub{PP}}}%shortcut for NP (postpoistional phrase) in subscript

\newcommand{\ipa}[1]{{\fontspec{Linux Libertine}#1}}%specifying font for IPA characters

\newcommand{\SEC}{§}%standardize section symbol and spacing afterwards
%\newcommand{\SEC}{§\,}%

\newcommand{\Nth}{{\footnotesize(\It{n})}}%used in table of numerals in ADJ chapter

%%newcommands for tables in introductionSDL.tex:
\newcommand{\cliticExs}[3]{\Tn{\begin{tabular}{p{28mm} c p{28mm} p{35mm}}\It{#1}&\ARROW &\It{#2} & ‘#3’\\\end{tabular}}}%specifically for the two clitic examples
\newcommand{\Grapheme}[1]{\It{#1}}%formatting for graphemes in orthography tables
%%new command for the section on orthographic examples; syntax: #1=orthography, #2=phonology, #3=gloss
\newcommand{\SpellEx}[3]{\Tn{\begin{tabular}{p{70pt} p{70pt} l}\ipa{/#2/}&\It{#1}& ‘#3’ \\\end{tabular}}}%formatting for orthographic examples (intro-Chapter)


%%new transl tier in gb4e; syntax: #1=free translation (in single quotes), #2=additional comments, z.B. literal meaning:
\newcommand{\Transl}[2]{\trans\Tn{‘#1’ #2}}%new transl tier in gb4e;
\newcommand{\TranslMulti}[2]{\trans\hspace{12pt}\Tn{‘#1’ #2}}%new transl tier in gb4e for a dialog to be included under a single example number


%% used for examples in the Prosody and Segmental phonology chapters:
\newcommand{\PhonGloss}[7]{%PhonGloss = Phonology Gloss;
%pattern: \PhonGloss{label}{phonemic}{phonetic}{orthographic}{gloss}{recording}{utterance}
\ea\label{#1}
\Tn{\begin{tabular}[t]{p{30mm} l}
\ipa{/#2/}	& \It{#4} \\
\ipa{[#3]}	&\HANG ‘#5’\\%no table row can start with square brackets! thus the workaround with \MC
\end{tabular}\hfill\hyperlink{#6}{{\small\textnormal[pit#6#7]}}%\index{Z\Red{rec}!\Red{pit#6}}\index{Z\Red{utt}!\Red{pit#6#7} \Blue{Phon}}
}
\z}
\newcommand{\PhonGlossWL}[6]{%PhonGloss = Phonology Gloss for words from WORDLIST, not from corpus!;
%pattern: \PhonGloss{label}{phonemic}{phonetic}{orthographic}{gloss}{wordListNumber}
\ea\label{#1}
\Tn{\begin{tabular}[t]{p{30mm} l}
\ipa{/#2/}	& \It{#4} \\
\ipa{[#3]}	&\HANG ‘#5’\\%no table row can start with square brackets! thus the workaround with \MC
\end{tabular}\hfill\hyperlink{explExs}{{\small\textnormal[#6]}}%\index{Z\Red{wl}!\Red{#6}\Blue{Phon}}
}
\z}

%%for derivation examples in the derivational morphology chapter!
%syntax: \DerivExam{#1}{#2}{#3}{#4}{#5}{#6}
%#1: base, #2: base-gloss, #3: derived form, #4: derived form gloss, #5: derived form translation, #6: pit-recording, #7: utterance number
\newcommand{\DW}{28mm}%for following three commands, to align arrows throughout
%%%%OLD:
%%%\newcommand{\DerivExam}[7]{\Tn{\begin{tabular}[t]{p{\DW}cl}\It{#1}&\ARROW&\It{#3}\\#2&&#4\\\end{tabular}\hfill\pbox{.3\textwidth}{\hfill‘#5’\\\hbox{}\hfill\hyperlink{pit#6}{{\small\textnormal[pit#6.#7]}}}
%%%%\index{Z\Red{rec}!\Red{pit#6}}\index{Z\Red{utt}!\Red{pit#6.#7}}
%%%}}
%NEW:
\newcommand{\DerivExam}[7]{\Tn{
\begin{tabular}[t]{p{\DW}x{5mm}l}\It{#1}&\ARROW&\It{#3}\\\end{tabular}\hfill‘#5’\\
\hspace{1mm}\begin{tabular}[t]{p{\DW}x{5mm}l}#2&&#4\\\end{tabular}\hfill\hyperlink{pit#6}{{\small\textnormal[pit#6.#7]}}
%\index{Z\Red{rec}!\Red{pit#6}}\index{Z\Red{utt}!\Red{pit#6.#7}}
}}
%%same as above, but supress any reference to a specific utterance
\newcommand{\DerivExamX}[7]{\Tn{
\begin{tabular}[t]{p{\DW}x{5mm}l}\It{#1}&\ARROW&\It{#3}\\\end{tabular}\hfill‘#5’\\
\hspace{1mm}\begin{tabular}[t]{p{\DW}x{5mm}l}#2&&#4\\\end{tabular}\hfill\hyperlink{pit#6}{{\small\textnormal[pit#6]\It{e}}}
%\index{Z\Red{rec}!\Red{pit#6}}\index{Z\Red{utt}!\Red{pit#6.#7}}
}}
\newcommand{\DerivExamWL}[6]{\Tn{
\begin{tabular}[t]{p{\DW}x{5mm}l}\It{#1}&\ARROW&\It{#3}\\\end{tabular}\hfill‘#5’\\
\hspace{1mm}\begin{tabular}[t]{p{\DW}x{5mm}l}#2&&#4\\\end{tabular}\hfill\hyperlink{explExs}{{\small\textnormal[#6]}}
%\index{Z\Red{wl}!\Red{#6}}
}}


%formatting of corpus source information (after \transl in gb4e-environments):
\newcommand{\Corpus}[2]{\hspace*{1pt}\hfill{\small\mbox{\hyperlink{pit#1}{\Tn{[pit#1.#2]}}}}%\index{Z\Red{rec}!\Red{pit#1}}\index{Z\Red{utt}!\Red{pit#1.#2}}
}%
\newcommand{\CorpusE}[2]{\hspace*{1pt}\hfill{\small\mbox{\hyperlink{pit#1}{\Tn{[pit#1.#2]}}\It{e}}}%\index{Z\Red{rec}!\Red{pit#1}}\index{Z\Red{utt}!\Red{pit#1.#2}\Blue{-E}}
}%
%%as above, but necessary for recording names which include an underline because the first variable in \href understands _ but the second variable requires \_
\newcommand{\CorpusLink}[3]{\hspace*{1pt}\hfill{\small\mbox{\hyperlink{pit#1}{\Tn{[pit#2.#3]}}}}%\index{Z\Red{rec}!\Red{pit#2}}\index{Z\Red{utt}!\Red{pit#2.#3}}
}%
%%as above, but for newer recordings which begin with sje20 instead of pit
\newcommand{\CorpusSJE}[2]{\hspace*{1pt}\hfill{\small\mbox{\hyperlink{sje20#1}{\Tn{[sje20#1.#2]}}}}%\index{Z\Red{rec}!\Red{sje20#1}}\index{Z\Red{utt}!\Red{sje20#1.#2}}
}%
\newcommand{\CorpusSJEE}[2]{\hspace*{1pt}\hfill{\small\mbox{\hyperlink{sje20#1}{\Tn{[sje20#1.#2]}}\It{e}}}%\index{Z\Red{rec}!\Red{sje20#1}}\index{Z\Red{utt}!\Red{sje20#1.#2}\Blue{-E}}
}%











%%hyphenation points for line breaks
%%add to TeX file before \begin{document} with:
%%%%hyphenation points for line breaks
%%add to TeX file before \begin{document} with:
%%%%hyphenation points for line breaks
%%add to TeX file before \begin{document} with:
%%\include{hyphenationSDL}
\hyphenation{
ab-es-sive
affri-ca-te
affri-ca-tes
Ahka-javv-re
al-ve-o-lar
com-ple-ments
%check this:
de-cad-es
fri-ca-tive
fri-ca-tives
gemi-nate
gemi-nates
gra-pheme
gra-phemes
ho-mo-pho-nous
ho-mor-ga-nic
mor-pho-syn-tac-tic
or-tho-gra-phic
pho-neme
pho-ne-mes
phra-ses
post-po-si-tion
post-po-si-tion-al
pre-as-pi-ra-te
pre-as-pi-ra-ted
pre-as-pi-ra-tion
seg-ment
un-voiced
wor-king-ver-sion
}
\hyphenation{
ab-es-sive
affri-ca-te
affri-ca-tes
Ahka-javv-re
al-ve-o-lar
com-ple-ments
%check this:
de-cad-es
fri-ca-tive
fri-ca-tives
gemi-nate
gemi-nates
gra-pheme
gra-phemes
ho-mo-pho-nous
ho-mor-ga-nic
mor-pho-syn-tac-tic
or-tho-gra-phic
pho-neme
pho-ne-mes
phra-ses
post-po-si-tion
post-po-si-tion-al
pre-as-pi-ra-te
pre-as-pi-ra-ted
pre-as-pi-ra-tion
seg-ment
un-voiced
wor-king-ver-sion
}
\hyphenation{
ab-es-sive
affri-ca-te
affri-ca-tes
Ahka-javv-re
al-ve-o-lar
com-ple-ments
%check this:
de-cad-es
fri-ca-tive
fri-ca-tives
gemi-nate
gemi-nates
gra-pheme
gra-phemes
ho-mo-pho-nous
ho-mor-ga-nic
mor-pho-syn-tac-tic
or-tho-gra-phic
pho-neme
pho-ne-mes
phra-ses
post-po-si-tion
post-po-si-tion-al
pre-as-pi-ra-te
pre-as-pi-ra-ted
pre-as-pi-ra-tion
seg-ment
un-voiced
wor-king-ver-sion
}\begin{document}\tableofcontents\clearpage

%%%%%%%%%%%%%%%%%%%%%%%%%%%%%%%%% ALL THE ABOVE TO BE COMMENTED OUT FOR COMPLETE DOCUMENT! %%%%%%%%%%%

\chapter{Nominals II: Pronouns}\label{pronouns}\index{pronouns}\index{parts of speech!pronouns}
%\section{Overview} %overview - what kind of pronouns exist?
\PS\ has a closed class of pronouns consisting of personal, demonstrative, reflexive, interrogative and relative pronouns. \korr{070}Pronouns are nominals and are defined syntactically by their ability to represent a nominal phrase. 
As nominals, all pronouns inflect for case (cf. Section \ref{case} on the case system); concerning number, personal and reflexive pronouns inflect for singular, dual and plural, while demonstrative, interrogative and relative pronouns only inflect for singular and plural. 
The \PS\ pronouns are described below, in the order listed above; paradigms for each pronoun class are also included. The pronouns are written using the working \PS\ orthography. The corpus does not provide sufficient data about the status of any pronouns in the abessive and essive cases,\footnote{Neither \citet{Lagercrantz1926} nor \citet{Lehtiranta1992} provide sufficient data, either.} 
so this must be left for future study. 
Note that there are also a number of non-nominal interrogative pro-forms which do not inflect for case or number; although not nominals, these pro-forms are covered in Section \ref{interrogativeProForms}, after the interrogative pronouns. 


%%%%%%%%%%%%%%%%%%%%%%%%%%%%%%%%%%%%%%%
%%%%%%%%%%%%%%%%%%%%%%%%%%%%%%%%%%%%%%%
%%%%%%%%%%%%%%%%PERSONAL%%%%%%%%%%%%%%%%%
%%%%%%%%%%%%%%%%%%%%%%%%%%%%%%%%%%%%%%%
%%%%%%%%%%%%%%%%%%%%%%%%%%%%%%%%%%%%%%%
\section{Personal pronouns}\label{personalPronouns}\index{pronouns!personal pronouns}
%The singular, dual and plural forms of the personal pronouns for the various cases can be found in Tables \ref{pronSG} through \ref{pronPL}.\marginpar{These pronouns are mostly from \citet[163]{Lehtiranta1992}, still need to be compared with/verified by PSDP field work}. The orthography based on current \PS\ orthography (and still under construction!).
%Personal pronouns replace a noun phrase which has already occurred or at least been implied in the preceding discourse. %UM: delete definitions!
Personal pronouns inflect for person and number (singular, dual or plural) as well as for case. They are listed in Table \vref{PersPronTable}. Personal pronouns do not inflect for the biological gender of their referents, but are restricted in referring only to humans (demonstrative pronouns are used when the referent is not human).

\begin{table}\centering
\caption{Personal pronouns}\label{PersPronTable}
\begin{tabular}{| r || c | c | c || c |}\hline
%\MC{4}{c}{}	&\MC{1}{c}{\MR{3}{*}{\rotatebox{270}{\hspace{5pt}\It{num}}}}\\\hline
&\MC{3}{c||}{\It{person}}&\\
%%SINGULAR
\It{case}	&\Sc{1\superS{st}}	&\Sc{2\superS{nd}}	&\Sc{3\superS{rd}}	&\It{num}\\\dline
\Sc{nom}	&mån\TILDE månnå	&dån\TILDE dånnå	&sån\TILDE sånnå	&\MR{7}{*}{\rotatebox{270}{\Sc{singular}}} \\\cline{1-4}%\hline
\Sc{gen}	&muv			&duv				&suv				&\\\cline{1-4}%\hline
\Sc{acc}	&muv			&duv				&suv				&\\\cline{1-4}%\hline
\Sc{ill}	&munje			&dunje			&sunje			&\\\cline{1-4}%\hline
\Sc{iness}	&muvne			&duvne			&suvne			&\\\cline{1-4}%\hline
\Sc{elat}	&muvvste			&duvvste			&suvvste			&\\\cline{1-4}%\hline
\Sc{com}	&mujna			&dujna			&sujna			&\\\dline%\cline{1-4}%\hline
%\Sc{ess}	&\MC{3}{c|}{\it månnan}								&\\\dline
%%DUAL
\Sc{nom}	&måj\TILDE måjå	&dåj\TILDE dåjå		&såj\TILDE såjå		&\MR{7}{*}{\rotatebox{270}{\Sc{dual}}} \\\cline{1-4}%\hline
\Sc{gen}	&munuo			&dunuo			&sunuo			&\\\cline{1-4}%\hline
\Sc{acc}	&månov			&dånov			&sånov			&\\\cline{1-4}%\hline
\Sc{ill}	&munnuj			&dunnuj			&sunnuj			&\\\cline{1-4}%\hline
\Sc{iness}	&munuon			&dunuon			&sunuon			&\\\cline{1-4}%\hline
\Sc{elat}	&munuost			&dunuost			&sunuost			&\\\cline{1-4}%\hline
\Sc{com}	&munujn			&dunujn			&sunujn			&\\\dline%\cline{1-4}%\hline
%\Sc{ess}	&\MC{3}{c|}{\it munnon}								&\\\dline
%%PLURAL
\Sc{nom}	&mij\TILDE mija		&dij\TILDE dija		&sij\TILDE sija		&\MR{7}{*}{\rotatebox{270}{\Sc{plural}}} \\\cline{1-4}%\hline
\Sc{gen}	&mijá			&dijá				&sijá				&\\\cline{1-4}%\hline
\Sc{acc}	&mijáv			&dijáv			&sijáv			&\\\cline{1-4}%\hline
\Sc{ill}	&mijjaj			&dijjaj			&sijjaj			&\\\cline{1-4}%\hline
\Sc{iness}	&miján			&diján			&siján			&\\\cline{1-4}%\hline
\Sc{elat}	&mijást			&dijást			&sijást			&\\\cline{1-4}%\hline
\Sc{com}	&mijájn			&dijájn			&sijájn			&\\\hline%\cline{1-4}%\hline
%\Sc{ess}	&\MC{3}{c|}{?}									&\\\dline
\end{tabular}
\end{table}

The nominative forms all have two possible forms, e.g. \It{mån}\TILDE\It{månnå} ‘\Sc{1sg.nom}’. In general, the monosyllabic form is the default, while the bisyllabic form is typically used as a citation form and when the pronoun is emphasized.

The person marking morphemes in personal pronouns are completely systematic and are %: \It{m-} for \Sc{1\superS{st} person}, \It{d-} for \Sc{2\superS{nd} person} and \It{s-} for \Sc{3\superS{rd} person}
listed in Table \vref{PersPronPersonMorph}. % and Table \vref{PersPronCaseMorph}, respectively.%, and the number markers in \ref{PersPronNumMorph}. Only the 
%These are followed by a portmanteau morpheme indicating both case and number, as indicated in Table \vref{PersPronPortmMorph}.
\begin{table}\centering
\caption{Person morphemes in personal pronouns}\label{PersPronPersonMorph}
\begin{tabular}{|c | c|}\hline
\It{person}			&\It{morpheme} \\\dline
\Sc{1\superS{st}}	&\It{m-}\\\hline
\Sc{2\superS{nd}}	&\It{d-}\\\hline
\Sc{3\superS{rd}}	&\It{s-}\\\hline
\end{tabular}
\end{table}

Case and number marking is not quite as systematic, but certain segmental patterns are present which closely resemble the singular case/number suffixes for nouns, particularly those in inflectional class I (cf. Section \ref{NclassI}). \korr{023}
Specifically, note the lack of a final consonant for all nominative pronouns (cf. the lack of a \Sc{nom.sg} or \Sc{nom.pl} case suffix); genitive pronouns in dual and plural (cf. the lack of a \Sc{gen.sg} case suffix), the final \It{-v} in the accusative personal pronouns (cf. \It{-v} for nouns in \Sc{acc.sg}), a final or nearly final \It{j}-element for illative pronouns (cf. \It{-j} for nouns in \Sc{ill.sg}), a final or nearly final \It{n}-element for inessive pronouns (cf. \It{-n} for nouns in \Sc{iness.sg}), a final or nearly final \It{st}-element for elative pronouns (cf. \It{-st} for nouns in \Sc{elat.sg} and \Sc{elat.pl}) and a final or nearly final \It{jn}-element for comitative pronouns (cf. \It{-jn(a)} for nouns in \Sc{com.sg}).



%%%%%%%%%%%%%%%%%%%%%%%%%%%%%%%%%%%%%%%
%%%%%%%%%%%%%%%%%%%%%%%%%%%%%%%%%%%%%%%
%%%%%%%%%%%%%%%DEMONSTRATIVE%%%%%%%%%%%%%%
%%%%%%%%%%%%%%%%%%%%%%%%%%%%%%%%%%%%%%%
%%%%%%%%%%%%%%%%%%%%%%%%%%%%%%%%%%%%%%%
\section{Demonstrative pronouns}\label{demonstrativePronouns}\index{demonstrative pronounss}\index{pronouns!demonstratives}%JW: demonstrative modifiers (this house) same as demonstrative pronouns (this)
Demonstrative pronouns are based on the stem \It{d}. They inflect for case and number (singular and plural, but not dual), as well as the proximity of the entity they refer to. The data from the corpus indicates that there is a three-way distinction between referents close to the speaker (proximal), those away from the speaker (distal), and those particularly far away (remote). %; these are referred to as \PROX, \DIST\ and \RMT, respectively. The \PROX\ form has \It{á} as the initial vowel, the \DIST\ form has \It{a}, and the \RMT\ form has \It{u}. 
They are listed in Table \vref{DemProTable}. Note that due to a lack of sufficient data on the remote forms in the corpus \korr{071}(either no forms exist or speakers were too uncertain to warrant inclusion here), this part of the paradigm is not complete; forms based on tentative data are marked by a question mark. 
\begin{table}[ht]\centering
\caption{Demonstrative pronouns}\label{DemProTable}
\begin{tabular}{| r || c | c | c | c| c | c |}\hline
		&\MC{6}{c|}{\It{number}}	\\
		&\MC{3}{c|}{\SG}	&\MC{3}{c|}{\PL}	\\
\It{case}	&\PROXs	&\DISTs	&\RMTs	&\PROXs	&\DISTs	&\RMTs	\\\dline
\NOMs	&dát		&dat		&dut		&dá(h)	&da(h)	&du(h)	\\\hline
\GENs	&dán		&dan		&dun		&dáj		&daj		&duj	\\\hline
\ACCs	&dáv		&dav		&duv		&dájt		&dajt		&dujt	\\\hline
\ILLs		&dása	&dasa	&dun?	&dájda	&dajda	&?	\\\hline
\INESSs	&dán		&dan		&dun		&dájtne	&dajtne	&duj?	\\\hline
\ELATs	&dásste	&dasste	&duj?	&dájste	&dajste	&duj?	\\\hline
\COMs	&dájna	&dajna	&dujn	&dáj		&daj		&duj	\\\hline
%%\ABESSs	&
%%\ESSs		&
\end{tabular}
\end{table}
%JW: not form ‘dusste’ ELAT.SG.DIST - probably far dist?!?, but dasste found e.g. in pit100404.345

Morphologically, demonstrative pronouns consist of the initial consonant \It{d-}, followed by \It{-á-}, \It{-a-} or \It{-u-} for proximal, distal and remote, respectively. This %first vowel 
is then followed by a case/number suffix, as summarized in Table \vref{DemProSuffixTable}. %\footnote{In many respects, the case/number suffixes are quite similar to the case/number suffixes in full nouns, but are only homophonic in \Sc{nom.pl}, \Sc{gen.pl}, \Sc{acc.sg/pl}, \Sc{ill.pl}, \Sc{iness.sg} and \Sc{com.sg/pl}.}
\begin{table}[ht]\centering
\caption{Demonstrative pronoun case/number suffixes}\label{DemProSuffixTable}
\begin{tabular}{| r || c | c |}\hline
		&\MC{2}{c|}{\It{number}}\\
\It{case}	&\SG	&\PL	\\\dline
%\it case	&\PROX	&\DIST	&\PROX	&\DIST	\\\dline
\NOMs	&-t		&(-h)		\\\hline
\GENs	&-n		&-j		\\\hline
\ACCs	&-v		&-jt		\\\hline
\ILLs		&-sa		&-jda		\\\hline%%in final copy of dissertation, ILL.PL listed as -j <-- MISTAKE!!
\INESSs	&-n		&-jtne	\\\hline
\ELATs	&-sste	&-jste	\\\hline
\COMs	&-jna		&-j		\\\hline
\end{tabular}
\end{table}

Both distal and remote demonstrative pronouns have a referent which is away from the speaker, but remote demonstrative pronouns indicate a greater distance than distal demonstrative pronouns. The referent of a remote demonstrative pronoun is clearly not located near the addressee. However, distal demonstrative pronouns do not necessarily denote a referent which is near the addressee, either, although this is certainly possible. Distal demonstrative pronouns are the most common in the corpus, and a sort of unmarked default demonstrative pronoun. A more precise description of when the various demonstrative forms are used must be left to future syntactic study. 
\korr{008}Note that demonstratives are identical in form, but differ syntactically as they modify the head of an NP; they are discussed in Section \ref{demonstratives}. 
%Demonstratives can function as pronouns or as determiners within a noun phrase. These functions are described in Sections \ref{demonstrativePronouns} and \ref{demonstrativeDeterminers}, respectively. 


%\subsection{Demonstratives as pronouns}\label{demonstrativePronouns}%not longer valid division here since demonstratives (modifiers of NP) are not in adjectives!
Demonstrative pronouns typically have non-human referents, as in \REF{demProWithNonHumanReferent1}. 
%\ea\label{demProWithNonHumanReferent1}%JW: +hum referent!!!
%\glll	maŋge jage del tjuvvuv mån dajt\\
%	maŋge jage del tjuvvu-v mån da-jt\\
%	many year\BS\Sc{nom.pl} then follow-\Sc{1sg.pst} \Sc{1sg.nom} \Sc{dem.dist-acc.pl}\\
%\Transl{for many years I followed them’	\Corpus{080924.306}
%\z
\ea\label{demProWithNonHumanReferent1}%JW: +hum referent!!!
\glll	muhtin sa del vuoptin dajt\\
	muhtin sa del vuopti-n d-a-jt\\
	sometimes so then sell-\Sc{1du.pst} \Sc{dem}-\Sc{dist}-\Sc{acc.pl}\\\nopagebreak
\Transl{so sometimes we sold those}{}	\Corpus{080924}{300}
\z
However, they can also be used to refer to third-person human referents, as in \REF{demProWithHumanReferent1}. 
\ea\label{demProWithHumanReferent1}
\glll	da lä jabmam, ber muv äddne'l viessomin dále\\
	d-a lä jabma-m ber mu-v äddne=l viesso-min dále\\
	\Sc{dem}-\Sc{dist}\BS\Sc{nom.pl} be\BS\Sc{3pl.prs} die-\Sc{prf} only \Sc{1sg.gen} mother\BS\Sc{nom.sg}=be\BS\Sc{3sg.prs} live-\Sc{prog} now\\\nopagebreak
\Transl{they have died, only my mother is living today}{}	\Corpus{100310b}{145}
%\glll	ja dat lä sjuksköterska, duv áhkka\\
%	ja dat lä sjuksköterska\footnotemark{} du-v áhkka\\
%	and \Sc{dem.prox\BS nom.sg} be\BS\Sc{3sg.prs} nurse \Sc{2sg-gen} grandmother\BS\Sc{nom.sg}\\
%\Transl{well, she is a nurse, your grandmother’	\Corpus{090702.424,426}
%\footnotetext{In \REF{demProWithHumanReferent1}, \It{sjuksköterska} ‘nurse’ is a borrowing from Swedish.}
\z

Distal demonstrative pronouns can also be used for anaphoric text deixis. For instance, \It{dat} in example \REF{demProTextDeixis} refers to the fact that the speaker has just dropped her ski pole. 
\ea\label{demProTextDeixis}%JW: +hum referent!!!
\glll	oj! ij dat aktagav dága\\
	oj ij d-a-t aktaga-v dága\\
	oh \Sc{neg}\BS\Sc{3sg.prs} \Sc{dem}-\Sc{dist}-\Sc{nom.sg} none-\Sc{acc.sg} make\BS\Sc{conneg}\\\nopagebreak
\Transl{oh! that’s no problem}{(lit.: that makes nothing)}	\Corpus{100404}{156}
\z

%Demonstratives can also refer to entire clauses, as in \REF{}.

\FloatBarrier


%%%%%%%%%%%%%%%%%%%%%%%%%%%%%%%%%%%%%%%
%%%%%%%%%%%%%%%%%%%%%%%%%%%%%%%%%%%%%%%
%%%%%%%%%%%%%%%%REFLEXIVE%%%%%%%%%%%%%%%%%
%%%%%%%%%%%%%%%%%%%%%%%%%%%%%%%%%%%%%%%
%%%%%%%%%%%%%%%%%%%%%%%%%%%%%%%%%%%%%%%

\section{Reflexive pronouns}\label{reflexivePronouns}\index{pronouns!reflexive pronouns}
The reflexive pronouns in Pite Saami are based on the stem \It{etj-} and inflect for the number (singular, dual and plural) and person of the noun they are co-referential with. Reflexive pronouns also inflect for case. These are listed in Table \vref{ReflPronTable}. 
The stem \It{etj-} can be translated as ‘self’, which could imply that it is a noun, but it is different from nouns for several reasons: 1) it is monosyllabic, 2) it has its own case and number marking suffixes, and 3) it inflects for dual number. For these reasons, it is glossed as ‘\REFLs’ instead of ‘self’. %/i͡eʧ/ "self" 
Note that reflexive pronouns are not common in the spontaneous language recordings in the corpus, but are mostly found in elicitation sessions. Even in elicitation sessions, my main consultant was not completely sure about some of the forms for less common cases (i.e., everything except nominative, accusative\footnote{However, note the form \It{etjav} ‘\Sc{refl-1sg.acc}’ was provided by a different speaker (A) than the speaker (B) who provided the forms in the rest of the paradigm, and speaker A was very uncertain of this form. \citet[162]{Lehtiranta1992} lists \It{etjam} and \It{etjamav}, and I suspect that the form \It{etjav} indicates that a simplification of the system has taken place (at least for speaker A) in which the root \It{etj-} is simply treated as a noun (such as ‘self’) which inflects using standard nominal case/number suffixes (here the \Sc{acc.sg} suffix \It{-v}), but ultimately a great deal more data is needed to verify this.} 
and genitive). Furthermore, a number of the elicited forms deviate from the forms provided in the complete paradigm in \citet[162]{Lehtiranta1992}.\footnote{But note also that the paradigm in \citet[162]{Lehtiranta1992} indicates a lack of consensus in the reflexive pronouns across speakers, as well. Whether the forms found in the \PSDP\ corpus indicate a simplification of the system or simply another speaker’s ideolect is impossible to determine at this point.}

For these reasons, the forms in Table \vref{ReflPronTable} should be considered preliminary at this point, and potentially subject to modification as a result of more thorough future study. My consultants were particularly uncertain about the forms in \It{italic} script, while forms listed in parenthesis are not attested in the corpus, but taken from the paradigms in \citet[162]{Lehtiranta1992} and adapted to the current \PS\ orthography.

\begin{table}[ht]\centering%\resizebox{1\textwidth}{!} {
%\begin{tabular}{| r || c | c | c || c |}
\caption{Reflexive pronouns}\label{ReflPronTable}
\begin{tabular}{| r || x{65pt} | x{65pt} | x{65pt} || c |}\hline
%\MC{4}{c}{}	&\MC{1}{c}{\MR{1}{*}{\It{num}}}\\\hline
%\MC{4}{c}{}	&\MC{1}{c}{\MR{1}{*}{\rotatebox{270}{\hspace{0pt}\It{num}}}}\\\hline
&\MC{3}{c||}{\It{person}}&\\
\It{case}	&\Sc{1\superS{st}}	&\Sc{2\superS{nd}}	&\Sc{3\superS{rd}}	&\It{num}\\\dline
\Sc{nom}	&etj				&etj				&etj				&\MR{7}{*}{\rotatebox{270}{\Sc{singular}}} \\\cline{1-4}%\hline
\Sc{gen}	&etjan			&etjad			&etjas			&\\\cline{1-4}%\hline
\Sc{acc}	&\It{etjav}			&etjavt			&etjavs			&\\\cline{1-4}%\hline
\Sc{ill}	&etjanij			&etjasad			&etjasis			&\\\cline{1-4}%\hline
\Sc{iness}	&ehtjanen			&etjanat			&etjanis			&\\\cline{1-4}%\hline
\Sc{elat}	&ehtjanist			&etjastit			&etjastis			&\\\cline{1-4}%\hline
\Sc{com}	&etjajnen			&(etjajnat)			&(etjajnis)			&\\\dline%\cline{1-4}%\hline
%\Sc{ess}	&\MC{3}{c|}{\it }								&\\\dline
\Sc{nom}	&etja				&etja				&etja				&\MR{7}{*}{\rotatebox{270}{\Sc{dual}}} \\\cline{1-4}%\hline
\Sc{gen}	&etjanij			&etjade			&etjajsga			&\\\cline{1-4}%\hline
\Sc{acc}	&(etjamenen)		&etjajd			&etjajdisa			&\\\cline{1-4}%\hline
\Sc{ill}	&ehtjasimen		&ehtjasiden		&ehtjasijga		&\\\cline{1-4}%\hline
\Sc{iness}	&(etjanenen)		&\It{etjajdin}		&(etjaneská)		&\\\cline{1-4}%\hline
\Sc{elat}	&etjanis			&etjastit			&etjastis			&\\\cline{1-4}%\hline
\Sc{com}	&(etjajnenen)		&(etjajneten)		&(etjajneská)		&\\\dline%\cline{1-4}%\hline
%\Sc{ess}	&\MC{3}{c|}{\it }								&\\\dline
\Sc{nom}	&etja				&etja				&etja				&\MR{7}{*}{\rotatebox{270}{\Sc{plural}}} \\\cline{1-4}%\hline
\Sc{gen}	&etjajme			&etjajde			&etjajse			&\\\cline{1-4}%\hline
\Sc{acc}	&(ehtjameh)		&etjajd			&etjajdisa			&\\\cline{1-4}%\hline
\Sc{ill}	&etjasijme			&etjasida			&etjasise			&\\\cline{1-4}%\hline
\Sc{iness}	&\It{ehtjanen}		&\It{etjajdin}		&\It{etjajnisan}		&\\\cline{1-4}%\hline
\Sc{elat}	&\It{etjanist}		&\It{etjastist}		&\It{etjajsist}		&\\\cline{1-4}%\hline
\Sc{com}	&(etjajneneh)		&(etjajneteh)		&(etjajneseh)		&\\\hline%\cline{1-4}%\hline
%\Sc{ess}	&\MC{3}{c|}{}									&\\\dline
\end{tabular}
\end{table}
\pagebreak

One example of a reflexive pronoun is shown in \REF{reflPron1}.
\ea\label{reflPron1}%
\glll	mån ságastav etjan birra\\
	mån ságasta-v etja-n birra\\
	\Sc{1sg.nom} speak-\Sc{1sg.prs} \Sc{refl-1sg.gen} about\\\nopagebreak
\Transl{I talk about myself}{}	\CorpusE{110521b2}{010}%JW: check number!!
\z

A reflexive pronoun is frequently used to add emphasis to the noun phrase it is coreferential with \korr{011}(as an intensifier), as in \REF{reflPron3}.
\ea\label{reflPron3}%
\glll	mån lev etj sábme\\
	mån le-v etj sábme\\
	\Sc{1sg.nom} be-\Sc{1sg.prs} \Sc{refl\BS 1sg.nom} Saami\BS\Sc{nom.sg}\\\nopagebreak
\Transl{I myself am Saami}{}	\Corpus{080703}{023}
\z

The noun phrase that a reflexive pronoun is coreferential with does not have to be realized overtly, %\marginpar{\REF{reflPron1} NOT truly reflexive but emphatic, b/c in \NOM\ case! find different example or drop it!} 
as illustrated by the utterance in \REF{reflPron2} (here as an intensifier as well).
\ea\label{reflPron2}%
\glll	etj lä lerram\\
	etj lä lerra-m\\
	\Sc{refl\BS 2sg.nom} be\BS\Sc{2sg.prs} learn-\Sc{prf}\\\nopagebreak
\Transl{(you) yourself have learned}{}	\Corpus{080924}{407}
\z

%The (ACC?) reflexive pronoun can occur in the presence of a regular personal pronoun, as in \ref{refPron01} and \ref{refPron02}, or alone, but then as a nominalized adjective?? as in \ref{refPron03}.
%\begin{exe}
%\z
%\ea\label{refPron01}%Jåssjo ber lijku suv ietjav
%\glll	Jåssjo ber lijku suv ietjav\\
%	Jåssjo ber lijku(str) suv ietja-v\\
%%	n adv v pn pn\\
%	Josh only like(str).\Sc{3sg.prs} \Sc{3sg.acc} self-\Sc{acc.sg}\\
%\trans	'Josh only likes himself'
%\z
%\ea\label{refPron02}
%\glll	Mån vuojnav ietjav\\
%	mån vuäjdna(wk)-v ietja-v\\
%	\Sc{1sg.nom} see-\Sc{1sg.prs} self-\Sc{acc.sg}\\
%\trans	'I see myself'
%\z
%\ea\label{refPron03}
%\glll	Jåssjo ber vuäjdna ietjasav\\
%	Jåssjo ber vuäjdna(str) ietja-sa-v\\
%	Josh only see(str).\Sc{3sg.prs} self-\Sc{nnlz}-\Sc{acc.sg}\\
%\trans	'Josh only sees himself'
%\z

%\vfill



\FloatBarrier

%%%%%%%%%%%%%%%%%%%%%%%%%%%%%%%%%%%%%%%
%%%%%%%%%%%%%%%%%%%%%%%%%%%%%%%%%%%%%%%
%%%%%%%%%%%%%%INTERROGATIVE%%%%%%%%%%%%%%%%
%%%%%%%%%%%%%%%%%%%%%%%%%%%%%%%%%%%%%%%
%%%%%%%%%%%%%%%%%%%%%%%%%%%%%%%%%%%%%%%
\section{Interrogative pronouns}\label{interrogativePronouns}\index{interrogatives}\index{pronouns!interrogative pronouns}\index{pronouns!interrogative pro-forms}
\PS\ has several classes of interrogative pronouns as well as a set of interrogative pro-forms which do not refer to NPs. \korr{010}While the latter set of non-nominal pro-forms refer to other word classes, they are covered in this section nonetheless due to their syntactic status as pro-forms. %, most of which refer to NPs. 
The pronouns can be divided into those with human referents (Section \ref{QpronounHUM}), which use the stem \It{ge-}, and those with non-human referents (Section \ref{QpronounNoHUM}), which use the stem \It{m-}. 
Furthermore, there are two classes of interrogatives which enquire about the selection of a particular item (semantically equivalent to English ‘which’; described in Section \ref{QpronounDEM}): the first refers to a choice from a selection in general and uses the stem \It{mikkir-}, while the other refers to a choice of one or two items and uses the stem \It{gåb-}. 
Interrogative pro-forms not referring to NPs feature the stem \It{g-} (Section \ref{interrogativeProForms}). This classification is summarized in Figure \vref{interrogativeProFormSummary}, which also indicates the stem for each type.
\begin{figure}\centering
\resizebox{\columnwidth}{!} {
\begin{tabular}{c}
\synttree{4}[interrogative pro-forms[pronouns[{\parbox{60pt}{\centering\PLUS hum\\\It{ge-}}}][{\parbox{60pt}{\centering\MINUS hum\\\It{m-}}}][selection[{\parbox{60pt}{\centering general\\\It{mikkir-}}}][{\parbox{60pt}{\centering limited\\\It{gåb-}}}]]][{\parbox{130pt}{\centering non-nominal pro-forms\\\It{g-}}}]]%synttree syntax: {3} = maximum tree depth, .b = send to bottom, [node]
\end{tabular}   }
\caption{A taxonomy of interrogative pro-form types and their stems}\label{interrogativeProFormSummary}
\end{figure}
%\parbox{100pt}{a \\ b}%\parbox[position]{width}{text}%NB: width and text are required

\FloatBarrier


\subsection{Interrogative pronouns with human referents}\label{QpronounHUM}
Interrogative pronouns with human referents use the stem \It{ge-} and inflect for the number (singular or plural) of the intended referent and for case. These pronouns are listed in Table \vref{QpronounHUMtable}; examples are provided in \REF{QpronounHUMex1} and \REF{QpronounHUMex2}. 
\begin{table}[ht]\centering
\caption{Interrogative pronouns with human referents}\label{QpronounHUMtable}
\begin{tabular}{| c | c | c |}\hline
		&\MC{2}{c|}{\It{number}}\\
\It{case}	&\SGs	&\PLs	\\\dline
\NOMs	&ge		&ge		\\\hline
\GENs	&gen		&gej		\\\hline
\ACCs	&gev		&gejd	\\\hline
\ILLs		&gesa	&gejda	\\\hline
\INESSs	&genne	&gejdne	\\\hline
\ELATs	&gesste	&gejsste	\\\hline
\COMs	&gejna	&gej		\\\hline
%\ABESSs	&		&		\\\hline
%\ESSs	&\MC{2}{c|}{?}		\\\hline
\end{tabular}
\end{table}

\ea\label{QpronounHUMex1}
\glll	nå, gejna dä tjuovo?\\
	nå ge-jna dä tjuovo\\
	well who-\Sc{com.sg} then accompany\BS\Sc{2sg.pst}\\\nopagebreak
\Transl{well, who did you go with?}{}	\Corpus{080924}{071}
\z
\ea\label{QpronounHUMex2}
\glll	gen gabmaga lä dá?\\
	ge-n gabmag-a lä d-á\\
	who-\Sc{gen.sg} shoe-\Sc{nom.pl} be\BS\Sc{3pl.prs} \Sc{dem}-\Sc{prox}\BS\Sc{nom.pl}\\\nopagebreak
\Transl{whose shoes are these?}{}	\Corpus{100404}{326}
\z


\subsection{Interrogative pronouns with non-human referents}\label{QpronounNoHUM}
Interrogative pronouns with non-human referents use the stem \It{m-} and inflect for the number (singular or plural) of the intended referent and for case. These pronouns are listed in Table \vref{QpronounNoHUMtable}; an example is provided in \REF{QpronounNoHUMex}.
\begin{table}[ht]\centering
\caption{Interrogative pronouns with non-human referents}\label{QpronounNoHUMtable}
\begin{tabular}{| c | c | c |}\hline
		&\MC{2}{c|}{\It{number}}\\
\It{case}	&\SGs	&\PLs	\\\dline
\NOMs	&mij		&ma(h)	\\\hline
\GENs	&man	&mej		\\\hline
\ACCs	&mav	&mejd\TILDE majd	\\\hline%JW: mejd DS, majd ER (but DS has majd for relative pronoun?)
\ILLs		&masa	&mejda	\\\hline
\INESSs	&manne	&majdne	\\\hline
\ELATs	&masste	&majsste	\\\hline
\COMs	&majna	&mej		\\\hline
%\ABESSs	&		&		\\\hline
%\ESSs	&\MC{2}{c|}{?}		\\\hline
\end{tabular}
\end{table}

\ea\label{QpronounNoHUMex}
\glll	mav dån sida?\\
	ma-v dån sida\\
	what-\Sc{acc.sg} \Sc{2sg.nom} want\BS\Sc{2sg.prs}\\\nopagebreak
\Transl{what do you want?}{}	\Corpus{090519}{194}
\z

\subsection{Interrogative pronouns concerning a selection}\label{QpronounDEM}
The two selective interrogative pronouns are used to enquire about the selection or choice of an item. The stem \It{mikkir-} refers to a selection in general, while the stem \It{gåb-} limits the selection to one or two choices. These are described in the following two sections.

\subsubsection{General selection using \It{mikkir-}}\label{QpronounDEMGeneral}
Interrogative pronouns based on the stem \It{mikkir-} are used to enquire about a choice or selection in general. They inflect for the number (singular and plural) of their referent and for case. These forms are listed in Table \vref{QpronounDEMGeneralTable}. 
Note that the illative forms are not attested in the corpus. 
\begin{table}[ht]\centering
\caption{Interrogative pronouns with a demonstrative referent using the \It{mikkir-} stem}\label{QpronounDEMGeneralTable}% and referring to a general selection.}
\begin{tabular}{| c | c | c |}\hline
		&\MC{2}{c|}{\It{number}}\\
\It{case}	&\SGs	&\PLs	\\\dline
\NOMs	&mikkir	&mikkira	\\\hline
\GENs	&mikkira	&mikkirij	\\\hline
\ACCs	&mikkirav	&mikkirijd	\\\hline
\ILLs		&mikkirij	&mikkirijda\\\hline
\INESSs	&?		&?	\\\hline
%\INESSs	&\it mikkir?&\it mikkir?\\\hline
\ELATs	&mikkirist	&mikkirijst	\\\hline
\COMs	&mikkirijna&mikkirij	\\\hline
%\ABESSs	&		&		\\\hline
%\ESSs	&\MC{2}{c|}{?}		\\\hline
\end{tabular}
\end{table}
%Note that, in the corpus, the \ILL\ forms are only used as pro-demonstratives (modifying a noun), and so the pronoun forms with case/number marking are not attested (thus the \ILLs\ forms are in \It{italic} font in Table \vref{QpronounDEMGeneralTable}).

\FloatBarrier

Two examples are provided in \REF{QpronounDEMGeneralEx1} and \REF{QpronounDEMGeneralEx2}.
%The interrogative pro-forms can be used as pronouns, as in \REF{QpronounDEMGeneralEx1} and \REF{QpronounDEMGeneralEx2}.
\ea\label{QpronounDEMGeneralEx1}
\glll	mikkirist lä dat dágaduvvum?\\
	mikkir-ist lä d-a-t dága-duvv-um\\
	which-\Sc{elat.sg} be\BS\Sc{3sg.prs} \Sc{dem}-\Sc{dist}-\Sc{nom.sg} make-\Sc{pass-prf}\\\nopagebreak
\Transl{what is that made of?}{}	\CorpusE{110521b1}{203}
\z
\ea\label{QpronounDEMGeneralEx2}
\glll	nå, mikkira lidjin dan Ákabakten?\\
	nå mikkir-a lidji-n d-a-n Ákabakte-n\\
	well which-\Sc{nom.pl} be-\Sc{3pl.pst} \Sc{dem}-\Sc{dist}-\Sc{iness.sg} Ákkabakkte-\Sc{iness.sg}\\\nopagebreak
\Transl{well which (people) were in Ákkabakkte?}{}	\Corpus{080924}{032}
\z

They can also modify the head of an NP, and are then a ‘pro-adjective’\footnote{Cf. \citet[31-34]{SchachterShopen2007} for more on non-pronoun ‘pro-forms’.} enquiring after a further characterization of the referent. In this case, they do not inflect for number or case (as is true of all attributive adjectives), and so the form is always \It{mikkir}, as illustrated by the examples in \REF{QpronounDEMGeneralEx3} and \REF{QpronounDEMGeneralEx4}. %, although more data is needed to be certain. 
\ea\label{QpronounDEMGeneralEx3}
\glll	mikkir málle lij?\\
	mikkir málle li-j\\
	which blood\BS\Sc{nom.sg} be-\Sc{3sg.pst}\\\nopagebreak
\Transl{which (kind of) blood was it?}{}	\Corpus{080924}{256}
\z
\ea\label{QpronounDEMGeneralEx4}
\glll	mikkir gulijd åtjojde?\\
	mikkir guli-jd åtjo-jde\\
	which fish-\Sc{acc.pl} buy-\Sc{2pl.pst}\\\nopagebreak
\Transl{which (kinds of) fish did you buy?}{}	\Corpus{080924}{025}
\z

Another possible form of the stem seems to be \It{makkar-}, but this is only attested twice in the corpus and by one speaker, while \It{mikkir-} was consistently preferred in elicitation sessions. %, but \It{makkar-} also accepted. 
An example with \It{makkar} is provided in \REF{QpronounDEMGeneralEx5}; here, \It{makkar} is a pro-adjective modifying a noun in a subordinate interrogative clause. 
\ea\label{QpronounDEMGeneralEx5}
\glll	ja dä lä aj väha gähtjamin makkar sarvajd gilgin njuovat aj\\
	ja dä lä aj väha gähtja-min makkar sarva-jd gilgi-n njuova-t aj\\
	and then be\BS\Sc{3sg.prs} also a\_little? look-\Sc{prog} which reindeer\_bull-\Sc{acc.pl} will-\Sc{3pl.pst} slaughter-\Sc{inf} also\\\nopagebreak
\TranslLong{and then he is also looking a little which reindeer bulls they should also slaughter}{}	\Corpus{080909}{006}
\z

%Ultimately, there is not enough data in the corpus to be certain of the behavior of these pronouns and future study is needed.


\subsubsection{Limited selection using \It{gåb-}}\label{QpronounDEMLimited}
A further interrogative pronoun used to limit a selection to only one or two is based on the stem \It{gåb-}. It inflects for case and for number (singular and plural), as described below. Table \vref{QpronounDEMLimitedTable} lists the various forms.
\begin{table}[ht]\centering
\caption{Interrogative pronouns with a demonstrative referent using the \It{gåb-} stem}\label{QpronounDEMLimitedTable}% and referring to a selection limited to one or two.}
\begin{tabular}{| c | c | c |}\hline
		&\MC{2}{c|}{\It{number}}\\
\It{case}	&\SGs	&\PLs	\\\dline
\NOMs	&gåbba	&?	\\\hline
\GENs	&gåban	&gåbaj	\\\hline
\ACCs	&gåbav	&gåbajd	\\\hline
\ILLs		&gåbbaj	&gåbajda\\\hline
\INESSs	&gåban	&gåbajn	\\\hline
\ELATs	&gåbast	&gåbajst	\\\hline
\COMs	&gåbajn(a)&gåbaj	\\\hline
%\ABESSs	&		&		\\\hline
%\ESSs	&\MC{2}{c|}{?}		\\\hline
\end{tabular}
\end{table}

When marked for singular, it indicates a selection of one out of two possible choices, as in \REF{QpronounDEMLimitedEx1}. 
\ea\label{QpronounDEMLimitedEx1}
\glll	gåban sajen lä dån årrom?\\
	gåba-n saje-n lä dån årro-m\\
	which-\Sc{iness.sg} place-\Sc{iness.sg} be\BS\Sc{2sg.prs} \Sc{2sg.nom} be-\Sc{prf}\\\nopagebreak
\Transl{at which of the two places have you been?}{}	\CorpusE{110521b1}{161}
\z
When marked for \PL, it indicates a selection of two out of three or more choices, as in \REF{QpronounDEMLimitedEx2}.
\ea\label{QpronounDEMLimitedEx2}
\glll	gåbaj birra ságasta?\\
	gåba-j birra ságasta\\
	which-\Sc{gen.pl} about speak\BS\Sc{2sg.prs}\\\nopagebreak
\Transl{which two are you talking about?}{}	\CorpusE{110521b1}{037}
\z

This interrogative pronoun is only attested in elicitation sessions in the corpus. A more thorough description/understanding must be left to future research.


\subsection{Non-nominal interrogative pro-forms}\label{interrogativeProForms}
There are a number of non-nominal interrogative pro-forms. These enquire about information typically expressed by a clause-level adverbial, an adjunct or a complement clause. They are listed and glossed in Table \vref{otherQpronounTable}, and three examples are provided in \REF{otherQpronounEx1} through \REF{otherQpronounEx3}.
\begin{table}[ht]\centering
\caption{Non-nominal interrogative pro-forms}\label{otherQpronounTable}
\begin{tabular}{| c | l |}\hline
\It{pro-form}		&\It{gloss} \\\dline
gåsse			&when	\\\hline
gusa\TILDE guse	&to where	\\\hline
gånne			&where	\\\hline
guste			&from where	\\\hline
manen			&why	\\\hline
man \PLUS\ {\It{adj.}}	&how (big)	\\\hline
maktes\TILDE gukte	&how	\\\hline
galla				&how many \\\hline
%		&how	\\\hline
\end{tabular}
\end{table}
\ea\label{otherQpronounEx1}
\glll	gånne dajt tjogijdä?\\
	gånne d-a-jt tjogi-jdä\\
	where \Sc{dem}-\Sc{dist}-\Sc{acc.pl} pick-\Sc{2pl.pst}\\\nopagebreak
\Transl{where did you pick them?}{}	\Corpus{080924}{168}
\z
\ea\label{otherQpronounEx2}
\glll	gukte almatj hålla ‘reta’?\\
	gukte almatj hålla reta\\
	how person\BS\Sc{nom.sg} say\BS\Sc{3sg.prs} ‘reta’\\\nopagebreak
\Transl{how does one say ‘reta’?}{(\It{reta} is Swedish for ‘tease’)}	\Corpus{080924}{377}
\z
\ea\label{otherQpronounEx3}
\glll	man mälgat lij gu lij hiejman, iv mån diede\\
	man mälgat li-j gu li-j hiejm-an, i-v mån diede\\
	how far be-\Sc{3sg.pst} when be-\Sc{3sg.pst} home-\Sc{iness.sg} neg-\Sc{1sg.prs} \Sc{1sg.nom} know\BS\Sc{conneg}\\\nopagebreak
\Transl{how far it was, when one was home, I don’t know}{}	\Corpus{100404}{317}
\z

The list in Table \vref{otherQpronounTable} is likely not complete, as there are several other non-nominal interrogative pro-forms listed in the \PS\ wordlist which are not attested in the corpus. Furthermore, the data does not indicate what the difference is between the various alternate forms for ‘to where’ and ‘how’, if there is any at all. 




\FloatBarrier
%%%%%%%%%%%%%%%%%%%%%%%%%%%%%%%%%%%%%%%
%%%%%%%%%%%%%%%%%%%%%%%%%%%%%%%%%%%%%%%
%%%%%%%%%%%%%%%%RELATIVE%%%%%%%%%%%%%%%%%%
%%%%%%%%%%%%%%%%%%%%%%%%%%%%%%%%%%%%%%%
%%%%%%%%%%%%%%%%%%%%%%%%%%%%%%%%%%%%%%%
\section{Relative pronouns}\label{relativePronouns}\index{pronouns!relative pronouns}
Relative pronouns in \PS\ are identical in form to the interrogative pronouns with non-human referents (cf. Section \ref{QpronounNoHUM}). However, unlike interrogative pronouns, relative pronouns do not reflect the human-ness of their referents. They agree in number with their referent, and inflect for the case 
required by their syntactic function within the relative clause %\footnote{Cf. Section \ref{relativeClauses} on relative clauses.} 
The relative pronouns are listed in Table \vref{relativePronounsTable}. %\marginpar{check if mejd/majd for relative pronouns; include +/-HUM!}%JW: no proof for majd and +hum, but very likely; all other combinations check out. 
%\marginpar{Ok to refer to Section \ref{relativeClauses}, or better to have a couple examples here, too?}
See Section \ref{relativeClauses} for a number of examples with relative pronouns as well as a description of relative clauses. %, and some examples are provided in \REF{relativePronounsEx1} through \REF{relativePronounsEx3}.
\begin{table}[ht]\centering
\caption{Relative pronouns}\label{relativePronounsTable}
\begin{tabular}{| c | c | c |}\hline
		&\MC{2}{c|}{\It{number}}\\
\It{case}	&\SGs	&\PLs	\\\dline
\NOMs	&mij		&ma(h)	\\\hline
\GENs	&man	&mej		\\\hline
\ACCs	&mav	&mejd\TILDE majd	\\\hline%JW: mejd DS, majd ER (but DS has majd for relative pronoun?)
\ILLs		&masa	&mejda	\\\hline
\INESSs	&manne	&majdne	\\\hline
\ELATs	&masste	&majsste	\\\hline
\COMs	&majna	&mej		\\\hline
%\ABESSs	&		&		\\\hline
%\ESSs	&\MC{2}{c|}{?}		\\\hline
\end{tabular}
\end{table}
%\begin{exe}
%\z
%\ea\label{relativePronounsEx1}
%%\glll	gånne dajt tjogijdä?\\
%%	gånne da-jt tjogi-jdä\\
%%	where \Sc{dem.dist\BS acc.pl} pick-\Sc{2pl.pst}\\
%%\Transl{where did you pick them?’	\CorpusE{080924.168}
%\z
%\ea\label{relativePronounsEx2}
%%\glll	gokte almatj hålla ‘reta’?\\
%%	gokte almatj hålla reta\\
%%	how person\BS\Sc{nom.sg} say\BS\Sc{3sg.prs} ‘reta’\\
%%\Transl{how does one say ‘reta’?’ (\It{reta} is Swedish for ‘tease’)	\CorpusE{080924.377}
%\z
%\ea\label{relativePronounsEx3}
%%\glll	man mälgat lij gu lij hiejman, iv mån diede\\
%%	man mälgat li-j gu li-j hiejm-an, i-v mån diede\\
%%	how far be-\Sc{3sg.pst} when be-\Sc{3sg.pst} home-\Sc{iness.sg} neg-\Sc{1sg.prs} \Sc{1sg.nom} know\BS\Sc{conneg}\\
%%\Transl{how far it was, when one was home, I don’t know’	\CorpusE{100404.317}
%\z












%%%%%%% THIS IS NOT USED FOR THE ENTIRE COMPILATION, but only for individual chapters!!!!

\clearpage
\addcontentsline{toc}{chapter}{Bibliography}\label{Bibliography}
\bibliography{PiteGrammarBibSDL}%for bibtex
%\printbibliography%[title=Works Cited]%%for biber!






%%%NAME INDEX doesn’t work!?!? why???
\cleardoublepage\phantomsection%this allows hyperlink in ToC to work
\addcontentsline{toc}{chapter}{Name index}
\ohead{Name index}
\printindex[aut]

\cleardoublepage\phantomsection%this allows hyperlink in ToC to work
\addcontentsline{toc}{chapter}{Language index}
\ohead{Language index}
\printindex[lan]

\cleardoublepage\phantomsection%this allows hyperlink in ToC to work
\addcontentsline{toc}{chapter}{Subject index}
\ohead{Subject index}
\printindex


\end{document}