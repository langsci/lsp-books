\documentclass[ number=5
			   ,series=sidl
			   ,isbn=xxx-x-xxxxxx-xx-x
			   ,url=http://langsci-press.org/catalog/book/17
			   ,output=long   % long|short|inprep              
			   %,blackandwhite
			   %,smallfont
			   ,draftmode   
			  ]{LSP/langsci}                          

\usepackage{LSP/lsp-styles/lsp-gb4e}		% verhindert Komma bei mehrfachen Fußnoten?
\usepackage{LSP/lsp-styles/avm}
\avmfont{\sc} 
\avmvalfont{\it}
                                                      
\usepackage{layout}  
\usepackage{lipsum}   

%%%following now in main document (XeTeX_pitePhDSDL.tex)
%\title{A corpus-based grammar \\ of spoken Pite Saami}  
%%\subtitle{2000+ Years of Language Science and no End in Sight}  
%\author{Joshua Wilbur}
%\dedication{Gijtov adnet!}
%\renewcommand{\lsBackBody}{This grammar of Pite Saami (Uralic; Sweden) is simply bad-ass.}%for back cover text
%\renewcommand{\lsBackTitle}{Biddumsáme giella}%for back title

%%%% ABOVE FOR LangSciPress %%%%
%%%% ABOVE FOR LangSciPress %%%%
%%%% ABOVE FOR LangSciPress %%%%

\usepackage{longtable}

\usepackage{multirow}
\usepackage{array} %allows, among other things, centering column content in a table while also specifying width, creates new column style "x" for center-alignment, "y" for right-alignment
\newcolumntype{x}[1]{%
>{\centering\hspace{0pt}}p{#1}}%
\newcolumntype{y}[1]{%
>{\raggedleft\hspace{0pt}}p{#1}}%

\usepackage[]{placeins}%using \FloatBarrier command, all floats still floating at that point will be typeset, and cannot cross that boundary. the option here \usepackage[section]{placeins} automatically adds \FloatBarrier to every \section command (only works for \section commands, nothing lower than that!)
\usepackage{afterpage}%by using the command \afterpage{\clearpage}, all floats will appear, but no new page will be started, thus avoiding bad page breaks around floats

\usepackage{vowel} %for vowel space chart


%%IS THIS NECESSARY??
%%%following allows you to refer to footnotes (from http://anthony.liekens.net/index.php/LaTeX/MultipleFootnoteReferences)
\newcommand{\footnoteremember}[2]{
  \footnote{#2}
  \newcounter{#1}
  \setcounter{#1}{\value{footnote}}
} \newcommand{\footnoterecall}[1]{
  \footnotemark[\value{#1}]} 
%%%previous allows you to refer to footnotes: use \footnoteremember{referenceText} in footnote, then \footnoterecall{referenceText} to refer.

\usepackage{tikz}
\usetikzlibrary{plothandlers,matrix,decorations.text,shapes.arrows,shadows,chains,positioning,scopes}

\usepackage{synttree} %zeichnet linguistische Bäume
\branchheight{36pt}%sets height between rows in synttree

\usepackage{lscape}%used for landscape pages in index (list of recordings)

\usepackage{polyglossia}
\setmainlanguage{english}



%%%TAKE OUT FOR FINAL VERSION:
%%%TAKE OUT FOR FINAL VERSION:
%%%TAKE OUT FOR FINAL VERSION:

%%%following readjusts margin text!
\setlength{\marginparwidth}{20mm}
\let\oldmarginpar\marginpar
\renewcommand\marginpar[1]{\-\oldmarginpar[\raggedleft\footnotesize\vspace{-7pt}\color{red}\It{→ #1}]%
{\raggedright\footnotesize\vspace{-7pt}\color{red}\It{→ #1}}}
%%%previous readjusts margin text!

%%%The following three line should be removed for final SDL version!
%\renewcommand{\contentsname}{Table of Contents} % überschrift des inhaltsverzeichnisses
\setcounter{secnumdepth}{5}%sets how deep section/subsection/subsubsections are numbered
\setcounter{tocdepth}{5}%sets the depth of the ToC %but this doesn't seem to work!!!
