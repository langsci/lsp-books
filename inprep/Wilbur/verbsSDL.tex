%\documentclass[ number=5
			   ,series=sidl
			   ,isbn=xxx-x-xxxxxx-xx-x
			   ,url=http://langsci-press.org/catalog/book/17
			   ,output=long   % long|short|inprep              
			   %,blackandwhite
			   %,smallfont
			   ,draftmode   
			  ]{LSP/langsci}                          

\usepackage{LSP/lsp-styles/lsp-gb4e}		% verhindert Komma bei mehrfachen Fußnoten?
                                                      
\usepackage{layout}
\usepackage{lipsum}

%%%% ABOVE FOR LangSciPress %%%%
%%%% ABOVE FOR LangSciPress %%%%
%%%% ABOVE FOR LangSciPress %%%%
\usepackage{libertine}%work-around solution for rendering problematic characters ʦ, ͡  (mostly in \textbf{})

\usepackage{longtable}%Double-lines (\hline\hline) aren’t typeset properly in ‘longtable’-environment across several pages! conflict with other package (maybe xcolor with option ‘tables’?)

\usepackage{multirow}

\usepackage{array} %allows, among other things, centering column content in a table while also specifying width, creates new column style "x" for center-alignment, "y" for right-alignment
\newcolumntype{x}[1]{>{\centering\hspace{0pt}}p{#1}}%
\newcolumntype{y}[1]{>{\raggedleft\hspace{0pt}}p{#1}}%

\usepackage[]{placeins}%using \FloatBarrier command, all floats still floating at that point will be typeset, and cannot cross that boundary. the option here \usepackage[section]{placeins} automatically adds \FloatBarrier to every \section command (only works for \section commands, nothing lower than that!)
%\usepackage{afterpage}%by using the command \afterpage{\clearpage}, all floats will appear, but no new page will be started, thus avoiding bad page breaks around floats

\usepackage{vowel} %for vowel space chart


%%%IS THIS NECESSARY??
%%%%following allows you to refer to footnotes (from http://anthony.liekens.net/index.php/LaTeX/MultipleFootnoteReferences)
%\newcommand{\footnoteremember}[2]{
%  \footnote{#2}
%  \newcounter{#1}
%  \setcounter{#1}{\value{footnote}}
%} \newcommand{\footnoterecall}[1]{
%  \footnotemark[\value{#1}]} 
%%%%previous allows you to refer to footnotes: use \footnoteremember{referenceText} in footnote, then \footnoterecall{referenceText} to refer.

\usepackage{tikz}%
\usetikzlibrary{plothandlers,matrix,decorations.text,shapes.arrows,shadows,chains,positioning,scopes}

\usepackage{synttree} %zeichnet linguistische Bäume
\branchheight{36pt}%sets height between rows in synttree

\usepackage{lscape}%used for landscape pages in index (list of recordings)

\usepackage{polyglossia}
\setmainlanguage{english}


%%%TAKE OUT FOR FINAL VERSION:
%%%TAKE OUT FOR FINAL VERSION:
%%%TAKE OUT FOR FINAL VERSION:

%%%%following readjusts margin text!
%\setlength{\marginparwidth}{20mm}
%\let\oldmarginpar\marginpar
%\renewcommand\marginpar[1]{\-\oldmarginpar[\raggedleft\footnotesize\vspace{-7pt}\color{red}\It{→ #1}]%
%{\raggedright\footnotesize\vspace{-7pt}\color{red}\It{→ #1}}}
%%%%previous readjusts margin text!

%%%The following lines set depth of ToC (LSP default is only 3 levels)!
%%%\renewcommand{\contentsname}{Table of Contents} % überschrift des inhaltsverzeichnisses
%\setcounter{secnumdepth}{5}%sets how deep section/subsection/subsubsections are numbered
%\setcounter{tocdepth}{5}%sets the depth of the ToC %but this doesn't seem to work!!!
%% new commands for LSP book (Grammar of Pite Saami, by J. Wilbur)

\newcommand{\PS}{Pite Saami}
\newcommand{\PSDP}{Pite Saami Documentation Project}
\newcommand{\WLP}{Wordlist Project}

\newcommand{\HANG}{\everypar{\hangindent15pt \hangafter1}}%also useful for table cells
\newcommand{\FB}{\FloatBarrier}%shortcut for this command to print all floats w/o pagebreak

\newcommand{\REF}[1]{(\ref{#1})}%adds parenthesis around the reference number, particularly useful for examples.%\Ref had clash with LSP!
\newcommand{\dline}{\hline\hline}%makes a double line in a table
\newcommand{\superS}[1]{\textsuperscript{#1}}%adds superscript element
\newcommand{\sub}[1]{$_{#1}$}%adds subscript element
\newcommand{\Sc}[1]{\textsc{#1}}%shortcut for small capitals (not to be confused with \sc, which changes the font from that point on)
\newcommand{\It}[1]{\textit{#1}}%shortcut for italics (not to be confused with \it, which changes the font from that point on)
\newcommand{\Bf}[1]{\textbf{#1}}%shortcut for bold (not to be confused with \bf, which changes the font from that point on)
\newcommand{\BfIt}[1]{\textbf{\textit{#1}}}
\newcommand{\BfSc}[1]{\textbf{\textsc{#1}}}
\newcommand{\Tn}[1]{\textnormal{#1}}%shortcut for normal text (undo italics, bolt, etc.)
\newcommand{\MC}{\multicolumn}%shortcut for multicolumn command in tabular environment - only replaces command, not variables!
\newcommand{\MR}{\multirow}%shortcut for multicolumn command in tabular environment - only replaces command, not variables!
\newcommand{\TILDE}{∼}%U+223C %OLD:~}%shortcut for tilde%command ‘\Tilde’ clashes with LSP!%
\newcommand{\BS}{\textbackslash}%backslash
\newcommand{\Red}[1]{{\color{red}{#1}}}%for red text
\newcommand{\Blue}[1]{{\color{blue}{#1}}}%for blue text
\newcommand{\PLUS}{+}%nicer looking plus symbol
\newcommand{\MINUS}{-}%nicer looking plus symbol
%    Was die Pfeile betrifft, kannst Du mal \Rightarrow \mapsto \textrightarrow probieren und dann \mathbf \boldsymbol oder \pbm dazutun.
\newcommand{\ARROW}{\textrightarrow}%→%dieser dicke Pfeil ➜ wird nicht von der LSP-Font unterstützt: %\newcommand{\ARROW}{{\fontspec{DejaVu Sans}➜}}
\newcommand{\DARROW}{\textleftrightarrow}%↔︎%DoubleARROW
\newcommand{\BULLET}{•}%
%%✓ does not exist in the default LSP font!
\newcommand{\CH}{\checkmark}%%\newcommand{\CH}{\fontspec{Arial Unicode MS}✓}%CH as in CHeck
%%following used to separate alternation forms for consonant gradation and umlaut patterns:
\newcommand{\Div}{‑}%↔︎⬌⟷⬄⟺⇔%non-breaking hyphen: ‑  
\newcommand{\QUES}{\textsuperscript{?}}%marks questionable/uncertain forms

\newcommand{\jvh}{\mbox{\It{j}-suffix} vowel harmony}%
%\newcommand{\Ptcl}{\Sc{ptcl} }%just shortcut for glossing ‘particle’
%\newcommand{\ATTR}{{\Sc{attributive}}}%shortcut for ATTRIBUTIVE in small caps
%\newcommand{\PRED}{{\Sc{predicative}}}%shortcut for PREDICATIVE in small caps
%\newcommand{\COMP}{{\Sc{comparative}}}%shortcut for COMPARATIVE in small caps
%\newcommand{\SUPERL}{{\Sc{superlative}}}%shortcut for SUPERLATIVE in small caps
\newcommand{\SG}{{\Sc{singular}}}%shortcut for SINGULAR in small caps
\newcommand{\DU}{{\Sc{dual}}}%shortcut for DUAL in small caps
\newcommand{\PL}{{\Sc{plural}}}%shortcut for PLURAL in small caps
%\newcommand{\NOM}{{\Sc{nominative}}}%shortcut for NOMINATIVE in small caps
%\newcommand{\ACC}{{\Sc{accusative}}}%shortcut for ACCUSATIVE in small caps
%\newcommand{\GEN}{{\Sc{genitive}}}%shortcut for GENITIVE in small caps
%\newcommand{\ILL}{{\Sc{illative}}}%shortcut for ILLATIVE in small caps
%\newcommand{\INESS}{{\Sc{inessive}}}%shortcut for INESSIVE in small caps
\newcommand{\ELAT}{{\Sc{elative}}}%shortcut for ELATIVE in small caps
%\newcommand{\COM}{{\Sc{comitative}}}%shortcut for COMITATIVE in small caps
%\newcommand{\ABESS}{{\Sc{abessive}}}%shortcut for ABESSIVE in small caps
%\newcommand{\ESS}{{\Sc{essive}}}%shortcut for ESSIVE in small caps
%\newcommand{\DIM}{{\Sc{diminutive}}}%shortcut for DIMINUTIVE in small caps
%\newcommand{\ORD}{{\Sc{ordinal}}}%shortcut for ORDINAL in small caps
%\newcommand{\CARD}{{\Sc{cardinal}}}%shortcut for CARDINAL in small caps
%\newcommand{\PROX}{{\Sc{proximal}}}%shortcut for PROXIMAL in small caps
%\newcommand{\DIST}{{\Sc{distal}}}%shortcut for DISTAL in small caps
%\newcommand{\RMT}{{\Sc{remote}}}%shortcut for REMOTE in small caps
%\newcommand{\REFL}{{\Sc{reflexive}}}%shortcut for REFLEXIVE in small caps
%\newcommand{\PRS}{{\Sc{present}}}%shortcut for PRESENT in small caps
%\newcommand{\PST}{{\Sc{past}}}%shortcut for PAST in small caps
%\newcommand{\IMP}{{\Sc{imperative}}}%shortcut for IMPERATIVE in small caps
%\newcommand{\POT}{{\Sc{potential}}}%shortcut for POTENTIAL in small caps
\newcommand{\PROG}{{\Sc{progressive}}}%shortcut for PROGRESSIVE in small caps
\newcommand{\PRF}{{\Sc{perfect}}}%shortcut for PERFECT in small caps
\newcommand{\INF}{{\Sc{infinitive}}}%shortcut for INFINITIVE in small caps
%\newcommand{\NEG}{{\Sc{negative}}}%shortcut for NEGATIVE in small caps
\newcommand{\CONNEG}{{\Sc{connegative}}}%shortcut for CONNEGATIVE in small caps
\newcommand{\ATTRs}{{\Sc{attr}}}%shortcut for ATTR in small caps
\newcommand{\PREDs}{{\Sc{pred}}}%shortcut for PRED in small caps
%\newcommand{\COMPs}{{\Sc{comp}}}%shortcut for COMP in small caps
%\newcommand{\SUPERLs}{{\Sc{superl}}}%shortcut for SUPERL in small caps
\newcommand{\SGs}{{\Sc{sg}}}%shortcut for SG in small caps
\newcommand{\DUs}{{\Sc{du}}}%shortcut for DU in small caps
\newcommand{\PLs}{{\Sc{pl}}}%shortcut for PL in small caps
\newcommand{\NOMs}{{\Sc{nom}}}%shortcut for NOM in small caps
\newcommand{\ACCs}{{\Sc{acc}}}%shortcut for ACC in small caps
\newcommand{\GENs}{{\Sc{gen}}}%shortcut for GEN in small caps
\newcommand{\ILLs}{{\Sc{ill}}}%shortcut for ILL in small caps
\newcommand{\INESSs}{{\Sc{iness}}}%shortcut for INESS in small caps
\newcommand{\ELATs}{{\Sc{elat}}}%shortcut for ELAT in small caps
\newcommand{\COMs}{{\Sc{com}}}%shortcut for COM in small caps
\newcommand{\ABESSs}{{\Sc{abess}}}%shortcut for ABESS in small caps
\newcommand{\ESSs}{{\Sc{ess}}}%shortcut for ESS in small caps
%\newcommand{\DIMs}{{\Sc{dim}}}%shortcut for DIM in small caps
%\newcommand{\ORDs}{{\Sc{ord}}}%shortcut for ORD in small caps
%\newcommand{\CARDs}{{\Sc{card}}}%shortcut for CARD in small caps
\newcommand{\PROXs}{{\Sc{prox}}}%shortcut for PROX in small caps
\newcommand{\DISTs}{{\Sc{dist}}}%shortcut for DIST in small caps
\newcommand{\RMTs}{{\Sc{rmt}}}%shortcut for RMT in small caps
\newcommand{\REFLs}{{\Sc{refl}}}%shortcut for REFL in small caps
\newcommand{\PRSs}{{\Sc{prs}}}%shortcut for PRS in small caps
\newcommand{\PSTs}{{\Sc{pst}}}%shortcut for PST in small caps
\newcommand{\IMPs}{{\Sc{imp}}}%shortcut for IMP in small caps
\newcommand{\POTs}{{\Sc{pot}}}%shortcut for POT in small caps
\newcommand{\PROGs}{{\Sc{prog}}}%shortcut for PROG in small caps
\newcommand{\PRFs}{{\Sc{prf}}}%shortcut for PRF in small caps
\newcommand{\INFs}{{\Sc{inf}}}%shortcut for INF in small caps
\newcommand{\NEGs}{{\Sc{neg}}}%shortcut for NEG in small caps
\newcommand{\CONNEGs}{{\Sc{conneg}}}%shortcut for CONNEG in small caps

\newcommand{\subNP}{{\footnotesize\sub{NP}}}%shortcut for NP (nominal phrase) in subscript
\newcommand{\subVC}{{\footnotesize\sub{VC}}}%shortcut for VC (verb complex) in subscript
\newcommand{\subAP}{{\footnotesize\sub{AP}}}%shortcut for NP (adjectival phrase) in subscript
\newcommand{\subAdvP}{{\footnotesize\sub{AdvP}}}%shortcut for AdvP (adverbial phrase) in subscript
\newcommand{\subPP}{{\footnotesize\sub{PP}}}%shortcut for NP (postpoistional phrase) in subscript

\newcommand{\ipa}[1]{{\fontspec{Linux Libertine}#1}}%specifying font for IPA characters

\newcommand{\SEC}{§}%standardize section symbol and spacing afterwards
%\newcommand{\SEC}{§\,}%

\newcommand{\Nth}{{\footnotesize(\It{n})}}%used in table of numerals in ADJ chapter

%%newcommands for tables in introductionSDL.tex:
\newcommand{\cliticExs}[3]{\Tn{\begin{tabular}{p{28mm} c p{28mm} p{35mm}}\It{#1}&\ARROW &\It{#2} & ‘#3’\\\end{tabular}}}%specifically for the two clitic examples
\newcommand{\Grapheme}[1]{\It{#1}}%formatting for graphemes in orthography tables
%%new command for the section on orthographic examples; syntax: #1=orthography, #2=phonology, #3=gloss
\newcommand{\SpellEx}[3]{\Tn{\begin{tabular}{p{70pt} p{70pt} l}\ipa{/#2/}&\It{#1}& ‘#3’ \\\end{tabular}}}%formatting for orthographic examples (intro-Chapter)


%%new transl tier in gb4e; syntax: #1=free translation (in single quotes), #2=additional comments, z.B. literal meaning:
\newcommand{\Transl}[2]{\trans\Tn{‘#1’ #2}}%new transl tier in gb4e;
\newcommand{\TranslMulti}[2]{\trans\hspace{12pt}\Tn{‘#1’ #2}}%new transl tier in gb4e for a dialog to be included under a single example number


%% used for examples in the Prosody and Segmental phonology chapters:
\newcommand{\PhonGloss}[7]{%PhonGloss = Phonology Gloss;
%pattern: \PhonGloss{label}{phonemic}{phonetic}{orthographic}{gloss}{recording}{utterance}
\ea\label{#1}
\Tn{\begin{tabular}[t]{p{30mm} l}
\ipa{/#2/}	& \It{#4} \\
\ipa{[#3]}	&\HANG ‘#5’\\%no table row can start with square brackets! thus the workaround with \MC
\end{tabular}\hfill\hyperlink{#6}{{\small\textnormal[pit#6#7]}}%\index{Z\Red{rec}!\Red{pit#6}}\index{Z\Red{utt}!\Red{pit#6#7} \Blue{Phon}}
}
\z}
\newcommand{\PhonGlossWL}[6]{%PhonGloss = Phonology Gloss for words from WORDLIST, not from corpus!;
%pattern: \PhonGloss{label}{phonemic}{phonetic}{orthographic}{gloss}{wordListNumber}
\ea\label{#1}
\Tn{\begin{tabular}[t]{p{30mm} l}
\ipa{/#2/}	& \It{#4} \\
\ipa{[#3]}	&\HANG ‘#5’\\%no table row can start with square brackets! thus the workaround with \MC
\end{tabular}\hfill\hyperlink{explExs}{{\small\textnormal[#6]}}%\index{Z\Red{wl}!\Red{#6}\Blue{Phon}}
}
\z}

%%for derivation examples in the derivational morphology chapter!
%syntax: \DerivExam{#1}{#2}{#3}{#4}{#5}{#6}
%#1: base, #2: base-gloss, #3: derived form, #4: derived form gloss, #5: derived form translation, #6: pit-recording, #7: utterance number
\newcommand{\DW}{28mm}%for following three commands, to align arrows throughout
%%%%OLD:
%%%\newcommand{\DerivExam}[7]{\Tn{\begin{tabular}[t]{p{\DW}cl}\It{#1}&\ARROW&\It{#3}\\#2&&#4\\\end{tabular}\hfill\pbox{.3\textwidth}{\hfill‘#5’\\\hbox{}\hfill\hyperlink{pit#6}{{\small\textnormal[pit#6.#7]}}}
%%%%\index{Z\Red{rec}!\Red{pit#6}}\index{Z\Red{utt}!\Red{pit#6.#7}}
%%%}}
%NEW:
\newcommand{\DerivExam}[7]{\Tn{
\begin{tabular}[t]{p{\DW}x{5mm}l}\It{#1}&\ARROW&\It{#3}\\\end{tabular}\hfill‘#5’\\
\hspace{1mm}\begin{tabular}[t]{p{\DW}x{5mm}l}#2&&#4\\\end{tabular}\hfill\hyperlink{pit#6}{{\small\textnormal[pit#6.#7]}}
%\index{Z\Red{rec}!\Red{pit#6}}\index{Z\Red{utt}!\Red{pit#6.#7}}
}}
%%same as above, but supress any reference to a specific utterance
\newcommand{\DerivExamX}[7]{\Tn{
\begin{tabular}[t]{p{\DW}x{5mm}l}\It{#1}&\ARROW&\It{#3}\\\end{tabular}\hfill‘#5’\\
\hspace{1mm}\begin{tabular}[t]{p{\DW}x{5mm}l}#2&&#4\\\end{tabular}\hfill\hyperlink{pit#6}{{\small\textnormal[pit#6]\It{e}}}
%\index{Z\Red{rec}!\Red{pit#6}}\index{Z\Red{utt}!\Red{pit#6.#7}}
}}
\newcommand{\DerivExamWL}[6]{\Tn{
\begin{tabular}[t]{p{\DW}x{5mm}l}\It{#1}&\ARROW&\It{#3}\\\end{tabular}\hfill‘#5’\\
\hspace{1mm}\begin{tabular}[t]{p{\DW}x{5mm}l}#2&&#4\\\end{tabular}\hfill\hyperlink{explExs}{{\small\textnormal[#6]}}
%\index{Z\Red{wl}!\Red{#6}}
}}


%formatting of corpus source information (after \transl in gb4e-environments):
\newcommand{\Corpus}[2]{\hspace*{1pt}\hfill{\small\mbox{\hyperlink{pit#1}{\Tn{[pit#1.#2]}}}}%\index{Z\Red{rec}!\Red{pit#1}}\index{Z\Red{utt}!\Red{pit#1.#2}}
}%
\newcommand{\CorpusE}[2]{\hspace*{1pt}\hfill{\small\mbox{\hyperlink{pit#1}{\Tn{[pit#1.#2]}}\It{e}}}%\index{Z\Red{rec}!\Red{pit#1}}\index{Z\Red{utt}!\Red{pit#1.#2}\Blue{-E}}
}%
%%as above, but necessary for recording names which include an underline because the first variable in \href understands _ but the second variable requires \_
\newcommand{\CorpusLink}[3]{\hspace*{1pt}\hfill{\small\mbox{\hyperlink{pit#1}{\Tn{[pit#2.#3]}}}}%\index{Z\Red{rec}!\Red{pit#2}}\index{Z\Red{utt}!\Red{pit#2.#3}}
}%
%%as above, but for newer recordings which begin with sje20 instead of pit
\newcommand{\CorpusSJE}[2]{\hspace*{1pt}\hfill{\small\mbox{\hyperlink{sje20#1}{\Tn{[sje20#1.#2]}}}}%\index{Z\Red{rec}!\Red{sje20#1}}\index{Z\Red{utt}!\Red{sje20#1.#2}}
}%
\newcommand{\CorpusSJEE}[2]{\hspace*{1pt}\hfill{\small\mbox{\hyperlink{sje20#1}{\Tn{[sje20#1.#2]}}\It{e}}}%\index{Z\Red{rec}!\Red{sje20#1}}\index{Z\Red{utt}!\Red{sje20#1.#2}\Blue{-E}}
}%











%%hyphenation points for line breaks
%%add to TeX file before \begin{document} with:
%%%%hyphenation points for line breaks
%%add to TeX file before \begin{document} with:
%%%%hyphenation points for line breaks
%%add to TeX file before \begin{document} with:
%%\include{hyphenationSDL}
\hyphenation{
ab-es-sive
affri-ca-te
affri-ca-tes
Ahka-javv-re
al-ve-o-lar
com-ple-ments
%check this:
de-cad-es
fri-ca-tive
fri-ca-tives
gemi-nate
gemi-nates
gra-pheme
gra-phemes
ho-mo-pho-nous
ho-mor-ga-nic
mor-pho-syn-tac-tic
or-tho-gra-phic
pho-neme
pho-ne-mes
phra-ses
post-po-si-tion
post-po-si-tion-al
pre-as-pi-ra-te
pre-as-pi-ra-ted
pre-as-pi-ra-tion
seg-ment
un-voiced
wor-king-ver-sion
}
\hyphenation{
ab-es-sive
affri-ca-te
affri-ca-tes
Ahka-javv-re
al-ve-o-lar
com-ple-ments
%check this:
de-cad-es
fri-ca-tive
fri-ca-tives
gemi-nate
gemi-nates
gra-pheme
gra-phemes
ho-mo-pho-nous
ho-mor-ga-nic
mor-pho-syn-tac-tic
or-tho-gra-phic
pho-neme
pho-ne-mes
phra-ses
post-po-si-tion
post-po-si-tion-al
pre-as-pi-ra-te
pre-as-pi-ra-ted
pre-as-pi-ra-tion
seg-ment
un-voiced
wor-king-ver-sion
}
\hyphenation{
ab-es-sive
affri-ca-te
affri-ca-tes
Ahka-javv-re
al-ve-o-lar
com-ple-ments
%check this:
de-cad-es
fri-ca-tive
fri-ca-tives
gemi-nate
gemi-nates
gra-pheme
gra-phemes
ho-mo-pho-nous
ho-mor-ga-nic
mor-pho-syn-tac-tic
or-tho-gra-phic
pho-neme
pho-ne-mes
phra-ses
post-po-si-tion
post-po-si-tion-al
pre-as-pi-ra-te
pre-as-pi-ra-ted
pre-as-pi-ra-tion
seg-ment
un-voiced
wor-king-ver-sion
}\begin{document}\tableofcontents\clearpage

%%%%%%%%%%%%%%%%%%%%%%%%%%%%%%% ALL THE ABOVE TO BE COMMENTED OUT FOR COMPLETE DOCUMENT! %%%%%%%%%%%

\chapter{Verbs}\label{verbs}\index{verbs}\index{parts of speech!verbs}
Verbs in Pite Saami form an open class of words which are defined syntactically by their ability to head a verbal complex, as well as morphologically by inflecting for person, number, tense and mood. 
Verbs consist of a \mbox{stem (∑)} which is followed by a class marker and an inflectional suffix or suffixes, % indicating tense, mood, person and/or number, 
as illustrated in Figure \vref{verbStructure}.\marginpar{Here and elsewhere, figures are below text (even below footnotes).}

\begin{figure}\centering
\fbox{∑ \PLUS\ class-marker \PLUS\ mood/tense/person/number}
\caption{The morphological structure of \PS\ verbs}\label{verbStructure}
\end{figure}

Verb stems can have up to five allomorphic forms throughout the verbal para\-digm due to a complex combination of morphophonological alternations. 
%Based on the somewhat limited corpus data, %UM: deleted
Verbs form at least five inflectional classes. 
The inflectional suffixes are exponents for person, number, tense and/or mood. 
\PS\ distinguishes three number categories (singular, dual and plural), two tense categories (present and past), and the three modal categories (indicative, imperative and potential). 


\korr{005}The first sections of this chapter (§\,\ref{inflectionalCatsVerbs} on the inflectional categories number, tense and mood; §\,\ref{nonFiniteVerbforms} on non-finite forms and periphrastically marked categories of future, aspect and negation; §\,\ref{passiveVinflection} on passive voice) provide a description of relevant morphological categories as a background for the discussion of morphological marking strategies for verbs in §\,\ref{markingVerbs}. Finally, §\,\ref{verbInflectionalClasses} draws on the initial sections to posit inflectional classes for verbs. 
% (\ref{inflectionalCatsVerbs}) discusses the inflectional categories number, tense and mood, before §\,\ref{nonFiniteVerbforms} goes on to discuss non-finite verb forms which are used to express aspect and negation in analytical constructions. Passive voice is described briefly in §\,\ref{passiveVinflection}. 
%Then, §\,\ref{markingVerbs} discusses linear and non-linear morphological marking strategies in order to then posit inflectional classes for verbs in §\,\ref{verbInflectionalClasses}. %A summary of the inflectional classes is provided in the final section (\ref{verbInflectionalClassesSummary}).


\section{Finite verbs and inflectional categories}\label{inflectionalCatsVerbs}
%Finite verbs in \PS\ agree in person and number with the subject of the clause, and \korr{026}are inflected for tense and mood, as described in the following sections. %The three examples in \REF{inflectionalCatsVerbsEx1} through \REF{inflectionalCatsVerbsEx3} should help to illustrate this.


\subsection{Person and number}\label{personNumberVerbs}
All finite verbs agree in number with the subject of the clause, and inflect for singular, dual or plural. 
Finite verbs in the indicative and the potential mood also agree in person. %\footnote{There are a few examples in the corpus in which speakers did not consistently inflect for dual, but instead used the corresponding plural suffix; I suspect this is due to language contact with Swedish, which does not have a grammatical dual category at all, but only a \SG-plural distinction, specifically caused by the dominance of Swedish in most speakers’ everyday situation.} %JW: UM says this belongs in main text - variation among speakers important part of descriptive grammars! 
\korr{047}Inflectional morphology is present even if the subject of the clause is not overt. 
For instance, in \REF{inflectionalCatsVerbsEx1}, the finite verbs \It{minne} and \It{gillge} both agree with \It{da}, the 3\PLs\ subject; in \REF{inflectionalCatsVerbsEx2}, the finite verb \It{lijmen} agrees with the 1\DUs\ subject \It{månnå ja Jåssjå}.

\ea\label{inflectionalCatsVerbsEx1}%3PL.PRS
\glll	ja dä da tjåhken minne gu gillge gåddålit nagan juhtusav\\
	ja dä d-a tjåhken minne gu gillge gåddåli-t nagan juhtusa-v\\
	and then \Sc{dem}-\Sc{dist}\BS\Sc{nom.pl} together go\BS\Sc{3pl.prs} when will\BS\Sc{3pl.prs} kill-\Sc{inf} some animal-\Sc{acc.sg}\\\nopagebreak
\TranslLong{and then they go together when they are going to kill some animal}{}	\Corpus{080703}{047-048}
\z
\ea\label{inflectionalCatsVerbsEx2}%1DU.PST
\glll	månnå ja Jåssjå lijmen ulgon sirijd tjåggemin\\
	månnå ja Jåssjå li-jmen ulgon siri-jd tjågge-min\\
	\Sc{1sg.nom} and Josh\BS\Sc{nom.sg} be-\Sc{1du.pst} outside blueberry-\Sc{acc.pl} pick-\Sc{prog}\\\nopagebreak
\Transl{Josh and I were picking blueberries outside}{}	\Corpus{100310b}{032}
%\ea\label{inflectionalCatsVerbsEx3}%2SG.IMP
%\glll	tjaske munje sobev\\
%	tjaske munje sobe-v\\
%	throw\BS\Sc{2sg.imp} \Sc{1sg.ill} pole-\Sc{acc.sg}\\
%\Transl{Throw a ski-pole to me’	\Corpus{100404.206}
\z
Note that there are a few examples in the corpus in which speakers do not consistently inflect for dual, but instead use the corresponding plural form. 
%MR: lieber weniger Spekulation zu Sprachkontakt hier, die Kategorie könnte auch durch einen internen Wandel abgebaut werden. Hebe Dir alle diese potentiellen Kontaktmerkmale für ein extra-Paper auf 
%This is possibly due to language contact with Swedish, which only has a singular-plural distinction.%, specifically caused by the dominance of Swedish in most speakers’ everyday situation.} 

%UM: deleted: In the imperative, person is not distinguished morphologically; however, all verbs marked for imperative have 2\superS{nd} person subjects. 
The imperative is not marked for person, but distinguishes the three number categories singular, dual and plural. 
For example, in \REF{inflectionalCatsVerbsEx3} the finite verb \It{tjaske} is inflected for the implied (2\superS{nd} person) singular subject. %\footnote{The finite verb \It{tjaske} in \REF{inflectionalCatsVerbsEx3} is not segmentable because the morpheme indicating singular is non-linear, namely the weak grade of the verb \It{tjassket} ‘throw’; cf. §\,\ref{nonLinearMorphVerbs} on non-linear verb morphology.} 
\ea\label{inflectionalCatsVerbsEx3}%2SG.IMP
\glll	tjaske munje sobev\\
	tjaske munje sobe-v\\
	throw\BS\Sc{sg.imp} \Sc{1sg.ill} pole-\Sc{acc.sg}\\\nopagebreak
\Transl{Throw a ski-pole to me}{}	\Corpus{100404}{206}
\z

\subsection{Tense}\label{tense}
For indicative clauses, verbs can inflect for present tense, as in \REF{inflectionalCatsVerbsEx1}, or past, as in \REF{inflectionalCatsVerbsEx2} above. %Not that non-past is glossed as \PRSs. While it can be used to indicate that a situation is true in the present, as in \REF{}. However, it can also be used to indicate historical present, as in \REF{}, or future situations, as in \REF{}, and is therefore not strictly present.
Verbs marked for present tense generally signify that a situation is true in the present, as in \REF{presentEx1} below, or they express general truths, as in example \REF{inflectionalCatsVerbsEx1} above (which indicates a general truth about wolves’ behavior). However, present tense can also be used to indicate historical present, as in \REF{presentEx2}, or planned future situations, as in \REF{presentEx3}. It is therefore not strictly a \It{present} tense, and could be considered \It{non-past}. %, but \It{non-past} is perhaps a preferable descriptor. %JW: what about historical present? also past, but marked present!
Nonetheless, the glossing standard ‘\PRSs’ is chosen to mark this, as it covers the most common \korr{027}function. %, but also to differentiate it from \PST. 
\ea\label{presentEx1}%present%he says ‘bar’, not ‘ber’
\glll	dale lä bar bievadak mij sudda muahtagav\\
	dale lä bar bievadak mij sudda muahtaga-v\\
	now be\BS\Sc{3sg.prs} only sunshine\BS\Sc{nom.sg} which\BS\Sc{nom.sg} melt\BS\Sc{3sg.prs} snow-\Sc{acc.sg}\\\nopagebreak
\Transl{now it’s only the sun which melts the snow}{}	\Corpus{100405a}{036}
\z
\ea\label{presentEx2}%historical present
\glll	tjävlav valdav ja dä tjanáv virbmev {dan\footnotemark} tjävvlaj ja hålåv raddnaj…\\
	tjävla-v valda-v ja dä tjaná-v virbme-v d-a-n tjävvla-j ja hålå-v raddna-j\\
	bobber-\Sc{acc.sg} take-\Sc{1sg.prs} and then tie-\Sc{1sg.prs} net-\Sc{acc.sg} \Sc{dem}-\Sc{dist}-?\Sc{ill.sg} bobber-\Sc{ill.sg} and say-\Sc{1sg.prs} friend-\Sc{ill.sg}\\\nopagebreak
\TranslLong{I take the bobber and then I tie the net to that bobber and I say to my friend…}{} 	\Corpus{090702}{029}
%\glll	ja gu lä nagin bäjjve mannam, dä vuolgav gåhtaj\\%just a procedural explanation about catching fish, not historical past
%	ja gu lä nagin bäjjve manna-m dä vuolga-v gåhta-j\\
%	and when be\BS\Sc{3sg.prs} some day\BS\Sc{nom.sg} go-\Sc{prf} then go-\Sc{1sg.prs} hut-\Sc{ill.sg}\\
%\Transl{and once a day has passed, I go to the hut’ 	\Corpus{090702.142}
\z
\ea\label{presentEx3}%future
\glll	ja dä maŋŋel dä vuolga Västeråsaj\\
	ja dä maŋŋel dä vuolga Västeråsa-j\\
	and then after.that then drive\BS\Sc{2sg.prs} Västerås-\Sc{ill.sg}\\\nopagebreak
\Transl{and then after that you’ll drive to Västerås}{}	\Corpus{080924}{677}
\z
\footnotetext{In the example in \REF{presentEx2}, it is not clear why the demonstrative \It{dan} is used, as this resembles either the genitive or the inessive demonstratives, but not the expected illative demonstrative \It{dasa}. Perhaps it is simply an error in natural speech.}


\subsection{Mood}\label{mood}\index{mood}
\PS\ has three moods: indicative, imperative and potential. Indicative mood is by far the most common, and considered the default, unmarked mood, as it is not overtly expressed morphologically, as in the examples in §\,\ref{tense} above. 
The following two sections deal with imperative and potential mood. %For imperative mood, person/number suffixes are portmanteau suffixes indicating \IMPs\ as well, while the potential mood is marked by a linearly segmentable morpheme \It{-tj} followed by a person/number suffix. The potential mood person/number suffixes are syncretic with those used in present tense, with the exception of \Sc{3sg}, which is unmarked in potential mood.

\subsubsection{Imperative mood}\label{IMPmood}\index{imperative}\index{mood!imperative}
Verbs inflected for imperative mood indicate that the speaker is instructing or commanding the addressee to carry out the action referred to by the verb; the implied subject is always 2\superS{nd} person. Verbs in the imperative are not marked for person, but do inflect for number (singular, dual and plural), as in \REF{inflectionalCatsVerbsEx3} above as well as in \REF{imperativeEx1} and \REF{imperativeEx2} below; see Table \vref{verbSuffixes} in §\,\ref{inflectionalSuffVerbs} for the imperative number suffixes. %Number and imperative mood are expressed simultaneously in portmanteau suffixes. 
\ea\label{imperativeEx1}%
\glll	nå, giehto naginav dan Luoddauvre birra\\
	nå giehto nagina-v d-a-n Luoddauvre birra\\
	well tell\BS\Sc{sg.imp} something-\Sc{acc.sg} \Sc{dem}-\Sc{dist}-\Sc{gen.sg} Luoddauvre\BS\Sc{gen.sg} about\\\nopagebreak
\Transl{well, say something about this ‘Luoddauvre’}{}	\Corpus{080924}{314}
\z
\ea\label{imperativeEx2}%
\glll	dáhken dal dav\\
	dáhke-n dal d-a-v\\
	do-\Sc{du.imp} now \Sc{dem}-\Sc{dist}-\Sc{acc.sg}\\\nopagebreak
\Transl{do that now}{}	\CorpusE{101208}{188}
\z
The example in \REF{imperativeEx3} below indicates that imperative can also be used as a kind performative speech-act. 
\ea\label{imperativeEx3}%
\glll	gijtov ednet\\
	gijtov edne-t\\
	thank-\Sc{acc.sg} have-\Sc{pl.imp}\\\nopagebreak
\Transl{thank you all}{(lit.: have thank)}	\CorpusE{101208}{290}
\z
Note that \citet[150-155]{Lehtiranta1992} includes  a second imperative category in his verb paradigms that inflects for all three person categories and is marked by a stem-final \It{-u-}; Lehtiranta terms this ‘imperative II’. \citet[22]{Lagercrantz1926} mentions ‘imperative II’ in passing as well, explaining that it is “less severe and more like a wish” (my translation), but  Lagercrantz only includes examples for \Sc{2}\SGs. %They both call this ‘imperative II’, and it inflects for all three person categories. In both sources, the verbs appear to be marked by \It{-u-} as It is not entirely clear what differentiates
The \PSDP\ corpus does not have any tokens of such verbs, so more study is needed to determine \korr{028}their current status. 



\subsubsection{Potential mood}\label{POTmood}\index{potential}\index{mood!potential}
Verbs can also be inflected for potential mood, indicating that the action referred to by the verb is likely to happen. Verbs in the potential mood are marked by a linearly segmentable morpheme \It{-tj-} followed by a person/number suffix.\footnote{Cf. §\,\ref{potClauses} for syntactic aspects of clauses in the potential mood.} 
Examples are provided in \REF{potentialEx1} through \REF{potentialEx3}.
\ea\label{potentialEx1}
\glll	nå hålåv, vuolgetjip del\\
	nå hålå-v vuolge-tji-p del\\
	well say-\Sc{1sg.prs} go-\Sc{pot}-\Sc{1pl} obviously\\\nopagebreak
\Transl{well then I say we should obviously go}{}	\Corpus{090702}{013}
\z
\ea\label{potentialEx2}
\glll	nä, virtitjav nuollat\\
	nä virti-tja-v nuolla-t\\
	no must-\Sc{pot}-\Sc{1sg} undress-\Sc{inf}\\\nopagebreak
\Transl{oh no, I’ll probably have to take off some clothes}{}	\Corpus{090519}{029}
\z
\ea\label{potentialEx3}
\glll	ikeb dat vuosjatja káfav\\
	ikeb d-a-t vuosja-tj-a káfa-v\\
	maybe \Sc{dem}-\Sc{dist}-\Sc{nom.sg} prepare.coffee-\Sc{pot}-\Sc{3sg} coffee-\Sc{acc.sg}\\\nopagebreak
\Transl{perhaps he’ll make some coffee}{}	\CorpusE{110404}{270}
\z
As the examples in \REF{potentialEx4} and \REF{potentialEx5} illustrate, the potential mood can be used as a friendly request. 
\ea\label{potentialEx4}%
\glll	vuosjatja káfav\\
	vuosja-tj-a káfa-v\\
	prepare.coffee-\Sc{pot}-\Sc{2sg} coffee-\Sc{acc.sg}\\\nopagebreak
\Transl{perhaps you could make some coffee}{}	\CorpusE{110404}{267}
\z
\ea\label{potentialEx5}%
\glll	gulatja dav mav mån hålåv\\
	gula-tj-a d-a-v ma-v mån hålå-v\\
	hear-\Sc{pot}-\Sc{2sg} \Sc{dem}-\Sc{dist}-\Sc{acc.sg} \Sc{rel}-\Sc{acc.sg} \Sc{1sg.nom} say-\Sc{1sg.prs}\\\nopagebreak
\Transl{please hear what I am saying}{}	\CorpusE{110404}{056}
\z
The person/number suffixes for potential mood are homophonous %UM notes: syncretism is homonymy of morphemes in the same paradigm 
with those used in present tense for Class V verbs (cf. §\,\ref{VclassV}); cf. §\,\ref{POTinflection} for a discussion of the status of verbs in the potential mood as inflectional and derivational forms. 


\section{Non-finite verb forms and periphrastically marked verbal categories}\label{nonFiniteVerbforms}
\korr{074}A number of non-finite verb forms exist in \PS. The most common of these are the infinitive, connegative, perfect and progressive forms. Each of these non-finite verb forms can co-occur with an auxiliary verb to periphrastically express the verbal categories of future tense, negation, as well as perfect and progressive aspect; these categories are described in §\,\ref{futureTense}, §\,\ref{aspect} and §\,\ref{negationVerb}, respectively. Syntactic aspects of clauses involving these non-finite forms are described in §\,\ref{multiVdeclarativeClauses} on declarative clauses with more than one verb form, as well as in §\,\ref{clausalSubordination} on clausal subordination.  
Table \vref{nonFiniteVTable} summarizes the morphological and syntactic features of these four non-finite forms, while examples of verbs in these forms are provided in Table \vref{nonFiniteVTableExs}. 
\begin{table}\centering
\caption{Summary of common non-finite verb forms and their features}\label{nonFiniteVTable}
\begin{tabular}{|l|l|p{150pt}|}\hline
\It{type}		&\It{morphological features}	&\It{syntactic features}	\\\hline
infinitive		&suffix \It{-t}, strong grade		&co-occurs with auxiliary \It{galgat} for future; complement to lexical verbs like \It{sihtat} ‘want’, \It{állget} ‘begin’, etc.\\
connegative	&no suffix, weak grade		&co-occurs with negation verb			\\
perfect		&suffix \It{-m}, strong grade	&co-occurs with auxiliary \It{årrot} ‘be’	\\
progressive	&suffix \It{-min}, strong grade	&co-occurs with auxiliary \It{årrot} ‘be’	\\
\hline\end{tabular}
\end{table}
\begin{table}\centering
\caption{Examples of non-finite verb forms}\label{nonFiniteVTableExs}
\begin{tabular}{|l|l|l|l|l|}\hline
\It{infinitive}	&\It{connegative}	&\It{perfect}	&\It{progressive}	&\It{gloss}\\\hline
%juhkat	&juga	&juhkam	&juhkamin	& ‘drink’\\
tjájbmat	&tjájma	&tjájbmam	&tjájbmamin	& ‘laugh’\\
viessot	&vieso	&viessom	&viessomin	& ‘live’\\
båhtet	&både	&båhtem	&båhtemin	& ‘come’\\
ságastit	&ságaste	&ságastam	&ságastamin	& ‘speak’\\
%gatjádit	&gatjáde	&gatjádam	&?	& ‘ask’\\
bargatjit	&bargatje	&bargatjam	&bargatjemin	& ‘work a little’\\
%årrot	&lä	&lam	&?	& ‘be’\\%not sure this should be here; it’s in the section on ‘to be’
\hline\end{tabular}
\end{table}


%\subsection{Other non-finite verbs forms}\label{otherNonFiniteVerbforms}
The literature on Saami languages often treats non-finite verb forms in addition to those mentioned above. These include the verb genitive, verb abessive or gerunds, for instance.\footnote{Cf. \citealt[103-104]{Sammallahti1998} and \citealt[67-73]{Svonni2009} for North Saami, or \citealt[104-111]{Spiik1989} for Lule Saami.} 
For \PS, \citet[95-106]{Lehtiranta1992} describes the morphological form a number of such non-finite forms,\footnote{These non-finite forms are also included in the verb paradigms in \citet[150-155]{Lehtiranta1992}.} 
while \citet{Lagercrantz1926} does not describe such verb forms. 

With this in mind, it is certainly plausible that \PS\ has other non-finite verb forms other than those mentioned here. However, there is no evidence of such forms in the present corpus. 
Ultimately, the morphological and syntactic behavior of other non-finite verb forms must be left for future study. 



\subsection{Future}\label{futureTense}
The verb \It{gallgat} ‘will’ plus the infinitive form of the lexical verb can together express a future activity. The examples in \REF{futureTenseEx1} through \REF{futureTenseEx3} illustrate this.
\ea\label{futureTenseEx1}%
\glll	nå gukte galga dåhkå ållit dajna\\
	nå gukte galga dåhkå ålli-t d-a-jna\\
	well how will\BS\Sc{2sg.prs} to.there reach-\Sc{inf} \Sc{dem}-\Sc{dist}-\Sc{com.sg}\\\nopagebreak
\TranslLong{well how are you going to reach it with that!}{}	\Corpus{080909}{052}
\z
\ea\label{futureTenseEx2}%
\glll	dä galgav mån gähttot\\
	dä galga-v mån gähtto-t\\
	then will-\Sc{1sg.prs} \Sc{1sg.nom} tell-\Sc{inf}\\\nopagebreak
\TranslLong{Then I will tell a story}{}	\Corpus{0906\_Ahkajavvre\_a}{115}
\z
\ea\label{futureTenseEx3}%
\glll	man ednag biejve galga danne årrot\\
	man ednag biejve galga danne årro-t\\
	how many day\BS\Sc{nom.pl} will\BS\Sc{2sg.prs} there be-\Sc{inf}\\\nopagebreak
\TranslLong{how many days are you going to be there?}{}	\Corpus{080924}{658}
\z
Note that, as mentioned in §\,\ref{tense} above, the present tense can also express planned future events. 

\subsection{Aspect}\label{aspect}
\PS\ features two aspects, perfect and progressive, as described in §\,\ref{perfectAspect} and §\,\ref{progressiveAspect} below. Both aspects are formed periphrastically using a combination of the auxiliary verb \It{årrot} ‘be’ and the relevant non-finite verb form. 
See also §\,\ref{auxV} on the syntactic structure of clauses with perfective and progressive verbs. 
%Other aspects \Red{(such as ??)} are expressed through verbal derivation, and are covered in §\,\ref{verbalDerivation} on verbal derivation.

\subsubsection{Perfect}\label{perfectAspect}
The perfect verb form is marked by the suffix \It{-m} (glossed as \PRFs); the verb stem is in the strong grade when consonant gradation is relevant. Verbs in the perfect generally indicate that an action in the past still has relevancy in the present situation. For instance, in \REF{perfectEx1} the speaker is slaughtering a reindeer, and is now able to cut out the stomach because the esophagus has been tied in a knot, preventing the stomach’s contents from running out. 
\ea\label{perfectEx1}%
\glll	men mån lev tjåjvev ruhtastemin ullgus, tjådågov lev tjadnam tjieboten\\
	men mån le-v tjåjve-v ruhtaste-min ullgus tjådågo-v le-v tjadna-m tjiebote-n\\
	but \Sc{1sg.nom} be-\Sc{1sg.prs} stomach-\Sc{acc.sg} cut-\Sc{prog} out esophagus-\Sc{acc.sg} be-\Sc{1sg.prs} knot-\Sc{prf} neck-\Sc{iness.sg}\\\nopagebreak
\TranslLong{but I am cutting out the stomach, I have knotted the esophagus in the neck}{}	\Corpus{080909}{054-055}
\z
In \REF{perfectEx2}, the speaker indicates that one can dip potatoes in fish fat only after one has fried the fat, thus melting it.
\ea\label{perfectEx2}%
\glll	gu lä dav bassam, dä máhta pironijd budnjut\\
	gu lä d-a-v bassa-m dä máhta pironi-jd budnju-t\\
	when be\BS\Sc{2sg.prs} \Sc{dem}-\Sc{dist}\BS\Sc{acc.sg} fry-\Sc{prf} then can\BS\Sc{2sg.prs} potato-\Sc{acc.pl} dip-\Sc{inf}\\\nopagebreak
\Transl{once you have fried it, you can dip potatoes (in it)}{}	\Corpus{090702}{088}
\z
Finally, in \REF{perfectEx3}, the perfect form of the verb \It{jábmet} ‘die’ is used 
%UM’s suggestion: 
to mark the state of being dead resulting from the event of dying as opposed to the present state of being alive. 
%JW’s formulation: because the deceased are, of course, still deceased, as opposed to the topic of the sentence (the speaker’s mother), who is still living. 
\ea\label{perfectEx3}%
\glll	da lä jábmam, ber muv äddne'l viessomin dále\\
	d-a lä jábma-m ber muv äddne=l viesso-min dále\\
	\Sc{dem}-\Sc{dist}\BS\Sc{nom.pl} be\BS\Sc{3pl.prs} die-\Sc{prf} only \Sc{1sg.gen} mother\BS\Sc{nom.sg}=be\BS\Sc{3sg.prs} live-\Sc{prog} now\\\nopagebreak
\Transl{they have died, only my mother is still living now}{}	\Corpus{100310b}{145}
\z

\subsubsection{Progressive}\label{progressiveAspect}
Verbs in the progressive indicate that an activity is on-going. The progressive verb form is marked by the suffix \It{-min} (glossed as \PROGs) appended to the verb stem, which is in the strong grade when consonant gradation is relevant. %\marginpar{footnote->häh?}\footnote{Another possible analysis could be that \It{-m} is a general aspect marking indicating ??, that perfect is not marked as such, and that the progressive suffix is \It{-in} and is suffixed to the \It{-m} aspect marker.} %JW: seems that different vowels can precede -min PROG and PRF, so maybe not a good analysis anyway - or just VH?
In \REF{perfectEx1} above, the speaker uses the progressive form \It{rhtastemin} because he is in the middle of cutting out the stomach as he utters the sentence. In \REF{perfectEx3}, the speaker’s mother is still living, as opposed to the deceased. 
The action expressed by a progressive verb does not have to be simultaneous with the moment of the utterance, but can be past tense, as shown by the %\marginpar{soll ich erwähnen, dass Beispiel \REF{inflectionalCatsVerbsEx2b} in einem anderen Kapitel auch zitiert wird?}
example in \REF{inflectionalCatsVerbsEx2b}. %(also presented as \REF{inflectionalCatsVerbsEx2} on page \pageref{inflectionalCatsVerbsEx2}). 
Here, the speaker is describing a picture which was taken while picking blueberries. 
\ea\label{inflectionalCatsVerbsEx2b}%1DU.PST
\glll	månnå ja Jåssjå, lijmen ulgon sirijd tjåggemin\\
	månnå ja Jåssjå li-jmen ulgon siri-jd tjågge-min\\
	\Sc{1sg.nom} and Josh\BS\Sc{nom.sg} be-\Sc{1du.pst} outside blueberry-\Sc{acc.pl} pick-\Sc{prog}\\\nopagebreak
\Transl{Josh and I were picking blueberries outside}{}	\Corpus{100310b}{032}
%\ea\label{progressiveEx2}%
%\glll	da lä jabmam, ber muv äddne'l viessomin dále\\
%	da lä jabma-m ber muv äddne=l viesso-min dále\\
%	\Sc{dem.dist}\BS\Sc{nom.pl} be\BS\Sc{3pl.prs} die-\Sc{prf} only \Sc{1sg.gen} mother\BS\Sc{nom.sg}=be\BS\Sc{3sg.prs} live-\Sc{prog} now\\
%\Transl{they have died, only my mother is living now’	\Corpus{100310b.145}
\z

\subsubsection{Progressive verb forms used adverbially}\label{ADVverbs}
%%from: http://www.eva.mpg.de/lingua/tools-at-lingboard/questionnaire/converbs_description.php
%%A converb (or coverb) is a non-finite verb form that serves to express adverbial subordination, i.e. notions like 'when', 'because', 'after', 'while'. … A converb depends syntactically on another verb form, but is not its argument. It can be an adjunct, i.e. an adverbial, but can neither be the only predicate of a simple sentence, nor clausal argument (i.e. it cannot depend on predicates such as 'begin', 'order', etc.), nor a nominal argument (i.e. it does not occur in subject and object position)
The progressive form of a verb can also be used in an adverbial function. For instance, \It{tjájbmamin} ‘laughing’ in \REF{ADVverbsEx1} and \It{gullamin} ‘listening‘ in \REF{ADVverbsEx2} are each used as a modal adverbial to indicate a simultaneous activity. %being carried out with the main finite verb. %UM: durchgestrichen
%time marked by the manner in which the subject referent was moving. 
\ea\label{ADVverbsEx1}%
\glll	\Bf{tjájbmamin} vádtsa\\
	tjájbma-min vádtsa\\
	laugh-\Sc{prog} go\BS\Sc{3sg.prs}\\\nopagebreak
\Transl{she walks while laughing}{}	\CorpusE{110522}{29m10s}
\z
\ea\label{ADVverbsEx2}%
\glll	\Bf{gullamin} mån tjálav\\
	gulla-min mån tjála-v\\
	listen-\Sc{prog} \Sc{1sg.nom} write-\Sc{1sg.prs}\\\nopagebreak
\Transl{I write while listening}{}	\CorpusE{110404}{089}
%\ea\label{ADVverbsEx3}%don’t know what to make of this; not enough context in the recording
%\glll	måj birrgin \Bf{tjállemin}\\
%	måj birrgi-n tjálle-min\\
%	\Sc{1du.nom} work-\Sc{1du.prs} write-\Sc{prog}\\\nopagebreak
%\Transl{we work with writing}{}	\Corpus{080926}{0m53s}
\z



\subsection{Negation}\label{negationVerb}
Negation in \PS\ is expressed periphrastically by a finite negation verb and a non-finite verb form. The inflectional behavior of the negation verb is presented in §\,\ref{theNegationVerb}, while syntactic aspects of negation in \PS\ are covered in more detail in §\,\ref{negation}; however, a brief description of negation is provided here. 

As with any finite verb, the negation verb agrees in person and number with the subject of the sentence, and inflects for tense or mood. The complement verb occurs in a special non-finite verb form called the connegative (glossed as \mbox{\CONNEGs}), \korr{075}which is in the weak grade (when gradation is relevant) and otherwise lacks any additional morphological marking. Examples for present and past indicative as well as imperative forms are provided in \REF{connegEx1} through \REF{negImpEx1}. 
\ea\label{connegEx1}%
\glll	mån iv vasja lipsusijd ja daggarijd válldet dán muddon\\
	mån i-v vasja lipsusi-jd ja daggari-jd vállde-t d-á-n muddo-n\\
	\Sc{1sg.nom} \Sc{neg}-\Sc{1sg.prs} feel.like\BS\Sc{conneg} rumen.fat-\Sc{acc.pl} and such-\Sc{acc.pl} take-\Sc{inf} \Sc{dem}-\Sc{prox}-\Sc{iness.sg} time-\Sc{iness.sg}\\\nopagebreak
\Transl{I don’t feel like taking the rumen fat and stuff at this time}{}	\Corpus{080909}{091}
\z
\ea\label{connegEx2}%
\glll	nå ittjij Henning dä skihpá, gu lij nåv gållum\\
	nå ittji-j Henning dä skihpá gu li-j nåv gållu-m\\
	well \Sc{neg}-\Sc{3sg.pst} Henning\BS\Sc{nom.sg} then become.sick\BS\Sc{conneg} when be-\Sc{3sg.pst} so freeze-\Sc{prf}\\\nopagebreak
\Transl{well Henning didn’t get sick after he had been freezing like that}{}	\Corpus{090702}{373}
\z
\ea\label{negImpEx1}%
\glll	ele tsábme!\\
	ele tsábme\\
	\Sc{neg}\BS\Sc{sg.imp} hit\BS\Sc{conneg}\\\nopagebreak
\Transl{don’t hit!}{(said to a child)}	\CorpusSJEE{121009}{11m27s}
%\ea\label{negImpEx1}%
%\glll	ele muv åjaldahte\\
%	ele muv åjaldahte\\
%	\Sc{neg}\BS\Sc{sg.imp} \Sc{1acc.sg} forget\BS\Sc{conneg}\\\nopagebreak
%\Transl{don’t forget me}{}	\Corpus{5089dontForgetMe}
\z




\section{Passive voice}\label{passiveVinflection}
Verbs in the passive voice can be derived from other verbs by the derivational suffix \It{-duvv}. Note that the vowel immediately following this suffix is the class marking morpheme for Class IV verbs; cf. §\,\ref{VclassIV}. 
Examples are provided in \REF{passEx4} through \REF{passEx1b}. 
\ea\label{passEx4}%
\glll	dat huvvsa bidtjiduvvuj Nisest\\
	d-a-t huvvsa bidtji-duvvu-j Nise-st\\
	\Sc{dem}-\Sc{dist}-\Sc{nom.sg} house\BS\Sc{nom.sg} build-\Sc{pass}-\Sc{3sg.pst} Nils-\Sc{elat.sg}\\\nopagebreak
\Transl{that house was built by Nils}{}	\CorpusE{110522}{33m03s}
\z
\ea\label{passEx1a}%
\glll	ja dat lä etjaláhkaj dä dat lij dal navte gårroduvvum\\
	ja d-a-t lä etjaláhkaj dä d-a-t lij dal navte gårro-duvvu-m\\
	and \Sc{dem}-\Sc{dist}-\Sc{nom.sg} be\BS\Sc{3sg.prs} different then \Sc{dem}-\Sc{dist}-\Sc{nom.sg} be\BS\Sc{3sg.pst} now like.that sew-\Sc{pass}-\Sc{prf}\\\nopagebreak
\Transl{and that is different as it has been sewn like that}{}	\CorpusLink{080708_Session08}{080708\_Session08}{011}
\z
\ea\label{passEx1b}%
\glll	men dá buhtsu ij lä mierkeduvvum\\
	men d-á buhtsu ij lä mierke-duvvu-m\\
	but \Sc{dem}-\Sc{prox}\BS\Sc{nom.pl} reindeer\BS\Sc{nom.pl} \Sc{neg}\BS\Sc{3pl.prs} be\BS\Sc{conneg} mark-\Sc{pass}-\Sc{prf}\\\nopagebreak
\Transl{but these reindeer have not been marked}{}	\Corpus{080703}{030}
\z
The data from the corpus concerning passive verbs is quite limited, but indicates that passive verbs can be finite verbs inflecting for tense, person and number, or non-finite forms, such as the perfect. However, due to a lack of data, it is not clear whether passives can be used for progressive aspect, or inflect for either imperative or potential mood. 

\korr{012}That being said, these examples do make clear that the passive marker is restricted to lexical verbs. %, and does not necessarily occur on the finite verb of a clause (as the examples with perfect passive participles attest). 
Passives are therefore not considered to be part of inflectional paradigms, but instead valency-decreasing verbal derivations. See also §\,\ref{VdervPassives} in the chapter on derivational morphology, and §\,\ref{passiveVoice} on syntactic aspects of clauses in the passive voice. 

Note that \citet{Ruong1945} includes other derivational suffixes which create passive verbs that are not attested in the corpus. 


\section{Morphological marking strategies on verbs}\label{markingVerbs}
As shown in §\,\ref{inflectionalCatsVerbs} above, finite verbs can be marked for the following four inflectional categories: 
\begin{itemize}
\item{agreement in person with the subject}
\item{agreement in number with the subject}
\item{tense}
\item{mood}
\end{itemize}
Just as with nouns, inflectional categories for verbs can be expressed by suffixes and by non-linear morphology, and frequently a combination of both. In the following, §\,\ref{inflectionalSuffVerbs} focusses on inflectional suffixes, while §\,\ref{nonLinearMorphVerbs} goes on to describe the behavior of non-linear morphology found in stem-consonant alternations (consonant gradation), stem-vowel alternations (umlaut), %bisyllabic stem allomorphy %häh?
and vowel harmony. The final section (\ref{verbInflectionalClasses}) then uses the various morphophonological inflectional patterns found across verb paradigms to posit five preliminary inflectional classes for verbs.


\newcommand{\Xp}[1]{\MC{1}{x{80pt}|}{#1}}%JW: added to allow centering in final set-length column in verb paradigm tables

\subsection{Inflectional suffixes for verbs}\label{inflectionalSuffVerbs}
The portmanteau suffixes expressing agreement in person and number as well as tense or mood in finite verbs are listed in Table \vref{verbSuffixes}. 
\korr{076}In this table, if only one suffix is given in a slot, then it is found in all inflectional classes. When more than one suffix is included in a slot, then the first allomorph is for inflectional classes I, II and III, the second allomorph for class IV, and the third allomorph for class V verbs. 
%The second suffix listed for \Sc{2du.prs}, \Sc{2pl.prs}, \Sc{du.imp} and \Sc{pl.imp} is used when the stem allomorph it occurs with is bisyllabic. 
%Note that this table does not include the deviant person/number suffixes for Class IV verbs; see §\,\ref{VclassIV} for more. 
The suffixes for the non-finite infinitive, connegative and perfect verb forms are included here and in the following sections because they are common verb forms in the corpus and particularly useful in recognizing patterns in verb paradigms. %A number of other non-finite forms also exist (cf. e.g. progressive and gerundium), but are uncommon and not considered here. 
\begin{table}\centering
\caption{Inflectional verb suffixes}\label{verbSuffixes}
\resizebox{1\linewidth}{!} {
\begin{tabular}{|cc||x{80pt}|x{80pt}|p{80pt}|}\hline
\It{tense/}			&			&\MC{3}{c|}{\It{number}}	\\
\It{mood}			&\It{person}	&\MC{1}{c}{\SGs}	&\MC{1}{c}{\DUs}			&\Xp{\PLs}	\\\dline
%PRESENT
\MR{3}{*}{\PRSs}	&1\superS{st}	&-v		&-n/-jin/-n			&\Xp{-p}		\\\cline{3-5}
				&2\superS{nd}	&-		&-bähten/-bähten/ -hpen&\Xp{-ähtet/-bähtet/ -hpit}\\\cline{3-5}
				&3\superS{rd}	&-/-ja/-	&-ba				&\Xp{-/-je/-}	\\\dline%\cline{2-5}%JW: removed ‘-hpa’ from 3DU.PRS because it doesn’t seem to exist in corpus
%PAST
\MR{3}{*}{\PSTs}	&1\superS{st}	&-v/-jiv/-jiv&-jmen			&\Xp{-jmä/-jme/-jme}	\\\cline{3-5}
				&2\superS{nd}	&-/-je/-je	&-jden			&\Xp{-jdä/-jde/-jde}		\\\cline{3-5}
				&3\superS{rd}	&-j		&-jga				&\Xp{-n/-jin/-n}		\\\dline%\cline{2-5}
%IMPERATIVE
\IMPs			&2\superS{nd}	&-		&-n/?/-hten		&\Xp{-t/?/-htet}	\\\hline%\cline{3-5}
%NON-FINITES
\MC{5}{l}{\It{non-finite verb forms:}}\\\hline
\MC{1}{|c}{\INFs}	&\MC{2}{c|}{-t}			&\MC{1}{c}{\CONNEGs}&-			\\\hline
\MC{1}{|c}{\PRFs}	&\MC{2}{c|}{-m}			&\MC{2}{c}{}			\\\cline{1-3}
\end{tabular}}
\end{table}


\subsubsection{Verbal suffixes and syncretism}\label{verbalSuffixesSyncretism}\index{syncretism}\index{suffixes!verbal}\index{verbs!suffixes}
Several of the verbal inflectional suffixes, considered by themselves, are homophonous: %(ignoring the deviant person/number suffixes for Class IV verbs): 
\begin{itemize}
\item{\It{-v} for \Sc{1sg.prs} and \Sc{1sg.pst} in classes I, II and III}
\item{\It{-n} for \Sc{1du.prs}, \Sc{3pl.pst} in all classes, and also \Sc{du.imp} in classes I, II and III}
\item{\It{-t} for \Sc{inf} and \Sc{pl.imp} in classes I, II and III}
\item{\It{no suffix} for \Sc{2sg.prs}, \Sc{sg.imp} and \Sc{conneg} in all classes; \Sc{3sg.prs}, \Sc{3pl.prs} in classes I, II, III and V; and \Sc{2sg.pst} in classes I, II and III}
%\item{(optional) \It{-h} for \Sc{nom.pl} and \Sc{gen.sg}}
\end{itemize}
%Note also that the \INF\ (not an inflectional category) and most \Sc{2pl.imp} forms are marked by a \It{-t} suffix. 

Despite these similarities, only the morphology of \Sc{1du.prs} and \Sc{3pl.pst} verb forms is syncretic in all verb classes because in most cases homophonous suffixes combine with different non-linear morphologically and/or with different class marking suffixes. 
%. In most cases, other inflected verb forms with homophonous suffixes are not syncretic due to non-linear morphophonological processes that co-occur with these suffixes and in addition to allomorphy in some class-marking suffixes within each class’s paradigm. 



\subsection{Non-linear morphology in verbs}\label{nonLinearMorphVerbs}
In addition to using the inflectional suffixes described above, inflectional categories for verbs can be marked %for tense, mood, aspect, person and number, as well as for non-finite forms, 
by one or more of the following stem allomorphy strategies: %\footnote{Cf. §\,\ref{morphophonology} on non-linear morphology.} %of stem consonants and vowels. 
\begin{itemize}
\item{stem consonant alternations (consonant gradation)}
\item{V1 vowel alternations (umlaut)}
\item{V1 vowel raising when followed by a high/mid-high V2 vowel (vowel harmony)}
%\item{additional stem syllable (stem extension)}
\end{itemize}
Because \Sc{2sg.prs}, \Sc{3sg.prs}, \Sc{3pl.prs}, \Sc{2sg.pst}, \Sc{3sg.pst}, \Sc{sg.imp} and \Sc{conneg} forms often lack suffixes (cf. §\,\ref{verbalSuffixesSyncretism} above), verbs in these inflectional categories are typically marked exclusively by these essentially non-linear morphological marking strategies. 

To illustrate this, the inflectional paradigm for the verb \It{buälldet} ‘ignite, burn’ is provided in Table \vref{verbBurn} and described here. 
\renewcommand{\Xp}[1]{\MC{1}{x{90pt}|}{#1}}%JW: added to allow centering in final set-length column in verb paradigm tables; redefined here
\begin{table}\centering
\caption{The inflectional paradigm for the verb \It{buälldet} ‘ignite, burn’}\label{verbBurn}
\resizebox{1\linewidth}{!} {
\begin{tabular}{|cc||x{90pt}|x{90pt}|p{90pt}|}\hline
\It{tense/}			&			&\MC{3}{c|}{\It{number}}	\\
\It{mood}			&\It{person}	&\MC{1}{c}{\SGs}		&\MC{1}{c}{\DUs}			&\Xp{\PLs}	\\\dline
%PRESENT
\MR{3}{*}{\PRSs}	&1\superS{st}	&buold-a-v	&bulld-e-n			&\Xp{buälld-e-p}		\\\cline{3-5}
				&2\superS{nd}	&buold-a		&buälld-e-bähten	&\Xp{buälld-e-bähtet}	\\\cline{3-5}
				&3\superS{rd}	&bualld-a		&buälld-e-ba		&\Xp{bulld-e}		\\\dline%\cline{2-5}
%PAST
\MR{3}{*}{\PSTs}	&1\superS{st}	&bulld-i-v		&buld-i-jmen		&\Xp{buld-i-jmä}	\\\cline{3-5}
				&2\superS{nd}	&bulld-e		&buld-i-jden		&\Xp{buld-i-jdä}		\\\cline{3-5}
				&3\superS{rd}	&buld-i-j		&buld-i-jga		&\Xp{bulld-e-n}		\\\dline%\cline{2-5}
%IMPERATIVE
\IMPs			&2\superS{nd}	&buold-e		&buälld-e-n			&\Xp{?}		\\\hline%\cline{3-5}
\MC{5}{l}{\It{non-finite verb forms:}}\\\hline
\MC{1}{|c}{\INFs}	&\MC{2}{c|}{buälld-e-t}		&\MC{1}{c}{\CONNEGs}&buold-e			\\\hline
\MC{1}{|c}{\PRFs}	&\MC{2}{c|}{bualld-a-m}		&\MC{2}{c}{}			\\\cline{1-3}
\end{tabular}}
\end{table}
Note that the vowel in V2 position (\It{a}, \It{e} and \It{i}) in all forms is the inflectional class marker for Class III verbs (cf. §\,\ref{VclassIII}); thus the stem has five allomorphs: \It{buälld-}, \It{bualld-}, \mbox{\It{buold-},} \It{bulld-} and \It{buld-}.\footnote{The examples used in this description of non-linear verb morphology is based on the current \PS\ orthography, which is still a work in progress. Because the orthography is to a great extent phonemic, orthographic representations are sufficient for the current discussion.} 
%Traditionally in Saami linguistics, the stem allomorph with more segmental material in the consonant center\footnote{Cf. §\,\ref{CCent} about the consonant center.} 
%(i.e., with a geminate consonant or with three consecutive consonants) is called the strong grade, while the shorter allomorph (i.e., with a singleton or with only two consecutive consonants) is termed the weak grade; this terminology is adopted here, as well. 
This reflects a consonant gradation pattern that alternates between strong \It{lld} and weak \It{ld}, and an umlaut pattern that alternates between \It{ua/uä} and \It{uo} in the vowel in V1 position.\footnote{Note that \It{ua} and \It{uä} are allophones of /ua/; cf. §\,\ref{Vua}.} 
Furthermore, the forms for \Sc{1du.prs}, \Sc{3pl.prs} and all past forms are subject to vowel harmony; here, the vowel in V1 position is raised to \It{u} in the presence of a mid-high (\It{e}) or a high front (\It{i}) vowel in V2 position. Note, however, that this vowel harmony is morphologically selected by these slots in the paradigms; the \It{e} in V2 in other inflected forms does not trigger vowel harmony (cf. \Sc{2du.prs} or \Sc{du.imp} forms). 

In summary, the inflectional paradigm for \It{buälldet} ‘ignite, burn’ is characterized by consonant gradation, umlaut and vowel harmony in the stem, and the morphological environment determines which of these allomorphs is selected. %\footnote{Historically, the choice of stem allomorphs was determined phonologically, depending on whether the second syllable was open or closed (\cite{Sammallahti1998}:191,etc.). However, this process is no longer a productive, and these stem alternations are now morphologically determined.} 
For instance, as a result, the \Sc{1sg.prs} form \It{buoldav} is marked for person, number and tense/mood by the weak \It{buold-} stem (with the \It{-uo-} umlaut form) and the \It{-v} suffix simultaneously, and the \Sc{1pl.pst} form \It{buldijmä} is marked by the weak \It{buld-} stem subjected to vowel harmony, and the \It{-jmä} suffix. 

The pattern of non-linear inflectional marking throughout the paradigm for \It{buälldet} is illustrated in Table \vref{burnParadigmPattern}. 
The patterns for both consonant gradation and for umlaut in verb classes subject to these morphophonological strategies align seamlessly. 
However, the two verbal inflection classes subject to vowel harmony each has a unique vowel harmony pattern. 
%Note that not all verbs or verb classes exhibit consonant gradation, umlaut and/or vowel harmony. 
\begin{table}\centering
\caption{Non-linear morphological marking in the paradigm for the verb \It{buälldet} ‘ignite, burn’}\label{burnParadigmPattern}
\resizebox{1\linewidth}{!} {
\begin{tabular}{|cc||x{90pt}|x{90pt}|p{90pt}|}\hline
\It{tense/}			&			&\MC{3}{c|}{\It{number}}	\\
\It{mood}			&\It{person}	&\MC{1}{c}{\SGs}		&\MC{1}{c}{\DUs}			&\Xp{\PLs}	\\\dline
%PRESENT
\MR{3}{*}{\PRSs}	&1\superS{st}	&uo\PLUS wk		&\It{VH}\PLUS str			&\Xp{uä\PLUS str}		\\\cline{4-4}
				&2\superS{nd}	&uo\PLUS wk		&\MC{1}{c}{uä\PLUS str}	&\Xp{uä\PLUS str}		\\\cline{3-3}\cline{5-5}
				&3\superS{rd}	&\MC{1}{c}{ua\PLUS str}	&uä\PLUS str			&\Xp{\It{VH}\PLUS str}		\\\hline%\cline{5-5}
%PAST
\MR{3}{*}{\PSTs}	&1\superS{st}	&\It{VH}\PLUS str		&\MC{1}{c}{\It{VH}\PLUS wk}	&\Xp{\It{VH}\PLUS wk}		\\%\cline{3-5}
				&2\superS{nd}	&\It{VH}\PLUS str		&\MC{1}{c}{\It{VH}\PLUS wk}	&\Xp{\It{VH}\PLUS wk}		\\\cline{3-3}\cline{5-5}
				&3\superS{rd}	&\MC{1}{c}{\It{VH}\PLUS wk}&\It{VH}\PLUS wk		&\Xp{\It{VH}\PLUS str}		\\\dline%\cline{2-5}
%IMPERATIVE
\IMPs			&2\superS{nd}	&uo\PLUS wk		&uä\PLUS str			&\Xp{?}			\\\hline%\cline{3-5}
\MC{5}{l}{\It{non-finite verb forms:}}\\\hline
\MC{2}{|c||}{\INFs}	&\MC{1}{c|}{uä\PLUS str}			&\MC{1}{c||}{\CONNEGs}&\MC{1}{c|}{uo\PLUS wk}		\\\cline{1-2}\cline{4-5}
\MC{2}{|c||}{\PRFs}	&\MC{1}{c|}{ua\PLUS str}			&\MC{2}{c}{}					\\\cline{1-3}
\end{tabular}}
\end{table}
\FloatBarrier


Not every verb undergoes consonant gradation and/or umlaut; instead, their presence are determined by the phonological form of a verb.\footnote{Consonant gradation is described in detail in §\,\ref{Cgrad}, and umlaut in §\,\ref{umlaut}.} 
Some examples of verbs with consonant gradation and umlaut alternations are shown in Table \ref{CGpatternsVerbs} and Table \ref{umlautPatternsVerbs}, respectively. Note that \It{ua} and \It{uä} are allophones of /ua/; cf. §\,\ref{Vua}.
\begin{table}\centering
\caption{Consonant gradation patterns for verbs, including examples}\label{CGpatternsVerbs}
\begin{tabular}{|c c c || c c c | l|}\hline
\MC{3}{|c||}{\It{C-grad pattern}}	&\MC{3}{c}{\It{examples}}&\MC{1}{c|}{}	\\
strong&\Div &weak	& strong	&\Div &weak	&\It{gloss}\\\dline
ʰx	&\Div &x		&/pɔʰta/	&\Div &/pɔta/	& ‘come’\\%
	&&		&\It{båhta}	&&\It{båda}&\\\hline
xː	&\Div &x		&/paːlːa/	&\Div &/paːla/	& ‘dig’\\%
	&&		&\It{bálla}	&&\It{bála}&\\\cline{4-7}
	&&		&/maːʰtːa/	&\Div &/maːʰta/	& ‘be able to’\\%
	&&		&\It{máhtta}&&\It{máhta}&\\\hline
xːy	&\Div & xy	&/parːka/	&\Div &/parka/	& ‘work’\\%
	&&		&\It{barrga}&&\It{barga}&\\\hline
xy	&\Div &y		&/atnaː/	&\Div &/anaː/	& ‘have’\\%
	&&		&\It{adná}	&&\It{aná}&\\\hline
xyz	&\Div & xz	&/ʧaːjpma/	&\Div &/ʧaːjma/	& ‘laugh’\\%\hline%
	&&		&\It{tjájbma}&&\It{tjájma}&\\\hline
\MC{1}{c}{}&&\MC{1}{c|}{}&3\SGs.\PRSs	& &2\SGs.\PRSs	&\MC{1}{c}{}\\\cline{4-6}
\end{tabular}
\end{table}

\begin{table}\centering
\caption{Umlaut alternation patterns for verbs, including examples}\label{umlautPatternsVerbs}
\begin{tabular}{|c c c || c c c | l|}\hline
\MC{3}{|c||}{\It{pattern}}	&\MC{3}{c}{\It{examples}}&\MC{1}{c|}{}	\\
x&\Div &y		&x	&\Div &y	&\It{gloss}\\\dline
ɛ	&\Div &e		&/kɛʰʧa/	&\Div &/keʧa/		& ‘look’\\%l
	&&		&\It{gähtja}&&\It{gietja}	& \\\hline%l
u͡a	&\Div &o		&/pu͡alːta/	&\Div &/polta/	& ‘ignite, burn’\\%
	&&		&\It{buallda}&&\It{buolda}	& \\\hline%
\MC{1}{c}{}&&\MC{1}{c|}{}&3\SGs.\PRSs	& &\MC{1}{c|}{2\SGs.\PRSs}	&\MC{1}{c}{}	\\\cline{4-6}
\end{tabular}
\end{table}

\FloatBarrier


\subsubsection{Vowel harmony patterns for verbs}\label{VHPatternSectionVerbs}
Vowel harmony in verb forms refers to a regressive assimilation of place of articulation between the two vowels of the final foot in a word. Specifically the raising of the vowel in V1 position triggered by the presence in specific, class-dependent paradigmatic slots of a close-mid /e/ (orthographic \It{e}) or a close front /i/ (orthographic \It{i}) vowel in V2 position. 
There are six attested vowel harmony patterns in the V1 vowel of a verb stem from Class II or Class III, as illustrated by Table \vref{VHPatternsVerbs}. 
\begin{table}\centering
\caption[Vowel harmony alternation patterns for verbs]{Vowel harmony alternation patterns for verbs (Class II and III)}\label{VHPatternsVerbs}
\begin{tabular}{|c c c || c c c | l|}\hline
\MC{3}{|c||}{\It{pattern}}	&\MC{3}{c}{\It{examples}}&	\\
A&\ARROW&B			&A		&\ARROW&B		&\It{gloss}\\\dline
á	&\ARROW&i		&tjájbmat	&\ARROW&tjijbme	& ‘laugh’\\%l
á	&\ARROW&ä		&sávvat	&\ARROW&sävve	& ‘wish’\\%l
%á	&\ARROW&i		&dáhkat	&\ARROW&dihke	& ‘do’\\%l
a	&\ARROW&i		&barrgat	&\ARROW&birrge	& ‘work’\\%l
a	&\ARROW&e		&adnet	&\ARROW&edne	& ‘have’\\%l
å	&\ARROW&u		&bårråt	&\ARROW&burre	& ‘eat’\\%l
uä	&\ARROW&u		&buälldet	&\ARROW&bullde	& ‘ignite’\\\hline%l
\MC{1}{c}{}&	&\MC{1}{c|}{}&\INFs	&		&2\SGs.\PSTs&\MC{1}{c}{}\\\cline{4-6}
\end{tabular}
\end{table}

The data from the corpus indicate that Class I and Class IV verbs do not exhibit vowel harmony, but there are no tokens of Class I or Class IV verbs with one of the vowels listed in Table \vref{VHPatternsVerbs} in V1 position. Consequently, the data must be considered inconclusive in this respect. On the other hand, it is quite evident that Class V verbs are not affected by vowel harmony because the V2 vowel in Class V verbs is never subject to the allomorphic alternations which trigger vowel harmony in the V1 vowel. 

It is not clear why \It{á} and \It{a} each have two different vowel harmony alternations (\It{i}/\It{ä} and \It{i}/\It{e}, respectively, as illustrated by the first four examples in Table \ref{VHPatternsVerbs}); these alternation patterns do not align with verb classes. 
Further research is needed to come to a better understanding of this vowel harmony. 


\subsection{The potential mood: inflection or derivation?}\label{POTinflection}\index{potential}\index{mood!potential}
%\subsection{A note on inflection, derivation and the potential mood}\label{POTinflection}\index{potential}\index{mood!potential}
The potential mood\footnote{Cf. §\,\ref{POTmood} for a general description, including examples, of the potential mood.} 
is not attested very often in the corpus, particularly outside elicitation settings, 
and was not considered in most elicitation sessions focussing on verb paradigms. As a result, the amount of data from the corpus available to inform a description of the inflectional behavior of the potential forms is quite limited. Nonetheless, the paradigm for the potential forms of the verb \It{gullat} ‘hear’ is provided in Table \vref{hearPOTforms}.
\renewcommand{\Xp}[1]{\MC{1}{x{80pt}|}{#1}}%JW: added to allow centering in final set-length column in verb paradigm tables; redefined here
\begin{table}\centering
\caption{Potential forms for the verb \It{gullat} ‘hear’}\label{hearPOTforms}
\resizebox{1\linewidth}{!} {
\begin{tabular}{|cc||x{80pt}|x{80pt}|p{80pt}|}\hline
				&			&\MC{3}{c|}{\It{number}}	\\
\It{mood}			&\It{person}	&\MC{1}{c}{\SGs}	&\MC{1}{c}{\DUs}			&\Xp{\PLs}	\\\dline
%POTENTIAL
\MR{3}{*}{\POTs}	&1\superS{st}	&gulatjav	&gulatjen			&\Xp{gulatjep}		\\\cline{3-5}
				&2\superS{nd}	&gulatja	&gulatjähpen		&\Xp{gulatjehpit}\\\cline{3-5}
				&3\superS{rd}	&gulatja	&gulatjäba		&\Xp{gulatje}		\\\hline
\end{tabular}}
\end{table}

Under consideration of the potential forms presented in the verb paradigms in \citet[150-155]{Lehtiranta1992} and in the examples in \citet[22-24]{Lagercrantz1926}, the paradigm of class marking suffixes and person/number suffixes used for potential verb forms is presented in Table \vref{POTinflections}. %Note that, as mentioned in §\,\ref{POTmood}, 
The stem allomorph of the verb is in the weak stage, when applicable. 
%\renewcommand{\Xp}[1]{\MC{1}{x{80pt}|}{#1}}%JW: added to allow centering in final set-length column in verb paradigm tables; redefined here
\begin{table}\centering
\caption{Class marking suffixes and person/number suffixes for potential verb forms}\label{POTinflections}
\resizebox{1\linewidth}{!} {
\begin{tabular}{|cc||x{80pt}|x{80pt}|p{80pt}|}\hline
				&			&\MC{3}{c|}{\It{number}}	\\
\It{mood}			&\It{person}	&\MC{1}{c}{\SGs}	&\MC{1}{c}{\DUs}			&\Xp{\PLs}	\\\dline
%POTENTIAL
\MR{3}{*}{\POTs}	&1\superS{st}	&-a-v	&-e-n			&\Xp{-e-p}		\\\cline{3-5}
				&2\superS{nd}	&-a		&-ä-hpen			&\Xp{-e-hpit}\\\cline{3-5}
				&3\superS{rd}	&-a		&-ä-ba			&\Xp{-e}		\\\hline
\end{tabular}}
\end{table}

In the literature on Saami languages, potential mood is normally treated as an inflectional category,\korr{072}\footnote{Cf., e.g., \citet[76-84]{Sammallahti1998}, \citet[88-89, 150-153]{Lehtiranta1992}, \citet[118-122]{Lagercrantz1926} and \citet[115]{Feist2010}.} 
and, for this reason as well as due to its seeming opposition to imperative or tense-marked forms, is treated as such in the present study. 

However, there are three morphosyntactic aspects of potential mood verb forms which make its classification as an inflectional category potentially questionable. 
First, verbs in the potential mood feature a segmentally separable marker (\It{-tj-}), rather than being part of a portmanteau morpheme simultaneously indicating mood/tense, number and normally person as is the case for other tenses and moods. 
Second, the stem allomorph chosen in all potential forms is consistently the weak form, which is quite consistent with the morphosyntactic behavior of other derived verbs which consistently have a specific consonant gradation type, while the mood and tense paradigms for non-derived verbs contain both strong and weak stem allomorphs. 
Finally, it is striking that the potential mood class marking suffixes and person/number suffixes (listed in Table \vref{POTinflections}) are homophonous with the class marking and present tense person/number suffixes for Class V verbs (cf. Table \vref{speakParadigm}).\footnote{Note that \Sc{3sg} potential forms in \citet[150-154]{Lehtiranta1992} %and \citet[]{Lagercrantz1926} %%no paradigms in Lagercratz for verbs in POT (except for irregular ‘be’-forms on. p. 115). Only mention is that the POT-suffix is -tj (p. 119)
do not have a class marker or person/number suffix. This deviates from the \Sc{3sg}.\PRSs\ forms of Class V verbs (even though \citet[88]{Lehtiranta1992} mentions that the potential forms are inflected in the same way as indicative present forms). On the other hand, all instances of \Sc{3sg} potential forms in the \PSDP\ corpus are marked with \It{-a}, just like the \Sc{3sg}.\PRSs\ forms of Class V verbs. Perhaps this \Sc{3sg} potential marker is a recent change to the \PS\ potential verb forms based on analogy to these present tense forms.} 
In all of these three aspects, the potential forms of verbs are identical in behavior to a number of derivational verb forms (cf. §\,\ref{verbDIM}, §\,\ref{vblzST} and §\,\ref{vblzD}), and unlike other inflectional tense/mood forms. 
At this point, the only morphological motivation to classify the potential mood as an inflectional category is its complementary distribution with other tense and mood forms. These characteristics are summarized in Table \vref{POTcomparison}. 
\begin{table}\centering
\caption{Features of potential verb forms characterized as typical for inflectional or derivational forms}\label{POTcomparison}
\begin{tabular}{|p{.64\textwidth}|c|c|}\hline
\It{}							&\MC{2}{c|}{\It{consistent with}}	\\
\It{features of potential forms}		&\It{inflection}	&\It{derivation}	\\\dline
consistently linearly segmentable marker	&	&\CH	\\\hline
consistently occurs with specific ∑-allomorph	&	&\CH	\\\hline
person/number marking like Class-V verbs	&	&\CH	\\\hline
complementary distribution with tense/mood forms	&\CH	&	\\\hline
\end{tabular}
\end{table}

With these facts in mind, potential forms could be analyzed as derived verb forms consisting of a lexical verbal root plus a verbalizer (the potential mood morpheme) followed by Class V inflectional suffixes. This possible analysis is illustrated in Figure \vref{POTparsedEx}, %on page \pageref{POTparsedEx}, 
in which the form \mbox{\It{gulatjav}} ‘I will likely hear’ is parsed morphologically.
\begin{figure}\centering
\begin{tabular}{cc cc cc cc}
%\MC{8}{r}{gulatjav}\\
\It{gula}	&-&\It{tj}	&-&\It{a}	&-&\It{v}	\\
hear		&-&\POTs	&-&V		&-&1\SGs	\\
∑		&-&mood	&-&class	&-&person/number\\
%\it gul	&-&\it a	&-&\it tj	&-&\it av	\\
%hear		&-&II		&-&\POTs	&-&1\SGs	\\
%∑		&-&class	&-&mood	&-&person/number\\
\end{tabular}
\caption{The morphological structure of the verb form \It{gulatjav} ‘I will likely hear’}\label{POTparsedEx}
\end{figure}
In such an analysis, potential verbs no longer stand in opposition to tense and imperative mood forms, but instead are subject to a semantic restriction to a non-past time, and are thus only marked for present (i.e.,~non-past) tense, and are marked according to the present tense slots of the inflectional paradigm for Class V verbs. 

It should be pointed out that the corpus contains insufficient data concerning the potential forms of any verbs in Class V. This is relevant because Class V verbs have bisyllabic stems that, together with the potential marker, may trigger allomorphy in other person/number suffixes, in which case not all potential forms would follow the standard Class V paradigm. Such information would be essential in fully evaluating the analysis proposed here. 
%, although the data in \cite[152]{Lehtiranta1992} indicate that only the second person dual form deviates from the standard Class V suffixes by showing a strong grade of the suffix, even though at least the second person plural and possibly the third person dual forms should exhibit such a deviation as well.
Due mainly to this lack of truly conclusive data, I continue to follow the standard classification of the potential mood as an inflectional category for the means of the present study, but point out this potentially problematic analysis for \PS\ as described above as a topic worthy of future study. 
%JW: would be necessary to look at POT-forms of 3-syllable stems; in Lehtiranta 1992: 152 only 2DU.POT deviates from the above suffix-paradigm!


\section{Inflectional classes for verbs}\label{verbInflectionalClasses}
Verbs in \PS\ can be grouped into inflectional classes based on recurring patterns across inflectional paradigms.\footnote{I am indebted to phonologist and Lule Saami scholar Bruce Morén-Duolljá for inspiring me to consider an approach to the data involving post-stem class marking morphology.} 
Each verb is marked by a class suffix which is attached directly after the verb stem and precedes inflectional suffixes (cf. Figure \vref{verbStructure}). %For the majority of verb classes, this suffix consists only of a vowel (in V2 position); however, the class marking suffix in classes III and IV deviate from this pattern. 
\korr{007}Unlike nouns, the potential to have umlaut alternations and/or consonant gradation present for a given verb is dependent on the verb’s membership in a specific class. However not every verb in the umlaut/gradation classes is subject to these alternations, as that is determined by whether the phonemes occupying the V1 position and the consonant center of the final foot, respectively, are susceptible to umlaut and/or consonant gradation. 
%\marginpar{check if -tj blocks CG/umlaut in verbs!}
Furthermore, some derivational suffixes (such as the diminutive suffix \It{-tj}) can block consonant gradation and umlaut from occurring in the derived form. 
Membership in a specific verb class does not seem to be semantically motivated. %

As described in the previous section, \PS\ verb paradigms present complex combinations of linear morphology (inflectional suffixes) and non-linear morphology (consonant gradation, umlaut, vowel harmony), and consist of a minimum of 21 finite forms and several non-finite forms. This minimum includes 1\superS{st}, 2\superS{nd} and 3\superS{rd} person forms for singular, dual and plural in both present and past, as well as singular, dual and plural forms for imperative.\footnote{Because of insufficient data concerning the potential forms of verbs, but also due to their regular predictability across classes (cf. §\,\ref{POTinflection}), these were not considered in determining inflectional classes for verbs.} 
These are by far the most common forms in non-elicited data from the corpus. Furthermore, the three non-finite forms infinitive, connegative and perfect were also considered in determining inflectional classes. %\footnote{In addition to a handful of further non-finite forms, \cite{Lehtiranta1992} also includes nine singular, dual and plural ‘potential’ forms, three 3\superS{rd} person ‘imperative I’ forms and nine singular, dual and plural ‘imperative II’ forms; there is simply not enough data in the present documentation corpus to come to any conclusions concerning ‘imperative II’.} %In addition to inflectional suffixes, consonant gradation, umlaut and vowel harmony are all relevant categories to consider when comparing verb forms within and across paradigms, as described in the previous sections. 
The non-elicited portions of the \PSDP\ corpus are simply too limited to even come close to providing complete paradigms for even a single verb, and so a majority of the verb forms composing the paradigms for the current study are from elicitation sessions. Approximately 30 more or less complete verb paradigms were recorded, which provides sufficient data to posit five inflectional classes. However, the true extent and finer details of the morphophonological patterns found across verb paradigms in \PS\ must be left to future study; it is possible that, with more research, more verb classes may result, or that the present classes may need revision. As a result, what follows must be considered of a preliminary nature. 


There are five main criteria for positing five different verb classesː
\begin{itemize}
\item{the regularity of the pattern of vowels occurring between the stem and inflectional suffixes (i.e., the class marking suffix)}
\item{the number of syllables in the infinitive form}
%\item{the allomorph of the stem in infinitive in relation to the other allomorphs in the inflectional paradigm (i.e., bisyllabic stem allomorph and consonant gradation/umlaut)}
\item{the presence of deviant person/number suffixes relative to the other verb classes}
\item{V1 and consonant center stem allomorphy patterning throughout the inflectional paradigm (i.e., umlaut and consonant gradation)}
\item{whether some verb forms are subject to vowel harmony, and which slots trigger such vowel harmony}
\end{itemize} 
To summarize these differences, it is sufficient to look at the class suffix and the syllable count in the infinitive form, the regularity of person/number suffixes across classes, the presence of consonant gradation (‘C-grad’) and umlaut, and the presence/absence of vowel harmony,\footnote{Note that, particularly historically, the verbs in inflectional classes I, II and II are similar to nouns in inflectional class I, while verbs in class IV are similar to nouns in class II, and verbs in class V correspond to nouns in class III. Cf. §\,\ref{nounClasses} for inflectional classes for nouns.} 
as illustrated in Table \vref{verbClassSummaryTable}. %\footnote{In Table \ref{verbClassSummaryTable}, the following abbreviations are used: } %Due to insufficient data, it is not clear whether Class IV verbs are always marked by \It{-u}; this is indicated by a question mark in this table. %on page \pageref{verbClassSummaryTable}. 
\begin{table}\centering
\caption{Summary of inflectional classes for verbs}\label{verbClassSummaryTable}
\begin{tabular}{c c c c c c}%\hline
\MC{1}{c}{}		&\MC{2}{c}{\It{infinitive}}						&\MC{1}{c}{\It{deviant}}&\MC{1}{c}{\It{C-grad /}}	&\MC{1}{c}{\It{VH}}		\\
\MC{1}{c}{\It{class}}	&\MC{1}{c}{\It{class suffix}}&\MC{1}{c}{σ-count}&\MC{1}{c}{\It{agr.\,sx.}}	&\MC{1}{c}{\It{umlaut}}	&\MC{1}{c}{\It{(pattern)}}	\\\hline
I				&o					&2				&					&\CH				&		\\%\hline
II				&a/å					&2				&					&\CH				&\CH(A)		\\%\hline
III				&e					&2				&					&\CH				&\CH(B)		\\%\hline
IV				&V					&2				&\CH				&					&		\\%\hline
V				&i					&3				&					&					&		\\\hline
\end{tabular}
\end{table}

Class I is the least complex class, %in which no morphophonology beyond consonant gradation and umlaut occurs within the noun paradigm, 
and is therefore dealt with first in §\,\ref{VclassI}, %Because Class I are very similar, they are dealt with together in the first section below (§\,\ref{NclassI2III}), 
while classes II, III, IV and V are described in §\,\ref{VclassII} %, \ref{VclassIII}, \ref{VclassIV} and 
through §\,\ref{VclassV}. %, respectively. 
§\,\ref{otherVerbClasses} briefly discusses the possibility of the existence of other verb classes. The verb \It{årrot} ‘be’ and the negation verb are dealt with in §\,\ref{theCopulaVerb} and §\,\ref{theNegationVerb}. The final section (\ref{verbInflectionalClassesSummary}) provides a brief summary of the verb classes, including a table listing examples from each of the verb classes. 


\subsection{Class I}\label{VclassI}
Verbs in Class I are relatively simple, and characterized as follows: 
\begin{itemize}
\item{a bisyllabic infinitive form}
\item{the class marking suffix is consistently \It{-o}}
\item{potentially subject to consonant gradation and umlaut, but not vowel harmony}
\end{itemize}
%because they are not affected by vowel harmony, and the class marking suffix is consistently \It{-o}. 
%This class is further defined by having the typical ‘str-wk’ consonant gradation pattern. 
%The \INF\ form is bisyllabic. 
The verb \It{viessot} ‘live, feel’ is provided in Table \vref{liveParadigm} as an example. Other examples of Class I verbs include: \It{årrot} ‘reside’, \It{gårrot} ‘sew’, \It{gähttjot} ‘tell’, \It{lávvlot} ‘sing’, and \It{såggot} ‘drown’.
\begin{table}\centering
\caption{The inflectional paradigm for the Class I verb \It{viessot} ‘live, feel’}\label{liveParadigm}
\resizebox{1\linewidth}{!} {
\begin{tabular}{|cc||x{80pt}|x{80pt}|p{80pt}|}\hline
\It{tense/}			&			&\MC{3}{c|}{\It{number}}	\\
\It{mood}			&\It{person}	&\MC{1}{c}{\SGs}	&\MC{1}{c}{\DUs}		&\Xp{\PLs}	\\\dline
%PRESENT
\MR{3}{*}{\PRSs}	&1\superS{st}	&vies-o-v	&viess-o-n		&\Xp{viess-o-p}		\\\cline{3-5}
				&2\superS{nd}	&vies-o	&viess-o-bähten	&\Xp{viess-o-bähtet}	\\\cline{3-5}
				&3\superS{rd}	&viess-o	&viess-o-ba		&\Xp{viess-o}		\\\dline%\cline{2-5}
%PAST
\MR{3}{*}{\PSTs}	&1\superS{st}	&viess-o-v	&vies-o-jmen		&\Xp{vies-o-jme}	\\\cline{3-5}
				&2\superS{nd}	&viess-o	&vies-o-jden		&\Xp{vies-o-jde}		\\\cline{3-5}
				&3\superS{rd}	&vies-o-j	&vies-o-jga		&\Xp{viess-o-n}		\\\dline%\cline{2-5}
%IMPERATIVE
\IMPs			&2\superS{nd}	&vies-o	&viess-o-n		&\Xp{viess-o-t}		\\\hline%\cline{3-5}
\MC{5}{l}{\It{non-finite verb forms:}}\\\hline
\MC{1}{|c}{\INFs}	&\MC{2}{c|}{viess-o-t}	&\MC{1}{c}{\CONNEGs}&vies-o			\\\hline
\MC{1}{|c}{\PRFs}	&\MC{2}{c|}{viess-o-m}	&\MC{2}{c}{}\\\cline{1-3}
\end{tabular}}
\end{table}

\pagebreak
Table \vref{VclassIsummaryTable} %on page \pageref{VclassIsummaryTable} 
summarizes the gradation pattern and class suffixes for Class I verbs. Note that umlaut alternations align with consonant gradation alternations. 
\begin{table}\centering
\caption{Summary of Class I verb paradigm features}\label{VclassIsummaryTable}
\begin{tabular}{|cc|| c|c|| c|c|| c|c|}\hline
				&			&\MC{6}{c|}{\It{number}}	\\
\It{tense/}			&			&\MC{2}{c||}{\SGs}			&\MC{2}{c||}{\DUs}			&\MC{2}{c|}{\PLs}	\\%\dline
\It{mood}			&\It{person}	&\MC{1}{c}{\It{C-grad}}&\It{cl.\,sx.}&\MC{1}{c}{\It{C-grad}}&\It{cl.\,sx.}	&\MC{1}{c}{\It{C-grad}}&\It{cl.\,sx.}	\\\cline{1-8}%\cline{1-8}
%PRESENT
\MR{3}{*}{\PRSs}	&1\superS{st}	&wk			&o			&str			&o			&str			&o		\\\cline{3-8}
				&2\superS{nd}	&wk			&o			&str			&o			&str			&o		\\\cline{3-8}
				&3\superS{rd}	&str			&o			&str			&o			&str			&o		\\\hline%\cline{3-8}
%PAST
\MR{3}{*}{\PSTs}	&1\superS{st}	&str			&o			&wk			&o			&wk			&o		\\\cline{3-8}
				&2\superS{nd}	&str			&o			&wk			&o			&wk			&o		\\\cline{3-8}
				&3\superS{rd}	&wk			&o			&wk			&o			&str			&o		\\\hline%\cline{3-8}
%%IMPERATIVE
\IMPs			&2\superS{nd}	&wk			&o			&str			&o			&str			&o		\\\hline%\cline{3-8}
\MC{8}{l}{\It{non-finite verb forms:}}\\\hline
\MC{2}{|r||}{\INFs}				&str			&o			&\MC{2}{r||}{\CONNEGs}		&wk			&o		\\\hline%\cline{2-8}
\MC{2}{|r||}{\PRFs}				&str			&o			&\MC{4}{c}{}		\\\cline{1-4}
%\MC{1}{c}{}		&\MC{7}{l}{\It{non-finite verb forms:}}\\\cline{3-4}\cline{7-8}%\hline
%\MC{2}{c|}{}					&\It{C-grad}	&\MC{1}{c|}{\it Cl. suffix}&\MC{2}{c|}{}				&\It{C-grad}	&\it Cl. suffix	\\\cline{2-4}\cline{6-8}
%\MC{1}{c|}{}	&\MC{1}{|c|}{\INFs}	&str			&\MC{1}{c|}{o}&\MC{1}{c|}{}	&\MC{1}{c|}{\CONNEGs}&wk			&o			\\\cline{2-4}\cline{6-8}
%\MC{1}{c|}{}	&\MC{1}{|c|}{\PRFs}	&str			&\MC{1}{c|}{o}&\MC{4}{c}{}\\\cline{2-4}
\end{tabular}
\end{table}

There are a number of verbs which seem to be marked by \It{-u} as a class marker in infinitive, such as %\marginpar{\Bf{gävdnut} perhaps Class V?} 
\It{gävdnut} ‘exist’ and \It{pruvkut} ‘use; usually do’. While the data in the corpus is incomplete, such verbs likely pattern in essentially the same way as the verbs mentioned above marked by \It{-o}, only they are consistently marked with \It{-u} as the class marking suffix. 

\FloatBarrier
%\clearpage


\subsection{Class II}\label{VclassII}
The characteristics of verbs in Class II are:
\begin{itemize}
\item{a bisyllabic infinitive form with a class suffix \It{-a} or \It{-å}}
\item{potentially subject to consonant gradation, umlaut and vowel harmony}
\end{itemize}
%This class is further defined by having the typical ‘str-wk’ consonant gradation pattern. 
%The \INF\ form is bisyllabic. 
For most inflected forms, the class marking suffix is consistent with the class marking suffix in the infinitive form; however, eight forms are assigned a specific class-marking vowel, as listed in Table \vref{deviantClassSuffixesVClassIITable}. %for 3\SGs.\PRSs, 1\DUs.\PRSs, 3\PLs.\PRSs, 1\SGs.\PSTs, 2\SGs.\PSTs, 
\begin{table}\centering
\caption{Specific class marking suffixes for all Class II verbs}\label{deviantClassSuffixesVClassIITable}
\begin{tabular}{|cc||cc|}\hline
\It{form}		&\It{cl.\,sx.}&\It{form}		&\It{cl.\,sx.}	\\\dline
3\SGs.\PRSs	&-a		&1\DUs.\PRSs	&-e	\\\hline
1\SGs.\PSTs	&-i		&2\SGs.\PSTs	&-e	\\\hline
3\PLs.\PRSs	&-e		&3\PLs.\PSTs	&-e	\\\hline
2\DUs.\IMPs	&-e		&2\PLs.\IMPs	&-i	\\\hline
\end{tabular}
\end{table}

Class II verbs can further be divided into two sub-classes, based on the class marking suffix in the infinitive form: Class IIa is marked by \It{a}, while Class IIb is marked by \It{å}. 
The verb \It{bassat} ‘wash’ is provided in Table \vref{washParadigm} as an example for a Class IIa verb, and \It{bårråt} ‘eat’ for a Class IIb verb in Table \vref{eatParadigm}. 

Other examples of Class IIa verbs include: \It{juhkat} ‘drink’, \It{tjájbmat} ‘write’, \It{barrgat} ‘work’, \It{gullat} ‘hear’, \It{gähtjat} ‘look’, and \It{sávvat} ‘wish’. 
The corpus only provides sufficient data for the Class IIb verb \It{bårråt}; the verbs \It{dåbbdåt} ‘recognize’, \It{gåpptjåt} ‘close’, \It{hållåt} ‘say’ and \It{låhkåt} ‘read’ are also likely candidates for Class IIb. Verbs in Class IIb all have \It{å} as the initial stem vowel while the class marking post-stem vowel is also \It{å}, just as the nouns in noun Class Id (also marked by \It{å}). %It is not clear whether this vowel harmony is triggered by the initial or second \It{å}.
\begin{table}\centering
\caption{The inflectional paradigm for the Class IIa verb \It{bassat} ‘wash’}\label{washParadigm}
\resizebox{1\linewidth}{!} {
\begin{tabular}{|cc||x{80pt}|x{80pt}|p{80pt}|}\hline
\It{tense/}			&			&\MC{3}{c|}{\It{number}}	\\
\It{mood}			&\It{person}	&\MC{1}{c}{\SGs}	&\MC{1}{c}{\DUs}		&\Xp{\PLs}	\\\dline
%PRESENT
\MR{3}{*}{\PRSs}	&1\superS{st}	&bas-a-v	&biss-i-n			&\Xp{bass-a-p}		\\\cline{3-5}
				&2\superS{nd}	&bas-a	&bass-a-bähten	&\Xp{bass-a-bähtet}	\\\cline{3-5}
				&3\superS{rd}	&bass-a	&bass-a-ba		&\Xp{biss-e}		\\\dline%\cline{2-5}
%PAST
\MR{3}{*}{\PSTs}	&1\superS{st}	&biss-i-v	&bas-a-jmen		&\Xp{bas-a-jmä}	\\\cline{3-5}
				&2\superS{nd}	&biss-e	&bas-a-jden		&\Xp{bas-a-jdä}		\\\cline{3-5}
				&3\superS{rd}	&bas-a-j	&bas-a-jga		&\Xp{biss-i-n}		\\\dline%\cline{2-5}
%IMPERATIVE
\IMPs			&2\superS{nd}	&bas-a	&bass-e-n			&\Xp{bess-i-t}		\\\hline%\cline{3-5}
\MC{5}{l}{\It{non-finite verb forms:}}\\\hline
\MC{1}{|c}{\INFs}	&\MC{2}{c|}{bass-a-t}	&\MC{1}{c}{\CONNEGs}&bas-a			\\\hline
\MC{1}{|c}{\PRFs}	&\MC{2}{c|}{bass-a-m}	&\MC{2}{c}{}\\\cline{1-3}
\end{tabular}}
\end{table}
\begin{table}\centering
\caption{The inflectional paradigm for the Class IIb verb \It{bårråt} ‘eat’}\label{eatParadigm}
\resizebox{1\linewidth}{!} {
\begin{tabular}{|cc||x{80pt}|x{80pt}|p{80pt}|}\hline
\It{tense/}			&			&\MC{3}{c|}{\It{number}}	\\
\It{mood}			&\It{person}	&\MC{1}{c}{\SGs}	&\MC{1}{c}{\DUs}		&\Xp{\PLs}	\\\dline
%PRESENT
\MR{3}{*}{\PRSs}	&1\superS{st}	&bår-å-v	&burr-e-n			&\Xp{bårr-å-p}		\\\cline{3-5}
				&2\superS{nd}	&bår-å	&bårr-å-bähtin		&\Xp{bårr-å-bähtet}	\\\cline{3-5}
				&3\superS{rd}	&bårr-a	&bårr-å-ba		&\Xp{burr-e}		\\\dline%\cline{2-5}
%PAST
\MR{3}{*}{\PSTs}	&1\superS{st}	&burr-e-v	&bår-å-jmen		&\Xp{bår-å-jme}	\\\cline{3-5}
				&2\superS{nd}	&burr-e	&bår-å-jden		&\Xp{bår-å-jde}		\\\cline{3-5}
				&3\superS{rd}	&bår-å-j	&bår-å-jga			&\Xp{burr-e-n}		\\\dline%\cline{2-5}
%IMPERATIVE
\IMPs			&2\superS{nd}	&bår-å	&bårr-e-n			&\Xp{burr-i-t}		\\\hline%\cline{3-5}
\MC{5}{l}{\It{non-finite verb forms:}}\\\hline
\MC{1}{|c}{\INFs}	&\MC{2}{c|}{bårr-å-t}			&\MC{1}{c}{\CONNEGs}&bår-å		\\\hline
\MC{1}{|c}{\PRFs}	&\MC{2}{c|}{bårr-å-m}	&\MC{2}{c}{}\\\cline{1-3}
\end{tabular}}
\end{table}

Table \vref{VclassIIsummaryTable} summarizes the gradation pattern, class suffixes and locations for vowel harmony for Class II verbs; here, \Bf{V} stands for the vowel which marks the infinitive form (\It{a} for Class IIa and \It{å} for Class IIb). Note that umlaut alternations align with consonant gradation alternations. 
\begin{sidewaystable}[p]\centering
%\begin{table}\centering
\caption{Summary of Class II verb paradigm features}\label{VclassIIsummaryTable}
\begin{tabular}{|cc|| c|c|c|| c|c|c|| c|c|c|}\hline
				&			&\MC{9}{c|}{\It{number}}	\\
\It{tense/}			&			&\MC{3}{c||}{\SGs}					&\MC{3}{c||}{\DUs}					&\MC{3}{c|}{\PLs}	\\%\dline
\It{mood}			&\It{person}	&\MC{1}{c}{\It{C-grad}}&\MC{1}{c}{\It{cl.\,sx.}}&\It{VH}&\MC{1}{c}{\It{C-grad}}&\MC{1}{c}{\It{cl.\,sx.}}&\It{VH}	&\MC{1}{c}{\It{C-grad}}&\MC{1}{c}{\It{cl.\,sx.}}	&\It{VH}	\\\cline{1-11}%\cline{1-8}
%PRESENT
\MR{3}{*}{\PRSs}	&1\superS{st}	&wk			&V			&		&str			&i			&\CH	&str			&V			&	\\\cline{3-11}
				&2\superS{nd}	&wk			&V			&		&str			&V			&		&str			&V			&	\\\cline{3-11}
				&3\superS{rd}	&str			&a			&		&str			&V			&		&str			&e			&\CH	\\\hline%\cline{3-8}
%PAST
\MR{3}{*}{\PSTs}	&1\superS{st}	&str			&i			&\CH	&wk			&V			&		&wk			&V			&	\\\cline{3-11}
				&2\superS{nd}	&str			&e			&\CH	&wk			&V			&		&wk			&i			&	\\\cline{3-11}
				&3\superS{rd}	&wk			&V			&		&wk			&V			&		&str			&i			&\CH	\\\hline%\cline{3-8}
%%IMPERATIVE
\IMPs			&2\superS{nd}	&wk			&V			&		&str			&e			&		&str			&i			&\CH	\\\hline%\cline{3-8}
\MC{8}{l}{\It{non-finite verb forms:}}\\\hline
\MC{2}{|r||}{\INFs}				&str			&V			&		&\MC{3}{r||}{\CONNEGs}				&wk			&V			&	\\\hline%\cline{2-8}
\MC{2}{|r||}{\PRFs}				&str			&V			&		&\MC{6}{c}{}		\\\cline{1-5}
\end{tabular}
%\caption{Summary of Class II verb paradigm features; here, \Bf{V} stands for \It{a} in Class IIa verbs, and for \It{å} in Class IIb verbs}\label{VclassIIsummaryTable}
%\end{table}
\end{sidewaystable}
%\afterpage{\clearpage}%\clearpage

%Examples:\\
%\begin{tabular}{|l|c|c|l|l|l|l|}\hline
%\INFs	&\it c-grad	&\it umlaut&2\SGs.\PRSs	& \CONNEGs	&1\DUs.\PRSs	&\it gloss	\\\dline
%bassat	&ss-s	&a-i		&basa		&basa		&bissin		&\it wash	\\\hline
%juhkat	&hk-g	&-		&juga		&juga		&juhken		&\it drink	\\\hline
%tjájbmat	&jbm-jm	&á-i		&tjájma		&tjájma		&tjijbmen		&\it laught	\\\hline
%sávvat	&vv-v	&á-ä		&sáva		&sáva		&sävvin		&\it wish	\\\hline
%dáhkat	&hk-g	&á-i		&dága		&dága		&dihkin		&\it do	\\\hline
%ballat	&ll-l		&a-i		&bala		&bala		&billin		&\it become scared	\\\hline
%ravvgat	&vv-v	&a-u		&ravga		&ravga		&ruvvgin		&\it dig	\\\hline
%gähttjat	&httj-htj	&ä-ie-i	&giehtja		&giehtja		&gihtjin		&\it look	\\\hline
%\end{tabular}


\FloatBarrier
\subsection{Class III}\label{VclassIII}
Verbs in Class III are characterized as follows:
\begin{itemize}
\item{a bisyllabic infinitive form with a class suffix \It{-e}}
\item{potentially subject to consonant gradation, umlaut and vowel harmony}
\end{itemize}
%Verbs in Class III can be subject to consonant gradation, umlaut and vowel harmony, and are characterized by the class marking suffix \It{-e} in the \INF\ form. 
Twelve forms are subject to vowel harmony (the same six as for Class II verbs, plus six more). 
%This class is further defined by having the typical ‘str-wk’ consonant gradation pattern. 
%The \INF\ form is bisyllabic. 
The verb \It{bassat} ‘wash’ is provided in Table \vref{fryParadigm} as an example for a Class III verb. Other examples of Class III verbs include: \It{vádtset} ‘go’, \It{adnet} ‘have, possess’, \It{diehtet} ‘know’, \It{båhtet} ‘come’, \It{buälldet} ‘ignite, burn’, and \It{máhttet} ‘can’. 
\begin{table}\centering
\caption{The inflectional paradigm for the Class III verb \It{basset} ‘fry’}\label{fryParadigm}
\resizebox{1\linewidth}{!} {
\begin{tabular}{|cc||x{80pt}|x{80pt}|p{80pt}|}\hline
\It{tense/}			&			&\MC{3}{c|}{\It{number}}	\\
\It{mood}			&\It{person}	&\MC{1}{c}{\SGs}	&\MC{1}{c}{\DUs}		&\Xp{\PLs}	\\\dline
%PRESENT
\MR{3}{*}{\PRSs}	&1\superS{st}	&bas-á-v	&biss-i-n			&\Xp{bass-e-p}		\\\cline{3-5}
				&2\superS{nd}	&bas-á	&bass-e-bähten	&\Xp{bass-e-bähtet}	\\\cline{3-5}
				&3\superS{rd}	&bass-a	&bass-e-ba		&\Xp{biss-e}		\\\dline%\cline{2-5}
%PAST
\MR{3}{*}{\PSTs}	&1\superS{st}	&biss-i-v	&bis-i-jmen		&\Xp{bis-i-jmä}		\\\cline{3-5}
				&2\superS{nd}	&biss-e	&bis-i-jden		&\Xp{bis-i-jdä}		\\\cline{3-5}
				&3\superS{rd}	&bis-i-j	&bis-i-jga			&\Xp{biss-i-n}		\\\dline%\cline{2-5}
%IMPERATIVE
\IMPs			&2\superS{nd}	&bas-e	&bass-e-n			&\Xp{biss-i-t}		\\\hline%\cline{3-5}
\MC{5}{l}{\It{non-finite verb forms:}}\\\hline
\MC{1}{|c}{\INFs}	&\MC{2}{c|}{bass-e-t}	&\MC{1}{c}{\CONNEGs}&bas-e			\\\hline
\MC{1}{|c}{\PRFs}	&\MC{2}{c|}{bass-a-m}	&\MC{2}{c}{}\\\cline{1-3}
\end{tabular}}
\end{table}

When the consonant center of a Class III stem consists of a single segment in the 1\SGs.\PRSs\ and 2\SGs.\PRSs\ forms and the V1 vowel is neither \It{á} nor \It{ua/uä/uo}, the class marking vowel is \It{á} instead of \It{a}, as illustrated by the verb \It{basset} in Table \vref{fryParadigm}. 

Table \vref{VclassIIIsummaryTable} summarizes the gradation pattern, class suffixes and locations for vowel harmony for Class III verbs. Note that umlaut alternations align with consonant gradation alternations. 
\begin{sidewaystable}[p]\centering
\caption{Summary of Class III verb paradigm features}\label{VclassIIIsummaryTable}
%\begin{table}\centering
\begin{tabular}{|cc|| c|c|c|| c|c|c|| c|c|c|}\hline
				&			&\MC{9}{c|}{\It{number}}	\\
\It{tense/}			&			&\MC{3}{c||}{\SGs}					&\MC{3}{c||}{\DUs}					&\MC{3}{c|}{\PLs}	\\%\dline
\It{mood}			&\It{person}	&\MC{1}{c}{\It{C-grad}}&\MC{1}{c}{\It{cl.\,sx.}}&\It{VH}&\MC{1}{c}{\It{C-grad}}&\MC{1}{c}{\It{cl.\,sx.}}&\It{VH}&\MC{1}{c}{\It{C-grad}}&\MC{1}{c}{\It{cl.\,sx.}}	&\It{VH}	\\\cline{1-11}%\cline{1-8}
%PRESENT
\MR{3}{*}{\PRSs}	&1\superS{st}	&wk			&a/á			&		&str			&i			&\CH	&str			&e			&	\\\cline{3-11}
				&2\superS{nd}	&wk			&a/á			&		&str			&e			&		&str			&e			&	\\\cline{3-11}
				&3\superS{rd}	&str			&a			&		&str			&e			&		&str			&e			&\CH	\\\hline%\cline{3-8}
%PAST
\MR{3}{*}{\PSTs}	&1\superS{st}	&str			&i			&\CH	&wk			&i			&\CH	&wk			&i			&\CH	\\\cline{3-11}
				&2\superS{nd}	&str			&e			&\CH	&wk			&i			&\CH	&wk			&i			&\CH	\\\cline{3-11}
				&3\superS{rd}	&wk			&i			&\CH	&wk			&i			&\CH	&str			&i			&\CH	\\\hline%\cline{3-8}
%%IMPERATIVE
\IMPs			&2\superS{nd}	&wk			&e			&		&str			&e			&		&str			&i			&\CH	\\\hline%\cline{3-8}
\MC{8}{l}{\It{non-finite verb forms:}}\\\hline
\MC{2}{|r||}{\INFs}				&str			&e			&		&\MC{3}{r||}{\CONNEGs}				&wk			&e			&	\\\hline%\cline{2-8}
\MC{2}{|r||}{\PRFs}				&str			&a			&		&\MC{6}{c}{}		\\\cline{1-5}
\end{tabular}
\end{sidewaystable}
%\end{table}

\FloatBarrier


\subsection{Class IV}\label{VclassIV}
%\subsection{Class IV (preliminary)}\label{VclassIV}
%At least one further inflectional class of verbs with a bisyllabic \INF\ form exists as well. It is %, however the corpus does not provide sufficient data on enough verbs to be sure about this class, so it is particularly preliminary at this point. Class V verbs are 
Class IV verbs are characterized by: 
\begin{itemize}
\item{a bisyllabic infinitive form}
\item{no allomorphic variation in the stem and in the class marker}
%\item{a lack of consonant gradation, umlaut and vowel harmony}
%\item{the class marking suffix lacks allomorphy within a paradigm}
\item{deviant person/number suffixes with a \It{-j-} element}
%\item{a bisyllabic stem allomorph ending in \It{-uj}} % for 3\SGs.\PRSs, 1\SGs.\PSTs, 2\SGs.\PSTs, 1\DUs.\PRSs, 3\PLs.\PRSs\ and 3\PLs.\PSTs\ forms. %\footnote{Inflectional noun classes III and IV also feature an ‘inverted’ consonant gradation pattern.} 
\end{itemize}
%Other verbs with a bisyllabic infinitive form only have monosyllabic stem allomorphs; Class IV is characterized by having a 
%bisyllabic stem allomorph, which is found in the 3\SGs.\PRSs, 1\SGs.\PSTs, 2\SGs.\PSTs, 1\DUs.\PRSs, 3\PLs.\PRSs\ and 3\PLs.\PSTs\ forms; these thus have one more syllable than the corresponding forms for other classes with a bisyllabic infinitive form. 
The stem and the class marking suffix are consistent in all forms throughout a paradigm, i.e., there is no allomorphy in the stem or class marker. 
The person/number suffixes for \Sc{3sg.prs}, \Sc{1du.prs}, \Sc{3pl.prs}, \Sc{1sg.pst}, \Sc{2sg.pst} and \Sc{3pl.pst} deviate from the corresponding person/number suffixes in other verb classes in featuring an initial \It{-j-} element. 
A nearly complete paradigm for the verb \It{välldut} ‘marry’ is provided in Table \vref{marryParadigm}.
\begin{table}\centering
\caption{The inflectional paradigm for the Class IV verb \It{välldut} ‘marry’}\label{marryParadigm}
\resizebox{1\linewidth}{!} {
\begin{tabular}{|cc||x{80pt}|x{80pt}|p{80pt}|}\hline
\It{tense/}			&			&\MC{3}{c|}{\It{number}}	\\
\It{mood}			&\It{person}	&\SGs		&\DUs			&\Xp{\PLs}	\\\dline
%PRESENT
\MR{3}{*}{\PRSs}	&1\superS{st}	&välld-u-v			&välld-u-jin		&\Xp{välld-u-p}		\\\cline{3-5}
%\MR{3}{*}{\PRSs}	&1\superS{st}	&välld-u-v			&\TabGray vällduj-i-n	&\Xp{välld-u-p}		\\\cline{3-5}
				&2\superS{nd}	&välld-u			&välld-u-bähten		&\Xp{välld-u-bähtet}	\\\cline{3-5}
				&3\superS{rd}	&välld-u-ja		&välld-u-ba		&\Xp{välld-u-je}		\\\dline%\cline{2-5}
%				&3\superS{rd}	&\TabGray vällduj-a-	&välld-u-bah		&\Xp{\TabGray vällduj-e-}		\\\dline%\cline{2-5}
%PAST
\MR{3}{*}{\PSTs}	&1\superS{st}	&välld-u-jiv		&välld-u-jmen		&\Xp{välld-u-jme}	\\\cline{3-5}
				&2\superS{nd}	&välld-u-je		&välld-u-jden		&\Xp{välld-u-jde}		\\\cline{3-5}
				&3\superS{rd}	&välld-u-j			&välld-u-jga		&\Xp{välld-u-jin}		\\\dline%\cline{2-5}
%\MR{3}{*}{\PSTs}	&1\superS{st}	&\TabGray vällduj-i-v	&välld-u-jmin		&\Xp{välld-u-jmä}	\\\cline{3-5}
%				&2\superS{nd}	&\TabGray vällduj-e-	&välld-u-jdin		&\Xp{välld-u-jdä}		\\\cline{3-5}
%				&3\superS{rd}	&välld-u-j			&välld-u-jgah		&\Xp{\TabGray vällduj-i-n}		\\\dline%\cline{2-5}
%IMPERATIVE
\IMPs			&2\superS{nd}	&?		&?			&\Xp{?}		\\\hline%\cline{3-5}
\MC{5}{l}{\It{non-finite verb forms:}}\\\hline
\MC{1}{|c}{\INFs}	&\MC{2}{c|}{välld-u-t}		&\MC{1}{c}{\CONNEGs}&välld-u				\\\hline
\MC{1}{|c}{\PRFs}	&\MC{2}{c|}{välld-u-m}	&\MC{2}{c}{}\\\cline{1-3}
\end{tabular}}
\end{table}

However, the data in the corpus are not nearly sufficient to provide much more than the paradigm in Table \vref{marryParadigm}. %, and a very partial paradigm for {\It{årrat}}\footnote{The verb \It{årrat} ‘fall asleep’ should not be confused with the Class III verb \It{årret} ‘sleep’ (which is exactly what happened in several elicitation sessions).} 
%‘fall asleep’, which is provided in Table \vref{fallAsleepParadigm}. 
%both suspected to be in this class. %Attempts to elicit the verbs purportedly in such a class failed. %to elicit the  consisted of either contradictory results or 
%A very partial paradigm for the verb 
Class IV is likely a relatively small class of verbs; other potential candidates are \mbox{\It{årrat}}\footnote{The verb \It{årrat} ‘fall asleep’ should not be confused with the Class III verb {\It{årret}} ‘sleep’.} %(which is exactly what happened in several elicitation sessions).} 
‘fall sleep’, \It{ádnot} ‘request’ and \It{tjerrot} ‘cry’. \citet[154]{Lehtiranta1992} includes a paradigm for \It{tjerrot}, which appears to patten like \It{välldut}.\footnote{But even the paradigm in \citet[154]{Lehtiranta1992} for \It{tjerrot} is marked by inconsistent forms across dialects; Furthermore, one of my main consultants from the northern side of the \PS\ territory stated that her dialect does not use the lexeme \It{tjerrot}, but instead \It{vállut} ‘cry’.} 
The class marking vowel in the infinitive form is thus not restricted to the \It{-u-} indicated in Table \vref{marryParadigm}. 
%Ultimately, more research is needed to come to a better understanding of Class IV verbs. 

Table \vref{VclassIVsummaryTable} summarizes the preliminary class suffix pattern for Class IV verbs, as well as the presence of a person/number suffix which deviates from the corresponding person/number suffixes in other verb classes (abbreviated as \mbox{\It{dev. agr. sx.}} for ‘deviant agreement suffix’ in the table) %locations for a bisyllabic stem allomorph in Class IV verbs, 
This is based on the paradigm for \It{välldut} in Table \vref{marryParadigm} and the paradigm for \It{tjerrot} provided in \citet[154]{Lehtiranta1992}.\footnote{Note the difference in orthographic forms between those used here, with \It{tjerrot} for the infinitive form, and the forms used in \citet{Lehtiranta1992}, with \It{tjierˈrut} for the infinitive form.} %Note that, umlaut alternations align with consonant gradation alternations. 
\begin{table}\centering
\caption{Preliminary summary of Class IV verb paradigm features}\label{VclassIVsummaryTable}
\begin{tabular}{|cc|| c|c|| c|c|| c|c|}\hline
				&			&\MC{6}{c|}{\It{number}}	\\
				&			&\MC{2}{c||}{\SGs}			&\MC{2}{c||}{\DUs}			&\MC{2}{c|}{\PLs}	\\%\dline
\It{tense/}			&			&\MC{1}{c}{}		&\It{dev.}		&\MC{1}{c}{}	&\It{dev.}	&\MC{1}{c}{}		&\It{dev.}	\\
\It{mood}			&\It{person}	&\MC{1}{c}{\It{cl.\,sx.}}&\It{agr. sx.}	&\MC{1}{c}{\It{cl.\,sx.}}&\It{agr. sx.}	&\MC{1}{c}{\It{cl.\,sx.}}		&\It{agr. sx.}\\\cline{1-8}%\cline{1-8}
%PRESENT
\MR{3}{*}{\PRSs}	&1\superS{st}	&V			&			&V			&\CH		&V			&		\\\cline{3-8}
				&2\superS{nd}	&V			&			&V			&			&V			&		\\\cline{3-8}
				&3\superS{rd}	&V			&\CH		&V			&			&V			&\CH	\\\hline%\cline{3-8}
%PAST
\MR{3}{*}{\PSTs}	&1\superS{st}	&V			&\CH		&V			&			&V			&		\\\cline{3-8}
				&2\superS{nd}	&V			&\CH		&V			&			&V			&		\\\cline{3-8}
				&3\superS{rd}	&V			&			&V			&			&V			&\CH	\\\hline%\cline{3-8}
%%IMPERATIVE
\IMPs			&2\superS{nd}	&?			&?			&?			&?			&?			&?		\\\hline%\cline{3-8}
\MC{8}{l}{\It{non-finite verb forms:}}\\\hline
\MC{2}{|r||}{\INFs}				&V			&			&\MC{2}{r||}{\CONNEGs}		&V			&		\\\hline%\cline{2-8}
\MC{2}{|r||}{\PRFs}				&V			&			&\MC{4}{c}{}		\\\cline{1-4}
\end{tabular}
\end{table}

\FloatBarrier

\subsection{Class V}\label{VclassV}
Verbs in Class V are characterized by 
\begin{itemize}
\item{a trisyllabic infinitive form with the class marking suffix \It{-i}}
\item{absence of consonant gradation, umlaut and vowel harmony.} 
\end{itemize}
Many Class V verbs are derived verbs based on a bisyllabic verb (cf. \It{gullat} ‘hear’ and \It{gulladit} ‘be in touch’ (lit.: let someone hear from you)).\footnote{Because many derived verbs are in Class V, the semantic aspects accompanying the relevant derivational suffixes align in Class V, but their membership in Class V is due to their \mbox{(morpho-)phonemic} structure, not their semantics. Cf. §\,\ref{vDerivation} on verbal derivation.} 
%\marginpar{ALL derived verbs?} derived verbs are in Class V. 
The paradigm in Table \vref{speakParadigm} provides an example for the verb \It{ságastit} ‘speak’; 
other Class V verbs include \It{bargatjit} ‘work a little’, \It{gatjadit} ‘ask’, \It{gullalit} ‘listen’, \It{málestit} ‘cook, boil’, \It{gávnadit} ‘meet’ and \It{leradit} ‘teach’. 
Table \vref{VclassVsummaryTable} summarizes the gradation pattern and class suffixes for Class V verbs. %Note that, umlaut alternations align with consonant gradation alternations. 
\begin{table}\centering
\caption{The inflectional paradigm for the Class V verb \It{ságastit} ‘speak’}\label{speakParadigm}
\resizebox{1\linewidth}{!} {
\begin{tabular}{|cc||x{80pt}|x{80pt}|p{80pt}|}\hline
\It{tense/}			&			&\MC{3}{c|}{\It{number}}	\\
\It{mood}			&\It{person}	&\SGs		&\DUs			&\Xp{\PLs}		\\\dline
%PRESENT
\MR{3}{*}{\PRSs}	&1\superS{st}	&ságast-a-v	&ságast-e-n		&\Xp{ságast-e-p}	\\\cline{3-5}
				&2\superS{nd}	&ságast-a		&ságast-ä-hpen	&\Xp{ságast-e-hpit}	\\\cline{3-5}
				&3\superS{rd}	&ságast-a		&ságast-ä-ba		&\Xp{ságast-e}		\\\dline%\cline{2-5}
%PAST
\MR{3}{*}{\PSTs}	&1\superS{st}	&ságast-i-jiv	&ságast-i-jmen		&\Xp{ságast-i-jme}	\\\cline{3-5}
				&2\superS{nd}	&ságast-i-je	&ságast-i-jden		&\Xp{ságast-i-jde}	\\\cline{3-5}
				&3\superS{rd}	&ságast-i-j	&ságast-i-jga		&\Xp{ságast-e-n}	\\\dline%\cline{2-5}
%IMPERATIVE
\IMPs			&2\superS{nd}	&ságast-e		&ságast-ä-hten		&\Xp{ságast-ä-htet}	\\\hline%\cline{3-5}
\MC{5}{l}{\It{non-finite verb forms:}}\\\hline
\MC{1}{|c}{\INFs}	&\MC{2}{c|}{ságast-i-t}		&\MC{1}{c}{\CONNEGs}&ságast-e		\\\hline
\MC{1}{|c}{\PRFs}	&\MC{2}{c|}{ságast-a-m}	&\MC{2}{c}{}\\\cline{1-3}
\end{tabular}}
\end{table}

\renewcommand{\Xp}[1]{\MC{1}{x{60pt}|}{#1}}%JW: added to allow centering in final set-length column in verb paradigm tables; redefined here
\begin{table}\centering
\caption[Summary of Class V verb paradigm features]{Summary of Class V verb paradigm features (only class suffix)}\label{VclassVsummaryTable}
\begin{tabular}{|cc|| x{60pt}|| x{60pt}|| c|}\hline
				&			&\MC{3}{c|}{\It{number}}	\\
\It{tense/}			&			&\MC{1}{c||}{\SGs}	&\MC{1}{c||}{\DUs}	&\MC{1}{c|}{\PLs}	\\%\dline
\It{mood}			&\It{person}	&\It{cl.\,sx.}		&\It{cl.\,sx.}		&\Xp{\It{cl.\,sx.}}	\\\cline{1-5}%\cline{1-8}
%PRESENT
\MR{3}{*}{\PRSs}	&1\superS{st}	&a			&e			&\Xp{e}		\\\cline{3-5}
				&2\superS{nd}	&a			&ä			&\Xp{e}		\\\cline{3-5}
				&3\superS{rd}	&a			&ä			&\Xp{e}		\\\hline%\cline{3-8}
%PAST
\MR{3}{*}{\PSTs}	&1\superS{st}	&i			&i			&\Xp{i}		\\\cline{3-5}
				&2\superS{nd}	&i			&i			&\Xp{i}		\\\cline{3-5}
				&3\superS{rd}	&i			&i			&\Xp{e}		\\\hline%\cline{3-8}
%%IMPERATIVE
\IMPs			&2\superS{nd}	&e			&ä			&\Xp{ä}		\\\hline%\cline{3-8}
\MC{5}{l}{\It{non-finite verb forms:}}\\\hline
\MC{2}{|r||}{\INFs}				&i			&\MC{1}{r||}{\CONNEGs}		&\Xp{e}		\\\hline%\cline{2-8}
\MC{2}{|r||}{\PRFs}				&a			&\MC{2}{c}{}		\\\cline{1-3}
\end{tabular}
\end{table}
%\clearpage
\FloatBarrier


\subsection{Other possible verb classes}\label{otherVerbClasses}
The data in the corpus are unfortunately not sufficient to be entirely confident concerning the five inflectional classes for verbs proposed here. With this in mind, the data concerning several verbs seems unusual, but also contradictory and inconsistent. Specifically, limited data on the verbs \It{årret} ‘sleep’, \It{årrat} ‘fall asleep’  %\It{åddet}\TILDE\It{årret} ‘sleep’, \It{åddat}\TILDE\It{årrat} ‘fall asleep’ 
and \It{ádnot} ‘request’ %, \It{tjerrot} ‘cry’ 
exists in the corpus indicating that these may belong to Class IV or some subset of Class IV verbs. Furthermore, a number of verbs with bisyllabic infinitive forms marked by \It{-i-} as a post-stem class-marking suffix exist in the data in the wordlist compiled by the \WLP\ (cf. §\,\ref{orthography}); however, in many cases, it seems that these verbs in fact belong to Class III, and the \It{-i-} class marker is simply an inconsistent spelling of \It{e}, as the realizations of /i/ and /e/ in unstressed syllables are more centralized, and thus easily confusable, particularly when applying what are otherwise Swedish graphemes representing more distinctly front Swedish vowels. For instance, the verb \It{virrtit} ‘must’ should perhaps be spelled \It{virrtet} and likely belongs to Class III. More data on this and other bisyllabic verbs with the \It{-i-} spelling need to be gathered to determine whether another inflectional class exists, or if these are only subclasses for Class IV and perhaps Class I, II or III. 



\subsection{The verb \It{årrot} ‘be’}\label{theCopulaVerb}
The verb \It{årrot} ‘be’ can be used as a copula (cf. §\,\ref{copulaClauses}) or as an auxiliary (cf. §\,\ref{auxV}); its paradigm is presented in Table \vref{beParadigm}. %on page \pageref{beParadigm}. 
\renewcommand{\Xp}[1]{\MC{1}{x{80pt}|}{#1}}%JW: added to allow centering in final set-length column in verb paradigm tables; redefined here
\begin{table}\centering
\caption{The inflectional paradigm for the verb \It{årrot} ‘be’}\label{beParadigm}
\resizebox{1\linewidth}{!} {
\begin{tabular}{|cc||x{80pt}|x{80pt}|p{80pt}|}\hline
\It{tense/}			&			&\MC{3}{c|}{\It{number}}	\\
\It{mood}			&\It{person}	&\MC{1}{c}{\SGs}	&\MC{1}{c}{\DUs}		&\Xp{\PLs}	\\\dline
%PRESENT
\MR{3}{*}{\PRSs}	&1\superS{st}	&lev		&lin				&\Xp{lep}		\\\cline{3-5}
				&2\superS{nd}	&lä/’l	&lähpen			&\Xp{lehpet}	\\\cline{3-5}
				&3\superS{rd}	&lä/’l	&lähpa			&\Xp{lea/’l}		\\\dline%\cline{2-5}
%PAST
\MR{3}{*}{\PSTs}	&1\superS{st}	&lidjiv	&lijmen			&\Xp{lijme}	\\\cline{3-5}
				&2\superS{nd}	&lidje	&lijden			&\Xp{lijde}		\\\cline{3-5}
				&3\superS{rd}	&lij		&lijga			&\Xp{lidjen}		\\\dline%\cline{2-5}
%IMPERATIVE
\IMPs			&2\superS{nd}	&?		&?				&\Xp{?}		\\\hline%\cline{3-5}
\MC{5}{l}{\It{other nonfinite verb forms:}}\\\hline
\MC{1}{|c}{\INFs}	&\MC{2}{c|}{årrot}			&\MC{1}{c}{\CONNEGs}&lä			\\\hline
\MC{1}{|c}{\PRFs}	&\MC{2}{c|}{urrum/lam}		&\MC{2}{c}{}			\\\cline{1-3}%\hline
\end{tabular}}
\end{table}

It is an unusual verb in a number of ways: \begin{itemize}
\item It is suppletive, featuring the two stems \It{årr-} and \It{l-}. 
\item Many of the \It{l-} stem forms are monosyllabic. 
\item The 2\SGs.\PRSs, 3\SGs.\PRSs\ and 3\PLs.\PRSs\ forms can be shortened to \It{’l} and encliticized onto the preceding word of an utterance if the preceding word has an open final syllable, as in \REF{cliticBE}.
%\end{itemize}
\ea\label{cliticBE}
\glll	duvne'l aj ‘svála’ båkså\\
	duvne=l aj svála båkså\\
	\Sc{2sg.iness}=be\BS\Sc{3pl.prs} also arctic.fox\BS\Sc{gen.sg} pant\BS\Sc{nom.pl}\\
\Transl{you also have \It{Fjällräven}\footnotemark\ pants on}{}	\Corpus{090519}{073}
\z
\footnotetext{\It{Fjällräven} refers to a Swedish clothing company named after ‘the arctic fox’ (lat.: \It{Vulpes lagopus}); in \REF{cliticBE}, the speaker literally translates the company’s name into \PS.}
% is the Swedish word for ‘arctic fox’ (lat.: \It{Vulpes lagopus}), but in \REF{cliticBE}, this refers to a Swedish clothing company by the same name; here, the speaker literally translates the Swedish name into \PS.}
%\begin{itemize}
\item The 1\SGs.\PSTs\ form \It{lidjiv} is often shortened to \It{lijiv}, and the 3\PLs.\PSTs\ form \It{lidjin} is often shortened to \It{lin}. 
%\marginpar{does it really make sense to posit/draw morpheme boundaries for the copula? or just mention the resemblance to regular verb morphology?}
%Table \vref{beParadigm} on page \pageref{beParadigm} presents the paradigm for the copula verb. 
\item The infinitive and perfect forms are the only forms in this basic paradigm which use the \It{årr-} stem, %\marginpar{the only form is \INFs?} 
which is homophonous (and cognate) with the verb \It{årrot} ‘reside, live’. 
\item Finally, the verb \It{årrot} ‘be’ is unique in having a contracted connegative and perfect form: \It{lam} ‘be-\Sc{prf}\BS\Sc{conneg}’ is a shortened form of \It{lä} ‘be\BS\Sc{conneg}’ and \It{urrum} ‘be-\Sc{prf}’, and is thus only used in conjunction with the verb of negation, as illustrated by the example in \REF{beVerbPRFex2}. 
%\end{itemize}
%\ea\label{beVerbPRFex1}
%\glll	ja danne muv äddne lij, lä årrom ja riegadam\\
%	ja danne muv äddne li-j lä årro-m ja riegada-m\\
%	and here \Sc{1sg.gen} mother\BS\Sc{nom.sg} be-\Sc{3sg.pst} be\BS\Sc{3sg.prs} be-\Sc{prf} and be\_born-\Sc{prf}\\
%\Transl{and here my mother was, has lived and was born’	\Corpus{100310b.050}
%%\glll	ja dán Álesgiehtjen lä buorak årrom gu liv mån mánná\\
%%	ja dá-n Álesgiehtje-n lä buorak årro-m gu li-v mån mánná\\
%%	and \Sc{dem.prox}-\Sc{iness.sg} Västerfjäll-\Sc{iness.sg} be\BS\Sc{3sg.prs} good be-\Sc{prf} when be-\Sc{1sg.prs?) \Sc{1sg.nom} child\BS\Sc{nom.sg}\\
%%\Transl{and it was good to be in this here Västerfjäll when I ?am? a child’	\Corpus{100404.141}
\ea\label{beVerbPRFex2}
\glll	men iv lam dä månnå del skålån giesen\\
	men i-v l-am dä månnå del skålå-n giese-n\\
	but \Sc{neg}-\Sc{1sg.prs} be-\Sc{prf}\BS\Sc{conneg} then \Sc{1sg.nom} then school-\Sc{iness.sg} summer-\Sc{iness.sg}\\\nopagebreak
\Transl{but I haven’t been in school during the summer}{}	\Corpus{080924}{622}
\z\end{itemize}
%\clearpage
\FloatBarrier


\subsection{The negation verb}\label{theNegationVerb}
The negation verb is unique because it only exists as a finite verb; thus there are no non-finite forms. % (such as \INF\ or \CONNEG). 
%\marginpar{does it really make sense to posit/draw morpheme boundaries for the neg-verb? or just mention the resemblance to regular verb morphology?}
Table \vref{NEGParadigm} %on page \pageref{NEGParadigm} 
presents the paradigm for the negation verb. 
\begin{table}\centering
\caption{The inflectional paradigm for the negation verb}\label{NEGParadigm}
\resizebox{1\linewidth}{!} {
\begin{tabular}{|cc||x{80pt}|x{80pt}|p{80pt}|}\hline
\It{tense/}			&			&\MC{3}{c|}{\It{number}}	\\
\It{mood}			&\It{person}	&\SGs	&\DUs		&\Xp{\PLs}	\\\dline
%PRESENT
\MR{3}{*}{\PRSs}	&1\superS{st}	&iv		&en			&\Xp{ep}		\\\cline{3-5}
				&2\superS{nd}	&i		&ehpen		&\Xp{ehpet}	\\\cline{3-5}
				&3\superS{rd}	&ij		&eba			&\Xp{eh}		\\\dline%\cline{2-5}
%PAST
\MR{3}{*}{\PSTs}	&1\superS{st}	&ittjiv	&ettjijmen		&\Xp{ittjijme}	\\\cline{3-5}
				&2\superS{nd}	&ittje		&ettjijden		&\Xp{ittjijde}	\\\cline{3-5}
				&3\superS{rd}	&ittjij		&ettjijga		&\Xp{ittjin}		\\\dline%\cline{2-5}
%IMPERATIVE
\IMPs			&2\superS{nd}	&ele/ilu	&ellen/illun	&\Xp{ellet/illut}	\\\hline%\cline{3-5}
%\MC{5}{l}{\It{non-finite verb forms:}}\\\hline
%\MC{1}{|c}{\INFs}	&\MC{2}{c|}{-}			&\MC{1}{c}{\CONNEGs}&-			\\\hline
\end{tabular}}
\end{table}
Concerning the imperative forms, both forms indicated for each number slot are attested in the corpus. 
%\clearpage



\subsection{Summary of verb classes}\label{verbInflectionalClassesSummary}
Table \vref{verbClassExamples} is provided to facilitate a cross-class comparison of paradigms with examples from the various inflectional classes for verbs, as well as the verb \It{årrot} ‘be’ and the negation verb. While the whole paradigm for each word is not listed due to a lack of space, the forms for \Sc{inf}, \Sc{2sg.prs}, \Sc{3sg.prs}, \Sc{2sg.pst}, \Sc{3sg.pst} and \Sc{conneg} are sufficient to convey the relevant morphological differences between the classes.
%\begin{table}\centering
\begin{sidewaystable}\centering
\caption{Comparison of verb class examples}\label{verbClassExamples}
%\hspace{-15mm}
\begin{tabular}{| cc | c | c | c | c | c | c || l |}\hline
\MC{2}{|c|}{\It{class}}&\Sc{inf}	&\Sc{2sg.prs}	&\Sc{3sg.prs}	&\Sc{2sg.pst}	&\Sc{3sg.pst}	&\Sc{conneg}	&\It{gloss}	\\\dline
I	&		&viess-o-t		&vies-o		&viess-o		&viess-o		&vies-o-j		&vies-o		& ‘live, feel’	\\\cline{3-9}%\hline
	&		&årr-o-t		&år-o		&årr-o		&årr-o		&år-o-j		&år-o		& ‘live, reside’	\\\cline{3-9}%\hline
	&		&gårr-o-t		&går-o		&gårr-o		&gårr-o		&går-o-j		&går-o		& ‘sew’	\\%\cline{3-9}%\hline
\dline
II	&a		&tjájbm-a-t	&tjájm-a		&tjájbm-a		&tjijbm-e		&tjájm-a-j		&tjájm-a		& ‘laugh’	\\\cline{3-9}%\hline
	&		&gähtj-a-t		&gietj-a		&gähtj-a		&gihtj-e	%\Red{?}
															&gietj-a-j		&gietj-a		& ‘look’	\\\cline{3-9}%\hline
	&		&bass-a-t		&bas-a		&bass-a		&biss-e		&bas-a-j		&bas-a		& ‘wash’	\\\cline{2-9}%\hline
	&b		&bårr-å-t		&bår-å		&bårr-a		&burr-e		&bår-å-j		&bår-å		& ‘eat’		\\\dline

III	&		&bass-e-t		&bas-á		&bass-a		&biss-e		&bis-i-j		&bas-e		& ‘fry’		\\\cline{3-9}%\hline
	&		&buälld-e-t	&buold-a		&bualld-a		&bulld-e		&buld-i-j		&buold-e		& ‘ignite, burn’	\\\cline{3-9}%\hline
	&		&adn-e-t		&an-á		&adn-a		&edn-e		&en-i-j		&an-e		& ‘have’	\\\cline{3-9}%\hline
	&		&vádts-e-t		&váts-a		&vádts-a		&vädts-e		&väts-i-j		&váts-e		& ‘go’		\\\dline

IV	&		&välld-u-t		&välld-u		&välld-u-ja	&välld-u-je	&välld-u-j		&välld-u		& ‘marry’		\\\dline
%	&		&årr-a-t		&årr-a		&årr-aj		&år-aj-e		&år-aji-j		&?			& ‘fall asleep’		\\\dline

V	&		&ságast-i-t	&ságast-a		&ságast-a		&ságast-e		&ságast-i-j	&ságast-e		& ‘say’		\\\cline{3-9}%\hline
	&		&málest-i-t	&málest-a		&málest-a		&málest-e		&málest-i-j	&málest-e		& ‘cook, boil’	\\\cline{3-9}%\hline
	&		&bargatj-i-t	&bargatj-a		&bargatj-a		&bargatj-e		&bargatj-i-j	&bargatj-e		& ‘work a little’	\\\dline%JW: corrected gloss AFTER final version sent in -- Jan 2013!!!

\MC{2}{|c|}{cop.}&årrot		&lä			&lä			&lidje		&lij			&lä			& ‘be	’	\\\dline

\MC{2}{|c|}{neg.}&-			&i			&ij			&ittje			&ittjij			&-			& ‘\NEGs’	\\\hline
\end{tabular}
\end{sidewaystable}






















%%%%%%% THIS IS NOT USED FOR THE ENTIRE COMPILATION, but only for individual chapters!!!!

\clearpage
\addcontentsline{toc}{chapter}{Bibliography}\label{Bibliography}
\bibliography{PiteGrammarBibSDL}%for bibtex
%\printbibliography%[title=Works Cited]%%for biber!






%%%NAME INDEX doesn’t work!?!? why???
\cleardoublepage\phantomsection%this allows hyperlink in ToC to work
\addcontentsline{toc}{chapter}{Name index}
\ohead{Name index}
\printindex[aut]

\cleardoublepage\phantomsection%this allows hyperlink in ToC to work
\addcontentsline{toc}{chapter}{Language index}
\ohead{Language index}
\printindex[lan]

\cleardoublepage\phantomsection%this allows hyperlink in ToC to work
\addcontentsline{toc}{chapter}{Subject index}
\ohead{Subject index}
\printindex


\end{document}