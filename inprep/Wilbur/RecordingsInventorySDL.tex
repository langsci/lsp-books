\hypertarget{pit080621}{pit080621} & elic. & A & 119 & Basic phrases, wordlist \\\hline %
\hypertarget{pit080622a}{pit080622a} & elic. & A & 168 & Verb paradigms (\Sc{prs}); numbers 1–1000 \\\hline %
\hypertarget{pit080622b}{pit080622b} & elic. & A & 6 & Phrase: “thank you” \\\hline %
\hypertarget{pit080627}{pit080627} & elic. & A & 39 & More exact numbers \\\hline %
\hypertarget{pit080701a}{pit080701a} & elic. & A & 4 & Phrase: "how do you say \_\_\_ in Saami" \\\hline %
\hypertarget{pit080701b}{pit080701b} & elic. & A & 116 & Swadesh list - words: 7–15, 17–21, 27–90, 91–114 \\\hline %
\hypertarget{pit080702a}{pit080702a} & elic. & A & 25 & A few short words/phrases (landscape words etc.) \\\hline %
\hypertarget{pit080702b}{pit080702b} & elic. & A & 226 & Swadesh list - words: 115–207; a few short phrases \\\hline %
\hypertarget{pit080703}{pit080703} & explan. & A/V & 334 & Descriptions of two pictures: 1: Saami camp 2: Reindeer in the tundra \\\hline %
\hypertarget{pit080708_Session01}{pit080708\_Session01} & explan. & A/V & 0 & Description of reindeer saddle carriers \\\hline %
\hypertarget{pit080708_Session02}{pit080708\_Session02} & explan. & A/V & 202 & Description of reindeer milking \\\hline %
\hypertarget{pit080708_Session03}{pit080708\_Session03} & explan. & A/V & 136 & Description of making butter, butter dishes and other dishes \\\hline %
\hypertarget{pit080708_Session04}{pit080708\_Session04} & explan. & A/V & 151 & Description of a Saami chest \\\hline %
\hypertarget{pit080708_Session05}{pit080708\_Session05} & explan. & A/V & 14 & Description of some traditional Pite Saami objects \\\hline %
\hypertarget{pit080708_Session06}{pit080708\_Session06} & explan. & A/V & 0 & Description of a Saami shirt design, reindeer-skin shoes \\\hline %
\hypertarget{pit080708_Session07}{pit080708\_Session07} & explan. & A/V & 74 & Description of some traditional Saami tools:  a weaving reed (a sort of mini-loom), a lasso ring and an unfinished sheath \\\hline %
\hypertarget{pit080708_Session08}{pit080708\_Session08} & explan. & A/V & 164 & Description of Saami kids’ shoes, two hats, reindeer-fur gloves \\\hline %
\hypertarget{pit080708_Session09}{pit080708\_Session09} & explan. & A/V & 124 & Description of animal traps \\\hline %
\hypertarget{pit080708_Session10}{pit080708\_Session10} & explan. & A/V & 0 & Description of how animal traps work \\\hline %
\hypertarget{pit080803a}{pit080803a} & elic. & A & 11 & Reindeer antler terms \\\hline %
\hypertarget{pit080803b}{pit080803b} & elic. & A & 118 & Ordinal numbers 1–10, >10 \\\hline %
\hypertarget{pit080811b1}{pit080811b1} & elic. & A & 4 & Numerals (\Sc{card}\PLUS\Sc{ord}), some verb paradigms \\\hline %
\hypertarget{pit080811b2}{pit080811b2} & elic. & A & 9 & Pronouns (\Sc{nom}), and other random words \\\hline %
\hypertarget{pit080813}{pit080813} & elic. & A & 163 & Verb paradigms \\\hline %
\hypertarget{pit080818}{pit080818} & elic., explan. & A/V & 0 & Reindeer-related words; numbers 1–10+ (\Sc{card}), 1–9 (\Sc{ord}) \\\hline %
\hypertarget{pit080819a}{pit080819a} & elic. & A & 448 & Adjective paradimgs \\\hline %
\hypertarget{pit080819b}{pit080819b} & elic. & A & 1 & Word: “sárrge” \\\hline %
\hypertarget{pit080819c}{pit080819c} & elic. & A & 17 & Phrase: “thank you for today” \\\hline %
\hypertarget{pit080825}{pit080825} & narr., song & A/V & 270 & Description of speaker’s family and her life in her childhood home; Singing of two hymns \\\hline %
\hypertarget{pit080909}{pit080909} & explan. & A/V & 736 & Film of reindeer roundup/slaughter, including footage of reindeer being selected, caught, slaughtered, and commentary on butchering a reindeer \\\hline %
\hypertarget{pit080917a}{pit080917a} & elic. & A & 201 & Some question words; some noun paradigms \\\hline %
\hypertarget{pit080917b}{pit080917b} & elic. & A & 8 & Numerals 20–30, 40, 50, 60, 1000, 2000 \\\hline %
\hypertarget{pit080917c}{pit080917c} & elic. & A & 36 & Some noun paradigms; some of question word paradigms “what” and “who” \\\hline %
\hypertarget{pit080924}{pit080924} & conv. & A/V & 2440 & Conversation about old times in Ákkapakte \\\hline %
\hypertarget{pit080926}{pit080926} & elic. & A & 107 & Word list from Pite-saami lessons from 25/26 september 2008 \\\hline %
\hypertarget{pit081011}{pit081011} & elic. & A & 355 & Random words collected during a previous Pite Saami lesson \\\hline %
\hypertarget{pit081012a}{pit081012a} & narr. & A & 0 & Descriptions of pictures from photo album, mostly of reindeer and calf marking \\\hline %
\hypertarget{pit081012b}{pit081012b} & elic. & A & 0 & Random words, mostly resulting from pit081012a \\\hline %
\hypertarget{pit081017}{pit081017} & elic. & A & 8 & Days of the week; months; seasons \\\hline %
\hypertarget{pit081021a_Story}{pit081021a\_Story} & read. & A/V & 0 & Reading of a story by Lars Rensund \\\hline %
\hypertarget{pit081021b}{pit081021b} & elic. & A & 0 & Demonstratives; some vocab from pit080708\_Session08 \\\hline %
\hypertarget{pit081028}{pit081028} & elic. & A & 29 & Words beginning with “sjnj-” (/ʃɲ/) \\\hline %
\hypertarget{pit081106}{pit081106} & explan. & A/V & 0 & Description of objects from Saami exhibit at Silvermuseet \\\hline %
\hypertarget{pit081111}{pit081111} & elic. & A & 55 & Adjective paradigms; some lexical items \\\hline %
\hypertarget{pit090411}{pit090411} & song, read., perf., writ. & A/V & 0 & Reading of scripture, singing of hymn \\\hline %
\hypertarget{pit090513}{pit090513} & elic. & A & 18 & Paradigm for noun \It{sábme} ‘Saami’ \\\hline %
\hypertarget{pit090519}{pit090519} & conv. & A/V & 1247 & A group of language activists have a picnic around a campfire, sometimes discussing words for a word list, but also just chatting \\\hline %
\hypertarget{pit090525a}{pit090525a} & elic. & A & 24 & Noun paradigms for: sábme (Saami), bena (dog) \\\hline %
\hypertarget{pit090525b}{pit090525b} & elic. & A & 77 & Six noun paradigms; short discussion of (near) minimal pairs \\\hline %
\hypertarget{pit090525c}{pit090525c} & conv. & A & 1 & Word \It{buris(t)} \\\hline %
\hypertarget{pit0906_Ahkajavvre_a}{pit0906\_Ahkajavvre\_a} & explan., narr. & A/V & 1105 & Description of the history and buildings at Ahkajavvre; performance of how to retrieve fishing nets and fish, and how to gut and wash fish; recorded on 9/10 June 2009 \\\hline %
\hypertarget{pit0906_Ahkajavvre_b}{pit0906\_Ahkajavvre\_b} & explan. & A/V & 301 & Description of the history of Ahkajavvre; recorded on 9/10 June 2009 \\\hline %
\hypertarget{pit090625}{pit090625} & elic. & A & 0 & A few words from the loanword typology list, mostly about geographic features \\\hline %
\hypertarget{pit090630}{pit090630} & conv., narr. & A/V & 103 & Conversation about a trip to Västerfjäll, driving across Tjeggelvas, going to school in Arjeplog; telling of a ghost story \\\hline %
\hypertarget{pit090702}{pit090702} & conv., narr. & A/V & 2245 & Conversation about fishing, hunting moose and preparing food in Västerfjäll/Álesgiehtje and Áhkkabakkte \\\hline %
\hypertarget{pit090705}{pit090705} & explan., narr. & A/V & 0 & On the way to and at reindeer calf marking the night of 5-6 July 2009 \\\hline %
\hypertarget{pit090821}{pit090821} & elic., explan. & A & 0 & A variety of words relating to berries, insects, house, etc. \\\hline %
\hypertarget{pit090822}{pit090822} & explan., narr. & A/V & 0 & Description of a variety of places around the speaker’s family homestead \\\hline %
\hypertarget{pit090823}{pit090823} & explan. & A/V & 0 & Description of the old house at the speaker’s family homestead \\\hline %
\hypertarget{pit090826}{pit090826} & explan. & A & 301 & Description of how reindeer herders look for unmarked calves \\\hline %
\hypertarget{pit090910}{pit090910} & elic. & A & 101 & Reflexive pronouns; some verb paradigms \\\hline %
\hypertarget{pit090912}{pit090912} & explan., conv., narr. & A/V & 0 & Video of reindeer slaughter, including first stages of butchering a reindeer \\\hline %
\hypertarget{pit090915a}{pit090915a} & narr. & A/V & 131 & Speaker talks about the hill on which Samegården stands \\\hline %
\hypertarget{pit090915b}{pit090915b} & narr. & A/V & 115 & Speaker talks about a pond that used to be in central Arjeplog, and how the Pite Saami name ‘Árjepluovve’ got its name from that pond \\\hline %
\hypertarget{pit090915c}{pit090915c} & narr. & A/V & 312 & Speaker talks about ‘Saami hill’ in central Arjeplog \\\hline %
\hypertarget{pit090915d}{pit090915d} & narr. & A/V & 103 & Speaker talks about ‘Knabben’-hill in central Arjeplog \\\hline %
\hypertarget{pit090915e}{pit090915e} & narr. & A/V & 116 & Speaker talks about the lake Hornavan/Tjårvek \\\hline %
\hypertarget{pit090915f}{pit090915f} & narr. & A/V & 225 & Speaker talks about the Skeppsviken/Hakksaluakkta in Arjeplog \\\hline %
\hypertarget{pit090915g}{pit090915g} & narr. & A/V & 201 & Speaker talks about the Skeppsholmen/Hakksasuolo in Arjeplog \\\hline %
\hypertarget{pit090915h}{pit090915h} & narr. & A/V & 87 & Speaker talks about a ‘njalla’ (raised storage shed) on Skeppsholmen in Arjeplog \\\hline %
\hypertarget{pit090915i}{pit090915i} & narr. & A/V & 123 & Speaker talks about a ‘luäppte’-storage shed on Skeppsholmen in Arjeplog \\\hline %
\hypertarget{pit090915j}{pit090915j} & narr. & A/V & 195 & Speaker talks about how reindeer and calves used to swim across the bay to Kraja \\\hline %
\hypertarget{pit090915k}{pit090915k} & narr. & A/V & 207 & Speaker talks about the location of the original marketplace in Arjeplog \\\hline %
\hypertarget{pit090926}{pit090926} & elic. & A & 494 & Adjective paradigms \\\hline %
\hypertarget{pit090927}{pit090927} & elic. & A & 445 & Adjective paradigms \\\hline %
\hypertarget{pit090930a}{pit090930a} & elic. & A & 829 & Adjectives in elliptical NPs; some color adjectives; more adjective paradigms \\\hline %
\hypertarget{pit090930b}{pit090930b} & elic. & A & 120 & Adjective paradigms \\\hline %
\hypertarget{pit091001}{pit091001} & elic. & A & 281 & Adjective paradigms \\\hline %
\hypertarget{pit100304}{pit100304} & elic. & A & 62 & Basic random wordlist (from the Leipzig-Jakarta list of basic vocabulary) \\\hline %
\hypertarget{pit100308a}{pit100308a} & elic. & A & 10 & Some basic elicitation forms, \Sc{nom.sg} and \Sc{acc.sg} noun paradigms \\\hline %
\hypertarget{pit100310b}{pit100310b} & narr., explan. & A/V & 0 & Description of a slide show concerning life in Álesgiehtje/Västerfjäll \\\hline %
\hypertarget{pit100323a}{pit100323a} & elic. & A & 481 & Verb paradigms \\\hline %
\hypertarget{pit100323b}{pit100323b} & song & A & 28 & Singing of a lullaby \\\hline %
\hypertarget{pit100324}{pit100324} & elic. & A/V & 291 & Expressions for spatial relations (mostly postpositions) \\\hline %
\hypertarget{pit100326}{pit100326} & elic. & A & 0 & Postpositions; some basic elicitation of existentials and demonstratives \\\hline %
\hypertarget{pit100403}{pit100403} & perf., read., writ. & A/V & 227 & Reading of scripture for Easter church service \\\hline %
\hypertarget{pit100404}{pit100404} & explan. & A/V & 1704 & Description of the winter landscape around Västerfjäll/Álesgiehtje, skiing, snowmobiles, playing there as a child, trapping ptarmigan, etc. \\\hline %
\hypertarget{pit100405a}{pit100405a} & explan., narr. & A & 758 & Description of the current winter from a reindeer herder’s perspective and of the activities that went on at the recording location near Blavvtajåhkå \\\hline %
\hypertarget{pit100405b}{pit100405b} & explan. & A/V & 461 & Description of different kinds of reindeer \\\hline %
\hypertarget{pit100703a}{pit100703a} & narr. & A & 312 & Story about waiting for the bus with the narrator’s aunt \\\hline %
\hypertarget{pit101208}{pit101208} & elic. & A & 674 & Verb paradigms \\\hline %
\hypertarget{pit110329}{pit110329} & elic. & A & 112 & Pronoun paradigms (personal, demonstrative, relative, interrogative); a few Saami village lexical items \\\hline %
\hypertarget{pit110331a}{pit110331a} & elic. & A & 405 & Pronoun paradigms (personal, demonstrative, interrogative, reflexive) for \Sc{nom},\Sc{acc}, \Sc{gen},\Sc{ill} \\\hline %
\hypertarget{pit110331b}{pit110331b} & elic. & A & 371 & Pronoun paradigms (personal, demonstrative, interrogative, reflexive) for \Sc{iness}, \Sc{elat}, \Sc{com} \\\hline %
\hypertarget{pit110404}{pit110404} & elic. & A & 530 & Verb paradigms \\\hline %
\hypertarget{pit110413a}{pit110413a} & elic. & A & 394 & Some noun paradigms \\\hline %
\hypertarget{pit110413b}{pit110413b} & elic. & A & 382 & Noun paradigms; ‘båtsoj’ includes numerals and quantifiers \\\hline %
\hypertarget{pit110415}{pit110415} & elic. & A & 158 & Kinship vocabulary; paradigm for ålmaj (man) \\\hline %
\hypertarget{pit110421}{pit110421} & elic. & A & 89 & Noun paradigms for juällge (leg/foot) and rejjdo (tool) \\\hline %
\hypertarget{pit110509a}{pit110509a} & elic. & A & 239 & Random verbs; random questions about subordinate clause linking; some noun paradigms \\\hline %
\hypertarget{pit110509b}{pit110509b} & elic. & A & 111 & Noun paradigms \\\hline %
\hypertarget{pit110517a}{pit110517a} & elic. & A & 870 & Some verb paradigms; includes some potential forms of verbs \\\hline %
\hypertarget{pit110517b1}{pit110517b1} & elic. & A & 0 & Some verb paradigms \\\hline %
\hypertarget{pit110517b2}{pit110517b2} & narr. & A & 380 & Short narrative about the orthography workshop on the previous weekend \\\hline %
\hypertarget{pit110518a}{pit110518a} & elic. & A & 187 & Verb paradigms; discussion of passives \\\hline %
\hypertarget{pit110518b}{pit110518b} & elic. & A & 0 & Some aspects in verbs; verbs for scratch/dig \\\hline %
\hypertarget{pit110519a}{pit110519a} & elic. & A & 0 & Some partial verb paradigms; some verbal derivations \\\hline %
\hypertarget{pit110519b}{pit110519b} & elic. & A & 47 & Some conjunction/subordinators;  question particle discussion;  some partial noun paradigms \\\hline %
\hypertarget{pit110521a}{pit110521a} & elic. & A & 150 & Pronoun paradigms: \Sc{nom}, \Sc{acc} for personal, most reflexive, demonstrative, question, selection, relative pronouns \\\hline %
\hypertarget{pit110521b1}{pit110521b1} & elic. & A & 529 & Some pronoun paradigms; a short narrative about what speaker did yesterday in Piteå \\\hline %
\hypertarget{pit110521b2}{pit110521b2} & narr. & A & 8 & A short narrative about what speaker did yesterday in Piteå \\\hline %
\hypertarget{pit110522}{pit110522} & elic. & A & 213 & Some verb and noun paradigms \\\hline %
\hypertarget{sje20121009}{sje20121009} & elic. & A & 252 & random grammatical topics: - DIM allomorphy - contracted verbs (gullut) - NEG.IMP (SG/DU/PL) - DEM - 3-way distinction - ADJ vs. ADV - passives - ADJ as head of NP - possessive suffixes \\\hline %
\hypertarget{sje20121014a}{sje20121014a} & elic. & A & 1 & Questions meant to elicit suspected differences between Pite Saami dialects \\\hline %
\hypertarget{sje20121014b}{sje20121014b} & conv. & A/V & 0 & Conversation about the old days, topics such as reindeer calves, coffee cheese, eating bear meat, seeing a bear, etc. \\\hline %
\hypertarget{sje20121011}{sje20121011} & elic. & A & 178 & Questions about \Sc{dim} suffix allomorphy in nouns; ‘contracted’ verb paradigm; imperative of negation verb; possessive suffixes, etc. \\\hline %
\hypertarget{sje20121014d1}{sje20121014d1} & elic. & A & 3 & NP-syntax, gapping; \Sc{comp} for Adj; coordination, complementizers \\\hline %
\hypertarget{sje20130523}{sje20130523} & narr. & A/V & 763 & A narrative about going to church and praying \\\hline %
\hypertarget{sje20130530b}{sje20130530b} & conv. & A & 1067 & a discussion about words, coffee and other topics while preparing and drinking coffee in the kitchen \\\hline %
\hypertarget{sje20131006}{sje20131006} & conv., explan., narr. & A/V & 181 & Discussion of the homestead; Speaker plays with his granddaughte; describes various traditional sheds on his property. \\\hline %
\hypertarget{sje20131012a}{sje20131012a} & conv. & A/V & 0 & Short conversation about Trollforsen and some of the old Saami huts there \\\hline %
\hypertarget{sje20131012b}{sje20131012b} & conv., narr., explan. & A/V & 0 & Tour of the homestead there, gathering reindeer moss, playing with a speaker’s granddaughter, tour of pictures and artefacts in the living room \\\hline %
\hypertarget{sje20131031}{sje20131031} & explan., narr. & A/V & 0 & Description of various things related to driving to check on the reindeer and then a potential moose hunt \\\hline %
