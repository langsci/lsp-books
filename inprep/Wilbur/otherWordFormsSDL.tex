{%\documentclass[ number=5
			   ,series=sidl
			   ,isbn=xxx-x-xxxxxx-xx-x
			   ,url=http://langsci-press.org/catalog/book/17
			   ,output=long   % long|short|inprep              
			   %,blackandwhite
			   %,smallfont
			   ,draftmode   
			  ]{LSP/langsci}                          

\usepackage{LSP/lsp-styles/lsp-gb4e}		% verhindert Komma bei mehrfachen Fußnoten?
                                                      
\usepackage{layout}
\usepackage{lipsum}

%%%% ABOVE FOR LangSciPress %%%%
%%%% ABOVE FOR LangSciPress %%%%
%%%% ABOVE FOR LangSciPress %%%%
\usepackage{libertine}%work-around solution for rendering problematic characters ʦ, ͡  (mostly in \textbf{})

\usepackage{longtable}%Double-lines (\hline\hline) aren’t typeset properly in ‘longtable’-environment across several pages! conflict with other package (maybe xcolor with option ‘tables’?)

\usepackage{multirow}

\usepackage{array} %allows, among other things, centering column content in a table while also specifying width, creates new column style "x" for center-alignment, "y" for right-alignment
\newcolumntype{x}[1]{>{\centering\hspace{0pt}}p{#1}}%
\newcolumntype{y}[1]{>{\raggedleft\hspace{0pt}}p{#1}}%

\usepackage[]{placeins}%using \FloatBarrier command, all floats still floating at that point will be typeset, and cannot cross that boundary. the option here \usepackage[section]{placeins} automatically adds \FloatBarrier to every \section command (only works for \section commands, nothing lower than that!)
%\usepackage{afterpage}%by using the command \afterpage{\clearpage}, all floats will appear, but no new page will be started, thus avoiding bad page breaks around floats

\usepackage{vowel} %for vowel space chart


%%%IS THIS NECESSARY??
%%%%following allows you to refer to footnotes (from http://anthony.liekens.net/index.php/LaTeX/MultipleFootnoteReferences)
%\newcommand{\footnoteremember}[2]{
%  \footnote{#2}
%  \newcounter{#1}
%  \setcounter{#1}{\value{footnote}}
%} \newcommand{\footnoterecall}[1]{
%  \footnotemark[\value{#1}]} 
%%%%previous allows you to refer to footnotes: use \footnoteremember{referenceText} in footnote, then \footnoterecall{referenceText} to refer.

\usepackage{tikz}%
\usetikzlibrary{plothandlers,matrix,decorations.text,shapes.arrows,shadows,chains,positioning,scopes}

\usepackage{synttree} %zeichnet linguistische Bäume
\branchheight{36pt}%sets height between rows in synttree

\usepackage{lscape}%used for landscape pages in index (list of recordings)

\usepackage{polyglossia}
\setmainlanguage{english}


%%%TAKE OUT FOR FINAL VERSION:
%%%TAKE OUT FOR FINAL VERSION:
%%%TAKE OUT FOR FINAL VERSION:

%%%%following readjusts margin text!
%\setlength{\marginparwidth}{20mm}
%\let\oldmarginpar\marginpar
%\renewcommand\marginpar[1]{\-\oldmarginpar[\raggedleft\footnotesize\vspace{-7pt}\color{red}\It{→ #1}]%
%{\raggedright\footnotesize\vspace{-7pt}\color{red}\It{→ #1}}}
%%%%previous readjusts margin text!

%%%The following lines set depth of ToC (LSP default is only 3 levels)!
%%%\renewcommand{\contentsname}{Table of Contents} % überschrift des inhaltsverzeichnisses
%\setcounter{secnumdepth}{5}%sets how deep section/subsection/subsubsections are numbered
%\setcounter{tocdepth}{5}%sets the depth of the ToC %but this doesn't seem to work!!!
%% new commands for LSP book (Grammar of Pite Saami, by J. Wilbur)

\newcommand{\PS}{Pite Saami}
\newcommand{\PSDP}{Pite Saami Documentation Project}
\newcommand{\WLP}{Wordlist Project}

\newcommand{\HANG}{\everypar{\hangindent15pt \hangafter1}}%also useful for table cells
\newcommand{\FB}{\FloatBarrier}%shortcut for this command to print all floats w/o pagebreak

\newcommand{\REF}[1]{(\ref{#1})}%adds parenthesis around the reference number, particularly useful for examples.%\Ref had clash with LSP!
\newcommand{\dline}{\hline\hline}%makes a double line in a table
\newcommand{\superS}[1]{\textsuperscript{#1}}%adds superscript element
\newcommand{\sub}[1]{$_{#1}$}%adds subscript element
\newcommand{\Sc}[1]{\textsc{#1}}%shortcut for small capitals (not to be confused with \sc, which changes the font from that point on)
\newcommand{\It}[1]{\textit{#1}}%shortcut for italics (not to be confused with \it, which changes the font from that point on)
\newcommand{\Bf}[1]{\textbf{#1}}%shortcut for bold (not to be confused with \bf, which changes the font from that point on)
\newcommand{\BfIt}[1]{\textbf{\textit{#1}}}
\newcommand{\BfSc}[1]{\textbf{\textsc{#1}}}
\newcommand{\Tn}[1]{\textnormal{#1}}%shortcut for normal text (undo italics, bolt, etc.)
\newcommand{\MC}{\multicolumn}%shortcut for multicolumn command in tabular environment - only replaces command, not variables!
\newcommand{\MR}{\multirow}%shortcut for multicolumn command in tabular environment - only replaces command, not variables!
\newcommand{\TILDE}{∼}%U+223C %OLD:~}%shortcut for tilde%command ‘\Tilde’ clashes with LSP!%
\newcommand{\BS}{\textbackslash}%backslash
\newcommand{\Red}[1]{{\color{red}{#1}}}%for red text
\newcommand{\Blue}[1]{{\color{blue}{#1}}}%for blue text
\newcommand{\PLUS}{+}%nicer looking plus symbol
\newcommand{\MINUS}{-}%nicer looking plus symbol
%    Was die Pfeile betrifft, kannst Du mal \Rightarrow \mapsto \textrightarrow probieren und dann \mathbf \boldsymbol oder \pbm dazutun.
\newcommand{\ARROW}{\textrightarrow}%→%dieser dicke Pfeil ➜ wird nicht von der LSP-Font unterstützt: %\newcommand{\ARROW}{{\fontspec{DejaVu Sans}➜}}
\newcommand{\DARROW}{\textleftrightarrow}%↔︎%DoubleARROW
\newcommand{\BULLET}{•}%
%%✓ does not exist in the default LSP font!
\newcommand{\CH}{\checkmark}%%\newcommand{\CH}{\fontspec{Arial Unicode MS}✓}%CH as in CHeck
%%following used to separate alternation forms for consonant gradation and umlaut patterns:
\newcommand{\Div}{‑}%↔︎⬌⟷⬄⟺⇔%non-breaking hyphen: ‑  
\newcommand{\QUES}{\textsuperscript{?}}%marks questionable/uncertain forms

\newcommand{\jvh}{\mbox{\It{j}-suffix} vowel harmony}%
%\newcommand{\Ptcl}{\Sc{ptcl} }%just shortcut for glossing ‘particle’
%\newcommand{\ATTR}{{\Sc{attributive}}}%shortcut for ATTRIBUTIVE in small caps
%\newcommand{\PRED}{{\Sc{predicative}}}%shortcut for PREDICATIVE in small caps
%\newcommand{\COMP}{{\Sc{comparative}}}%shortcut for COMPARATIVE in small caps
%\newcommand{\SUPERL}{{\Sc{superlative}}}%shortcut for SUPERLATIVE in small caps
\newcommand{\SG}{{\Sc{singular}}}%shortcut for SINGULAR in small caps
\newcommand{\DU}{{\Sc{dual}}}%shortcut for DUAL in small caps
\newcommand{\PL}{{\Sc{plural}}}%shortcut for PLURAL in small caps
%\newcommand{\NOM}{{\Sc{nominative}}}%shortcut for NOMINATIVE in small caps
%\newcommand{\ACC}{{\Sc{accusative}}}%shortcut for ACCUSATIVE in small caps
%\newcommand{\GEN}{{\Sc{genitive}}}%shortcut for GENITIVE in small caps
%\newcommand{\ILL}{{\Sc{illative}}}%shortcut for ILLATIVE in small caps
%\newcommand{\INESS}{{\Sc{inessive}}}%shortcut for INESSIVE in small caps
\newcommand{\ELAT}{{\Sc{elative}}}%shortcut for ELATIVE in small caps
%\newcommand{\COM}{{\Sc{comitative}}}%shortcut for COMITATIVE in small caps
%\newcommand{\ABESS}{{\Sc{abessive}}}%shortcut for ABESSIVE in small caps
%\newcommand{\ESS}{{\Sc{essive}}}%shortcut for ESSIVE in small caps
%\newcommand{\DIM}{{\Sc{diminutive}}}%shortcut for DIMINUTIVE in small caps
%\newcommand{\ORD}{{\Sc{ordinal}}}%shortcut for ORDINAL in small caps
%\newcommand{\CARD}{{\Sc{cardinal}}}%shortcut for CARDINAL in small caps
%\newcommand{\PROX}{{\Sc{proximal}}}%shortcut for PROXIMAL in small caps
%\newcommand{\DIST}{{\Sc{distal}}}%shortcut for DISTAL in small caps
%\newcommand{\RMT}{{\Sc{remote}}}%shortcut for REMOTE in small caps
%\newcommand{\REFL}{{\Sc{reflexive}}}%shortcut for REFLEXIVE in small caps
%\newcommand{\PRS}{{\Sc{present}}}%shortcut for PRESENT in small caps
%\newcommand{\PST}{{\Sc{past}}}%shortcut for PAST in small caps
%\newcommand{\IMP}{{\Sc{imperative}}}%shortcut for IMPERATIVE in small caps
%\newcommand{\POT}{{\Sc{potential}}}%shortcut for POTENTIAL in small caps
\newcommand{\PROG}{{\Sc{progressive}}}%shortcut for PROGRESSIVE in small caps
\newcommand{\PRF}{{\Sc{perfect}}}%shortcut for PERFECT in small caps
\newcommand{\INF}{{\Sc{infinitive}}}%shortcut for INFINITIVE in small caps
%\newcommand{\NEG}{{\Sc{negative}}}%shortcut for NEGATIVE in small caps
\newcommand{\CONNEG}{{\Sc{connegative}}}%shortcut for CONNEGATIVE in small caps
\newcommand{\ATTRs}{{\Sc{attr}}}%shortcut for ATTR in small caps
\newcommand{\PREDs}{{\Sc{pred}}}%shortcut for PRED in small caps
%\newcommand{\COMPs}{{\Sc{comp}}}%shortcut for COMP in small caps
%\newcommand{\SUPERLs}{{\Sc{superl}}}%shortcut for SUPERL in small caps
\newcommand{\SGs}{{\Sc{sg}}}%shortcut for SG in small caps
\newcommand{\DUs}{{\Sc{du}}}%shortcut for DU in small caps
\newcommand{\PLs}{{\Sc{pl}}}%shortcut for PL in small caps
\newcommand{\NOMs}{{\Sc{nom}}}%shortcut for NOM in small caps
\newcommand{\ACCs}{{\Sc{acc}}}%shortcut for ACC in small caps
\newcommand{\GENs}{{\Sc{gen}}}%shortcut for GEN in small caps
\newcommand{\ILLs}{{\Sc{ill}}}%shortcut for ILL in small caps
\newcommand{\INESSs}{{\Sc{iness}}}%shortcut for INESS in small caps
\newcommand{\ELATs}{{\Sc{elat}}}%shortcut for ELAT in small caps
\newcommand{\COMs}{{\Sc{com}}}%shortcut for COM in small caps
\newcommand{\ABESSs}{{\Sc{abess}}}%shortcut for ABESS in small caps
\newcommand{\ESSs}{{\Sc{ess}}}%shortcut for ESS in small caps
%\newcommand{\DIMs}{{\Sc{dim}}}%shortcut for DIM in small caps
%\newcommand{\ORDs}{{\Sc{ord}}}%shortcut for ORD in small caps
%\newcommand{\CARDs}{{\Sc{card}}}%shortcut for CARD in small caps
\newcommand{\PROXs}{{\Sc{prox}}}%shortcut for PROX in small caps
\newcommand{\DISTs}{{\Sc{dist}}}%shortcut for DIST in small caps
\newcommand{\RMTs}{{\Sc{rmt}}}%shortcut for RMT in small caps
\newcommand{\REFLs}{{\Sc{refl}}}%shortcut for REFL in small caps
\newcommand{\PRSs}{{\Sc{prs}}}%shortcut for PRS in small caps
\newcommand{\PSTs}{{\Sc{pst}}}%shortcut for PST in small caps
\newcommand{\IMPs}{{\Sc{imp}}}%shortcut for IMP in small caps
\newcommand{\POTs}{{\Sc{pot}}}%shortcut for POT in small caps
\newcommand{\PROGs}{{\Sc{prog}}}%shortcut for PROG in small caps
\newcommand{\PRFs}{{\Sc{prf}}}%shortcut for PRF in small caps
\newcommand{\INFs}{{\Sc{inf}}}%shortcut for INF in small caps
\newcommand{\NEGs}{{\Sc{neg}}}%shortcut for NEG in small caps
\newcommand{\CONNEGs}{{\Sc{conneg}}}%shortcut for CONNEG in small caps

\newcommand{\subNP}{{\footnotesize\sub{NP}}}%shortcut for NP (nominal phrase) in subscript
\newcommand{\subVC}{{\footnotesize\sub{VC}}}%shortcut for VC (verb complex) in subscript
\newcommand{\subAP}{{\footnotesize\sub{AP}}}%shortcut for NP (adjectival phrase) in subscript
\newcommand{\subAdvP}{{\footnotesize\sub{AdvP}}}%shortcut for AdvP (adverbial phrase) in subscript
\newcommand{\subPP}{{\footnotesize\sub{PP}}}%shortcut for NP (postpoistional phrase) in subscript

\newcommand{\ipa}[1]{{\fontspec{Linux Libertine}#1}}%specifying font for IPA characters

\newcommand{\SEC}{§}%standardize section symbol and spacing afterwards
%\newcommand{\SEC}{§\,}%

\newcommand{\Nth}{{\footnotesize(\It{n})}}%used in table of numerals in ADJ chapter

%%newcommands for tables in introductionSDL.tex:
\newcommand{\cliticExs}[3]{\Tn{\begin{tabular}{p{28mm} c p{28mm} p{35mm}}\It{#1}&\ARROW &\It{#2} & ‘#3’\\\end{tabular}}}%specifically for the two clitic examples
\newcommand{\Grapheme}[1]{\It{#1}}%formatting for graphemes in orthography tables
%%new command for the section on orthographic examples; syntax: #1=orthography, #2=phonology, #3=gloss
\newcommand{\SpellEx}[3]{\Tn{\begin{tabular}{p{70pt} p{70pt} l}\ipa{/#2/}&\It{#1}& ‘#3’ \\\end{tabular}}}%formatting for orthographic examples (intro-Chapter)


%%new transl tier in gb4e; syntax: #1=free translation (in single quotes), #2=additional comments, z.B. literal meaning:
\newcommand{\Transl}[2]{\trans\Tn{‘#1’ #2}}%new transl tier in gb4e;
\newcommand{\TranslMulti}[2]{\trans\hspace{12pt}\Tn{‘#1’ #2}}%new transl tier in gb4e for a dialog to be included under a single example number


%% used for examples in the Prosody and Segmental phonology chapters:
\newcommand{\PhonGloss}[7]{%PhonGloss = Phonology Gloss;
%pattern: \PhonGloss{label}{phonemic}{phonetic}{orthographic}{gloss}{recording}{utterance}
\ea\label{#1}
\Tn{\begin{tabular}[t]{p{30mm} l}
\ipa{/#2/}	& \It{#4} \\
\ipa{[#3]}	&\HANG ‘#5’\\%no table row can start with square brackets! thus the workaround with \MC
\end{tabular}\hfill\hyperlink{#6}{{\small\textnormal[pit#6#7]}}%\index{Z\Red{rec}!\Red{pit#6}}\index{Z\Red{utt}!\Red{pit#6#7} \Blue{Phon}}
}
\z}
\newcommand{\PhonGlossWL}[6]{%PhonGloss = Phonology Gloss for words from WORDLIST, not from corpus!;
%pattern: \PhonGloss{label}{phonemic}{phonetic}{orthographic}{gloss}{wordListNumber}
\ea\label{#1}
\Tn{\begin{tabular}[t]{p{30mm} l}
\ipa{/#2/}	& \It{#4} \\
\ipa{[#3]}	&\HANG ‘#5’\\%no table row can start with square brackets! thus the workaround with \MC
\end{tabular}\hfill\hyperlink{explExs}{{\small\textnormal[#6]}}%\index{Z\Red{wl}!\Red{#6}\Blue{Phon}}
}
\z}

%%for derivation examples in the derivational morphology chapter!
%syntax: \DerivExam{#1}{#2}{#3}{#4}{#5}{#6}
%#1: base, #2: base-gloss, #3: derived form, #4: derived form gloss, #5: derived form translation, #6: pit-recording, #7: utterance number
\newcommand{\DW}{28mm}%for following three commands, to align arrows throughout
%%%%OLD:
%%%\newcommand{\DerivExam}[7]{\Tn{\begin{tabular}[t]{p{\DW}cl}\It{#1}&\ARROW&\It{#3}\\#2&&#4\\\end{tabular}\hfill\pbox{.3\textwidth}{\hfill‘#5’\\\hbox{}\hfill\hyperlink{pit#6}{{\small\textnormal[pit#6.#7]}}}
%%%%\index{Z\Red{rec}!\Red{pit#6}}\index{Z\Red{utt}!\Red{pit#6.#7}}
%%%}}
%NEW:
\newcommand{\DerivExam}[7]{\Tn{
\begin{tabular}[t]{p{\DW}x{5mm}l}\It{#1}&\ARROW&\It{#3}\\\end{tabular}\hfill‘#5’\\
\hspace{1mm}\begin{tabular}[t]{p{\DW}x{5mm}l}#2&&#4\\\end{tabular}\hfill\hyperlink{pit#6}{{\small\textnormal[pit#6.#7]}}
%\index{Z\Red{rec}!\Red{pit#6}}\index{Z\Red{utt}!\Red{pit#6.#7}}
}}
%%same as above, but supress any reference to a specific utterance
\newcommand{\DerivExamX}[7]{\Tn{
\begin{tabular}[t]{p{\DW}x{5mm}l}\It{#1}&\ARROW&\It{#3}\\\end{tabular}\hfill‘#5’\\
\hspace{1mm}\begin{tabular}[t]{p{\DW}x{5mm}l}#2&&#4\\\end{tabular}\hfill\hyperlink{pit#6}{{\small\textnormal[pit#6]\It{e}}}
%\index{Z\Red{rec}!\Red{pit#6}}\index{Z\Red{utt}!\Red{pit#6.#7}}
}}
\newcommand{\DerivExamWL}[6]{\Tn{
\begin{tabular}[t]{p{\DW}x{5mm}l}\It{#1}&\ARROW&\It{#3}\\\end{tabular}\hfill‘#5’\\
\hspace{1mm}\begin{tabular}[t]{p{\DW}x{5mm}l}#2&&#4\\\end{tabular}\hfill\hyperlink{explExs}{{\small\textnormal[#6]}}
%\index{Z\Red{wl}!\Red{#6}}
}}


%formatting of corpus source information (after \transl in gb4e-environments):
\newcommand{\Corpus}[2]{\hspace*{1pt}\hfill{\small\mbox{\hyperlink{pit#1}{\Tn{[pit#1.#2]}}}}%\index{Z\Red{rec}!\Red{pit#1}}\index{Z\Red{utt}!\Red{pit#1.#2}}
}%
\newcommand{\CorpusE}[2]{\hspace*{1pt}\hfill{\small\mbox{\hyperlink{pit#1}{\Tn{[pit#1.#2]}}\It{e}}}%\index{Z\Red{rec}!\Red{pit#1}}\index{Z\Red{utt}!\Red{pit#1.#2}\Blue{-E}}
}%
%%as above, but necessary for recording names which include an underline because the first variable in \href understands _ but the second variable requires \_
\newcommand{\CorpusLink}[3]{\hspace*{1pt}\hfill{\small\mbox{\hyperlink{pit#1}{\Tn{[pit#2.#3]}}}}%\index{Z\Red{rec}!\Red{pit#2}}\index{Z\Red{utt}!\Red{pit#2.#3}}
}%
%%as above, but for newer recordings which begin with sje20 instead of pit
\newcommand{\CorpusSJE}[2]{\hspace*{1pt}\hfill{\small\mbox{\hyperlink{sje20#1}{\Tn{[sje20#1.#2]}}}}%\index{Z\Red{rec}!\Red{sje20#1}}\index{Z\Red{utt}!\Red{sje20#1.#2}}
}%
\newcommand{\CorpusSJEE}[2]{\hspace*{1pt}\hfill{\small\mbox{\hyperlink{sje20#1}{\Tn{[sje20#1.#2]}}\It{e}}}%\index{Z\Red{rec}!\Red{sje20#1}}\index{Z\Red{utt}!\Red{sje20#1.#2}\Blue{-E}}
}%











%%hyphenation points for line breaks
%%add to TeX file before \begin{document} with:
%%%%hyphenation points for line breaks
%%add to TeX file before \begin{document} with:
%%%%hyphenation points for line breaks
%%add to TeX file before \begin{document} with:
%%\include{hyphenationSDL}
\hyphenation{
ab-es-sive
affri-ca-te
affri-ca-tes
Ahka-javv-re
al-ve-o-lar
com-ple-ments
%check this:
de-cad-es
fri-ca-tive
fri-ca-tives
gemi-nate
gemi-nates
gra-pheme
gra-phemes
ho-mo-pho-nous
ho-mor-ga-nic
mor-pho-syn-tac-tic
or-tho-gra-phic
pho-neme
pho-ne-mes
phra-ses
post-po-si-tion
post-po-si-tion-al
pre-as-pi-ra-te
pre-as-pi-ra-ted
pre-as-pi-ra-tion
seg-ment
un-voiced
wor-king-ver-sion
}
\hyphenation{
ab-es-sive
affri-ca-te
affri-ca-tes
Ahka-javv-re
al-ve-o-lar
com-ple-ments
%check this:
de-cad-es
fri-ca-tive
fri-ca-tives
gemi-nate
gemi-nates
gra-pheme
gra-phemes
ho-mo-pho-nous
ho-mor-ga-nic
mor-pho-syn-tac-tic
or-tho-gra-phic
pho-neme
pho-ne-mes
phra-ses
post-po-si-tion
post-po-si-tion-al
pre-as-pi-ra-te
pre-as-pi-ra-ted
pre-as-pi-ra-tion
seg-ment
un-voiced
wor-king-ver-sion
}
\hyphenation{
ab-es-sive
affri-ca-te
affri-ca-tes
Ahka-javv-re
al-ve-o-lar
com-ple-ments
%check this:
de-cad-es
fri-ca-tive
fri-ca-tives
gemi-nate
gemi-nates
gra-pheme
gra-phemes
ho-mo-pho-nous
ho-mor-ga-nic
mor-pho-syn-tac-tic
or-tho-gra-phic
pho-neme
pho-ne-mes
phra-ses
post-po-si-tion
post-po-si-tion-al
pre-as-pi-ra-te
pre-as-pi-ra-ted
pre-as-pi-ra-tion
seg-ment
un-voiced
wor-king-ver-sion
}\begin{document}\tableofcontents\clearpage

%%%%%%%%%%%%%%%%%%%%%%%%%%%%%%%%% ALL THE ABOVE TO BE COMMENTED OUT FOR COMPLETE DOCUMENT! %%%%%%%%%%%

\chapter{Other word classes}\label{otherWordClasses}
This chapter describes the word classes:
\begin{itemize}
\item{adverbs in \SEC\ref{adverbs},}
\item{adpositions in \SEC\ref{adpositions},}
\item{conjunctions in \SEC\ref{conjunctions},}
%\item{particles in \SEC\ref{particles}}
\item{and interjections in \SEC\ref{interjections}}
\end{itemize}
The information provided here is of a preliminary nature due to limited data in the corpus, and stands to gain much from future research. %Sections \ref{conjunctions} through \ref{interjections} deal with postpositions, conjunctions, particles and interjections, in that order. 

%\vfill

\section{Adverbs}\label{adverbs}
Adverbs compose an open word class and are defined by their ability to head an adverbial phrase; they can be further divided into two main groups: 
\begin{itemize}
\item{derived adverbs}
\item{lexical adverbs}
\end{itemize}
%Derived adverbs form an open set of adverbs, while lexical adverbs are a closed set. 
Here, \SEC\ref{derivedADVs} deals with the former, while \SEC\ref{lexicalADVs} presents with the latter. 

%\vfill
%\pagebreak

\subsection{Derived adverbs}\label{derivedADVs}
%The data in the corpus is very limited concerning derived adverbs, but a
At least one derivational affix seems to exist which derives an adverb from an adjective: the suffix \It{-t}, as illustrated by Table \vref{derivedADVsTable}. 
This adverbializing suffix triggers the weak consonant grade, when applicable. Two examples from the corpus are provided in \REF{derivedADVsEx1} and \REF{derivedADVsEx2} below.
%\pagebreak
\begin{table}[h]\centering
\caption{Derived adverbs and their adjectival stems}\label{derivedADVsTable}
\begin{tabular}{llll}\mytoprule
\ATTRs-adjective	&&{adverb}	&	\\\hline
\It{várogis}	&\ARROW&\It{várogit}		& ‘careful(ly)’	\\%\hline
\It{buoragis}	&\ARROW&\It{buoragit}		& ‘good/well’	\\\mybottomrule
%suojmas	&\ARROW&suojmagit	& ‘slow’	\\\hline
%\it  adverb 	&\it adjective	&\it gloss	\\\mybottomrule
%várogit	&várrok		& ‘careful(ly)’	\\\hline
%buoragit	&buorrak		& ‘good/well’	\\\hline
\end{tabular}
\end{table}

\ea\label{derivedADVsEx1}
\glll	dån virte várogit válldet\\
	dån virte várogi-t vállde-t\\
	\Sc{2sg.nom} must\BS\Sc{2sg.prs} careful-\Sc{advz} take-\Sc{inf}\\\nopagebreak
\Transl{you have to take it carefully}{}	\Corpus{080909}{062}
\z
\ea\label{derivedADVsEx2}
\glll	viesojmä vanj ganska buoragit dajna guollemijn aj\\
	vieso-jmä vanj {ganska\footnotemark\-} buoragi-t d-a-jna guollemi-jn aj\\
	live-\Sc{1pl.pst} definitely quite good-\Sc{advz} \Sc{dem}-\Sc{dist}-\Sc{com.sg} fishing-\Sc{com.sg} also\\\nopagebreak
\Transl{we definitely lived pretty well with the fishing, too}{}	\Corpus{0906\_Ahkajavvre\_a}{164}
\footnotetext{Note that \It{ganska} is a nonce borrowing from Swedish; cf. Swedish \It{ganska} ‘quite’.}
\z



\subsection{Lexical adverbs}\label{lexicalADVs}
A group of lexical items exclusively used as adverbs in \PS\ forms a subset of adverbs. 
A list of some lexical adverbs is provided in Table \vref{lexicalAdverbTable}.\footnote{The adverbs \It{ber}\TILDE\It{bar}, \It{kan} and \It{så} are Swedish loans; cf. Swedish \It{bara} ‘only’, \It{kanske} ‘maybe’ and \It{så} ‘so’.} 
 Examples containing the sentence adverbs \It{ber} ‘only’, \It{kan} ‘maybe’, \It{aj} ‘too’ and \It{vanj} ‘definitely’ are provided in \REF{ADVsEx1} through \REF{ADVsEx5}.
% and some examples from the corpus are found below. 
%for the adverbs \It{ber} ‘only’, \It{ilá} ‘too’, \It{kan} ‘maybe’, \It{dä} ‘then’, \It{aj} ‘also’ and \It{vanj} ‘really’ are provided in \REF{ADVsEx1} through \REF{ADVsEx5}.
% \It{ber}\TILDE\It{bar} ‘only’,\footnote{The adverb \It{ber}\TILDE\It{bar} is a Swedish loan; cf. Swedish \It{bara} ‘only’.} 
%\It{ilá} ‘too’, \It{mudiŋ} ‘sometimes’, \It{åbbå} ‘quite’, \It{del} ‘obviously’, \It{dä} ‘then’, \It{så} ‘so’,\footnote{The adverb \It{så} is a Swedish loan; cf. Swedish \It{så} ‘so’.} 
%\It{ihkep} ‘maybe’ and \It{kan} ‘maybe’.\footnote{The adverb \It{kan} is a Swedish loan; cf. Swedish \It{kanske} ‘maybe’, which can be parsed into \It{kan} ‘can’ and \it{ske} ‘happen’.} 
\begin{table}\centering%\resizebox{1\textwidth}{!} {
\caption{A selection of lexical adverbs}\label{lexicalAdverbTable}
\begin{tabular}{ll}\mytoprule
%{lexical adverb}&{}\\\hline
\It{aj			} & ‘also, too’	\\
\It{ber\TILDE bar	} & ‘only’	\\
\It{del		} & ‘obviously’	\\
\It{dä			} & ‘then’	\\
\It{gal		} & ‘actually’\\
%\Red{gu\TILDE gus}	} & question particle\\
\It{ihkep		} & ‘maybe’	\\
\It{ilá			} & ‘too’ (excessive)	\\
\It{kan		} & ‘maybe’	\\
\It{mudiŋ		} & ‘sometimes’	\\
\It{så			} & ‘so’	\\
\It{vanj		} & ‘really’	\\
\It{åbbå		} & ‘quite’	\\\mybottomrule
\end{tabular}%}
\end{table}

\ea\label{ADVsEx1}%ber
\glll	buhtsu mielkest ijtjen ber vuostajd dága\\
	buhtsu mielke-st ittj-in ber vuosta-jd dága\\
	reindeer\BS\Sc{gen.sg} milk-\Sc{elat.sg} \Sc{neg}-\Sc{3pl.pst} only cheese-\Sc{acc.pl} make\BS\Sc{conneg}\\\nopagebreak
\Transl{they didn’t only make cheese from reindeer milk}{}	\CorpusLink{080708_Session03}{080708\_Session03}{001}
\z
\ea\label{ADVsEx3}%kan
\glll	kan Edde diehta\\
	kan Edde diehta\\
	maybe Edgar\BS\Sc{nom.sg} know\BS\Sc{3sg.prs}\\\nopagebreak
\Transl{maybe Edgar knows}{}	\Corpus{090519}{355}
\z
\ea\label{ADVsEx3b}%dä
\glll	ja dä bedja dun nubbe bielen aj risijd\\
	ja dä bedja d-u-n nubbe biele-n aj risi-jd\\
	and then put\BS\Sc{3sg.prs} \Sc{dem}-\Sc{rmt}-\Sc{iness.sg} other side-\Sc{iness.sg} also twig-\Sc{acc.pl}\\\nopagebreak
\Transl{and then one puts twigs on the other side, too}{}	\Corpus{100404}{228}
\z
\ea\label{ADVsEx5}
\glll	gajk vuorasumos saddje'l vanj dát urrum dulutjist\\
	gajk vuoras-umos saddje=l vanj d-á-t urru-m dulutji-st\\
	all old-\Sc{superl} place\BS\Sc{nom.sg}=be\BS\Sc{3sg.prs} definitely \Sc{dem}-\Sc{prox}-\Sc{nom.sg} be-\Sc{prf} old.days-\Sc{elat.sg}\\\nopagebreak
\Transl{this was definitely the absolute oldest place from the old days}{}	\CorpusLink{0906_Ahkajavvre_a}{0906\_Ahkajavvre\_a}{059}
\z

A further lexical adverb is \It{gal} ‘actually’, which can be used to emphasize a contradiction or surprise, as in \REF{particleEx6}. The interjection \It{nä} ‘no’, a borrowing from Swedish,\footnote{< Swedish \It{nej} ‘no’; cf. local dialect pronunciation [nɛː].} 
is also used in this example, in addition to the native Saamic negation verb. 
\ea\label{particleEx6}
\glll	\Tn{A:} udtju sáme gielav danne sagastit?\\
	{} udtju sáme giela-v danne sagasti-t\\
	{} be.allowed\BS\Sc{2sg.pst} Saami\BS\Sc{gen.sg} language-\Sc{acc.sg} there speak-\Sc{inf}\\\nopagebreak
\TranslMulti{were you allowed to speak the Saami language there?}{}\\
\glll	\Tn{B:} nä, ij gal, ittjiv åtjo\\
	{} nä ij gal i-ttjiv åtjo\\
	{} no \Sc{neg}\BS\Sc{3sg.prs} actually \Sc{neg}-\Sc{1sg.pst} be.allowed\BS\Sc{conneg}\\\nopagebreak
\TranslMulti{no, actually no, I wasn’t allowed to}{}	\Corpus{080924}{351-352}
\z

In \REF{ADVsEx2}, the adverb \It{ilá} ‘too’ modifies the adjective \It{nuora} ‘young’. 
\ea\label{ADVsEx2}%ilá
\glll	ilá nuora lijme\\
	ilá nuora li-jme\\
	too young\BS\Sc{pl} be-\Sc{1pl.pst}\\\nopagebreak
\Transl{we were too young}{}	\Corpus{080924}{437}
\z


\subsubsection{The question marker \It{gu}\TILDE\It{gus}}\label{QpartWordform}
In several Saami languages, including closely related Lule Saami, a grammatical unit often referred to in the literature as a ‘question particle’ is used to mark polar interrogative clauses.\footnote{Cf. North Saami \It{-go} \citep[cf.][90]{Svonni2009} and Lule Saami \It{-ga/-k/-ge} \citep[cf.][94-94]{Spiik1989}, which are cognate with \PS\ \It{gu}\TILDE\It{gus}. According to \citet[245]{Sammallahti1998}, the question marker was originally borrowed from Finnish. Skolt Saami also has a question marker \It{-a} \citep[cf.][319-320]{Feist2010}, which is not cognate.} 
For Pite Saami, \citet[20-21]{Lagercrantz1926} %and \cite{Lehtiranta1992}:?? both indicate
indicates that \PS\ also has a question marker \It{gu} identifying polar interrogatives, although he shows that it is not obligatory.\footnote{\citet[21]{Lagercrantz1926} notes that polar questions often are only marked by being verb-initial, so even in 1921 (when he conducted fieldwork for his book) the question marker was not obligatory in Pite Saami.} 
In the entire \PSDP\ corpus, there are only three clear tokens of a polar interrogative with the question marker, and even then, the marker has two forms: \It{gu} and \It{gus}. These tokens are provided in examples \REF{Qpart1}\footnote{Note that the question marker in example \REF{Qpart1} was recorded serendipitously in an elicitation session concerning a different topic.} 
through \REF{Qpart3}. 
\ea\label{Qpart1}
\glll	lä gu nällgomin?\\
	lä gu nällgo-min\\
	be\BS\Sc{2sg.prs} \Sc{Q} hunger-\Sc{prog}\\\nopagebreak
\Transl{are you hungry?}{(lit.: are you hungering)} \CorpusE{110518a}{18m36s} %(lit.: ‘are you hungering?’)
\z
\ea\label{Qpart2}
\glll	aná gus dån naginav, mujtojd?\\
	aná gus dån nagina-v mujt-o-jd\\
	have\BS\Sc{2sg.prs} \Sc{Q} \Sc{2sg.nom} something-\Sc{acc.sg} remember-\Sc{nmlz1}-\Sc{acc.pl}\\\nopagebreak
\Transl{do you have something, memories?}{} \Corpus{090702}{483} %
\z
\ea\label{Qpart3}
\glll	nå, juga gu guäsmagav?\\
	nå juga gu guäsmaga-v\\
	well drink\BS\Sc{2sg.prs} \Sc{Q} coffee-\Sc{acc.sg}\\\nopagebreak
\Transl{well, do you drink coffee?}{} \CorpusSJE{130530b}{015}%
%\Transl{well, do you want to drink coffee?}{(lit: do you drink coffee?)} \Corpus{20130530b}{015} %
%\ex\label{Qpart2}%JW: not really clear upon a second listen
%\glll	sida gus ienabujd gullat\\
%	sida gus iena-bu-jd gulla-t\\
%	want\BS\Sc{2sg.prs} \Sc{Qptcl} much-\Sc{comp-acc} hear-\Sc{inf}\\
%\Transl{do you want to hear more?}{} \Corpus{090702.515} %(lit.: ‘are you hungering?’)
\z
Based on this lack of data, and on the description provided in \citet{Lagercrantz1926}, 
one can only conclude that the question marker is no longer required to identify polar interrogative clauses, and has all but disappeared from current \PS\ usage. %\marginpar{what about \ref{Qpart2}? when \It{gu}, when \It{gus}?}

In determining which word class the question marker belongs to, several facts should be considered. %lead to its current classification as a lexical adverb. 
Most importantly, like the adverbs in examples \REF{ADVsEx1} through \REF{ADVsEx5} above, the scope of the question marker is the entire sentence; \It{gu}\TILDE\It{gus} indicates an epistemic lack on behalf of the speaker concerning the proposition expressed by the interrogative clause it marks. While its monosyllabicity is remarkable, and implies a strong degree of grammaticalization (since lexical items in general are minimally bisyllabic), a number of other monosyllabic lexical adverbs also exist (cf. Table \vref{lexicalAdverbTable}). On this basis, the question marker can be classified as a lexical adverb. 

However, although the data are much too limited to be certain, the question marker in all three examples occurs directly after the finite verb. If it indeed can only occur here, then this may be sufficient reason to consider the question marker to be the sole member of a unique word class (perhaps best named ‘particle’) defined by its clause-level scope and syntactic position restriction.\footnote{Note that the brief description of the cognate Lule Saami ‘question particle’ in \mbox{\citet[95]{Spiik1989}} indicates that the Lule Saami equivalent may in fact be a focus particle used exclusively in polar interrogatives, as it “is placed near the word on which the most emphasis rests”, while always occurring “after the helping verb” (my translations).} % \It{gus}\TILDE\It{-k} 
%Nonetheless, until more data are available, this status of the question marker’s membership in a specific word class must remain a preliminary classification. 





\section{Adpositions}\label{adpositions}\index{adpositions}\index{parts of speech!adpositions}
Adpositions in \PS\ constitute a closed class of words that are defined syntactically by their ability to head an adpositional phrase (abbreviated ‘PP’ as these are either postpositional or prepositional phrases). 
Postpositions, which are clearly preferred over prepositions, are covered in \SEC\ref{postpositions}. The limited data on prepositions, which, with one exception, can all be used as postpositions as well, are described in \SEC\ref{prepositions}. 

\subsection{Postpositions}\label{postpositions}\index{postpositions}\index{parts of speech!postpositions}
Table \vref{postpositionTable} provides a selection of postpositions found in the corpus and includes English translation equivalents. %However, as there is obviously no one-to-one correspondence between post-positions, there English equivalents are intended only to serve as a rough 
It is possible that other postpositions also exist but were not attested in the corpus. %; however, postpositions remain a closed class, as new postpositions cannot be created.
\begin{table}[h]\centering%\resizebox{1\textwidth}{!} {
\caption[A selection of postpositions and their translation equivalents]{A selection of postpositions and their English translation equivalents}\label{postpositionTable}
\begin{tabular}{ll}\mytoprule
%			&\it approx. \\
%{postposition}&{translation equivalent}\\\hline
\It{badjel, bajel	} & ‘above; over’ \\%cf. pit100324.25m51s
\It{birra } & ‘about; around’ \\
\It{duogen } & ‘behind’ \\%cf. pit100404.045
\It{gaskan	} & ‘between’ \\%cf. pit100324.49m19s
\It{guoran } & ‘next to; near’ \\%cf. pit0906_Ahkajavvre_ES.004
\It{lahka	} & ‘near’ \\%cf. pit080702DSb
\It{nala	} & ‘upon, up, towards’ \\
\It{nanne\TILDE nan } & ‘on’ \\
%baltan } & ‘next to; beside ’ \\%in the corpus?!?
\It{sidån } & ‘next to; beside’ \\%pit 041324.12m03s
\It{sinne\TILDE sin } & ‘inside’ \\
\It{sissa\TILDE sis } & ‘into’ \\
\It{siste } & ‘out of’ \\
\It{tjadá	} & ‘throughout’ \\%cf. pit0906\_Ahkajavvre\_b.047
\It{vuolen } & ‘under’ \\
\It{vusste } & ‘against’ \\%cf. pit110413a.085/.087/.089
\It{åvdon } & ‘in front of’ \\
\It{åvdost } & ‘for’ \\\mybottomrule
\end{tabular}%}
\end{table}

Postpositions are complemented by NPs in the genitive case. 
Two examples are provided in \REF{postpositionEx1} and \REF{postpositionEx2}; for more on the syntactic behavior of postpositions in postpositional phrases as well as more examples, see \SEC\ref{postpositionalPhrases}. %
\ea\label{postpositionEx1}%
\glll	ja tsáhpat biergov káfa sis\\
	ja tsáhpa-t biergo-v káfa sis\\
	and cut-\Sc{inf} meat-\Sc{acc.sg} coffee\BS\Sc{gen.sg} into\\\nopagebreak
\Transl{and to cut meat into the coffee}{}	\Corpus{100405a}{136}
\z
\ea\label{postpositionEx2}%
\glll	ja dä skåvlåmáná minnin dan bajel sparkijin\\
	ja dä skåvlå-máná minni-n d-a-n bajel sparki-jin\\
	and then school-child\BS\Sc{nom.pl} go-\Sc{3pl.pst} \Sc{dem}-\Sc{dist}-\Sc{gen.sg} over kick.sled-\Sc{com.sg}\\\nopagebreak
\Transl{and then the school children went over that by kick-sled}{}	\Corpus{090915}{031}
\z


\subsection{Prepositions}\label{prepositions}\index{prepositions}\index{parts of speech!prepositions}
With the exception of \It{dugu} ‘like’, which governs a noun in either the essive or the nominative case (cf. \SEC\ref{essive}), 
a few words that are normally used as postpositions may also occur as prepositions. 
%Prepositions seem to exist in \PS, but very marginally, and only as words than are normally used as postpositions. 
The corpus provides only a very limited amount of data concerning the existence and behavior of prepositions; the two examples are presented here. 
%There does appear to be one preposition which does not occur as a postposition: \It{dugu} ‘like’, as illustrated in \REF{prepositionEx1} and \REF{prepositionEx2}. 

In \REF{prepositionEx1}, \It{birra} ‘about, around’ is used as a preposition, and governs the genitive case on the complement demonstrative. 
\ea\label{prepositionEx1}%
\glll	ja badde, åsto badde birra danne\\
	ja badde åst-o badde birra d-a-nne\\
	and ribbon\BS\Sc{nom.sg} buy-\Sc{nmlz1}\BS\Sc{nom.sg} ribbon\BS\Sc{nom.sg} around \Sc{dem}-\Sc{dist}-\Sc{gen.sg}\\\nopagebreak
\Transl{and ribbon, purchased ribbon around that}{}	\CorpusLink{080708_Session08}{080708\_Session08}{012}
\z

In \REF{prepositionEx2}, \It{badjel} ‘over’ is used as a preposition. %\footnote{There are several examples in the corpus for \It{badjel} as a postposition.} 
However, the complement \It{nällje kronor} ‘four crowns’ (referring to the Swedish currency) consists of the \PS\ numeral \It{nällje} and a Swedish borrowing \It{kronor} which is inflected according to Swedish grammar (\It{kron-or} ‘crown-\PLs’), and not \PS\ grammar, so it is impossible to know with these data which case \It{badjel} governs as a preposition. 
\ea\label{prepositionEx2}%
\glll	så åtjojmä badjel nällje kronor tjilos dalloj\\
	så åtjo-jmä badjel nällje kronor tjilos dalloj\\
	so receive-\Sc{1pl.pst} over four crowns kilogram at.that.time\\\nopagebreak
\Transl{so we received more than four Swedish crowns per kilogram back then}{}	\CorpusLink{0906_Ahkajavvre_a}{0906\_Ahkajavvre\_a}{159}
\z
%\ex\label{prepositionEx1}%
%\glll	dat vuodja dugu goullen\\
%	d-a-t vuodja dugu goullen\\
%	\Sc{dem}-\Sc{dist}-\Sc{nom.sg} swim\BS\Sc{3sg.prs} like fish-\Sc{ess}\\\nopagebreak
%\Transl{it swims like a fish}{}	\CorpusE{110413a.059}
%\ex\label{prepositionEx2}%
%\glll	mån lev dugu unna guolátj\\
%	mån le-v dugu unna guolá-tj\\
%	\Sc{1sg.nom} be-\Sc{1sg.prs} like little fish-\Sc{dim}\BS\Sc{nom.sg}\\\nopagebreak
%\Transl{I am like a little fish}{}	\CorpusE{110413a.114}
%%\glll	lihka gassak dugu biehtse\\
%%	lihka gassak dugu biehtse\\
%%	and one branch\BS\Sc{nom.sg} \Sc{dem}-\Sc{dist}\BS\Sc{elat.pl} get-\Sc{1pl.pst} milk-\Sc{acc.sg} butter-\Sc{acc.sg} cheese-\Sc{acc.sg} and meat-\Sc{acc.sg}\\\nopagebreak
%%\Transl{we got milk, butter, cheese and milk from these}{}	\Corpus{080825.015}
%\z
%The case marking for NPs complementing \It{dugu} is inconsistent, but is either \NOM\ or \ESS\ case. Furthermore, it is not clear if only NPs can complement \It{dugu}, or if a full clause could also complement it. Due to a lack of data in the corpus, and the inconsistencies in usage, prepositions are currently only considered a potential subgroup of the adpositional word class. 

Note that there are numerous examples for both \It{birra} and \It{badjel} as postpositions. 
It is not surprising that prepositions are infrequent and marginal in \PS\ as other Saami languages also only have a small set of prepositions with significant restrictions in frequency and meaning.\footnote{Cf. \citet[91-92]{Spiik1989} for Lule Saami, \citet[84-85]{Svonni2009} for North Saami and \citet[314-317]{Feist2010} for Skolt Saami. As for \PS, neither \citet{Lagercrantz1926} nor \citet{Lehtiranta1992} mention anything about prepositions or the syntactic behavior (constituent order) of adpositions in general.} % for \PS (cf. \cite[115-117]{Lagercrantz1926}).} 




\section{Conjunctions}\label{conjunctions}\index{conjunctions}\index{parts of speech!conjunctions}
Conjunctions in \PS\ form a closed class of words that connect phrases or clauses. A list of some conjunctions, what they can connect, as well as their English translation equivalents can be found in Table \vref{conjunctionTable}. Note that the conjunctions \It{att}, \It{eller} and \It{men} are borrowings from Swedish. 
\begin{table}[h]\centering%
\caption{Some \PS\ conjunctions and their English translation equivalents, as well as whether these connect phrases and/or clauses}%{Some \PS\ conjunctions and their English translation equivalents}
\label{conjunctionTable}
\begin{tabular}{llcc}\mytoprule
%			&		&\MC{2}{c}{{connects}} \\
{}			&		&{phrases}	&{clauses}	\\\hline
\It{att	}		& ‘(in order) to’	&			&\CH			\\
\It{eller}		& ‘or’		&\CH		&\CH			\\
\It{gu	}		& ‘when’	&\CH		&\CH		\\
\It{ja}			& ‘and’	&\CH		&\CH		\\
\It{jala}		& ‘or’		&\CH		&\CH		\\
\It{jus}		& ‘if’		&			&\CH		\\
\It{maŋŋel}	& ‘after’	&			&\CH		\\%JW: conjunction??
\It{men}		& ‘but’	&			&\CH		\\
\It{vaj}		& ‘or’		&			&\CH			\\
%\Red{vala}	&			&\CH	 	&but	\Red{(only in bible reading!)}\\
%gen		&			&			&either	\\%JW: not a conjunction!?! oder?
%ahte		&			&\CH		&in order to	\\&JW: not attested in corpus
\It{åvdål}		& ‘before’	&			&\CH			\\\mybottomrule%JW: conjunction??
\end{tabular}
%\begin{tabular}{l  c  c  l}\mytoprule
%			&\MC{2}{c}{{connects}}		&{} \\
%{}	&{phrases}	&{clauses}	&{}	\\\hline
%\It{att	}		&			&\CH		& ‘(in order) to’	\\
%\It{eller}		&\CH		&\CH		& ‘or’	\\
%\It{gu	}		&			&\CH		& ‘when’	\\
%\It{ja}			&\CH		&\CH		& ‘and’	\\
%\It{jala}		&\CH		&\CH		& ‘or’	\\
%\It{jus}		&			&\CH		& ‘if’	\\
%\It{maŋŋel}	&			&\CH		& ‘after’	\\%JW: conjunction??
%\It{men}		&			&\CH		& ‘but’	\\
%\It{vaj}		&			&\CH		& ‘or’	\\
%%\Red{vala}	&			&\CH	 	&but	\Red{(only in bible reading!)}\\
%%gen		&			&			&either	\\%JW: not a conjunction!?! oder?
%%ahte		&			&\CH		&in order to	\\&JW: not attested in corpus
%\It{åvdål}		&			&\CH		& ‘before’	\\%JW: conjunction??
%\dline
%\end{tabular}
\end{table}

Conjunctions connecting clauses are discussed in \SEC\ref{clausalCoordination} on coordination and \SEC\ref{clausalSubordination} on subordination. 
Conjunctions connecting phrases are briefly described here. 
NPs, APs and verbs can be connected to another phrase of the same type by a conjunction; however, it is not clear from the data whether PPs or AdvPs can be connected. 
Some examples can be found in \REF{conjunctionEx1} through \REF{conjunctionEx6}. 

In \REF{conjunctionEx1} and \REF{conjunctionEx2}, \It{ja} ‘and’ connects NPs and APs, respectively. 
\ea\label{conjunctionEx1}%andNPs
\glll	dájste åtjojmä mielkev, vuojav, vuostav ja biergov\\
	d-á-jste åtjo-jmä mielke-v vuoja-v vuosta-v ja biergo-v\\
	\Sc{dem}-\Sc{prox}-\Sc{elat.pl} get-\Sc{1pl.pst} milk-\Sc{acc.sg} butter-\Sc{acc.sg} cheese-\Sc{acc.sg} and meat-\Sc{acc.sg}\\\nopagebreak
\Transl{we got milk, butter, cheese and meat from these}{}	\Corpus{080825}{015}
\z
\ea\label{conjunctionEx2}%andAPs
\glll	buhtsu lä nav buojde ja tjábbe\\
	buhtsu lä nav buojde ja tjábbe\\
	reindeer\BS\Sc{nom.pl} be\BS\Sc{3pl.prs} so fat\BS\Sc{pl} and beautiful\BS\Sc{pl} \\\nopagebreak
\Transl{the reindeer are so fat and beautiful}{}	\Corpus{080703}{014}
%\ex\label{conjunctionEx3}%jalaNPs%JW: not really phrasal-conjunction here, but clausal!!
%\glll	muorajd jala majd ana?\\
%	muora-jd jala ma-jd ana\\
%	wood-\Sc{acc.pl} or what-\Sc{acc.pl} have\BS\Sc{2sg.prs}\\
%\Transl{wood or what do you have?}{}	\Corpus{090702.148}
\z

In \REF{conjunctionEx3}, \It{jala} ‘or’ connects numeral-APs, while in \REF{conjunctionEx4} it connects NPs. 
\ea\label{conjunctionEx3}%jalaNPs(num)
\glll	men mån jahkav gu lidjiv mån aktalåknelldje jala aktalåkvihta jáge…\\
	men mån jahka-v gu li-djiv mån akta-låk-nelldje jala akta-låk-vihta jáge\\
	but \Sc{1sg.nom} believe-\Sc{1sg.prs} when be-\Sc{1sg.pst} \Sc{1sg.nom} one-ten-four or one-ten-five year\BS\Sc{nom.pl}\\\nopagebreak
\Transl{but I believe when I was fourteen or fifteen years (old)…}{}	\Corpus{100404}{273-274}
\z
\ea\label{conjunctionEx4}%jalaNPs
\glll	válda káfav suhkorijn jala suhkorahta\\
	válda káfa-v suhkor-ijn jala suhkor-ahta\\
	take\BS\Sc{2sg.prs} coffee-\Sc{acc.sg} sugar-\Sc{com.sg} or sugar-\Sc{abess.sg}\\\nopagebreak
\Transl{do you take coffee with or without sugar?}{}	\CorpusE{110509b}{11m42s}
\z

The conjuction \It{jala} ‘or’ can also connect non-finite verbs, as in \REF{conjunctionEx5}. 
\ea\label{conjunctionEx5}%jalaVPs
\glll	ja dálasj ájgen ij almatj, aktak almatj danne vieso jala åro\\
	ja dálasj ájge-n ij almatj aktak almatj danne vieso jala åro\\
	and nowadays time-\Sc{iness.sg} \Sc{neg}\BS\Sc{3sg.prs} person\BS\Sc{nom.sg} none person\BS\Sc{nom.sg} there live\BS\Sc{conneg} or reside\BS\Sc{conneg}\\\nopagebreak
\Transl{and nowadays no one lives or resides there}{}	\Corpus{100310b}{131}
\z

Finally, in \REF{conjunctionEx6}, the loan conjunction \It{eller} ‘or’\footnote{< Swedish \It{eller} ‘or’.} connects NPs. 
\ea\label{conjunctionEx6}%ellerVPs
\glll	inijmä eller bårojmä sirijd ja láddagijd\\
	ini-jmä eller båro-jmä siri-jd ja láddagi-jd\\
	have-\Sc{1pl.pst} or eat-\Sc{1pl.pst} blueberry-\Sc{acc.pl} and cloudberry-\Sc{acc.pl}\\\nopagebreak
\Transl{we had or ate blueberries and cloudberries}{}	\Corpus{100310b}{035}
\z

%\vfill

\section{Interjections}\label{interjections}\index{interjections}\index{parts of speech!interjections}
An \Bf{interjection} is an individual word that is syntactically an utterance of its own at the same level as entire clauses. As such, interjections are not a part of another clause. Interjections often indicate a speaker’s feelings or attitude towards an event. The data in the corpus are quite limited, and it is beyond the scope of the current study to describe interjections in detail, so the 
list of interjections %, two of which are used as curse words, 
and their English translation equivalents provided in Table \vref{interjectionTable} is preliminary and subject to amendment pending future study. %A list of interjections %, two of which are used as curse words, 
%and their approximate English equivalents is provided in Table \vref{interjectionTable}. %Many of the interjections are borrowings from Swedish. 
Nonetheless, examples from the corpus of \It{nå} and \It{jå} are provided below. 
Note that the interjections \It{jå}, \It{nä}, \It{så}, \It{å}/\It{oj}, \It{mmm} and \It{jaha} are borrowings from Swedish. %, while  are also common in Swedish, but it. 

%\vfill

\begin{table}[h]\centering%
\caption[Some \PS\ interjections and their translation equivalents]{Some \PS\ interjections and their English translation equivalents}\label{interjectionTable}
\begin{tabular}{ll}\mytoprule
%\It{interjection}	&\It{translation equivalent}	\\\hline
\It{burist	} & ‘hello’ (greeting)	\\
\It{jaha		} & ‘ok, I see’	(understanding) \\
\It{jå		} & ‘yes, definitely’	\\%<Sv
\It{mmm	} & ‘hmmm’ (pondering)	\\
\It{nå		} & ‘well, yes’	\\
\It{nä		} & ‘no’	\\%<Sv
\It{så		} & ‘so’	\\%<Sv
\It{å\TILDE oj		} & ‘oh’ (surprise)	\\\mybottomrule
%oj	&oh!	\\
%guadna	&vagina	&dammit!	\\%\hline
%buohtja	&penis	&dammit!	\\%\hline
\end{tabular}
\end{table}
\FB
%\vfill

The interjection \It{nå} ‘well, yes, ok’ is very common in the corpus. It has at least three possible meanings. At the beginning of the conversation presented in \REF{particleEx3}, \It{nå} is a kind of declaration that a speaker is beginning to speak. As in the final utterance in this example, \It{nå} also indicates a switch to a new topic. 
\ea\label{particleEx3}
\glll	\Tn{A:} nå buris, Henning\\
	{} nå buris Henning\\
	{} well hello Henning\\\nopagebreak
\TranslMulti{well hello, Henning}{}\\	%\Corpus{}
\glll	\Tn{B:} nå, buris dä\\
	{} nå buris dä\\
	{} well hello then\\\nopagebreak
\TranslMulti{well hello there}{}\\	%\Corpus{}
\glll	\Tn{A:} nå, guste dån bådá?\\
	{} nå guste dån bådá\\
	{} well from.where \Sc{2sg.nom} come\BS\Sc{2sg.prs}\\\nopagebreak
\TranslMulti{well where are you coming from?}{}	\Corpus{080924}{001-003}
\z

%\vfill
%\clearpage

The interjection \It{nå} can also be a confirmation of the preceding utterance, as in \REF{particleEx4}.
\ea\label{particleEx4}
\glll	\Tn{A:} ja dä muv bena Rahka\\
	{} ja dä muv bena Rahka\\
	{} and then \Sc{1sg.gen} dog\BS\Sc{nom.sg} Rahka\\\nopagebreak
\TranslMulti{and then my dog Rahka}{}\\
\glll	\Tn{B:} nå, duv bena aj\\
	{} nå duv bena aj\\
	{} well \Sc{2sg.gen} dog\BS\Sc{nom.sg} also\\\nopagebreak
\TranslMulti{ok, your dog, too}{}	\Corpus{080924}{037-038}
\z

Finally, \It{nå} can be used to indicate that a speaker is finished speaking, usually at the conclusion of a narrative and after a pause. One example can be found in the narrative in the recording ‘\hyperlink{pit100404}{pit100404}’ between utterance ‘.324’ and utterance ‘.361’; due to space constraints, only the last two utterances of this long narrative are presented in \REF{noEx1}. %and \REF{noEx2}: %, but it ends with a short pause, then the particle \It{nå} at utterance ‘.361’. 
\ea\label{noEx1}
\glll	\Tn{A:} så dä måj mähtijmen, måj Jåsjåjn mähtijmen {\hspace{14pt}dä} tjuäjjgat dán Stuornjárga nalá.\\
	{} så dä måj mähti-jmen måj Jåsjå-jn mähti-jmen {\hspace{14pt}dä} tjuäjjga-t d-á-n Stuor-njárga nalá\\
	{} so then \Sc{1du.nom} can-\Sc{1du.pst} \Sc{1du.nom} Josh-\Sc{com.sg} can-\Sc{1du.pst} {\hspace{14pt}then} ski-\Sc{inf} \Sc{dem}-\Sc{prox}-\Sc{gen.sg} big-point\BS\Sc{gen.sg} onto\\\nopagebreak
\TranslMulti{…so then we were able to, Josh and I were able to ski up to this\\\nopagebreak\hspace{10pt} here Big Point}{}\\	%\Corpus{100404.360}
\glll	\Tn{B:} nå.\\
	{} nå\\
	{} well\\\nopagebreak
\TranslMulti{that’s all}{}	\Corpus{100404}{360-361}
\z


The interjection \It{jå} ‘yes’, also a Swedish borrowing,\footnote{< Swedish \It{jå}, a common pronunciation of \It{ja} ‘yes’.} 
is often used to confirm something, or contradict a negative utterance, as in \REF{particleEx7}, in which two speakers debate about what the correct \PS\ word for ‘esophagus’ is. 
\ea\label{particleEx7}
\glll	\Tn{A:} ‘tjåddjåk’, nä, ij lä ‘tjåddjåk’\\
	{} tjåddjåk nä ij lä tjåddjåk\\
	{} esophagus\BS\Sc{nom.sg} no \Sc{neg}\BS\Sc{3sg.prs} be\BS\Sc{conneg} esophagus\BS\Sc{nom.sg}\\\nopagebreak
\TranslMulti{‘tjåddjåk’, no, it’s not ‘tjåddjåk’}{}\\
\glll	\Tn{B:} jå, lä tjåddjåk\\
	{} jå lä tjåddjåk\\
	{} yes be\BS\Sc{3sg.prs} esophagus\BS\Sc{nom.sg}\\\nopagebreak
\TranslMulti{yes, it’s ‘tjåddjåk’}{}	\Corpus{080909}{115-116}
\z









%%%%%%% THIS IS NOT USED FOR THE ENTIRE COMPILATION, but only for individual chapters!!!!

\clearpage
\addcontentsline{toc}{chapter}{Bibliography}\label{Bibliography}
\bibliography{PiteGrammarBibSDL}%for bibtex
%\printbibliography%[title=Works Cited]%%for biber!






%%%NAME INDEX doesn’t work!?!? why???
\cleardoublepage\phantomsection%this allows hyperlink in ToC to work
\addcontentsline{toc}{chapter}{Name index}
\ohead{Name index}
\printindex[aut]

\cleardoublepage\phantomsection%this allows hyperlink in ToC to work
\addcontentsline{toc}{chapter}{Language index}
\ohead{Language index}
\printindex[lan]

\cleardoublepage\phantomsection%this allows hyperlink in ToC to work
\addcontentsline{toc}{chapter}{Subject index}
\ohead{Subject index}
\printindex


\end{document}