%\documentclass[ number=5
			   ,series=sidl
			   ,isbn=xxx-x-xxxxxx-xx-x
			   ,url=http://langsci-press.org/catalog/book/17
			   ,output=long   % long|short|inprep              
			   %,blackandwhite
			   %,smallfont
			   ,draftmode   
			  ]{LSP/langsci}                          

\usepackage{LSP/lsp-styles/lsp-gb4e}		% verhindert Komma bei mehrfachen Fußnoten?
                                                      
\usepackage{layout}
\usepackage{lipsum}

%%%% ABOVE FOR LangSciPress %%%%
%%%% ABOVE FOR LangSciPress %%%%
%%%% ABOVE FOR LangSciPress %%%%
\usepackage{libertine}%work-around solution for rendering problematic characters ʦ, ͡  (mostly in \textbf{})

\usepackage{longtable}%Double-lines (\hline\hline) aren’t typeset properly in ‘longtable’-environment across several pages! conflict with other package (maybe xcolor with option ‘tables’?)

\usepackage{multirow}

\usepackage{array} %allows, among other things, centering column content in a table while also specifying width, creates new column style "x" for center-alignment, "y" for right-alignment
\newcolumntype{x}[1]{>{\centering\hspace{0pt}}p{#1}}%
\newcolumntype{y}[1]{>{\raggedleft\hspace{0pt}}p{#1}}%

\usepackage[]{placeins}%using \FloatBarrier command, all floats still floating at that point will be typeset, and cannot cross that boundary. the option here \usepackage[section]{placeins} automatically adds \FloatBarrier to every \section command (only works for \section commands, nothing lower than that!)
%\usepackage{afterpage}%by using the command \afterpage{\clearpage}, all floats will appear, but no new page will be started, thus avoiding bad page breaks around floats

\usepackage{vowel} %for vowel space chart


%%%IS THIS NECESSARY??
%%%%following allows you to refer to footnotes (from http://anthony.liekens.net/index.php/LaTeX/MultipleFootnoteReferences)
%\newcommand{\footnoteremember}[2]{
%  \footnote{#2}
%  \newcounter{#1}
%  \setcounter{#1}{\value{footnote}}
%} \newcommand{\footnoterecall}[1]{
%  \footnotemark[\value{#1}]} 
%%%%previous allows you to refer to footnotes: use \footnoteremember{referenceText} in footnote, then \footnoterecall{referenceText} to refer.

\usepackage{tikz}%
\usetikzlibrary{plothandlers,matrix,decorations.text,shapes.arrows,shadows,chains,positioning,scopes}

\usepackage{synttree} %zeichnet linguistische Bäume
\branchheight{36pt}%sets height between rows in synttree

\usepackage{lscape}%used for landscape pages in index (list of recordings)

\usepackage{polyglossia}
\setmainlanguage{english}


%%%TAKE OUT FOR FINAL VERSION:
%%%TAKE OUT FOR FINAL VERSION:
%%%TAKE OUT FOR FINAL VERSION:

%%%%following readjusts margin text!
%\setlength{\marginparwidth}{20mm}
%\let\oldmarginpar\marginpar
%\renewcommand\marginpar[1]{\-\oldmarginpar[\raggedleft\footnotesize\vspace{-7pt}\color{red}\It{→ #1}]%
%{\raggedright\footnotesize\vspace{-7pt}\color{red}\It{→ #1}}}
%%%%previous readjusts margin text!

%%%The following lines set depth of ToC (LSP default is only 3 levels)!
%%%\renewcommand{\contentsname}{Table of Contents} % überschrift des inhaltsverzeichnisses
%\setcounter{secnumdepth}{5}%sets how deep section/subsection/subsubsections are numbered
%\setcounter{tocdepth}{5}%sets the depth of the ToC %but this doesn't seem to work!!!
%% new commands for LSP book (Grammar of Pite Saami, by J. Wilbur)

\newcommand{\PS}{Pite Saami}
\newcommand{\PSDP}{Pite Saami Documentation Project}
\newcommand{\WLP}{Wordlist Project}

\newcommand{\HANG}{\everypar{\hangindent15pt \hangafter1}}%also useful for table cells
\newcommand{\FB}{\FloatBarrier}%shortcut for this command to print all floats w/o pagebreak

\newcommand{\REF}[1]{(\ref{#1})}%adds parenthesis around the reference number, particularly useful for examples.%\Ref had clash with LSP!
\newcommand{\dline}{\hline\hline}%makes a double line in a table
\newcommand{\superS}[1]{\textsuperscript{#1}}%adds superscript element
\newcommand{\sub}[1]{$_{#1}$}%adds subscript element
\newcommand{\Sc}[1]{\textsc{#1}}%shortcut for small capitals (not to be confused with \sc, which changes the font from that point on)
\newcommand{\It}[1]{\textit{#1}}%shortcut for italics (not to be confused with \it, which changes the font from that point on)
\newcommand{\Bf}[1]{\textbf{#1}}%shortcut for bold (not to be confused with \bf, which changes the font from that point on)
\newcommand{\BfIt}[1]{\textbf{\textit{#1}}}
\newcommand{\BfSc}[1]{\textbf{\textsc{#1}}}
\newcommand{\Tn}[1]{\textnormal{#1}}%shortcut for normal text (undo italics, bolt, etc.)
\newcommand{\MC}{\multicolumn}%shortcut for multicolumn command in tabular environment - only replaces command, not variables!
\newcommand{\MR}{\multirow}%shortcut for multicolumn command in tabular environment - only replaces command, not variables!
\newcommand{\TILDE}{∼}%U+223C %OLD:~}%shortcut for tilde%command ‘\Tilde’ clashes with LSP!%
\newcommand{\BS}{\textbackslash}%backslash
\newcommand{\Red}[1]{{\color{red}{#1}}}%for red text
\newcommand{\Blue}[1]{{\color{blue}{#1}}}%for blue text
\newcommand{\PLUS}{+}%nicer looking plus symbol
\newcommand{\MINUS}{-}%nicer looking plus symbol
%    Was die Pfeile betrifft, kannst Du mal \Rightarrow \mapsto \textrightarrow probieren und dann \mathbf \boldsymbol oder \pbm dazutun.
\newcommand{\ARROW}{\textrightarrow}%→%dieser dicke Pfeil ➜ wird nicht von der LSP-Font unterstützt: %\newcommand{\ARROW}{{\fontspec{DejaVu Sans}➜}}
\newcommand{\DARROW}{\textleftrightarrow}%↔︎%DoubleARROW
\newcommand{\BULLET}{•}%
%%✓ does not exist in the default LSP font!
\newcommand{\CH}{\checkmark}%%\newcommand{\CH}{\fontspec{Arial Unicode MS}✓}%CH as in CHeck
%%following used to separate alternation forms for consonant gradation and umlaut patterns:
\newcommand{\Div}{‑}%↔︎⬌⟷⬄⟺⇔%non-breaking hyphen: ‑  
\newcommand{\QUES}{\textsuperscript{?}}%marks questionable/uncertain forms

\newcommand{\jvh}{\mbox{\It{j}-suffix} vowel harmony}%
%\newcommand{\Ptcl}{\Sc{ptcl} }%just shortcut for glossing ‘particle’
%\newcommand{\ATTR}{{\Sc{attributive}}}%shortcut for ATTRIBUTIVE in small caps
%\newcommand{\PRED}{{\Sc{predicative}}}%shortcut for PREDICATIVE in small caps
%\newcommand{\COMP}{{\Sc{comparative}}}%shortcut for COMPARATIVE in small caps
%\newcommand{\SUPERL}{{\Sc{superlative}}}%shortcut for SUPERLATIVE in small caps
\newcommand{\SG}{{\Sc{singular}}}%shortcut for SINGULAR in small caps
\newcommand{\DU}{{\Sc{dual}}}%shortcut for DUAL in small caps
\newcommand{\PL}{{\Sc{plural}}}%shortcut for PLURAL in small caps
%\newcommand{\NOM}{{\Sc{nominative}}}%shortcut for NOMINATIVE in small caps
%\newcommand{\ACC}{{\Sc{accusative}}}%shortcut for ACCUSATIVE in small caps
%\newcommand{\GEN}{{\Sc{genitive}}}%shortcut for GENITIVE in small caps
%\newcommand{\ILL}{{\Sc{illative}}}%shortcut for ILLATIVE in small caps
%\newcommand{\INESS}{{\Sc{inessive}}}%shortcut for INESSIVE in small caps
\newcommand{\ELAT}{{\Sc{elative}}}%shortcut for ELATIVE in small caps
%\newcommand{\COM}{{\Sc{comitative}}}%shortcut for COMITATIVE in small caps
%\newcommand{\ABESS}{{\Sc{abessive}}}%shortcut for ABESSIVE in small caps
%\newcommand{\ESS}{{\Sc{essive}}}%shortcut for ESSIVE in small caps
%\newcommand{\DIM}{{\Sc{diminutive}}}%shortcut for DIMINUTIVE in small caps
%\newcommand{\ORD}{{\Sc{ordinal}}}%shortcut for ORDINAL in small caps
%\newcommand{\CARD}{{\Sc{cardinal}}}%shortcut for CARDINAL in small caps
%\newcommand{\PROX}{{\Sc{proximal}}}%shortcut for PROXIMAL in small caps
%\newcommand{\DIST}{{\Sc{distal}}}%shortcut for DISTAL in small caps
%\newcommand{\RMT}{{\Sc{remote}}}%shortcut for REMOTE in small caps
%\newcommand{\REFL}{{\Sc{reflexive}}}%shortcut for REFLEXIVE in small caps
%\newcommand{\PRS}{{\Sc{present}}}%shortcut for PRESENT in small caps
%\newcommand{\PST}{{\Sc{past}}}%shortcut for PAST in small caps
%\newcommand{\IMP}{{\Sc{imperative}}}%shortcut for IMPERATIVE in small caps
%\newcommand{\POT}{{\Sc{potential}}}%shortcut for POTENTIAL in small caps
\newcommand{\PROG}{{\Sc{progressive}}}%shortcut for PROGRESSIVE in small caps
\newcommand{\PRF}{{\Sc{perfect}}}%shortcut for PERFECT in small caps
\newcommand{\INF}{{\Sc{infinitive}}}%shortcut for INFINITIVE in small caps
%\newcommand{\NEG}{{\Sc{negative}}}%shortcut for NEGATIVE in small caps
\newcommand{\CONNEG}{{\Sc{connegative}}}%shortcut for CONNEGATIVE in small caps
\newcommand{\ATTRs}{{\Sc{attr}}}%shortcut for ATTR in small caps
\newcommand{\PREDs}{{\Sc{pred}}}%shortcut for PRED in small caps
%\newcommand{\COMPs}{{\Sc{comp}}}%shortcut for COMP in small caps
%\newcommand{\SUPERLs}{{\Sc{superl}}}%shortcut for SUPERL in small caps
\newcommand{\SGs}{{\Sc{sg}}}%shortcut for SG in small caps
\newcommand{\DUs}{{\Sc{du}}}%shortcut for DU in small caps
\newcommand{\PLs}{{\Sc{pl}}}%shortcut for PL in small caps
\newcommand{\NOMs}{{\Sc{nom}}}%shortcut for NOM in small caps
\newcommand{\ACCs}{{\Sc{acc}}}%shortcut for ACC in small caps
\newcommand{\GENs}{{\Sc{gen}}}%shortcut for GEN in small caps
\newcommand{\ILLs}{{\Sc{ill}}}%shortcut for ILL in small caps
\newcommand{\INESSs}{{\Sc{iness}}}%shortcut for INESS in small caps
\newcommand{\ELATs}{{\Sc{elat}}}%shortcut for ELAT in small caps
\newcommand{\COMs}{{\Sc{com}}}%shortcut for COM in small caps
\newcommand{\ABESSs}{{\Sc{abess}}}%shortcut for ABESS in small caps
\newcommand{\ESSs}{{\Sc{ess}}}%shortcut for ESS in small caps
%\newcommand{\DIMs}{{\Sc{dim}}}%shortcut for DIM in small caps
%\newcommand{\ORDs}{{\Sc{ord}}}%shortcut for ORD in small caps
%\newcommand{\CARDs}{{\Sc{card}}}%shortcut for CARD in small caps
\newcommand{\PROXs}{{\Sc{prox}}}%shortcut for PROX in small caps
\newcommand{\DISTs}{{\Sc{dist}}}%shortcut for DIST in small caps
\newcommand{\RMTs}{{\Sc{rmt}}}%shortcut for RMT in small caps
\newcommand{\REFLs}{{\Sc{refl}}}%shortcut for REFL in small caps
\newcommand{\PRSs}{{\Sc{prs}}}%shortcut for PRS in small caps
\newcommand{\PSTs}{{\Sc{pst}}}%shortcut for PST in small caps
\newcommand{\IMPs}{{\Sc{imp}}}%shortcut for IMP in small caps
\newcommand{\POTs}{{\Sc{pot}}}%shortcut for POT in small caps
\newcommand{\PROGs}{{\Sc{prog}}}%shortcut for PROG in small caps
\newcommand{\PRFs}{{\Sc{prf}}}%shortcut for PRF in small caps
\newcommand{\INFs}{{\Sc{inf}}}%shortcut for INF in small caps
\newcommand{\NEGs}{{\Sc{neg}}}%shortcut for NEG in small caps
\newcommand{\CONNEGs}{{\Sc{conneg}}}%shortcut for CONNEG in small caps

\newcommand{\subNP}{{\footnotesize\sub{NP}}}%shortcut for NP (nominal phrase) in subscript
\newcommand{\subVC}{{\footnotesize\sub{VC}}}%shortcut for VC (verb complex) in subscript
\newcommand{\subAP}{{\footnotesize\sub{AP}}}%shortcut for NP (adjectival phrase) in subscript
\newcommand{\subAdvP}{{\footnotesize\sub{AdvP}}}%shortcut for AdvP (adverbial phrase) in subscript
\newcommand{\subPP}{{\footnotesize\sub{PP}}}%shortcut for NP (postpoistional phrase) in subscript

\newcommand{\ipa}[1]{{\fontspec{Linux Libertine}#1}}%specifying font for IPA characters

\newcommand{\SEC}{§}%standardize section symbol and spacing afterwards
%\newcommand{\SEC}{§\,}%

\newcommand{\Nth}{{\footnotesize(\It{n})}}%used in table of numerals in ADJ chapter

%%newcommands for tables in introductionSDL.tex:
\newcommand{\cliticExs}[3]{\Tn{\begin{tabular}{p{28mm} c p{28mm} p{35mm}}\It{#1}&\ARROW &\It{#2} & ‘#3’\\\end{tabular}}}%specifically for the two clitic examples
\newcommand{\Grapheme}[1]{\It{#1}}%formatting for graphemes in orthography tables
%%new command for the section on orthographic examples; syntax: #1=orthography, #2=phonology, #3=gloss
\newcommand{\SpellEx}[3]{\Tn{\begin{tabular}{p{70pt} p{70pt} l}\ipa{/#2/}&\It{#1}& ‘#3’ \\\end{tabular}}}%formatting for orthographic examples (intro-Chapter)


%%new transl tier in gb4e; syntax: #1=free translation (in single quotes), #2=additional comments, z.B. literal meaning:
\newcommand{\Transl}[2]{\trans\Tn{‘#1’ #2}}%new transl tier in gb4e;
\newcommand{\TranslMulti}[2]{\trans\hspace{12pt}\Tn{‘#1’ #2}}%new transl tier in gb4e for a dialog to be included under a single example number


%% used for examples in the Prosody and Segmental phonology chapters:
\newcommand{\PhonGloss}[7]{%PhonGloss = Phonology Gloss;
%pattern: \PhonGloss{label}{phonemic}{phonetic}{orthographic}{gloss}{recording}{utterance}
\ea\label{#1}
\Tn{\begin{tabular}[t]{p{30mm} l}
\ipa{/#2/}	& \It{#4} \\
\ipa{[#3]}	&\HANG ‘#5’\\%no table row can start with square brackets! thus the workaround with \MC
\end{tabular}\hfill\hyperlink{#6}{{\small\textnormal[pit#6#7]}}%\index{Z\Red{rec}!\Red{pit#6}}\index{Z\Red{utt}!\Red{pit#6#7} \Blue{Phon}}
}
\z}
\newcommand{\PhonGlossWL}[6]{%PhonGloss = Phonology Gloss for words from WORDLIST, not from corpus!;
%pattern: \PhonGloss{label}{phonemic}{phonetic}{orthographic}{gloss}{wordListNumber}
\ea\label{#1}
\Tn{\begin{tabular}[t]{p{30mm} l}
\ipa{/#2/}	& \It{#4} \\
\ipa{[#3]}	&\HANG ‘#5’\\%no table row can start with square brackets! thus the workaround with \MC
\end{tabular}\hfill\hyperlink{explExs}{{\small\textnormal[#6]}}%\index{Z\Red{wl}!\Red{#6}\Blue{Phon}}
}
\z}

%%for derivation examples in the derivational morphology chapter!
%syntax: \DerivExam{#1}{#2}{#3}{#4}{#5}{#6}
%#1: base, #2: base-gloss, #3: derived form, #4: derived form gloss, #5: derived form translation, #6: pit-recording, #7: utterance number
\newcommand{\DW}{28mm}%for following three commands, to align arrows throughout
%%%%OLD:
%%%\newcommand{\DerivExam}[7]{\Tn{\begin{tabular}[t]{p{\DW}cl}\It{#1}&\ARROW&\It{#3}\\#2&&#4\\\end{tabular}\hfill\pbox{.3\textwidth}{\hfill‘#5’\\\hbox{}\hfill\hyperlink{pit#6}{{\small\textnormal[pit#6.#7]}}}
%%%%\index{Z\Red{rec}!\Red{pit#6}}\index{Z\Red{utt}!\Red{pit#6.#7}}
%%%}}
%NEW:
\newcommand{\DerivExam}[7]{\Tn{
\begin{tabular}[t]{p{\DW}x{5mm}l}\It{#1}&\ARROW&\It{#3}\\\end{tabular}\hfill‘#5’\\
\hspace{1mm}\begin{tabular}[t]{p{\DW}x{5mm}l}#2&&#4\\\end{tabular}\hfill\hyperlink{pit#6}{{\small\textnormal[pit#6.#7]}}
%\index{Z\Red{rec}!\Red{pit#6}}\index{Z\Red{utt}!\Red{pit#6.#7}}
}}
%%same as above, but supress any reference to a specific utterance
\newcommand{\DerivExamX}[7]{\Tn{
\begin{tabular}[t]{p{\DW}x{5mm}l}\It{#1}&\ARROW&\It{#3}\\\end{tabular}\hfill‘#5’\\
\hspace{1mm}\begin{tabular}[t]{p{\DW}x{5mm}l}#2&&#4\\\end{tabular}\hfill\hyperlink{pit#6}{{\small\textnormal[pit#6]\It{e}}}
%\index{Z\Red{rec}!\Red{pit#6}}\index{Z\Red{utt}!\Red{pit#6.#7}}
}}
\newcommand{\DerivExamWL}[6]{\Tn{
\begin{tabular}[t]{p{\DW}x{5mm}l}\It{#1}&\ARROW&\It{#3}\\\end{tabular}\hfill‘#5’\\
\hspace{1mm}\begin{tabular}[t]{p{\DW}x{5mm}l}#2&&#4\\\end{tabular}\hfill\hyperlink{explExs}{{\small\textnormal[#6]}}
%\index{Z\Red{wl}!\Red{#6}}
}}


%formatting of corpus source information (after \transl in gb4e-environments):
\newcommand{\Corpus}[2]{\hspace*{1pt}\hfill{\small\mbox{\hyperlink{pit#1}{\Tn{[pit#1.#2]}}}}%\index{Z\Red{rec}!\Red{pit#1}}\index{Z\Red{utt}!\Red{pit#1.#2}}
}%
\newcommand{\CorpusE}[2]{\hspace*{1pt}\hfill{\small\mbox{\hyperlink{pit#1}{\Tn{[pit#1.#2]}}\It{e}}}%\index{Z\Red{rec}!\Red{pit#1}}\index{Z\Red{utt}!\Red{pit#1.#2}\Blue{-E}}
}%
%%as above, but necessary for recording names which include an underline because the first variable in \href understands _ but the second variable requires \_
\newcommand{\CorpusLink}[3]{\hspace*{1pt}\hfill{\small\mbox{\hyperlink{pit#1}{\Tn{[pit#2.#3]}}}}%\index{Z\Red{rec}!\Red{pit#2}}\index{Z\Red{utt}!\Red{pit#2.#3}}
}%
%%as above, but for newer recordings which begin with sje20 instead of pit
\newcommand{\CorpusSJE}[2]{\hspace*{1pt}\hfill{\small\mbox{\hyperlink{sje20#1}{\Tn{[sje20#1.#2]}}}}%\index{Z\Red{rec}!\Red{sje20#1}}\index{Z\Red{utt}!\Red{sje20#1.#2}}
}%
\newcommand{\CorpusSJEE}[2]{\hspace*{1pt}\hfill{\small\mbox{\hyperlink{sje20#1}{\Tn{[sje20#1.#2]}}\It{e}}}%\index{Z\Red{rec}!\Red{sje20#1}}\index{Z\Red{utt}!\Red{sje20#1.#2}\Blue{-E}}
}%











%%hyphenation points for line breaks
%%add to TeX file before \begin{document} with:
%%%%hyphenation points for line breaks
%%add to TeX file before \begin{document} with:
%%%%hyphenation points for line breaks
%%add to TeX file before \begin{document} with:
%%\include{hyphenationSDL}
\hyphenation{
ab-es-sive
affri-ca-te
affri-ca-tes
Ahka-javv-re
al-ve-o-lar
com-ple-ments
%check this:
de-cad-es
fri-ca-tive
fri-ca-tives
gemi-nate
gemi-nates
gra-pheme
gra-phemes
ho-mo-pho-nous
ho-mor-ga-nic
mor-pho-syn-tac-tic
or-tho-gra-phic
pho-neme
pho-ne-mes
phra-ses
post-po-si-tion
post-po-si-tion-al
pre-as-pi-ra-te
pre-as-pi-ra-ted
pre-as-pi-ra-tion
seg-ment
un-voiced
wor-king-ver-sion
}
\hyphenation{
ab-es-sive
affri-ca-te
affri-ca-tes
Ahka-javv-re
al-ve-o-lar
com-ple-ments
%check this:
de-cad-es
fri-ca-tive
fri-ca-tives
gemi-nate
gemi-nates
gra-pheme
gra-phemes
ho-mo-pho-nous
ho-mor-ga-nic
mor-pho-syn-tac-tic
or-tho-gra-phic
pho-neme
pho-ne-mes
phra-ses
post-po-si-tion
post-po-si-tion-al
pre-as-pi-ra-te
pre-as-pi-ra-ted
pre-as-pi-ra-tion
seg-ment
un-voiced
wor-king-ver-sion
}
\hyphenation{
ab-es-sive
affri-ca-te
affri-ca-tes
Ahka-javv-re
al-ve-o-lar
com-ple-ments
%check this:
de-cad-es
fri-ca-tive
fri-ca-tives
gemi-nate
gemi-nates
gra-pheme
gra-phemes
ho-mo-pho-nous
ho-mor-ga-nic
mor-pho-syn-tac-tic
or-tho-gra-phic
pho-neme
pho-ne-mes
phra-ses
post-po-si-tion
post-po-si-tion-al
pre-as-pi-ra-te
pre-as-pi-ra-ted
pre-as-pi-ra-tion
seg-ment
un-voiced
wor-king-ver-sion
}\begin{document}\tableofcontents\clearpage

%%%%%%%%%%%%%%%%%%%%%%%%%%%%%%%%% ALL THE ABOVE TO BE COMMENTED OUT FOR COMPLETE DOCUMENT! %%%%%%%%%%%


\chapter{Derivational morphology}\label{derivMorph}
\PS\ is rich in derivational morphology. While it is beyond the scope of the present work to provide a thorough description of all the various derivational processes and of their semantic nuances and productivity,\footnote{Israel Ruong (himself a native speaker of \PS) dedicated his entire PhD thesis to verbal derivation in \PS\ \citep{Ruong1943}.} %: \It{Lappische Verbalableitung dargestellt auf Grundlage des Pitelappischen} ‘Saami Verbal Derivation as illustrated by the \PS\ language’ (my translation).} 
the following should provide a general impression of how derivational morphology works in \PS, as well as an overview of some of the more common derivational morphemes attested in the corpus and extant in the wordlist compiled by the \WLP\ (\cite{insamlingPS2011}; cf. Section \ref{orthography}). 

In the following, derivational meanings are assigned to suffixes for simplicity in classification; however, as with inflectional suffixes, derivational suffixes coincide with non-linear morphology when the derivational base is subject to non-linear morphological alternations. %As can be expected cross-linguistically, derivational suffixes occur before inflectional
%It goes without saying that only open word classes (nouns (but not the closed nominal sub-category of pronouns), verbs, adjectives and adverbs) can be derived. 
There are many nominalizing and verbalizing derivational processes, and derivations can apply to already derived forms. 
%It is not always possible to unambiguously determine the base that a derived lexeme is based on. 
On the other hand, there are only two adjectivalizers and one adverbializer. 

Nominal derivation and verbal derivation are especially complex because the semantics of a derived word do not consistently equal the sum of the meanings of its components. Furthermore, the borderline between polysemy and homonymy of suffixes cannot always be clearly determined, and the decision whether two formally identical, but semantically different forms should be ascribed to the same morpheme or to distinct morphemes is not always obvious. 
This is reflected in the glossing standards used here in which most nominalizers and verbalizers are simply allotted numbers, as in \Sc{nmlz1} or \Sc{vblz3}, as opposed to more meaningful glosses such as \Sc{dim}. 

In the following, nominal derivation is dealt with first, in Section \ref{nDerivation}, before moving on to verbal derivation %(including passives) 
in \ref{vDerivation}, while adjectival and adverbial derivation are described briefly in Sections \ref{adjDerivation} and \ref{advDerivation}, respectively. The final section (\ref{derivMorphSummary}) provides a summary of the derivational morphemes discussed here. 

Note that examples in the present chapter include references to either the documentation corpus or an entry in the database from the \WLP\ \citep{insamlingPS2011}. Nearly all references to the documentation corpus are for elicitation sessions, and these are marked accordingly. Unlike examples in the other chapters, in which references indicate a particular utterance of a recording, references here may not be not more specific than the recording name alone because the relevant data were obtained during the course of a longer discussion, and not just in a single utterance. References referring to the \WLP’s database consist only of the four-digit entry number.  


\section{Nominal derivation}\label{nDerivation}
Nouns can be derived from verbs, adjectives, or other nouns. Some of the more common derivational suffixes are \It{-tj}, \It{-k}, \It{-o}, \It{-däddje}, \It{-vuohta}, and these are discussed in the following sections. The bases they can be applied to are summarized in Table \vref{NMLZtable}. 
\begin{table}\centering
\caption{Summary of the types of bases accessible to the nominal derivation suffixes discussed here}\label{NMLZtable}
\begin{tabular}{|c|ccc|}\hline
		&\MC{3}{c|}{\It{type of base}}		\\
\It{suffix}	&\It{nominal}&\It{verbal}&\It{adjectival}	\\\dline
-tj		&\CH	&		&			\\\hline
-k		&\CH	&\CH	&\CH		\\\hline
-o		&		&\CH	&			\\\hline
-däddje	&\CH	&\CH	&			\\\hline
-vuohta	&\CH	&		&\CH		\\\hline
\end{tabular}
\end{table}


\subsection{The diminutive suffix \It{-tj}}\label{dim}\index{diminutive}
The diminutive suffix \It{-tj} (glossed as \Sc{dim}) can be affixed to a nominal base to form a denominal noun with a diminutive meaning. Examples can be found in \REF{DIMex1} through \REF{DIMex6}. 
\ea\label{DIMex1}
\DerivExamX{vájbmo}{heart\BS\Sc{nom.sg}}{vájmu-tj}{heart-\Sc{dim}\BS\Sc{nom.sg}}{little heart}{110413a}{238}
\z
\ea\label{DIMex2}
\DerivExamX{guolla}{testicle\BS\Sc{nom.sg}}{guola-tj}{testicle-\Sc{dim}\BS\Sc{nom.sg}}{little testicle}{110413a}{150}
\z
\ea\label{DIMex3}
\DerivExamX{guolle}{fish\BS\Sc{nom.sg}}{guolá-tj}{fish-\Sc{dim}\BS\Sc{nom.sg}}{little fish}{110413a}{067}
\z
\ea\label{DIMex4}
\DerivExamX{båtsoj}{reindeer\BS\Sc{nom.sg}}{buhtsu-tj}{reindeer-\Sc{dim}\BS\Sc{nom.sg}}{little reindeer}{110413b}{118}
\z
\ea\label{DIMex5}
\DerivExamX{sabek}{ski\BS\Sc{nom.sg}}{sabega-tj}{ski-\Sc{dim}\BS\Sc{nom.sg}}{little ski}{090525b}{01m05s}
\z
\ea\label{DIMex6}
\DerivExamX{bena}{dog\BS\Sc{nom.sg}}{bednaga-tj}{dog-\Sc{dim}\BS\Sc{nom.sg}}{little dog}{080819a}{119}
\z

The diminutive form features the same stem as found in the \NOMs.\PLs\ form of a noun paradigm (differences in the segments occurring between the consonant center and the right edge of the nominal base in these examples are due to regular alternations in inflectional noun class suffixes on the base). %and are not indicated in the morphological structure or in the glosses in order to keep the glosses from getting overly complicated. 
The resulting diminutive nouns are class IIIa nouns; a nearly complete paradigm for the derived noun \It{guolátj} ‘little fish’ is provided \marginpar{check refs are correct here!}in Table \vref{fishDIMparadigm} in Section \ref{NclassIII} in the chapter on nouns. Note also that there is a diminutive verbalizer suffix \It{-tj}; cf. Section \ref{verbDIM}. 


\subsection{The general nominalizer suffix \It{-k}}\label{nmlz2}\index{nominalizer}
The nominalizer suffix \It{-k} (spelled \It{-g-} intervocalically; glossed as \Sc{nmlz}\footnote{Due to its frequency and extensive use as a general nominalizer, the nominalizer \It{-k} is glossed simply as \Sc{nmlz} without any additional number to specify it, unlike the other, less frequent nominalizers described in sections \ref{nmlz1} through \ref{vuohta}.}) 
can be affixed to a noun, a verb or an adjective. The resulting derived nouns have a wide variety of meanings, but generally have a referent which is someone or something with a property referred to by the base. %, but in general refers to a person or thing somehow related to the base for derivation. 
A number of examples are provided in \REF{NMLZ2ex1} through \REF{NMLZ2ex9} below, but this is hardly an exhaustive sample. %generally indicates ?. 

In \REF{NMLZ2ex1} the derived noun \It{guhkajuolgagijd} ‘long-legger-\Sc{acc.pl}’ is based on a compound noun \It{guhka-juällge} ‘long-leg’, and is used several times in the corpus to refer to moose. 
\ea\label{NMLZ2ex1}
\glll	vuojdne edna guhkajuolgagijd?\\
	vuojdne edna guhka-juolga-g-ijd\\
	see\BS\Sc{2sg.pst} some long-leg-\Sc{nmlz}-\Sc{acc.pl}\\\nopagebreak
\trans 	‘did you see some moose?’	\Corpus{080924}{007}
\z
The derivation of the base compound’s head \It{guhka-juällge} ‘long-leg’ into the derived form is illustrated in \REF{NMLZ2ex2}. 
\ea\label{NMLZ2ex2}
\DerivExam{guhka-juällge}{long-leg\BS\Sc{nom.sg}}{[guhka-juolga]-k}{[long-leg]-\Sc{nmlz}\BS\Sc{nom.sg}}{long-legger}{080924}{007}
\z

Two other examples of denominal nouns derived with \It{-k} are the word \It{jagak} ‘yearling, one-year-old’, which is derived from the nominal base \It{jahke} ‘year’, as given in \REF{NMLZ2ex3a}, and the word \It{nástak} ‘police’, which is derived from the nominal base \It{násste} ‘star’: %, as shown in \REF{NMLZ2ex3c}. 
\ea\label{NMLZ2ex3a}
\DerivExamWL{jahke}{year\BS\Sc{nom.sg}}{jaga-k}{year-\Sc{nmlz}\BS\Sc{nom.sg}}{yearling}{4911}
\z
\ea\label{NMLZ2ex3c}
\DerivExamWL{násste}{star\BS\Sc{nom.sg}}{násta-k}{star-\Sc{nmlz}\BS\Sc{nom.sg}}{police, police officer}{1249}
\z

The word \It{máhtak} ‘knowledgable person’ is derived from the verb \mbox{\It{máhttet}} ‘can’: %, as shown in \REF{NMLZ2ex3b}. 
\ea\label{NMLZ2ex3b}
\DerivExamWL{máhtte-t}{can-\Sc{inf}}{máhta-k}{can-\Sc{nmlz}\BS\Sc{nom.sg}}{knowledgable person}{1110}
\z

The word \It{villguk} ‘white reindeer’ is based on a form \It{vällg-} ‘white’ (cf.\,the attributive adjective \It{villgis} and predicative adjective \It{vällgat} ‘white’): %, as shown in \REF{NMLZ2ex4}. %\marginpar{this maybe needs re-writing - what is the base for derivation?} 
\ea\label{NMLZ2ex4}
\DerivExamWL{vällg-}{white}{villgu-k}{white-\Sc{nmlz}\BS\Sc{nom.sg}}{white reindeer}{2219}
\z
%It is likely the case that the stem \It{villg-} never occurs without some ending, such as \It{-s} in the attributive adjective form and \It{-t} in the predicative adjective form. %In this case, the nominalizer \It{-k} seems to trigger the weak consonant gradation allomorph \It{vilg-}.

Similarly, the word \It{suojmek} ‘slow person’ is derived from the stem \It{suajbm-} ‘slow’ (cf. the attributive adjective \It{suojmas} and the predicative adjective \mbox{\It{suajbma}} ‘slow’): %, as shown in \REF{NMLZ2ex5}. 
\ea\label{NMLZ2ex5}
\DerivExamWL{suajbm-}{slow}{suojmek}{slow-\Sc{nmlz}\BS\Sc{nom.sg}}{slow person}{2650}
\z

The word \It{vidak} ‘five-crown coin’ is derived from the numeral base \It{vihta} ‘five’, while \It{vidalågåk} ‘fifty-crown bank note’ is derived from the numeral base \It{vidalåhkå} ‘fifty’: %These are summarized in \REF{NMLZ2ex6} and \REF{NMLZ2ex7}. 
\ea\label{NMLZ2ex6}
\DerivExamWL{vihta}{five}{vida-k}{five-\Sc{nmlz}\BS\Sc{nom.sg}}{five-crown coin}{4051}
\z
\ea\label{NMLZ2ex7}
\DerivExamWL{vidalåhkå}{fifty}{vidalågå-k}{fifty-\Sc{nmlz}\BS\Sc{nom.sg}}{fifty-crown banknote}{4053}
\z

There are still other cases of nouns derived by \It{-k} which are based on a verb form at the deepest level, but feature subsequent derivational affixes between the root lexeme and the final nominalizing suffix \It{-k}, so that it is not clear what base lexeme \It{-k} is directly attached to. For instance, the noun \It{gaskaldak} ‘bite, bit’ is ultimately based on the verb \It{gassket} ‘bite’, but it is not clear what the derivational affix or affixes expressed by the segments \It{-alda-} could indicate: %. This is illustrated in \REF{NMLZ2ex8}. 
\ea\label{NMLZ2ex8}
\DerivExamWL{gasske-t}{bite-\Sc{inf}}{gask-alda-k}{bite-\Sc{vblz}?-\Sc{nmlz}\BS\Sc{nom.sg}}{bite, bit}{3278}
\z

Finally, several place names are also likely derived using \It{-k}. For instance, the name \It{Tjårvek} ‘Lake Hornavan’\footnote{\It{Tjårvek} is the large lake on which Arjeplog, the main \PS\ community, is located. Even the Swedish name \BfIt{Horn}\It{avan} seems to refer to antlers or horns, and could be the result of a loan translation.} 
is derived from the nominal base \It{tjårrve} ‘horn, antler’: %, as summarized in \REF{NMLZ2ex9}. 
\ea\label{NMLZ2ex9}
\DerivExam{tjårrve}{antler\BS\Sc{nom.sg}}{tjårve-k}{antler-\Sc{nmlz}\BS\Sc{nom.sg}}{Lake Hornavan}{110517b2}{083}
\z

It is clear that, morphophonologically, the suffix \It{-k} triggers the weak consonant grade, when applicable. The vowel immediately preceding the suffix \It{-k} is not consistent; this could be due to various noun class bases, or perhaps there are in fact more than one nominalization suffixes of the form \It{-Vk}. These questions as well as other questions concerning the variety of uses of this very flexible and common derivational morpheme must be left for future study. 
%Future research could shed light on the variety of uses for the very flexible derivational morpheme \It{-k}. 



\subsection{The action nominalizer suffix \It{-o}}\label{nmlz1}\index{nominalizer}
%UM comment: instead of ‘action’, better ‘state of affairs’ because ‘dry’ is not an action; JW: but ‘dry’ is transitive, thus an action
The nominalizer suffix \It{-o} (glossed as \Sc{nmlz1}) can be affixed to a verbal base to form a deverbal noun. %\marginpar{re-write this description?} %UM didn’t seem to think this is nec.
In general, the resulting noun refers to the action or the result of the action denoted by the stem, as in \REF{NMLZ1ex1} through \REF{NMLZ1ex7}. 
\ea\label{NMLZ1ex1}
\DerivExamX{barrga-t}{work-\Sc{inf}}{barrg-o}{work-\Sc{nmlz1}\BS\Sc{nom.sg}}{job, work}{110404}{320}
\z
%\ea\label{NMLZ1ex2}
%\DerivExam{gåjjkåt}{dry-\Sc{inf}}{gåjjk-o}{dry-\Sc{nmlz}\BS\Sc{nom.sg}}{drought; thirst}{?}
%\z
%\ea\label{NMLZ1ex3a}
%\DerivExam{jáhkket}{believe-\Sc{inf}}{jáhk-o}{believe-\Sc{nmlz}\BS\Sc{nom.sg}}{belief}{?}
%\z
\ea\label{NMLZ1ex3b}
\DerivExamWL{bivvde-t}{catch-\Sc{inf}}{bivvd-o}{catch-\Sc{nmlz1}\BS\Sc{nom.sg}}{catch (fishing)}{6574}
\z
\ea\label{NMLZ1ex4}
\DerivExamWL{gähtto-t}{tell-\Sc{inf}}{gähtt-o}{tell-\Sc{nmlz1}\BS\Sc{nom.sg}}{story, report}{6686}
\z
\ea\label{NMLZ1ex5}
\DerivExam{lávvlo-t}{sing-\Sc{inf}}{lávvl-o}{sing-\Sc{nmlz1}\BS\Sc{nom.sg}}{song, hymn}{080825}{030}
\z
\ea\label{NMLZ1ex6}
\DerivExamX{dårro-t}{fight-\Sc{inf}}{dårr-o}{fight-\Sc{nmlz1}\BS\Sc{nom.sg}}{fight, battle}{080701b}{024}
\z
\ea\label{NMLZ1ex7}
\DerivExamWL{dårrjo-t}{support-\Sc{inf}}{dårrj-o}{support-\Sc{nmlz1}\BS\Sc{nom.sg}}{support}{4732}
\z
However, as \REF{NMLZ1ex2} and \REF{NMLZ1ex3a} indicate, the deverbalized noun does not have to refer exactly to the action or result of the verb, but to only a related concept. 
\ea\label{NMLZ1ex2}
\DerivExamWL{gåjjkå-t}{dry-\Sc{inf}}{gåjjk-o}{dry-\Sc{nmlz1}\BS\Sc{nom.sg}}{drought; thirst}{4225}
\z
\ea\label{NMLZ1ex3a}
\DerivExamWL{jáhkke-t}{believe-\Sc{inf}}{jáhkk-o}{believe-\Sc{nmlz1}\BS\Sc{nom.sg}}{belief}{0909}
\z

\korr{048}Such nouns are in nominal inflectional class Ic; these examples present the nominal singular forms, and are thus in the strong stage of consonant gradation. 
%In these examples, the deverbal noun is in the strong stage of consonant gradation, and belongs to the %\marginpar{confirm Ic for these nouns!}
%nominal inflectional class Ic. 
As the example in \REF{nominalizerEx1} shows, these are full fledged nouns that inflect for case and number and can fill syntactic slots reserved for NPs (here, the object of the transitive verb \It{adnet} ‘have’).
\ea\label{nominalizerEx1}
\glll	jut almatj adna jáhkov\\
	jut almatj adna jáhk-o-v\\
	if person\BS\Sc{nom.sg} have\BS\Sc{3sg.prs} believe-\Sc{nmlz1}-\Sc{acc.sg}\\\nopagebreak
\Transl{if one has faith}{}	\CorpusSJE{130523}{158}
\z




\subsection{The agent nominalizer suffix \It{-däddje}}\label{agentNMLZ}\index{nominalizer}
The nominalizing suffix \It{-däddje} (glossed as \Sc{nmlz2}) creates an agent noun, indicating that the referent of the noun is involved in the activity denoted by the base. Examples are provided in \REF{agentNMLZex1} through \REF{agentNMLZex5}. 
\ea\label{agentNMLZex1}
\DerivExamWL{vuäjdne-t}{see-\Sc{inf}}{vuojna-däddje}{see-\Sc{nmlz2}\BS\Sc{nom.sg}}{clairvoyant}{6532}
\z
\ea\label{agentNMLZex2}
\DerivExamWL{åhpå-t}{learn-\Sc{inf}}{åhpa-däddje}{learn-\Sc{nmlz2}\BS\Sc{nom.sg}}{teacher}{2243}
\z
\ea\label{agentNMLZex3}
\DerivExamWL{málesti-t}{cook-\Sc{inf}}{máles-däddje}{cook-\Sc{nmlz2}\BS\Sc{nom.sg}}{cook, chef}{5377}
\z
\ea\label{agentNMLZex4}
\DerivExamWL{gieles}{lie\BS\Sc{nom.sg}}{gieles-däddje}{lie-\Sc{nmlz2}\BS\Sc{nom.sg}}{liar}{4826}
\z
\ea\label{agentNMLZex5}
\DerivExamWL{jåhta-t}{drive-\Sc{inf}}{báhko\PLUS jåde-däddje}{word\PLUS drive-\Sc{nmlz2}\BS\Sc{nom.sg}}{chairperson}{0109}
\z
The base is typically a verb, but can be a noun, as in \REF{agentNMLZex4}. The stem of the derived agent noun is in the weak grade. 
As illustrated by \REF{agentNMLZex2} and \REF{agentNMLZex3}, the resulting agent noun (with a root \It{máles} and \It{åhpa}) may no longer be directly derivable from the comparable verb (there is no verb \It{*málle-t} ‘cook-\Sc{inf}’, only \It{málestit}, nor a verb \It{*åhpa-t}, \korr{054}but instead \It{åhpådit} ‘teach-\Sc{inf}’). 

Note that the noun \It{báhkojådedäddje} ‘chairperson’ in \REF{agentNMLZex3} is a compound calque based on the Swedish equivalent \It{ordförande}, which literally means ‘word-driver’. It is not clear whether ?\It{jådedäddje} ‘driver’ exists on its own. 




\subsection{The state nominalizer suffix \It{-vuohta}}\label{vuohta}\index{nominalizer}
The nominalizing suffix \It{-vuohta} (glossed as \Sc{nmlz3}) typically derives nouns from adjectives, 
%indicates a state\marginpar{better term than ‘state’?} of being the property referred to by the base, which is usually an adjective, 
as in \REF{vuohtaEx1} through \REF{vuohtaEx3}. 
\ea\label{vuohtaEx1}
\DerivExamWL{vassjalis}{active}{vassjalis-vuohta}{active-\Sc{nmlz3}\BS\Sc{nom.sg}}{activity}{3082}
\z
\ea\label{vuohtaEx2}
\DerivExamWL{sádnes}{true}{sádnes-vuohta}{true-\Sc{nmlz3}\BS\Sc{nom.sg}}{truth}{1476}
\z
\ea\label{vuohtaEx3}
\DerivExamWL{luossis}{heavy}{luossis-vuohta}{heavy-\Sc{nmlz3}\BS\Sc{nom.sg}}{melancholy}{2519}
\z

The suffix \It{-vuohta} can also be applied to a derived adjectival base. In \REF{vuohtaEx5}, the stem \It{máhtelis} ‘possible’ is itself a derived adjectival based on the verb \It{máhttet} ‘can’. The deepest derivational base in the example in \REF{vuohtaEx6} is roughly analogous, but one step farther removed from the final derived form: the highest-level base \It{bargodis} ‘unemployed’ is an adjectival form of the noun \It{bargo} ‘work’, which itself is a deverbal form based on the verb \It{barrgat} ‘work’ (cf. example \REF{NMLZ1ex1} in Section \ref{nmlz1} above). 
\ea\label{vuohtaEx5}
\DerivExamWL{máhtelis}{possible}{máhtelis-vuohta}{possible-\Sc{nmlz3}\BS\Sc{nom.sg}}{possibility}{6533}
\z
\ea\label{vuohtaEx6}
\DerivExamWL{bargodis}{unemployed}{bargodis-vuohta}{unemployed-\Sc{nmlz3}\BS\Sc{nom.sg}}{unemployment}{3131}
\z

Much as with \REF{vuohtaEx6} above, the base \It{tjalmedis} ‘blind’ in \REF{vuohtaEx7} is itself based on the noun \It{tjalbme} ‘eye’ derived by the suffix \It{-dis} indicating a lack of the base referent. Thus, \It{tjalmedisvuohta} could be literally translated as ‘eye-less-ness’. 
\ea\label{vuohtaEx7}
\DerivExamWL{tjalmedis}{blind}{tjalmedis-vuohta}{blind-\Sc{nmlz3}\BS\Sc{nom.sg}}{blindness}{6201}
\z

However, as \REF{vuohtaEx4} indicates, the base from which \It{-vuohta} derives a new noun can also be a noun. 
\ea\label{vuohtaEx4}
\DerivExamWL{mánná}{child\BS\Sc{nom.sg}}{mánná-vuohta}{child-\Sc{nmlz3}\BS\Sc{nom.sg}}{childhood}{3221}
\z



\section{Verbal derivation}\label{vDerivation}%include non-finite verbs here!
Verbal derivation in \PS\ is a particularly complex area, and the interested reader is first and foremost referred to Israel Ruong’s PhD thesis \It{Lappische Verbalableitung dargestellt auf Grundlage des Pitelappischen}\footnote{‘Saami verbal derivation as illustrated by the \PS\ language’ (my translation).} 
\citep{Ruong1943}. This work presents a comprehensive typology of non-derived verbs and verbal derivation suffixes in \PS. %, made possible surely in part by his native speaker intuitions and knowledge. 
It includes an extensive semantic sub-classification of the derivational suffixes into the varied and overlapping meanings each one can have. The forty suffixes Ruong presents, and the myriad functions he assigns them to, further attest to the complicated nature of verbal derivation in \PS. 

The present discussion cannot hope to improve on Ruong’s work, and instead attempts to use the \PSDP\ corpus to achieve the following: 
%two things in this section on verbal derivation:
\begin{itemize}
\item{Using the diminutive verbalizer \It{-tj} as a starting point, illustrate the complexity of verbal derivation in \PS\ due to the persistent irregularities between forms and functions (Section \ref{verbDIM});}
\item{Present a sample of verbal derivations (Section \ref{vblzST} through \ref{vblzDALL});}
\item{Provide a basic description of the important valency-decreasing verbal derivation creating passive verb forms (Section \ref{VdervPassives}).}
\end{itemize}



\subsection[The diminutive verbalizer suffix \It{-tj}]{The diminutive verbalizer suffix \It{-tj} and the complexities of \PS\ derivational verb morphology}\label{verbDIM}\index{verbalizer}
The diminutive verbalizing suffix \It{-tj} (glossed as \Sc{dim}) expresses doing the activity referred to by the verbal base a little bit or to a limited extent, as in \REF{verbDIMex1} through \REF{verbDIMex3}.\footnote{Note the similarity in form and semantics to the diminutive nominalizing suffix \It{-tj} discussed in Section \ref{dim}.} %It is essentially identical in form  
\ea\label{verbDIMex1}
\DerivExamX{barrga-t}{work-\Sc{inf}}{barga-tji-t}{work-\Sc{dim}-\Sc{inf}}{work a little bit}{110404}{285}
\z
\ea\label{verbDIMex2}
\DerivExamWL{vádtse-t}{walk-\Sc{inf}}{vádtsa-tji-t}{walk-\Sc{dim}-\Sc{inf}}{walk slowly}{2047}
\z
\ea\label{verbDIMex3}
\DerivExamWL{bällke-t}{quarrel-\Sc{inf}}{bielka-tji-t}{quarrel-\Sc{dim}-\Sc{inf}}{have a small quarrel}{4698}
\z

The weak form of the base verb is selected by \It{-tj}, and the final vowel in the base becomes \It{a}. 
The \It{i} following the \It{-tj} suffix is the verb class marker for the resulting Class V verb. %, to which verbs derived by \It{-tj} belong. 

%\subsubsection{other DIM forms}
Note, however, that other derivational suffixes can produce diminutive meanings as well, as illustrated by the examples in \REF{verbDIMex4} through \REF{verbDIMex6}. 
\ea\label{verbDIMex4}
\DerivExamWL{gähtja-t}{see-\Sc{inf}}{gietja-sti-t}{see-\Sc{vblz1}-\Sc{inf}}{glance}{2530}
\z
\ea\label{verbDIMex5}
\DerivExamWL{gåsså-t}{cough-\Sc{inf}}{gåsså-di-t}{cough-\Sc{vblz2}-\Sc{inf}}{cough a little bit}{4898}
\z
%\ea\label{verbDIMex6}%from Ruong p. 154
%\DerivExam{suovvadi-t}{smoke-\Sc{inf}}{suovva-dalla-t}{smoke-\Sc{vblz}-\Sc{inf}}{smoke a little}{}{?}
%\z
\ea\label{verbDIMex6}
\DerivExamWL{rassjo-t}{rain-\Sc{inf}}{rässjo-dalla-t}{rain-\Sc{vblz3}-\Sc{inf}}{rain lightly}{5073}
\z
%dä lä suohta dåjna sáhgadallat \Corpus{080924.330}
In these three examples, the derivational suffixes \It{-st}, \It{-d} and \It{-dall}, respectively,\footnote{The vowel following each of these verbalizers is a class marker.} 
also derive deverbal verbs which add similar diminutive meanings to the base. 

If these suffixes were restricted to a diminutive meaning, then this would simply be a case of many forms corresponding to a single function. However, these suffixes, which are all quite common, only occasionally carry a diminutive meaning. In other instances, they impart a variety of different meanings to the base form. This is illustrated by just a few examples below, and is even more obvious in \citet{Ruong1943}. Despite the variety of and inconsistencies in the meanings that verbal derivational suffixes express, their limited number relative to the number of functions they fulfill is reason enough to describe each of these suffixes as a single derivational affix with multiple functions, rather than multiple, homonymous affixes, each aligned to a separate function. 



%\begin{itemize}\item{The derivational suffix \BfIt{-st}:}\end{itemize}
\subsection{The verbal derivational suffix \It{-st}}\label{vblzST}
In addition to the diminutive meaning in \REF{verbDIMex4} above, 
the derivational suffix \It{-st} (glossed as \Sc{vblz1}) is applied to a postposition in \REF{vblzSTex1}, and functions as a verbalizer. In \REF{vblzSTex2}, the nominal base is not only verbalized, but has a causative or perhaps an inchoative meaning. The derived verb in \REF{vblzSTex3} is a figurative extension of the verbal base’s meaning. 
Furthermore, \It{-st} can indicate that an action is carried out briefly or for a short period of time, as in \REF{vblzSTex4}. 
\ea\label{vblzSTex1}
\DerivExamWL{birra}{around}{bira-sti-t}{around-\Sc{vblz1}-\Sc{inf}}{cruise around}{0185}
\z
\ea\label{vblzSTex2}
\DerivExamWL{dållå}{fire\BS\Sc{nom.sg}}{dålå-sti-t}{fire-\Sc{vblz1}-\Sc{inf}}{start a fire}{0422}
\z
\ea\label{vblzSTex3}
\DerivExamWL{båhtje-t}{milk-\Sc{inf}}{båtje-sti-t}{milk-\Sc{vblz1}-\Sc{inf}}{wring out}{0262}
\z
\ea\label{vblzSTex4}
\DerivExamWL{basse-t}{fry-\Sc{inf}}{base-sti-t}{fry-\Sc{vblz1}-\Sc{inf}}{fry quickly}{5501}
\z
Note that the \It{i} following the \It{-st} suffix is the verb class marker for the resulting Class V verb. 



\subsection{The verbal derivational suffix \It{-d}}\label{vblzD}
In addition to the diminutive meaning in \REF{verbDIMex5} above, 
the two examples of the verbalizer \It{-d} (glossed as \Sc{vblz2}) in \REF{vblzDex1} and \REF{vblzDex2} each has a reflexive meaning; note that the base in \REF{vblzDex2} is a noun, not a verb. The example in \REF{vblzDex4} has a transitivizing effect on the verbal base, while there is no clear difference in meaning between the base and the resulting derived form in \REF{vblzDex5} and \REF{vblzDex5b}. The last example, \It{sykel}\footnote{< Swedish \It{cykel} ‘bicycle’.} 
‘bicycle’ in \REF{vblzDex6}, illustrates that this suffix is quite productive, as it is used as a verbalizer for a loanword serving as a nominal base. 
Note that the \It{i} following the \It{-d} suffix is the verb class marker for the resulting Class V verb. 
\ea\label{vblzDex1}
\DerivExamX{bassa-t}{wash-\Sc{inf}}{basá-di-t}{wash-\Sc{vblz2}-\Sc{inf}}{wash oneself}{090910}{81m08s}
\z
\ea\label{vblzDex2}
\DerivExamWL{gárrvo}{clothing\BS\Sc{nom.sg}}{gärvo-di-t}{clothing-\Sc{vblz2}-\Sc{inf}}{dress oneself}{0793}
\z
%\ea\label{vblzDex3}
%\DerivExamWL{tjájbma-t}{laugh-\Sc{inf}}{tjájma-di-t}{laugh-\Sc{vblz2}-\Sc{inf}}{laugh, smile}{1865}%UM: not a reflexive meaning!
%\z
\ea\label{vblzDex4}
\DerivExamWL{busso-t}{blow-\Sc{inf}}{buso-di-t}{blow-\Sc{vblz2}-\Sc{inf}}{blow out}{4704}
\z
\ea\label{vblzDex5}
\DerivExamWL{bulle-t}{ignite-\Sc{inf}}{bulle-di-t}{ignite-\Sc{vblz2}-\Sc{inf}}{ignite}{2664}
\z
\ea\label{vblzDex5b}
\DerivExamWL{tjájbma-t}{laugh-\Sc{inf}}{tjájma-di-t}{laugh-\Sc{vblz2}-\Sc{inf}}{laugh, smile}{1865}%UM: not a reflexive meaning!
\z
\ea\label{vblzDex6}
\DerivExamWL{sykel}{bicycle\BS\Sc{nom.sg}}{sykel-di-t}{bicycle-\Sc{vblz2}-\Sc{inf}}{ride a bicycle}{1810}
\z



\subsection{The verbal derivational suffix \It{-dall}}\label{vblzDALL}
In addition to the diminutive meaning in \REF{verbDIMex6} above, 
the first two examples \mbox{(\ref{vblzDALLex1}-\ref{vblzDALLex2})} of the verbalizer \It{-dall} (glossed as \Sc{vblz3}) show deadjectival verbs which express being characterized by the base adjective. The adjectival base is even semantically restricted in the derived form in Example \REF{vblzDALLex2}. 
Similarly, the denominal verb in \REF{vblzDALLex3} is based on the \Sc{3}\SGs.\NOMs\ reflexive pronoun \It{etjas} ‘oneself’, and expresses a more forceful state of being the meaning of the base noun. The final two examples also restrict the semantic scope of the verbal base. 
The \It{a} following the \It{-dall} suffix is the verb class marker for the resulting Class IIa verb. 
\ea\label{vblzDALLex1}
\DerivExamWL{lajjkes}{lazy}{lajkas-dalla-t}{lazy-\Sc{vblz3}-\Sc{inf}}{be lazy}{2959}
\z
\ea\label{vblzDALLex2}
\DerivExamWL{bahás}{evil}{bahás-dalla-t}{evil-\Sc{vblz3}-\Sc{inf}}{be against something}{4705}
\z
\ea\label{vblzDALLex3}
\DerivExamWL{etjas}{oneself\BS\Sc{nom.sg}}{etjas-dalla-t}{oneself-\Sc{vblz3}-\Sc{inf}}{be stubborn}{0460}
\z
\ea\label{vblzDALLex4}
\DerivExamWL{gähtja-t}{see-\Sc{inf}}{giehtja-dalla-t}{see-\Sc{vblz3}-\Sc{inf}}{check out, look into}{3875}
\z
\ea\label{vblzDALLex5}
\DerivExamWL{tjehka-t}{hide-\Sc{inf}}{tjehka-dalla-t}{hide-\Sc{vblz3}-\Sc{inf}}{play hide and seek}{1889}
\z


%-dalla- cf. Ruong p. 152-169!




\subsection{Passivization with the derivational suffix \It{-duvv}}\label{VdervPassives}\index{passivization}\index{verbs!passives}
Transitive verbs can be passivized using the suffix \It{-duvv} (glossed as \Sc{pass}). %\marginpar{or only \Bf{-uvvu}, with \Bf{-t-} as \Sc{inf}?}, %JW: decided to ignore this possibility
The resulting derived verb belongs to the inflectional verb Class IV, and thus features the class marker \It{-a} following the passivizing suffix in the infinitive form, and \It{-u} in the perfect form. 
For instance, compare the verb in \REF{passEx2} (in the active voice) with the equivalent passivized verb in \REF{passEx3}, including the oblique agent (in \ELAT\ case). %in the passive construction in \REF{passEx4}.
\ea\label{passEx2}%
\glll	máná lä tsiggim gådev\\
	máná lä tsiggi-m gåde-v\\
	child\BS\Sc{nom.pl} be\BS\Sc{3pl.prs} build-\Sc{prf} hut-\Sc{acc.sg}\\\nopagebreak
\trans	‘children have built the hut’	\CorpusE{110518a}{28m14s}
\z
\ea\label{passEx3}%
\glll	gåhte lä tsiggijduvvum mánájst\\
	gåhte lä tsiggij-duvvu-m máná-jst\\
	hut\BS\Sc{nom.sg} be\BS\Sc{3sg.prs} build-\Sc{pass}-\Sc{prf} child-\Sc{elat.pl}\\\nopagebreak
\trans	‘the hut has been built by children’	\CorpusE{110518a}{28m41s}
\z
%\ea\label{passEx4}%
%\glll	dat huvvsa bidtjiduvvuj Nisest\\
%	dat huvvsa bidtji-duvvu-j Nise-st\\
%	\Sc{dem.dist}\BS\Sc{nom.sg} house\BS\Sc{nom.sg} build-\Sc{pass}-\Sc{3sg.pst} Nils-\Sc{elat.sg}\\
%\trans	‘that house was built by Nils’	\CorpusE{110522.33m03s}
%\z

Passivization is a valency-decreasing device because the resulting verb is intransitive, as it only features the patient-like argument as its sole core argument in nominative case. Note that \citet[92]{Svonni2009} claims, for North Saami, that “one cannot indicate the agent in any way” (my translation) in passive clauses using the cognate North Saami passivizing suffix. \PS\ differs significantly from North Saami in this respect, as even Ruong – himself a native speaker of \PS\ – verifies \citep[cf.][41]{Ruong1943}. It is very possible that the \PS\ strategy of placing the agent in an oblique case could be due to extensive language contact with Swedish, a language which clearly allows the agent in a passivized clause to be expressed obliquely using a prepositional phrase headed by the preposition \It{av} ‘of, from’. 
Indeed, Swedish PPs headed by \It{av} in other contexts are best translated into \PS\ as an NP in elative case, the same oblique case in which the agent NP in a passive \PS\ sentence is found. 
%Moreover, \PS\ NPs in \ELAT\ case are often translated into Swedish as a PP with \It{av} ‘from’ as the phrasal head; this is the sam preposition used to head a PP indicating the agent in passive Swedish sentences. 
%It should be noted that the passive examples in the corpus with agents in \ELAT\ case are all from translational elicitation sessions conducted mainly in Swedish, and Swedish triggers used to elicit these examples (such as \It{kåtan ställdes upp av barnen} ‘the tent was set up by the children’) may also have affected speakers’ judgment and production of \PS\ grammatical structures. 

Some other examples of transitive verbs and their passivized equivalents using \It{-duvv} are shown in \REF{vblzDUVVex1} through \REF{vblzDUVVex3}.
\ea\label{vblzDUVVex1}
\DerivExamWL{tjåvvde-t}{untie-\Sc{inf}}{tjåvde-duvva-t}{untie-\Sc{pass}-\Sc{inf}}{be liberated}{3233}
\z
\ea\label{vblzDUVVex2}
\DerivExamX{dahka-t}{make-\Sc{inf}}{daga-duvva-t}{make-\Sc{pass}-\Sc{inf}}{be made}{110331b}{069}%pit110331b only has -PRF forms! (starts at .069)
\z
\ea\label{vblzDUVVex3}
\DerivExamWL{adne-t}{utilize-\Sc{inf}}{ane-duvva-t}{utilize-\Sc{pass}-\Sc{inf}}{be used}{2682}
\z
%\ea\label{vblzDUVVex4}
%\DerivExam{bállkot}{throw-\Sc{inf}}{bálkes-duvva-t}{throw-\Sc{pass}-\Sc{inf}}{be thrown}{}{?}
%\z

There is not sufficient data in the corpus to state any more about passive derivation, particularly concerning morphophonological effects of passivization on verb stems, and this and other related topics must be left for future study. The reader is referred to \citet{Ruong1943} for a more thorough morphological and semantic account of \PS\ passives. 
Inflectional aspects of passivized verbs are treated in \ref{passiveVinflection}, while syntactic aspects of clauses with passive verbs are presented briefly in \ref{passiveVoice}. 

Note that the derivational suffix \It{-duvv} can have meanings other than passive when attached to a nominal or adjectival base. Typically it then expresses a change of state that is related to the referent of the root involved. A few examples are provided in \REF{vblzDUVVex5} through \REF{vblzDUVVex8}. 
\ea\label{vblzDUVVex5}
\DerivExamWL{vuoras}{old}{vuoras-duvva-t}{old-\Sc{pass}-\Sc{inf}}{age (verb)}{2188}
\z
\ea\label{vblzDUVVex6}
\DerivExamWL{bevas}{sweat\BS\Sc{nom.sg}}{bevas-duvva-t}{sweat-\Sc{pass}-\Sc{inf}}{become sweaty}{6084}
\z
\ea\label{vblzDUVVex7}
\DerivExamWL{giella}{language\BS\Sc{nom.sg}}{giela-duvva-t}{language-\Sc{pass}-\Sc{inf}}{become hoarse}{3876}
\z
\ea\label{vblzDUVVex8}
\DerivExamWL{tjálbme}{eye\BS\Sc{nom.sg}}{tjálme-duvva-t}{eye-\Sc{pass}-\Sc{inf}}{become blind}{1876}
\z



\section{Adjectival derivation}\label{adjDerivation}
Only two derivational processes exist for adjectivals: the non-productive derivation of adjectives by \It{-s}, and the productive derivation of ordinal numerals from cardinal ones. 
%, and the first, concerning the formation of adjectives is likely no longer productive. The latter concerns the derivation of ordinal numerals from cardinal numerals. 
These are described below. 

\subsection{Adjective derivation}\label{ATTRadjDerivation}
It seems conceivable that adjectives can be derived by the suffix \mbox{\It{-s}} (glossed as \Sc{adjz}). 
For instance, \It{bahá} is a nominal meaning ‘evil’, as in \REF{derivADJex1}, and \It{bahás} is the equivalent attributive adjective form, as in \REF{derivADJex2}. 
\ea\label{derivADJex1}
\glll	dat almatj lä bahá\\
	d-a-t almatj lä bahá\\
	\Sc{dem}-\Sc{dist}-\Sc{nom.sg} person\BS\Sc{nom.sg} be\BS\Sc{3sg.prs} evil\BS\Sc{nom.sg} \\\nopagebreak
\trans	‘that person is evil’	\CorpusE{090926}{13m36s}
\z
\ea\label{derivADJex2}
\glll	bahás almatj\\
	bahá-s almatj\\
	evil-\Sc{adjz} person\BS\Sc{nom.sg}\\\nopagebreak
\trans	‘evil person’	\CorpusE{090926}{13m40s}
\z
In addition, the nominalized form \It{bahá-k} ‘evil’ can be further derived into an adjective \It{bahágis} ‘painful’ as in \REF{derivADJex3}. %, although it is not clear if this is an attributive or a predicative adjective. 
\ea\label{derivADJex3}
\DerivExamWL{bahá-k}{evil-\Sc{nmlz}}{bahá-gi-s}{evil-\Sc{nmlz}-\Sc{adjz}}{painful}{0102}
\z

However, as pointed out in detail in Sections \ref{adjectivesATTR} through \ref{notePredNounsAdjs}, not all adjectives follow this pattern. In fact, based on the current data, the \It{-s} suffix marks attributive adjectives (as in the example in \REF{derivADJex2}) and as well as predicative adjectives, %or even if it is %\marginpar{find out if \Bf{-s} is productive!} 
and, synchronically, it is not considered to be productive for either attributive or predicative forms at all. 



\subsection{Ordinal numeral derivation with \It{-át}}\label{ordNUMderiv}
Numerals, a sub-category of adjectivals (cf. Section \ref{numerals}), are subject to derivation. 
%\subsection{Numeral Derivation}\label{numDerivation}
Specifically, the basic ordinal numerals can be derived by applying the derivational suffix \It{-át} (or its allomorph \It{-et}) %\TILDE\It{-et} %just gonna consider -et an allomorph
to the respective cardinal numeral, although the forms \It{vuostas} ‘first’ and \It{mubbe} ‘second’ are suppletive. Ordinal derivation is discussed in Section \ref{ordinalNums} in more detail, including a comparison of cardinal and ordinal numbers in Table \vref{basicNumsTable}. 





\section{Adverbial derivation}\label{advDerivation}
Adverbs are not common in the corpus (as opposed to other word classes and phrase types with adverbial functions), but do appear to be derivable from an adjective base using the suffix \It{-git}. This is dealt with in more detail in Section \ref{derivedADVs}. 






\section{Summary of derivational morphology}\label{derivMorphSummary}
Table \vref{derivMorphSummaryTable} provides an overview of the derivational morphology discussed in this chapter. 
%\marginpar{not really sure Table \vref{derivMorphSummaryTable} is necessary or useful}%UM: it is useful, but ‘result’ column needs revision because you sometime mix object and meta-language
\begin{table}\centering
\caption{Summary of derivational morphology discussed in this chapter}\label{derivMorphSummaryTable}
\begin{tabular}{|c |c | l l c|}\hline
\It{type}&\It{suffix}	&\It{base}	&\It{result}			&\It{section}\\\dline
\MR{9}{*}{\begin{sideways}nominal\end{sideways}}
	&-tj	&noun	&diminutive			&\ref{dim}	\\\cline{2-5}
	&-k	&noun	&characterized by base referent	&\ref{nmlz2}	\\%\cline{2-5}
	&		&adjective	&					&	\\
	&		&verb	&					&	\\\cline{2-5}
	&-o	&verb	&the action itself		&\ref{nmlz1}	\\\cline{2-5}
	&-däddje&verb	&person involved in state of affairs	&\ref{agentNMLZ}	\\%\cline{2-5}
	&		&noun	&					&	\\\cline{2-5}
	&-vuohta&adjective&characterized by base referent	&\ref{vuohta}	\\%\cline{2-5}
	&		&noun	&					&	\\\hline
\MR{8}{*}{\begin{sideways}verbal\end{sideways}}
	&-tj	&verb	&diminutive			&\ref{verbDIM}	\\\cline{2-5}
	&-st	&verb	&causative; inchoative	&\ref{vblzST}	\\%\hline
	&		&noun	&					&	\\\cline{2-5}
	&-d	&verb	&reflexive; other		&\ref{vblzD}	\\%\cline{2-5}
	&		&noun	&					&	\\\cline{2-5}
	&-dall	&verb	&diminutive; characterized by		&\ref{vblzDALL}	\\%\cline{2-5}
	&		&noun	&\hspace{6pt}base referent; etc.		&	\\\cline{2-5}
	&-duvv	&verb	&passive				&\ref{VdervPassives}	\\\cline{3-5}
	&		&adjective	&change of state		&\ref{VdervPassives}	\\%\cline{2-5}
	&		&noun	&					&	\\\hline%\cline{2-5}
%\MR{1}{*}{\begin{sideways}adj.\end{sideways}}
\MR{2}{*}{adj.}	
	&-s	&noun	&attributive adjective		&\ref{adjDerivation}	\\\cline{2-5}
	&-át	&cardinal num.	&ordinal num.		&\ref{ordNUMderiv}	\\\hline%\cline{2-5}
%\MR{1}{*}{\begin{sideways}adv.\end{sideways}}
adv.	&-git	&adjective	&adverb				&\ref{advDerivation}	\\\hline%\cline{2-5}
\end{tabular}
\end{table}





%%%%%%% THIS IS NOT USED FOR THE ENTIRE COMPILATION, but only for individual chapters!!!!

\clearpage
\addcontentsline{toc}{chapter}{Bibliography}\label{Bibliography}
\bibliography{PiteGrammarBibSDL}%for bibtex
%\printbibliography%[title=Works Cited]%%for biber!






%%%NAME INDEX doesn’t work!?!? why???
\cleardoublepage\phantomsection%this allows hyperlink in ToC to work
\addcontentsline{toc}{chapter}{Name index}
\ohead{Name index}
\printindex[aut]

\cleardoublepage\phantomsection%this allows hyperlink in ToC to work
\addcontentsline{toc}{chapter}{Language index}
\ohead{Language index}
\printindex[lan]

\cleardoublepage\phantomsection%this allows hyperlink in ToC to work
\addcontentsline{toc}{chapter}{Subject index}
\ohead{Subject index}
\printindex


\end{document}