%2 

\section{Les phrases liées}
\subsection{Typologie des phrases liées} 
Les phrases telles que je les ai définies dans le chapitre 1 peuvent être indépendantes ou liées par des relations de coordination ou de subordination (cf. la classification des phrases donnée en \REF{ex:2:1}). Ainsi, une séquence comme \textit{la Bucureşti ninge} `à Bucarest il neige' peut être employée comme une phrase indépendante, qui n'a pas de relation syntaxique avec une autre phrase \REF{ex:2:2}a. La même séquence peut apparaître liée à une autre phrase par une relation de coordination, marquée dans l'exemple \REF{ex:2:2}b par la conjonction \textit{iar} `et'. Elle peut fonctionner aussi comme phrase racine par rapport à une subordonnée ajout \REF{ex:2:2}c. En revanche, en \REF{ex:2:2}d-e la séquence mentionnée précédemment perd son statut de phrase racine, car elle est cette fois-ci subordonnée : en \REF{ex:2:2}d, elle fonctionne comme une complétive du verbe \textit{a spune} `dire' de la phrase racine (et, dans ce cas, elle est liée à celle-ci par le complémenteur \textit{că} `que'), alors que, en \REF{ex:2:2}e, elle est ajout à la phrase racine\footnote{La plupart des grammaires traditionnelles utilisent la notion de \textit{phrase principale} pour désigner la phrase tête ou la phrase qui contient le mot tête. Si elle peut être séparée d'une subordonnée ayant la fonction ajout (i)a, en revanche elle n'a pas d'autonomie syntaxique sans son complément phrastique (i)b. Comme Abeillé, Delaveau \& \citet{Godard2007} suggèrent, le terme de \textit{phrase racine} rend mieux compte de cette distribution.
(i)  a  La Braşov trebuie să fie foarte frig, (dacă la Bucureşti ninge).
    \textit{A Braşov il doit faire très froid, (si à Bucarest il neige)}
  b  Ion mi-a spus *(că la Bucureşti ninge).
    \textit{Ion m'a dit qu'à Bucarest il neige}} et liée par le complémenteur \textit{dacă} `si'. Finalement, on observe qu'en \REF{ex:2:2}f la séquence \textit{la Bucureşti ninge} est simultanément subordonnée (par le complémenteur \textit{că} `que') à~une phrase racine, et coordonnée (par la conjonction \textit{şi} `et') à une autre subordonnée.  


\begin{enumerate}
\item   \label{bkm:Ref300148662}Classification des phrases (cf. Abeillé et al. \textit{en prép.})


\end{enumerate}
{   [Warning: Image ignored] % Unhandled or unsupported graphics:
%\includegraphics[width=4.7543in,height=0.6681in,width=\textwidth]{fe443409cd384d3fb0f6390ffd77f513-img20.svm}
} 


\begin{enumerate}
\item \label{bkm:Ref268704975}a  \textbf{La Bucureşti ninge}.


\end{enumerate}
{\itshape
A Bucarest il neige}

  b  \textbf{La Bucureşti ninge}, iar la Braşov plouă.

{\itshape
A Bucarest il neige, et à Braşov il pleut} 

  c  \textbf{La Bucureşti ninge}, deşi la Braşov e cald 

{\itshape
A Bucarest il neige, même si à Braşov il fait chaud}

  d  Ion mi-a spus că \textbf{la Bucureşti ninge}.

{\itshape
Ion m'a dit qu'à Bucarest il neige} 

  e  Dacă \textbf{la Bucureşti ninge}, la Braşov trebuie să fie foarte frig.

    \textit{Si à Bucarest il neige, à Braşov il doit faire très froid}

  f  Ion mi-a spus că \textbf{la Bucureşti ninge} şi că la Braşov e cald.

    \textit{Ion m'a dit qu'à Bucarest il neige et qu'à Braşov il fait chaud}

La manière dont on lie les phrases a intéressé les chercheurs dès les années 1980 (cf. \textit{clause combining} dans Haiman \& \citet{Thompson1988} ou encore \textit{clause linkage} dans \citet{Lehmann1988}). Dans la tradition grammaticale, il y a bien consensus sur l'importance de la coordination et de la subordination pour la description grammaticale. En revanche, comme on verra par la suite, les critères utilisés pour distinguer ces deux termes varient d'un auteur à l'autre.\footnote{Certains auteurs utilisent les notions de \textit{coordination} et \textit{subordination} en variation libre avec les termes de \textit{parataxe} et respectivement \textit{hypotaxe}. Pour éviter toute confusion terminologique, je garde les premiers termes.}   

Minimalement, on peut décrire les deux relations en suivant la distinction que \citet{Longacre2007} fait entre les structures de type \textit{co-ranking} (c.-à-d. deux éléments qui se trouvent au même niveau) et les structures de type \textit{chaining} (c.-à-d. deux éléments qui se trouvent dans une relation de dominance), en remarquant qu'il y a des cas où les deux sont possibles\footnote{Le choix d'une structure avec coordination ou subordination sera justifiée ici par des facteurs sémantico-discursifs.} : on a le choix entre une coordination de deux phrases racines \REF{ex:2:3}a et une subordonnée dépendant d'une phrase racine \REF{ex:2:3}b.


\begin{enumerate}
\item \label{bkm:Ref271467474}a  John chopped the tree into firewood \textbf{and} carried it into his house.


\end{enumerate}
  b  \textbf{When} John had chopped the tree into firewood, he carried it into his house.

{\raggedleft
            (\citet[376]{Longacre2007})
}

Dans ce chapitre, je m'intéresse à l'étude détaillée du fonctionnement de la coordination et de la subordination des phrases. Tout d'abord, je présente les deux approches majeures de la coordination et de la subordination (section \ref{sec:2.2}), pour ensuite délimiter le domaine empirique de ces deux relations syntaxiques par l'utilisation des tests distinctifs (section \ref{sec:2.3}) qui nous permettent d'une part de séparer les différents types d'introducteurs (conjonctions, adverbes connecteurs ou complémenteurs) et d'autre part de distinguer les deux types de structures (coordonnées vs. subordonnées). Je discute ensuite la notion de parallélisme dans la coordination, en regardant les asymétries qu'on peut observer aux niveaux morpho-syntaxique et sémantico-discursif (section \ref{sec:2.4}). L'hypothèse de départ est qu'aucune corrélation ne peut être faite entre la coordination et l'identité morpho-syntaxique ou encore les relations discursives symétriques. La section suivante (section \ref{sec:2.5}) étudie la coordination des subordonnées, avec leurs contraintes spécifiques. Dans les trois sections qui suivent, j'étudie les phrases liées (coordonnées et subordonnées) en rapport avec trois phénomènes : la juxtaposition (section \ref{sec:2.6}), les éléments corrélatifs (section \ref{sec:2.7}) et l'incidence (section \ref{sec:2.8}). Je montre que ces trois phénomènes ne sont pas réservés à un type de structure (coordonnée ou subordonnée), mais ils caractérisent les deux structures (bien qu'avec des propriétés différentes). Dans la section \ref{sec:2.9}, j'étudie le fonctionnement de la conjonction \textit{iar} `et', qui est le prototype de la coordination contrastive et symétrique en roumain, en donnant les contraintes imposées par \textit{iar} au niveau sémantique, discursif et syntaxique. Enfin, dans la section \ref{sec:2.10}, je donne l'analyse syntaxique des structures coordonnées, en étudiant les relations fonctionnelles qui s'établissent d'une part entre la conjonction et le terme qui le suit, et d'autre part entre les termes coordonnés. Pour l'analyse formelle de la coordination dans cette section, j'utilise le cadre HPSG. 

\subsection{Approches de la coordination et de la subordination}
\label{bkm:Ref302033474}Traditionnellement, on décrit la coordination et la subordination en opposition mutuelle, en utilisant les notions de \textit{dépendance} et \textit{symétrie} (\citet{Haspelmath2007}). On définit ainsi la coordination comme une relation symétrique / non-hiérarchique entre au moins deux termes {\guillemotleft}~équivalents~{\guillemotright} qui sont indépendants syntaxiquement, alors que la subordination est considérée comme une relation asymétrique / hiérarchique entre deux éléments dont un est dépendant syntaxiquement de l'autre (\citet{Lang1984}, \citet{Lehmann1988}, \citet{Dik1997}, Huddleston \& \citet{Pullum2002}, \textit{GALR} (2005), Haspelmath (2004, 2007), Fabricius-Hansen \& \citet{Ramm2008}). Généralement, on considère que, si A est coordonné avec B, alors B est aussi coordonné avec A. En revanche, si A subordonne B, B ne peut pas subordonner A (\citet{Dik1997}).

Cependant, il est souvent difficile de distinguer coordination et subordination, car il y a beaucoup d'occurrences~{\guillemotleft}~intermédiaires~{\guillemotright}, enregistrées d'habitude comme des cas de frontière entre la coordination et la subordination (comme l'indique le terme \textit{subordinate} \textit{conjoining}, utilisé par \citet{Lang1984}, ou \textit{pseudo}-\textit{coordination}, discuté dans \citet{Johannessen1998}). Le statut exact de ces cas hybrides est interprété en fonction de la manière dont on envisage la coordination et la subordination. Deux approches majeures peuvent être distinguées.

\subsubsection{Une approche en termes de continuum}
\label{bkm:Ref300158658}\textbf{Approche A.} \textit{Clause linking as a continuum / a gradient phenomenon.} Pour certains, la coordination et la subordination sont les limites d'un continuum, des prototypes sur une échelle complexe. 

Un exemple dans ce sens est \citet{Cristofaro2003} : {\guillemotleft}~[...] clause linkage types should not be described in terms of the binary opposition between coordination and subordination. Rather, they should be defined in terms of a set of mutually independent and freely combinable features, which form a more or less articulated continuum. Each clause linkage type may be more or less coordinate-like or subordinate-like depending on the parameter taken into account.~{\guillemotright} (Cristofaro (2003 : 22-23))

Dans cette perspective, entre la coordination et la subordination canoniques, il y a un gradient de constructions qui sont plus ou moins coordonnées ou subordonnées (cf. Foley \& Van \citet{Valin1984}, Quirk \textit{et al.} (1985), Matthiessen \& \citet{Thompson1988}, \citet{Lehmann1988}, \citet{Cristofaro2003}, \citet{Cosme2008}). Selon \citet{Aarts2007}, la coordination et la subordination sont dans une relation de gradiance constructionnelle intersective (\textit{Intersective Constructional Gradience}), ce qui est justifié, selon lui, par les nombreuses séquences qui présentent à la fois des propriétés appartenant à la coordination et des propriétés appartenant à la subordination. 

La manière dont on organise ce continuum varie d'un auteur à l'autre. Une approche assez connue est celle de Foley \& Van \citet{Valin1984}. Leur continuum va de la coordination à la subordination en passant par ce qu'ils appellent la~{\guillemotleft}~co-subordination~{\guillemotright}. D'une part, la coordination est caractérisée par l'absence de dépendance et d'enchâssement des verbes têtes \REF{ex:2:4}a. D'autre part, la subordination est définie par la présence de ces deux traits \REF{ex:2:4}b. En revanche, la co-subordination partage avec la coordination l'absence d'enchâssement et avec la subordination le fait qu'elle montre un comportement dépendant \REF{ex:2:4}c. Un exemple pour chacun des trois types est donné en \REF{ex:2:5}. Selon eux, l'exemple \REF{ex:2:5}c serait un cas de co-subordination, car formellement il n'y a aucune marque d'enchâssement, mais une des deux phrases est dépendante syntaxiquement de l'autre (la séquence \textit{plouând afară} n'a pas d'autonomie syntaxique). De manière plus générale, les phrases avec des verbes non-finis, constitueraient des occurrences de co-subordination.  


\begin{enumerate}
\item \label{bkm:Ref300150231}Le continuum de Foley \& Van \citet{Valin1984}, Van \citet{Valin2005}~


\end{enumerate}
  a  coordination :   [-- enchâssement, -- dépendance]  

  b  subordination :   [+ enchâssement, + dépendance]

  c  co-subordination :   [-- enchâssement, + dépendance]


\begin{enumerate}
\item \label{bkm:Ref300150652}a  Ion doarme, \textbf{iar} Maria citeşte.  


\end{enumerate}
{\itshape
Ion dort, et Maria lit}

  b  Ion mi-a spus \textbf{că} Maria citeşte.

{\itshape
Ion m'a dit que Maria lisait}

  c  Plouând  afară,  Maria  citeşte.

    pleuvoir.\textsc{gerond}  dehors,  Maria  lit

    \textit{Comme il pleut dehors, Maria lit}

Une échelle plus complexe est illustrée dans Quirk \textit{et al.} (1985). Ils établissent six critères syntaxiques pour délimiter le prototype de la coordination : (i) position initiale de la conjonction, (ii) non-antéposition du syntagme introduit par une conjonction, (iii) non-combinabilité des conjonctions entre elles, (iv) coordination de catégories et syntagmes divers, (v) coordination de subordonnées, (vi) coordination multiple. Plus une structure remplit de critères, plus elle se rapproche de la coordination et s'éloigne de la subordination. Dans cette perspective, \citet{Cosme2008} arrive à un continuum à quatre niveaux : la coordination de phrases, la coordination de syntagmes verbaux, l'hypotaxe et l'enchâssement (cf. l'échelle donnée en \REF{ex:2:6}). A une extrémité de l'échelle, on trouve les constructions qui se définissent par les traits [- dépendant] et [- enchâssé], p.ex. la coordination de phrases en \REF{ex:2:7}a. L'autre extrémité de l'échelle contient les constructions qui se caractérisent par les traits [+ dépendant] et [+ enchâssé], p.ex. \REF{ex:2:7}e et \REF{ex:2:7}f. En revanche, il considère (en suivant \citet{Cristofaro2003}) que la coordination de syntagmes verbaux \REF{ex:2:7}b est moins prototypique que la coordination de phrases, en termes de dépendance. De même, les phrases en hypotaxe \REF{ex:2:7}c et \REF{ex:2:7}d sont moins subordonnées que les phrases enchâssées en \REF{ex:2:7}e et \REF{ex:2:7}f.  


\begin{enumerate}
\item \label{bkm:Ref300155561}Le continuum de \citet{Cosme2008}


\end{enumerate}
{   [Warning: Image ignored] % Unhandled or unsupported graphics:
%\includegraphics[width=5.4236in,height=1.0354in,width=\textwidth]{fe443409cd384d3fb0f6390ffd77f513-img21.svm}
} 


\begin{enumerate}
\item \label{bkm:Ref300155698}a  The winter had come at last, and snow lay thick on the ground.  


\end{enumerate}
  b  Peter ate the fruit and drank the beer.

  c  Before I could sit down, she offered me a cup of tea.

  d  Driving home after work, I accidentally went through a red light.

  e  I noticed that he spoke English with an Australian accent.

  f  He likes everyone to relax.            (\citet[111]{Cosme2008})

Ce qui ressort de tous ces travaux est la distinction entre deux types de subordonnées, en fonction du degré d'intégration d'une subordonnée dans une phrase racine (\citet{Aarts2006}, Fabricius-Hansen \& \citet{Ramm2008}). D'une part, on a les subordonnées en hypotaxe (ou co-subordonnées), et d'autre part, les phrases enchâssées. Dans ces approches, l'hypotaxe caractérise les subordonnées ajouts ou périphériques, qui ne sont pas contenues dans la phrase racine. La phrase racine peut fonctionner de manière indépendante (cf. l'optionalité des subordonnées en \REF{ex:2:8}).


\begin{enumerate}
\item \label{bkm:Ref300156807}a  Nu vin azi, (pentru că sunt bolnavă).


\end{enumerate}
{\itshape
Je ne viens pas aujourd'hui, car je suis malade}

  b  (Deşi mănâncă mult), nu e grasă deloc.

    \textit{Bien qu'elle mange beaucoup, elle n'est pas du tout grosse}

En revanche, on attribue l'étiquette de phrase enchâssée à toute subordonnée qui est un constituant obligatoire de la phrase racine, c.-à-d. en fonction de valent. Dans ces cas, la phrase racine contient la subordonnée, elle ne peut pas être isolée de son valent pas plus qu'une tête de son valent non-phrastique.\footnote{De ce point de vue, la notion de \textit{phrase principale}, utilisée dans les grammaires traditionnelles, n'a pas de sens. Voir note \pageref{fnt:ftn0}.}


\begin{enumerate}
\item a  Maria ştie [că voi veni mâine].


\end{enumerate}
{\itshape
Maria sait que je viendrai demain}

b   Maria ştie [adevărul].

  \textit{Maria sait la vérité}

\subsubsection{Une approche en termes de système binaire}
\textbf{Approche B.} \textit{Clause linking as a binary system.} Il y a une deuxième possibilité d'analyse selon laquelle la coordination et la subordination constituent un système binaire dans la grammaire. 

Dans cette perspective, le débat joue autour d'une question fondamentale. Si l'on veut inclure les cas {\guillemotleft}~intermédiaires~{\guillemotright} dans ce système binaire, quel est le niveau linguistique auquel on doit définir les notions de coordination et subordination ? Certaines approches définissent et utilisent les deux termes sur une base strictement syntaxique (\citet{Johannessen1998}, Haspelmath (2004, 2007)), d'autres les emploient pour décrire des faits relevant du discours (voir le terme \textit{discourse subordination} utilisé dans \citet{Frank2002}~ou encore la distinction entre les relations coordonnantes et subordonnantes utilisée en SDRT par \citet{Asher1993}, Asher \& \citet{Lascarides2003} et Asher \& \citet{Vieu2005} pour désigner les relations symétriques vs. asymétriques au niveau discursif). Enfin, les deux notions peuvent s'appliquer à plusieurs composantes de la grammaire : une structure peut être coordonnée en syntaxe et subordonnée en sémantique (Culicover \& \citet{Jackendoff2005}) ou vice-versa (Yuasa \& \citet{Sadock2002}). 

Il est généralement admis qu'une coordination canonique permet une lecture symétrique, c.-à-d. les conjoints peuvent être interchangés, sans affecter le sens ou la grammaticalité de la structure coordonnée, comme on voit en \REF{ex:2:10}a-b. 


\begin{enumerate}
\item \label{bkm:Ref300157651}a  Ion citeşte, iar Maria doarme.


\end{enumerate}
{\itshape
Ion lit, et Maria dort}

=  b  Maria doarme, iar Ion citeşte.

    \textit{Maria dort, et Ion lit}

Cependant, la littérature enregistre, comme on le verra dans la section \ref{sec:2.4.2}, beaucoup de coordinations avec une interprétation asymétrique, dans lesquelles l'ordre des conjoints ne peut être changé sans affecter (au moins) l'interprétation de l'ensemble (cf. \citet{Ross1967}, \citet{Schmerling1975}, \citet{Lakoff1986}, etc.). Si dans l'exemple \REF{ex:2:10} l'interchangeabilité des conjoints ne produisait aucun changement significatif concernant l'interprétation ou l'acceptabilité de la coordination en question, dans l'exemple \REF{ex:2:11} le changement dans l'ordre des conjoints affecte l'interprétation, voire l'acceptabilité de l'énoncé \REF{ex:2:11}b.


\begin{enumerate}
\item \label{bkm:Ref272684772}a  Mai bea un pahar de vin şi plec !


\end{enumerate}
{\itshape
Bois encore un verre de vin et je pars}

${\neq}$  b  ??Plec şi mai bea un pahar de vin !

    \textit{Je pars et bois encore un verre de vin}

Par conséquent, comment analyser l'exemple \REF{ex:2:11}a ? Il y a trois possibilités d'analyse. 

(i) Si on se place dans une perspective non-binaire (en termes de continuum) de la coordination et de la subordination (comme l'approche présentée précédemment dans la section \ref{sec:2.2.1}), on est amené à dire que cet exemple constitue un cas intermédiaire, hybride entre la coordination et la subordination. L'inconvénient majeur de ce type d'approche est qu'on arrive à un nombre significatif de cas `intermédiaires', en négligeant les propriétés qu'ont en commun les phrases coordonnées et respectivement les phrases subordonnées. 

(ii) Si on se place dans une perspective binaire stricte, qui postule un isomorphisme syntaxe-sémantique-discours pour la définition de la coordination et de la subordination (p.ex. une coordination doit avoir une interprétation symétrique au niveau discursif), l'exemple \REF{ex:2:11}a ne doit pas être considéré comme une occurrence de la coordination : une des deux phrases est subordonnée (\citet{Goldsmith1985}, \citet{Postal1993}). Cette analyse présente, elle aussi, un inconvénient majeur : la subordination est invoquée pour décrire une classe très hétérogène d'éléments, donc elle devient une sorte de poubelle où on met tout ce qui n'est pas coordination selon les tests diagnostics (p.ex. \textit{Coordinate Structure Constraint, Across the Board Extraction, No Backward Anaphora,} etc.).

Comme les deux approches présentées ci-dessus posent des problèmes significatifs, je présente une troisième possibilité d'analyse, que j'adopterai par la suite dans la description des deux relations. 

(iii) On peut maintenir l'idée d'un système binaire, mais en relâcheant la corrélation entre la syntaxe et les autres composantes de la grammaire, en particulier le discours. Dans cette troisième perspective, les relations de coordination et subordination sont donc définies sur une base strictement syntaxique, sans lien avec les types de relations discursives. Par conséquent, l'exemple \REF{ex:2:11}a doit être analysé comme une structure coordonnée (selon les critères syntaxiques que je présente plus bas dans la section \ref{sec:2.3}), qui entretient une relation asymétrique au niveau sémantico-discursif.

Dans les deux sections qui suivent (2.3 et 2.4), je veux montrer~les points suivants :

- Il y a des critères formels pour bien distinguer coordination et subordination (contre une approche en termes de continuum ou gradient). La coordination et la subordination sont des notions syntaxiques dénotant les relations qui peuvent exister entre les éléments d'une unité syntaxique complexe.

- La notion de symétrie / asymétrie discursive est orthogonale à la distinction coordination / subordination. En particulier, les coordinations semblent être sous-spécifiées quant aux types de relations discursives établies. Par conséquent, on peut avoir des coordinations qui entretiennent des relations discursives asymétriques.

- Les cas de discordance peuvent être facilement traités si l'on dissocie les niveaux linguistiques, en particulier la syntaxe vs. le discours. Pas d'isomorphisme entre la syntaxe, la sémantique et le discours. Pas de parallélisme entre la coordination comme phénomène syntaxique et les relations symétriques au niveau du discours (voir aussi Blühdorn (2008)).

Le besoin d'avoir des critères syntaxiques pour distinguer coordination et subordination est justifié typologiquement. Il y a des langues qui ont des marqueurs `communs' pour plusieurs types de relations sémantiques, mais leur distribution syntaxique est déterminée par le type de relation en question. Selon \citet{Dixon2009}, il y a certaines langues (en particulier le groupe océanique des langues austronésiennes, comme c'est le cas en toqabaqita dans l'exemple \REF{ex:2:13}) qui utilisent le même marqueur pour exprimer la disjonction (l'équivalent de la conjonction \textit{ou} en français) et la condition (l'équivalent du complémenteur \textit{si} en français). Quand il est utilisé pour indiquer une disjonction, le marqueur ne peut introduire que la deuxième phrase (comme toute conjonction de coordination) ; en revanche, s'il impose une interprétation conditionnelle aux phrases liées, la phrase introduite par ce marqueur apparaît obligatoirement antéposée à l'autre (l'antéposition étant un critère pour la subordination syntaxique). Les règles sont résumées en~\REF{ex:2:12}. En \REF{ex:2:13}, on a une illustration de ce phénomène en toqabaqita (une langue austronésienne), qui présente un marqueur \textit{mada} compatible avec la disjonction et la condition~(cf. \citet[15]{Dixon2009}).


\begin{enumerate}
\item \label{bkm:Ref270622612}a  X `ou/si' Y = X ou Y  Disjonction


\end{enumerate}
  b  `ou/si' X, Y = si X, Y  Condition


\begin{enumerate}
\item \label{bkm:Ref300162349}a  faka  ba=e  fula  \textbf{mad}=e  aqi?


\end{enumerate}
bateau  that=\textsc{3sg.non-fut}  arriver  ou=\textsc{3sg.non-fut  neg}

{\itshape
Le bateau est-il arrivé à temps ou pas~}

  b  \textbf{mada}  s=o  sua-na  iqa  naqi,  rake-na

si  \textsc{irreal=2sg.non-fut } toucher.3\textsc{obj } poisson  ce,  ventre.\textsc{3sg } 

    ka  boe  nena

\textsc{3sg.sequent}  gonfler  \textsc{non-past}

    \textit{Si tu touches ce poisson, son ventre se gonflera}

\subsection{Identification des phrases liées}
\label{bkm:Ref301427467}Dans cette section, je présente d'abord la distribution des éléments qui introduisent les phrases coordonnées et subordonnées en roumain (section \ref{sec:2.3.1}). Les tests diagnostics utilisés permettront de distinguer les différents types d'introducteurs (conjonctions, adverbes connecteurs ou complémenteurs). Ensuite, je donne les tests diagnostics pour distinguer la coordination et la subordination comme deux phénomènes à part, avec des contraintes syntaxiques spécifiques (section \ref{sec:2.3.2}). 

\subsubsection{Tests diagnostics des introducteurs}
\label{bkm:Ref300787466}Les phrases coordonnées sont composées d'au moins deux phrases reliées par une conjonction (de coordination) ou simplement juxtaposées (sans conjonction). En revanche, les phrases subordonnées sont introduites soit par des complémenteurs, soit par des syntagmes extraits contenant un mot interrogatif ou exclamatif, ou bien, dans certains cas, elles peuvent être juxtaposées à la phrase racine.

Les ouvrages moins formels utilisent parfois la notion de \textit{conjonction} en variation libre avec le terme de \textit{connecteur} et mettent sous ces étiquettes des éléments assez hétérogènes~du point de vue syntaxique, en particulier de vraies conjonctions (de coordination) et des adverbes. Je distingue ici la notion syntaxique de \textit{conjonction} et la notion sémantique de \textit{connecteur}. Les connecteurs sont des éléments appartenant à différentes catégories syntaxiques, dont le rôle est de coder les relations sémantico-discursives qui s'établissent entre deux ou plusieurs unités ; ils explicitent l'existence d'une relation entre deux segments consécutifs du même discours ou dialogue, ce qui justifie le terme de \textit{connecteurs pragmatiques} (\citet{Berrendonner1983}, Blühdorn (2008)).~Dans ce sens, on va dire que ce qui rapproche les conjonctions des adverbes `introducteurs' est justement la possibilité de fonctionner comme connecteurs au niveau discursif.

Pour une coordination comme en \REF{ex:2:14}a, la relation discursive qui s'établit entre les conjoints (c.-à-d. relation temporelle de succession) est implicite. Elle devient explicite grâce à l'emploi de l'adverbe \textit{apoi} `ensuite' qui peut être co-occurrent avec la conjonction \textit{şi} `et' \REF{ex:2:14}b. L'adverbial \textit{apoi} peut apparaître aussi dans les juxtapositions \REF{ex:2:14}c, toujours comme indicateur d'une certaine relation discursive. Ce dernier cas a amené beaucoup de linguistes à considérer que les adverbes ont un emploi conjonctif (cf. \textit{linking adverbs} dans \citet{Haspelmath2007}) quand ils sont utilisés seuls sans conjonction. 


\begin{enumerate}
\item \label{bkm:Ref272700213}a  Mi-am terminat temele \textbf{şi} am ieşit cu copiii la joacă.


\end{enumerate}
{\itshape
J'ai fini mes devoirs et je suis sortie pour jouer avec les enfants}

  b  Mi-am terminat temele \textbf{şi} apoi am ieşit cu copiii la joacă.

{\itshape
J'ai fini mes devoirs et ensuite je suis sortie pour jouer avec les enfants}

  c  Mi-am terminat temele, apoi am ieşit cu copiii la joacă.

    \textit{J'ai fini mes devoirs, ensuite je suis sortie pour jouer avec les enfants}

Cependant, les propriétés distributionnelles des conjonctions et des adverbes connecteurs sont différentes. Je mentionne brièvement les critères qui nous permettent de distinguer d'une part conjonctions et adverbes, et d'autre part conjonctions et complémenteurs. Les conjonctions se distinguent par les propriétés suivantes :

(i) La non-combinabilité des conjonctions entre elles. Une conjonction ne peut introduire une expression déjà marquée par une conjonction \REF{ex:2:15}a. En revanche, une conjonction peut se combiner avec un adverbe \REF{ex:2:15}b ou un complémenteur \REF{ex:2:15}c. 


\begin{enumerate}
\item \label{bkm:Ref300166642}a  *Maria  lucrează  \textbf{şi  dar } Ion  nu  face  nimic.


\end{enumerate}
Maria  travaille  et  mais  Ion  \textsc{neg } fait  rien 

{\itshape
Maria travaille, mais Ion ne fait rien}

  b  Mi-am terminat temele \textbf{şi} apoi am ieşit cu copiii la joacă.

    \textit{J'ai fini mes devoirs et ensuite je suis sortie pour jouer avec les enfants}

  c  Mama m-a întrebat dacă Ion doarme \textbf{şi} dacă Maria citeşte.

{\itshape
    Ma mère m'a demandé si Ion dort et si Maria lit}

Sur la base de ce critère, on doit analyser les items corrélatifs \textit{şi...şi} `et...et' et \textit{nici...nici} `ni...ni' en roumain comme des adverbes et non comme des conjonctions\footnote{De ce point de vue, le roumain se distingue des autres langues romanes où les paires correspondantes sont des conjonctions. Voir discussion dans Bîlbîie (2008).}  (cf. Bîlbîie (2008), \textit{contra GALR} (2005)), car le deuxième élément peut immédiatement suivre une conjonction (p.ex. \textit{şi} `et' ou \textit{dar} `mais' en \REF{ex:2:16}). Ainsi, on doit distinguer deux emplois homonymes de \textit{şi} en roumain : un emploi conjonctif et un emploi adverbial.


\begin{enumerate}
\item \label{bkm:Ref300305729}a  Ion  ştie  (şi)  să scrie,  (\{\textbf{dar}  {\textbar} \textbf{şi}\})  şi  să citească.


\end{enumerate}
Ion  sait  (\textsc{correl) } écrire.\textsc{subj},  (\{mais  {\textbar} et\})  \textsc{correl}  lire.\textsc{subj} 

{\itshape
Ion sait et lire, et écrire}

  b  Ana  nu  ştie  (nici)  să scrie,  (\{\textbf{dar}  {\textbar} \textbf{şi}\})  nici  să citească.

    Ana  \textsc{neg}  sait  (\textsc{correl)}  écrire.\textsc{subj},  (\{mais  {\textbar} et\})  \textsc{correl}  lire\textsc{.subj}

    \textit{Ana ne sait ni écrire, ni lire}

(ii) Placement initial et absence de mobilité dans le domaine dans lequel elles apparaissent. Une conjonction apparaît toujours à l'initiale du conjoint et jamais insérée parmi les constituants du conjoint. Sur la base de ce critère, on doit distinguer en roumain deux connecteurs adversatifs (qui reçoivent la même analyse dans les grammaires traditionnelles) : la conjonction \textit{dar} `mais' ne peut pas apparaître en dehors de sa position initiale dans le conjoint, alors que le connecteur \textit{însă} a un placement extrêmement mobile, comme on le voit en \REF{ex:2:17}b, ce qui justifie son statut d'adverbe.


\begin{enumerate}
\item \label{bkm:Ref300167827}a  Maria  lucrează,  (\textbf{dar})  Ion  (*\textbf{dar})  nu  face  nimic.


\end{enumerate}
Maria  travaille,  (mais)  Ion  (mais)  \textsc{neg } fait  rien 

{\itshape
Maria travaille, mais Ion ne fait rien}

  b  Maria  lucrează,  (\textbf{însă})  Ion  (\textbf{însă})  nu  face  (\textbf{însă})  nimic  (\textbf{însă}).

    Maria  travaille,  (pourtant)  Ion  (pourtant)  \textsc{neg } fait  (pourtant)  rien  (pourtant)

    \textit{Maria travaille, pourtant Ion ne fait rien}

Revenant aux items corrélatifs \textit{şi...şi} `et...et' et \textit{nici...nici} `ni...ni', on observe que parfois ils peuvent apparaître à l'intérieur de chaque conjoint \REF{ex:2:18}, et pas nécessairement à l'initiale du conjoint, ce qui justifie encore une fois leur comportement adverbial.


\begin{enumerate}
\item \label{bkm:Ref300305703}a  [Mi-am  făcut  (şi)  patul]  şi  [mi-am  scris  şi  temele].


\end{enumerate}
\textsc{cl}-ai  fait  (\textsc{correl) } le-lit  et  \textsc{cl}-ai  écrit  \textsc{correl } les-devoirs 

{\itshape
J'ai et rangé le lit et fait mes devoirs}

  b  [Nu  ți-ai  făcut  (nici)  patul]  şi  [nu  ți-ai  scris  nici  temele].

    \textsc{neg  cl}-as  fait  \textsc{correl}  le-lit  et  \textsc{neg  cl}-as  écrit  \textsc{correl } les-devoirs

    \textit{Tu n'as ni rangé ton lit ni fait tes devoirs}

Si une conjonction se combine avec un adverbe (comme c'était le cas en \REF{ex:2:15}b), elle est nécessairement suivie et jamais précédée par l'adverbe en question. Les adverbes qui apparaissent dans l'environnement d'une conjonction sont certains adverbes focalisants (comme p.ex. \textit{şi} `aussi'\textit{, nici} `non plus'\textit{, doar / numai} `seulement' ; voir exemple \REF{ex:2:19}a), énonciatifs (comme p.ex. \textit{de fapt~}`en fait'\textit{, în realitate} `en réalité' ; voir exemple \REF{ex:2:19}b) ou connecteurs (comme p.ex. \textit{apoi} `ensuite'\textit{, atunci} `alors', \textit{prin urmare} `par conséquent' ; voir exemple \REF{ex:2:19}c). Ces adverbes peuvent, en revanche, précéder un complémenteur (comparer le comportement de l'adverbe \textit{numai} `seulement' avec une conjonction en \REF{ex:2:20}a et avec un complémenteur en \REF{ex:2:20}b). 


\begin{enumerate}
\item \label{bkm:Ref300169766}a  Maria  nu  m-a  sunat  (*nici)  \textbf{şi } (nici)  Ion  nu  mi-a  scris.


\end{enumerate}
Maria  \textsc{neg  cl-}a  appelé  \textsc{adv}  et  \textsc{adv}  Ion  \textsc{neg  cl}-a  écrit 

{\itshape
Maria ne m'a pas appelé et Ion ne m'a pas écrit non plus}

  b  Nu  vreau  să  te  deranjez  (*de fapt)  \textbf{şi}  (de fapt)  nu  mi-e  foame.

    \textsc{neg}  veux  \textsc{mrq  cl}  dérange  (en fait)  et  (en fait)  \textsc{neg  cl}-est  faim

    \textit{Je ne veux pas te déranger et en fait je n'ai pas faim}

  c  Ion  n-a  muncit  (*prin urmare)  \textbf{şi}  (prin urmare)  nu  merită  nimic.

    Ion  \textsc{neg}-a  travaillé  (par conséquent)  et  (par conséquent)  \textsc{neg}  mérite  rien

{\itshape
    Ion n'a rien travaillé et par conséquent il ne mérite rien}


\begin{enumerate}
\item \label{bkm:Ref300169830}a  *Am  fost  la  Maria  numai  \{\textbf{şi {\textbar} dar}\}  am  stat  puțin.


\end{enumerate}
\textsc{aux.1}  été  chez  Maria,  seulement  \{et {\textbar} mais\}  \textsc{aux.1}  resté  peu 

{\itshape
J'ai été chez Maria \{et {\textbar} mais\} j'y suis resté peu}

  b  Merg la Maria numai \{\textbf{când {\textbar} dacă}\} am timp.

    \textit{Je vais chez Maria uniquement \{quand {\textbar} si\} j'ai le temps}

(iii) La non-mobilité du syntagme introduit par une conjonction. Le syntagme introduit par une conjonction ne peut jamais être extrait ou antéposé à l'initiale de la phrase \REF{ex:2:21}a-b. En revanche, la séquence introduite par un complémenteur peut être antéposée à la phrase racine \REF{ex:2:22}a-b. 


\begin{enumerate}
\item \label{bkm:Ref300178038}a  Maria lucrează, \textbf{iar} Ion citeşte.


\end{enumerate}
{\itshape
Maria travaille, et Ion lit}

  b  *\textbf{Iar}  Ion  citeşte,  Maria  lucrează.

    et  Ion  lit,  Maria  travaille

    \textit{Maria travaille et Ion lit}


\begin{enumerate}
\item \label{bkm:Ref300178055}a  Ies  cu  copiii  în  parc,  \textbf{dacă}  nu  plouă.


\end{enumerate}
sors\textsc{.2sg}  avec  les-enfants  en  parc,  si  \textsc{neg}  pleut 

{\itshape
Je sors avec les enfants dans le parc, s'il ne pleut pas}

  b  \textbf{Dacă}  nu  plouă,  ies  cu  copiii  în  parc.

    si  \textsc{neg}  pleut,  sors\textsc{.2sg}  avec  les-enfants  en  parc

    \textit{S'il ne pleut pas, je sors avec les enfants dans le parc}

(iv) Une conjonction introduit généralement des catégories variées (mots, syntagmes ou phrases), alors que les complémenteurs introduisent uniquement des syntagmes verbaux et des phrases.

Sur la base des critères (i) et (ii), on élimine les adverbes connecteurs (conclusifs, adversatifs ou autre), ainsi que les adverbes corrélatifs \textit{şi...şi} `et...et' et \textit{nici...nici} `ni...ni' (auxquels je reviendrai dans la section \ref{sec:2.7.1}), de la liste des conjonctions\footnote{\citet{Haspelmath2007} ajoute un critère supplémentaire pour la distinction adverbe connecteur vs. conjonction. En allemand, les adverbes (et non les conjonctions) occupent la position préverbale et imposent au sujet la position postverbale, cf. (i).
(i)  a  \textbf{und}  Lisa  kam  {\textbar} *\textbf{und}  kam  Lisa
et  Lisa  est-venue  {\textbar} et  est-venue  Lisa
et Lisa est venue
  b  *\textbf{dann}  Lisa  kam  {\textbar} \textbf{dann}  kam  Lisa
    ensuite  Lisa  est-venue  {\textbar} ensuite  est-venue  Lisa
    \textit{ensuite Lisa est venue}}. Les critères (iii)-(iv) séparent les conjonctions des complémenteurs.

Les connecteurs adverbiaux ne relient donc pas syntaxiquement les phrases comme les conjonctions ou les complémenteurs, mais ils établissent des connexions au niveau sémantico-discursif (cf. Blühdorn (2008)) ; en particulier, ils marquent explicitement le type de relation discursive établie entre les éléments liés.  

Quant aux conjonctions, les éléments qui seront pertinents pour mon étude sont : \textit{şi} `et', \textit{iar} `et', \textit{dar} `mais', \textit{ci} `mais', \textit{sau} `ou', \textit{ori} `ou' et la paire corrélative \textit{fie...fie} `soit...soit'. En laissant de côté les conjonctions qui marquent la disjonction (c.-à-d. \textit{sau, ori, fie...fie}), je présente~les conjonctions les plus importantes dans le tableau en \REF{ex:2:23}, dans une perspective typologique, afin d'observer les particularités du roumain. On observe ainsi que les fonctions sémantiques habituelles des conjonctions (c.-à-d. emploi additif, contrastif, adversatif et correctif) sont assurées par un nombre différent d'items lexicaux à travers les langues. Le français (comme l'anglais) présente un système à deux items (c.-à-d. la conjonction additive \textit{et} et la conjonction adversative \textit{mais}). L'espagnol (comme l'allemand) possède un système à trois éléments dans lequel on lexicalise la distinction entre l'emploi correctif \textit{sino} et l'emploi non-correctif \textit{pero}. En revanche, le système du roumain est encore plus complexe, car il opère la distinction entre l'emploi correctif (\textit{ci}) et l'emploi adversatif (\textit{dar}) comme en espagnol, mais il contient encore une quatrième conjonction (\textit{iar}), spécialisée pour les emplois contrastifs, ce qui rapproche le roumain des langues slaves, comme le russe, qui possèdent une conjonction similaire (\textit{a}). Comme la conjonction \textit{iar} du roumain est très fréquente dans les constructions à gapping, je présenterai son comportement discursif, syntaxique et sémantique dans une section à part (section \ref{sec:2.9}).  


\begin{enumerate}
\item   \label{bkm:Ref300180933}L'espace des conjonctions en quatre langues


\end{enumerate}

\begin{table}


\begin{tabular}{llll}

 {\bfseries Français}\par & {\bfseries Espagnol}\par & {\bfseries Roumain}\par & {\bfseries Russe}\par\\
 {\itshape et}\par & {\itshape y}\par & {\itshape şi}\par & {\itshape i}\par\\
 &  & {\itshape iar}\par & {\itshape a}\par\\
 {\itshape mais}\par & {\itshape pero}\par &  & \\
 &  & {\itshape dar}\par & {\itshape no}\par\\
\hhline{~---} & {\itshape sino}\par & {\itshape ci}\par & {\itshape a}\par\\
\hhline{~---}

\end{tabular}

\caption{}
%\label{}
\end{table}

Avant de finir cette section, je veux juste préciser que le roumain dispose aussi de certaines expressions lexicalisées comme \textit{precum şi, ca şi, cât şi} `ainsi que', qui fonctionnent aujourd'hui comme des conjonctions (au moins dans certains de leurs emplois, car la séquence qu'elles introduisent ne peut être ni antéposée, ni coordonnée à une séquence introduite par une autre conjonction), mais avec des contraintes spéciales, comme c'est le cas du français \textit{ainsi que} (cf. \citet{Mouret2007}) ou encore de l'anglais \textit{as well as} (cf. Culicover \& \citet{Jackendoff2005}), qui ne peuvent pas se combiner avec une catégorie finie. Je ferai appel à ces conjonctions dans le chapitre 4 dédié aux constructions à gapping. 

\subsubsection{Tests diagnostics de la coordination et la subordination}
\label{bkm:Ref300868872}Outre les critères qui nous permettent de distinguer entre différents types d'introducteurs (conjonctions, adverbes connecteurs ou complémenteurs), on dispose des critères formels qui caractérisent différemment la coordination vs. la subordination. Nous procèdons à une distinction entre les deux constructions sur la base des critères suivants :

(i) Le nombre de phrases liées : relation obligatoirement binaire vs. relation n-aire. La coordination peut relier plus de deux phrases \REF{ex:2:24}a, alors que la subordination est une relation entre une phrase `dominante' (c.-à-d. la phrase racine) et une phrase `dominée' (c.-à-d. la phrase subordonnée) \REF{ex:2:24}b.


\begin{enumerate}
\item \label{bkm:Ref300183728}a  Maria lucrează, Ion citeşte \textbf{şi} Dan doarme.


\end{enumerate}
{\itshape
Maria travaille, Ion lit et Dan dort}

  b  Ion mi-a spus \textbf{că} Dan doarme.

    \textit{Ion m'a dit que Dan dort}

(ii) Le type de construction syntaxique dans laquelle les phrases liées apparaissent. Intuitivement, les grammaires traditionnelles considèrent que dans une relation de coordination les phrases ont le même statut (chaque phrase pouvant apparaître toute seule), ce qui justifie l'analyse de la coordination comme une construction sans tête (dans certaines grammaires non-dérivationnelles, comme HPSG). En revanche, les phrases qui établissent une relation de subordination n'ont pas le même statut (phrase racine vs. phrase subordonnée) ; contrairement aux structures coordonnées, une structure avec subordination est une construction avec tête (la tête étant identifiée à la phrase racine ou à un élément de la phrase racine qui souvent sous-catégorise la subordonnée en question). 

(iii) Comportement différent des introducteurs. Une phrase coordonnée est introduite d'habitude par une conjonction ou rien \REF{ex:2:25}a, alors qu'une phrase subordonnée est introduite par un complémenteur \REF{ex:2:25}b, un élément relatif \REF{ex:2:25}c (interrogatif ou exclamatif) ou rien \REF{ex:2:25}d\footnote{Dans le chapitre 1 (section \ref{sec:1.3.1.2}), on a établi que la marque \textit{să} dans les subordonnées au subjonctif n'est pas un complémenteur en roumain.}.


\begin{enumerate}
\item \label{bkm:Ref300184862}a  Maria lucrează, (\textbf{iar}) Ion doarme.


\end{enumerate}
{\itshape
Maria travaille, (et) Ion dort}

  b  Ion mi-a spus [\textbf{că} va veni].

    \textit{Ion m'a dit qu'il viendrait}

  c  Ion mi-a spus [\textbf{ce carte} a cumpărat].

{\itshape
    Ion m'a dit quel bouquin il a acheté}

  d  Ion mi-a spus [să cumpăr o carte].

\textit{    Ion m'a dit que j'achète un bouquin} 

Si on lie ce critère au critère (ii) mentionné ci-dessus, on observe que ni la conjonction ni le conjoint qu'elle introduit ne sont requis par le premier conjoint \REF{ex:2:26}a, alors que la subordonnée peut être sous-catégorisée par (un élément de) la phrase-racine \REF{ex:2:26}b~(cf. \citet{Johannessen1998}).


\begin{enumerate}
\item \label{bkm:Ref300185582}a  Maria lucrează, (\textbf{iar} Ion doarme). 


\end{enumerate}
\textit{Maria travaille, (et Ion dort)}   

  b  Ion se întreabă ??(\textbf{dacă} va veni Maria).

    \textit{Ion se demande si Maria va venir}

Par conséquent, le complémenteur est séléctionné par le verbe de la phrase racine, tandis que la conjonction n'est jamais séléctionnée par un élément du premier conjoint.

(iv) La position de l'introducteur. Une conjonction est toujours à l'initiale de la séquence qu'elle introduit ; le complémenteur ou l'élément jonctif est d'habitude à l'initiale, mais on peut trouver des exemples où l'élément jonctif suit un constituant antéposé, cf. \REF{ex:2:27} et \REF{ex:2:28}.


\begin{enumerate}
\item \label{bkm:Ref300186399}a  \textbf{Dacă} mai scoți o vorbă, te-am băgat la culcare.


\end{enumerate}
{\itshape
Si tu dis un mot de plus, tu iras te coucher}

  b  O vorbă \textbf{dacă} mai scoți, te-am băgat la culcare.

    \textit{Un mot si tu dis encore, tu iras te coucher}


\begin{enumerate}
\item \label{bkm:Ref300186401}a  Băiatul  în  \textbf{ale cărui}  vorbe  m-am  încrezut  m-a  dezamăgit.


\end{enumerate}
le-garçon  dans  \textsc{rel.gen}  paroles  me-suis  fié  m'a  deçu

{\itshape
Le garçon dans les paroles duquel j'ai eu confiance m'a deçu}

  b  Băiatul  în  vorbele  \textbf{căruia}  m-am  încrezut  m-a  dezamăgit.

 le-garçon  dans  paroles  \textsc{rel.gen}  me-suis  fié  m'a  deçu

  \textit{Le garçon dans les paroles duquel j'ai eu confiance m'a deçu}

(v) La permutabilité de la phrase liée. La position de la séquence introduite par une conjonction est contrainte : elle n'est jamais antéposée à la séquence qui n'est pas introduite par conjonction \REF{ex:2:29}. En revanche, une subordonnée peut être antéposée à la phrase racine \REF{ex:2:30} sans affecter le sens de l'ensemble (surtout avec les subordonnées ajouts, périphériques ou certains valents, en particulier les sujets)\footnote{\citet{Piot1993} ajoute une petite exception à ce critère : les subordonnées qui marquent la conséquence ne peuvent pas être en antéposition par rapport à la phrase racine.} : 


\begin{enumerate}
\item \label{bkm:Ref300186642}a  Nu am mai ieşit din casă, [\textbf{căci} era prea cald].


\end{enumerate}
{\itshape
Je ne suis plus sortie de chez moi, car il faisait trop chaud}

  b  *[\textbf{Căci} era prea cald], nu am mai ieşit din casă.

{\itshape
Car il faisait trop chaud, je ne suis plus sortie de chez moi}


\begin{enumerate}
\item \label{bkm:Ref300186661}a  Nu am mai ieşit din casă, [\textbf{pentru că} era prea cald].


\end{enumerate}
{\itshape
Je ne suis plus sortie de chez moi, parce qu'il faisait trop chaud}

  b  [\textbf{Pentru că} era prea cald], nu am mai ieşit din casă.

{\itshape
Parce qu'il faisait trop chaud, je ne suis plus sortie de chez moi}

(vi) Ordre des mots différent dans certaines langues (p.ex. les langues germaniques). En néerlandais et allemand, un paramètre important pour la distinction phrase racine / phrase subordonnée est l'ordre dans lequel apparaissent le verbe fini et le complément objet. Les phrases racines (et les coordonnées) ont un ordre de type SVO (le verbe apparaissant en deuxième position), tandis que les subordonnées ont un ordre de type SOV (le verbe occupant la position finale). Ainsi, en néerlandais, dans la séquence introduite par \textit{maar} `mais' en \REF{ex:2:31}a, le verbe suit le sujet et précède le clitique pronominal et la négation, alors que dans la séquence introduite par \textit{hoewel} `bien que' en \REF{ex:2:31}b, le verbe apparaît en dernière position, après le clitique pronominal et la négation.  


\begin{enumerate}
\item \label{bkm:Ref300187099}a  De  arbeiders  bleven  werken,  \textbf{maar}  de  directeur  betaalde  hen  niet.


\end{enumerate}
les  ouvriers  continuer.\textsc{pst}  travailler,  mais  le  directeur  payer.\textsc{pst  cl  neg}  

{\itshape
Les ouvriers ont continué à travailler, mais le directeur ne les a pas payés}

  b  \textbf{Hoewel}  de  directeur  hen  niet  betaalde,  bleven  de  arbeiders  werken.

    bien-que  le  directeur  \textsc{cl  neg}  payer.\textsc{pst, } continuer.\textsc{pst}  les  ouvriers  travailler

    \textit{Bien que le directeur ne les ait pas payés, les ouvriers ont continué à travailler                       } (\citet[612]{Verstraete2005})

En danois et suédois, c'est la position de la négation adverbiale et d'autres adverbiaux phrastiques qui nous indique le type de relation. Les phrases racines ont un ordre verbe-adverbe, alors que les phrases subordonnées ont l'ordre inverse adverbe-verbe, comme le montrent les deux exemples du danois en \REF{ex:2:32}a-b.


\begin{enumerate}
\item \label{bkm:Ref300187381}a  Arbejderne  fortsatte  med  at  arbejde,  \textbf{men}  direktoeren  betalte  dem ikke.


\end{enumerate}
les-ouvriers  continuer.\textsc{pst}  avec  le  travail,  mais  le-directeur  payer.\textsc{pst cl  neg}  

{\itshape
Les ouvriers ont continué à travailler, mais le directeur ne les a pas payés}

  b  \textbf{Skoent}  direktoeren  ikke  betalte  dem,  fortsatte  arbejderne  med  at

    bien-que  le-directeur  \textsc{neg}  payer.\textsc{pst  cl, } continuer.\textsc{pst } les-ouvriers  avec  le 

  arbejde

    travail

    \textit{Bien que le directeur ne les ait pas payés, les ouvriers ont continué à travailler                       } (\citet[612]{Verstraete2005})

(vii) Anaphore ou cataphore pronominale. Les phrases subordonnées se caractérisent par la présence possible d'une cataphore pronominale (angl. \textit{Backwards Anaphora}), co-indicée avec une expression référentielle de la phrase racine. En revanche, dans les coordonnées, l'expression référentielle précède généralement l'expression pronominale qui lui est co-indicée (angl. \textit{No Backward Anaphoricity}, cf. \citet{Lang1984}, \citet{Haspelmath2004}, etc.). Le premier conjoint est ainsi un contexte accessible au moment du processing du deuxième conjoint et non vice-versa. 

Dans une relation de coordination, comme en \REF{ex:2:33}a, la co-indiciation entre une expression référentielle du premier conjoint et une expression anaphorique du deuxième conjoint est possible et naturelle ; en \REF{ex:2:33}b, la co-indiciation entre une expression cataphorique du premier conjoint et une expression référentielle se trouvant dans le second conjoint est peu plausible (voir l'observation de Zribi-\citet{Hertz1996} sur les contextes émotifs et contrastifs qui permettent la violation de cette contrainte). En revanche, dans la subordonnée ajout en \REF{ex:2:34}, la co-indiciation est possible, indépendamment de la position de la subordonnée.


\begin{enumerate}
\item \label{bkm:Ref271663057}a  Ion\textsubscript{i} aduce de toate în casă, \textbf{iar} nevastă-sa\textsubscript{i} îşi bate joc de el\textsubscript{i}.


\end{enumerate}
\textit{Ion}\textit{\textsubscript{i}}\textit{ apporte tout ce qu'il faut à la maison, et sa}\textit{\textsubscript{i}}\textit{ femme se moque de lui}\textit{\textsubscript{i}}

  b  ??Nevastă-sa\textsubscript{i} îşi bate joc de el\textsubscript{i}, \textbf{iar} Ion\textsubscript{i} aduce de toate în casă.

\textit{Sa}\textit{\textsubscript{i}}\textit{ femme se moque de lui}\textit{\textsubscript{i}}\textit{, et Ion}\textit{\textsubscript{i}}\textit{ apporte tout ce qu'il faut à la maison}


\begin{enumerate}
\item \label{bkm:Ref271663349}a  Dan\textsubscript{i} e foarte calm, \textbf{deşi} toată lumea îl\textsubscript{i} acuză.


\end{enumerate}
\textit{Dan}\textit{\textsubscript{i}}\textit{ est très calme, même si tout le monde l}\textit{\textsubscript{i}}\textit{'accuse}

  b  \textbf{Deşi} toată lumea îl\textsubscript{i} acuză, Dan\textsubscript{i} e foarte calm.

\textit{Bien que tout le monde l}\textit{\textsubscript{i}}\textit{'accuse, Dan}\textit{\textsubscript{i}}\textit{ est très calme}

(viii) Possibilité d'extraction.~La coordination interagit de façon intéressante avec la syntaxe de l'extraction. Depuis \citet{Ross1967}, on sait que l'extraction d'éléments dans les structures coordonnées obéit à des restrictions fortes. En \REF{ex:2:35}a-b, je reproduis la contrainte définie par \citet{Grosu1973}. 


\begin{enumerate}
\item \label{bkm:Ref272240177}\label{bkm:Ref300868828}Contrainte sur les Structures Coordonnées\footnote{Je reprends la version française donnée par \citet[34]{Mouret2007}.} (angl. \textit{Coordinate Structure Constraint})\textit{~}: 


\end{enumerate}
  Dans une structure coordonnée,

  a  aucun des termes conjoints ne peut être extrait ou cliticisé (\textit{Conjunct Constraint}) ;

  b  l'extraction ou la cliticisation d'un constituant hors d'un terme conjoint est interdite à moins d'opérer simultanément hors de chacun des termes (\textit{Element Constraint}).

On observe ainsi que, contrairement aux subordonnées \REF{ex:2:36}, une structure coordonnée obéit à cette contrainte d'extraction parallèle, que ça soit dans les phrases racines \REF{ex:2:37} ou avec les coordinations de subordonnées \REF{ex:2:38}. Néanmoins, il faut noter qu'il y a certaines exceptions à cette contrainte. Pour une discussion sur le facteur linguistique qui joue sur la contrainte d'extraction dans les coordinations, voir discussion à la fin de la section \ref{sec:2.4.2}). 


\begin{enumerate}
\item \label{bkm:Ref300245582}a  Ion i-a spus Mariei [\textbf{că} merge la mare cu părinții].


\end{enumerate}
{\itshape
Ion a dit à Maria qu'il va à la mer avec ses parents}

 b  Cui i-a spus \_ Ion [\textbf{că} merge la mare cu părinții] ?

  \textit{A qui a dit Ion qu'il va à la mer avec ses parents}

  c  Cu cine ți-a spus Ion [\textbf{că} merge la mare \_ ] ?

{\itshape
Avec qui t'a dit Ion qu'il va à la mer}


\begin{enumerate}
\item \label{bkm:Ref300245589}a  Am vorbit cu Maria \textbf{şi} am stabilit o întâlnire cu ea lunea viitoare.


\end{enumerate}
{\itshape
J'ai parlé avec Maria et j'ai établi un rendez-vous avec elle lundi prochain}

 b  *Cu cine ai vorbit \_ \textbf{şi} ai stabilit o întâlnire cu ea ?

  \textit{Avec qui tu as parlé \_ et tu as établi un rendez-vous avec elle}

  c  Cu cine ai vorbit \_ \textbf{şi} ai stabilit o întâlnire~\_ ?

{\itshape
Avec qui tu as parlé et tu as établi un rendez-vous}


\begin{enumerate}
\item \label{bkm:Ref272879507}a  Știu pe cineva [căruia i s-a făcut \_ transplant de inimă] \textbf{şi} [căruia i-a eşuat \_ operația].


\end{enumerate}
{\itshape
Je connais quelqu'un à qui on a fait un transplant de c{\oe}ur et à qui a échoué l'opération} 

  b  *Știu pe cineva [căruia i s-a făcut \_ transplant de inimă] \textbf{şi} [a eşuat operația acestuia].

{\itshape
Je connais quelqu'un à qui on a fait un transplant de c{\oe}ur et son opération a échoué} 

  c  *Știu pe cineva [căruia i s-a făcut transplant de inimă acestuia] \textbf{şi} [căruia i-a eşuat \_ operația].

{\itshape
Je connais quelqu'un à qui on a fait un transplant de c{\oe}ur à celui-ci et à qui a échoué l'opération} 

(ix) Type d'unités liées (cf. \citet{Haspelmath2007}). La coordination concerne toute unité syntaxique -- mots, syntagmes ou phrases --, alors que la subordination intervient seulement au niveau phrastique, c.-à-d. une subordonnée est toujours une phrase (verbale, averbale ou fragmentaire). 

Dans ce sens, \citet{Piot1993} ajoute un test distinguant la coordination et la subordination qui s'applique au français et à l'anglais (mais pas aux langues pro-drop comme le roumain) : l'impossibilité d'~{\guillemotleft}~élider~{\guillemotright} le sujet de la subordonnée lorsque celui-ci est identique ou coréférent à celui de la phrase racine \REF{ex:2:39}b, alors qu'une conjonction autorise les coordinations de verbes ou syntagmes verbaux partageant le même sujet \REF{ex:2:39}a.


\begin{enumerate}
\item \label{bkm:Ref300266165}a  Marie\textsubscript{i} parle et (elle\textsubscript{i}) lit.  


\end{enumerate}
  b  Marie\textsubscript{i} chante plus qu'*(elle\textsubscript{i}) rêve. 

(x) Présence ou absence d'acte illocutoire. Comme on l'a vu dans le chapitre 1 (section \ref{sec:1.2.1.2}), les subordonnées se comportent différemment des autres phrases (racines indépendantes ou racines coordonnées) en ce qui concerne la force illocutoire (cf. Foley \& Van \citet{Valin1984}, \citet{Cristofaro2003}). Tandis que les dernières ont un acte illocutoire (dans le cas des phrases racines coordonnées : soit un seul acte pour la coordination dans son ensemble, soit un acte pour chaque conjoint), une subordonnée n'a pas de force illocutoire, sauf dans les cas où le verbe enchâssant de la phrase racine est un {\guillemotleft}~recteur faible~{\guillemotright} parenthétique (cf. Blanche-Benveniste \& \citet{Willems2007}), c.-à-d. une forme verbale qui ne constitue pas la prédication principale de l'énoncé, même lorsqu'elle est en position de verbe enchâssant.\footnote{Les recteurs faibles prototypiques en français sont : \textit{je trouve, je crois, je pense, il me semble, paraît-il}.} Généralement, ce type de {\guillemotleft}~recteurs faibles~{\guillemotright} joue dans l'énoncé un rôle modal, épistémique, évidentiel, etc.


\begin{enumerate}
\item a  Ion mi-a spus [\textbf{că} mâine plouă].


\end{enumerate}
{\itshape
Ion m'a dit que demain il pleuvrait}

  b  Cred [\textbf{că} mâine plouă].

{\itshape
Je crois que demain il pleuvra}

(xi) Dans les langues qui présentent une construction clivée spécifique (p.ex. le français \textit{c'est...que}), on observe que seulement les subordonnées peuvent être clivées. C'est ainsi qu'on arrive à faire la différence entre deux connecteurs qui établissent le même type de relation sémantico-discursive (c.-à-d. la cause) en \REF{ex:2:41}, mais qui ont des propriétés syntaxiques différentes.


\begin{enumerate}
\item \label{bkm:Ref300265540}a  Elle se méfiait de lui \{\textbf{car} {\textbar} \textbf{parce qu'}\}elle le connaissait.


\end{enumerate}
  b  C'est \textbf{parce qu}'elle le connaissait qu'elle se méfiat de lui.

  c  *C'est \textbf{car} elle le connaissait qu'elle se méfiait de lui.

Pour conclure, il y a des critères formels pour bien distinguer coordination et subordination (contre une approche en termes de continuum ou gradient). La coordination et la subordination sont des notions \textit{syntaxiques} dénotant les relations qui peuvent exister entre les éléments d'une unité syntaxique complexe.

\subsection{Le parallélisme dans la coordination}
\label{bkm:Ref301814089}La coordination repose sur un mécanisme d'itération (obligatoire et potentiellement non-bornée) d'une catégorie syntaxique (Sag \textit{et al.} (1985)). Traditionnellement, on considère que cette itération doit être définie en termes d'identité ou similarité des termes coordonnés. Cependant, la manière dont on définit cette contrainte d'identité sur les termes coordonnés est source d'ambiguïté et confusion : on considère souvent que les unités coordonnées sont syntaxiquement {\guillemotleft}~équivalentes~{\guillemotright}, {\guillemotleft}~ont le même statut~{\guillemotright}, {\guillemotleft}~jouent le même rôle~{\guillemotright} dans un contexte syntaxique donné (Riegel \textit{et al.} (1994), \citet{Dik1997}, Huddleston \& \citet{Pullum2002}, \citet{Haspelmath2004}, \textit{GALR} (2005), Carston \& \citet{Blakemore2005}, etc.). Le principe de similarité qui caractérise les tours coordonnés varie d'un auteur à l'autre entre identité de catégorie syntaxique, identité de fonction syntaxique, identité sémantique ou encore identité discursive. 

Ce principe d'identité que dans les années 1970 et 1980 les grammaires syntagmatiques considéraient utile pour rendre compte des propriétés des coordinations est résumé en \REF{ex:2:42} (cf. \citet{Jackendoff1977}, Gazdar \textit{et al.} (1985)) :


\begin{enumerate}
\item \label{bkm:Ref268609534}X $\rightarrow $ X\textsubscript{1}, X\textsubscript{2} ... \textbf{Conj} X\textsubscript{n}


\end{enumerate}
Cependant, on sait aujourd'hui que cette règle rend compte uniquement des cas prototypes de la coordination et qu'il y a bien des données plus compliquées qui ne rentrent pas dans ce schéma. Une grammaire compléte de la coordination doit en tenir compte et disposer des outils nécessaires pour la description de ces cas problématiques. 

La violation de l'identité stricte dans la coordination peut être observée à plusieurs niveaux linguistiques. Dans un premier temps, je me concentre sur les dissemblances qu'on observe au niveau morpho-syntaxique (section \ref{sec:2.4.1}) et ensuite je discute les asymétries qui relèvent de la sémantique et du discours (section \ref{sec:2.4.2}). Une synthèse sera faite dans la section \ref{sec:2.4.3}.  

\subsubsection{Degré d'identité au niveau morpho-syntaxique} 
\label{bkm:Ref300780566}Les travaux en grammaire syntagmatique généralisée (angl. \textit{Generalized Phrase Structure Grammar} ou GPSG, cf. Gazdar \textit{et al.} (1985), Sag \textit{et al.} (1985)) nous ont fait découvrir beaucoup de cas où la contrainte d'identité stricte n'est pas respectée. Un premier exemple dans ce sens est la coordination de catégories dissemblables (angl. \textit{coordination of unlikes}, cf. \citet{Yatabe2004}, \citet{Sag2005}), où les conjoints n'appartiennent pas à la même catégorie syntaxique. Les cas les plus simples et les plus fréquents concernent la coordination de syntagmes sous-phrastiques (syntagmes nominaux, adjectivaux, prépositionnels et adverbiaux, comme en \REF{ex:2:43}), mais on enregistre aussi des coordinations d'un syntagme sous-phrastique (un syntagme nominal ou adjectival) et une phrase (une subordonnée {\guillemotleft}~libre~{\guillemotright}, sans complémenteur, au subjonctif \REF{ex:2:44}, ou bien une subordonnée dont l'introducteur est un pronom relatif \REF{ex:2:45}).


\begin{enumerate}
\item \label{bkm:Ref269821144}a  Ion este [mincinos]\textsubscript{AdjP} şi [mare măgar]\textsubscript{NP}.


\end{enumerate}
{\itshape
Ion est menteur et un grand âne}

  b  Maria este [relaxată]\textsubscript{AdjP} şi [în foarte bună formă]\textsubscript{PP}.

{\itshape
Maria est détendue et en très bonne forme} 

  c  Maria lucrează [repede]\textsubscript{AdvP} şi [cu multă atenție]\textsubscript{PP}. 

{\itshape
Maria travaille vite et avec beaucoup d'attention}


\begin{enumerate}
\item \label{bkm:Ref300509910}a  Românului  îi  place  [grătarul]\textsubscript{NP}  şi  [să bârfească  cu  prietenii]\textsubscript{S}.


\end{enumerate}
    le-roumain\textsc{.dat  cl}  plaît  le-barbecue  et  causer.\textsc{subj}  avec  les-amis

{\itshape
Les roumains aiment le barbecue et bavarder avec leurs amis}

  b  [O vacanță la Roma]\textsubscript{NP} şi [să schiez în Alpi]\textsubscript{S} e tot ce îmi doresc anul acesta.

{\itshape
Des vacances à Rome et skier dans les Alpes est tout ce que je veux pour cette année}


\begin{enumerate}
\item \label{bkm:Ref300509933}a  Iți cumpăr [o carte]\textsubscript{NP} şi [\textbf{ce} mai doreşti]\textsubscript{S}.


\end{enumerate}
\textsc{    cl.2sg} achète.1\textsc{sg} un livre et \textsc{rel adv} désires.2\textsc{sg}

{\itshape
Je t'achète un livre et ce que tu désires encore}

  b  Angajăm vorbitori de limbă engleză [foarte experimentați]\textsubscript{AdjP} sau [\textbf{care} au fost şcoliți la o universitate de renume în SUA]\textsubscript{S}. 

{\itshape
On emploie des locuteurs de langue anglaise très expérimentés ou qui ont fait leurs études dans une université renommée aux Etats-Unis}

  c  [Perfecționarea cadrelor]\textsubscript{NP} şi [\textbf{cum} să lucrezi cât mai mult cu cât mai puțin efort]\textsubscript{S} reprezintă cele două teme majore ale reuniunii de azi.

{\itshape
Le perfectionnement des cadres et comment travailler plus avec moins d'efforts représentent les deux thèmes majeurs de la réunion d'aujourd'hui}

La généralisation dite {\guillemotleft}~de Wasow~{\guillemotright} (Gazdar \textit{et al.} (1985), Pullum \& \citet{Zwicky1986}) permet de capter les données mentionnées plus haut concernant la possibilité d'avoir une identité partielle entre les termes coordonnés. 


\begin{enumerate}
\item \label{bkm:Ref300654637}Généralisation de Wasow (cf. \citet[26]{Mouret2007})


\end{enumerate}
Une construction coordonnée est syntaxiquement bien formée dans un contexte phrastique si et seulement si chacun des termes coordonnés peut apparaître seul dans ce contexte sans en altérer les propriétés.

La généralisation de Wasow permet ainsi une explication des variations d'acceptabilité observées ci-dessous en \REF{ex:2:47}a et \REF{ex:2:48}a : dans le premier exemple, chaque terme coordonné peut apparaître seul dans la phrase, cf. \REF{ex:2:47}b-c, mais en \REF{ex:2:48}, on coordonne deux termes imposant des restrictions différentes sur le verbe en facteur : le syntagme nominal \textit{cărțile} `les livres' impose l'accord du verbe au pluriel, cf. \REF{ex:2:48}b, tandis que la phrase au subjonctif \textit{să gătească} `cuisiner' est compatible seulement avec un singulier \REF{ex:2:48}c.


\begin{enumerate}
\item \label{bkm:Ref269754757}a  Românului  îi  place  [grătarul]\textsubscript{NP}  şi  [să discute  cu  prietenii]\textsubscript{S}.


\end{enumerate}
    le-roumain\textsc{.dat  cl}  plaît  le-barbecue  et  causer.\textsc{subj}  avec  les-amis

{\itshape
Les roumains aiment le barbecue et bavarder avec leurs amis}

  b  Românului  îi  place  grătarul.

    le-roumain\textsc{.dat  cl}  plaît  le-barbecue

{\itshape
Les roumains aiment le barbecue}

  c  Românului  îi  place  să discute  cu  prietenii.

    le-roumain\textsc{.dat  cl}  plaît  causer.\textsc{subj}  avec  les-amis

{\itshape
Les roumains aiment bavarder avec leurs amis}


\begin{enumerate}
\item \label{bkm:Ref269755045}a  ??Mamei  îi  plac  [cărțile]\textsubscript{NP}  şi  [să gătească]\textsubscript{S}.


\end{enumerate}
la-mère\textsc{.dat  cl}  plaisent  les-livres  et  \textsc{mrq} cuisiner.\textsc{subj}

{\itshape
Ma mère aime lire et cuisiner}

  b  Mamei  îi  plac  cărțile.

la-mère\textsc{.dat  cl}  plaisent  les-livres

{\itshape
    Ma mère aime lire}

  c  *Mamei  îi  plac  să gătească.

la-mère\textsc{.dat  cl}  plaisent  cuisiner.\textsc{subj}

{\itshape
    Ma mère aime cuisiner}

Il faut préciser que certaines approches (voir van \citet{Oirsouw1987} et \citet{Wilder1997}, entre autres, dans le cadre des Principes et Paramètres, ainsi que \citet{Crysmann2003}, Beavers \& \citet{Sag2004} et \citet{Chaves2007} dans le cadre HPSG) dérivent la coordination de termes dissemblables du phénomène plus général de l'ellipse\footnote{Dans ce type d'approches, la coordination de termes dissemblables est un sous-type de \textit{Left Peripheral Ellipsis} ou \textit{Conjunction Reduction}, mettant en jeu une règle d'ellipse sur la frontière gauche du second conjoint. Cependant, cette analyse est problématique pour les coordinations en \REF{ex:2:44}b et \REF{ex:2:45}c, où on n'arrive pas bien à voir quel serait le matériel élidé.} . L'exemple \REF{ex:2:47}a, repris en \REF{ex:2:49}a, serait donc une version elliptique de \REF{ex:2:49}b. 


\begin{enumerate}
\item \label{bkm:Ref272339508}a  Românului îi place grătarul şi să discute cu prietenii.


\end{enumerate}
{\itshape
Les roumains aiment le barbecue et causer avec leurs amis}

  b  Românului îi place grătarul şi îi place să discute cu prietenii.

Cependant, Mouret (2007, 2008) donne trois arguments empiriques convaincants en faveur d'une analyse sans ellipse des coordinations de termes dissemblables en français. Je les présente brièvement par la suite, avec des exemples repris de \citet{Mouret2008}. Tout d'abord, il observe (à la suite d'Abeillé \& \citet{Godard1996}) que certains items lexicaux ne peuvent jamais être suivis d'une catégorie verbale finie ; or, ces items peuvent introduire le dernier conjoint dans une coordination de termes dissemblables. Ainsi, la conjonction \textit{ainsi que}, présente dans une coordination de termes dissemblables en \REF{ex:2:50}a, ne permet pas la reconstruction syntaxique du verbe \textit{adorer} dans le deuxième terme coordonné en \REF{ex:2:50}a. Le même type de contraste est observé avec l'adverbe \textit{non pas} en \REF{ex:2:50}c, réservé à la négation de constituant, qui peut introduire le deuxième terme coordonné dans une coordination de catégories dissemblables \REF{ex:2:50}c, mais il ne peut jamais se combiner avec une forme verbale finie \REF{ex:2:50}d.


\begin{enumerate}
\item \label{bkm:Ref300512867}a  Marie adore le cinéma \textbf{ainsi que} faire les boutiques.


\end{enumerate}
  b  *Marie adore le cinéma \textbf{ainsi que} adore faire les boutiques.

  c  Marie adore le cinéma et \textbf{non pas} faire les boutiques.

  d  *Marie adore le cinéma et \textbf{non pas} adore faire les boutiques.

Un deuxième argument concerne la position des conjonctions initiales dans les coordinations omnisyndétiques (ou corrélatives, voir section \ref{sec:2.7.1}). Dans les coordinations de termes dissemblables, la conjonction initiale précède immédiatement le premier terme coordonné (comme on voit avec la conjonction \textit{et} en \REF{ex:2:51}a ou \textit{ni} en \REF{ex:2:51}c) et ne peut pas précéder le verbe mis en facteur en \REF{ex:2:51}b et \REF{ex:2:51}d comme cela est attendu dans une approche à base d'ellipse.


\begin{enumerate}
\item \label{bkm:Ref272087433}a  Marie adore (\textbf{et}) le cinéma \textbf{et} faire les boutiques.


\end{enumerate}
  b  Marie (*\textbf{et}) adore le cinéma \textbf{et} faire les boutiques.

  c  Marie n'adore (\textbf{ni}) le cinema \textbf{ni} faire les boutiques

  d  Marie (*\textbf{ni}) n'adore le cinema \textbf{ni} faire les boutiques

Finalement, une approche à base d'ellipse rencontre des difficultés à expliquer les phénomènes d'association des adverbes additifs ou restrictifs. Un adverbe comme l'additif \textit{aussi} en \REF{ex:2:52}a ou encore le restrictif \textit{seulement} en \REF{ex:2:52}b peut immédiatement précéder le premier terme coordonné et prendre comme associé sémantique la coordination dans son ensemble. Cette association large des adverbes ne peut pas être expliquée de manière adéquate si l'on considère que l'adverbe est enchâssé dans le premier terme conjoint. Il faut que l'adverbe ait accès localement au contenu de la coordination dans son ensemble pour pouvoir dériver l'interprétation additive ou restrictive. 


\begin{enumerate}
\item \label{bkm:Ref300521332}a  Marie adore \textbf{aussi} le cinéma et faire les boutiques.


\end{enumerate}
  b  Marie adore \textbf{seulement} le cinéma et faire les boutiques.

A la lumière de ces arguments, on doit donc postuler que le parallélisme syntaxique ne se réduit pas à une identité stricte entre les éléments coordonnés. Les termes coordonnés peuvent avoir une catégorie syntaxique différente, à condition que le prédicat avec lequel ils se combinent accepte alternativement chacun d'eux (Sag \textit{et al.} (1985)), ce qui peut être facilement formalisé dans une grammaire à base de contraintes, comme le HPSG, grâce au mécanisme de sous-spécification (pour des détails sur la formalisation, voir la section \ref{sec:2.10.3}). On a donc besoin d'une théorie du degré d'identité des termes conjoints indépendamment d'une théorie de l'ellipse dans la grammaire.

De manière générale, cette contrainte syntaxique gère non seulement les différences catégorielles, mais aussi les différences de mode ou encore les différences de marqueur qu'on peut observer dans une coordination à travers les langues. Comme le montrent les exemples repris de Delais-Roussarie \& \citet{Mouret2008} pour le français, on peut ainsi coordonner une subordonnée à l'indicatif et une subordonnée au subjonctif à condition que chaque subordonnée soit compatible individuellement avec le prédicat mis en facteur, ce qui explique l'acceptabilité de l'exemple \REF{ex:2:53} et l'agrammaticalité de l'exemple \REF{ex:2:54}. De même, on peut coordonner deux syntagmes prépositionnels introduits par des prépositions différentes (comme c'est le cas des prépositions \textit{de} et \textit{par} sélectionnées par un verbe à la voix passive en français) si le verbe mis en facteur est compatible alternativement avec les deux marqueurs, comme le montre le contraste en \REF{ex:2:55}a et \REF{ex:2:56}a entre une expression verbale comme \textit{être apprécié}, qui accepte les deux cf. \REF{ex:2:55}, et le verbe \textit{être critiqué}, qui n'accepte que la préposition \textit{par} cf. \REF{ex:2:56}. 


\begin{enumerate}
\item \label{bkm:Ref300597513}a  Il n'est pas certain que ton père pourra t'emmener ni que ta mère puisse venir te chercher.


\end{enumerate}
  b  Il n'est pas certain que ton père pourra t'emmener.

  c  Il n'est pas certain que ta mère puisse venir te chercher.


\begin{enumerate}
\item \label{bkm:Ref300597641}a  *Il est certain que ton père pourra t'emmener ni que ta mère puisse venir te chercher.


\end{enumerate}
  b  Il est certain que ton père pourra t'emmener.

  c  *Il est certain que ta mère puisse venir te chercher.


\begin{enumerate}
\item \label{bkm:Ref300598417}a  Pierre est apprécié de ses collègues et par tous ses étudiants.


\end{enumerate}
  b  Pierre est apprécié de ses collègues.

  c  Pierre est apprécié par tous ses étudiants.


\begin{enumerate}
\item \label{bkm:Ref300598419}a  *Pierre est critiqué de ses collègues et par tous ses étudiants.


\end{enumerate}
  b  *Pierre est critiqué de ses collègues.

  c  Pierre est critiqué par tous ses étudiants.

En roumain, on peut coordonner deux syntagmes nominaux qui reçoivent le marquage casuel de manière différente. Ainsi, en \REF{ex:2:57}, le premier syntagme nominal coordonné reçoit le cas datif par la marque \textit{-lor} suffixée au nom (\textit{băieților}), alors que le deuxième terme coordonné reçoit plutôt la préposition \textit{la} sélectionnant une forme nominale d'accusatif ; donc, l'information casuelle est donnée ici par des marques différentes (cf. la distinction traditionnelle entre le datif synthétique, obtenu à l'aide d'une marque affixale, et le datif analytique, obtenu à l'aide d'une marque prépositionnelle). 


\begin{enumerate}
\item \label{bkm:Ref300653737}a  Le-am  dat  mere  băieților  şi  la  trei  fete.


\end{enumerate}
\textsc{cl}-ai  donné  pommes  les-garçons.\textsc{dat } et  à  trois  filles\textsc{.acc}

  \textit{J'ai donné des pommes aux garçons}\textsc{} \textit{et à trois filles}

b  Le-am  dat  mere  \{băieților  {\textbar} la  trei  fete\}.

  \textsc{cl}-ai  donné  pommes  \{les-garçons.\textsc{dat  {\textbar}} à  trois  filles\textsc{.acc}

    \textit{J'ai donné des pommes \{aux garçons}\textsc{} \textit{{\textbar} à trois filles}\textsc{\}}

Un autre défi pour les approches de la coordination en termes d'identité syntaxique est le problème de la neutralisation lexicale (angl. \textit{feature neutrality} ou \textit{feature indeterminacy}, cf. \citet{Daniels2002}, Levy \& \citet{Pollard2002}, \citet{Sag2005}, \citet{Crysmann2006}, etc.). Cette fois-ci, un seul et même élément apparaît dans un contexte où il doit satisfaire simultanément deux contraintes grammaticales mutuellement incompatibles, autrement dit un élément qui est mis en facteur dans une coordination peut être compatible avec chaque terme coordonné, ayant pour chaque conjoint une valeur grammaticale distincte. 

Un exemple est le syncrétisme casuel qui caractérise certaines formes pronominales en français et en roumain. En français, le clitique pronominal \textit{nous} est à la fois compatible avec un verbe sous-catégorisant un objet direct et un verbe sélectionnant un oblique, ce qui explique la possibilité d'avoir en \REF{ex:2:58}a une coordination lexicale de deux verbes avec des régimes différents, dans la dominance syntaxique d'un auxiliaire accompagné du clitique \textit{nous}. En revanche, à la troisième personne du singulier on n'a pas de forme pronominale neutre qui soit compatible simultanément avec les deux verbes mentionnés, mais deux formes distinctes \textit{lui} et \textit{le}, chacune étant compatible uniquement avec un régime du verbe, cf. \REF{ex:2:59}. Les mêmes faits sont observés en roumain, mais dans des exemples marginaux, car généralement la coordination de participes passés n'est pas privilégiée en roumain (voir section \ref{sec:1.3.1.1} du chapitre 1). Cependant, le roumain présente des formes pronominales neutres, comme les affixes de première et respectivement deuxième personne du pluriel \textit{ne} et \textit{vă}, qui ne distinguent pas morphologiquement entre l'accusatif et le datif~et qui pourraient s'appliquer a priori a une coordination de verbes de régimes différents, comme le montre l'exemple attesté en \REF{ex:2:60}a où la forme \textit{vă} (réduite phonologiquement à \textit{v-} devant une voyelle) apparaît simultanément compatible avec les deux verbes conjoints, alors que cela semble impossible avec une forme pronominale distinguant entre le datif et l'accusatif, comme le clitique de troisième personne du pluriel en \REF{ex:2:60}a. 


\begin{enumerate}
\item \label{bkm:Ref300600124}a  Il nous a écrit et appelé maintes fois.


\end{enumerate}
  b  Il nous a écrit maintes fois.

  c  Il nous a appelé maintes fois.


\begin{enumerate}
\item \label{bkm:Ref300600409}a  *Il \{lui {\textbar} l'\} a écrit et appelé maintes fois.


\end{enumerate}
  b  Il lui a écrit maintes fois.

  c  Il l'a appelé maintes fois


\begin{enumerate}
\item \label{bkm:Ref269816990}a  V-ați  bătut  şi  dat  palme  unul  altuia  degeaba.


\end{enumerate}
\textsc{2pl.acc{\textbar}dat-aux}  battu  et  donné  gifles  l'un  l'autre.\textsc{dat}  gratuitement

{\itshape
  Vous vous êtes battus et donné des gifles réciproquement pour rien}

  b  *\{S- {\textbar} şi-\}au  bătut  şi  dat  palme  unul  altuia  degeaba.

\textsc{\{3pl.acc{\textbar}3pl.dat-aux}  battu  et  donné  gifles  l'un  l'autre.\textsc{dat}  gratuitement

{\itshape
Ils se sont battus et ils se sont donné des gifles réciproquement pour rien}

\citet{Mouret2007} ajoute un deuxième cas de figure où cette fois-ci deux formes verbales sont neutralisées : une coordination de deux verbes imposant des modes différents (indicatif vs. subjonctif) à la subordonnée complétive qu'ils sous-catégorisent est possible s'il existe une forme verbale qui neutralise la distinction entre ces deux modes, comme en \REF{ex:2:61}. On peut ainsi coordonner les verbes \textit{constater} et \textit{regretter} qui sélectionnent un complément phrastique à l'indicatif et respectivement au subjonctif si et seulement si le verbe de la complétive a la même forme aux deux modes (comparer le verbe \textit{constituer} en \REF{ex:2:61}a et le verbe \textit{être} en \REF{ex:2:61}b).


\begin{enumerate}
\item \label{bkm:Ref269816994}a  On ne peut que constater et regretter que cette mesure constitue un échec.


\end{enumerate}
  b  *On ne peut que constater et regretter que cette mesure \{est {\textbar} soit\} un échec.

Pour conclure, toute coordination doit obéir à deux contraintes qui prennent en compte la fonction syntaxique des conjoints par rapport à l'élément qu'ils partagent. D'une part, si les conjoints correspondent à des arguments, le prédicat avec lequel se combine la coordination dans son ensemble doit accepter \textit{alternativement} chacun d'eux. D'autre part, si les conjoints correspondent à des têtes, l'argument qu'ils partagent doit respecter \textit{simultanément} les différentes contraintes imposées par les têtes coordonnées sur l'argument en question. Avec ces deux contraintes, qui découlent directement de la généralisation de Wasow formulée en \REF{ex:2:46}, on peut donc prédire l'acceptabilité de tous les exemples discutés plus haut, mettant en jeu une coordination de termes non-identiques au niveau morpho-syntaxique.

Un autre trait morpho-syntaxique qui amène à des dissimilarités dans une coordination est l'accord. Comme cela concerne plutôt la coordination sous-phrastique (qui n'est pas l'objet de ce chapitre), je me limite juste à signaler l'existence de ce type d'asymétries et à l'exemplifier, sans entrer dans les détails descriptifs et analytiques. Pour une étude approfondie de l'accord dans la coordination en roumain, voir \citet{Croitor2011}.

En roumain, comme dans beaucoup de langues, la coordination est compatible dans certaines configurations avec un accord asymétrique, c.-à-d. une stratégie d'accord qui ne tient pas compte des critères sémantiques, mais de l'ordre linéaire. Le verbe s'accorde ainsi uniquement avec un des conjoints et, dans la plupart des cas, c'est le conjoint le plus proche (angl. \textit{single conjunct agreement} ou \textit{closest conjunct agreement}, cf. \citet{Corbett1991}, Moosally (1998, 1999), \citet{Sadler2003} ; \textit{each-conjunct agreement}, cf. \citet{Yatabe2004}).\footnote{\citet{Peterson2004a} suggère que ce type d'accord n'est pas un phénomène grammatical, mais uniquement une stratégie utilisée par les locuteurs pour combler les trous de la grammaire.} 

Comme en polonais ou tchèque (Bril \& \citet{Rebuschi2006}, \citet{Borsley2009}), en roumain l'accord est sensible à la position du sujet coordonné par rapport au verbe (\textit{GA} (1963), \citet{Avram2001}, \textit{GALR} (2005)). Si le sujet coordonné suit le verbe, le verbe s'accorde soit avec l'ensemble, soit avec le premier terme coordonné\footnote{Dans la tradition grammaticale roumaine, on appelle ce type d'accord \textit{accord par attraction}.} , cf. \REF{ex:2:62}a. En revanche, si le sujet coordonné précède le verbe, l'accord du verbe se fait obligatoirement avec le syntagme coordonné dans son ensemble, cf. \REF{ex:2:62}b. Les deux autres critères qui gèrent les accords asymétriques en roumain sont décrits dans \citet{Croitor2011}. 


\begin{enumerate}
\item \label{bkm:Ref300735120}a  \{\textbf{Se aude  {\textbar}~se aud}\}  ploaia  şi  vântul.


\end{enumerate}
s'entendre.\textsc{pres.3sg  {\textbar} 3pl}  la-pluie  et  le-vent

{\itshape
On entend la pluie et le vent}

  b  Ploaia  şi  vântul  \{\textbf{fac}  {\textbar}~*\textbf{face}\}  ravagii  în  estul  României.

    la-pluie  et  le-vent  faire.\textsc{pres.3pl  {\textbar} 3sg}  ravages  dans  l'est  Roumanie.\textsc{gen}

    \textit{La pluie et le vent font des ravages dans l'est de la Roumanie}

Toujours dans le domaine de la coordination sous-phrastique, on peut noter les situations de ce qu'on appelle la coordination {\guillemotleft}~commitative~{\guillemotright} (\citet{Haspelmath2007}), où les termes coordonnés reçoivent des marques casuelles différentes. Une coordination de syntagmes nominaux en position sujet se caractérise normalement par une identité casuelle entre les deux conjoints (c.-à-d. le nominatif), mais certains syntagmes nominaux [+ humain] permettent une dissimilarité casuelle : le premier conjoint est toujours au nominatif, alors que le deuxième conjoint est introduit par la préposition \textit{cu} `avec' exigeant l'accusatif (comparer \REF{ex:2:63}a-b : on a la forme pronominale de première personne singulier au nominatif \textit{eu} `je' en \REF{ex:2:63}a et sa contrepartie à l'accusatif \textit{mine} en \REF{ex:2:63}b). Un argument qui relie ces constructions aux coordinations ordinaires concerne les phénomènes d'accord. Le verbe dans une coordination commitative est généralement au pluriel \REF{ex:2:64}a, de plus il reçoit le genre masculin s'il y a au moins un nominal dénotant une entité au masculin \REF{ex:2:64}b, et la personne du verbe dépend des marques de personne qu'on retrouve dans la coordination, c.-à-d. première personne du pluriel si l'un des termes est à la première personne \REF{ex:2:64}a, deuxième personne du pluriel si l'un des termes est à la deuxième personne et si aucun des termes n'est à la première personne \REF{ex:2:64}c, et troisième personne du pluriel dans les autres cas \REF{ex:2:64}b. Pour une analyse syntaxique de la coordination comitative, voir Mc\citet{Nally1993} et Trawi\'nski (2005).  


\begin{enumerate}
\item \label{bkm:Ref272395209}a  \textbf{Eu } şi  soția  mea  lucrăm  la  bucătărie,  iar  Ion  în  sală.


\end{enumerate}
\textsc{  1sg.nom}\textsc{ } et  femme  \textsc{poss } travaillons  en  cuisine,  et  Ion  en  salle

  \textit{Moi et ma femme travaillons en cuisine, et Ion en salle}

b  Soția  mea  şi  \textbf{cu}  \textbf{mine}  lucrăm  la  bucătărie,  iar  Ion  în  sală.

  femme  \textsc{poss}  et  avec  1\textsc{sg.acc } travaillons  en  cuisine,  et  Ion  en  salle

  \textit{Ma femme et moi travaillons en cuisine et Ion en salle}


\begin{enumerate}
\item \label{bkm:Ref272401966}a  Tata  şi  cu  mine  \{\textbf{bem} {\textbar} *bea {\textbar} *beau\}  câte  o  cafea.


\end{enumerate}
  papa.\textsc{nom}  et  avec  1\textsc{sg.acc } boire.\textsc{pres.\{1pl {\textbar} 3sg {\textbar} 1sg}\}  \textsc{distrib}  un  café

  \textit{Mon père et moi buvons chacun un café}

b  Ion  şi  cu  Maria  sunt  proaspăt  \textbf{căsătoriți}  şi  totuşi  se  ceartă.

  Ion  et  avec  Maria  sont  récemment  mariés  et  pourtant  se  bagarrent

  \textit{Ion et Maria sont récemment mariés et pourtant ils se bagarrent}  

c  \{Tu  şi  cu  Andra {\textbar} Andra  şi  cu  tine\}  \textbf{sunteți}  cei  mai  tari.

  \{2\textsc{sg.nom}  et  avec  Andra {\textbar} Andra  et  avec  2\textsc{sg.acc\} } êtes  les  plus  forts

  \textit{Andra et toi, vous êtes les plus forts}

Revenons aux coordinations phrastiques, qui sont au c{\oe}ur de ce chapitre. Au niveau strictement syntaxique, on observe que la coordination de phrases n'opère pas nécessairement avec le même type syntaxique de phrase, qu'il s'agisse d'une coordination de phrases racines ou bien d'une coordination de subordonnées. En ce qui concerne les phrases racines, on peut coordonner des phrases de différents types sans contrainte particulière : type déclaratif et type désidératif \REF{ex:2:65}a, type déclaratif et type interrogatif \REF{ex:2:65}b, type déclaratif et type exclamatif \REF{ex:2:65}c, type désidératif et type interrogatif \REF{ex:2:65}d, type désidératif et type exclamatif \REF{ex:2:65}e, type exclamatif et type interrogatif \REF{ex:2:65}f. On peut même coordonner des phrases averbales avec des phrases verbales de différents types \REF{ex:2:66}a-b-c.


\begin{enumerate}
\item \label{bkm:Ref273567144}a  Te joci pe stradă, dar fii atent la maşini !


\end{enumerate}
{\itshape
Tu joues dans la rue, mais fais attention aux voitures}

  b  In stilul ăsta va ajunge în sapă de lemn şi cine îi va plăti datoriile ?

{\itshape
A ce rythme il deviendra très pauvre et qui va payer ses dettes} 

  c  Dimineața mă simt bine, dar ce rău îmi e seara !

{\itshape
Le matin je me sens bien, mais comme j'ai mal le soir}

  d  Să plece (dacă vrea), dar cine va avea grijă de copil ?

{\itshape
Qu'il parte (s'il veut), mais qui va garder l'enfant}

  e  Să divorțeze (dacă vrea), dar ce-o să mai regrete !

{\itshape
Qu'il divorce (s'il le veut), mais ô combien il va le regretter}

  f  \c{T}in atât de mult la cărțile astea, dar unde să le mai pun ?

{\itshape
J'aime tellement ces bouquins, mais où les mettre encore}


\begin{enumerate}
\item \label{bkm:Ref269903039}a  A avut tot ce şi-a dorit inima, dar [la ce bun] ?


\end{enumerate}
{\itshape
Il a eu tout ce que son c{\oe}ur a désiré, mais à quoi bon}

  b  Grecia a fost salvată, dar [cu ce preț] !

{\itshape
La Grèce a été sauvée, mais avec quel prix} 

  c  [Incă o bere] şi plec.

{\itshape
Encore une bière et je m'en vais}

Quand on coordonne des phrases subordonnées, il y a des phrases complétives qui peuvent ne pas avoir le même type, à condition que chacune soit indépendamment un complément possible du verbe qui les enchâsse. J'illustre cela avec les exemples en \REF{ex:2:67} où l'on coordonne une complétive interrogative et une complétive exclamative comme en \REF{ex:2:67}a, ou bien une déclarative et une désidérative comme en \REF{ex:2:67}b.


\begin{enumerate}
\item \label{bkm:Ref300774271}a  Ion ştie cine a murit şi cât de mult suferise acesta.  


\end{enumerate}
{\itshape
Ion sait qui est mort et combien avait souffert celui-ci}

  b  Maria mi-a spus că te aşteaptă, dar să te grăbeşti.    

{\itshape
Maria m'a dit qu'elle t'attendrait, mais que tu te dépêches} 

On observe donc que le parallélisme syntaxique peut être partiel en ce qui concerne la catégorie, le mode, le marqueur, le type de phrase, etc. Il n'y a pas de contrainte d'identité syntaxique stricte entre les conjoints, à condition que chaque terme coordonné soit compatible avec l'élément mis en facteur (cf. la généralisation de Wasow). 

En revanche, ce parallélisme syntaxique semble être plus strict en ce qui concerne la fonction syntaxique des éléments coordonnés. En règle générale, les membres d'une coordination doivent avoir la même fonction syntaxique. Cette contrainte (à laquelle on ajoutera, dans la section suivante, une contrainte d'identité sémantique) explique l'agrammaticalité des exemples en \REF{ex:2:68}, où l'on coordonne un complément ou un sujet avec un ajout. La seule exception est représentée par les {\guillemotleft}~conjoints incidents~{\guillemotright} (cf. Abeillé (2005)) figurant en \REF{ex:2:69}, qui sont analysés comme des ajouts par rapport à leur hôte syntaxique. Leurs propriétés semblent être très différentes d'une coordination canonique (cf. Abeillé (2005), \citet{Mouret2007}, ce qui nous amène à les exclure du domaine empirique des coordinations (pour une discussion de la notion d'ajout incident, voir la section \ref{sec:2.8}).


\begin{enumerate}
\item \label{bkm:Ref300776667}a  *Ion  mănâncă  repede  şi  un  sandviş.


\end{enumerate}
Ion  mange  vite  et  un  sandwich

{\itshape
Ion mange vite un sandwich}

b  *Maria  locuieşte  în  Bretania  şi  cu  socrii. 

  Maria  habite  en  Bretagne  et  avec  les-beaux-parents

{\itshape
Maria habite en Bretagne avec ses beaux-parents}

c  *Maria  şi  ieri  a  fost  la  plajă. 

  Maria  et  hier  est  allée  à  plage

{\itshape
    Hier Maria est allée à la plage}


\begin{enumerate}
\item \label{bkm:Ref300777234}a  Marea problemă, [\textbf{şi} cu asta vreau să închei], este că nu oferiți nicio soluție viabilă.


\end{enumerate}
{\itshape
Le grand problème, et je conclus avec ça, est que vous n'offrez aucune solution viable}

  b  Medicina nu este gratuită, ea costă, [\textbf{şi} încă foarte mult].

{\itshape
La médecine n'est pas gratuite, elle coûte, et beaucoup même}

  c  Are un corp de milioane... are noroc fata, [\textbf{şi} pare-se că şi voce].

{\itshape
Elle a un corps très joli...elle a de la chance la fille, et semble-t-il une bonne voix aussi}

\subsubsection{Degré d'identité au niveau sémantico-discursif}
\label{bkm:Ref301431030}{\bfseries
Contrainte sémantique générale}

Si l'on veut garder la notion d'identité pour décrire la relation entre les conjoints, elle opère plutôt au niveau sémantique. Les termes coordonnés doivent avoir le même type sémantique (Partee \& \citet{Rooth1983}, Sag \textit{et al.} (1985), Gazdar \textit{et al.} (1985), \citet{Munn1993}, \citet{Johannessen1998}, \citet{Haspelmath2007}). En particulier, \citet{Haspelmath2007} considère que l'identité sémantique est le critère le plus important pour identifier une coordination. Les exemples en \REF{ex:2:70}a montrent qu'on ne peut pas coordonner un complément qui dénote un individu et un complément dénotant une activité, bien que les deux respectent la généralisation de Wasow discutée dans la section précédente. En revanche, si l'on coordonne deux compléments dénotant une activité, la coordination devient parfaitement acceptable \REF{ex:2:70}b. 


\begin{enumerate}
\item \label{bkm:Ref269838245}a  \#Ador actorii americani şi să fac cumpărături.


\end{enumerate}
{\itshape
J'adore les acteurs américains et faire les boutiques}

b  Ador sportul şi să fac cumpărături. 

    \textit{J'adore le sport et faire les boutiques}

Les terms conjoints doivent avoir en commun une archipropriété qui justifie leur association. Cette contrainte correspond à la notion d'intégrateur commun (proposée par \citet{Lang1984}), qui regroupe sémantiquement tous les éléments coordonnés par ce qu'ils ont en commun et aussi par ce qu'ils ont en contraste l'un par rapport à l'autre (je reviendrai sur ce point dans la section \ref{sec:2.9.2} de ce chapitre et dans le chapitre 4, section \ref{sec:4.3.4.2}). Une violation de cette contrainte de parallélisme sémantique a lieu dans ce que la rhétorique appelle \textit{zeugme} ou \textit{syllepse}\footnote{Ces exemples peuvent apparaître dans certains contextes (surtout dans les écrits littéraires) avec des effets stylistiques.}~(voir discussion dans \citet{Zafiu2001}), où les termes conjoints ont des traits sémantiques différents [abstrait] vs. [concret] comme en \REF{ex:2:71}a-b, ou bien le statut lexical des termes n'est pas le même, comme en \REF{ex:2:71}c (ici, le premier terme coordonné \textit{rămas bun} fait partie d'une expression figée \textit{a-şi lua rămas bun} `faire ses adieux', alors que le deuxième terme \textit{pălăria} est un mot autonome du point de vue lexical ; par conséquent, ils ne partagent pas le même foncteur, c.-à-d. le verbe \textit{a lua} `prendre' a deux entrées lexicales distinctes). 


\begin{enumerate}
\item \label{bkm:Ref272834660}a  \#Și-a pierdut optimismul şi un portofel de piele.


\end{enumerate}
{\itshape
Il a perdu son optimisme et son portefeuille en cuir}

b  \#Angajez vânzător cu experiență şi maşină. 

{\itshape
Je cherche vendeur avec expérience et voiture}

c  \#Și-a  luat  rămas bun  şi  pălăria. 

    \textsc{refl.3}-a  pris  au revoir  et  le-chapeau

{\itshape
    Il a dit au revoir et pris son chapeau}

On a vu dans la section précédente que les termes coordonnés devaient avoir la même fonction syntaxique. Cela est en fait une conséquence de la contrainte d'identité sémantique. Les éléments ayant des fonctions syntaxiques différentes ont des rôles sémantiques différents. Or, la contrainte d'identité sémantique requiert que les termes coordonnés aient les mêmes rôles sémantiques. L'incompatibilité des rôles sémantiques explique ainsi l'inacceptabilité des exemples en \REF{ex:2:72} : 


\begin{enumerate}
\item \label{bkm:Ref272836229}a  \#Ion şi (cu) ciocanul au spart zidul.      (agent \& instrument)


\end{enumerate}
{\itshape
Ion et le marteau ont cassé le mur} 

  b  \#Ion mănâncă repede şi un sandviş.      (manière \& thème)

{\itshape
Ion mange vite et un sandwich} 

  c  \#Ion mănâncă cu mama lui şi cu foarte multă poftă.  (commitatif \& manière)

{\itshape
Ion mange avec sa mère et avec un très grand appétit}

  d  \#Ion merge la film şi la ora 5.         (direction \& temps)

{\itshape
Ion va au cinéma et à 5 heures}

  e  \#Ion a plecat încet şi acolo.          (manière \& direction)

{\itshape
Ion est parti lentement et là-bas} 

Les seules exceptions à cette règle sont les coordinations des intérrogatifs multiples\footnote{Voir Bîlbîie \& \citet{Gazdik2011} pour une analyse détaillée des coordinations des interrogatifs multiples en roumain. Il faut noter que les interrogatifs coordonnés partagent la même fonction syntaxique, c.-à-d. la fonction extrait en HPSG.}  en \REF{ex:2:73} et la coordination de mots négatifs en \REF{ex:2:74}. 


\begin{enumerate}
\item \label{bkm:Ref268618145}a  \textbf{Cine}  şi  \textbf{cu  ce}  a  spart  zidul ?        (agent \& instrument)


\end{enumerate}
qui  et  avec  quoi  a  cassé  le-mur

{\itshape
  Qui a cassé le mur, et avec quoi}

  b  \textbf{Ce}  şi  \textbf{cum}  mănâncă ?        (thème \& manière)

    quoi  et  comment  mange

{\itshape
  Qu'est-ce qu'il mange, et comment} 

c  Vine  \textbf{cine } şi  \textbf{cum } poate.        (agent \& manière)

  vient  qui  et  comme  peut

{\itshape
Viendra qui peut et comme il peut}

  d  Fiecare  face  \textbf{ce}  şi  \textbf{cum}  vrea.      (thème \& manière)

    chacun  fait  quoi  et  comme  veut

{\itshape
Chacun fait ce qu'il veut et comme il veut}


\begin{enumerate}
\item \label{bkm:Ref300779730}a  \textbf{Nimeni} şi \textbf{nimic} nu mă va despărți de tine.


\end{enumerate}
{\itshape
Personne et rien ne me séparera de toi}

b  \emph{\textbf{\textup{Nicăieri}}}\emph{\textup{ şi} }\emph{\textbf{\textup{niciodată}}} nu s-a ieşit din vreo criză prin măsuri de austeritate. 

{\itshape
Nulle part et jamais on n'est pas sorti d'une crise par des mesures d'austérité} 

c  \emph{\textbf{\textup{Nimeni}}}\emph{\textup{ şi} }\emph{\textbf{\textup{niciodată}}} nu poate ocupa locul unei mame. 

  \textit{Personne et jamais ne peut occuper la place d'une mère}

{\bfseries
Coordination de phrases avec des types de contenu distincts}

Dans le chapitre 1, section \ref{sec:1.2.2}, j'ai précisé qu'à chaque type syntaxique de phrase correspondait de façon biunivoque un type de contenu stable. Dans la section \ref{sec:2.4.1}, on a vu qu'on pouvait coordonner des phrases ayant des types syntaxiques différents. Par conséquent, les phrases coordonnées n'ont pas nécessairement le même type de contenu. Avant d'illustrer cela, je veux reprendre la hiérarchie des types de contenu, telle qu'elle a été établie dans le chapitre 1, section \ref{sec:1.4.2}.


\begin{enumerate}
\item   \label{bkm:Ref301425933}Hiérarchie des types de contenu


\end{enumerate}
{   [Warning: Image ignored] % Unhandled or unsupported graphics:
%\includegraphics[width=4.2535in,height=1.1035in,width=\textwidth]{fe443409cd384d3fb0f6390ffd77f513-img22.svm}
} 

Dans les phrases racines, il n'y a pas a priori de contrainte particulière. Je retiens ici l'exemple des \textit{pseudo-impératifs} (cf. \citet{Franke2008}) en \REF{ex:2:76}, où l'on coordonne une phrase désidérative (de type visée) et une phrase déclarative (de type proposition). 


\begin{enumerate}
\item \label{bkm:Ref272839516}a  Inchide uşa \textbf{şi}-ți voi spune secretul.


\end{enumerate}
{\itshape
Ferme la porte et je te dirai le secret}

  b  Continuă să țipi \textbf{şi} ai să ți-o iei.

{\itshape
Continue à crier et tu seras frappé}

  c  Mănâncă deja ce ți-am pus pe masă, \textbf{iar} între timp eu mă duc să cumpăr pâine.

{\itshape
Mange ce que je t'ai déjà mis sur table, et entre temps je vais acheter du pain}

  d  Scoate toți banii pe care îi ai \textbf{sau} te omor.\footnote{La conjonction \textit{sau} `ou' coordonne les contenus sémantiques des phrases et non les actes illocutoires. Il y a donc un seul acte de discours réalisé dans ce cas. Voir \citet{Piscone2009} pour plus de détails.}

    \textit{Sors tout l'argent que tu as ou je te tue} 

Dans les subordonnées, le type de contenu des subordonnées complétives est contraint par la sémantique de la tête qui factorise la construction coordonnée (\citet{Piscone2009}). Certains verbes enchâssent tout sous-type de \textit{message} (c.-à-d. \textit{question} \& \textit{prop-excl}, cf. \REF{ex:2:77}a), alors que d'autres imposent des contraintes particulières. Un prédicat comme \textit{a şti} `savoir' est compatible uniquement avec des contenus de type \textit{propositionnel} (c.-à-d. \textit{prop-excl} \& \textit{prop}, cf. \REF{ex:2:77}b), alors qu'un prédicat comme \textit{a ruga} `demander/prier' enchâsse uniquement des contenus de type \textit{abstractionnel} (c.-à-d. \textit{question} \& \textit{visée}, cf. \REF{ex:2:77}c). 


\begin{enumerate}
\item \label{bkm:Ref269938414}a  Mă gândesc uneori [dacă va mai veni oare Ion vreodată] şi [cât de dor trebuie să-i fie mamei lui de el].


\end{enumerate}
{\itshape
Je réfléchis parfois si Ion reviendra un jour et ô combien il doit manquer à sa mère}

  b  Ion ştie [cât de mult a suferit mama lui] şi [că nu exista niciun remediu].  

{\itshape
Ion sait combien a souffert sa mère et qu'il n'y avait aucun remède}

  c  Ion m-a rugat [dacă-i pot duce un pachet Anei] şi [să nu uit s-o salut din partea lui]. 

\textit{Ion m'a demandé si je pouvais amener un paquet pour Ana et que je n'oublie pas de la saluer de sa part }  

{\bfseries
Coordination de phrases avec des actes illocutoires distincts}

Toujours dans le chapitre 1, j'avais précisé qu'on définissait les actes illocutoires uniquement pour les phrases racines (voir section \ref{sec:1.2.1.2}). Les actes illocutoires majeurs discutés dans le chapitre 1 sont l'assertion, l'exclamation, l'interrogation et l'injonction. On observe donc qu'on peut coordonner des phrases racines avec des actes illocutoires distincts, mais dans certains cas l'appel à l'interlocuteur doit être explicite afin de préserver la grammaticalité de l'énoncé. Comparer dans ce sens les exemples anglais repris de Blakemore \& \citet[588]{Carston2005} en \REF{ex:2:78}.


\begin{enumerate}
\item \label{bkm:Ref300782480}a  I went to the lecture and who \textbf{do you think} I saw ?


\end{enumerate}
  b  *I went to the lecture and who I saw ?

  d  I am doing the dishes and \textbf{don't try} to stop me ! 

  e  *Your mother has already left and go home !

Pour ce qui est du roumain, j'illustre la coordination de phrases avec des actes illocutoires distincts en \REF{ex:2:79}-\REF{ex:2:80}-\REF{ex:2:81}. L'acte illocutoire correspondant à la première phrase coordonnée dans tous ces exemples est une assertion. En revanche, la deuxième phrase coordonnée a comme acte illocutoire une interrogation en \REF{ex:2:79}, une exclamation en \REF{ex:2:80} et une injonction en \REF{ex:2:81}.  


\begin{enumerate}
\item \label{bkm:Ref300783484}a  Am fost aseară la film şi pe cine crezi că am văzut ? 


\end{enumerate}
{\itshape
Je suis allée hier soir au cinéma et qui tu crois que j'ai vu}

  b  J'\textbf{asserte} : \textit{Je suis allée hier soir au cinéma.} et je \textbf{demande} : \textit{Qui tu crois que j'ai vu ?}


\begin{enumerate}
\item \label{bkm:Ref300783487}a  Am trecut aseară pe la Maria şi, vai, în ce hal arăta !


\end{enumerate}
{\itshape
Je suis passé hier soir voir Maria et, mon Dieu, dans quel état elle était}

  b  J'\textbf{asserte} : \textit{Je suis passé hier soir voir Maria.} et j'\textbf{exclame} : \textit{Mon Dieu, dans quel état elle était !}


\begin{enumerate}
\item \label{bkm:Ref300783489}a  Te ajut cu mare plăcere, dar aşteaptă puțin !


\end{enumerate}
{\itshape
Je t'aide volontiers, mais attends un peu}

  b  J'\textbf{asserte} : \textit{Je t'aide volontiers.} et j'\textbf{ordonne} : \textit{Attends un peu !}

{\bfseries
Relations discursives asymétriques}

\citet{Lakoff1971}, \citet{Schmerling1975}, Levin \& \citet{Prince1986} sont parmi les premiers à noter que les structures coordonnées (au moins celles coordonnées par la conjonction \textit{and} en anglais) se prêtent à deux types d'interprétations, en fonction du type de relation discursive entretenue par les conjoints. 

Selon \citet{Asher1993} et Asher \& \citet{Lascarides2003}, une relation discursive est définie par toute paire d'unités discursives qui se succèdent immédiatement. Ces unités peuvent être  organisées hiérarchiquement ou non, ce qui fait la distinction entre deux types de relations discursives : relations asymétriques vs. relations symétriques\footnote{Asher \& \citet{Lascarides2003} utilisent les notions de {\guillemotleft}~subordinating~{\guillemotright} vs. {\guillemotleft}~coordinating~{\guillemotright} relations pour désigner les deux types de relations discursives. Vu l'ambiguïté des termes, je garde les termes de coordination et subordination uniquement pour les relations syntaxiques.}. Dans une relation asymétrique, une des deux unités du discours change la {\guillemotleft}~granularité~{\guillemotright} de la description dans le texte, en apportant plus d'information sur un élément présent dans l'autre unité du discours. Dans cette perspective, une unité du discours domine l'autre (Asher \& \citet{Lascarides2003}, \citet{Asher2004}). Dans une relation symétrique, l'unité discursive est liée au contexte précédent de telle façon qu'elle continue la description sans changer la {\guillemotleft}~granularité~{\guillemotright}.\footnote{J'emploie les termes tels qu'ils ont été définis en SDRT (\textit{Segmented Discourse Representation Theory --} SDRT) par Asher (1993, 2004) et Asher \& \citet{Lascarides2003}. Cependant, les relations discursives asymétriques et symétriques sont définies également dans une autre approche discursive (\textit{Rhetorical Structure Theory} -- RST), par Mann \& \citet{Thompson1988}. Même si les termes utilisés sont différents (relations discursives de type noyau-satellite ou multinucléaires), la dichotomie est la même. Ce qui distingue toutefois les deux approches est le critère sous-jacent qui permet l'assignation d'une relation discursive : en SDRT, une relation discursive est assignée en fonction de la manière dont le discours est construit, plus précisement en fonction des relations temporelles entre les éventualités. En RST, l'assignation d'une relation discursive dépend principalement des intentions de l'écrivain, des effets communicatifs qu'il vise quand il énonce une unité discursive. }   

Généralement, le test utilisé pour distinguer les deux types de relations discursives dans la coordination est lié à l'influence de l'ordre dans lequel sont présentés les conjoints. Une coordination qui entretient une relation discursive symétrique permet le changement dans l'ordre des conjoints sans conséquences sur la distribution ou l'interprétation de l'énoncé. Les deux conjoints sont ainsi considérés comme mutuellement indépendants. Ainsi, en roumain, dans une coordination avec \textit{iar} `et' comme en \REF{ex:2:82}, les deux phrases coordonnées sont interchangeables, sans qu'il y ait de différence sémantique significative. 


\begin{enumerate}
\item \label{bkm:Ref300784197}a  Maria curăță cartofii, iar Ion toacă ceapa.


\end{enumerate}
{\itshape
Maria épluche les pommes de terre et Ion hache l'oignon}

=  b  Ion toacă ceapa, iar Maria curăță cartofii. 

{\itshape
Ion hache l'oignon et Maria épluche les pommes de terre}

En revanche, si le changement dans l'ordre des conjoints produit soit un énoncé inacceptable, soit un énoncé avec une interprétation différente, la relation qui s'établit entre les conjoints en question est discursivement asymétrique. Les relations discursives asymétriques dans la coordination ont été beaucoup étudiées. Par la suite, je présente brièvement les différents termes utilisés dans la littérature pour désigner ces coordinations asymétriques, en illustrant ces phénomènes avec des exemples du roumain. Pour \citet{Johannessen1998}, toute coordination dans laquelle changer l'ordre des conjoints rend la phrase inacceptable est une occurrence de ce qu'il appelle \textit{unbalanced coordination}. Rentrent dans cette catégorie les coordinations de phrases de types syntaxiques (et sémantiques) différents, comme c'est le cas des {\guillemotleft}~pseudo-impératifs~{\guillemotright} discutés plus haut et exemplifiés en \REF{ex:2:83}.  


\begin{enumerate}
\item \label{bkm:Ref300784757}a  [X Conj Y] vs. \#[Y Conj X]  


\end{enumerate}
  b  Continuă să țipi şi ai să ți-o iei ! 

{\itshape
Continue à crier et tu seras frappé}

  c  \#Ai să ți-o iei~şi continuă să țipi !

  \textit{Tu seras frappé et continue à crier}

Un autre cas de figure mettant en jeu une relation discursive asymétrique est ce que \citet{Johannessen1998} appelle\textit{ pseudocoordination} (ou \textit{subordinate conjoining}, dans \citet{Lang1984}). Cela concerne toute coordination de phrases (ou syntagmes verbaux dans d'autres langues) dans laquelle le premier conjoint contient un verbe indiquant le lieu ou la direction qui perd en quelque sorte son sens de base et acquiert une interprétation aspectuelle. Selon \citet[48]{Johannessen1998}, le phénomène est assez fréquent dans les langues scandinaves. Dans ces cas, la conjonction peut être remplacée par un complémenteur. En roumain, ce genre de coordination est illustré en \REF{ex:2:84}, où l'on coordonne des phrases désidératives ayant des verbes à l'impératif. Le premier verbe à l'impératif (souligné dans chaque exemple ci-dessous) est un verbe de mouvement et acquiert effectivement une interprétation aspectuelle. On obtient le même type de relation discursive si l'on remplace le deuxième verbe à l'impératif par un verbe au subjonctif (comparer les exemples en \REF{ex:2:84} et \REF{ex:2:85}), la relation syntaxique qui s'établit entre les conjoints étant cette fois-ci une relation de subordination. 


\begin{enumerate}
\item \label{bkm:Ref300785138}a  Aşează-te \textbf{şi} mănâncă ceva !


\end{enumerate}
{\itshape
Assieds-toi et mange quelque chose}

  b  Ionel, vino \textbf{şi} dă-mi o mână de ajutor !

{\itshape
Ionel, viens et donne-moi un coup de main}

  c  Du-te \textbf{şi} vezi ce fac copiii în grădină !

{\itshape
Va-t'en et vérifie ce que font les enfants dans le jardin}

  d  Incearcă \textbf{şi} vezi cum îți vine rochia !

{\itshape
Essaie et vois comment te va la robe}


\begin{enumerate}
\item \label{bkm:Ref300785895}a  Aşează-te \textbf{să} mănânci ceva !


\end{enumerate}
{\itshape
Assieds-toi pour manger quelque chose}

  b  Ionel, vino \textbf{să}-mi dai o mână de ajutor !

{\itshape
Ionel, viens pour me donner un coup de main}

  c  Du-te \textbf{să} vezi ce fac copiii în grădină !

{\itshape
Va-t'en pour vérifier ce que font les enfants dans le jardin}

  d  Incearcă \textbf{să} vezi cum îți vine rochia !

{\itshape
Essaie pour voir comment te va la robe}

A part la conjonction \textit{şi} `et', le roumain dispose d'un autre connecteur pour marquer ce type de relation asymétrique. Il s'agit du marqueur \textit{de} dont la catégorie syntaxique est sujette à discussion\footnote{Les analyses proposées sont : conjonction cf. Guțu \citet{Romalo1973}, complémenteur cf. \citet{Avram2001} ou catégorie hybride cf. \citet{Niculescu1965}. Pană \citet[202]{Dindelegan2003} analyse \textit{de} comme un non-subordonnant, car il peut être utilisé avec un impératif, ce qui est spécifique, selon elle, aux phrases racines et non aux subordonnées.} , qui apparaît en distribution libre avec la conjonction \textit{şi} dans les constructions exprimant le but : ainsi, en \REF{ex:2:86} \textit{de} peut introduire une phrase à l'impératif, tout comme la conjonction \textit{şi} en \REF{ex:2:84}.\textit{~}


\begin{enumerate}
\item \label{bkm:Ref300786403}a  Mergi \textbf{de}-mi adu dosarul !


\end{enumerate}
{\itshape
Va-t'en et apporte-moi le dossier}

  b  Du-te \textbf{de} te culcă !

    \textit{Va-t'en et couche-toi}

Un dernier phénomène que je veux mentionner ici est ce que \citet{Levinson2000} appelle \textit{conjunction buttressing}. Cela correspond à une coordination de deux événements qui se succèdent dans le temps et qui sont liés par des relations causales ; c'est pour cela que l'interprétation change si on change l'ordre des conjoints (Blakemore \& \citet{Carston2005}). Cette fois-ci, changer l'ordre des conjoints n'affecte pas l'acceptabilité de l'énoncé, mais simplement l'interprétation, comme on le voit dans les exemples en \REF{ex:2:87}-\REF{ex:2:88}-\REF{ex:2:89}.  


\begin{enumerate}
\item \label{bkm:Ref300786844}a  Ion pune cheia în contact şi maşina porneşte.


\end{enumerate}
{\itshape
Ion met le contact et la voiture démarre}

${\neq}$  b  Maşina porneşte şi Ion pune cheia în contact.

{\itshape
La voiture démarre et Ion met le contact} 


\begin{enumerate}
\item \label{bkm:Ref300786848}a  Işi ia geanta şi pleacă enervat din clasă.


\end{enumerate}
{\itshape
Il prend sa serviette et, énervé, quitte la classe } 

${\neq}$  b  Pleacă enervat din clasă şi îşi ia geanta.  

{\itshape
Enervé, il quitte la classe et prend sa serviette}


\begin{enumerate}
\item \label{bkm:Ref300786858}a  N-am fost atent şi am spart vaza preferată a mamei.


\end{enumerate}
{\itshape
Ion n'a pas fait attention et a cassé la vase favorite de sa mère}

${\neq}$  b  Am spart vaza preferată a mamei şi n-am fost atent.

{\itshape
Ion a cassé la vase favorite de sa mère et n'a pas fait attention}

Maintenant qu'on a vu que les relations discursives asymétriques sont compatibles avec une relation syntaxique de coordination, il nous reste à dresser la liste de ces relations discursives. Je me contente de mentionner ici uniquement quelques relations discursives asymétriques. Pour des typologies exhaustives, consulter les typologies de Matthiessen \& \citet{Thompson1988}, \citet{Kehler2002}, ou bien \citet{Asher1993}, Asher \& \citet{Lascarides2003}, Asher \& \citet{Vieu2005}. Une discussion concernant les relations discursives symétriques apparaît dans le chapitre 4, section \ref{sec:4.3.4.3}. 

Il est généralement accepté que les relations de parallélisme et de contraste, par exemple, relèvent de la classe des relations symétriques, alors qu'une relation de type contiguité (c.-à-d. succession temporelle), cause-effet, condition ou concession est discursivement asymétrique. Comme mentionné plus haut en 2.3.1, les relations asymétriques sont parfois explicitées à l'aide des adverbiaux qui accompagnent les conjonctions (p.ex. \textit{apoi} `ensuite' en \REF{ex:2:90}a, \textit{astfel} `ainsi' en \REF{ex:2:90}b, \textit{prin urmare} `par conséquent' en \REF{ex:2:90}c).  


\begin{enumerate}
\item \label{bkm:Ref300787925}a  Dimineața iau micul dejun \textbf{şi} apoi plec la lucru.


\end{enumerate}
{\itshape
Le matin je prends le petit-déjeuner et ensuite je vais au travail}

  b  Am urmat cursuri de bucătărie \textbf{şi} astfel am învățat să gătesc. 

{\itshape
J'ai suivi des cours de cuisine et ainsi j'ai appris à cuisiner}

  c  Ieri a plouat toată ziua \textbf{şi} prin urmare n-am ieşit din casă.

{\itshape
Hier il a plu toute la journée et par conséquent je ne suis pas sorti de chez moi}

Revenant aux conjonctions du roumain que j'ai données dans le tableau en \REF{ex:2:23}, on peut dire que certaines conjonctions sont spécialisées pour les relations discursives symétriques (p.ex. \textit{iar} `et', cf. section \ref{sec:2.9.1} et chapitre 4, section \ref{sec:4.3.4.3}), d'autres apparaissent essentiellement dans des relations asymétriques (p.ex. \textit{dar} `mais' qui, dans son usage argumentatif, demande un conjoint droit argumentativement plus fort que celui de gauche, donc, par définition, cette conjonction entretient une relation asymétrique), alors que d'autres n'ont aucune contrainte quant au type de relation discursive mise en jeu (p.ex. \textit{şi} `et'). 

Par la suite, j'illustre la variété de relations asymétriques dans les coordinations avec la conjonction \textit{şi} `et'. Cette conjonction peut lier des phrases qui entretiennent une relation discursive de succession temporelle (c.-à-d. X \textit{et ensuite} Y), comme en \REF{ex:2:91}a. La coordination avec la conjonction \textit{şi} peut avoir aussi une interprétation conditionnelle. On en trouve au moins trois sous-types (cf. \citet{Franke2008} pour l'anglais) : (i) le premier conjoint est une phrase impérative \REF{ex:2:91}b ou déclarative \REF{ex:2:91}c et la coordination dans son ensemble est interpretée comme menace ou promesse ; (ii) le premier conjoint a comme constituant immédiat un syntagme nominal contenant un item du type \textit{one more}\footnote{Ce sous-type de conditionnelle dans la coordination a été étudié en anglais par Culicover \& Jackendoff (1997, 2005).} \REF{ex:2:91}d, alors que le deuxième conjoint est une phrase déclarative ; (iii) on coordonne une phrase de type \textit{sufficiency-modal construction} (p.ex. \textit{only have to {\textbar} it's sufficient to} en anglais) et une phrase déclarative \REF{ex:2:91}e-f. 


\begin{enumerate}
\item \label{bkm:Ref300861185}a  Ajung acasă \textbf{şi} vă răspund imediat.


\end{enumerate}
{\itshape
Je rentre chez moi et je vous réponds tout de suite}

  b  Continuă să țipi \textbf{şi} ai să ți-o iei !

{\itshape
Continue à crier et tu seras frappé}

  c  Votezi cu cine-ți zic eu \textbf{şi} îți dau 100 de euro.

{\itshape
Tu votes pour qui je te dis et je te donnerai 100 euros}

  d  \{Doar {\textbar} Incă\} un pas \textbf{şi} eram mort.

{\itshape
Un pas de plus et j'étais mort}

  e  Trebuie doar să suni \textbf{şi} răspunsul vine imediat. 

{\itshape
Tu dois juste appeler et la réponse vient tout de suite}

  f  E suficient să strigi o dată \textbf{şi} uşa se deschide.

    \textit{Il suffit d'appeler une seule fois et la porte s'ouvre}

La conjonction \textit{şi} `et' peut apparaître aussi avec des relations asymétriques exprimant la concession \REF{ex:2:92}a, la cause ou la conséquence \REF{ex:2:92}b-c, ou encore le but \REF{ex:2:92}d.~


\begin{enumerate}
\item \label{bkm:Ref300864102}a  Mănânc extrem de mult \textbf{şi} tot nu reuşesc să mă îngraş.


\end{enumerate}
{\itshape
Je mange énormément et cependant je ne réussis pas à grossir}

  b  M-am certat cu părinții \textbf{şi} n-am putut să dorm toată noaptea.

{\itshape
Je me suis fâchée contre mes parents et je n'ai pas pu dormir toute la nuit}

  c  Am luat o pastilă \textbf{şi} m-am simțit mult mai bine.

    \textit{J'ai pris une tablette et je me suis sentie beaucoup mieux}

  d  Du-te la brutărie \textbf{şi} cumpără două pâini !

    \textit{Va à la boulangerie et achète deux pains}

Ce qui est intéressant à noter est que la conjonction \textit{şi} `et' peut lier plus de deux conjoints. Cependant, dans tous les exemples qu'on a vus en \REF{ex:2:91} et \REF{ex:2:92}, le nombre des conjoints était limité à deux. Ainsi, on peut faire l'hypothèse qu'une coordination discursivement symétrique n'a pas de limite quant au nombre de ses constituants, alors qu'une coordination discursivement asymétrique est généralement binaire. Cela reste à être vérifié sur plus d'exemples.

Certaines approches ont établi une corrélation entre le type de relation syntaxique (c.-à-d. coordination \textit{vs}. subordination) et le type de relation discursive (c.-à-d. symétrique \textit{vs}. asymétrique). Par conséquent, toute coordination serait par défaut discursivement symétrique, tandis que la subordination serait asymétrique. \citet{Txurruka2000} essaie ainsi de démontrer que toute coordination avec la conjonction \textit{and} en anglais entretient une relation discursive symétrique. Inversement, \citet{Johannessen1998} considère que la {\guillemotleft}~pseudocoordination~{\guillemotright} définie par des relations discursives asymétriques, telle qu'elle est utilisée dans les langues scandinaves, relève plutôt de la subordination (cf. possibilité d'extraction, l'ordre des conjoints ne peut être changé sans changer l'interprétation, etc.).  

Je considère qu'aucune corrélation ne doit être faite a priori entre une certaine construction syntaxique et un certain type de relation discursive. La notion de coordination n'implique pas nécessairement que les relations discursives soient symétriques. Ainsi, j'ai montré plus haut qu'il y a des conjonctions qui, bien qu'elles établissent par défaut une relation symétrique, sont compatibles avec toute sorte de relations asymétriques (p.ex. \textit{şi} `et' en roumain). De même, comme Culicover \& \citet{Jackendoff2005} le notent, il n'y a pas de corrélation entre une relation de discours asymétrique et la subordination. Une même relation entre contenus de phrases peut apparaître à la fois dans une relation de coordination et de subordination. Ainsi, on peut avoir une relation asymétrique exprimant la cause dans une subordination marquée par le complémenteur \textit{pentru că} `parce que' \REF{ex:2:93}a ou bien dans une coordination avec la conjonction \textit{şi} `et' \REF{ex:2:93}b. De même, une relation concessive apparaît dans une subordination avec le complémenteur \textit{deşi} `bien que' \REF{ex:2:94}a, mais aussi dans une coordination avec la conjonction adversative \textit{dar} `mais' \REF{ex:2:94}b. Un dernier exemple concerne la relation asymétrique à interprétation conditionnelle, qui dans une subordination est rendue explicite par le complémenteur \textit{dacă} `si' \REF{ex:2:95}a, et dans une coordination est compatible avec la conjonction \textit{şi} `et' \REF{ex:2:95}b.


\begin{enumerate}
\item \label{bkm:Ref300867086}a  \textbf{Pentru că} a stat mulți ani în Franța, Ion vorbeşte foarte bine franceza.


\end{enumerate}
{\itshape
Puisqu'il a vécu pendant des années en France, Ion parle très bien français}

  b  Ion a stat mulți ani în Franța \textbf{şi}, prin urmare, vorbeşte foarte bine franceza. 

{\itshape
Ion a vécu pendant des années en France et, par conséquent, il parle très bien français}


\begin{enumerate}
\item \label{bkm:Ref300867524}a  \textbf{Deşi} a stat în Franța multă vreme, Ion nu vorbeşte bine franceza.


\end{enumerate}
{\itshape
Bien qu'il ait vécu longtemps en France, Ion ne parle pas bien français}

  b  Ion a stat multă vreme în Franța, \textbf{dar} nu vorbeşte bine franceza. 

{\itshape
Ion a vécu longtemps en France, mais il ne parle pas bien français}


\begin{enumerate}
\item \label{bkm:Ref300867825}a  \textbf{Dacă} aş fi făcut un singur pas în plus, aş fi fost călcat de maşină.


\end{enumerate}
{\itshape
Si je faisais un pas de plus, j'aurais été écrasé par la voiture}

  b  Incă un pas (să mai fi făcut) \textbf{şi} aş fi fost călcat de maşină.

{\itshape
Un pas de plus et j'aurais été écrasé par la voiture}

On observe ainsi qu'une même relation asymétrique peut caractériser des phrases liées par un complémenteur dans une construction subordonnée comme en \REF{ex:2:96}a, ou bien par une conjonction dans une construction coordonnée comme en \REF{ex:2:96}b, et elle peut être aussi rendue explicite par la présence d'un connecteur adverbial (surtout s'il s'agit d'une juxtaposition, comme en \REF{ex:2:96}c).


\begin{enumerate}
\item \label{bkm:Ref273392060}a  \textbf{Dacă} vrei bani, munceşte ! 


\end{enumerate}
{\itshape
Si tu veux de l'argent, travaille}

  b  Iți dau bani, \textbf{dar} munceşte !

{\itshape
Je te donne de l'argent, mais travaille}

 c  Vrei bani, \textbf{atunci} munceşte !  

    \textit{Tu veux de l'argent, alors travaille}

Par la suite, je veux présenter le lien qui peut être établi entre les relations discursives et un phénomène syntaxique tellement discuté dans la littérature et attribué aux structures coordonnées, à savoir la Contrainte sur les Structures Coordonnées (\citet{Ross1967}) que j'ai donnée en \REF{ex:2:35} dans la section \ref{sec:2.3.2} de ce chapitre. De manière générale, cette contrainte interdit toute extraction de constituant hors d'un conjoint si elle n'opère pas simultanément hors des autres. Cette contrainte est donc considérée comme un test essentiel pour distinguer la coordination des autres constructions.  

Cependant, comme l'ont noté entre autres \citet{Ross1967}, \citet{Schmerling1972}, \citet{Goldsmith1985}, \citet{Lakoff1986}, \citet{Kehler2002}, il y a des exceptions à la contrainte d'extraction parallèle. Selon ces auteurs, il existe des coordinations dans lesquelles on observe une extraction asymétrique uniquement d'un des conjoints. J'ai repris en \REF{ex:2:97} quelques exemples en anglais, mais on observe (marginalement) les mêmes violations en roumain.\footnote{Pour le français, l'acceptabilité semble être dégradée (cf. \citet{Mouret2007} et Mouret \& \citet{Desmets2008} :
(i)   \%Je me demande combien de bières on peut [boire \_] et [quand même rester sobre].
(ii)  *C'est le genre de régime que Marie aimerait bien [faire \_] et [enfin rentrer dans sa robe]. 
(iii)  *Voici le whisky que Paul [est allé au supermarché] et [a acheté \_].  } 


\begin{enumerate}
\item \label{bkm:Ref272881233}a  How much can you [drink \_] and [not end up with a hangover the next morning] ?


\end{enumerate}
{\raggedleft
 (\citet[135]{Goldsmith1985})
}

  b  That's the stuff that the guys in the Caucasus [drink \_] and [live to be a hundred].

{\raggedleft
 (\citet[156]{Lakoff1986}) 
}

  c  Here's the whiskey I [went to the store] and [bought \_].       (\citet[94]{Ross1967})

  d  Sam is not the sort of guy you can [just sit there] and [listen to \_].  (\citet{Lakoff1986})


\begin{enumerate}
\item \label{bkm:Ref272882957}a  Mă întreb câte beri poți [bea \_] şi [totuşi să rămâi întreg la minte].


\end{enumerate}
{\itshape
Je me demande combien de bières on peut boire et quand même rester sobre}

  b  Asta e băutura pe care japonezii [adoră s-o bea \_ în fiecare dimineață] şi [trăiesc bine-mersi până la adânci bătrâneți].

{\itshape
C'est ça la boisson que les japonais aiment boire chaque matin et ils vivent bien-merci jusqu{\textasciigrave}à leur profonde vieillesse } 

  c  ??Uite maşina pentru care [m-am dus în Germania] şi [am luat-o \_].

{\itshape
Voici la voiture pour laquelle je suis allé en Allemagne et je l'ai prise}

L'acceptabilité des exemples en \REF{ex:2:97} a déterminé \citet{Ross1967}, \citet{Goldsmith1985}, Postal, etc. à considérer que les coordinations qui sont compatibles avec une extraction asymétrique sont en fait des structures subordonnées en syntaxe. Dans leur perspective (basée sur un principe d'isomorphisme entre la syntaxe, la sémantique et le discours), la syntaxe permettrait ainsi de dériver l'interprétation sémantique associée à~la relation en question. Cependant, comme \citet{Lakoff1986} et \citet{Kehler2002} le démontrent, il n'y a pas d'arguments empiriques pour aligner les coordinations avec extraction asymétrique sur les structures subordonnées (car, contrairement aux structures subordonnées, les phrases coordonnées peuvent être itérées, elles peuvent avoir des gaps multiples et elles ne sont pas mobiles)\footnote{Voir exemples dans \citet{Chaves2007}.}. 

La violation de la contrainte d'extraction parallèle s'explique mieux si l'on postule une analyse discursive (ou un principe pragmatique, cf. \citet{Lakoff1986}, \citet{Kehler2002}, Kubota \& \citet{Lee2008}). Si l'on regarde le type de relation discursive qui s'établit entre les conjoints qui violent la contrainte d'extraction parallèle en \REF{ex:2:97} et \REF{ex:2:98}, on observe qu'il s'agit toujours d'une relation discursive asymétrique (p.ex. cause-effet avec un effet contraire aux attentes \REF{ex:2:97}a, conséquence \REF{ex:2:97}b, succession temporelle \REF{ex:2:97}c, etc.). Cette contrainte semble être donc dépendante du type de relation sémantico-discursive établie entre les éléments coordonnés : les relations discursives symétriques ne permettent qu'une extraction parallèle (afin de préserver le parallélisme des arguments, cf. \citet{Kehler2002}), tandis que les relations discursives asymétriques sont compatibles avec une extraction asymétrique. Comme la contrainte d'extraction parallèle ne concerne qu'une partie des coordinations (c.-à-d. celles avec des relations symétriques), on ne devrait plus la considérer comme un test suffisant pour identifier une coordination en anglais ou en roumain.  

Le type de relation discursive influence non seulement les possibilités d'extraction, mais aussi d'autres faits syntaxiques, comme par exemple la possibilité d'avoir une ellipse verbale de type gapping ou encore la possibilité d'employer des expressions corrélatives. Levin \& \citet{Prince1986}, \citet{Kehler2002} et \citet{Hendriks2004} observent qu'une coordination non-elliptique \REF{ex:2:99}a peut avoir à la fois une interprétation symétrique et asymétrique. Selon l'interprétation symétrique, les événements dénotés par les deux conjoints sont indépendants l'un par rapport à l'autre (ce qui rend possible la paraphrase par \textit{and independently}), alors qu'avec une interprétation asymétrique les événements entretiennent une relation de type cause-effet (pouvant être paraphrasée par \textit{and consequently}). En revanche, une coordination à gapping \REF{ex:2:99}b est compatible uniquement avec une relation discursive symétrique. Pour plus de détails à ce sujet, voir le chapitre 4, section \ref{sec:4.3.4.3}.  


\begin{enumerate}
\item \label{bkm:Ref272884847}a  Sue become upset and Dan become downright angry.


\end{enumerate}
  b  Sue become upset and Dan downright angry.

De même, les coordinations omnisyndétiques, avec des éléments corrélatifs qui redoublent chaque conjoint, sont incompatibles avec une relation asymétrique entre les conjoints~(\citet{Schmerling1972}, \citet{Mouret2007}), comme l'illustrent les exemples suivants en anglais \REF{ex:2:100}a, français \REF{ex:2:100}b et respectivement roumain \REF{ex:2:100}c.  


\begin{enumerate}
\item \label{bkm:Ref300871886}a  *Would you \textbf{both} be an angel \textbf{and} make me some coffee ?  (\citet[30]{Chaves2007}) 


\end{enumerate}
  b  *\textbf{Et} il suit mes conseils \textbf{et} tout se passera bien.     (\citet[191]{Mouret2007})

  c  *\textbf{Și}  ajung  acasă  (şi)  \textbf{şi}  vă  răspund  imediat.

    \textsc{correl } arrive.1\textsc{sg}  à-maison  (et)  \textsc{correl}  vous  réponds  immédiatement

    \textit{J'arrive à la maison et je vous réponds immédiatement}

Finalement, on observe que l'asymétrie discursive, dans beaucoup de coordinations, est soutenue par une asymétrie morpho-syntaxique : l'emploi des verbes à des modes différents crée souvent une dépendance logique (p.ex. l'interprétation conditionnelle est facilitée par l'emploi du subjonctif vs. indicatif en \REF{ex:2:101}a, ou de l'indicatif vs. impératif en \REF{ex:2:101}b) ; la coordination de phrases ayant des types syntaxiques différents (comme dans les exemples donnés en \REF{ex:2:65} et \REF{ex:2:66} ci-dessus) facilite une interprétation asymétrique dans beaucoup de cas. 


\begin{enumerate}
\item \label{bkm:Ref273392770}a  Să plătească şi vom vedea noi mai pe urmă.


\end{enumerate}
{\itshape
Qu'il paie et on verra ensuite}

  b  Te las să te joci pe stradă, dar fii atent la maşini !

    \textit{Je te laisse jouer dans la rue, mais fais attention aux voitures}

\subsubsection{Synthèse}
\label{bkm:Ref302036956}La coordination n'implique pas nécessairement une identité stricte au niveau morpho-syntaxique et sémantico-discursif. En particulier, il y a un parallélisme syntaxique dans la coordination, mais beaucoup plus relâché que ce qui est généralement postulé. Les asymétries morpho-syntaxiques peuvent être prises en compte par la généralisation de Wasow, selon laquelle tout conjoint est admis dans une construction coordonnée, à condition qu'il puisse apparaître seul en lieu et place de la coordination dans son ensemble.

Au niveau discursif, on ne peut pas établir de corrélation entre la coordination et les relations discursives symétriques, car il y a des structures coordonnées qui mettent en jeu des relations asymétriques. En revanche, il semble y avoir une corrélation entre les relations discursives asymétriques et certaines asymétries syntaxiques dans la coordination, comme l'extraction non-parallèle, l'impossibilité du gapping, l'incompatibilité avec les items corrélatifs, les dissemblances de modes verbaux, les dissemblances de type de phrase dans les conjoints, etc. 

Le parallélisme qui semble le plus strict apparaît au niveau sémantique. Les éléments coordonnés doivent avoir le même type sémantique et partager une archipropriété qui justifie leur association.

En conclusion, les symétries ou les asymétries qui apparaissent dans une construction coordonnée peuvent apparaître à plusieurs niveaux d'analyse linguistique. Il n'y a pas toujours de corrélation entre la coordination et les symétries morpho-syntaxiques ou discursives. Une grammaire de la coordination doit rendre compte de toutes les asymétries observées dans cette section. 

\subsection{Coordination des subordonnées}
\label{bkm:Ref302035143}Conformément au mécanisme d'itération syntaxique spécifique à la coordination, la coordination peut avoir lieu entre des phrases racines \REF{ex:2:102}a ou entre des phrases subordonnées \REF{ex:2:102}b-c. 


\begin{enumerate}
\item \label{bkm:Ref273454770}a  [Ion doarme] \textbf{şi} [Maria citeşte].


\end{enumerate}
{\itshape
Ion dort et Maria lit}

  b  Mama m-a întrebat [\textbf{dacă} Ion doarme] \textbf{şi} [\textbf{dacă} Maria citeşte].

{\itshape
Ma mère m'a demandé si Ion dormait et si Maria lisait}

  c  Cred \textbf{că} Ion doarme \textbf{şi} (\textbf{că}) Maria citeşte.

    \textit{Je crois que Ion dort et (que) Maria lit}

La coordination de phrases subordonnées est contrainte par les propriétés syntaxiques et sémantiques du verbe qui enchâsse les phrases coordonnées. D'une part, tous les verbes ne sélectionnent pas les mêmes complémenteurs et les mêmes modes verbaux dans leurs complétives : par exemple, un verbe comme \textit{a vrea} `vouloir' sélectionne une complétive au subjonctif (avec la marque \textit{să}) comme en \REF{ex:2:103}a, tandis qu'un verbe comme \textit{a crede} `croire' sous-catégorise une complétive en \textit{că} comme en \REF{ex:2:103}b\textit{,} incompatible avec le subjonctif, ce qui montre que les complétives sont sensibles au contexte dans lequel elles apparaissent. D'autre part, on a une contrainte sémantique sur le contenu des complétives en fonction de la classe sémantique à laquelle appartient~le verbe : si le verbe de la phrase racine est un prédicat comme \textit{a vrea} `vouloir' \REF{ex:2:103}a, la complétive doit avoir un contenu de type non-propositionnel / abstractionnel (c.-à-d. question ou visée) ; si, en revanche, le verbe est un prédicat comme \textit{a crede} `croire' \REF{ex:2:103}b, la subordonnée doit avoir un contenu de type propositionnel (c.-à-d. proposition ou proposition-exclamative).\footnote{Voir la hiérarchie des types de contenu figurant en \REF{ex:2:75}.}  Les subordonnées qui ne subissent aucune contrainte concernant le type de contenu sont les complétives dont la tête est un verbe de discours rapporté comme \textit{a spune} `dire' \REF{ex:2:103}c ; ces verbes peuvent enchâsser n'importe quel type de phrase ayant un contenu de type message.


\begin{enumerate}
\item \label{bkm:Ref301346779}a  Mama  vrea  [\{să  fiu  {\textbar} *că  sunt\}  cuminte].


\end{enumerate}
maman  veut  \{\textsc{mrq } être.\textsc{subj.1sg}  {\textbar} que  être.\textsc{ind.1sg\}}  sage

{\itshape
Ma mère veut que je sois sage}

  b  Mama  crede  [\{\textbf{că}  sunt  {\textbar} *să  fiu\}  cuminte].

    maman  croit  \{que  être.\textsc{ind.1sg}  {\textbar} \textsc{mrq } être.\textsc{subj.1sg\}}  sage 

{\itshape
Ma mère croit que je suis sage}

  c  Mama  mi-a  spus  [\{\textbf{că}  sunt  {\textbar} să  fiu\}  cuminte].

    maman  m'a  dit  \{que  être.\textsc{ind.1sg}  {\textbar} \textsc{mrq } être.\textsc{subj.1sg\}}  sage 

{\itshape
Ma mère m'a dit \{que je suis sage {\textbar} que je sois sage\}}

De manière générale, les propriétés canoniques des constructions coordonnées s'appliquent aussi aux coordinations de subordonnées. Si par exemple on coordonne deux subordonnées introduites par des complémenteurs différents \REF{ex:2:104}a ou qui n'ont pas le même type syntaxique (interrogative vs. exclamative en \REF{ex:2:104}b), le prédicat avec lequel se combine la coordination dans son ensemble doit accepter \textit{alternativement} chacune d'elles, conformément à la généralisation de Wasow discutée dans la section \ref{sec:2.4.1}. 


\begin{enumerate}
\item \label{bkm:Ref273464899}a  Știu [\textbf{când} îi e foame bebeluşului] şi [\textbf{dacă} trebuie să-l schimb].


\end{enumerate}
{\itshape
Je sais quand le bébé a faim et si je dois lui changer la couche}

  b  Am văzut [cine e alături de mine] şi [cât de greu e să fii singur].

{\itshape
J'ai vu qui était à mes côtés et combien il était difficile d'être seul}

Cependant, les coordinations de subordonnées peuvent présenter des propriétés particulières liées à la présence ou non d'un complémenteur, à leur ordre des mots et, plus généralement, à leur distribution.

La présence de la coordination dans une construction subordonnée change parfois sa distribution. \citet{Johannessen1998} discute le cas de l'allemand et du néerlandais où la coordination de deux subordonnées entraîne des différences dans leur ordre des mots. Dans ces langues, le verbe se trouve habituellement en position seconde dans une phrase racine et en position finale dans une phrase subordonnée (voir l'exemple néerlandais en \REF{ex:2:105}). En revanche, si deux subordonnées sont coordonnées, elles n'ont pas le même ordre des mots. La première subordonnée introduite par un complémenteur a le verbe en position finale (comme attendu dans une subordonnée), alors que la deuxième subordonnée se comporte comme une phrase racine et a le verbe en deuxième position \REF{ex:2:106}.


\begin{enumerate}
\item \label{bkm:Ref301351743}a  Willem  heeft  de  kat  vermoord.


\end{enumerate}
Willem  a  le  chat  tué

{\itshape
Willem a tué le chat}

  b  Ik  denk  dat  Willem  de  kat  heeft  vermoord.

    je  pense  que  Willem  le  chat  a  tué

    \textit{Je pense que Willem a tué le chat          } (\citet[249]{Aarts2006})


\begin{enumerate}
\item \label{bkm:Ref301351983}Als  [je  te  laat  thuis  komt]  en  [je  hebt  geen  sleutel  bij  je] 


\end{enumerate}
  si  tu  trop  tard  maison  viens  et  tu  as  \textsc{neg}  clé  avec  toi

\textit{  Si tu rentres à la maison trop tard et que tu n'as pas de clé   } (\citet[40]{Johannessen1998})

En français, si l'on coordonne deux subordonnées circonstancielles introduites par le même complémenteur, le deuxième complémenteur peut être remplacé par \textit{que} (cf. \citet{Piot1993}, \textit{GGF en prép}.), cf. \REF{ex:2:107}. Cela n'est possible qu'avec les subordonnées ajouts~(cf. l'exemple \REF{ex:2:108} repris de \citet{Piot1993}).


\begin{enumerate}
\item \label{bkm:Ref301352172}a  Marie est partie \textbf{quand} nous étions là \textbf{et} \textbf{quand} Jacques est arrivé.


\end{enumerate}
=  b  Marie est partie \textbf{quand} nous étions là \textbf{et} \textbf{que} Jacques est arrivé. 


\begin{enumerate}
\item \label{bkm:Ref270594898}a  Marie ne savait \textbf{quand} nous étions là \textbf{et} \textbf{quand} Pierre reviendrait.


\end{enumerate}
  b  *Marie ne savait \textbf{quand} nous étions là \textbf{et} \textbf{que} Pierre reviendrait. 

Un autre aspect important à noter qui est toujours lié à la distribution concerne la possibilité de réitérer le complémenteur dans la deuxième subordonnée. De ce point de vue, le français et le roumain sont différents\footnote{Les exemples du français sont repris de la \textit{GGF en prép}.}. En français, d'une manière générale, le complémenteur doit être répété devant chaque subordonnée, qu'il s'agisse d'un valent \REF{ex:2:109} ou d'un ajout \REF{ex:2:110} :


\begin{enumerate}
\item \label{bkm:Ref301357683}a  Je pense \textbf{qu}'il va faire beau et \textbf{qu}'on pourra sortir.


\end{enumerate}
  b  *Je pense \textbf{qu}'il va faire beau et on pourra sortir. 


\begin{enumerate}
\item \label{bkm:Ref301357699}a  \textbf{Comme} il faisait beau et \{\textbf{comme} {\textbar} \textbf{qu}'\}on avait le temps, on est sortis.


\end{enumerate}
  b  *\textbf{Comme} il faisait beau et on avait le temps, on est sortis. 

Selon \textit{GGF en prép.}, cette contrainte peut être enlevée s'il y a un parallélisme marqué formellement entre les conjoints, soit par la répétition d'une conjonction ou item corrélatif (p.ex. \textit{soit...soit}) comme en \REF{ex:2:111}a, soit par l'emploi des expressions qui entretiennent une relation symétrique de contraste entre les conjoints (p.ex. \textit{les uns...les autres}) comme en \REF{ex:2:111}b.


\begin{enumerate}
\item \label{bkm:Ref301357778}a  La situation est telle \textbf{que} soit on diminue les dépenses de l'Etat, soit on emprunte davantage.


\end{enumerate}
  b  ?Je pense \textbf{que} les uns vont aller pêcher et les autres préféreront le cinéma. 

En revanche, le roumain n'impose pas la répétition du complémenteur (cf. Dobrovie-\citet{Sorin1994}), qui reste optionnelle dans la plupart des cas \REF{ex:2:112}. 


\begin{enumerate}
\item \label{bkm:Ref273480950}a  Sper \textbf{că} vei învăța mai bine anul acesta şi (\textbf{că}) vei lua premiul întâi.


\end{enumerate}
{\itshape
J'espère que tu travailleras mieux cette année et (que) tu gagneras le premier prix}

  b  Nu am venit \textbf{fiindcă} am fost bolnav şi (\textbf{fiindcă}) nu m-a invitat nimeni. 

{\itshape
Je ne suis pas venu parce que j'ai été malade et (parce que) personne ne m'a invité}

  c  \textbf{Dacă} n-ai ce face şi (\textbf{dacă}) vrei să treci pe la mine, eşti binevenită.

{\itshape
Si tu n'as rien à faire et (si) tu veux passer chez moi, tu es bienvenue}

  d  Ion m-a întrebat \textbf{dacă} am fost la primărie şi (\textbf{dacă}) am semnat contractul.

    \textit{Ion m'a demandé si j'avais été à la mairie et (si) j'avais signé le contrat}

Toutefois, dans certains cas, la répétition du complémenteur est douteuse \REF{ex:2:113}a, voire agrammaticale (il est impossible de réitérer le complémenteur après la conjonction \textit{iar} `et', cf. \REF{ex:2:113}b, ou bien de répéter le complémenteur \textit{încât} marquant la conséquence, cf. \REF{ex:2:113}c).


\begin{enumerate}
\item \label{bkm:Ref273479978}a  \textbf{Deoarece} era cald afară şi (??\textbf{deoarece}) aveam timp la dispoziție, am ieşit în parc.


\end{enumerate}
{\itshape
Puisqu'il faisait chaud dehors et (puisque) j'avais du temps libre, je suis allé dans le parc}

  b  Mi s-a spus \textbf{că} Ion ar fi luat nota 10, iar (*\textbf{că}) Maria ar fi picat examenul.

{\itshape
On m'a dit que Ion aurait eu 10, et (que) Maria aurait raté l'examen}

  c  Am lucrat atât de mult \textbf{încât} m-am îmbolnăvit şi (*\textbf{încât})\textbf{} mi-am neglijat copiii. 

{\itshape
J'ai tellement travaillé que je suis tombé malade et (que) j'ai négligé mes enfants}

Au-delà des préférences des locuteurs ou des restrictions spécifiques à chaque item lexical, la présence ou l'absence du complémenteur dans la deuxième subordonnée peut être expliquée en termes discursifs. Le type de relation discursive influence le comportement des coordonnées sous l'enchâssement. Si la coordination des subordonnées met en jeu une relation discursive symétrique (c.-à-d. les événements décrits par les deux subordonnées sont indépendants l'un par rapport à l'autre), le complémenteur apparaît optionnellement devant chaque conjoint \REF{ex:2:114}a. Si, en revanche, les deux conjoints entretiennent une relation discursive asymétrique (c.-à-d. les deux événements sont dans une relation de dépendance discursive, p.ex. de type cause-effet), la coordination est analysée comme un seul bloc et le deuxième conjoint ne peut pas garder son complémenteur \REF{ex:2:114}b. 


\begin{enumerate}
\item \label{bkm:Ref269990986}a  Mi se pare \textbf{că} eşti cam leneş \textbf{şi} (\textbf{că}) aştepți să ți se aducă totul pe tavă.


\end{enumerate}
{\itshape
J'ai l'impression que tu es un peu paresseux et (que) tu attends que tout~te soit apporté sur un plateau} 

  b  Se  pare  \textbf{că}  îl  atingi  \textbf{şi}  (*\textbf{că})  te  omoară. 

    \textsc{refl.3 } paraît  que  \textsc{cl.3sg}  touches  et  (que)  \textsc{cl.2sg } tue

{\itshape
Il paraît qu'on le touche et il nous tue}

Pour expliquer l'optionalité du complémenteur dans les exemples donnés en \REF{ex:2:112}, on note que l'absence du complémenteur dans le dernier conjoint entraîne d'habitude une relation asymétrique entre les conjoints, alors que la présence du complémenteur met les événements plutôt au même plan discursif. 

En français aussi, le type de relation discursive entraîne une différence dans la distribution du complémenteur dans le deuxième conjoint. Selon \citet{Piscone2009}, le complémenteur est incompatible avec une relation asymétrique de type cause-effet~\REF{ex:2:115}a ; sinon, il est obligatoire \REF{ex:2:115}b. 


\begin{enumerate}
\item \label{bkm:Ref273482378}a  Il paraît qu'on le touche et (*qu')il nous tue.


\end{enumerate}
  b  Il paraît que Jean est rentré et *(qu')il a embrassé ses enfants.  

Finalement, on note la coordination de mots ou syntagmes interrogatifs, qui mettent en jeu une coordination entre deux subordonnées, dont la première est elliptique (cf. Bîlbîie \& \citet{Gazdik2011}). Habituellement, il s'agit d'une expression \textit{qu-} extraite en début de phrase, mais on peut avoir aussi des complémenteurs {\guillemotleft}~non-opaques~{\guillemotright} comme \textit{dacă} `si' (mais pas \textit{că} `que').


\begin{enumerate}
\item a  Cum putem şti \textbf{când} şi \textbf{dacă} e cazul să mai mergem la medic ?


\end{enumerate}
{\itshape
Comment peut-on savoir quand et s'il faut encore aller chez le médecin}

  b  Am tot încercat să aflu \emph{\textbf{\textup{dacă}}}\emph{} \emph{\textup{şi}}\emph{} \emph{\textbf{\textup{cum}}} poate fi schimbat destinul. 

{\itshape
J'ai essayé d'apprendre si et comment on pouvait changer le destin}

  c  Aştept să-mi spuneți \textbf{dacă} şi \textbf{când} am putea vorbi.

{\itshape
J'attends que vous me disiez si et quand on pourrait discuter}

  d  Am aflat \textbf{unde} şi \textbf{de} \textbf{cine} a fost ascunsă.

{\itshape
J'ai appris où et par qui elle a été cachée}

  e  Cercetătorii explică \textbf{cum} şi \textbf{de} \textbf{ce} căsătoria ține la distanță diferite boli.

{\itshape
Les chercheurs expliquent comment et pourquoi le mariage tient certaines maladies à distance}

\subsection{Les phrases liées et la juxtaposition} 
\label{bkm:Ref301448824}\label{bkm:Ref302035340}Si la coordination et la subordination se définissent par la présence d'un élément jonctif liant les phrases entre elles (c.-à-d. une conjonction ou un complémenteur), la juxtaposition\footnote{Dans la littérature, les termes \textit{juxtaposition, parataxe} et \textit{asyndète} sont généralement interchangeables (ils ont une étymologie différente : \textit{juxtaposition} vient du latin, alors que les deux autres viennent du grec).}  se caractérise par l'absence de marque de relation syntaxique (c.-à-d. asyndèse), ce qui amène Culicover \& \citet[528]{Jackendoff2005} à postuler que la juxtaposition est la manière la plus primitive de lier des unités. Le rôle des jonctifs revient, en leur absence, à la ponctuation (à l'écrit), à la prosodie (à l'oral) ou encore à certains connecteurs adverbiaux qui explicitent le type de relation discursive qui s'établit entre les phrases (cf. section \ref{sec:2.3}). 

Contrairement à certaines approches qui analysent la juxtaposition comme un sous-type de coordination (p.ex. \textit{GALR} (2005) et d'autres travaux traditionnels), je considère que la notion de juxtaposition est orthogonale à la coordination et à la subordination. Plus précisément, quand on juxtapose deux unités, il me semble que ce qui prime n'est pas la relation syntaxique entre les termes, mais les relations sémantiques et discursives. Donc, on ne s'intéresse pas à décider si c'est de la coordination ou de la subordination (car de toute façon on a la même analyse syntaxique), mais plutôt à voir le type de relation sémantico-discursive qui existe entre les unités juxtaposées.

La juxtaposition est un type de relation sous-spécifié~au niveau syntaxique. Par conséquent, je propose la même structure syntaxique, c.-à-d. une structure plate, pour toutes les occurrences des éléments juxtaposés. Je m'oppose ainsi à toute approche qui assume une conjonction~{\guillemotleft}~vide~{\guillemotright} pour les juxtapositions (voir \citet[84]{Johannessen1998}). En revanche, c'est au niveau discursif qu'on fait la différence entre les relations symétriques et les relations asymétriques.

Cette perspective a l'avantage de ne pas se limiter aux juxtapositions à l'intérieur des {\guillemotleft}~phrases orthographiques~{\guillemotright} (angl. \textit{orthographic sentences}, cf. Huddleston \& \citet[1728]{Pullum2002}, unités qui commencent par une majuscule et qui finissent par un point, point d'interrogation ou point d'exclamation), mais d'inclure aussi des phrases qui syntaxiquement sont des unités indépendantes, mais qui tout de même entretiennent des relations discursives, comme en \REF{ex:2:117}.


\begin{enumerate}
\item \label{bkm:Ref273636002}a  Ion n-a mers la şcoală azi. E bolnav cobză.


\end{enumerate}
{\itshape
Ion n'est pas allé à l'école aujourd'hui. Il est très malade}

  b  Nu vreau să ies din casă. E prea frig afară.

{\itshape
Je ne veux pas sortir de chez moi. Il fait trop froid dehors}

  c  Sunt foarte obosită. Am nevoie de un weekend la munte.

    \textit{Je suis très fatiguée. J'ai besoin d'un weekend à la montagne}

Je montre par la suite pourquoi postuler une analyse syntaxique similaire à la coordination ou à la subordination n'aboutit pas à une correspondance parfaite avec l'une ou l'autre de ces deux constructions. 

En ce qui concerne l'assimilation de la juxtaposition au domaine plus général de la coordination, le test utilisé est la possibilité d'insérer une conjonction entre les termes juxtaposés sans rendre l'énoncé agrammatical, cf. \REF{ex:2:118}.


\begin{enumerate}
\item \label{bkm:Ref301428345}a  Ion citeşte, Maria doarme.


\end{enumerate}
{\itshape
Ion lit, Maria dort}

  b  Ion citeşte \textbf{şi} Maria doarme.

    \textit{Ion lit et Maria dort}

Cependant, Blakemore \& \citet[572]{Carston2005} relèvent des faits sémantiques et discursifs qui montrent qu'on ne peut pas substituer de manière systématique une coordination à une juxtaposition. En \REF{ex:2:119}a, le deuxième segment est interprété comme une explication pour la situation décrite par le premier segment, interprétation qui n'est pas possible pour l'exemple \REF{ex:2:119}b où on utilise la conjonction \textit{and}. De même, dans l'exemple \REF{ex:2:120} le deuxième segment ne peut être interprété comme conclusion que dans la version asyndétique. 


\begin{enumerate}
\item \label{bkm:Ref272261087}a  Max fell asleep ; he was tired.


\end{enumerate}
\textit{${\neq}$}  b  Max fell asleep~and he was tired.


\begin{enumerate}
\item \label{bkm:Ref272261210}a  These are his footprints ; he's been here recently.


\end{enumerate}
\textit{${\neq}$}  b  These are his footprints~and he's been here recently.

Une autre différence entre la coordination et la juxtaposition concerne l'ordre des phrases. Nous observons que les unités en juxtaposition peuvent être facilement interverties sans changer l'interprétation \REF{ex:2:121}, alors que l'ordre des conjoints coordonnés par la conjonction \textit{şi} `et' ne peut pas changer sans entraîner des effets sémantiques \REF{ex:2:122}.


\begin{enumerate}
\item \label{bkm:Ref273638437}a  N-am mers azi la şcoală, sunt bolnav cobză.


\end{enumerate}
{\itshape
  Je ne suis pas allé à l'école, je suis très malade}

=  b  Sunt bolnav cobză, n-am mers azi la şcoală.

    \textit{Je suis très malade, je ne suis pas allé à l'école}


\begin{enumerate}
\item \label{bkm:Ref273638457}a  N-am mers azi la şcoală şi sunt bolnav cobză.


\end{enumerate}
{\itshape
  Je ne suis pas allé à l'école et je suis très malade}

${\neq}$  b  Sunt bolnav cobză şi n-am mers azi la şcoală.

    \textit{Je suis très malade et je ne suis pas allé à l'école}

Si on explicite la relation discursive par un connecteur adverbial, les différences sont encore plus claires. Si dans une juxtaposition le connecteur \textit{de fapt} `en fait' explicite une relation asymétrique de type cause-effet (c.-à-d. le deuxième segment est une explication du premier) comme \REF{ex:2:123}a, la présence du même connecteur dans une coordination avec \textit{şi} `et' produit un énoncé difficilement interprétable \REF{ex:2:123}b. Pour plus de détails sur les différences sémantiques entre les versions syndétique et asyndétique, voir \citet{Winterstein2010}.


\begin{enumerate}
\item \label{bkm:Ref273638711}a  Ion n-a mâncat nimic ; de fapt, nu-i e foame.


\end{enumerate}
{\itshape
  Ion n'a rien mangé ; en fait, il n'a pas faim}

  b  \#Ion n-a mâncat nimic şi de fapt nu-i e foame.

    \textit{Ion n'a rien mangé et en fait il n'a pas faim}

Pour ce qui est du côté de la subordination, Choi-Jonin \& Delais-\citet{Roussarie2006} argumentent contre le fait d'assimiler les phrases sans marque segmentale entretenant une relation spatiale ou temporelle \REF{ex:2:124}a à leurs contreparties reliées par un complémenteur \REF{ex:2:124}b en français. Ces phrases en asyndèse obéissent à des contraintes spécifiques qui ne sont pas observées dans leurs contreparties subordonnées : (i) la première phrase ne peut pas être négative \REF{ex:2:125} ; (ii) l'ordre des temps verbaux est restreint \REF{ex:2:126} ; (iii) la première phrase doit avoir un prédicat de type ponctuel s'il n'y a pas d'indication temporelle dans la deuxième \REF{ex:2:127}, etc. Syntaxiquement, les phrases en \REF{ex:2:124}a ne semblent manifester aucune relation de dépendance, ce qui détermine Choi-Jonin \& Delais \citet{Roussarie2006} à analyser ce type de phrases comme des {\guillemotleft}~enchaînements parataxiques~{\guillemotright}.


\begin{enumerate}
\item \label{bkm:Ref273720759}a  je suis venue sur Toulouse j'avais environ deux ans


\end{enumerate}
  b  je suis venue sur Toulouse \textbf{quand} j'avais environ deux ans


\begin{enumerate}
\item \label{bkm:Ref273721354}a  ??je ne suis pas venue sur Toulouse j'avais environ deux ans


\end{enumerate}
  b  je suis venue sur Toulouse j'avais même pas deux ans


\begin{enumerate}
\item \label{bkm:Ref273721371}a  on ouvrait la porte il y avait la mer en face


\end{enumerate}
  b  ??on ouvrait la porte il y a eu la mer en face

  c  ??on a ouvert la porte il y a eu la mer en face


\begin{enumerate}
\item \label{bkm:Ref273721388}a  quand je reste longtemps à Toulouse j'apprécie de rentrer à Roulins


\end{enumerate}
  b  ??je reste longtemps à Toulouse j'apprécie de rentrer à Roulins

  c  je suis venue sur Toulouse j'avais environ deux ans

J'ai suggéré plus haut que la juxtaposition était plutôt une relation sous-spécifiée au niveau syntaxique. Ce type d'analyse ne nous oblige pas à {\guillemotleft}~classer~{\guillemotright} les exemples délicats comme celui donné en \REF{ex:2:128}a\footnote{Les exemples sont repris de Choi-Jonin \& \citet[88]{Roussarie2006}.} qui se prête à une double analyse : coordination (cf. l'insertion d'une conjonction en \REF{ex:2:128}b) ou subordination (cf. l'insertion d'un complémenteur en \REF{ex:2:128}c). 


\begin{enumerate}
\item \label{bkm:Ref273722739}a  il arrivait il lançait son cartable sur le bureau


\end{enumerate}
  b  il arrivait \textbf{et} il lançait son cartable sur le bureau

  c  \textbf{quand} il arrivait il lançait son cartable sur le bureau

J'ai montré que, contrairement à la tradition grammaticale, la juxtaposition doit être définie en dehors de la coordination et de la subordination, et je propose qu'elle soit définie plutôt par rapport au type de relation discursive qui caractérise les éléments juxtaposés. 

\subsubsection{Juxtapositions symétriques}
Je distingue en première approximation deux types majeurs de juxtapositions symétriques, en fonction du type d'unité syntaxique concerné. Au niveau sous-phrastique (syntagmatique ou lexical), la juxtaposition symétrique se définit par un certain parallélisme syntaxique et / ou sémantique. Au niveau phrastique, elle se définit surtout par une relation discursive symétrique (voir la typologie de \citet{Kehler2002} et les autres références mentionnées dans la section \ref{sec:2.4.2}), explicitée dans bien des cas par un connecteur adverbial ou par des items corrélatifs. 

Au niveau sous-phrastique, je précise les trois emplois de la juxtaposition, tels qu'ils apparaissent dans \citet{Mouret2007} : (i) listes et énumérations \REF{ex:2:129}a ; (ii) reformulations où la référence des termes juxtaposés est superposée partiellement \REF{ex:2:129}b, et (iii) {\guillemotleft}~réparations~{\guillemotright} du discours oral spontané \REF{ex:2:129}c. 


\begin{enumerate}
\item \label{bkm:Ref273639363}a  Trebuie să cumpăr [roşii, mere, banane, cartofi]...


\end{enumerate}
{\itshape
Je dois acheter des tomates, des pommes, des bananes, des pommes de terre } 

  b  [Analiza, interpretarea] faptelor nu poate fi făcută fără prea multe dovezi.

{\itshape
L'analyse, l'interprétation des faits ne peut être faite sans avoir suffisamment d'épreuves } 

  c  Aveam [doi ani şi ceva, doi ani şi patru luni], când a murit mama.

    \textit{J'avais deux ans et quelque, deux ans et quatre mois, quand ma mère est morte}

Concernant l'emploi de la juxtaposition comme liste ou énumération, un exemple (en français) qui met bien en évidence le parallélisme syntaxique et sémantique que je viens d'évoquer est donné en \REF{ex:2:130} : une suite d'une dizaine de syntagmes est interprétée comme deux listes / deux énumérations.


\begin{enumerate}
\item \label{bkm:Ref273691528}\textit{Welovewords} invite tous ceux qui écrivent [un peu, beaucoup, passionément, à la folie], [des romans, des essais, des souvenirs, des poèmes, des nouvelles, des contes, des articles]... à partager leurs mots sur \href{http://www.welovewords.com/}{{www.welovewords.com}}.


\end{enumerate}
Le nombre de termes juxtaposés, selon \citet{Noailly1986}, peut indiquer le type de rapport envisagé. Selon elle, la juxtaposition de deux termes crée le plus souvent un rapport hiérarchisant, avec une nuance de précision ou de rectification, tandis que l'accumulation de plus de deux éléments crée plutôt une énumération, une liste et donne ainsi l'impression d'une juxtaposition symétrique.

Concernant les reformulations comme sous-type de juxtaposition, on y ajoute les contextes de correction avec la négation de constituant \textit{nu} `non' en roumain \REF{ex:2:131}a-b-c, y compris la négation métalinguistique (cf. \citet{Zafiu2005}) \REF{ex:2:131}d. Dans tous les exemples de reformulation, le dernier terme (c.-à-d. le constituant {\guillemotleft}~réparateur~{\guillemotright}) remplace littéralement l'autre terme du point de vue syntaxique et sémantique (voir l'accord du verbe au singulier en \REF{ex:2:131}a).


\begin{enumerate}
\item \label{bkm:Ref273693866}a  A  venit  Ion,  nu  Maria.


\end{enumerate}
\textsc{aux}  venu  Ion,  \textsc{neg}  Maria

{\itshape
Il est venu Ion, et non Maria}

  b  Am  mâncat  mere,  nu  pere.

    \textsc{aux}  mangé  pommes,  \textsc{neg}  poires

{\itshape
J'ai mangé des pommes et non des poires}

  c  Vin  luni,  nu  marți.

    viens  lundi,  \textsc{neg } mardi

{\itshape
Je viendrais lundi et non mardi}

  d  Se spune \textit{preşedinție}, nu \textit{preşedenție}.

    \textit{On dit {\guillemotleft}~preşedinție~{\guillemotright}, non {\guillemotleft}~preşedenție~{\guillemotright}}

On inclut dans la série des reformulations les relations de recapitulation ou de paraphrase (cf. \citet[387]{Longacre2007}), cf. \REF{ex:2:132}. 


\begin{enumerate}
\item \label{bkm:Ref301431913}a  Fred m-a uimit în dimineața asta, m-a uimit cât de mult a mâncat.


\end{enumerate}
{\itshape
Fred m'a impressionné ce matin, c'est la quantité qu'il mange qui m'a impréssionné}

  b  Azi m-am dus la primărie, m-am dus să-i spun primarului situația.

{\itshape
Aujourd'hui je suis allé à la mairie, j'y suis allé pour expliquer au maire la situation}

  c  E un monstru, e o brută.

{\itshape
C'est un monstre, c'est une brute}

  d  I-am strigat pe toți din casă, am țipat cât m-a ținut gura. 

    \textit{J'ai appelé tous ceux qui étaient à la maison, j'ai crié à tue-tête}

Aux juxtapositions symétriques mentionnées plus haut, j'ajoute les constructions corrélatives où chaque terme juxtaposé est introduit par un item corrélatif. En roumain, elles sont compatibles avec tous les types de syntagmes, y compris les phrases. Les corrélatifs peuvent être les adverbes associatifs \textit{şi}\footnote{A ne pas confondre \textit{şi}\textit{\textsubscript{1}} `et' conjonction et \textit{şi}\textit{\textsubscript{2}} `aussi' adverbe associatif. Voir la discussion dans la section \ref{sec:2.3.1} et 2.7. Voir aussi Bîlbîie (2008) pour des arguments contre une analyse en termes de conjonctions des items corrélatifs \textit{şi...şi} `et...et' et \textit{nici...nici} `ni...ni'.} `aussi' et \textit{nici} `non plus' \REF{ex:2:133}, des adverbes de {\guillemotleft}~circonstance~{\guillemotright} comme \textit{acum} `maintenant' ou \textit{când} `quand' qui marquent l'alternance temporelle en \REF{ex:2:134}a, des structures comparatives comme en \REF{ex:2:134}b-c, ou encore des adverbes structurant le discours comme \textit{pe de-o parte...pe de alta} `d'une part...d'autre part' \REF{ex:2:135}a ou \textit{întâi...apoi} `d'abord...ensuite' \REF{ex:2:135}a. Si les items corrélatifs en \REF{ex:2:133} et \REF{ex:2:135} permettent et la juxtaposition et la coordination (cf. la présence optionnelle d'une conjonction), les tours corrélatifs {\guillemotleft}~alternatifs~{\guillemotright} en \REF{ex:2:134}a et comparatifs en \REF{ex:2:134}b-c n'apparaissent que dans la juxtaposition en roumain (contrairement au français, voir plus de détails dans Abeillé \& \citet{Borsley2006}).


\begin{enumerate}
\item \label{bkm:Ref273695832}a  Am  mâncat  \textbf{şi}  mere  (şi)  \textbf{şi}  pere.


\end{enumerate}
ai  mangé  \textsc{correl}  pommes  (et)  \textsc{correl}  poires

{\itshape
J'ai mangé et des pommes et des poires}

  b  N-am  mâncat  \textbf{nici}  mere  (şi)  \textbf{nici}  pere.

    \textsc{neg-}ai  mangé  \textsc{correl } pommes  (et)  \textsc{correl}  poires

    \textit{Je n'ai mangé ni pommes ni poires}


\begin{enumerate}
\item \label{bkm:Ref273696343}a  \{\textbf{Acum {\textbar} Când}\}  ninge,  (*şi)  \{\textbf{acum {\textbar} când}\}  plouă.


\end{enumerate}
  \{maintenant {\textbar} quand\}  neige,  (et)  \{maintenant {\textbar} quand\}  pleut

{\itshape
Tantôt il neige, (et) tantôt il pleut}

  b  \textbf{Pe cât} e de isteț, (*şi) \textbf{pe-atât} e de leneş.

{\itshape
Autant il est perspicace, (et) autant il est paresseux}

  c  \textbf{Cu cât} e mai bogat, (*şi) \textbf{cu atât} e mai rău.

{\itshape
Plus il est riche, (et) plus il est méchant}


\begin{enumerate}
\item \label{bkm:Ref301458990}a  \textbf{Pe de-o parte}, criza generează panică ; (iar) \textbf{pe de alta}, stimulează creativitatea.


\end{enumerate}
{\itshape
D'une part, la crise génère de la panique ; (et) d'autre part, elle stimule la créativité}

 b  Au plecat \textbf{întâi} la Sibiu, (iar) \textbf{apoi} la Constanța.

    \textit{Ils sont partis d'abord à Sibiu, (et) ensuite à Constanța}

Finalement, je mentionne les juxtapositions qui permettent une relation discursive de contraste explicitée par un adverbe connecteur comme \textit{însă} `cependant' ou \textit{în schimb} `en revanche' comme en \REF{ex:2:136}. 


\begin{enumerate}
\item \label{bkm:Ref301447062}a  Imi place să citesc, (însă) nu-mi place (însă) să scriu.


\end{enumerate}
{\itshape
J'aime lire, (cependant) je n'aime pas (cependant) écrire}

  b  Imi place să citesc, în schimb nu-mi place să scriu.

    \textit{J'aime lire, en revanche je n'aime pas écrire}

\subsubsection{Juxtapositions asymétriques}
Dans l'introduction de cette section, j'ai procédé à la distinction juxtaposition symétrique vs. juxtaposition asymétrique sur une base essentiellement discursive, en prenant en compte surtout le niveau phrastique. Je m'intéresse donc ici aux juxtapositions qui mettent en jeu une relation discursive asymétrique (p.ex. condition, concession, résultat / but, explication, etc.). 

Les phrases qui se prêtent facilement à un emploi asyndétique sont les ajouts appelés {\guillemotleft}~circonstanciels~{\guillemotright} par la tradition grammaticale : elles peuvent indiquer une relation conditionnelle \REF{ex:2:137}, la concession \REF{ex:2:138}-\REF{ex:2:139}, le résultat \REF{ex:2:140}, etc. Dans certains cas, on observe une interaction entre les propriétés morphosyntaxiques des phrases et les processus établissant la cohérence discursive (cf. \citet{Kehler2002}). Ainsi, en roumain l'interprétation concessive d'une phrase dérive parfois soit d'un verbe au subjonctif (sans la marque \textit{să}) suivi d'une relative au présomptif \REF{ex:2:138}, soit d'une structure {\guillemotleft}~antithétique~{\guillemotright} (cf. \textit{GALR} (2005)) où le même verbe apparaît à la fois à la forme affirmative et négative \REF{ex:2:139}). Ou encore pour obtenir l'interprétation finale en l'absence d'un introducteur {\guillemotleft}~spécialisé~{\guillemotright}, on répète l'adverbe \textit{doar} `seulement', ce qui donne \textit{doar-doar} en \REF{ex:2:140}).


\begin{enumerate}
\item \label{bkm:Ref273715353}a  Ai  carte,  ai  parte.


\end{enumerate}
avoir.\textsc{ind.2sg}  livre,  avoir.\textsc{ind.2sg}  part

{\itshape
Du savoir vient avoir}

  b  La  calic  slujeşti,  calic  rămâi.

    à  avare  servir.\textsc{ind.2sg},  avare  rester.\textsc{ind.2sg}

    \textit{Si on est au service d'un avare, on devient soi-même avare } 


\begin{enumerate}
\item \label{bkm:Ref268716051}a  Zică  ce-o  zice  lumea,  eu  nu  mă  însor.


\end{enumerate}
dire.\textsc{subj.3  rel}  dire.\textsc{ind.presom.3sg}  le-monde,  je  \textsc{neg } me  marier.\textsc{ind.1sg}

{\itshape
Peu importe ce que disent les autres, je ne me marie pas}

  b  Facă  ce-o  vrea,  eu  nu-l  voi  ajuta.

    faire.\textsc{subj.3  rel}  vouloir.\textsc{ind.presom.3sg},  je  \textsc{neg-}le  vais  aider

{\itshape
  Qu'il fasse ce qu'il veut, je ne vais pas l'aider}


\begin{enumerate}
\item \label{bkm:Ref273710146}a  Vreau,  nu  vreau,  trebuie  să-l  ajut.


\end{enumerate}
vouloir.\textsc{ind.1sg},  \textsc{neg}  vouloir\textsc{.ind.1sg},  faut  \textsc{mrq-cl } aider.\textsc{subj.1sg}

{\itshape
Je veux, je ne veux pas, je dois l'aider}

  b  Am,  nu  am,  trebuie  să-i  dau  ceva.

    avoir.\textsc{ind.1sg},  \textsc{neg}  avoir\textsc{.ind.1sg},  faut  \textsc{mrq-cl } donner.\textsc{subj.1sg } qq-ch\textsc{ } 

{\itshape
J'en ai, je n'en ai pas, je dois lui donner quelque chose}


\begin{enumerate}
\item \label{bkm:Ref273710400}Cerşetorul  stă  ore  în  şir  la  uşă,  \textbf{doar-doar}  o primi  un  ban.


\end{enumerate}
  le-mendiant  passe  heures  en  suite  à  porte,  seulement-seulement  recevra  un  sou

{\itshape
Le mendiant passe des heures entières devant la porte, il espérait recevoir ne serait-ce qu'un sou}

Culicover \& \citet[480]{Jackendoff2005} précisent que dans les contextes avec juxtaposition un rôle très important dans l'interprétation des énoncés revient parfois à la prosodie des phrases concernées. Une lecture conditionnelle va ainsi être obtenue en \REF{ex:2:141} si l'on a un contour ascendant dans la première phrase et un contour descendant dans la deuxième. 


\begin{enumerate}
\item \label{bkm:Ref273716163}a  Ai început să te droghezi$\uparrow $, nu mai poți să scapi uşor$\downarrow $.


\end{enumerate}
{\itshape
On a commencé à se droguer, on ne peut plus s'en sortir facilement}

  b  Te-ai căsătorit cu cine nu trebuie$\uparrow $, regreți toată viața$\downarrow $.

    \textit{On a épousé la personne qui ne fallait pas, on le regrette toute sa vie}

A l'écrit, la ponctuation~à elle seule peut parfois désambiguïser le type de relation discursive qui s'établit entre deux phrases, p.ex. les deux points en \REF{ex:2:142} explicitent un résultat ou une explication. 


\begin{enumerate}
\item \label{bkm:Ref273714782}a  M-am hotărât : nu mai plec nicăieri !


\end{enumerate}
{\itshape
Je me suis décidé : je ne pars plus nulle part } 

  b  Ion nu mi-a răspuns nicio vorbă : îl enervasem prea tare.

    \textit{Ion ne m'a pas répondu un mot : je l'avais trop énérvé}

D'autres constructions juxtaposées imposant une relation discursive asymétrique sont les phrases ayant comme tête une forme verbale non-finie, p.ex. participe présent (appelé gérondif en roumain) \REF{ex:2:143}a, participe passé \REF{ex:2:143}b ou passif~\REF{ex:2:143}c. 


\begin{enumerate}
\item \label{bkm:Ref273714325}a  Zgomotul încetând, m-a luat somnul imediat.


\end{enumerate}
{\itshape
Le bruit cessant, je me suis endormi rapidement}

  b  Odată Maria ieşită din casă, am început să-mi fac de cap. 

{\itshape
Une fois Maria sortie de la maison, je me suis mis à faire les cent coups}

  c  Odată lucrul terminat, am putut ieşi în parc liniştiți. 

\textit{Une fois le travail fini, on a pu sortir dans le parc tranquillement}  

Un dernier exemple de phrases juxtaposées est constitué par les phrases de citation directe (angl. \textit{direct quotation clauses}) qui sont les compléments d'un verbe de discours direct.\footnote{A la suite de Bonami \& \citet{Godard2008b}, nous faisons la différence entre le discours direct (ia) et l'incise (ib), sur une base purement empirique (différences liées au placement de la citation, placement du verbe de citation, inversion du sujet, type de prosodie, enchâssement et clivage de la citation, emploi de certains verbes comme \textit{imaginer} et\textit{ hoqueter} uniquement en incise, etc.). Voir aussi la section \ref{sec:2.8}.
 (i)  a  Paul a dit : {\guillemotleft}~Je viendrai demain~{\guillemotright}. 
  b  {\guillemotleft}~Je viendrai demain~{\guillemotright}, a dit Paul.} Parmi les trois classes de verbes mentionnés par Bonami \& \citet{Godard2008b}, ceux qui peuvent introduire une citation phrastique en roumain sont les verbes du genre \textit{a spune / a zice} `dire' comme \REF{ex:2:144}a et les verbes du type \textit{a afirma} `affirmer', \textit{a întreba} `demander', \textit{a porunci} `ordonner' (qui sélectionnent le type d'acte de langage qui est {\guillemotleft}~imité~{\guillemotright} dans la citation, c.-à-d. assertion \REF{ex:2:144}b, interrogation \REF{ex:2:144}c et respectivement ordre \REF{ex:2:144}d).  


\begin{enumerate}
\item \label{bkm:Ref275335879}a  Maria mi-a spus : {\guillemotleft}~Nu te voi părăsi niciodată~{\guillemotright}.


\end{enumerate}
{\itshape
Maria m'a dit : {\guillemotleft}~Je ne te quitterai jamais~{\guillemotright}}

  b  Profesorul a afirmat : {\guillemotleft}~Astăzi vom învăța regula de trei simplă.~{\guillemotright}

{\itshape
Le professeur a affirmé : {\guillemotleft}~Aujourd'hui nous apprenons la règle de trois simple~{\guillemotright}}

 c  Profesorul m-a întrebat : {\guillemotleft}~\c{T}i-ai făcut tema ?~{\guillemotright}

{\itshape
    Le professeur m'a demandé : {\guillemotleft}~Est-ce que tu as fait ton devoir~{\guillemotright}}

  d  Profesorul mi-a poruncit : {\guillemotleft}~Fă-ți tema !~{\guillemotright}

\textit{Le professeur m'a ordonné : {\guillemotleft}~Fais ton devoir~{\guillemotright}}  

\subsection{Les phrases liées et les éléments corrélatifs}
\label{bkm:Ref301432144}Dans les sections précédentes, on a pu observer que les phrases peuvent être liées par un introducteur (et on a ainsi une relation syndètique) ou bien par rien (et dans ce cas on a une relation asyndétique ou de juxtaposition, cf. section \ref{sec:2.6}). On ajoute les constructions dites {\guillemotleft}~corrélatives~{\guillemotright} dans lesquelles on observe des items (jonctifs ou adverbiaux) formellement interdépendants. Les éléments corrélatifs le plus étudiés sont ceux relevant du domaine de la coordination, dans ce qu'on appelle les coordinations omnisyndétiques, car le corrélatif est redoublé devant chaque conjoint (p.ex. les \textit{conjonctions doubles} en français,\textbf{} cf.\textbf{} \citet{Piot2000}). Mais on trouve des structures corrélatives dans le domaine de la subordination aussi. 

Dans ce qui suit, je me concentre dans un premier temps sur quelques aspects relevant des coordinations corrélatives, en mettant l'accent sur leurs propriétés spécifiques, par rapport aux propriétés générales des coordinations simples. Ensuite, je présente brièvement quelques emplois corrélatifs dans le domaine de la subordination.

\subsubsection{Les corrélatifs dans la coordination}
\label{bkm:Ref302037437}En fonction de la catégorie syntaxique des éléments corrélatifs, on en distingue trois types majeurs en roumain (cf. Bîlbîie (2008)) : 

(i) Constructions corrélatives conjonctives \textit{Conj...Conj}. Ce type se rapproche des conjonctions {\guillemotleft}~doubles~{\guillemotright} en français (\citet{Piot2000}, \citet{Mouret2007}) et dans d'autres langues romanes, mais il est réservé en roumain aux conjonctions disjonctives \textit{fie...fie} `soit...soit', qui ont seulement un emploi corrélatif\footnote{S'il s'agit d'une coordination de subordonnées, on a \textit{fie că...fie că} `soit...soit' :

\begin{enumerate}
\item \textbf{Fie că}-şi dă seama de situație, \textbf{fie că} nu, el acționează în acelaşi fel.


\end{enumerate}
\textsc{corr} qu'il s'en rende compte de la situation, \textsc{corr} que non, il agit de la même façon
Qu'il s'en rende compte de la situation ou pas, il agit de la même façon} \REF{ex:2:147}a, et \textit{sau...sau},\textit{ ori...ori} `ou...ou', cf. \REF{ex:2:147}b. 


\begin{enumerate}
\item a  Ca să slăbeşti, \textbf{fie} mănânci mai puțin, \textbf{fie} faci mai mult sport.


\end{enumerate}
{\itshape
Pour maigrir, soit on mange moins, soit on fait plus de sport}

  b  (\textbf{Ori}) plec în vacanță, \textbf{ori} lucrez.

{\itshape
Ou bien je pars en vacances, ou bien je travaille}

(ii) Constructions corrélatives adverbiales \textit{Adv...Adv}. Ce type regroupe essentiellement les paires \textit{şi...şi} `et...et' \REF{ex:2:146}a et \textit{nici...nici} `ni...ni' \REF{ex:2:146}b, qui, contrairement à leurs correspondants dans les autres langues romanes, ont un comportement adverbial et non conjonctif (cf. discussion dans la section \ref{sec:2.3.1} et dans Bîlbîie (2008)).


\begin{enumerate}
\item \label{bkm:Ref301463209}a  Ion  ştie  \textbf{şi}  să  citească,  (şi)  \textbf{şi}  să  scrie.


\end{enumerate}
Ion  sait  \textsc{correl  mrq  } lire.\textsc{subj},  (et)  \textsc{correl  mrq}  écrire.\textsc{subj} 

{\itshape
Ion sait et lire, et écrire}

  b  Ana  nu  ştie  \textbf{nici}  să  scrie,  (şi)  \textbf{nici}  să  citească.

    Ana  \textsc{neg}  sait  \textsc{correl  mrq}  écrire.\textsc{subj},  (et)  \textsc{correl  mrq}  lire\textsc{.subj}

    \textit{Ana ne sait ni écrire, ni lire}

(iii) Constructions à modifieur initial \textit{Adv...Conj}. C'est le seul type qui se rapproche des constructions corrélatives qu'on retrouve dans les langues germaniques, où le premier élément corrélatif est un adverbe associatif (avec une distribution relativement libre, des effets de portée, des restrictions sur l'association avec le focus, etc., cf. \citet{Hendriks2004}, \citet{Johannessen2005}, \citet{Hofmeister2010}), qui modifie la coordination dans son ensemble, d'où le terme de \textit{coordination initiale}. En roumain, ce type est représenté par la paire \textit{nu numai...ci şi} `non seulement...mais aussi' \REF{ex:2:147}a, et dans certains emplois la paire homonyme \textit{şi...şi} `et...et', différente syntaxiquement de la paire \textit{şi...şi} `et...et' présentée en (ii). Le premier \textit{şi} est un adverbe associatif, car il peut ne pas apparaître sur chacun des éléments coordonnés \REF{ex:2:147}b ou encore il n'est pas suivi de \textit{şi}, mais de \textit{sau} qui est clairement une conjonction \REF{ex:2:147}c. 


\begin{enumerate}
\item \label{bkm:Ref301458422}a  \textbf{Nu numai} fetele, \textbf{ci şi} băieții vor ajuta la curățarea clasei.


\end{enumerate}
    \textit{Non seulement les filles, mais aussi les garçons vont aider au nettoyage de la classe}

  b  20 de țări, printre care \textbf{şi} Rusia, Franța \textbf{şi} Italia, spionează intens Marea Britanie.

{\itshape
20 pays, parmi lesquels aussi la Russie, la France et l'Italie, espionnent intensivement la Grande Bretagne}

  c  6 companii, printre care \textbf{şi} Vodafone \textbf{sau} Cosmote România, au primit amendă pentru trimiterea de mesaje nesolicitate.

\textit{6 compagnies, parmi lesquelles aussi Vodafone ou Cosmote România, ont reçu des amendes pour l'envoi des messages non-sollicités}   

Les coordinations corrélatives obéissent à des contraintes spécifiques, par rapport aux coordinations simples (cf. Mouret (2007, \textit{à paraître}). Distributionnellement, ces éléments corrélatifs sont réservés à la coordination de catégories syntagmatiques variées, y compris des phrases (de ce point de vue, le français présente plus de restrictions surtout avec les paires \textit{et...et} et \textit{ni...ni}, qui sont moins acceptables dans la coordination de syntagmes verbaux ou phrases racines). Au niveau sémantique, les coordinations avec des éléments corrélatifs reçoivent uniquement une interprétation de conjonction ou de disjonction de contenus propositionnels : en particulier, elles ne peuvent pas dénoter une pluralité (explicitée par un ajout comme \textit{împreună} `ensemble') avec l'emploi de la paire \textit{şi...şi} `et...et' (comparer la coordination simple \REF{ex:2:148}a et la coordination corrélative \REF{ex:2:148}b). De plus, la disjonction reçoit uniquement une interprétation exclusive dans les tours corrélatifs en \REF{ex:2:149}b, contrairement à la disjonction simple \REF{ex:2:149}a qui est compatible avec une lecture exclusive ou inclusive. Une contrainte supplémentaire concerne l'impossibilité d'utiliser les corrélatifs disjonctifs pour former un ensemble d'alternatives dans les interrogatives \REF{ex:2:150}.


\begin{enumerate}
\item \label{bkm:Ref301505562}a  Ion \textbf{şi} Maria au venit (împreună) la petrecere.


\end{enumerate}
{\itshape
Ion et Maria sont venus (ensemble) à la fête}

  b  \textbf{Și} Ion (şi) \textbf{şi} Maria au venit (*împreună) la petrecere.

{\itshape
Et Ion et Maria sont venus (ensemble) à la fête}


\begin{enumerate}
\item \label{bkm:Ref301510943}a  Veți găsi informații la ghişeul 9 \textbf{sau} la ghişeul 10.


\end{enumerate}
{\itshape
Vous trouverez des informations au guichet 9 ou au guichet 10}

  b  Veți găsi informații \textbf{fie} la ghişeul 9 \textbf{fie} la ghişeul 10.

{\itshape
Vous trouverez des informations soit au guichet 9 soit au guichet 10}


\begin{enumerate}
\item \label{bkm:Ref301533034}a  Vii cu noi \textbf{sau} rămâi acasă ?


\end{enumerate}
{\itshape
Tu viens avec nous ou tu restes à la maison}

  b  \#\textbf{Fie} vii cu noi \textbf{fie} rămâi acasă ?

{\itshape
Soit tu viens avec nous soit tu restes à la maison}

Au niveau discursif, les coordinations corrélatives doivent entretenir une relation symétrique, contrairement aux coordinations simples qui sont compatibles avec les deux types de relations (cf. section \ref{sec:2.4.2}). Les relations de discours asymétriques, c.-à-d. non-préservées lorsqu'on permute les termes conjoints, ne sont pas autorisées (donc, on n'a pas de succession temporelle, pas de conséquence, pas de condition). Du point de vue pragmatique, on note qu'une coordination corrélative met en relief la relation dénotée par l'élément corrélatif par exclusion des relations alternatives. Elle introduit plus d'informations que ce qui est descriptivement nécessaire et ces informations deviennent ainsi particulièrement pertinentes pour le discours en question. Enfin, une dernière contrainte concerne le niveau prosodique où on observe que chaque conjoint contenant un élément corrélatif doit appartenir à un groupe prosodique distinct, marqué par une saillance prosodique sur l'élément corrélatif (notée en majuscules dans les exemples en \REF{ex:2:151}.


\begin{enumerate}
\item \label{bkm:Ref301510387}a  Ion  ştie  \textbf{ȘI}  să  citească,  (şi)  \textbf{ȘI}  să  scrie.


\end{enumerate}
Ion  sait  \textsc{correl  mrq  } lire.\textsc{subj},  (et)  \textsc{correl  mrq}  écrire.\textsc{subj} 

{\itshape
Ion sait et lire, et écrire}

  b  Ana  nu  ştie  \textbf{NICI}  să  scrie,  (şi)  \textbf{NICI}  să  citească.

    Ana  \textsc{neg}  sait  \textsc{correl  mrq}  écrire.\textsc{subj},  (et)  \textsc{correl  mrq}  lire\textsc{.subj}

    \textit{Ana ne sait ni écrire, ni lire}  

\subsubsection{Les corrélatifs dans la subordination}
Outre les éléments corrélatifs qui apparaissent dans la coordination, il y a certains adverbes {\guillemotleft}~circonstanciels~{\guillemotright} qui peuvent apparaître en emploi corrélatif dans la phrase racine, leur forme étant déterminée par le type de subordonnant qui introduit la phrase circonstancielle ajout. Ainsi, dans des constructions subordonnées avec des complémenteurs, on a la paire conditionnelle \textit{dacă...atunci} `si...alors' \REF{ex:2:152}a, la paire concessive \textit{chiar dacă...tot} `même si...tout de même' \REF{ex:2:152}b, la paire causale \textit{fiindcă...de aceea} `puisque...c'est pour cela' \REF{ex:2:152}c. On a aussi des constructions contenant des formes \textit{qu-} où on retrouve les formes comparatives \textit{cât...atât} `autant...autant' \REF{ex:2:153}a, \textit{cum...aşa} `de même que...de même' \REF{ex:2:153}b, ou encore la paire \textit{cum...cum} marquant la succession temporelle \REF{ex:2:153}c. Dans toutes ces occurrences, l'adverbe corrélatif de la deuxième phrase (ici, la phrase racine) est optionnel. L'insertion d'une conjonction entre les deux phrases est impossible.


\begin{enumerate}
\item \label{bkm:Ref301515796}a  \textbf{Dacă} vrei bani, (\textbf{atunci}) munceşte !


\end{enumerate}
{\itshape
Si tu veux de l'argent, alors travaille}

  b  \textbf{Chiar dacă} nu-ți place rochia, eu (\textbf{tot}) o cumpăr.

{\itshape
Même si tu n'aimes pas la robe, je l'achète tout de même}

  c  \textbf{Fiindcă} ştiam ce pierdem, (\textbf{de} \textbf{aceea}) am insistat.

{\itshape
Puisque je savais ce qu'on perd, c'est pour cela que j'ai insisté}


\begin{enumerate}
\item \label{bkm:Ref301516573}a  \textbf{Cât} câştigă, (\textbf{atât}) cheltuie.


\end{enumerate}
{\itshape
Autant il gagne, autant il dépense}

  b  \textbf{Cum} vine, \textbf{aşa} pleacă.

{\itshape
De même qu'il vient, de même il part} 

  c  Pisoiul, \textbf{cum} intră în bucătărie, (\textbf{cum}) începe să cotrobăie după mâncare.

    le-chat, \textsc{adv} rentre en cuisine, \textsc{corr} commence chercher.\textsc{subj} après nourriture

{\itshape
Le chat, dès qu'il rentre dans la cuisine, commence à chercher de la nourriture}

Je finis cette section par une petite discussion sur les {\guillemotleft}~corrélatives comparatives~{\guillemotright} \REF{ex:2:154}, dont le statut syntaxique n'est pas clair : certains les analysent comme des constructions coordonnées (Abeillé \& \citet{Borsley2006}), d'autres les traitent comme relevant du domaine de la subordination (Den \citet{Dikken2005}) et d'autres encore les analysent comme ayant un comportement hybride (Culicover \& \citet{Jackendoff2005}, Bril \& \citet{Rebuschi2006}). 


\begin{enumerate}
\item \label{bkm:Ref301516794}a  \textbf{The} long\textbf{er} he has to wait, \textbf{the} angri\textbf{er} John gets.


\end{enumerate}
  b  \textbf{Plus} il doit attendre longtemps, \textbf{plus} Jean devient furieux.

De manière générale, on décrit la construction corrélative comparative comme une structure binaire avec deux phrases racines généralement juxtaposées, qui discursivement entretiennent une relation conditionnelle (Mc\citet{Cawley1988}, Culicover \& \citet{Jackendoff2005}, etc.).

Les propriétés des corrélatives comparatives en roumain semblent être différentes des propriétés que les mêmes constructions ont en anglais et en français. La comparative initiale est introduite par un syntagme prépositionnel extrait contenant une préposition (\textit{cu} `avec' ou \textit{pe} `sur') et la forme \textit{qu-} \textit{cât} `combien'. La deuxième phrase commence aussi par un syntagme prépositionnel extrait, mais la préposition est suivie cette fois-ci par l'adverbe \textit{atât} `tant'.


\begin{enumerate}
\item a  \textbf{Cu  cât}  eşti  mai  slab,  \textbf{cu  atât}  eşti  mai  sănătos.


\end{enumerate}
avec  combien  être.\textsc{ind.2sg}  plus  mince,  avec  tant  être.\textsc{ind.2sg}  plus  sain 

{\itshape
Plus on est mince, plus on est sain}

  b  \textbf{Pe  cât}  e  de  frumoasă,  \textbf{pe  atât}  e  de  leneşă.

    sur  combien  être.\textsc{ind.3sg}  de  belle,  sur  tant  être.\textsc{ind.3sg}  de  paresseuse

    \textit{Autant elle est belle, autant elle est paresseuse}

Contrairement au français, on ne peut pas insérer de conjonction entre les deux phrases \REF{ex:2:156}a, ce qui est un argument contre l'hypothèse selon laquelle les corrélatives comparatives sont des constructions coordonnées. Le syntagme prépositionnel contenant la forme \textit{qu-} \textit{cât} dans la comparative initiale peut fonctionner comme élément relatif dans les subordonnées complétives ordinaires \REF{ex:2:156}b. Contrairement aux constructions coordonnées, les corrélatives comparatives permettent non seulement l'anaphore \REF{ex:2:156}c, mais aussi la cataphore \REF{ex:2:156}d, ce qui les rapproche des constructions subordonnées. Sur la base de ces propriétés, j'aligne les corrélatives comparatives du roumain sur les autres structures à subordination illustrées précédemment. 


\begin{enumerate}
\item \label{bkm:Ref301522035}a  \textbf{Cu  cât}  eşti  mai  slab,  (*şi)  \textbf{cu  atât}  eşti  mai  sănătos.


\end{enumerate}
avec  combien  être.\textsc{ind.2sg}  plus  mince,  (et)  avec  tant  être.\textsc{ind.2sg}  plus  sain 

{\itshape
Plus on est mince, plus on est sain}

  b  El  mi-a  spus  [\textbf{cu  cât}   a  cumpărat  apartamentul].

    il  m'a  dit  avec  combien  a  acheté  l'appartement

    \textit{Il m'a dit pour combien il avait acheté l'appartement}

  c  \textbf{Cu cât}  Ana  citeşte  mai mult,  \textbf{cu atât}  devine  mai  înțeleaptă.

    \textsc{correl}  Ana  lit  davantage,  \textsc{correl } devient  plus  sage

    \textit{Plus Ana lit davantage, plus elle devient plus sage} 

  d  \textbf{Cu cât } citeşte  mai mult,  \textbf{cu atât } Ana  devine  mai  înțeleaptă.

    \textsc{correl}  lit plus  beaucoup,  \textsc{correl}  Ana  devient  plus  sage

    \textit{Plus elle lit davantage, plus Ana devient plus sage}

\subsection{Les phrases liées et l'incidence}
\label{bkm:Ref302035390}\label{bkm:Ref302037872}\label{bkm:Ref301443208}A côté des structures coordonnées et subordonnées ordinaires, on trouve des incidents phrastiques qui utilisent les mêmes introducteurs que les deux structures mentionnées, mais qui présentent des propriétés différentes. Je commence cette section par une définition du terme \textit{incident} (section \ref{sec:2.8.1}), pour ensuite donner une typologie des incidents phrastiques introduits par une conjonction, par un élément subordonnant ou par rien (section \ref{sec:2.8.2}). Finalement, je discute les corrélations souvent faites dans la littérature entre l'incidence et d'autres facteurs linguistiques (section \ref{sec:2.8.3}), en particulier la (fausse) corrélation entre l'incidence et la parenthéticité. L'étude des incidents phrastiques sera d'ailleurs utile dans la description des ajouts relatifs averbaux, dans le chapitre 5. 

\subsubsection{La notion d'incident}
\label{bkm:Ref301534217}On appelle \textit{incidents} les constituants qui sont isolés prosodiquement du reste de la phrase dans laquelle ils apparaissent (Bonami \& Godard (2007a, 2007b, 2007c, 2008a)), contrairement aux constituants ordinaires qui reçoivent une prosodie integrée. Cette propriété prosodique d'incidence (appelée {\guillemotleft}~comma intonation~{\guillemotright} dans la tradition grammaticale) est souvent transcrite à l'écrit par l'utilisation de virgules (ou tirets) \REF{ex:2:157}a-b, et à l'oral par la possibilité de pause aux marges, ainsi que par la possibilité d'allongement des syllabes finales de l'incident et du constituant précédant l'incident (cf. Delais-\citet{Roussarie2005} pour le français), ce qui fait des constituants en question des syntagmes prosodiques autonomes. 


\begin{enumerate}
\item \label{bkm:Ref274849789}a  Mama, \textbf{Dumnezeu s-o ierte}, a muncit enorm pentru mine.


\end{enumerate}
{\itshape
Ma mère, que Dieu lui pardonne, a énormément travaillé pour moi}

  b  Fata care a luat postul e -- \textbf{după câte se pare} -- nepoata preşedintelui de comisie.

{\itshape
La fille qui a eu le poste est -- semble-t-il -- la nièce du président du jury}

Selon Bonami \& Godard (2007c, 2008a), il existe une large variété de constructions mettant en jeu l'incidence. A part les incidents sous-phrastiques (p.ex. adverbes incidents, dislocations, vocatifs, topicalisations, appositions nominales, etc.), il y a les incidents phrastiques qui peuvent être introduits par une conjonction, un complémenteur ou rien. Du point de vue syntaxique, ces phrases incidentes s'interpolent parmi les constituants majeurs d'un hôte phrastique ou non-phrastique. Leur placement dans la phrase est relativement libre, ce qui distingue ces constructions incidentes de leurs contreparties coordonnées, subordonnées ou juxtaposées.

\subsubsection{Typologie des incidents phrastiques} 
\label{bkm:Ref302039771}\paragraph[Les incidents sans introducteur]{Les incidents sans introducteur}
Je mentionne tout d'abord les incises qui sont, selon Bonami \& Godard \textit{à paraître}, des phrases à verbe conjugué, sans terme introducteur, et qui reçoivent obligatoirement la prosodie incidente. Apartiennent à cette classe les incises ordinaires \REF{ex:2:158} et les incises de citation \REF{ex:2:159}-\REF{ex:2:160}. 


\begin{enumerate}
\item \label{bkm:Ref275288250}a  Inițiativa a aparținut, pare-se, celor doi manageri ai întreprinderii.


\end{enumerate}
{\itshape
L'initiative a appartenu, semble-t-il, aux deux managers de l'entreprise}

  b  Maria are, țineți-vă bine\textbf{,} şapte ani de arte marțiale.

{\itshape
Maria a, tenez-vous bien, sept ans d'arts martiaux}

  c  Până acum am făcut, să zicem, aproape jumătate din ce aveam de făcut.

{\itshape
Jusqu'à maintenant j'ai fait, disons, presque la moitié de ce que j'avais à faire}

  d  Tu, am eu impresia, nu-ți dai seama cât de mult sufăr.

{\itshape
Toi, j'ai l'impression, ne te rends pas compte combien je souffre}


\begin{enumerate}
\item \label{bkm:Ref275288301}a  {\guillemotleft}~Totul e pierdut pentru mine\textsubscript{i}~{\guillemotright}, îmi spuse el\textsubscript{i}.


\end{enumerate}
{\itshape
{\guillemotleft}~Tout est perdu pour moi~{\guillemotright}, me dit-il}

  b  {\guillemotleft}~Ai mâncat ?~{\guillemotright}, mă întrebă Ion.

{\itshape
{\guillemotleft}~Est-ce que tu as mangé ?~{\guillemotright}, me demanda Ion}

  c  {\guillemotleft}~Nu vei reuşi~{\guillemotright}, adăugă Maria cu o privire disprețuitoare.

{\itshape
{\guillemotleft}~Tu ne réussiras pas~{\guillemotright}, ajouta Maria avec un regard méprisant}


\begin{enumerate}
\item \label{bkm:Ref275288287}a  Va continua să lucreze în cadrul grupului, \emph{\textup{spune}}\emph{} \emph{\textup{el}}\emph{\textup{\textsubscript{i}}}, până îl\textsubscript{i} vor lăsa puterile.


\end{enumerate}
{\itshape
Il continuera à travailler dans ce groupe, dit-il, jusqu'à la fin de ses forces}

  b  Oare va putea vreodată, se întreba Ion\textsubscript{i}, să-şi\textsubscript{i} mai vadă copilul ?

{\itshape
Pourrait-il encore, se demandait Ion, revoir son enfant}

  c  După concedierea lui\textsubscript{i}, ne amenință Ion\textsubscript{i}, {\guillemotleft}~toți vom avea de suferit~{\guillemotright}\footnote{Dans cet exemple, l'hôte comporte ce qu'on appelle une {\guillemotleft}~citation mixte~{\guillemotright} (\citet{Davidson1979}) ou encore une {\guillemotleft}~citation de constituant~{\guillemotright} (\citet{Potts2007}).} .

    \textit{Après son licenciement, nous ménace-t-il, nous tous allons souffrir}

Les incises de citation constituent leur hôte comme une citation qui a toutes les propriétés d'un constituant ordinaire : elle peut apparaître au milieu d'un syntagme verbal \REF{ex:2:161}a, elle peut être enchâssée sous une subordonnée \REF{ex:2:161}b ou reprise par le clitique pronominal \textit{o}\footnote{En roumain, la reprise pronominale des citations se réalise, théoriquement, par l'emploi du clitique accusatif de troisième personne au féminin (\textit{accusatif à valeur neutre}, cf. \textit{GALR} (2005 : 207)). En réalité, c'est très difficile de dire si l'on a affaire à une reprise pronominale, car la plupart des exemples attestés peuvent être interprétés comme appartenant à des expressions (semi)lexicalisées (i). C'est pour cela que je ne prends pas en compte le critère de la reprise pronominale pour faire la différence entre les incises de citation et d'autres types d'incises (comme Bonami \& Godard le font). 
(i)  {\guillemotleft}~Dar dumneavoastră cine sunteți ?~{\guillemotright} mi-a întors-o ea degajată.
  \textit{{\guillemotleft}~}\textit{Mais vous, qui est-ce que vous êtes?} \textit{{\guillemotright}, me l'a-t-elle repliqué d'une manière détendue} } dans l'incise \REF{ex:2:161}c, elle peut apparaître dans une structure clivée \REF{ex:2:161}d ou bien être coordonnée \REF{ex:2:161}e.


\begin{enumerate}
\item \label{bkm:Ref275445357}a  Maria tot striga~disperată :~{\guillemotleft} Vreau să mor !~{\guillemotright} de față cu copilul ei. 


\end{enumerate}
{\itshape
Maria criait sans cesse désespérée : {\guillemotleft}~Je veux mourir !~{\guillemotright} en présence de son enfant}

  b  Nu eram sigur că Paul a spus : {\guillemotleft}~Vreau să mor !~{\guillemotright} de față cu copilul lui.

{\itshape
Je n'étais pas sûr que Paul ait dit : {\guillemotleft}~Je veux mourir !~{\guillemotright} en présence de son enfant} 

  c  {\guillemotleft}~Eşti o vrăjitoare mică~{\guillemotright}, mi-a spus-\textbf{o} în urmă cu mulți ani cineva.

{\itshape
{\guillemotleft}~Tu es une petite sorcière~{\guillemotright}, me l'a dit quelqu'un il y a longtemps}

  d  Ceea ce spunea Maria în disperarea ei era : {\guillemotleft}~Vreau să mor !~{\guillemotright}.

{\itshape
Ce que disait Maria dans son désespoir était :~{\guillemotleft}~Je veux mourir !~{\guillemotright}}

  e  Ana a spus : {\guillemotleft}~Vreau să mor !~{\guillemotright} şi nicidecum {\guillemotleft}~Aş mai fi vrut să trăiesc !~{\guillemotright}.

    \textit{Ana a dit : {\guillemotleft}~Je veux mourir !~{\guillemotright} et pas~{\guillemotleft}~J'aurais bien voulu vivre un peu plus~{\guillemotright}}

En revanche, ce qui distingue les citations par rapport aux constituants ordinaires est la manière dont la référence est établie. Selon Clark \& \citet{Gerrig1990}, les citations réfèrent par \textit{imitation}\footnote{Le terme de Clark \& \citet{Gerrig1990} est \textit{demonstration} en anglais, mais il est traduit par \textit{imitation} en français (cf. Bonami \& \citet{Godard2008b}).} , contrairement à la plupart des expressions référentielles, dont la référence est établie par \textit{description}. Selon Bonami \& Godard (2008b, 2008c, \textit{à paraître}), dans une citation le locuteur imite le \textit{comportement} d'un individu, ce qui nous permet de rendre compte non seulement des \textit{citations énonciativement fidèles} comme en \REF{ex:2:159}, où la perspective discursive est celle de la source citée (ce qui fait que les indexicaux trouvent leur référence dans la situation de discours décrite par l'incise : p.ex. toute expression pronominale référant à l'agent cité est à la première  personne), mais aussi des \textit{citations énonciativement infidèles} comme en \REF{ex:2:160}, où la perspective discursive est celle du locuteur (par conséquent, les indexicaux trouvent leur référence dans la situation d'énonciation, p.ex. toute expression pronominale référant à l'individu cité est à la troisième personne ; cette fois-ci, un déictique de première personne y renvoie au locuteur). Ce qui nous oblige à supposer deux sous-types de citation est aussi la manière dont l'hôte est pris en charge illocutoirement : dans le cas des citations énonciativement fidèles, le contenu de la citation n'est pas assumé par le locuteur, alors que dans le cas des citations énonciativement infidèles, il l'est (en cela, le dernier type de citation mentionné se rapproche des incises ordinaires). 

Ce qui est commun à ces deux types de citation est le fait que l'hôte présente le type de phrase approprié pour l'acte illocutoire de l'individu cité dans l'incise. Ainsi, dans les deux exemples en \REF{ex:2:159}b et \REF{ex:2:160}b, l'interrogative citée rend compte de la question que se pose Ion (et non le locuteur). 

L'ensemble des phrases incises (ordinaires et de citation) a la même combinatoire, c.-à-d. les incises sont des ajouts incidents qui sélectionnent comme tête la citation. Les arguments en faveur de cette analyse sont, en dehors du détachement prosodique dont on parlait dans la section \ref{sec:2.8.1} (justifiant le terme d'\textit{incident}), la possibilité d'omettre l'incise et sa liberté de placement. 

Contrairement à d'autres phrases incidentes, les incises ne peuvent pas apparaître en début d'énoncé, mais elles peuvent apparaître en début de phrase s'il s'agit d'une phrase coordonnée ou subordonnée. Sinon, ils ont une grande liberté positionnelle, mais, comme le remarquent Cori \& \citet{Marandin1995}, aucun incident ne peut pénétrer un constituant majeur. 


\begin{enumerate}
\item a  (*Se pare,) Trei persoane (,se pare,) au oferit (,se pare,) flori (,se pare,) preşedintelui (,se pare,).


\end{enumerate}
{\itshape
(Paraît-il,) trois personnes (,paraît-il,) ont offert (,paraît-il,) des fleurs (,paraît-il,) au président (,paraît-il)}

  b  S-a născut într-o familie săracă din Iaşi \textbf{şi}, pare-se, n-a prea avut o copilărie fericită.

{\itshape
Il est né dans une famille pauvre et, paraît-il, n'a pas eu une enfance heureuse}

  c  Mă cert mereu cu el, \textbf{deşi}, pare-se, mă place. 

{\itshape
J'ai toujours des querelles avec lui, quoique, paraît-il, je lui plais}


\begin{enumerate}
\item a  (*Adaugă Preşedintele,)\footnote{En position initiale, on préfère le positionnement préverbal du sujet (p.ex. \textit{Preşedintele adaugă...}). Bonami \& \citet{Godard2008b} font la différence entre le discours direct (i)a et l'incise de citation (i)b. Sur la base des arguments empiriques, ils analysent la citation dans le premier cas comme le complément d'un verbe de citation (cf. le schéma classique sujet-tête-complément), alors que, dans le deuxième cas, la citation en entier est la tête, et la séquence contenant le verbe de citation est un ajout incident.  
(i)  a  Paul dit: {\guillemotleft} Qu'est-ce que je fais ? {\guillemotright}
  b  {\guillemotleft} Qu'est-ce que je fais {\guillemotright}, dit Paul.} Situația precară a funcționarilor (,adaugă Preşedintele,) se va rezolva (,adaugă Preşedintele,) înainte de 2012 (,adaugă Preşedintele).


\end{enumerate}
{\itshape
(Ajoute le Président,) La situation précaire des fonctionnaires (,ajoute le Président,) sera résolue (,ajoute le Président,) avant 2012 (,ajoute le Président)}

  b  Ioan Căprar este în pensie de boală \textbf{şi}, spune primarul comunei, nu semnează procesele verbale ale şedințelor de frică să nu-şi piardă pensia.

{\itshape
Ioan Căprar est à la retraite et, dit le maire de la commune, ne signe pas les procès verbaux des réunions, de crainte qu'il ne perde son allocation}

  c  Elevul în cauză nu comunica cu ceilalți colegi, \textbf{deşi}, spune directorul şcolii, aceştia au încercat să-l ajute să se integreze.

    \textit{L'élève en question ne communiquait pas avec les autres collègues, quoique, dit le directeur de l'école, ceux-ci avaient essayé de l'aider à s'intégrer   } 

\paragraph[Les incidents introduits par un subordonnant]{Les incidents introduits par un subordonnant}
Les constituants incidents peuvent être des subordonnées introduites par un complémenteur \REF{ex:2:164}a ou une expression \textit{qu-}. En particulier, on mentionne ici les ajouts {\guillemotleft}~reportifs~{\guillemotright} (cf. \citet{Desmets2001}) en \textit{cum} et \textit{după cum} `comme' \REF{ex:2:164}b-c, et les relatives non-restrictives \REF{ex:2:164}d-e, comme en particulier les ajouts relatifs averbaux \REF{ex:2:164}f-g, qu'on étudie en détail dans le chapitre 5.  


\begin{enumerate}
\item \label{bkm:Ref275532759}a  Medicamentul acesta, \textbf{dacă} nu mă înşel, se ia de două ori pe zi.


\end{enumerate}
{\itshape
Ce médicament, si je ne me trompe pas, se prend deux fois par jour}

  b  Vă respect, \textbf{cum} spuneam şi mai devreme, dar asta nu înseamnă că sunt de acord cu tot ce spuneți.

{\itshape
Je vous respecte, comme je le disais plus tôt, mais cela ne signifie pas que je suis d'accord avec tout ce que vous dîtes}

  c  Povestea s-a terminat, \textbf{după cum} povesteşte ea însăşi, cu o {\guillemotleft}~lovitură de teatru~{\guillemotright}.

{\itshape
L'histoire s'est terminée, comme elle raconte elle-même, avec un {\guillemotleft}~coup de théâtre~{\guillemotright}}

  d  Fata, \textbf{care} nu mai ştia ce să facă, a sunat Poliția.

{\itshape
La fille, qui ne savait plus quoi faire, a appelé la Police}

  e  Ion a început să dea în ea, \textbf{ceea ce} m-a făcut să chem poliția.

{\itshape
Ion a commencé à la frapper, ce qui m'a déterminée à appeler la police}

  f  La  întâlnire  au  venit  trei  persoane,  \textbf{printre}  \textbf{care}  (şi)  Maria. 

  à  rendez-vous\textsc{  aux } venu  trois  personnes,  parmi  lesquelles  (aussi)  Maria

{\itshape
Au rendez-vous, trois personnes sont venues, parmi lesquelles Maria (aussi)}

  g  Au  venit  trei  persoane,  \textbf{dintre  care}  una  ieri. 

\textsc{    aux } venu  trois  personnes,  parmi  lesquelles  une  hier

    \textit{Plusieurs personnes sont venues, dont une hier}  

Toutes ces subordonnées sont des ajouts incidents, en vertu des mêmes propriétés qu'on avait mentionnées pour les incises. En revanche, contrairement aux incises et aux subordonnées relatives, les autres subordonnées en \REF{ex:2:164}a-b-c peuvent être en début absolu d'énoncé. C'est ce qui distingue plus précisément les deux types d'ajouts de citation : les incises de citation mentionnées plus haut et les subordonnées {\guillemotleft}~reportives~{\guillemotright} en \textit{cum / după cum} `comme'.\footnote{Pour une analyse détaillée des ressemblances et des différences qui existent entre les incises de citation et les subordonnées {\guillemotleft} reportives~{\guillemotright} en \textit{comme} en français, voir Bonami \& Godard (\textit{à paraître}).}  

\paragraph[Les incidents introduits par une conjonction]{Les incidents introduits par une conjonction}
\label{bkm:Ref301809092}Finalement, je mentionne les phrases incidentes introduites par une conjonction, que Abeillé (2005) appelle \textit{conjoints incidents} \REF{ex:2:165}. Selon elle, les conjoints incidents se regroupent en trois sous-types majeurs, en fonction de la constitution du syntagme qui suit la conjonction dans le syntagme conjoint : (i) les conjoints en incise, qui contiennent une phrase de type {\guillemotleft}~prédicative~{\guillemotright} \REF{ex:2:165}a, {\guillemotleft}~modale~{\guillemotright} \REF{ex:2:165}b ou {\guillemotleft}~énonciative~{\guillemotright} \REF{ex:2:165}c (cf. la distinction établie par \citet{Marandin1999}) ; (ii) les conjoints {\guillemotleft}~emphatiques~{\guillemotright} \REF{ex:2:165}d, qui contiennent un modifieur ou un complément optionnel sans équivalent dans la première phrase, et (iii) les conjoints {\guillemotleft}~différés~{\guillemotright} \REF{ex:2:165}e, qui contiennent un syntagme mis en relation avec un syntagme parallèle dans la première phrase. 


\begin{enumerate}
\item \label{bkm:Ref301679065}a  Marea problemă, \textbf{şi} e trist s-o spun chiar eu, este că nu mi-am făcut datoria.


\end{enumerate}
{\itshape
Le grand problème, et c'est triste que je le dise moi-même, est que je n'ai pas fait mon devoir}

  b  Ioana e cam arogantă, \textbf{sau} poate mă înşel eu.

{\itshape
Ioana est un peu arrogante, ou peut-être je me trompe}

  c  Marea problemă, \textbf{şi} cu asta vreau să închei, este că nu oferiți nicio soluție viabilă.

{\itshape
Le grand problème, et je conclus avec ça, est que vous n'offrez aucune solution viabl}

  d  Medicina nu este gratuită, ea costă, \textbf{şi} încă foarte mult.

{\itshape
La médecine n'est pas gratuite, elle coûte, et beaucoup même}

  e  Va putea veni ION în locul tău, \textbf{sau} poate MaRIa.

{\itshape
C'est Ion qui pourrait venir à ta place, ou peut-être Maria}

Bien qu'ils soient introduits par une conjonction, les conjoints incidents ne peuvent pas être analysés en termes d'itération syntaxique comme c'était le cas pour les coordinations ordinaires. Les conjoints incidents se distinguent généralement des conjoints ordinaires par les propriétés suivantes : ils ont une prosodie incidente (non intégrée à la prosodie du reste de la phrase, avec un décrochement mélodique), ils sont plus mobiles que les conjoints ordinaires qui ont un positionnement fixe, ils ne sont pas compatibles avec des items corrélatifs (conjonctions {\guillemotleft}~doubles~{\guillemotright} ou adverbes corrélatifs), ils apparaissent toujours dans une construction binaire (les conjoints ordinaires peuvent apparaître dans une coordination multiple), ils sont optionnels et ils peuvent être employés en phrases indépendantes.\footnote{Je ne prends pas en compte l'argument de Abeillé (2005) et \citet{Mouret2007} concernant la violation de la contrainte d'extraction parallèle (\textit{ATB} \textit{extraction}), car, comme on l'a discuté dans la section \ref{sec:2.4.2}, j'adopte la perspective de \citet{Kehler2002}, c.-à-d. la contrainte d'extraction parallèle est plutôt une contrainte discursive, et non syntaxique.} Sur la base de ces propriétés, Abeillé (2005) analyse le conjoint incident comme un ajout syntaxique, et non comme membre d'une construction coordonnée. Le seul cas problématique est celui des syntagmes conjoints {\guillemotleft}~différés~{\guillemotright} qui se trouvent en fin de phrase (cf. \citet{Mouret2007} pour le français). Les séquences soulignées en \REF{ex:2:166}a-b sont clairement des ajouts incidents (car position mobile, prosodie incidente, etc.). Mais la même séquence en \REF{ex:2:166}c, cette fois-ci en fin de phrase, se prête à une double analyse : ajout incident ou membre d'une coordination canonique. L'argument invoqué par \citet{Mouret2007} pour justifier le rapprochement des coordinations canoniques est la compatibilité avec les conjonctions {\guillemotleft}~doubles~{\guillemotright} \REF{ex:2:166}d. Le même argument s'applique au roumain avec d'autres éléments corrélatifs \REF{ex:2:167}.


\begin{enumerate}
\item \label{bkm:Ref274176341}a  Les socialistes, \textbf{et} les verts aussi, ont appelé à une grève générale.


\end{enumerate}
  b  Les socialistes ont, \textbf{et} les verts aussi, appelé à une grève générale.

  c  Les socialistes ont appelé à une grève générale, \textbf{et} les verts aussi.

  d  Demain, \textbf{ou bien} les socialistes seront présents, \textbf{ou bien} les verts.


\begin{enumerate}
\item \label{bkm:Ref301718240}a  (\textbf{Nici})  Ion  nu  m-ajută,  \textbf{şi}  \textbf{nici}  Maria.


\end{enumerate}
\textsc{correl}  Ion  \textsc{neg } m'aide,  et  \textsc{correl}  Maria 

{\itshape
Ion ne m'aide pas, et Maria non plus}

  b  (\textbf{Și})  medicii,  \textbf{dar}  \textbf{şi}  studenții,  au  protestat  ieri.

    \textsc{correl}  les-médecins,  mais  \textsc{correl}  les-étudiants,  ont  protesté  hier

    \textit{Les fonctionnaires, et les étudiants aussi, ont protesté hier}

\subsubsection{L'incidence et d'autres facteurs linguistiques}
\label{bkm:Ref302039778}Contrairement à ce qu'on dit souvent, l'incidence n'est pas nécessairement corrélée à d'autres propriétés (cf. Bonami \& \citet{Godard2008a}). S'il y a des corrélations entre cette propriété prosodique et d'autres facteurs linguistiques, ces corrélations ne valent pas pour toutes les constructions incidentes. 

Au niveau syntaxique, en ce qui concerne la mobilité des incidents, on observe que certains incidents ont une distribution moins libre que d'autres : d'une part, certains incidents ne peuvent pas apparaître en début absolu de phrase (p.ex. les incises par rapport aux ajouts {\guillemotleft}~reportifs~{\guillemotright} ou encore les conjoints incidents, qui ne peuvent pas être en tête de phrase) ; d'autre part, certains incidents sont toujours des ajouts à l'ensemble de la phrase, indépendamment de leur position syntaxique, alors que d'autres sont des ajouts soit à la phrase, soit à un constituant de celle-ci (p.ex. les conjoints en incise sont forcément des ajouts à la phrase, alors que les conjoints {\guillemotleft}~emphatiques~{\guillemotright} sont des ajouts au verbe principal). 

Au niveau sémantique, quant au statut illocutoire des incidents, on a souvent considéré qu'il y avait un lien entre l'incidence et les constituants parenthétiques. Selon Jayez \& \citet{Rossari2004} et \citet{Potts2005}, le contenu sémantique d'un élément parenthétique n'est pas intégré au contenu {\guillemotleft}~principal~{\guillemotright} asserté de la phrase dans laquelle il apparaît, il constitue donc une implicature conventionnelle. Par conséquent, il n'entre pas dans les conditions de vérité de la proposition dénotée par la phrase hôte, et inversement, la contribution de la phrase hôte est indépendante de celle véhiculée par le parenthétique. Le matériel parenthétique reçoit ainsi le statut de {\guillemotleft}~commentaire~{\guillemotright} (ou \textit{at-issue content}, cf. \citet{Potts2005}) par rapport au contenu principal asserté d'un énoncé ou par rapport à l'acte illocutoire de l'ensemble. En termes dialogiques, cela revient à dire que, contrairement à ce qui se passe dans une assertion, un parenthétique n'est pas mis en discussion : le locuteur s'engage sur la vérité de la proposition sans demander l'assentiment de son interlocuteur (le parenthétique est un {\guillemotleft}~engagement solitaire~{\guillemotright}, cf. Bonami \& \citet{Godard2008d}). C'est ce qui explique pourquoi les moyens habituellement employés dans le dialogue pour nier une assertion ne peuvent pas être utilisés avec un parenthétique : l'interlocuteur ne peut pas remettre en cause le contenu parenthétique en utilisant \textit{c'est faux}. On obtient ainsi une différence illocutoire entre les deux types d'incidents {\guillemotleft}~citationnels~{\guillemotright} : les incises de citation ont un contenu non-parenthétique (car le contenu de l'incise correspond à l'acte principal) \REF{ex:2:168}, alors que les ajouts {\guillemotleft}~reportifs~{\guillemotright} en \textit{după cum} `cum' ont un contenu parenthétique \REF{ex:2:169}.  


\begin{enumerate}
\item \label{bkm:Ref275769598}A :  - Ați mâncat bine ?, a întrebat ospătarul. 


\end{enumerate}
{\itshape
Avez-vous bien mangé, a demandé le serveur}

  B :  - Fals, nu ospătarul a pus întrebarea.

{\itshape
C'est faux, ce n'est pas le serveur qui a posé la question}


\begin{enumerate}
\item \label{bkm:Ref275769602}A :  - După cum a întrebat şi Paul, ştiți cumva unde pot vedea filmul {\guillemotleft}~Lolita~{\guillemotright} ? 


\end{enumerate}
{\itshape
Comme avait demandé Paul aussi, savez-vous où je peux voir le film {\guillemotleft}~Lolita~{\guillemotright}}

  B :  - \#Fals, nu Paul a întrebat.

    \textit{C'est faux, ce n'est pas Paul qui a posé la question}

Un autre exemple qui montre le manque de corrélation biunivoque entre l'incidence et le fonctionnement parenthétique concerne les subordonnées relatives non-restrictives. Bien que le contenu de la plupart des modifieurs non-restrictifs soit parenthétique, le contenu des ajouts relatifs averbaux ne l'est pas (voir plus de détails dans la section \ref{sec:5.3.1} du chapitre 5). Les conjoints incidents ne sont pas forcément non plus des parenthétiques, car le contenu des conjoints {\guillemotleft}~emphatiques~{\guillemotright} et {\guillemotleft}~différés~{\guillemotright} peut faire une contribution très importante au contenu de l'acte illocutoire principal.

Pour conclure, les constituants incidents sont distingués des constituants intégrés en vertu d'une propriété prosodique qui les situe à part du reste des constituants. Ils peuvent être introduits par un subordonnant, une conjonction ou rien. Les ajouts incidents ont généralement une distribution plus libre que leurs contreparties non-incidentes et présentent d'autres propriétés spécifiques. On doit cependant conclure que l'incidence n'a pas de corrélat biunivoque avec d'autres facteurs linguistiques. Quant on parle de l'incidence, il faut procéder à une distinction claire entre les propriétés prosodiques, syntaxiques et sémantiques des éléments analysés comme incidents.\footnote{Dans la littérature, on trouve au moins deux corrélations supplémentaires. On considère que l'incidence peut avoir des conséquences sur les relations de portée~(les ajouts incidents ont portée large sur les ajouts non-incidents et portée libre entre eux, cf. Bonami \& \citet{Godard2008a}) ou sur le statut informationnel des incidents (voir discussion dans \citet{Maekawa2007}). }  

\subsection{La coordination symétrique avec \textit{iar} `et' en roumain }
\label{bkm:Ref302035451}\label{bkm:Ref300182216}Dans la section \ref{sec:2.4}, on a vu que la coordination n'impose pas un parallélisme strict et que les symétries ou les asymétries qui apparaissent dans une construction coordonnée peuvent apparaître à plusieurs niveaux d'analyse linguistique (niveau morpho-syntaxique ou sémantico-discursif). En particulier, dans la section \ref{sec:2.4.2}, on a fait la distinction entre une interprétation symétrique et une interprétation asymétrique de la coordination, en fonction de la relation discursive qui s'établit entre les conjoints. Selon \citet{Abraham2006}, Büring (2007), etc., la coordination symétrique par excellence est celle qui met en jeu une relation discursive de parallélisme et contraste. Dans cette section, j'étudie le comportement de la conjonction \textit{iar} `et', spécialisée en roumain pour ce type de relations discursives. L'étude des contraintes spécifiques qui pèsent sur cette conjonction est nécessaire pour mieux comprendre le fonctionnement des coordinations à gapping qu'on étudiera dans le chapitre 4. Cette section s'appuie sur les résultats figurant dans Bîlbîie \& \citet{Winterstein2011}.

\subsubsection{Présentation générale}
\label{bkm:Ref302038810}Dans une perspective typologique, le roumain se distingue des autres langues romanes et se rapproche des langues slaves par l'emploi d'une conjonction spéciale \textit{iar} `et' dans des contextes {\guillemotleft}~contrastifs~{\guillemotright}, dont le correspondent est \textit{a} en biélorusse, bulgare, polonais, russe, serbo-croate (\citet{Niculescu1965}, \citet{Mauri2008}, Jasinskaja \& \citet{Zeevat2009}, \citet{Repp2009}). Je reprends en \REF{ex:2:170} le tableau que j'ai donné en \REF{ex:2:23}.  


\begin{enumerate}
\item   \label{bkm:Ref301962063}L'espace des conjonctions en quatre langues


\end{enumerate}

\begin{table}


\begin{tabular}{llll}

 {\bfseries Français}\par & {\bfseries Espagnol}\par & {\bfseries Roumain}\par & {\bfseries Russe}\par\\
 {\itshape et}\par & {\itshape y}\par & {\itshape şi}\par & {\itshape i}\par\\
 &  & {\bfseries\itshape iar}\par & {\itshape a}\par\\
 {\itshape mais}\par & {\itshape pero}\par &  & \\
 &  & {\itshape dar}\par & {\itshape no}\par\\
\hhline{~---} & {\itshape sino}\par & {\itshape ci}\par & {\itshape a}\par\\
\hhline{~---}

\end{tabular}

\caption{}
%\label{}
\end{table}

Dans la tradition grammaticale du roumain, on considère que la conjonction \textit{iar} a un statut intermédiaire entre une interprétation additive et une interprétation adversative (\citet{Niculescu1965}, \citet{Zafiu2005}, \textit{GALR} (2005)). Ainsi, dans la plupart des exemples la conjonction \textit{iar} peut être remplacée par la conjonction \textit{şi} `et' \REF{ex:2:171} et parfois par la conjonction \textit{dar} `mais' \REF{ex:2:172}. 


\begin{enumerate}
\item \label{bkm:Ref301963738}a  Ioana citeşte o carte, \{\textbf{iar {\textbar} şi}\} Maria vorbeşte la telefon.


\end{enumerate}
{\itshape
Ioana lit un livre, \{IAR {\textbar} et\} Maria parle au téléphone} 

  b  Ioana vine azi, \{\textbf{iar {\textbar} şi}\} Maria mâine.

{\itshape
Ioana vient aujourd'hui, \{IAR {\textbar} et\} Maria demain}


\begin{enumerate}
\item \label{bkm:Ref301963751}a  Eu îi fac toate mofturile, \{\textbf{iar {\textbar} dar}\} el îşi bate joc de mine.


\end{enumerate}
{\itshape
Je lui fais tous les caprices, \{IAR {\textbar} mais\} il se moque de moi}

  b  Afară sunt 40 de grade, \{\textbf{iar {\textbar} dar}\} Ion e îmbrăcat cu pulovăr de iarnă.

{\itshape
Dehors il fait 40 degrés, \{IAR {\textbar} mais\} Ion est habillé avec un pull d'hiver} 

Cependant, contrairement à la conjonction \textit{şi} `et' qui peut lier des conjoints plus ou moins symétriques ou encore la conjonction \textit{dar} `mais' qui, dans la plupart des occurrences, est asymétrique, la coordination avec \textit{iar} ne peut pas entraîner des relations discursives asymétriques (donc, pas d'effets de cause \REF{ex:2:173}a ou de conséquence \REF{ex:2:173}b) ; la relation discursive qui s'établit entre les conjoints coordonnés par \textit{iar} est toujours symétrique. Par conséquent, l'ordre des conjoints est interchangeable, sans changer les conditions de vérité de la phrase. Contrairement à la conjonction \textit{dar} `mais', \textit{iar} marque le contraste sémantique non-orienté. 


\begin{enumerate}
\item \label{bkm:Ref301964614}a  A : - Cum de-ți merge maşina ? B : - Am pus pur şi simplu cheia în contact \{şi {\textbar} \#iar\} imediat maşina a pornit.


\end{enumerate}
{\itshape
A : - Comment tu as réussi à faire démarrer ta voiture ? B : - J'ai simplement mis le contact \{et {\textbar} IAR\} tout de suite la voiture a démarré}

  b  A : - De ce eşti trist ? B : - Am spart o vază foarte scumpă \{şi {\textbar} \#iar\} mama m-a pedepsit.

\textit{A : - Pourquoi tu es triste ? B : - J'ai cassé un vase très cher \{et {\textbar} IAR\} ma mère m'a puni}  

\subsubsection{La sémantique de \textit{iar}}
\label{bkm:Ref302038000}La contrainte majeure sur les coordinations avec \textit{iar} est la condition du double contraste : les conjoints coordonnés par \textit{iar} doivent contenir au moins deux paires contrastives \REF{ex:2:174}a. D'une part, chaque conjoint doit fournir un élément pour chaque paire (avec la mention qu'un des éléments d'une paire peut être implicite dans le premier conjoint \REF{ex:2:174}b). D'autre part, les éléments d'une paire contrastive doivent appartenir au même ensemble d'alternatives, mais ils doivent être différents (cf. \citet{Zeevat2004}, \citet{Umbach2005}). Par conséquent, \textit{iar} n'est pas licite dans les coordinations avec une seule paire contrastive (p.ex. \REF{ex:2:174}c, où l'on a une seule paire contrastive {\textless}\textit{un măr, o pară}{\textgreater}) ou encore dans les contextes dans lesquels la paire est non-contrastive (p.ex. la paire {\textless}\textit{un măr, un fruct}{\textgreater} en \REF{ex:2:174}d, qui est une paire de type {\textless}hyperonyme, hyponyme{\textgreater}\footnote{Ce type de paire devient une paire contrastive et donc licite avec \textit{iar} uniquement si les éléments de la paire donnent lieu à une implicature de quantité (cf. \citet{Geurts2010}). Comparer la paire {\textless}\textit{un măr, un fruct}{\textgreater} dans \REF{ex:2:174}d avec la paire {\textless}\textit{toate, câteva}{\textgreater} dans (i). Dans l'exemple (i), on exhaustifie le membre le plus faible de la paire (c.-à-d. \textit{quelques-unes} = `pas toutes les questions') et on obtient ainsi le contraste dont on a besoin pour rendre licite \textit{iar}. Pour une explication plus détaillée, consulter Bîlbîie \& \citet{Winterstein2011}.
(i)  Paul a răspuns la toate întrebările, \textbf{iar} Maria la câteva.
  Paul a répondu à toutes les questions, IAR Maria à quelques-unes}). En revanche, en \REF{ex:2:174}a, les conditions sont respectées : chaque conjoint fournit un élément pour la constitution d'une paire et chaque paire est contrastive : {\textless}\textit{Ioana, Maria}{\textgreater} et {\textless}\textit{un măr, o pară}{\textgreater}).~


\begin{enumerate}
\item \label{bkm:Ref302038976}a  Ioana mănâncă un măr, \textbf{iar} Maria o pară.


\end{enumerate}
{\itshape
Ioana mange une pomme, IAR Maria (mange) une poire}

  b  Ioana mănâncă (mai întâi) un măr, \textbf{iar} apoi o pară.

{\itshape
Ioana mange (d'abord) une pomme, IAR ensuite une poire}

  c  *Ioana mănâncă un măr, \textbf{iar} o pară.

    \textit{Ioana mange une pomme, IAR une poire}

  d  ??Ioana a mâncat un măr, \textbf{iar} Maria un fruct.

    \textit{Ioana a mangé une pomme, IAR Maria un fruit}

\subsubsection{La structure informationnelle de \textit{iar} }
On observe que, dans ses emplois contrastifs, \textit{iar} semble être en variation libre avec la conjonction \textit{şi} `et'. Le but de cette section est de montrer les contraintes discursives qui pèsent sur le conjoint introduit par \textit{iar} en roumain et qui ne s'appliquent pas au conjoint introduit par la conjonction \textit{şi} `et'. Ainsi, en \REF{ex:2:175} et \REF{ex:2:176}, bien que les deux conjonctions soient possibles, les locuteurs manifestent une préférence forte pour l'emploi de la conjonction \textit{iar} à la place de \textit{şi}\footnote{Le signe d'interrogation dans les exemples qui suivent indique tout simplement un emploi moins naturel de la conjonction dans le contexte en question. } . Descriptivement, la différence majeure entre \REF{ex:2:175} et \REF{ex:2:176} réside dans l'ordre relatif des lieux (p.ex. Paris et Rome) et des dates (p.ex. 1\textsuperscript{er} et 10 juillet). La position préférée de l'élément qui résout la question est à la fin du conjoint, alors que l'élément présent déjà dans la question suit immédiatement la conjonction \textit{iar}.


\begin{enumerate}
\item \label{bkm:Ref301969812}A  - Unde vei fi pe 1 şi pe 10 iulie 2011 ?


\end{enumerate}
\textit{Où est-ce que tu seras le 1}\textit{\textsuperscript{er}}\textit{ et le 10 juillet 2011} 

  B  - Pe 1 iulie (voi fi) la Paris, \{\textbf{iar {\textbar}} ?\textbf{şi}\} pe 10 iulie (voi fi) la Roma.

\textit{Le 1}\textit{\textsuperscript{er}}\textit{ juillet (je serai) à Paris, \{IAR {\textbar} et\} le 10 juillet (je serai) à Rome}


\begin{enumerate}
\item \label{bkm:Ref301969814}A  - Când vei ajunge la Paris şi când la Roma ?


\end{enumerate}
{\itshape
Quand est-ce que tu arriveras à Paris et quand à Rome}

  B  - La Paris (voi ajunge) pe 1 iulie, \{\textbf{iar {\textbar}} ?\textbf{şi}\} la Roma (voi ajunge) pe 10.

\textit{A Paris (j'arriverai) le 1}\textit{\textsuperscript{er}}\textit{ juillet, \{IAR {\textbar} et\} à Rome (j'arriverai) le 10} 

Cette hypothèse se vérifie sur d'autres exemples aussi. Ainsi, pour une séquence comme \REF{ex:2:177}, on peut utiliser deux questions ouvertes différentes, qui rendent explicite la structure informationnelle de l'énoncé qui résout la question. 


\begin{enumerate}
\item \label{bkm:Ref301970680}La film cu Ioana, la teatru cu Maria. 


\end{enumerate}
{\itshape
  Au cinéma avec Ioana, au théâtre avec Maria}


\begin{enumerate}
\item \label{bkm:Ref301971540}[\textit{Contexte : le locuteur A sait que le locuteur B va et au cinéma et au théâtre}]


\end{enumerate}
  A~  - Cu cine vei merge la film şi cu cine la teatru ?

{\itshape
Avec qui tu iras au cinéma et avec qui au théâtre}

  B\textsubscript{1}  - La film cu Ioana, \textbf{iar} la teatru cu Maria.

{\itshape
Au cinéma avec Ioana, IAR au théâtre avec Maria}

  B\textsubscript{2}  - Cu Ioana la film \textbf{şi} cu Maria la teatru.

{\itshape
Avec Ioana au cinéma et avec Maria au théâtre} 


\begin{enumerate}
\item \label{bkm:Ref301971542}[\textit{Contexte : le locuteur A sait que le locuteur B a deux filles, Ioana et Maria, et qu'il sort avec chacune dans un endroit différent}]


\end{enumerate}
  A~  - Unde ieşi cu copiii weekendul ăsta ?

{\itshape
Où est-ce que tu sors avec tes enfants ce weekend}

  B\textsubscript{1}  - Cu Ioana la film, \textbf{iar} cu Maria la teatru.

{\itshape
Avec Ioana au cinéma, IAR avec Maria au théâtre}

  B\textsubscript{2}  - La film cu Ioana \textbf{şi} la teatru cu Maria.

{\itshape
Au cinéma avec Ioana et au théâtre avec Maria} 

Pour décrire les différences qu'on observe entre \REF{ex:2:178} et \REF{ex:2:179}, on fait appel aux notions de \textit{topique} \textit{contrastif} et \textit{focus informationnel} de Büring (2003) : le constituant qui résout la question est identifié comme étant un focus informationnel (habituellement, il est marqué par un contour prosodique spécifique), alors que l'élément qui vient d'être mentionné ou qui est saillant dans le discours correspond à un topique contrastif. Revenant aux coordinations avec \textit{iar}, l'hypothèse qui résulte des variations d'ordre observées dans les exemples \REF{ex:2:175}-\REF{ex:2:176} et \REF{ex:2:178}-\REF{ex:2:179} est la suivante : la conjonction \textit{iar} doit être immédiatement suivie par un topique contrastif et non par un focus informationnel. Cette hypothèse se vérifie facilement par le placement de l'accent prosodique (marqué par des lettres majuscules dans les exemples ci-dessous), qui force l'identification du focus informationnel. Dans les exemples en \REF{ex:2:180}, le premier élément suivant \textit{iar} ne peut pas recevoir l'accent prosodique (il ne peut pas être donc un focus informationnel). En revanche, la conjonction \textit{şi} ne semble pas être si contrainte que \textit{iar} en ce qui concerne le statut informationnel de l'élément qui la suit immédiatement (voir \REF{ex:2:180}b).


\begin{enumerate}
\item \label{bkm:Ref301977126}a  La film voi merge cu \textsc{Ioa}na, \{\textbf{iar {\textbar}} şi\} la teatru (voi merge) cu Ma\textsc{ria}.


\end{enumerate}
{\itshape
Au cinéma j'irai avec Ioana, \{IAR {\textbar} et\} au théâtre (j'irai) avec Maria}

  b  Cu I\textsc{oa}na merg la film, \{\textbf{şi} {\textbar} \#\textbf{iar}\} cu Ma\textsc{ria} (merg) la teatru.

{\itshape
Avec Ioana je vais au cinéma \{et {\textbar} IAR\} avec Maria (je vais) au théâtre}

  c  Cu I\textsc{oa}na merg la film, \{\textbf{iar} {\textbar} ?\textbf{şi}\} la teatru (merg) cu Ma\textsc{ria}.

{\itshape
Avec Ioana je vais au cinéma \{IAR {\textbar} et\} au théâtre (je vais) avec Maria} 

Cette contrainte liée à la structure informationnelle explique pourquoi les indéfinis sont moins acceptables en première position après la conjonction \textit{iar} (voir la réponse B\textsubscript{2} en \REF{ex:2:181} et la réponse B\textsubscript{1} en \REF{ex:2:182}). Selon la contrainte mentionnée précédemment, une expression indéfinie qui suit immédiatement la conjonction \textit{iar} doit être interprétée comme appartenant au fond (angl. \textit{background}) de l'énoncé. Or, cela est en conflit avec le fait qu'on choisit habituellement comme moyen de référer à une entité déjà introduite dans le discours une description définie, et non une description indéfinie. Les contextes naturels avec \textit{iar} et une expression indéfinie sont les contextes avec des indéfinis génériques \REF{ex:2:183}. 


\begin{enumerate}
\item \label{bkm:Ref301979316}A  - Ce i-ai oferit Mariei şi ce i-ai oferit Ioanei ?


\end{enumerate}
{\itshape
Qu'est-ce que tu as offert à Maria et qu'est-ce que tu as offert à Ioana}

  B\textsubscript{1}  - Mariei (i-am oferit) o carte, \{\textbf{iar} {\textbar} şi\} Ioanei un stilou.

{\itshape
A Maria (j'ai offert) un livre \{IAR {\textbar} et\} à Ioana un stylo}

  B\textsubscript{2}  - (I-am oferit) o \textsc{car}te Mariei \{\textbf{şi} {\textbar} \#\textbf{iar}\} un stil\textsc{ou} Ioanei.

{\itshape
(J'ai offert) un livre à Maria \{et {\textbar} IAR\} un stylo à Ioana} 


\begin{enumerate}
\item \label{bkm:Ref301979318}A  - Cui i-ai oferit o carte şi cui i-ai oferit un stilou ?


\end{enumerate}
{\itshape
A qui as-tu offert un livre et à qui as-tu offert un stylo}

  B\textsubscript{1}  - I-am oferit o carte Ma\textsc{ri}ei \{\textbf{şi} {\textbar} ?\textbf{iar}\} un stilou I\textsc{oa}nei.\footnote{L'emploi de \textit{iar} dans ce contexte est amélioré si l'expression indéfinie \textit{un stilou} `un stylo' est interprétée comme \textit{unul dintre stilouri} `un des stylos' ou bien si son corrélat dans le premier conjoint (p.ex. \textit{o carte} `un livre') est isolé prosodiquement à la périphérie gauche de la phrase :
(i)  O carte, i-am oferit(-o) Ma\textsc{ri}ei, \{\textbf{şi} / \textbf{iar}\} un stilou, I\textsc{oa}nei.
  \textit{Un livre, je l'ai offert à Maria, \{et {\textbar} IAR\} un stylo, à Ioana}}

{\itshape
J'ai offert un livre à Maria \{et {\textbar} IAR\} un stylo à Ioana} 

  B\textsubscript{2}  - I-am oferit cartea Ma\textsc{ri}ei \{\textbf{iar} {\textbar} şi\} stiloul I\textsc{oa}nei.

{\itshape
J'ai offert le livre à Maria \{IAR {\textbar} et\} le stylo à Ioana} 


\begin{enumerate}
\item \label{bkm:Ref301979294}O casă costă 200 000 de euro, \textbf{iar} o maşină 20 000. 


\end{enumerate}
{\itshape
Une maison coûte 200~000 euros, IAR une voiture 20 000}

Le fait que le premier constituant suivant \textit{iar} doit être un topique contrastif explique aussi l'agrammaticalité de \textit{iar} dans les exemples \REF{ex:2:184}a-b, dans lesquels le syntagme nominal désigné comme topique contrastif est modifié par un adverbe associatif comme \textit{şi} `aussi' ou \textit{nici} `non plus'. Les adverbes associatifs \textit{şi} `aussi' et \textit{nici} `non plus' ont une portée étroite sur leur associé, qui habituellement est un focus informationnel (noté F) dans le discours. Selon l'hypothèse mentionnée précédemment, on ne peut pas employer \textit{iar} dans ces contextes. En revanche, l'adverbial \textit{de asemenea} `de même' en \REF{ex:2:184}c peut avoir portée large sur le deuxième conjoint en entier (grâce à sa mobilité). Par conséquent, la structure informationnelle du deuxième conjoint dans ces trois exemples n'est pas la même : \textit{Maria} reçoit le focus en \REF{ex:2:184}a-b, tandis qu'en \REF{ex:2:184}c il est interprété comme topique contrastif. 


\begin{enumerate}
\item \label{bkm:Ref301981893}a  Ioanei  îi  plac  merele,  \{\textbf{şi} {\textbar} *\textbf{iar}\}  [şi  Mariei]\textsubscript{F}  perele.


\end{enumerate}
Ioana\textsc{.dat  cl}  plaisent  les-pommes,  \{et {\textbar} IAR\}  \textsc{adv}  Maria\textsc{.dat}  les-poires

{\itshape
Ioana aime les pommes \{et {\textbar} IAR\} Maria aussi les poires}

  b  Ioanei  nu-i  plac  merele,  \{\textbf{şi} {\textbar} *\textbf{iar}\}  [nici  Mariei]\textsubscript{F}  perele.

    Ioana\textsc{.dat  neg-cl}  plaisent  les-pommes,  \{et {\textbar} IAR\}  \textsc{adv}  Maria\textsc{.dat}  les-poires

{\itshape
Ioana n'aime pas les pommes \{et {\textbar} IAR\} Maria non plus les poires}

  c  Ioanei  îi  plac  merele,  \textbf{iar}  [Mariei]\textsubscript{CT}  de asemenea.

    Ioana\textsc{.dat  cl}  plaisent  les-pommes,  IAR  Maria\textsc{.dat}  de même

{\itshape
Ioana aime les pommes, IAR Maria aussi} 

Assigner au premier constituant suivant \textit{iar} le statut de topique contrastif rejoint la notion de {\guillemotleft}~clé de tri~{\guillemotright} (angl. \textit{sorting key}) utilisée par \citet{Kuno1982} dans les couples question-réponse, pour identifier l'élément qui indique comment attaquer~la résolution d'une question \REF{ex:2:185}. Cette analyse se rapproche aussi de la notion d'\textit{intégrateur commun} de \citet{Lang1984}. 


\begin{enumerate}
\item \label{bkm:Ref301983241}a  A : - Ce au făcut copiii azi-dimineață ?


\end{enumerate}
{\itshape
Qu'est-ce que les enfants ont fait ce matin} 

  b  B : - Ion şi-a făcut temele, \textbf{iar} Maria a fost la film.

{\itshape
Ion a fait ses devoirs, IAR Maria a été au cinéma}

\subsubsection{La syntaxe de \textit{iar}}
\label{bkm:Ref301983837}Selon les grammaires traditionnelles, la conjonction \textit{iar} lie uniquement des phrases finies. On observe toutefois que la conjonction \textit{iar} peut coordonner (i) des phrases~avec une tête verbale (p.ex. le verbe \textit{doarme} `dort' en \REF{ex:2:186}a), (ii) des phrases avec une tête averbale (p.ex. le syntagme nominal \textit{câtă tristețe} `quelle tristesse' en \REF{ex:2:186}b), ou bien (iii) des phrases elliptiques qu'on analyse comme fragments (p.ex. la phrase à gapping en \REF{ex:2:186}c, où il n'y a pas de tête, cf. chapitre 4), à condition que le premier syntagme suivant \textit{iar} soit un topique contrastif (cf. discussion de la section \ref{sec:2.9.4}). Donc, on va dire que la conjonction \textit{iar} lie toujours des phrases (pas nécessairement finies) dont le contenu sémantique est un sous-type de \textit{message} (cf. Ginzburg \& \citet{Sag2000}).


\begin{enumerate}
\item \label{bkm:Ref301984133}a  Ioana citeşte, \textbf{iar} [[Maria]\textsubscript{NP} doarme]\textsubscript{S}.


\end{enumerate}
{\itshape
Ioana lit, IAR Maria dort}

  b  Să-ți vezi copilul crescând, câtă bucurie, \textbf{iar} [[să-l asişti  murind]\textsubscript{VP}, [câtă tristețe]\textsubscript{NP}]\textsubscript{S~}!

{\itshape
Voir son enfant grandir, quelle joie, IAR l'assister à sa mort, quelle tristesse} 

  c  Ioana mănâncă mere, \textbf{iar} [[Maria]\textsubscript{NP} [pere]\textsubscript{NP}]\textsubscript{S}.

{\itshape
Ioana mange des pommes, et Maria des poires} 

Une contrainte syntaxique majeure, qui est une conséquence de la contrainte discursive discutée précédemment dans la section \ref{sec:2.9.4}, exige que le constituant distingué comme topique contrastif dans le conjoint introduit par \textit{iar} soit un {\guillemotleft}~constituant majeur~{\guillemotright} dans le sens de \citet{Hankamer1971}, c.-à-d. le dépendant d'une tête prédicative, mais pas la tête elle-même. Par conséquent, la première position après \textit{iar} peut correspondre à un sujet \REF{ex:2:187}a, un complément préverbal \REF{ex:2:187}b ou un ajout initial \REF{ex:2:187}c. La tête, s'il y en a, doit suivre le topique contrastif (voir l'impossibilité d'avoir la tête verbale immédiatement après \textit{iar} en \REF{ex:2:187}b).


\begin{enumerate}
\item \label{bkm:Ref301985166}a  Ioana citeşte, \textbf{iar} [Maria]\textsubscript{NP} doarme.


\end{enumerate}
{\itshape
Ioana lit, IAR Maria dort}

  b  I-am dat Ioanei un măr, \textbf{iar} (*i-am dat) [lui Petre]\textsubscript{NP} (i-am dat) o banană.

{\itshape
J'ai donné à Ioana une pomme, IAR (j'ai donné) à Petre (j'ai donné) une banane} 

  c  La mare plouă, \textbf{iar} [la munte]\textsubscript{PP} ninge.

{\itshape
A la mer il pleut, IAR à la montagne il neige} 

Cela nous oblige à postuler une contrainte de précédence linéaire \REF{ex:2:188} qui s'applique aux conjoints introduits par la conjonction \textit{iar} en roumain. Elle dit que la conjonction \textit{iar} introduit une phrase S et précède immédiatement le topique contrastif de cette phrase, qui à son tour précède une liste non-vide d'éléments (y compris la tête, s'il y en a).


\begin{enumerate}
\item \label{bkm:Ref301985533}Contrainte de précédence linéaire 


\end{enumerate}
\textbf{\textit{iar} }\textbf{{\textless}}\textbf{ [}\textbf{XP}\textbf{\textsubscript{CONTRASTIVE TOPIC}}\textbf{} \textbf{{\textless}}\textbf{ nelist(...)]}\textbf{\textsubscript{S}}

Maintenant on a les outils nécessaires pour expliquer l'agrammaticalité de \REF{ex:2:189}, exemple décrit traditionnellement comme une violation liée à la finitude (c.-à-d. un verbe fini ne peut jamais suivre immédiatement la conjonction \textit{iar}). Si l'on compare \REF{ex:2:189} et \REF{ex:2:190}, on observe qu'il ne s'agit pas d'une contrainte liée à la finitude du verbe, mais plutôt d'une contrainte liée à la fonction syntaxique : une forme verbale finie (comme le verbe au subjonctif \textit{ să bea} `qu'il boive' en \REF{ex:2:190}) peut être le premier constituant après \textit{iar}, à condition qu'il ne soit pas la tête dans la phrase en question (en \REF{ex:2:190}, la tête de la phrase est le syntagme adverbial \textit{nici atât} `pas du tout').  


\begin{enumerate}
\item \label{bkm:Ref301985717}*Ninge la Braşov, \textbf{iar} [plouă [la Bucureşti]]\textsubscript{S}. 


\end{enumerate}
{\itshape
Il neige à Braşov, IAR il pleut à Bucarest}


\begin{enumerate}
\item \label{bkm:Ref301985869}Bebeluşul [nu prea vrea [să pape]], \textbf{iar} [să bea] [nici atât]. 


\end{enumerate}
{\itshape
Le petit bébé ne veut pas manger, IAR boire pas du tout}

La conjonction \textit{iar} peut coordonner des phrases racines et des phrases subordonnées. Si \textit{iar} coordonne deux phrases subordonnées, il ne permet pas la réitération du complémenteur dans le deuxième conjoint, contrairement à ce qui se passe avec la conjonction \textit{şi} `et' (comparer \REF{ex:2:191}a et \REF{ex:2:191}b). Cela dérive directement de la contrainte discursive discutée précédemment dans la section \ref{sec:2.9.4}, car rien ne peut s'insérer entre la conjonction \textit{iar} et le topique contrastif. Par conséquent, le complémenteur peut suivre immédiatement \textit{iar} seulement s'il introduit un topique contrastif \REF{ex:2:191}c.


\begin{enumerate}
\item \label{bkm:Ref301986761}a  Mă întreb dacă Ion citeşte \{\textbf{şi} / *\textbf{iar}\} dacă Ana lucrează.


\end{enumerate}
{\itshape
Je me demande si Ion lit \{et {\textbar} IAR\} si Ana travaille}

  b  Mă întreb dacă Ion citeşte, \textbf{iar} Ana lucrează.

{\itshape
Je me demande si Ion lit, IAR si Ana travaille}

  c  De Paşte, mergem la munte, \textbf{iar} [\textbf{dacă} e frumos]\textsubscript{S}, şi la mare.

{\itshape
Aux Paques, nous allons à la montagne, IAR, s'il fait beau, aussi à la mer} 

La même contrainte discursive explique aussi pourquoi il n'y a pas toujours de parallélisme strict entre les conjoints contrastifs coordonnés par \textit{iar}. L'ordre des éléments d'une paire contrastive peut varier d'un conjoint à l'autre, afin d'éviter d'avoir un focus informationnel immédiatement après \textit{iar}.  


\begin{enumerate}
\item a  [Ninge]\textsubscript{F} la Braşov, \textbf{iar} la Bucureşti [plouă]\textsubscript{F}.


\end{enumerate}
{\itshape
Il neige à Braşov, IAR à Bucarest il pleut}

  b  [Cu I\textsc{oa}na]\textsubscript{F} merg la film, \textbf{iar} la teatru [cu Ma\textsc{ri}a]\textsubscript{F}.

{\itshape
Avec Ioana je vais au cinéma, IAR au théâtre avec Maria} 

Pour conclure, on observe que toutes les contraintes de linéarisation (sur le placement de la tête verbale, sur les complémenteurs, sur l'ordre des mots en général) sont dues à une contrainte discursive majeure : le premier constituant suivant \textit{iar} doit être un topique contrastif.

\subsubsection{Formalisation sémantique des conjonctions}
Par la suite, je reprends brièvement les traits utilisés par Jasinskaja \& \citet{Zeevat2009} pour décrire formellement la sémantique des connecteurs majeurs en russe, anglais et allemand et je les applique aux conjonctions correspondantes en roumain (figurant dans le tableau présenté ci-dessus en \REF{ex:2:170}). Les traits définissant chaque conjonction synthétisent les conditions sur le type de question à laquelle répondent les conjoints coordonnés par l'item en question. On obtient ainsi la formalisation sémantique suivante : 


\begin{enumerate}
\item   \label{bkm:Ref302028826}La formalisation sémantique de 4 conjonctions en roumain


\end{enumerate}

\begin{table}


\begin{tabular}{lll}

 {\itshape şi}\par & SINGLE & \\
 {\itshape iar}\par &  & ${\lnot}$SINGLE,${\lnot}$CORRECTION,${\lnot}$(WHETHER,2\textsuperscript{nd})\\
 {\itshape dar}\par & WHETHER, 2\textsuperscript{nd} & ${\lnot}$CORRECTION\\
 {\itshape ci}\par & CORRECTION & \\
\end{tabular}

\caption{}
%\label{}
\end{table}

La conjonction \textit{şi} `et' est décrite comme ayant le trait SINGLE, qui indique que chaque conjoint répond à une question avec un seul élément \textit{qu-}, comme en \REF{ex:2:194}. 


\begin{enumerate}
\item \label{bkm:Ref302029219}A  - Ce face Cristina ?


\end{enumerate}
{\itshape
Qu'est-ce qu'elle fait Cristina}

  B  - Cristina face duş \textbf{şi} vorbeşte la telefon.

{\itshape
Cristina prend une douche et parle au téléphone} 

La négation du trait SINGLE (${\lnot}$SINGLE) indique que chaque conjoint doit être une réponse à une question multiple, avec au moins deux éléments \textit{qu-}, comme en \REF{ex:2:195}. En utilisant ce trait, on rend ainsi compte de la contrainte du double contraste requis dans les coordinations avec \textit{iar}. Si l'on regarde la définition de la conjonction \textit{iar} en \REF{ex:2:193}, on observe qu'elle est décrite uniquement par des négations : elle s'oppose non seulement à la conjonction \textit{şi} par le trait ${\lnot}$SINGLE, mais aussi à la conjonction corrective \textit{ci} `mais' (par le trait ${\lnot}$CORRECTION) et à la conjonction argumentative \textit{dar} `mais' (par le trait ${\lnot}$(WHETHER,2\textsuperscript{nd})).


\begin{enumerate}
\item \label{bkm:Ref302029658}A  - Cine ce face ?


\end{enumerate}
{\itshape
Qui fait quoi}

  B  - Cristina face duş, \textbf{iar} Maria vorbeşte la telefon.

{\itshape
Cristina prend une douche, et Maria parle au téléphone} 

La conjonction adversative \textit{dar} `mais' est décrite par le trait WHETHER, indiquant que chaque conjoint répond à une question polaire (comme en \REF{ex:2:196}) et en même temps par le trait 2\textsuperscript{nd}, indiquant le fait que le deuxième conjoint est plus fort argumentativement que le premier (comparer les conclusions différentes qui résultent des réponses B\textsubscript{1} et respectivement B\textsubscript{2}).


\begin{enumerate}
\item \label{bkm:Ref302030627}A  - Au venit Ion şi Maria ?


\end{enumerate}
{\itshape
Ion et Maria sont venus}

  B  - Ion a venit, \textbf{dar} Maria nu.

{\itshape
Ion est venu, mais Maria non} 


\begin{enumerate}
\item A  - Ce zici : să cumpăr rochia asta sau nu ?


\end{enumerate}
{\itshape
Qu'est-ce que tu en penses : je devrais acheter cette robe ou pas}

  B\textsubscript{1}  - Rochia e frumoasă, \textbf{dar} e cam scumpă.

{\itshape
La robe est belle, mais elle est un peu chère} 

  B\textsubscript{2}  - Rochia e cam scumpă, \textbf{dar} e frumoasă.

{\itshape
La robe est un peu chère, mais elle est belle} 

Finalement, la conjonction \textit{ci} `mais' est décrite essentiellement par le trait CORRECTION, qui est utilisé pour indiquer que le deuxième conjoint est une correction du premier conjoint. La conjonction \textit{ci} se rapproche ainsi de la conjonction corrective \textit{sino} en espagnol ou encore de la conjonction \textit{sondern} en allemand.


\begin{enumerate}
\item Nu vin azi, \textbf{ci} mâine. 


\end{enumerate}
{\itshape
Je ne viens pas aujourd'hui, mais demain}

Si l'on compare les conjonctions du roumain et leurs équivalents en français, on obtient le tableau en \REF{ex:2:199}. La conjonction \textit{et} n'a pas besoin de spécification particulière en ce qui concerne le nombre d'éléments \textit{qu-} dans la question, donc elle couvre simultanément l'espace de \textit{şi} et \textit{iar} en roumain. La conjonction \textit{mais} reçoit presque la même description que la conjonction \textit{dar} en roumain, la seule différence étant l'absence du trait CORRECTION, ce qui permet l'emploi de la conjonction \textit{mais} dans les deux contextes : argumentatif et correctif.  


\begin{enumerate}
\item   \label{bkm:Ref302031147}La formalisation sémantique de \textit{et} et \textit{mais} en français


\end{enumerate}

\begin{table}


\begin{tabular}{lll}

 {\itshape et}\par &  & ${\lnot}$(WHETHER,2\textsuperscript{nd})\\
 {\itshape mais}\par & WHETHER,2\textsuperscript{nd} & \\
\end{tabular}

\caption{}
%\label{}
\end{table}

Pour conclure, dans cette section j'ai étudié le comportement de la conjonction \textit{iar}, spécialisée pour le contraste sémantique non-orienté en roumain. Deux contraintes générales s'imposent sur les conjoints coordonnés par \textit{iar~}: du point de vue sémantique, la conjonction \textit{iar} relie des conjoints qui contiennent au moins deux paires contrastives ; du point de vue de la structure informationnelle, le constituant qui suit immédiatement \textit{iar} doit être un topique contrastif. La description formelle du fonctionnement sémantique des conjonctions en roumain nous montre que la conjonction \textit{iar} a un comportement différent de la conjonction additive \textit{şi}, de la conjonction adversative \textit{dar} ou encore de la conjonction corrective \textit{ci}. Par conséquent, toute approche qui essaie d'inclure la conjonction \textit{iar} soit dans la classe des additifs, soit dans la classe des adversatifs est réductionniste. 

\subsection{La structure interne des coordinations}
\label{bkm:Ref302035599}Je m'intéresse dans cette section à l'analyse formelle des constructions coordonnées, suivant en particulier les travaux de Sag (2003, 2005), Abeillé (2003, 2005), Mouret (2006, 2007) et Bîlbîie (2008). Le modèle que je retiens ici est une version constructionnelle de HPSG (\citet{Sag1997}, Ginzburg \& \citet{Sag2000}, Sag \textit{et al.} (2003), Abeillé (2007)). 

Cette section est organisée de la manière suivante. D'abord, je discute l'organisation générale d'une coordination, en montrant qu'une structure hiérarchique rend mieux compte des propriétés combinatoires des conjonctions qu'une structure plate (section \ref{sec:2.10.1}). Ensuite, je reprends brièvement quelques arguments qui montrent que la conjonction forme un constituant avec la séquence qui suit (au moins, dans les langues à tête initiale), cf. section \ref{sec:2.10.2}. Dans la même section, je m'intéresse aux relations fonctionnelles qui s'établissent entre la conjonction et le constituant avec lequel elle se combine, en adoptant une structure endocentrique du syntagme conjoint. Finalement, j'étudie le comportement syntaxique de la structure coordonnée dans son ensemble (section \ref{sec:2.10.3}) : après avoir observé que la structure coordonnée ne peut pas être réduite aux syntagmes ordinaires endocentriques, je donne les règles des différents types de structures coordonnées et je montre comment on peut traiter la coordination de termes dissemblables en faisant appel au mécanisme de sous-spécification.  

\subsubsection{Une structure hiérarchique de la coordination} 
\label{bkm:Ref301808090}Pour rendre compte des propriétés combinatoires des conjonctions, deux structures syntaxiques ont été proposées : (i) une structure plate \REF{ex:2:200}a, dans laquelle la conjonction et les éléments coordonnés se trouvent au même niveau (cf. \citet{Jackendoff1977}, Maxwell \& \citet{Manning1996}, \citet{Dalrymple2001}), et (ii) une structure hiérarchique \REF{ex:2:200}b, dans laquelle la conjonction forme un constituant avec la séquence qui suit, au moins pour les langues à tête initiale (\citet{Ross1967}, Sag \textit{et al.} (1985), \citet{Munn1993}, \citet{Paritong1992}, \citet{Kayne1994}, \citet{Johannessen1998}, Abeillé (2003, 2005, 2006), \citet{Skrabalova2004}, \citet{Mouret2007}).


\begin{enumerate}
\item \label{bkm:Ref301776256}a  Structure plate        b  Structure hiérarchique


\end{enumerate}
  [Warning: Image ignored] % Unhandled or unsupported graphics:
%\includegraphics[width=1.4984in,height=1.1154in,width=\textwidth]{fe443409cd384d3fb0f6390ffd77f513-img23.svm}
           [Warning: Image ignored] % Unhandled or unsupported graphics:
%\includegraphics[width=1.8161in,height=1.5583in,width=\textwidth]{fe443409cd384d3fb0f6390ffd77f513-img24.svm}
 

Cependant, il y a des arguments empiriques qui montrent qu'il faut préférer la deuxième à la première des deux structures ci-dessus. Une cohésion particulière est observée entre la conjonction et le syntagme qu'elle introduit (\citet{Ross1967}, Sag \textit{et al.} (1985), \citet{Munn1993}, \citet{Kayne1994}, \citet{Johannessen1998}, Abeillé (2003, 2005), \citet{Mouret2007}), ce qui justifie l'existence d'un syntagme conjoint [Conj X]. Premièrement, la séquence [Conj X] peut apparaître en dehors des constructions coordonnées ordinaires : le syntagme conjoint peut être une phrase indépendante \REF{ex:2:201}a, un fragment dans un dialogue \REF{ex:2:201}b ou bien un ajout incident \REF{ex:2:201}c. Deuxièmement, on observe une alternance distributionnelle régulière [Conj X]~/ [X] à travers les trois types de structures coordonnées : syndétique, asyndétique et omnisyndétique \REF{ex:2:202}. 


\begin{enumerate}
\item \label{bkm:Ref301775535}a  \textbf{Dar} Maria când vine ?


\end{enumerate}
{\itshape
Mais Maria quand elle vient}

  b  A : - Eu vreau un măr ! B : - \textbf{Iar} eu o pară !

{\itshape
A : - Je veux une pomme B : - Et moi une poire}

  c  Marea problemă, \textbf{şi} cu asta vreau să închei, este că nu oferiți nicio soluție viabilă.

\textit{Le grand problème, et je conclus avec ça, est que vous n'offrez aucune solution viabl}e


\begin{enumerate}
\item \label{bkm:Ref301776100}a  Recomand tuturor să consume [fructe, legume \textbf{ori} nuci].


\end{enumerate}
{\itshape
Je recommande à tous de consommer des fruits, des légumes ou des noix}

  b  Recomand tuturor să consume [fructe, legume, nuci].

{\itshape
Je recommande à tous de consommer des fruits, des légumes, des noix}

  c  Recomand tuturor să consume [\textbf{ori} fructe, \textbf{ori} legume, \textbf{ori} nuci].

{\itshape
Je recommande à tous de consommer ou bien des fruits, ou bien des légumes ou bien des noix}

Troisièmement, on note parfois l'existence d'une rupture prosodique toujours avant la conjonction et non après, de même pour la ponctuation à l'écrit \REF{ex:2:203}, ce qui justifie un regroupement prosodique de la conjonction avec le terme subséquent.\footnote{\citet{Mouret2007} considère cet argument pas convaincant, en s'appuyant sur les résultats de Bonami \& Delais-\citet{Roussarie2006} qui montrent qu'il n'y a pas toujours d'isomorphisme entre la structure syntaxique et la structure prosodique. On ne peut pas \textit{a priori} garder cette observation au moins pour une partie des conjonctions en roumain (\textit{iar} `et'\textit{, dar} `mais'\textit{, ci} `mais', etc.) qui manifestement se regroupent prosodiquement avec le terme subséquent. }  Quatrièmement, dans de nombreuses langues, la conjonction est cliticisée, c.-à-d. elle est réalisée sous forme d'affixe qui s'attache à la fin du terme coordonné, p.ex. la particule \textit{--que} en latin \REF{ex:2:204}. Finalement, un autre argument qui est avancé dans la littérature concerne les asymétries de liage qu'on peut observer entre un quantifieur et une variable liée \REF{ex:2:205}, mais, comme le note \citet[68]{Mouret2007}, cet argument se base sur une hypothèse contestable (selon Culicover \& Jackendoff (2005 : 122-124), la relation entre un quantifieur et une variable liée n'est pas nécessairement contrainte par une condition de c-commande).


\begin{enumerate}
\item \label{bkm:Ref301776855}a  Ion citeşte, {\textbar} \textbf{iar} Maria doarme.


\end{enumerate}
{\itshape
Ion lit, et Maria dort}

  b  *Ion citeşte \textbf{iar}, {\textbar} Maria doarme.\footnote{L'exemple est grammatical si \textit{iar} est adverbe (avec le sens `à nouveau'), mais ici ce qui nous intéresse est le comportement de la conjonction \textit{iar.} }

{\itshape
Ion lit et Maria dort}


\begin{enumerate}
\item \label{bkm:Ref301778499}Melior  tutior\textbf{que}  est  certa  pax  quam  incerta  victoria. 


\end{enumerate}
  meilleur  plus-sûr.\textsc{conj}  est  sûre  paix  que  incertaine  victoire

{\itshape
  Une paix sûre est meilleure et moins dangereuse qu'une victoire incertaine} 


\begin{enumerate}
\item \label{bkm:Ref301779023}a  Fiecare\textsubscript{i} mamă şi copilul ei\textsubscript{i} au fost vaccinați împotriva tuberculozei.


\end{enumerate}
{\itshape
Chaque mère et son enfant ont été vaccinés contre la tuberculose}

  b  *Copilul ei\textsubscript{i} şi fiecare\textsubscript{i} mamă au fost vaccinați împotriva tuberculozei.

{\itshape
Son enfant et chaque mère ont été vaccinés contre la tuberculose}

Sur la base de ces arguments empiriques, on considère donc que la structure la plus adéquate pour les coordinations simples est une structure hiérarchique \REF{ex:2:200}b, qui contient habituellement (au moins) un syntagme conjoint de type [Conj X].

\subsubsection{Une analyse endocentrique du syntagme conjoint [Conj X]}
\label{bkm:Ref301808307}\label{bkm:Ref301808905}En ce qui concerne les relations fonctionnelles à l'intérieur du syntagme conjoint, il est généralement analysé comme un syntagme de type tête-complément, donc il a une structure endocentrique. Cependant, deux analyses sont envisageables quant à l'attribution de la fonction tête : (i) la fonction tête est associée à la conjonction (\citet{Paritong1992}, \citet{Munn1993}, \citet{Kayne1994}, \citet{Johannessen1998}, Abeillé (2003, 2005, 2006), \citet{Skrabalova2004}, \citet{Mouret2007}), ou (ii) la fonction tête est associée au terme subséquent (\citet{Ross1967}, Sag \textit{et al.} (1985), Gazdar \textit{et al.} (1985), Beavers \& \citet{Sag2004}).  

Une des motivations majeures pour assigner la fonction tête à la conjonction est la corrélation entre la position de la conjonction et la position de la tête (verbale) à travers les langues. \citet{Kayne1994} observe ainsi que, dans les langues à tête finale, la conjonction a tendance à être postposée dans~le syntagme conjoint (p.ex. en japonais \REF{ex:2:206}, exemple repris de \citet[69]{Mouret2007}), alors que, dans les langues à tête initiale, la conjonction occupe la première position dans le syntagme conjoint (p.ex. en français \REF{ex:2:207}).


\begin{enumerate}
\item \label{bkm:Ref301719480}a  Taroo  \textbf{to}  Akiko  \textbf{to}  wa  Nara  e  ikimashita.


\end{enumerate}
Taroo  et  Akiko  et  \textsc{foc } Nara  à  aller.\textsc{passe} 

{\itshape
Et Taroo et Akiko sont allés à Nara}

  b  *\textbf{To}  Taroo  \textbf{to}  Akiko  wa  Nara  e  ikimashita.

    et  Taroo  et  Akiko  \textsc{foc } Nara  à  aller.\textsc{passe}

    \textit{Et Taroo et Akiko sont allés à Nara} 


\begin{enumerate}
\item \label{bkm:Ref301719944}a  \textbf{Et} Jean \textbf{et} Marie sont venus à la fête.


\end{enumerate}
  b  *Jean \textbf{et} Marie \textbf{et} sont venus à la fête.

Cependant, on ne peut pas établir une corrélation parfaite entre la position de la conjonction et la position du verbe à travers les langues (cf. \citet{Chaves2007}). D'une part, parmi les 68 langues à tête finale étudiées par \citet{Zwart2005}, la plupart utilisent des conjonctions en début de conjoint. D'autre part, il y a des langues à ordre de mots libre, dans lesquelles la tête peut avoir une distribution plus ou moins flexible, mais dans lesquelles on n'observe pas une distribution similaire de la conjonction, qui a toujours une position fixe (p.ex. la conjonction \textit{i} `et' en russe). De plus, comme le note \citet{Mouret2007}, cette corrélation est pertinente uniquement si la conjonction constitue un mot syntaxiquement autonome (or, dans les langues avec postposition de la conjonction, celle-ci est souvent cliticisée comme affixe syntagmatique).

Un deuxième argument en faveur d'une analyse de la conjonction comme tête concerne les restrictions de sous-catégorisation. Les conjonctions sous-catégorisent le constituant avec lequel elles se combinent. Dans une perspective typologique, on observe que certaines conjonctions sont compatibles uniquement avec certaines catégories syntaxiques (p.ex. syntagmes nominaux vs. phrases) ou uniquement avec certains types sémantiques (p.ex. individus vs. événements). Ainsi, en français, la conjonction \textit{car} est compatible avec une phrase finie \REF{ex:2:208}a, mais pas avec un groupe verbal~\REF{ex:2:208}b ; ou encore la conjonction lexicalisée \textit{ainsi que} qui est compatible avec une catégorie verbale non-finie \REF{ex:2:208}d, mais incompatible avec une catégorie verbale finie \REF{ex:2:208}c, cf. \citet{Mouret2007}. De même, en roumain, la conjonction \textit{iar} `et' est compatible avec une phrase \REF{ex:2:209}a, mais pas avec un syntagme nominal \REF{ex:2:209}b.


\begin{enumerate}
\item \label{bkm:Ref301782840}a  Paul est contrarié, \textbf{car} il a oublié d'acheter du vin.


\end{enumerate}
  b  *Paul est contrarié, \textbf{car} a oublié d'acheter du vin.

  c  *Paul écoute la radio, \textbf{ainsi que} Marie lit le journal.

  d  Paul aime écouter la radio, \textbf{ainsi que} lire le journal.


\begin{enumerate}
\item \label{bkm:Ref301783815}a  Maria doarme, \textbf{iar} Ion lucrează.


\end{enumerate}
{\itshape
Maria dort, et Ion travaille}

  b  *Maria mănâncă mere roşii \textbf{iar} pere galbene.

{\itshape
Maria mange des pommes rouges et des poires jaunes}

Une autre propriété qui est parfois exploitée comme argument en faveur de cette analyse est l'assignation du cas. On observe que la conjonction peut bloquer l'assignation du cas : ainsi, selon \citet{Johannessen1998}), en espagnol une coordination de pronoms au nominatif peut être l'objet d'une préposition comme \textit{para} `pour' \REF{ex:2:210}a, qui normalement assigne le cas accusatif à son complément \REF{ex:2:210}c-d. Un fait similaire est observé en anglais \REF{ex:2:211}, cf. Huddleston \& \citet{Pullum2002}.  


\begin{enumerate}
\item \label{bkm:Ref301786034}a  para  tú  \textbf{y } yo


\end{enumerate}
pour  toi\textsc{.nom}  et  moi.\textsc{nom}

{\itshape
pour toi et moi}

  b  *para  ti  \textbf{y } mí

    pour  toi\textsc{.acc}  et  moi.\textsc{acc}

{\itshape
pour toi et moi}

  c  para  \{ti  {\textbar} *tú\}

pour  \{toi.\textsc{acc  {\textbar}} toi.\textsc{nom\}}

{\itshape
pour toi}

  d  para  \{mí  {\textbar} *yo\}

    pour  \{moi.\textsc{acc  {\textbar}} moi.\textsc{nom\}}

{\itshape
pour moi}


\begin{enumerate}
\item \label{bkm:Ref301786194}a  He invited Kim and me.


\end{enumerate}
  b  \%He invited Kim and I.

  c  He invited \{me {\textbar} *I\}.

Sur la base de ces propriétés, on peut conclure que la conjonction régit le constituant avec lequel elle se combine. En même temps, on a observé que la conjonction ne détermine que partiellement la syntaxe externe du constituant qu'elle introduit (\citet{Paritong1992}, \citet{Johannessen1998}, Abeillé (2003, 2005, 2006)), ayant ainsi un comportement différent par rapport aux têtes ordinaires (p.ex. un verbe ou un adjectif). De ce point de vue, le syntagme conjoint, tout comme d'autres syntagmes contenant une catégorie {\guillemotleft}~mineure~{\guillemotright} (p.ex. déterminants, complémenteurs ou prépositions {\guillemotleft}~incolores~{\guillemotright}), présente un mixte des propriétés de leurs constituants immédiats (voir discussion et illustrations dans Mouret (2007 : 71-73)). 

Je considère que l'analyse des conjonctions en termes de têtes {\guillemotleft}~faibles~{\guillemotright}, telle qu'elle a été proposée par Abeillé (2003, 2005), rend compte parfaitement de ce comportement spécial des conjonctions (\citet{Sag1997} et \citet{Tseng2002} proposent une analyse similaire pour d'autres catégories {\guillemotleft}~mineures~{\guillemotright}). Dans cette perspective, une catégorie {\guillemotleft}~mineure~{\guillemotright} hérite du constituant qu'elle sélectionne la plupart des propriétés morpho-syntaxiques qu'elle transmet au syntagme. 

La conjonction sous-catégorise donc le constituant avec lequel elle se combine (ce qui justifie le terme de \textit{tête}), mais hérite une partie des propriétés morpho-syntaxiques de celui-ci (cf. le partage de variables en \REF{ex:2:212}), sauf le trait CONJ.~Par conséquent, syntaxiquement, ce sont les conjoints qui déterminent, entre autres, la catégorie de la structure coordonnée dans son ensemble, alors que, sémantiquement, c'est la conjonction qui gère l`interprétation de l'ensemble à partir de la contribution sémantique de chaque conjoint.

Les traits de tête (HEAD) et de marque (MARKING) de la conjonction sont les mêmes que ceux de son complément. La conjonction hérite des traits SUBJ, SPR et COMPS de son complément, permettant ainsi au syntagme conjoint de rendre accessible les contraintes de sélection de ce complément. La conjonction a un trait CONJ à valeur non-nulle, tandis que son complément possède un trait CONJ à valeur nulle.


\begin{enumerate}
\item \label{bkm:Ref290400569}Entrée lexicale d'une conjonction


\end{enumerate}
  [Warning: Image ignored] % Unhandled or unsupported graphics:
%\includegraphics[width=4.8417in,height=2.2118in,width=\textwidth]{fe443409cd384d3fb0f6390ffd77f513-img25.svm}
  

Pour conclure, le constituant introduit par une conjonction a une structure de type tête-complément, avec la mention que la tête est déficiente (cf. spécificité des catégories {\guillemotleft}~mineures~{\guillemotright}). On obtient ainsi une structure endocentrique des séquences [Conj X], comme illustrée en \REF{ex:2:213} pour le conjoint \textit{sau pe masă} `ou sous table'. 


\begin{enumerate}
\item \label{bkm:Ref301789151}Syntaxe simplifiée du syntagme conjoint


\end{enumerate}
{   [Warning: Image ignored] % Unhandled or unsupported graphics:
%\includegraphics[width=2.1484in,height=2.2583in,width=\textwidth]{fe443409cd384d3fb0f6390ffd77f513-img26.svm}
} 

\subsubsection{Une analyse exocentrique de la coordination dans son ensemble}
\label{bkm:Ref301808531}Si l'on regarde la structure hiérarchique proposée en \REF{ex:2:200}b, on observe qu'une coordination est une structure à deux étages : le premier niveau est celui du syntagme conjoint, qu'on a décrit dans la section précédente 2.10.2, et le deuxième niveau correspond à la structure coordonnée dans son ensemble.

La plupart des travaux alignent la structure coordonnée sur la structure endocentrique du syntagme conjoint. Ainsi, \citet{Munn1993}, \citet{Kayne1994}, \citet{Johannessen1998}, \citet{Camacho2003}, \citet{Rebuschi2005}, etc. considèrent les structures coordonnées comme des constructions avec tête. En particulier, les coordinations sont, selon eux, soit des structures de type spécifieur-tête-compléments\footnote{Dans ce cas, la structure coordonnée est un syntagme ConjP, dans laquelle la conjonction est la tête, le premier conjoint est le spécifieur et le deuxième conjoint est le complément.} (\citet{Kayne1994}, \citet{Johannessen1998}), soit des structures de type tête-ajout\footnote{Selon cette analyse, le syntagme introduit par la conjonction est adjoint au premier terme coordonné qui est la tête de la coordination dans son ensemble. Cette analyse est reprise par Abeillé (2005) pour rendre compte des conjoints incidents en français.} (\citet{Munn1993}). Cependant, par rapport aux syntagmes classiques avec tête, les propriétés de la structure coordonnée dans son ensemble ne sont pas projetées par une seule tête syntaxique, mais elles sont déterminées par tous les conjoints (voir discussion et arguments convaincants dans Borsley (1994, 2005)).\footnote{Pour une critique détaillée du syntagme ConjP, voir \citet{Chaves2007} et \citet{Mouret2007}.} 

Syntaxiquement, aucun conjoint n'est dépendant d'un autre conjoint : il y a une bonne partie des coordinations qui permettent le changement dans l'ordre des conjoints sans entraîner des changements significatifs au niveau de l'interprétation \REF{ex:2:214}.


\begin{enumerate}
\item \label{bkm:Ref301799373}\label{bkm:Ref301797712}a  Ion doarme \textbf{şi} Maria citeşte.  \\
\textit{Ion dort et Maria} \textit{lit}


\end{enumerate}
=  b  Maria citeşte \textbf{şi} Ion doarme.

    \textit{Maria lit et Ion dort}

Puisqu'il n'y a pas de dépendance syntaxique entre les conjoints, on ne peut donc attribuer la fonction tête à aucun conjoint. Par conséquent, les propriétés d'une coordination dépendent des propriétés de \textit{chacun} des termes qui la composent et non pas des propriétés de la conjonction (\textit{contra} les approches de la coordination en termes de schéma X-barre de type spécifieur-tête-compléments, cf. \citet{Kayne1994}, \citet{Johannessen1998}) ou encore des propriétés du premier terme conjoint (\textit{contra} les approches en termes de schéma tête-ajout, cf. \citet{Munn1993}).

Pour rendre compte du comportement syntaxique des structures coordonnées, je fais appel à la distinction (classique en HPSG) entre les syntagmes avec tête (angl. \textit{headed-phrases}, abrégé \textit{hd-ph}) et les syntagmes sans tête (angl. \textit{non-headed-phrases}, abrégé \textit{non-hd-ph}), cf. la hiérarchie de signes figurant en \REF{ex:2:215}. Dans un syntagme avec tête, on a une relation de sélection entre les constituants : une branche dominante gère la catégorie et la distribution syntaxique de l'ensemble. Dans un syntagme sans tête, il n'y a aucune relation de sélection au niveau syntaxique.


\begin{enumerate}
\item \label{bkm:Ref301798760}Hiérarchie de signes


\end{enumerate}
{   [Warning: Image ignored] % Unhandled or unsupported graphics:
%\includegraphics[width=3.2681in,height=1.6665in,width=\textwidth]{fe443409cd384d3fb0f6390ffd77f513-img27.svm}
} 

La structure coordonnée se prête mieux à une analyse en termes de construction sans tête, c.-à-d. un syntagme de type \textit{non-headed-ph} (cf. Pollard \& \citet{Sag1994}, Sag \textit{et al.} (2003), Abeillé (2005), \citet{Mouret2007}, etc.). En \REF{ex:2:216}, je donne l'arbre simplifié de la phrase \REF{ex:2:214}a, qui illustre les deux niveaux syntaxiques de la coordination. 


\begin{enumerate}
\item \label{bkm:Ref301799331}Syntaxe simplifiée d'une coordination de phrases


\end{enumerate}
{   [Warning: Image ignored] % Unhandled or unsupported graphics:
%\includegraphics[width=2.7717in,height=1.6929in,width=\textwidth]{fe443409cd384d3fb0f6390ffd77f513-img28.svm}
} 

Une coordination comporte au moins deux termes, qui peuvent ou non être introduits par une conjonction. La règle générale de la coordination est donnée en \REF{ex:2:217}.


\begin{enumerate}
\item \label{bkm:Ref301713060}Règle générale de la coordination


\end{enumerate}
  [Warning: Image ignored] % Unhandled or unsupported graphics:
%\includegraphics[width=5.4217in,height=0.4736in,width=\textwidth]{fe443409cd384d3fb0f6390ffd77f513-img29.svm}
  

En fonction de la distribution des conjonctions, on peut distinguer entre trois types de constructions coordonnées (cf. Mouret (2006, 2007) pour le français, Bîlbîie (2008) pour le roumain, et de manière plus générale dans les langues romanes) : (i) coordinations simples, avec au moins une conjonction sur le dernier conjoint \REF{ex:2:218} ; (ii) coordinations omnisyndétiques ou corrélatives, avec un élément corrélatif répété sur chaque conjoint (y compris le conjoint initial), cf. \REF{ex:2:219}, et (iii) les coordinations asyndétiques ou juxtaposées (sans aucune conjonction lexicalisée), cf. \REF{ex:2:220}. 


\begin{enumerate}
\item \label{bkm:Ref290401511}a  \textit{simplex-coord-ph} ={\textgreater} [DTRS nelist([CONJ nil]) ${{{\oplus}}}$nelist([CONJ 1 ${\neq}$nil])]  


\end{enumerate}
  b  Ion, (şi) Maria şi Gheorghe

    \textit{Ion, (et) Maria et Gheorghe}


\begin{enumerate}
\item \label{bkm:Ref290401542}a  \textit{omnisyndetic-coord-ph} ={\textgreater} [DTRS nelist([CONJ 1 ${\neq}$nil])]    


\end{enumerate}
  b  fie Ion, fie Maria, fie Gheorghe

    \textit{soit Ion, soit Maria, soit Gheorghe}


\begin{enumerate}
\item \label{bkm:Ref290401557}a  \textit{asyndetic-coord-ph} ={\textgreater} [DTRS nelist([CONJ nil])]  


\end{enumerate}
  b  Ion, Maria, Gheorghe

    \textit{Ion, Maria, Gheorghe}

Une hiérarchie de syntagmes qui contient les trois types de coordinations est donnée en \REF{ex:2:221}. 


\begin{enumerate}
\item \label{bkm:Ref301799112}La structure coordonnée dans une hiérarchie de syntagmes


\end{enumerate}
{   [Warning: Image ignored] % Unhandled or unsupported graphics:
%\includegraphics[width=5.5602in,height=1.6874in,width=\textwidth]{fe443409cd384d3fb0f6390ffd77f513-img30.svm}
} 

Les termes d'une coordination ont souvent les mêmes propriétés syntaxiques, mais ils peuvent aussi différer sur certains aspects (catégorie syntaxique, cas, personne, mode verbal, temps, forme de la préposition, etc.). Le problème des coordinations de termes dissemblables, comme en \REF{ex:2:222}a, peut être résolu si on assume, cf. Sag (2003, 2005), que les valeurs de certains traits peuvent rester sous-spécifiées, même lorsque la structure de traits est `complète' du point de vue de la grammaire. Ainsi, on peut supposer que les entrées lexicales ne fixent pas une valeur précise pour la valeur de HEAD, mais imposent une borne supérieure. \REF{ex:2:223}a dit que la catégorie syntaxique de \textit{naiv} `naïf' est moins spécifiée que, ou égale à \textit{adj}, alors que \REF{ex:2:223}b dit que la catégorie de \textit{imbecil} `imbécile' est une valeur moins spécifiée que, ou égale à \textit{noun}. 


\begin{enumerate}
\item \label{bkm:Ref290403129}a  Ion este [fie naiv]\textsubscript{AdjP}, [fie un imbecil]\textsubscript{NP}.


\end{enumerate}
{\itshape
Ion est soit naïf, soit un imbécile}

  b  Ion este \{naiv {\textbar} un imbecil\}.  


\begin{enumerate}
\item \label{bkm:Ref299914157}a  naiv (`naïf') : [HEAD 1 {\textbar} 1 ${\leq}$ \textit{adj}]


\end{enumerate}
  b  imbecil (`imbécile') : [HEAD 2 {\textbar} 2 ${\leq}$ \textit{noun}

Par la suite, les traits de tête de la construction coordonnée sont obtenus par unification des traits de tête des branches (DTRS), qui peuvent, quant à elles, rester sous-spécifiées. On peut donc coordonner le syntagme adjectival et le syntagme nominal, car la coordination reçoit par unification une catégorie sous-spécifiée \textit{nominal}, qui est un super-type des noms et des adjectifs. Cette catégorie sous-spécifiée peut donc s'unifier sans problèmes avec les deux compléments prédicatifs sélectionnés par le verbe copule \textit{a fi} `être' (car le verbe \textit{a fi} accepte comme complément et un syntagme adjectival et un syntagme nominal, cf. \REF{ex:2:222}b), comme on voit en \REF{ex:2:225}. En revanche, cette catégorie sous-spécifiée ne peut pas s'unifier avec le complément sélectionné par un verbe comme \textit{a întâlni} `rencontrer' \REF{ex:2:224}a, car ce verbe ne peut pas avoir comme complément un syntagme adjectival \REF{ex:2:224}b.  


\begin{enumerate}
\item \label{bkm:Ref302068906}a  *Ion  a  întâlnit  fie  naiv,  fie  un  imbecil.


\end{enumerate}
Ion  a  rencontré  soit  naïf,  soit  un  imbécile

{\itshape
Ion a rencontré soit un naïf, soit un imbécile}

  b  Ion a întâlnit \{*naiv {\textbar} un imbecil\}.


\begin{enumerate}
\item \label{bkm:Ref301713260}Coordination de termes dissemblables, cf. \REF{ex:2:222}a


\end{enumerate}
{   [Warning: Image ignored] % Unhandled or unsupported graphics:
%\includegraphics[width=5.552in,height=3.5453in,width=\textwidth]{fe443409cd384d3fb0f6390ffd77f513-img31.svm}
} 

On peut ainsi postuler une version finale pour la règle de la coordination en \REF{ex:2:226}. Cette règle impose qu'il y ait non seulement un partage des traits SLASH et VALENCE entre le syntagme coordonné dans son ensemble et les membres conjoints, mais aussi un partage, par défaut, du trait HEAD. L'identité de valeur pour le trait SLASH (trait qui enregistre les constituants manquants en cas d'extraction) autorise uniquement l'extraction parallèle hors de chaque conjoint. L'identité de valeur pour le trait VALENCE n'autorise que la coordination de prédicats avec les mêmes contraintes de sous-catégorisation. 


\begin{enumerate}
\item \label{bkm:Ref290407646}Règle de la coordination (version finale)


\end{enumerate}
  [Warning: Image ignored] % Unhandled or unsupported graphics:
%\includegraphics[width=4.2618in,height=1.1563in,width=\textwidth]{fe443409cd384d3fb0f6390ffd77f513-img32.svm}
  

Les contraintes observées jusqu'ici s'appliquent aux constructions coordonnées ordinaires. J'ai présenté uniquement les aspects importants de la coordination qui seront utiles pour l'analyse des cooordinations à gapping (chapitre 4), en laissant de côté les conjoints incidents que j'ai mentionnés dans la section \ref{sec:2.8.2.3}. Pour une analyse détaillée des syntagmes conjoints en dehors des structures coordonnées canoniques, consulter Abeillé (2005, 2006). Je reprends en \REF{ex:2:227} deux arbres simplifiés (figurant dans Abeillé (2005)), qui montrent les deux analyses dont on a besoin pour rendre compte des conjoints standard \REF{ex:2:227}a et respectivement des conjoints incidents \REF{ex:2:227}b. Dans les deux cas, la conjonction de coordination est une tête {\guillemotleft}~faible~{\guillemotright} qui forme un sous-constituant avec le terme suivant et qui projette la catégorie de son complément. La différence majeure entre les deux constructions réside dans la fonction syntaxique associée aux termes coordonnés : dans les coordinations canoniques, les constituants reçoivent la même fonction (c.-à-d. \textit{non-têtes}, abrégé \textit{N-HD}), alors que dans les constructions avec un conjoint incident, celui-ci est un ajout syntaxique (abrégé \textit{ADJ}).


\begin{enumerate}
\item \label{bkm:Ref301813041}a  Conjoints ordinaires      b  Conjoints incidents


\end{enumerate}
  [Warning: Image ignored] % Unhandled or unsupported graphics:
%\includegraphics[width=2.3681in,height=1.5709in,width=\textwidth]{fe443409cd384d3fb0f6390ffd77f513-img33.svm}
         [Warning: Image ignored] % Unhandled or unsupported graphics:
%\includegraphics[width=2.3008in,height=1.6272in,width=\textwidth]{fe443409cd384d3fb0f6390ffd77f513-img34.svm}
 

\subsection{Conclusion}
Dans ce chapitre, j'ai défini la coordination et la subordination sur une base strictement syntaxique, sans corrélation biunivoque avec les types de relations discursives. J'ai montré que la coordination n'impose pas de parallélisme strict au niveau morpho-syntaxique ou encore au niveau sémantico-discursif. D'une part, la contrainte syntaxique majeure qui rend compte des dissemblances qu'on peut avoir dans une structure coordonnée est résumée par la généralisation de Wasow, selon laquelle chaque conjoint dans une structure coordonnée doit pouvoir apparaître seul en lieu et place de la coordination dans son ensemble. D'autre part, la notion de symétrie / asymétrie discursive est orthogonale à la distinction coordination / subordination. De manière générale, les coordinations semblent être sous-spécifiées quant aux types de relations discursives établies. Les seules coordinations qui spécifient toujours une relation discursive symétrique (c.-à-d. qui permettent l'ordre libre des conjoints sans changer les conditions de vérité de l'énoncé) sont : les coordinations avec la conjonction \textit{iar} `et', certaines coordinations avec des éléments corrélatifs, les coordinations de subordonnées avec répétition du complémenteur et, comme on verra dans le chapitre 4, les coordinations à gapping. La juxtaposition, quant à elle, est un type de relation sous-spécifié~au niveau syntaxique. Elle doit être définie plutôt par rapport au type de relation discursive (symétrique vs. asymétrique) qui caractérise les éléments juxtaposés. Quant aux éléments corrélatifs, ils apparaissent et avec la coordination et avec la subordination, mais leurs propriétés sont différentes (les plus contraintes sont les coordinations corrélatives). En ce qui concerne les conjoints ou les subordonnées incidents, ils se distinguent prosodiquement et syntaxiquement des conjoints ou subordonnées intégrés. De plus, contrairement à ce qui est souvent postulé, un constituant incident n'est pas nécessairement parenthétique du point de vue sémantique. 

Bien que la coordination permette beaucoup de dissemblances syntaxiques et d'asymétries discursives, le roumain dispose d'une conjonction spécialisée pour la relation symétrique de parallélisme et contraste. La conjonction \textit{iar} impose des contraintes spécifiques au niveau sémantique, discursif et syntaxique, qui la distinguent de toute autre conjonction en roumain (en particulier, de la conjonction additive \textit{şi} `et' et de la conjonction adversative \textit{dar} `mais'). 

En ce qui concerne la structure interne des coordinations, j'ai montré que la coordination se prête mieux à une structure hiérarchique, dans laquelle la conjonction forme un constituant avec la séquence qui suit (c.-à-d. un syntagme conjoint de la forme [Conj X]). Ce syntagme conjoint a une structure endocentrique de type tête-complément. La tête du syntagme conjoint est la conjonction, mais, contrairement aux têtes ordinaires, elle hérite une partie des propriétés morpho-syntaxiques (en particulier, la catégorie syntaxique) de son complément, ce qui justifie son statut de \textit{tête {\guillemotleft}~faible~{\guillemotright}}. La coordination dans son ensemble est plutôt une construction sans tête, ayant une structure exocentrique dans laquelle il n'y a pas de dépendance syntaxique entre les conjoints.

