%0

\setcounter{page}{1}
{
Université Paris Diderot-Paris 7
}

{\bfseries
Ecole Doctorale de Sciences du Langage
}

{\bfseries
UFR de Linguistique
}

{\itshape
Laboratoire de Linguistique Formelle
}

{\bfseries
DOCTORAT\textmd{} 
}

{
de 
}

{
Linguistique Théorique, Descriptive et Automatique
}

{
Gabriela BILBIIE
}

{\bfseries
GRAMMAIRE DES CONSTRUCTIONS ELLIPTIQUES
}

{\bfseries
Une étude comparative des phrases sans verbe en roumain et en français
}

{
Thèse dirigée par Anne Abeillé
}

{
Soutenue le~16 novembre 2011
}

{\bfseries
Composition du jury :
}

Anne Abeillé      Université Paris Diderot -- Paris 7 (directeur)

Carmen Dobrovie-Sorin  CNRS, LLF

Jonathan Ginzburg    Université Paris Diderot -- Paris 7 

Emil Ionescu      Université de Bucarest (pré-rapporteur)

Jason Merchant    Université de Chicago (pré-rapporteur)

Marleen Van Peteghem   Université de Gand

 
{\itshape
{\guillemotleft}~Et les voilà embarqués dans une querelle interminable sur les femmes ; l'un prétendant qu'elles étaient bonnes, l'autre méchantes : et ils avaient tous deux raison ; l'un sottes, l'autre pleines d'esprit : et ils avaient tous deux raison ; l'un fausses, l'autre vraies : et ils avaient tous deux raison ; l'un avares, l'autre libérales : et ils avaient tous deux raison ; l'un belles, l'autres laides : et ils avaient tous deux raison ; l'un bavardes, l'autre discrètes ; l'un franches, l'autre dissimulées ; l'un ignorantes, l'autre éclairées ; l'un sages, l'autre libertines ;~l'un folles, l'autre sensées, l'un grandes, l'autre petites : et ils avaient tous deux raison.~{\guillemotright} (D. Diderot, \textup{Jacques le fataliste et son maître})}

{\itshape
{\guillemotleft}~Se încinge o conversație :} 

{\itshape
- Hei, mă, din casă !}

{\itshape
- Cine ?}

{\itshape
- Tu.}

{\itshape
- Eu ?}

{\itshape
- Păi cine ?}

{\itshape
- Ce-i ?}

{\itshape
- Cum ce-i ?}

{\itshape
- Păi ce-i ?}

{\itshape
- Ai o scrisoare.}

{\itshape
- Cine, mă ?}

{\itshape
- Tu.}

{\itshape
- Eu ?}

{\itshape
- Păi cine ?}

{\itshape
- Ad-o-ncoa !~{\guillemotright}}

{\itshape
(I.L. Caragiale, \textup{Cum se înțeleg țăranii})}



{\bfseries
Remerciements}

Plusieurs personnes ont contribué, de près ou de loin, à l'aboutissement de cette thèse et je tiens à les en remercier chaleureusement. Dans ce qui suit, j'aimerais remercier en particulier les personnes qui ont eu un apport significatif à cette thèse.

Je voudrais exprimer toute ma reconnaissance à ma directrice, Anne Abeillé, qui a dirigé mes recherches depuis le master et m'a permis de réaliser cette thèse dans les meilleures conditions possibles. Je lui suis reconnaissante de sa souplesse et de son ouverture d'esprit, car Anne m'a laissé une grande marge de liberté pendant tout ce travail de recherche. Je la remercie pour sa confiance, pour sa disponibilité et la rapidité avec laquelle elle a répondu à mes questions et surtout pour ses sages conseils tant au niveau scientifique qu'administratif. 

Je suis très reconnaissante à François Mouret, qui m'a aidée à comprendre plusieurs aspects liés à la coordination et à l'ellipse. Son suivi, le sérieux et la précision de ses commentaires m'ont toujours aidée à avancer dans mon travail. 

Je remercie aussi les membres de l'UFR de Linguistique et du Laboratoire de Linguistique Formelle, et tout particulièrement Olivier Bonami et Jean-Marie Marandin pour leur aide inestimable à la fin de ma thèse ; leurs commentaires m'ont aidée à améliorer certains chapitres de la thèse et à résoudre certains problèmes relevant de la méthodologie, la description ou encore la formalisation de faits linguistiques. Je tiens à remercier Danièle Godard pour son suivi et pour ses encouragements continus. 

Je remercie les personnes qui ont accepté d'être membres de mon jury, et en particulier les pré-rapporteurs, qui ont manifesté une grande compréhension et ont fait l'effort de lire la thèse qui leur parvenait par chapitres. Je suis très reconnaissante à Carmen Dobrovie-Sorin pour toute son aide apportée à divers titres depuis mon arrivée à l'Université Paris 7 comme étudiante ERASMUS. Je remercie Jonathan Ginzburg pour avoir accepté de faire partie de mon jury, pour sa gentillesse et pour sa flexibilité. Je tiens à exprimer ma reconnaissance à Emil Ionescu, un des premiers professeurs m'ayant initiée à la linguistique à l'Université de Bucarest. Ma gratitude s'adresse aussi à Jason Merchant, qui, par l'intermédiaire de ses présentations et articles, m'a aidée à approfondir le champ d'étude de l'ellipse ; j'apprécie beaucoup son honnêteté intellectuelle et son ouverture d'esprit. Enfin, je remercie Marleen Van Peteghem d'avoir accepté de lire ma thèse malgré des conditions difficiles.  

Un grand merci à mes chers collègues Frédéric Laurens et Grégoire Winterstein, qui ont eu un apport non-négligeable aux résultats de cette thèse, par leurs collaborations et leurs discussions stimulantes (merci, Grégoire, pour ton soutien technique). Merci à Margot Colinet pour sa disponibilité même à des heures indues, ma référence en matière de français. Merci à mes collègues et amis à Paris 7, et notamment à Anna Gazdik, Fabiola Henri, Jana Strnadova et Delphine Tribout pour leur soutien lors des derniers mois de rédaction. Merci à Clément, qui m'a soutenue dans mes problèmes informatiques.

Je clos enfin ces remerciements en dédiant cette thèse à ma famille et aux quelques amis très proches (sans oublier Daniel Lavalette), qui m'ont soutenue tout au long de ces années de travail. Et, plus que quiconque, je remercie Răzvan, mon soutien sans faille, d'avoir sublimé les 2500 km entre Bucarest et Paris pour venir à mes côtés.


{\bfseries
Table des matières}


 
{\bfseries
Introduction
}

{\bfseries
Problématique}

Le fait que les langues sont organisées de manière économique devient évident quand on parle d'ellipse. Les utilisateurs d'une langue emploient chaque jour des structures fragmentaires ou incomplètes (comme celles données dans le préambule de cette thèse), en comptant sur le fait que la bonne interprétation sera de toute façon obtenue grâce au contexte ou à la connaissance du monde. Cependant, la manière dont on élide l'information considérée comme redondante n'est pas arbitraire : elle est guidée systématiquement par des facteurs syntaxiques, sémantiques et discursifs. 

Le but de cette thèse est essentiellement de définir les facteurs qui entrent en jeu dans la description de deux constructions elliptiques en roumain et en français. La première, appelée \textit{gapping}\footnote{Le terme de \textit{gapping} est emprunté à l'anglais. Dans cette thèse, j'utilise généralement le terme anglais et, quand je parle strictement de la phrase elliptique, j'utilise \textit{phrase trouée}.} , est exemplifiée en \REF{ex:0:1}a pour le roumain et \REF{ex:0:1}b pour le français : on coordonne une phrase complète et une phrase elliptique qui compte au moins deux éléments résiduels et où manque le verbe principal. La deuxième construction étudiée, qu'on va appeler \textit{ajouts relatifs averbaux} (dorénavant VRA, cf. angl. \textit{Verbless Relative Adjuncts}), est exemplifiée en \REF{ex:0:2}a pour le roumain et \REF{ex:0:2}b~pour le français : à une phrase complète s'adjoint une phrase elliptique qui est introduite par un élément relatif suivi d'un ou plusieurs constituants à l'exclusion d'un verbe fini. 


\begin{enumerate}
\item \label{bkm:Ref306208053}a  Ioana studiază lingvistica, [\textbf{iar} Maria dreptul].  (roumain)


\end{enumerate}
{\itshape
Ioana étudie la linguistique, et Maria le droit~}

  b  Jean étudie la linguistique [\textbf{et} Marie le droit].  (français)


\begin{enumerate}
\item \label{bkm:Ref306208494}a  Au  sunat  mai multe  persoane,  [\textbf{printre  care}  şi  Ion  (ieri)]. (roumain)


\end{enumerate}
ont  appelé  plusieurs  personnes,  parmi  lesquelles  aussi  Ion  (hier)

{\itshape
Plusieurs personnes ont appelé, parmi lesquelles Ion (hier)~}

  b  Plusieurs personnes, [\textbf{parmi lesquelles} Jean], ont appelé hier.      (français)

Dans les deux constructions, il s'agit d'une relation entre une séquence de constituants dont l'interprétation requiert plus que ce qui est donné par les mots qui la composent et une expression présente dans le contexte linguistique, qui fournit à cette séquence le matériel manquant dont elle a besoin pour être interprétée (en l'occurrence, le verbe \textit{étudier} dans les exemples en \REF{ex:0:1} et le verbe \textit{appeler} dans les exemples en \REF{ex:0:2}).

L'ellipse constitue ainsi un vrai défi à la définition saussurienne du signe linguistique qui associe une forme (le signifiant) à un contenu (le signifié), car dans les constructions mentionnées ci-dessus, comme dans beaucoup d'autres, on arrive à obtenir une interprétation en l'absence d'une forme (\textit{significatio ex nihilo}). 

La question générale qui surgit alors est de savoir comment on articule cette dichotomie classique dans le cas de l'ellipse. Une solution simple, couramment admise dans les approches génératives, serait d'aligner ces phrases elliptiques sur leurs contreparties complètes et de considérer que le matériel qui manque a une forme {\guillemotleft}~invisible~{\guillemotright}, auquel cas l'association forme/contenu est préservée. Pour pouvoir adopter cette solution, il faut démontrer que les phrases elliptiques et leurs contreparties complètes ont exactement les mêmes propriétés. L'objectif majeur de cette thèse est de montrer que cette solution n'est pas adéquate pour les constructions en question, car le comportement de ces phrases elliptiques n'est pas toujours le même que celui des phrases complètes correspondantes. Argumenter contre l'existence d'une forme {\guillemotleft}~invisible~{\guillemotright} dans ces structures elliptiques et, par conséquent, contre une reconstruction en syntaxe du matériel manquant nous demande, d'une part, de revoir l'ontologie des unités syntaxiques, ainsi que leur pertinence pour la description des phénomènes elliptiques, et, d'autre part, d'examiner l'importance d'autres facteurs. Dans cette thèse, j'insiste surtout sur le premier point : revoir la syntaxe et argumenter pour l'existence d'une catégorie \textit{fragment} dans la grammaire (cf. Ginzburg \& \citet{Sag2000}). La perspective syntaxique que j'adopte est donc celle résumée par Culicover \& \citet[5]{Jackendoff2005} :


\begin{enumerate}
\item   \textit{Simpler Syntax Hypothesis~}


\end{enumerate}
The most explanatory syntactic theory is one that imputes the minimum structure necessary to mediate between phonology and meaning.

La plupart des travaux sur l'ellipse se sont concentrés sur la coordination et moins sur la subordination. Dans cette thèse, j'ai choisi de traiter une construction elliptique pour chaque type de relation syntaxique (le gapping pour la coordination et les VRA pour la subordination), afin de vérifier si mes hypothèses s'appliquent uniquement à un type de relation syntaxique ou si elles sont plus générales. 

Le gapping a constitué l'objet de recherches nombreuses faites sur des langues différentes (dont, en particulier, l'anglais, l'allemand, le japonais et le coréen), mais il n'y a pas eu de recherche équivalente sur les langues romanes. L'avantage de choisir le roumain comme langue privilégiée pour l'étude du gapping est double : d'une part, le roumain nous permet de confronter les contraintes de parallélisme (tellement discutées pour le gapping) à certaines particularités typologiques (complexe verbal riche, pro-drop, ordre libre des mots, marquage casuel, etc.) et, d'autre part, le roumain dispose d'un inventaire des conjonctions plus riche que celui des autres~langues romanes. En particulier, le roumain dispose d'une conjonction particulière \textit{iar} `et', qui obéit à des contraintes spécifiques, dont certaines sont requises de manière indépendante dans les constructions à gapping aussi. De plus, comme cette conjonction lie uniquement des contenus propositionnels, elle est utile dans l'analyse de certaines coordinations elliptiques qui sont ambiguës entre une coordination de phrases ou bien une coordination sous-phrastique. 

En choisissant les VRA comme deuxième construction d'étude, j'ai voulu d'une part étudier l'ellipse dans la subordination et d'autre part examiner une construction qui n'a pas été étudiée avant, mais qui est disponible dans plusieurs langues romanes. 

{\bfseries
Contributions de la thèse}

Sur un plan théorique, les phénomènes elliptiques ne fournissent pas tous des arguments pour supposer une structure syntaxique quelconque pour le matériel manquant. On l'admet pour certains énoncés fragmentaires dans le dialogue (cf. Ginzburg \& \citet{Sag2000}, Ginzburg \& \citet{Cooper2004}, Ginzburg \textit{à paraître}), mais cela est moins évident pour des constructions, comme les deux étudiées dans cette thèse (le gapping et les VRA), qui en apparence manifestent des effets de \textit{connectivité}. Une étude approfondie nous fait découvrir que, dans les deux cas, on ne peut pas aligner ces phrases elliptiques sur leurs contreparties complètes, car (i) parfois la reconstruction syntaxique est impossible, (ii) quand elle est disponible, elle ne peut pas s'appliquer de manière uniforme et systématique à toutes les occurrences elliptiques, et (iii) les propriétés syntaxiques et sémantiques des phrases elliptiques et de leurs contreparties complètes ne sont pas les mêmes. Comme il s'agit d'un mécanisme ad-hoc et superflu, l'hypothèse de la reconstruction syntaxique doit être abandonnée pour ces constructions. 

Le refus de toute approche structurale dans la description de ces constructions implique une perspective différente sur la syntaxe de l'ellipse. Il s'agit d'une syntaxe {\guillemotleft}~plus simple~{\guillemotright} (dans le sens de Culicover \& \citet{Jackendoff2005}), sans effacement, sans éléments vides, sans mouvement. L'unité essentielle pour décrire le comportement de ces phrases elliptiques est le \textit{fragment} (cf. Ginzburg \& \citet{Sag2000}), défini comme une expression dont le contenu sémantique n'est pas déductible de la forme prise en isolation et dépend de l'interprétation d'un antécédent dans le contexte. Le contenu sémantique du fragment dépend du type du fragment (différent pour chaque construction, en raison de contraintes différentes), du contenu sémantique des constituants du fragment et des informations contextuelles. 

La littérature sur l'ellipse est vaste et diverse aujourd'hui, mais la plupart des travaux insistent sur les points théoriques suscités par le phénomène de l'ellipse dans la grammaire et laissent en arrière plan l'établissement et la classification des données. Dans cette thèse, je mets en valeur la description des données ; les points théoriques, bien que nécessaires, ne prennent pas la place de la description. L'établissement des données nous fait découvrir le rôle très important des facteurs non-syntaxiques dans la description des constructions elliptiques : la sémantique (p.ex. le contraste pour le gapping), le discours (p.ex. le type de relations discursives) ou encore la structure informationnelle.

{\bfseries
Méthodes et moyens}

Afin d'arriver à ces résultats, deux préalables se sont avérés nécessaires. D'une part, empiriquement, j'ai eu besoin de mener une étude approfondie du fonctionnement de la phrase simple en roumain et ensuite de la coordination et de la subordination. Bien que le roumain soit une langue relativement bien décrite (Dobrovie-Sorin (1987, 1994), \textit{GALR} (2005), \textit{GBLR} (2010), etc.), il n'existe en effet pas de synthèse générale à jour sur le sujet. D'autre part, théoriquement, j'ai voulu avoir l'image d'ensemble sur le phénomène de l'ellipse, tel qu'il est discuté aujourd'hui et bien situer la problématique. Ces deux préalables nécessaires expliquent l'existence des trois premiers chapitres dans ma thèse : un chapitre sur la phrase simple en roumain, un chapitre sur les phrases liées en roumain et un chapitre sur l'ellipse en général.

Pour ce qui est du cadre formel utilisé, j'ai choisi le modèle de la la grammaire syntagmatique guidée par les têtes (dorénavant HPSG, cf. angl. \textit{Head-driven Phrase Structure Grammar}), dont le fonctionnement sera décrit dans la section \ref{sec:1.4} du chapitre 1.

Dans cette thèse, de manière générale, je distingue autant que possible la partie descriptive de la partie analytique. C'est pour cela que la formalisation en HPSG apparaît généralement en fin de chapitre. 

Pour ce qui est des données examinées dans cette thèse, la plupart sont des exemples construits, soumis au jugement de locuteurs natifs habitant le pays d'origine. Le reste des données constituent des exemples attestés, que j'ai trouvés sur internet, dans les textes de presse ou dans mes lectures. Les données du français figurant dans le chapitre 5 se basent sur des relevés du Corpus Arboré de Paris 7 (Abeillé \textit{et al.} (2003)). 

{\bfseries
Plan de la thèse}

La thèse est composée de cinq chapitres.

Chapitre 1. La phrase simple

Comme les deux constructions elliptiques étudiées mettent en jeu respectivement une coordination et une subordination de phrases, il résulte que l'unité syntaxique phrase est au c{\oe}ur de cette thèse. Je lui dédie donc une partie de ce chapitre, afin de proposer une définition satisfaisante, en prenant en compte les trois dimensions syntaxique, sémantique et pragmatique. Pour être sûre de travailler sur une base empirique solide, je fais mon propre choix d'analyses sur la syntaxe de la phrase roumaine. Les points abordés concernent (i) le statut syntaxique des éléments appartenant au complexe verbal, (ii) la non-réalisation du sujet (y compris le pro-drop), (iii) la distribution du sujet par rapport au verbe et par rapport aux autres dépendants verbaux, et (iv) l'ordre des mots dans la phrase. Le chapitre se clôt avec une brève présentation du cadre formel HPSG et de l'architecture générale de la phrase roumaine dans ce modèle.

Chapitre 2. Les phrases liées

Après un bref aperçu de la phrase simple en roumain, je passe à l'étude des phrases~liées (coordonnées et subordonnées). Je montre que, au-delà du flou terminologique et descriptif, on peut concevoir une bonne description de la coordination et de la subordination sur une base strictement syntaxique, en faisant appel à des tests distinctifs, qui nous permettent d'ailleurs d'opérer une classification plus adéquate des différents types d'introducteurs (conjonctions, adverbes connecteurs ou complémenteurs) qui interviennent dans ces constructions. Je discute ensuite la notion de parallélisme dans la coordination, tant au niveau morpho-syntaxique qu'au niveau sémantico-discursif, pour en conclure qu'aucune corrélation biunivoque ne peut être faite entre la coordination et l'identité morpho-syntaxique des éléments coordonnés ou encore entre la coordination et les relations discursives symétriques. Ensuite, je regarde la coordination et la subordination par rapport à (i) la juxtaposition, (ii) les éléments corrélatifs, et (iii) l'incidence. Ces trois phénomènes me serviront à différents endroits dans le chapitre 4 et 5. Une partie de ce chapitre présente en détails une analyse de la conjonction \textit{iar} `et', qui est le prototype de la coordination contrastive et symétrique en roumain. L'étude sémantique, discursive et syntaxique de cette conjonction me sera très utile pour la compréhension des différentes contraintes qui entrent en jeu dans les constructions à gapping. Le chapitre se termine par l'analyse syntaxique de la coordination, en regardant les relations fonctionnelles qui s'établissent d'une part entre la conjonction et le terme qui le suit, et d'autre part entre les termes coordonnés. 

Chapitre 3. Les phrases elliptiques

Ce chapitre se veut une présentation générale du phénomène de l'ellipse, tel qu'il est étudié dans la littérature. Après avoir délimité ce qu'est l'ellipse, je montre qu'elle ne peut pas toujours être expliquée, comme cela a été proposé, par le principe du moindre effort. Il y a bien des contextes dans lesquels l'ellipse est obligatoire ou encore des contextes dans lesquels la présence de l'ellipse entraîne des propriétés différentes qu'on ne retrouve pas dans leurs contreparties complètes. Je reprends la notion de tête prédicative du chapitre 1 et je montre qu'une telle fonction ne doit pas être nécessairement corrélée à la notion de phrase verbale finie ; elle est pertinente aussi dans les phrases verbales non-finies ou encore dans les phrases averbales. Ensuite, je donne un aperçu des principales constructions autorisant l'ellipse, en les classifiant par rapport à la nature du matériel manquant, le type de contexte syntaxique dans lequel l'ellipse apparaît, ainsi que par rapport à la directionnalité de l'ellipse. Je montre ensuite la difficulté d'une étude sur l'ellipse dans une perspective typologique, quant à la disponibilité des différentes constructions inventoriés et aux problèmes d'identification de certaines constructions. Le reste du chapitre présente le grand débat de la littérature entre les différentes manières d'envisager la résolution de l'ellipse, réduites ici à la reconstruction syntaxique (dans les approches structurales) vs. reconstruction sémantique (dans les approches non-structurales).

Chapitre 4. Les conjoints fragmentaires : le gapping

Le quatrième chapitre est dédié aux coordinations à gapping en roumain. Je montre que la définition classique du gapping, qui retient parmi ses critères la position médiane du verbe manquant et la présence d'un élément résiduel sujet, n'est pas adéquate. Adopter une définition plus large en termes de couverture empirique complique en revanche le travail de délimitation du gapping (à l'intérieur d'une même langue et à travers les langues). D'une part, en roumain et en français, le gapping pose des problèmes de description par rapport à d'autres constructions elliptiques, en particulier les comparatives elliptiques et le stripping. D'autre part, en roumain, les configurations avec le verbe prédicat en position initiale se prêtent a priori à une double analyse : gapping (donc, une coordination au niveau des phrases) ou bien coordination de séquences (et donc une coordination sous-phrastique dans la portée syntaxique d'un prédicat). 

Dans un premier temps, je laisse de côté les configurations ambiguës et je présente les propriétés du gapping dans les contextes non-ambigus (c.-à-d. avec un verbe prédicat en position médiane ou finale). J'observe d'abord les contraintes générales qui s'appliquent au matériel manquant (en particulier, le type d'identité qui s'établit entre le verbe antécédent et le verbe manquant) et ensuite les contraintes auxquelles obéissent les éléments résiduels. Les contraintes de parallélisme, souvent discutées dans la littérature, sont reprises ici en détails. Je montre que le parallélisme le plus strict opère au niveau sémantico-discursif. Sur le plan syntaxique, différentes asymétries peuvent apparaître, à condition que chaque conjoint puisse apparaître seul en lieu et place de la coordination dans son ensemble (selon la généralisation dite {\guillemotleft}~de Wasow~{\guillemotright}, cf. Gazdar \textit{et al.} (1985), Pullum \& \citet{Zwicky1986}). L'importance des différentes contraintes pesant sur le gapping est évaluée pour le roumain à la lumière de la conjonction \textit{iar} `et', spécialisée en roumain pour marquer le contraste.

Une autre partie du chapitre est dédiée aux différentes analyses proposées dans la littérature pour le traitement du gapping, qui peuvent être regroupées en trois approches majeures : reconstruction syntaxique (donc, une ellipse syntaxique) ; mouvement (donc, pas d'ellipse), et reconstruction sémantique (donc, une ellipse sémantique). Après avoir inventorié les différents arguments invoqués dans la littérature pour ou contre l'une de ces approches, je donne des arguments empiriques en faveur d'une approche constructionnelle des coordinations à gapping (c.-à-d. reconstruction sémantique) et contre les approches alternatives en termes de reconstruction syntaxique ou mouvement. Dans cette perspective, la construction à gapping est une coordination entre une phrase finie non-elliptique et une \textit{phrase} non-finie \textit{fragmentaire}. Une formalisation de cette approche est donnée ensuite dans le cadre HPSG. 

Dans~la dernière section du chapitre, je reviens aux configurations ambiguës, dans lesquelles le verbe de la phrase complète se trouve en position initiale. Je montre que le problème de l'ambiguïté est résolu pour les coordinations en \textit{iar} `et', ce type de structures recevant la même analyse que les distributions typiques de gapping. En revanche, pour les coordinations avec d'autres conjonctions, l'analyse est systématiquement ambiguë entre une coordination phrastique (avec ellipse) et une coordination sous-phrastique (sans ellipse).  

Chapitre 5. Les subordonnées fragmentaires : les ajouts relatifs averbaux

Ce dernier chapitre est consacré à l'étude des ajouts relatifs averbaux (VRA) en roumain et en français. Dans un premier temps, je discute leurs propriétés syntaxiques, en regardant la constituance (en particulier, les propriétés distributionnelles de l'introducteur et celles du corps du VRA) et la linéarisation, pour en conclure que les VRA se comportent, dans les deux langues, comme des ajouts incidents par rapport à la phrase hôte. Dans un deuxième temps, je m'intéresse aux propriétés sémantiques des VRA. Je montre que les VRA ont un comportement hybride, car ils ont une interprétation non-intersective (comme les relatives non-restrictives), mais leur contenu fait partie du contenu asserté (comme dans le cas d'une relative restrictive). De plus, leur sémantique est partitive, rendant possible deux types d'interprétations : une interprétation exemplifiante et une interprétation partitionnante. Les deux interprétations me permettent de décrire plus précisement les différences qu'on observe d'un introducteur à l'autre en roumain et en français, ainsi que les préférences des locuteurs pour certains introducteurs dans certaines configurations des VRA.

Après avoir observé les propriétés syntaxiques et sémantiques des VRA, je montre que l'hypothèse selon laquelle les VRA sont dérivés à partir des phrases relatives ordinaires est intenable empiriquement. D'une part, la reconstruction syntaxique d'une forme verbale n'est pas toujours possible et, quand elle est disponible, des contraintes lexicales, syntaxiques ou sémantiques doivent être prises en compte au cas par cas. Donc il n'y a pas de mécanisme général de reconstruction syntaxique qui s'applique à tous les VRA. D'autre part, la contribution sémantique des VRA n'est pas la même que celle de leurs contreparties complètes. Par conséquent, je considère que les VRA ne mettent pas en jeu une ellipse syntaxique et que toutes les différences enregistrées entre les phrases relatives verbales et les VRA s'expliquent par le fait que les VRA sont des \textit{ajouts fragmentaires}. A la fin du chapitre, je montre comment une approche constructionnelle des VRA en termes de \textit{fragments} peut être formalisée dans le cadre HPSG, en utilisant le langage \textit{Minimal Recursion Semantics}, qui permet, entre autres, la description des représentations sémantiques incomplètes. Dans cette partie, je présente aussi les relations fonctionnelles à l'intérieur d'un VRA.

