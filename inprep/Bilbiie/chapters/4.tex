%4

\section{Les conjoints fragmentaires : le gapping}
\subsection{Introduction}
Ce chapitre examine en détail le comportement d'une phrase fragmentaire dans la coordination, à travers la construction à gapping\footnote{Abeillé \& \citet{Mouret2010} proposent le terme français \textit{construction trouée}. Je garde l'étiquette anglaise, afin de faciliter la lecture.}, terme qui apparaît pour la première fois dans les travaux de Ross (1967, 1970).

Je regroupe sous le nom de gapping les constructions dans lesquelles une séquence de syntagmes apparemment sans tête verbale, ayant néanmoins le contenu d'une phrase, se combine avec une phrase complète qui détermine sa forme et son interprétation. Ainsi, en \REF{ex:4:1} on coordonne une phrase ordinaire (\textit{John eats apples}) et une séquence elliptique (\textit{and Mary pears}) à laquelle manque la tête verbale \textit{eats}. Dans la description de ce phénomène, j'utilise les étiquettes suivantes : la phrase elliptique qui présente un \textit{trou} (angl. \textit{gap}) verbal est une \textit{phrase trouée~}; la phrase complète qui contribue à la reconstruction du contenu dans la phrase trouée est la \textit{phrase source~}; le \textit{trou} ou le \textit{matériel manquant} désigne l'élément qui manque dans la phrase trouée\textit{~}; l'\textit{antécédent} est le matériel présent dans la phrase source, qui contribue à l'interprétation de la phrase trouée ; les \textit{éléments résiduels} (angl. \textit{remnants}) font référence aux constituants qui composent la séquence elliptique ; les \textit{éléments corrélats} sont les constituants parallèles aux éléments résiduels dans la phrase complète. 


\begin{enumerate}
\item \label{bkm:Ref287281281}a   (A) John eats apples, and Mary pears.  


\end{enumerate}
  b  (F) Jean mange des pommes et Marie des poires.  

Depuis les travaux fondateurs de Ross (1967, 1970), le gapping a constitué l'objet de beaucoup de recherches.\footnote{Parmi les plus importants figurent les travaux de \citet{Koutsoudas1970}, \citet{Jackendoff1971}, \citet{Maling1972}, \citet{Hankamer1973}, Hudson (1975, 1976, 1989), \citet{Kuno1976}, \citet{Neijt1979}, \citet{Siegel1984}, \citet{Oehrle1987}, \citet{Jayaseelan1990}, Steedman (1990, 2000), \citet{Gardent1991}, \citet{Kehler1994}, Johnson (1996/2004, 2009), Kim (1997, 2003, 2006), \citet{Hartmann2000}, Zoerner \& \citet{Agbayani2000}, \citet{Coppock2001}, Carlson (2001, 2002), \citet{Lin2002}, Winkler \& López (2002), Carlson \textit{et al.} (2005), \citet{Chaves2005}, \citet{Winkler2005}, \citet{Reich2006}, Hernàndez (2006), Zwart (2007, 2009), \citet{Hoyt2008}, \citet{Vicente2010}, etc.} On observe une certaine disproportion quant aux langues étudiées : l'anglais est de loin la langue la plus étudiée, suivi par l'allemand (\citet{Hartmann2000}, \citet{Winkler2005}, \citet{Osborne2006}, \citet{Reich2006}, \citet{Repp2009}), le japonais et le coréen (Kim (1997, 1998), Abe \& \citet{Hoshi1999}, \citet{Lee2005}, \citet{Yosuke2009}). D'autres langues représentées sont : le chinois (\citet{Paul1999}, \citet{Tang2001}, Ruixi \citet{Ressy2008}), l'espagnol (\citet{Tran1999}), le français (Zribi-\citet{Hertz1986}, \citet{Tassin1998}, Abeillé \& \citet{Mouret2010}), le grec classique (Gaeta \& \citet{Luraghi2001}), le latin (\citet{Panhuis1979}), le russe (\citet{Kazenin2001}, Agafonova \textit{à paraître}), le turc (\citet{Ince2009}), et aussi des langues rares comme le quechua (\citet{Pulte1971}), le zapotec (\citet{Rosenbaum1977}), le dargwa et le chuvash (\citet{Kazenin2001}), les langues bantu (Manus \& \citet{Patin2011}). 

Ce chapitre est une monographie du gapping en roumain, faite dans une approche comparative avec le français (et parfois avec l'anglais). L'avantage d'étudier le roumain est double : d'une part, cela permet de combler une lacune dans la description empirique des phénomènes elliptiques en roumain, et d'autre part, cela permet de vérifier la pertinence des contraintes et des analyses postulées pour le gapping dans d'autres langues, afin d'avoir une perspective générale sur le fonctionnement de ce type d'ellipse dans la grammaire. En particulier, le roumain nous permet de confronter les contraintes de parallélisme (tellement discutées pour le gapping) à certaines particularités typologiques (p.ex. ordre libre des mots, pro-drop, marquage casuel, conjonctions spéciales, etc.). Pour cela, je m'appuie sur les travaux de Bîlbîie (2009), Abeillé \& \citet{Mouret2010} et Abeillé, Bîlbîie \& Mouret \textit{à paraître}.

Les données utilisées dans cette étude sont de nature différente. Les distributions simples du gapping ne posent aucun problème d'acceptabilité pour les locuteurs. De façon générale, les données utilisées sont des exemples construits, soumis au jugement des locuteurs natifs. Pour certains aspects cependant, j'ai dû chercher des exemples attestés sur internet dans les textes de presse. 

Le chapitre est organisé de la manière suivante : Dans la section \ref{sec:4.2}, je présente les critères définitoires pour l'identification du gapping par rapport à d'autres constructions réputées elliptiques (en particulier la coordination de séquences, connue sous le nom de \textit{Conjunction Reduction} ou encore \textit{Argument Cluster Coordination}) dans une perspective typologique. Pour les langues à ordre des mots relativement libre, comme le roumain, certaines configurations sont syntaxiquement ambiguës, se prêtant a priori à une double analyse selon que la coordination se place au niveau phrastique ou bien à un niveau sous-phrastique. Dans les trois sections qui suivent (4.3, 4.4 et 4.5), je me concentre sur les configurations non-ambiguës de gapping en roumain (c.-à-d. celles dans lesquelles le verbe antécédent dans la phrase source est en position médiane ou finale). Dans la section \ref{sec:4.3}, j'établis les propriétés du gapping, en discutant dans un premier temps les contraintes générales qui s'appliquent au matériel manquant et aux éléments résiduels, et ensuite les contraintes de parallélisme au niveau syntaxique, sémantique et discursif. La description porte essentiellement sur le roumain, mais la plupart des observations faites sont applicables aussi au français. Lorsque des différences existent entre les deux langues, je les signale. Je synthétise ensuite dans la section \ref{sec:4.4} les analyses proposées à travers les différents cadres théoriques, en insistant sur les problèmes que la plupart de ces analyses n'arrivent pas à résoudre. Ensuite, dans la section \ref{sec:4.5}, je donne une analyse constructionnelle du gapping, en m'inspirant des analyses {\guillemotleft}~non-structurales~{\guillemotright}, qui ne postulent pas de reconstruction syntaxique ou mouvement. Finalement, après avoir décrit en détail les distributions non-ambiguës de gapping, je reviens dans la section \ref{sec:4.6} aux configurations ambiguës relevées dans la section \ref{sec:4.2} (c.-à-d. celles dans lesquelles le verbe se trouve en position initiale), qui se prêtent a priori à deux analyses : gapping ou coordination de séquences. Après avoir invalidé l'hypothèse d'une reconstruction syntaxique, je montre qu'il faut distinguer les configurations avec la conjonction \textit{iar} `et' des autres. Je donne des arguments pour aligner les structures en \textit{iar} `et' sur les cas standard de gapping discutés dans les sections précédentes, on a donc dans ces cas une coordination d'une phrase complète avec un fragment. En revanche, les configurations avec d'autres conjonctions (en particulier celles avec la conjonction \textit{şi} `et') restent ambiguës, n'étant pas incompatibles avec une analyse en termes de coordination de séquences dans la dominance syntaxique d'un prédicat verbal. Une analyse formelle est donnée en fin de chapitre pour rendre compte de cette deuxième possibilité.  

\subsection{Le gapping et l'ordre des mots à travers les langues}
\label{bkm:Ref300921865}De manière générale, les critères minimaux pris en compte pour l'identification des phrases trouées concernent le nombre des éléments résiduels et la catégorie du trou : il s'agit d'une phrase elliptique qui compte au moins deux éléments résiduels et où manque au moins le verbe principal. Le premier critère distingue ainsi le gapping des constructions différées \REF{ex:4:2}a et des ellipses polaires \REF{ex:4:2}b, où la phrase elliptique est réduite à un seul élément résiduel, éventuellement accompagné d'un adverbe.\footnote{Dans la littérature, on trouve le terme \textit{stripping}, mais distributionnellement il réfère à des constructions hétérogènes, c'est pour cela que Abeillé (2006, \textit{en prép.}) utilise des termes différents. Elle distingue le stripping (i) des constructions différées (ii) et des ellipses polaires (iii). Dans le premier cas, il n'y a pas d'ellipse, car il y a toujours un adverbe propositionnel qui constitue la tête de la phrase (cet adverbe propositionnel est souligné dans les exemples en (i)). Dans les deux derniers cas, le deuxième conjoint est elliptique, dans le sens où on doit récupérer une partie du contenu du premier conjoint, pour qu'on puisse interpréter le deuxième conjoint.
(i)  a  Paul viendra, [mais Marie non].
  b  Paul viendra, [mais Marie certainement pas].
  c  Paul ne viendra pas, [mais Marie peut-être].
(ii)  a  Paul viendra, [ou bien Marie].
  b  Paul viendra, [et même Marie].
  c  Paul est venu hier, [mais pas Marie].
(iii)  a  Paul viendra [et Marie aussi].
  b  Paul ne viendra pas [et Marie non plus].} Le deuxième critère sépare le gapping du pseudogapping par exemple, où il y a deux éléments résiduels, mais accompagnés d'un verbe auxiliaire ou modal \REF{ex:4:2}c. 


\begin{enumerate}
\item \label{bkm:Ref287347142}a   John can play the guitar, but not Mary.  


\end{enumerate}
  b  John can play the guitar, and Mary too.

  c  John can play the guitar and Mary can the violin.  

En revanche, ces deux critères ne permettent pas a priori de faire la différence entre le gapping et la coordination de séquences (angl. \textit{Conjunction Reduction} ou \textit{Argument Cluster Coordination}, dorénavant ACC) en \REF{ex:4:3}, surtout si on les regarde dans une perspective typologique. Par conséquent, dans cette section je présente deux aspects dont on doit tenir compte quand on analyse le gapping. Le premier concerne la directionnalité du gapping à travers les langues et le deuxième la position de la tête verbale.


\begin{enumerate}
\item \label{bkm:Ref302128626}a   We play [poker] [at our house], and [bridge] [at Betsy's house]. 


\end{enumerate}
  b  [At our house] we play [poker], and [at Betsy's house], [bridge].

\subsubsection{Directionnalité du gapping}
{\bfseries
Analepse ou catalepse}

\citet{Ross1970} est le premier à établir une corrélation entre la directionnalité du gapping et l'ordre des mots dans une langue, en particulier la position de la tête verbale dans la phrase. Ainsi, les langues à tête non-finale strictement SVO, comme l'anglais \REF{ex:4:4} ou le français \REF{ex:4:5}, ou strictement VSO, comme le gaélique irlandais \REF{ex:4:6}, se caractérisent par la présence de l'analepse ou ellipse progressive (angl. \textit{forward gapping}), ce qui implique que la phrase source précède toujours la séquence elliptique (SVO+SO, *SO+SVO), alors qu'une langue à tête finale strictement SOV, comme le japonais \REF{ex:4:7} ou le coréen, présente la catalepse ou ellipse régressive (angl. \textit{backward gapping}), où la phrase source suit la séquence elliptique (SO+SOV, *SOV+SO). 


\begin{enumerate}
\item \label{bkm:Ref287311583}a  John likes apples and Mary pears.


\end{enumerate}
  b  *John apples and Mary likes pears.


\begin{enumerate}
\item \label{bkm:Ref287311587}a  Jean aime les pommes et Marie les poires.


\end{enumerate}
  b  *Jean les pommes et Marie aime les poires.


\begin{enumerate}
\item \label{bkm:Ref288384400}a  Chonaic  Eoghan  Siobhàn  agus  Eoghnaí  Ciaràn.


\end{enumerate}
voir.\textsc{passe } Eoghan  Siobhàn  et  Eoghnai  Ciaràn

{\itshape
Eoghan a vu Siobhàn et Eoghnaí Ciaràn     } 

  b  *Eoghan  Siobhàn  agus  chonaic  Eoghnaí  Ciaràn.

Eoghan  Siobhàn  et  voir.\textsc{passe } Eoghnai  Ciaràn 

\textit{Eoghan a vu Siobhàn et Eoghnaí Ciaràn     } (\citet[177]{Steedman2000})


\begin{enumerate}
\item \label{bkm:Ref287347982}a  Watakusi  wa  sakana  o,  Biru  wa  gohan  o  tabeta.


\end{enumerate}
\textsc{1sg  top.prt}\textsc{  } poisson  \textsc{acc.prt } Biru  \textsc{top.prt}  riz   \textsc{acc.prt}   a-mangé

\textit{Je mange du poisson et Biru du riz         }   

  b  *Watakusi  wa  sakana  o  tabeta,  Biru  wa  gohan  o.

\textsc{1sg  top.prt}\textsc{ } poisson  \textsc{acc.prt } ate,  Biru  \textsc{top.prt}  riz   \textsc{acc.prt}

\textit{Je mange du poisson et Biru du riz         }  (\citet[251]{Ross1970})

Si la généralisation de Ross est pertinente pour les langues à tête strictement non-finale (voir aussi \citet{Jackendoff1971}, \citet{Lobeck1995}, etc.), elle est loin de capter tous les faits empiriques qu'on observe à travers les langues, en particulier le comportement inattendu de certaines langues à tête finale ou encore les différences qu'on peut trouver parmi les langues à ordre de mots libre. Les langues à tête finale ne rentrent pas toutes dans la généralisation de Ross. Un premier cas de figure est relevé par Hernàndez (2007) pour le persan standard, qui est une langue à tête finale, mais qui, contrairement à des langues comme le japonais, le coréen ou le siouan, permet uniquement le gapping progressif, c.-à-d. SOV+SO, comme on voit en \REF{ex:4:8}.


\begin{enumerate}
\item \label{bkm:Ref287461964}a  {\AE}li    sib  xord  v{\ae}  M{\ae}rzi  hulu.


\end{enumerate}
Ali  pomme  a-mangé  et  Marzo  pêche

\textit{Ali a mangé des pommes et Marzo des pêches}       (Hernàndez (2007 : 2123))

  b  *{\AE}li  sib  v{\ae}  M{\ae}rzi  hulu  xord.

Ali  pomme  et  Marzo  pêche  mange

{\itshape
Ali mange des pommes et Marzo des pêches}

Un deuxième cas de figure est représenté par l'existence de langues à tête finale qui présentent à première vue les deux directions de gapping~(SO+SOV et aussi SOV+SO) : basque, chuvash, hindi, punjabi, turc, les subordonnées en allemand (cf. \citet{Maling1972}, Mallinson \& \citet{Blake1981}, \citet{Kazenin2001}, Hernàndez (2007), \citet{Haspelmath2007}, etc.). Je reprends en \REF{ex:4:9} l'exemple du basque, cité par \citet{Haspelmath2007}.


\begin{enumerate}
\item \label{bkm:Ref287315344}a  Linda-k  ardau  eta  Ander-ek  esnea  edaten  dabez.


\end{enumerate}
Linda-\textsc{erg } vin.\textsc{abs}  et  Ander\textsc{-erg}  lait\textsc{.abs}  boire    \textsc{3pl.3sg} 

{\itshape
Linda va boire du vin et Ander du lait}

 b  Linda-k  ardau  edaten  du,  eta  Ander-ek  esnea.

Linda-\textsc{erg } vin.\textsc{abs}  boire  \textsc{3sg.fut}  et  Ander\textsc{-erg}  lait\textsc{.abs} 

\textit{Linda va boire du vin et Ander du lait     } (Haspelmath (2007 : 42-43))

En allemand, on observe que dans les phrases racine il s'agit plutôt d'un ordre SVO et donc on a uniquement l'analepse, alors que dans les subordonnées, il présente à première vue les deux directions de l'ellipse.


\begin{enumerate}
\item a  Peter  trank  Wein,  und  Marie  Bier.


\end{enumerate}
Peter  a-bu  vin  et  Marie  bière

{\itshape
  Peter a bu du vin et Marie de la bière     } 

  b  *Peter  Wein,  und  Marie  trank  Bier.

Peter  vin  et  Marie  a-bu  bière

{\itshape
Peter a bu du vin et Marie de la bière}


\begin{enumerate}
\item a  Ich  vermute  da{\ss}  Peter  Wein,  und  Marie  Bier  getrunken  hat.


\end{enumerate}
je  suppose  que  Peter  vin  et  Marie  bière  bu  a

{\itshape
  Je suppose que Peter a bu du vin et Marie de la bière     } 

  b  Ich  vermute  da{\ss}  Peter  Wein  getrunken  hat,  und  Marie  Bier.

je  suppose  que  Peter  vin  bu  a  et  Marie  bière

\textit{Je suppose que Peter a bu du vin et Marie de la bière}      (\citet[192]{Crysmann2006})

On a observé le comportement des langues SVO et des langues à tête finale. Mais que se passe-t-il dans les langues à ordre de mots libre ? Là aussi, on ne peut pas parler d'homogénéité des faits. Il y a des langues qui peuvent avoir les deux directions du gapping, avec différents ordres possibles, dont le nombre varie d'une langue à l'autre. C'est le cas du russe, du latin ou du grec classique (qui ont au moins les ordres typiques SVO+SO, SO+SOV, SOV+SO), auxquels on peut ajouter le zapotec qui est encore plus libre (SVO+SO, SO+SVO, SOV+SO, SO+SVO\footnote{\citet{Ross1970} considère qu'il n'y a pas de langue ayant cet ordre dans le gapping. Cette hypothèse s'avère fausse pour le quechua (cf. Pulte (1971, 1973)) et le zapotec (cf. \citet{Rosenbaum1977}).}, OVS+OS, OS+OVS, VSO+SO, *SO+VSO, cf. \citet{Rosenbaum1977}) ou encore le tojolabal qui peut avoir toutes les combinaisons possibles dans les deux sens (cf. \citet{Furbee1974}). Je me limite à un exemple du russe en \REF{ex:4:12}, qui vient de \citet{Ross1970}.


\begin{enumerate}
\item \label{bkm:Ref287463623}a  Ja pil vodu, i Anna vodku.


\end{enumerate}
je bois eau, et Anna vodka

{\itshape
Je bois de l'eau et Anna de la vodka}

 b  Ja vodu, i Anna vodku pila.

je eau, et Anna vodka boit.\textsc{fem}

{\itshape
Je bois de l'eau, et Anna de la vodka}

 c  Ja vodu pil, i Anna vodku.

je eau bois, et Anna vodka 

\textit{Je bois de l'eau, et Anna de la vodka       } (Ross (1970 : 251-252))

Enfin, on observe qu'il y a des langues qui peuvent avoir (plus ou moins) tous les ordres possibles, mais qui permettent uniquement l'ellipse progressive (SVO+SO, SVO+OS, SOV+SO, SOV+OS, VSO+SO, VSO+OS, OVS+OS, OVS+SO). C'est le cas du quechua bolivien, du cherokee (cf. Pulte (1971, 1973)), etc. J'ajoute ici le roumain qui ne permet pas le gapping régressif \REF{ex:4:13}, mais qui est une langue à ordre de mots relativement libre, donc on s'attend à avoir plusieurs ordres qui soient possibles dans les deux conjoints, ce qui s'avère être le cas \REF{ex:4:14}.


\begin{enumerate}
\item \label{bkm:Ref287466258}a  Ion mănâncă mere, iar Maria pere.


\end{enumerate}
Ion mange pommes, et Maria poires

{\itshape
Ion mange des pommes et Maria des poires}

 b  *Ion mere, iar Maria pere mănâncă.

Ion pommes, et Maria poires mange

{\itshape
Ion mange des pommes, et Maria des poires}


\begin{enumerate}
\item \label{bkm:Ref287466272}a  Ion spală vasele,     iar Maria rufele. (SVO+SO)


\end{enumerate}
Ion lave~ vaisselle.\textsc{def.pl}, et Maria linge.\textsc{def.pl}

{\itshape
Ion fait la vaisselle, et Maria la lessive}

 b  Vasele     le     spală Ion, iar rufele     Maria. (OVS+OS)

vaisselle.\textsc{def.pl cl.pl.acc} lave  Ion, et  linge.\textsc{def.pl} Maria

{\itshape
Ion fait la vaisselle, et Maria la lessive}

 c  Mâine  va  spăla Ion vasele,    iar Maria rufele. (VSO+SO)

demain va laver Ion  vaisselle.\textsc{def.pl,} et Maria linge.\textsc{def.pl}  

{\itshape
Demain Ion fera la vaisselle, et Maria la lessive}

 d  (litt.) Ion VAsele le spală, iar Maria RUfele. (SOV+SO)

Ion vaisselle.\textsc{def.pl cl.pl.acc} lave, et Maria linge.\textsc{def.pl}

{\itshape
C'est la vaisselle que Ion fait, et Maria c'est la lessive}

 e  (litt.) Vasele    ION le     spală, iar rufele     MaRIa. (OSV+OS)

vaisselle.\textsc{def.pl} Ion  \textsc{cl.pl.acc} lave,  et  linge.\textsc{def.pl} Maria 

{\itshape
C'est Ion qui fait la vaisselle, et Maria la lessive}

 f  Mâine  va spăla vasele     Ion, iar rufele     Maria. (VOS+OS)

demain va laver vaisselle.\textsc{def.pl} Ion, et  linge.\textsc{def.pl} Maria

{\itshape
Demain Ion fera la vaisselle, et Maria la lessive}

Pour conclure, on ne peut pas postuler un principe universel quant à la directionnalité du gapping. On observe une tendance à avoir le gapping progressif plutôt que régressif, mais, en l'absence d'une étude empirique adéquate, il est difficile de fournir une explication convaincante. A priori, ce fait est susceptible de deux explications : (i) soit il s'agit d'une contrainte interne à la grammaire, qui dérive du paramètre de la dépendance en syntaxe, (ii) soit la contrainte mise en jeu est plutôt externe à la grammaire et doit être mise en relation avec les faits de processing, comme le proposent Mallinson \& \citet{Blake1981}, Gaeta \& \citet{Luraghi2001}, etc. Invoquer les facteurs de processing expliquerait aussi pourquoi on a tendance à utiliser plutôt l'anaphore que la cataphore. \citet[90]{Ramat1987} : {\guillemotleft}~an ellipsis which refers to a constituent not previously introduced, places a heavy burden on short-term memory. [...] It is thus only natural that gapping of what is contextually known should be preferred.~{\guillemotright}  

\subsubsection{Identification du gapping}
\label{bkm:Ref289905212}{\bfseries
Ellipse médiane ou périphérique}

Dans ce paragraphe, je présente les critères définitoires du gapping, nous permettant de distinguer cette construction de la coordination de séquences (ou ACC). 

Comme la plupart des travaux sur l'ellipse ont été consacrés à l'anglais (donc, implicitement aux langues de type SVO), on définit le gapping par rapport à la position du matériel manquant. Par conséquent, on considère souvent que le phénomène du gapping implique un trou en position médiane, c.-à-d. on a au moins deux éléments résiduels qui `encadrent' le matériel manquant, ce qui distingue le gapping d'autres types d'ellipse (\citet{Jackendoff1971}, \citet{Lobeck1995})\footnote{Contrairement à \citet{Jackendoff1971} et \citet{Lobeck1995}, je considère qu'on ne peut utiliser ce critère pour distinguer le gapping et le sluicing, car on peut trouver des exemples avec sluicing où il y a deux éléments résiduels.
(i)  Cineva a lovit pe cineva, dar nu ştiu cine pe cine.
  quelqu'un a frappé \textsc{mrq} quelqu'un, mais \textsc{neg} sais\textsc{.1sg} qui \textsc{mrq} qui
  \textit{Quelqu'un a frappé quelqu'un d'autre, mais je ne sais pas qui a frappé qui}}. \citet{Jackendoff1971}, par exemple, prend ce critère au sérieux et distingue le gapping des structures à {\guillemotleft}~réduction de conjoints~{\guillemotright} (qui, selon lui, regroupe ACC et RNR), dans le dernier cas le matériel manquant n'ayant pas une position médiane (dans les ACC, il a une position initiale ; dans les RNR, il a une position finale). Cependant, pour les langues n'ayant pas un ordre SVO ou pour celles qui ont un ordre de mots relativement libre, il n'est pas toujours évident que gapping soit le terme approprié pour décrire les faits observés (voir aussi \citet{Haspelmath2007}). Dans beaucoup de ces langues (pour lesquelles on peut considérer que le sujet et l'objet sont au même niveau dans la structure syntaxique), une séquence d'au moins deux éléments résiduels sans tête verbale se prête \textit{a priori} à deux analyses possibles, au moins dans certaines de leurs configurations. 

Ce qui nous permet de faire la distinction entre le gapping et une éventuelle coordination de séquences est, à première vue, l'adjacence des éléments (résiduels et corrélats) par rapport à la tête de la phrase complète. Si tous ces éléments suivent ou précèdent la tête, on devrait trouver des arguments empiriques pour décider si c'est du gapping ou bien si c'est une coordination de séquences. Les deux distributions majeures qui \textit{a priori} se prêtent à l'une ou l'autre des deux constructions sont :

i) l'ordre SO+SOV, pour les langues qui permettent la catalepse, comme c'est le cas de l'allemand dans les phrases subordonnées, du japonais ou du coréen. Voir dans ce sens les discussions de \citet{Lee2005} pour le coréen, \citet{Osborne2006} pour l'allemand, (\citet{Yosuke2009} pour le japonais. Voir aussi \citet{Kazenin2001} pour des discussions sur le dargwa et le chuvash.

ii) l'ordre VSO+SO, pour les langues qui permettent l'analepse, voir le russe (\citet{Kazenin2001}), le roumain (Bîlbîie (2010)), etc. 

En revanche, le terme de gapping semble approprié (sans confusion possible avec ACC) pour trois distributions à travers les langues : 

i) position médiane de la tête avec les deux directions du gapping (p.ex. SVO+SO, SO+SVO, etc.) ; 

ii) position finale de la tête, corrélée avec l'analepse (p.ex. SOV+SO) ;

iii) position initiale de la tête, corrélée avec la catalepse (p.ex. SO+VSO).

Je vais illustrer les problèmes liés à l'identification du gapping en utilisant l'exemple de l'allemand. \citet{Haspelmath2007} donne trois exemples en allemand, avec à chaque fois un ordre différent. Selon les distributions inventoriées ci-dessus, on va considérer \REF{ex:4:16}b comme un cas incontestable de gapping, mais quel type envisager pour l'exemple \REF{ex:4:15} ou bien \REF{ex:4:16}a : gapping ou ACC ?\footnote{Pour les subordonnées en allemand, \citet{Maling1972} donne deux arguments qui montrent qu'on a affaire à deux constructions différentes en \REF{ex:4:16}a et \REF{ex:4:16}b : i) on ne peut pas élider uniquement l'auxiliaire dans la distribution SOV+SO, alors que cela est possible pour la distribution SO+SOV ; ii) l'ordre SO+SOV exige des contraintes de linéarisation plus fortes que l'ordre SOV+SO. Voir aussi \citet{Ince2009} qui considère que l'analepse en turc doit être analysée différemment de la catalepse (un des ses arguments est le fait que la catalepse ne permet qu'une identité parfaite entre l'antécédent et le matériel manquant).} 


\begin{enumerate}
\item \label{bkm:Ref287549104}Liebt Julia Romeo und Kleopatra Cäsar ?  


\end{enumerate}
  aime Julia Romeo et Kleopatra Cäsar

  \textit{Est-ce que Julia aime Romeo et Cleopatra Cäsar}      (\citet[44]{Haspelmath2007})


\begin{enumerate}
\item \label{bkm:Ref287549189}a  ... dass Georg Wein und Barbara Bier trinkt


\end{enumerate}
que Georg vin et Barbara bière boit

\textit{... que Georg boit du vin et Barbara de la bière}          

  b  ... dass Georg Wein trinkt und Barbara Bier

que Georg vin boit et Barbara bière 

\textit{    ... que Georg boit du vin et Barbara de la bière }   (\citet[43]{Haspelmath2007})

L'accord permet de définir un test. Ainsi,~l'accord au pluriel peut être un argument pour considérer la structure en question plutôt comme une coordination de séquences, comme c'est le cas en dargwa, une langue ayant l'ordre SO+SOV en \REF{ex:4:17}.


\begin{enumerate}
\item \label{bkm:Ref289274278}dul  mutal,     dil  rasul     malHal[Qalalij \{\textbf{b-}atalRibda \textbf{/*w}-atalRibda\}.  


\end{enumerate}
  1\textsc{erg} Mutal\textsc{.abs, 2erg} Rasul.\textsc{abs} à-Makhachkala\textbf{\textsc{} }\textsc{1pl-}envoyer.\textsc{passé/1sg-}envoyer.\textsc{passé}

  \textit{J'ai envoyé Mutal à Makhachkala, et toi Rasul}         (\citet{Kazenin2001})

Mon hypothèse semble être confirmée par les données du russe et du chuvash. En russe, avec l'ordre VSO+SO, on a accord au pluriel, alors que l'ordre SVO+SO présente plutôt un accord au singulier \REF{ex:4:18}.~Le chuvash, langue SOV qui permet et l'analepse et la catalepse, présente une variation d'accord (singulier ou pluriel) avec l'ordre SO+SOV \REF{ex:4:19}b, mais uniquement l'accord au singulier pour l'ordre SOV+SO \REF{ex:4:19}a.\footnote{Pour d'autres asymétries entre l'analepse et la catalepse en chuvash, voir \citet{Kazenin2001}.}  


\begin{enumerate}
\item \label{bkm:Ref287483754}a  Kolja \{poedet / *poedut\}     zavtra v Moskvu, a Vasja v Petersburg.


\end{enumerate}
Kolja aller.\textsc{fut.}\textbf{\textsc{sg}}\textsc{/} aller.\textsc{fut.}\textbf{\textsc{pl}} demain à Moscou, et Vasja à Petersburg

{\itshape
Kolja ira demain à Moscou, et Vasja à Saint Petersburg}

 b  Zavtra \{poedut / *poedet\} :  Kolja v Moskvu, a Vasja v Petersburg.

demain aller.\textsc{fut.}\textbf{\textsc{pl}}\textsc{/} aller.\textsc{fut.}\textbf{\textsc{sg}} Kolja à Moscou, et Vasja à Petersburg

\textit{Demain Kolja ira à Moscou et Vasja à Saint Petersburg}      (\citet{Kazenin2001})


\begin{enumerate}
\item \label{bkm:Ref287483765}a  Vasja KanaS-a  \{kaja-T / *kaja-C-C-e\},     Petja SupaSkar-a.


\end{enumerate}
Vasja Kanash-\textsc{dat} aller-\textsc{pres.3}\textbf{\textsc{sg}}\textsc{/} aller-\textsc{pres-}\textbf{\textsc{pl}}\textsc{-3,} Petja Cheboksary\textsc{.dat}

{\itshape
Vasja va à Kanash, et Petja à Cheboksary}

 b  Vasja KanaS-a,   Petja SupaSkar-a   \{kaja-T / kaja-C-C-e\}.

Vasja Kanash-\textsc{dat,} Petja Cheboksary\textsc{.dat} aller-\textsc{pres.3}\textbf{\textsc{sg}}\textsc{/} aller-\textsc{pres-}\textbf{\textsc{pl}}\textsc{-3}

\textit{Vasja va à Kanash, et Petja à Cheboksary       }   (\citet{Kazenin2001})

Par conséquent, l'accord au pluriel est possible uniquement si le verbe précède (ou suit) les deux conjoints.

Est-ce que le gapping pose un problème d'identification en français et en roumain, les deux étant compatibles uniquement avec l'analepse ?

Comme le français est une langue à ordre des mots fixe, c.-à-d. généralement SVO, le trou se trouve dans la plupart des cas en position médiane. Cependant, il y a les cas d'inversion du sujet qui se prêtent \textit{a priori} à une double analyse. Voir dans ce sens l'analyse de Mouret (2007, 2008). Les séquences V Cplt Cplt + Cplt Cplt se prêtent aussi parfois à une double analyse (\textit{à Jean un livre} vs. \textit{un livre à Jean}). Voir plus de détails dans la section \ref{sec:4.6}.

Quant au roumain, on peut considérer comme cas de gapping tous les exemples avec tête médiane (SVO+SO ou OVS+OS) et aussi les exemples avec tête finale (SOV+SO ou OSV+OS). Si le verbe est en position initiale (VSO+SO ou bien VOS+OS), il reste à vérifier si c'est du gapping ou de l'ACC. C'est pour cela que la suite de ce chapitre se concentrera tout d'abord sur les cas non-ambigus de gapping, en montrant sa distribution, ses contraintes, les analyses proposées dans la littérature, ainsi que l'analyse la plus adéquate pour rendre compte de ces contraintes, et ensuite finira par une présentation détaillée des cas ambigüs où la tête est en première position, afin de décider le type de construction envisagé : gapping ou ACC. 

Pour conclure, on a observé dans cette section que la perspective typologique rend encore plus difficile l'étude de l'ellipse. Les faits discutés ci-dessus nous montrent qu'il faudrait se méfier de l'adéquation des étiquettes proposées pour certaines constructions elliptiques dans certaines langues et qu'une étude empirique des données devrait être faite avant de postuler l'existence d'un certain type elliptique dans une langue. Ce problème explique le flou terminologique qui existe dans la littérature sur les types d'ellipse et en partie la pléthore d'analyses proposées pour expliquer un même phénomène. 

Deux points ont été abordés dans une perspective typologique : d'une part, la direction de l'ellipse, et d'autre part, le contraste potentiel entre gapping et ACC. On a observé que le gapping ne se comporte pas comme un phénomène uniforme quant à sa distribution à travers les langues. De plus, dans certaines distributions et dans certaines langues, il doit être délimité sur une base empirique des occurrences d'ACC. Donc, les critères doivent être relativisés aux contraintes d'ordre des mots des différentes langues.

\subsection{Propriétés du gapping}
\label{bkm:Ref300921908}Comme le gapping en roumain n'a été jamais décrit, j'insiste surtout sur les données de cette langue ; les données du français seront présentées quand il y a des différences entre le fonctionnement du gapping dans les deux langues. Pour une analyse détaillée du gapping en français, voir Abeillé \& \citet{Mouret2010} et Abeillé, Bîlbîie \& Mouret \textit{à paraître}. En ce qui concerne le roumain, comme je l'ai mentionné plus haut, je commence l'étude du gapping en regardant essentiellement les cas non-ambigus, c.-à-d. les structures dans lesquelles la phrase source a le verbe tête en position médiane ou en position finale.

{\bfseries
\label{bkm:Ref289613798}Contextes phrastiques du gapping}

{\bfseries
Gapping et types de phrase}

Une phrase trouée est composée d'au moins deux syntagmes interprétés comme arguments ou ajouts (d'un verbe antécédent de la phrase source), ayant un contenu propositionnel qui est récupéré à partir de la phrase source. Ce contenu propositionnel apparaît a priori avec tous les types de phrase, bien que les types interrogatif et exclamatif soient plus contraints, en particulier en français. Dans les deux langues, les types déclaratif et désidératif ne posent aucun problème en termes d'acceptabilité (voir les déclaratives \REF{ex:4:20}a et \REF{ex:4:21}a, ainsi que les désidératives \REF{ex:4:20}b et \REF{ex:4:21}b). 


\begin{enumerate}
\item \label{bkm:Ref289033162}a  Ion a cumpărat o carte pentru Dana, iar Petre un stilou pentru Maria.


\end{enumerate}
{\itshape
Ion a acheté un livre pour Dana, et Petre un stylo pour Maria  } 

  b  Mâine pregăteşte-mi o pizza, iar poimâine o friptură de vițel !

{\itshape
Demain cuisine-moi une pizza, et après-demain un rôti de veau} 


\begin{enumerate}
\item \label{bkm:Ref289033177}a   Jean a acheté un livre à Marie et Paul un stylo à Anne.


\end{enumerate}
b  Demain va à la piscine et après-demain au stade ! 

Quant aux types interrogatif et exclamatif, les contextes qui sont parfaitement acceptables sont ceux dans lesquels le syntagme interrogatif ou exclamatif est mis en facteur et prend donc portée large sur la coordination dans son ensemble. D'ailleurs, l'ellipse du verbe est même requise dans ces contextes (c.-à-d. la répétition du verbe dans le deuxième conjoint dégrade significativement l'acceptabilité de la phrase). Un petit point comparatif à noter est le fait qu'en roumain les interrogatives partielles avec un syntagme \textit{qu-} extrait demandent généralement un sujet postverbal\footnote{Le syntagme interrogatif \textit{de ce} `pourquoi' permet le sujet préverbal sous certaines conditions :
(i)  a  [De ce] (Ion) merge (Ion) cu maşina, iar Maria pe jos ?
    \textit{Pourquoi Ion va en voiture, et Maria à pied}
  b  De ce (Ion) vine (Ion) azi (Ion) ?
    \textit{Pourquoi Ion vient aujourd'hui}
  c  De ce (*Ion) vine (Ion) ? 
    \textit{Pourquoi Ion vient}} (comparer \REF{ex:4:22}a-\REF{ex:4:22}b, \REF{ex:4:22}c-\REF{ex:4:22}d pour les interrogatives et \REF{ex:4:23}a-\REF{ex:4:23}b, \REF{ex:4:23}c-\REF{ex:4:23}d pour les exclamatives), alors qu'en français on n'observe pas cette contrainte (c.-à-d. les deux placements du sujet, préverbal et postverbal, sont possibles), cf. \REF{ex:4:24} et \REF{ex:4:25}. Par conséquent, en roumain, les interrogatives ou exclamatives avec la mise en facteur du syntagme \textit{qu-} sont à priori ambiguës quant à l'identification du type de construction : une construction à gapping (donc, une coordination de phrases) ou bien une construction ACC (donc, une coordination de séquences), car le verbe dans ces contextes précède et les éléments corrélats et les éléments résiduels (voir sections 4.2.2 et 4.6.). 


\begin{enumerate}
\item \label{bkm:Ref299612187}a  [Pe cine] sună Ion dimineața şi Maria (*sună) seara ?


\end{enumerate}
  \textsc{mrq} qui appelle Ion le-matin et Maria le-soir 

{\itshape
Qui est-ce que Ion appelle le matin et Maria le soir   } 

  b  *[Pe cine] Ion sună dimineața şi Maria seara ?

    \textsc{mrq} qui Ion appelle le-matin et Maria le-soir

{\itshape
Qui est-ce que Ion appelle le matin et Maria le soir}

 c  [Ce ziar] citeşte Ion dimineața şi Maria (*citeşte) seara ? 

{\itshape
Quel journal lit Ion le matin et Maria le soir   } 

  d  *[Ce ziar] Ion citeşte dimineața şi Maria seara ?

    \textsc{mrq} qui Ion appelle le-matin et Maria le-soir

\textit{Quel journal Ion lit le matin et Maria le soir       }  


\begin{enumerate}
\item \label{bkm:Ref299612221}a  [Ce răbdare] are Maria cu copiii ei şi Ion (*are) cu studenții lui !


\end{enumerate}
{\itshape
Quelle patience a Maria avec ses enfants et Ion avec ses étudiants   } 

  b  *[Ce răbdare] Maria are cu copiii ei şi Ion cu studenții lui !

{\itshape
Quelle patience Maria a avec ses enfants et Ion avec ses étudiants   } 

  c  [Ce oameni săraci] a întâlnit Ion în Dobrogea şi Maria (*a întâlnit) în Moldova !

{\itshape
  Quels gens pauvres a rencontré Ion en Dobrogea et Maria en Moldavie}

  d  *[Ce oameni săraci] Ion a întâlnit în Dobrogea şi Maria în Moldova !

{\itshape
    Quels gens pauvres Ion a rencontré en Dobrogea et Maria en Moldavie}


\begin{enumerate}
\item \label{bkm:Ref299614058}a   [Quels livres] Paul a-t-il lus hier et Marie (*a-t-elle lus) aujourd'hui ? 


\end{enumerate}
  b  [Quels livres] a lus Paul hier et (??a lus) Marie aujourd'hui ?


\begin{enumerate}
\item \label{bkm:Ref299614060}a   [Quelle patience] Paul a montré avec ses enfants et Marie (*a montré) avec ses étudiants ! 


\end{enumerate}
  b  [Quelle patience] a montré Paul avec ses enfants et (??a montré) Marie avec ses étudiants ! 

En revanche, si le syntagme interrogatif ou exclamatif n'est pas mis en facteur (donc, on a un syntagme interrogatif ou exclamatif dans chaque conjoint), les jugements sont très difficiles à faire, en particulier en français. La seule observation qui est claire en français concerne l'acceptabilité des phrases avec des syntagmes interrogatifs sujet \REF{ex:4:28}a. Il reste à préciser les facteurs qui jouent sur les degrés d'acceptabilité qu'on observe.\footnote{Les éléments qu'on pourrait examiner sont : (i) le marquage d'un des éléments résiduels, (ii) la présence d'un sujet dans la séquence trouée et (iii) s'il y a un sujet dans la phrase trouée, vérifier si le placement du sujet dans la phrase source joue un rôle (en particulier, si les phrases avec sujet préverbal sont préférées aux phrases avec sujet postverbal).} 


\begin{enumerate}
\item a  Cine vine azi şi cine mâine ? 


\end{enumerate}
{\itshape
Qui vient aujourd'hui et qui demain  } 

  b  (Mă întreb) Ce cărți citeşte Ion şi ce cărți Maria.

{\itshape
(Je me demande) Quels livres lit Ion et quels livres Maria   } 

  c  De când te-ai sculat tu şi de când eu ? [question de reproche]

{\itshape
    Depuis quand tu t'es réveillé et depuis quand moi          } 


\begin{enumerate}
\item a  \%Ce rochie frumoasă are Ioana şi ce pantaloni demodați soțul ei !


\end{enumerate}
{\itshape
Quelle jolie robe a Ioana et quel pantalon démodé son mari   } 

  b  \%E uimitor  cât  de  frig  e  înăuntru  şi  cât  de  cald   afară !

    est  étonnant  combien  de  froid  est  dedans  et  combien  de  chaud  dehors

{\itshape
  C'est étonnant comme il fait froid à l'intérieur et comme il fait chaud dehors}


\begin{enumerate}
\item \label{bkm:Ref299615593}a   Qui va à Rome et qui à Florence ?


\end{enumerate}
  b  \%Je me souviens quels livres je lisais à 6 ans et quels livres à 8 ans.

  c  \%Quels livres a-t-il donnés à Paul et quels livres à Marie ?

  d  ??Quels livres a lus Paul et quels livres Marie ?


\begin{enumerate}
\item a   ??Quel bonheur Paul a connu à Paris et quelle tristesse Marie à Londres !\footnote{Certaines exclamatives sont parfaitement acceptables, mais dans ces cas-là, la séquence trouée peut apparaître toute seule comme phrase averbale. 
(i)  a  (C'est incroyable) Quelle chance on a eu à Londres et quelle poisse à Berlin !
  b  Quelle poisse à Berlin !
(ii)  a  Quelle tristesse (on voyait) parmi les soldats, mais quelle joie parmi les officiers !
  b  Quelle joie parmi les officiers !} 


\end{enumerate}
b  ??C'est incroyable quelle chance a eu Paul et quelle malchance son frère ! 

{\bfseries
Gapping et les phrases coordonnées}

On considère souvent que le gapping est compatible uniquement avec la coordination, son emploi étant exclu de la subordination. Le gapping apparaît toujours dans des constructions {\guillemotleft}~parallèles~{\guillemotright} du point de vue sémantique et discursif (la notion de parallélisme étant développée par la suite), ce qui explique l'occurrence massive de ce type d'ellipse dans des phrases liées par la coordination ou la juxtaposition. Tout type de coordination y est présent : coordination de phrases racines \REF{ex:4:30}a ou coordination de phrases subordonnées\footnote{Le gapping n'est pas un phénomène réservé aux phrases racines (\textit{contra} Hankamer (1971, 1979), \citet{Ince2009}).} \REF{ex:4:30}b, coordination simple \REF{ex:4:30}a-b ou coordination omnisyndétique ou {\guillemotleft}~corrélative~{\guillemotright} \REF{ex:4:30}c. 


\begin{enumerate}
\item \label{bkm:Ref289076905}a  Ion vine azi, (\textbf{iar}) Maria mâine.


\end{enumerate}
{\itshape
Ion vient aujourd'hui, (et) Maria demain  } 

  b  Mi s-a spus că Ion vine azi, \textbf{iar} Maria mâine.

{\itshape
On m'a dit que Ion vient aujourd'hui, et Maria demain} 

  c  \textbf{Fie} Dan va cânta la vioară, \textbf{fie} Maria la pian.

{\itshape
Soit Dan va jouer au violon, soit Maria du piano      } 

Si l'on a une coordination de subordonnées, la phrase trouée ne peut pas être introduite par un complémenteur. Cela s'explique si l'on suppose que le complémenteur hérite du statut verbal fini de la phrase avec laquelle il se combine (cf. \citet{Godard1989}). Or, la contrainte générale qui pèse sur le matériel manquant est d'omettre au moins la tête verbale (y compris le complémenteur). 


\begin{enumerate}
\item \label{bkm:Ref289277143}a  Mi s-a spus \textbf{că} Ion vine azi şi (*\textbf{că}) Maria mâine.


\end{enumerate}
{\itshape
On m'a dit que Ion vient aujourd'hui et Maria demain  } 

  b  Vreau \textbf{ca} Ion să vină azi şi (*\textbf{ca}) Maria mâine.

{\itshape
Je veux que Ion vienne aujourd'hui et Maria demain} 

Beaucoup d'auteurs (\citet{Jackendoff1971}, \citet{Koutsoudas1971}, \citet{Hankamer1979}, \citet{Wilder1994}, \citet{Johnson2009}, etc.) considèrent qu'il est impossible d'avoir une phrase trouée enchâssée sous une phrase source racine \REF{ex:4:32} ; \citet{Sag1976} notamment observe que le gapping opère uniquement au n{\oe}ud phrastique (S) le plus haut et non à un n{\oe}ud enchâssé. Cependant, \citet{Izutsu2008} note quelques exemples de gapping avec des complémenteurs comme \textit{whereas} ou\textit{ while}, qui semblent être acceptables en anglais \REF{ex:4:33}. 


\begin{enumerate}
\item \label{bkm:Ref289077384}\label{bkm:Ref299639546}a   *Sam played tuba \textbf{whenever} Max sax.          


\end{enumerate}
b  *McTavish plays bagpipe \textbf{despite the fact that} McCawley the contrafagotto d'amore.                  (\citet{Jackendoff1971})


\begin{enumerate}
\item \label{bkm:Ref289077590}a   Men are valued for their economic status, \textbf{whereas} women for their appearance.


\end{enumerate}
b  Boys are encouraged to go out for work, \textbf{while} girls to stay at home. 

{\raggedleft
(\citet[654]{Izutsu2008})
}

On observe les mêmes contraintes en roumain. La phrase trouée ne peut pas être enchâssée~dans une phrase source racine \REF{ex:4:34}. On note cependant quelques exemples acceptables, où le connecteur est un élément analysé comme subordonnant par la tradition grammaticale, mais qui maintient une relation discursive de parallélisme et contraste entre les deux phrases, p.ex. \textit{în timp ce} `alors que' \REF{ex:4:35}.~


\begin{enumerate}
\item \label{bkm:Ref289080977}a  *Maria cântă la vioară, \{\textbf{pentru că / deşi}\} Ion la pian.


\end{enumerate}
{\itshape
Maria joue au violon, \{parce que / quoique\} Ion du piano  } 

  b  *Maria mănâncă o pară, \{\textbf{înainte ca / după ce}\} Ion un măr.

{\itshape
Maria mange une poire, \{avant que / après que\} Ion une pomme} 


\begin{enumerate}
\item \label{bkm:Ref289081035}a  Unele pisici vomită tot timpul, \textbf{în timp ce} altele foarte rar.


\end{enumerate}
{\itshape
Certains chats vomissent tout le temps, alors que d'autres très rarement  } 

  b  Serviciul şi alte îndatoriri exterioare ocupă prea mult timp, \textbf{în timp ce} familia prea puțin.

{\itshape
Le travail et d'autres obligations extérieures prennent trop de temps, alors que la famille très peu} 

Cependant, la question se pose de savoir si ces éléments comme le roumain \textit{în timp ce} `alors que', qui figurent normalement sur la liste des complémenteurs (ou conjonctions de subordination, selon la terminologie traditionnelle), se comportent effectivement comme des subordonnants dans les constructions à gapping (et dans les coordinations en général). Il semble y avoir au moins deux arguments empiriques pour distinguer entre un \textit{în timp ce} subordonnant et un \textit{în timp ce} coordonnant. D'une part, une subordonnée temporelle introduite par \textit{în timp ce} peut précéder et suivre la phrase racine \REF{ex:4:36}, alors qu'une coordonnée introduite par \textit{în timp ce} ne peut pas être antéposée \REF{ex:4:37}. D'autre part, le subordonnant temporel \textit{în timp ce} impose des contraintes sur le temps du verbe (il exige généralement un verbe à l'indicatif imparfait, cf. \REF{ex:4:38}a), alors que le coordonnant \textit{în timp ce} n'impose pas de contrainte particulière \REF{ex:4:38}b). Enfin, on constate que le subordonnant \textit{în timp ce} a une interprétation temporelle (c.-à-d. il marque la simultanéité des événements dans les phrases liées), tandis que le coordonnant \textit{în timp ce} a plutôt un sens abstrait (il marque une relation de parallélisme et contraste entre les phrases, sans qu'il s'agisse de simultanéité temporelle).


\begin{enumerate}
\item \label{bkm:Ref299645088}a  Cineva ne-a spart casa, \textbf{în timp ce} noi eram în vacanță.


\end{enumerate}
{\itshape
Quelqu'un est entré par effraction dans la maison, pendant qu'on était en vacances} 

  b  \textbf{In timp ce} noi eram în vacanță, cineva ne-a spart casa.

    \textit{Pendant qu'on était en vacances, quelqu'un est entré par effraction dans la maison}


\begin{enumerate}
\item \label{bkm:Ref299645112}a  Unele pisici vomită tot timpul, \textbf{în timp ce} altele foarte rar.


\end{enumerate}
{\itshape
Certains chats vomissent tout le temps, alors que d'autres très rarement  } 

  b  \#\textbf{In timp ce} altele vomită foarte rar, unele pisici vomită tot timpul.

    Alors que d'autres vomissent très rarement, certains chats vomissent tout le temps

{\itshape
Certains chats vomissent tout le temps, alors que d'autres très rarement} 


\begin{enumerate}
\item \label{bkm:Ref299645771}a  ??Cineva ne-a spart casa, \textbf{în timp ce} noi am fost în vacanță.


\end{enumerate}
{\itshape
Quelqu'un est entré par effraction dans la maison, pendant qu'on a été en vacances}

  b  Am constatat că unele pisici au reacționat destul de violent la administrarea medicamentului, \textbf{în timp} ce altele n-au reacționat în niciun fel.

    \textit{On a constaté que certains chats ont réagi assez violamment à l'administration du médicament, alors que d'autres n'ont pas réagi du tout}

Pour revenir au gapping dans une subordonnée, on observe qu'en roumain l'enchâssement est possible dans certains contextes avec {\guillemotleft}~amalgamation syntaxique~{\guillemotright} (\citet{Lakoff1974}), où un verbe épistémique comme \textit{a crede} `croire' (à une personne déictique, en particulier à la première personne) ou impersonnel comme \textit{a părea} `paraître' exprime une attitude propositionnelle par rapport au contenu de la phrase trouée. Leur emploi dans ces contextes spécifiques semble être très différent et assez marginal par rapport à l'emploi ordinaire de ces verbes, ce qui impose une analyse syntaxique différente (p.ex. verbes {\guillemotleft}~faibles~{\guillemotright}, cf. Blanche-Benveniste \& \citet{Willems2007}, angl. \textit{grafts}, cf. van \citet{Riemsdijk2006}, ou encore angl. \textit{hedges}, cf. \citet{Lakoff1973}).


\begin{enumerate}
\item a  Andrei a luat cartea şi \textbf{cred că} Marga atlasul.


\end{enumerate}
{\itshape
Andrei a pris le livre et je crois que Maria l'atlas  } 

  b  ??Andrei a luat cartea şi \textbf{mama crede că} Marga atlasul.

{\itshape
Andrei a pris le livre et maman croit que Maria l'atlas  } 

  c  Ion are trei copii şi \textbf{pare-se că} Maria doar unul.

{\itshape
Ion a trois enfants et semble-t-il que Maria seulement un} 

En dehors de ces exemples, la phrase trouée est toujours au même niveau syntaxique que la phrase contenant l'antécédent (\citet{Lobeck1995}).

Revenons aux emplois typiques du gapping dans les phrases coordonnées. On observe que le gapping peut apparaître dans une coordination multiple, avec plusieurs phrases coordonnées. Dans ce cas, plusieurs options se présentent : soit il y a une seule phrase trouée (et en général c'est le dernier conjoint) \REF{ex:4:40}a, soit il y en a plusieurs \REF{ex:4:40}b, coordonnées à une phrase source.\footnote{A priori, l'anglais serait différent, selon Mc\citet{Cawley1988}, qui note que, dans une coordination multiple, tous les conjoints sont troués, excepté le premier.} Les exemples avec une phrase trouée encadrée par deux phrases complètes sont marginaux \REF{ex:4:40}c, mais ils s'améliorent si l'on a plus de trois conjoints \REF{ex:4:41}a-b. 


\begin{enumerate}
\item \label{bkm:Ref289092187}a  La petrecere, Dan a băut bere, Maria a băut vin, iar Ioana suc.


\end{enumerate}
{\itshape
A la fête, Dan a bu de la bière, Maria a bu du vin, et Ioana du jus  } 

  b  La petrecere, Dan a băut bere, Maria vin, iar Ioana suc.

{\itshape
A la fête, Dan a bu de la bière, Maria du vin, et Ioana du jus } 

  c  ?La petrecere, Dan a băut bere, Maria vin, iar Ioana a băut suc.

{\itshape
A la fête, Dan a bu de la bière, Maria du vin, et Ioana a bu du jus}


\begin{enumerate}
\item \label{bkm:Ref289092617}a  Mama vrea o casă, tata o maşină, Ion vrea un câine, iar Maria o pisică.


\end{enumerate}
{\itshape
  A la fête, Dan a bu de la bière, Maria a bu du vin, et Ioana du jus}

  b  \textstyleapplestylespan{O specialitate} \textstyleapplestylespan{o făceam}\textstyleapplestylespan{ cu domnul profesor, alta cu doamna domnului, o alta cu fiul lor, iar alte specialități} \textstyleapplestylespan{le-am făcut}\textstyleapplestylespan{ cu frații, finii şi nora marilor profesori}.\textstyleapplestylespan{} 

\textstyleapplestylespan{   } \textstyleapplestylespan{(phrase librement adaptée de :} \href{http://www.ziare.com/scoala/educatie/seism-in-mijlocul-anului-universitar-1077918}{{www.ziare.com/scoala/educatie/seism-in-mijlocul-anului-universitar-1077918}}\textstyleapplestylespan{)}

{\itshape
Une spécialité on la faisait avec M. le professeur, une autre avec la dame du monsieur, une autre avec leur fils, et d'autres spécialités on les a faites avec les frères, les filleuls et la belle-fille de ces grands professeurs } 

Des exemples plus complexes de coordination multiple apparaissent dans les coordinations récursives (avec des éléments corrélatifs), qui permettent la coordination de phrases trouées entre elles.


\begin{enumerate}
\item a  Dan va cânta la vioară, iar apoi [fie Maria la pian, fie Ion la trompetă].


\end{enumerate}
{\itshape
Dan va jouer du violon, et ensuite soit Maria du piano, soit Ion de la trompette} 

  b  Fie [Ion va merge cu trenul şi Dan cu maşina], fie [Dan cu trenul şi Ion cu maşina].

{\itshape
Soit Ion va prendre le train et Dan la voiture, soit Dan le train et Ion la voiture } 

{\bfseries
Inventaire des coordonnants}

Le choix des conjonctions qui peuvent être utilisées dans les constructions à gapping est conditionné par les contraintes sémantiques et discursives qu'on étudiera dans la section . Le gapping permet l'emploi de toute conjonction qui est compatible avec une relation discursive symétrique, ce qui exclut donc les conjonctions \textit{or} et \textit{car} en français. Toutes les autres conjonctions sont possibles : fr. \textit{et, ou, mais, ni, soit...soit...~}; roum. \textit{şi} `et'\textit{, sau} `ou'\textit{, iar} `et'\textit{, dar} `mais'\textit{, ci} `mais', \textit{fie...fie...} `soit...soit...', ainsi que des éléments comme fr. \textit{ainsi que, comme, que} comparatif\footnote{Voir \citet{Mouret2007} et Mouret \& \citet{Desmets2008} pour des discussions sur le statut conjonctif de ces éléments en français. A noter la différence entre le complémenteur \textit{que} qui n'autorise pas le gapping, et le comparatif \textit{que}, qui apparaît dans une phrase trouée. } ; roum. \textit{ca (şi), precum (şi)} `ainsi que'\textit{, la fel ca} `comme'. 

Les premiers travaux considéraient que le gapping est beaucoup moins acceptable avec \textit{but} en anglais ou \textit{mais} en français. A l'instar de \citet{Repp2009}, on observe que \textit{but} ne pose aucun problème s'il apparaît dans un contexte approprié, certes plus contraint que les autres conjonctions typiques. Il est légitimé surtout par la présence de certains opérateurs sémantiques dans le deuxième conjoint (p.ex. \textit{only, even}) ou bien par une différence de polarité entre la phrase source et la phrase trouée. Ainsi, en roumain, l'adversatif \textit{dar} `mais' est permis dans la phrase trouée, s'il est accompagné d'un adverbe associatif comme l'additif \textit{şi} `aussi' en \REF{ex:4:43}a, ou~le restrictif \textit{numai} `seulement' en \REF{ex:4:43}b, ou bien si la phrase trouée contient un mot négatif qui renverse la polarité par rapport à la phrase source \REF{ex:4:44}. Tous ces éléments (\textit{şi, numai, nici măcar, nimic}) renforcent le contraste et contribuent au mouvement argumentatif de la conjonction adversative \textit{dar} `mais'. 


\begin{enumerate}
\item \label{bkm:Ref289098401}a  Ne e greu, pretențiile sunt mari, \textbf{dar} \textbf{şi} răsplata pe măsură.


\end{enumerate}
{\itshape
C'est difficile pour nous, les exigences sont grandes, mais la recompense aussi à la hauteur } 

  b  Ion schiază pe orice fel de pistă, \textbf{dar} Maria \textbf{numai} pe cele mai uşoare.

{\itshape
Ion skie toutes les pistes, mais Maria seulement les plus faciles}


\begin{enumerate}
\item \label{bkm:Ref289098379}a  Ioanei îi plac toate dulciurile, \textbf{dar} Mariei \textbf{nici măcar} tortul făcut în casă.


\end{enumerate}
{\itshape
Ioana aime toutes les sucreries, mais Maria même pas le gâteau fait maison } 

  b  Băiatul a mâncat ceva, \textbf{dar} fata \textbf{nimic}.

{\itshape
Le garçon a mangé quelque chose, mais la fille rien du tout} 

Le correctif \textit{ci} `mais' est autorisé lui aussi dans les constructions à gapping, à condition que la phrase source contienne la négation (de constituant) \textit{nu} qui a portée uniquement dans la phrase source (cf. \citet{Toosarvandani2011} pour les données de l'anglais). Toujours pour la correction, on peut utiliser la conjonction \textit{şi} avec la négation de constituant \textit{nu} dans la phrase trouée cette fois-ci, alors que la phrase source est positive.


\begin{enumerate}
\item a  \textbf{NU} Ioana\textsubscript{i} îl iubeşte pe Dan\textsubscript{j}, \textbf{ci} EL\textsubscript{j} pe ea\textsubscript{i}.


\end{enumerate}
\textsc{neg} Ioana \textsc{cl} aime \textsc{mrq} Dan, mais lui \textsc{mrq} elle

{\itshape
Ce n'est pas Ioana qui aime Dan, mais c'est Dan qui l'aime } 

  b  DAN\textsubscript{j} o iubeşte pe Ioana\textsubscript{i}, \textbf{şi NU} ea\textsubscript{i} pe el\textsubscript{j}.

    Dan \textsc{cl} aime \textsc{mrq} Ioana, et \textsc{neg} elle \textsc{mrq} lui

{\itshape
C'est Dan qui aime Ioana, et pas l'inverse}

Néanmoins, les conjonctions qui apparaissent le plus souvent avec le gapping en roumain sont \textit{şi} et \textit{iar} `et', avec une fréquence beaucoup plus importante de la dernière conjonction. Si \textit{şi} est assez sous-spécifiée, pouvant être utilisée dans tous les contextes, à tous les niveaux, en dehors du gapping, la conjonction \textit{iar} impose plusieurs contraintes, cf. la section \ref{sec:2.9} du chapitre 2. La préférence de cette conjonction dans les constructions à gapping s'explique essentiellement par la contrainte sémantico-discursive qu'elle impose aux conjoints : \textit{iar} lie (uniquement) des phrases qui présentent au moins deux paires contrastives (c.-à-d. chaque paire réunit des éléments qui ont un \textit{intégrateur commun}, cf. \citet{Lang1984}, et qui marquent une opposition sémantique), dont une est constituée par des topiques (ce qui justifie l'intitulé de \textit{contraste thématique} attribué habituellement à cette conjonction, cf. \citet{Zafiu2005}). Or, on verra dans la section  que la contrainte la plus importante dans le gapping est exactement celle-ci, c.-à-d. le double contraste. On observe ainsi qu'il y a une superposition entre le gapping et la conjonction \textit{iar~}: les deux imposent la même contrainte sémantico-discursive. L'exemple typique avec \textit{iar} est \REF{ex:4:46}a, où l'on a deux paires contrastives obligatoires : une paire d'individus (\textit{Ioana, Maria}) et une paire de fruits (\textit{un măr, o pară}). En dehors du gapping, on observe que la conjonction \textit{iar} est autorisée en \REF{ex:4:46}b, où l'on coordonne des phrases, mais pas en \REF{ex:4:46}c, où l'on coordonne des syntagmes nominaux, bien qu'il y ait un contraste. Cette conjonction spécialisée pour le double contraste en roumain n'existe pas dans les autres langues romanes, mais elle a un correspondant dans les langues slaves (voir le fonctionnement de la conjonction \textit{a} en russe, cf. Jasinskaja \& \citet{Zeevat2009}, \citet{Kazenin2001}, Agafonova \textit{à paraître}, etc.).  


\begin{enumerate}
\item \label{bkm:Ref289110958}a  Ioana a mâncat un măr, iar *(Maria) (a mâncat) o pară.


\end{enumerate}
{\itshape
Ioana a mangé une pomme, et Maria (a mangé) une poire  } 

  b  La Bucureşti plouă, iar *(la Braşov) ninge.

{\itshape
A Bucarest il pleut, et à Braşov il neige } 

  c  Ioana mănâncă mere verzi \{şi / *iar\} pere galbene.

\textit{Ioana mange des pommes vertes et des poires jaunes}  

Comme je l'ai déjà mentionné, la phrase trouée peut ne pas être introduite par une conjonction. Dans certains cas de juxtaposition, on remarque la présence des connecteurs adverbiaux qui rendent explicite la relation discursive~de contraste requise par les constructions à gapping.


\begin{enumerate}
\item \label{bkm:Ref299647036}a  Eu apreciez mai mult valorile spirituale, ea \textbf{însă}, mai mult~pe cele materiale.


\end{enumerate}
{\itshape
Moi, j'apprécie surtout les valeurs spirituelles, elle en revanche, surtout les valeurs matérielles  } 

  b  Studenții erau încântați, profesorii, \textbf{dimpotrivă}, extrem de abătuți.

{\itshape
Les étudiants étaient enchantés, les professeurs, au contraire, extrêmement abattus}

{\bfseries
Antécédent linguistique} 

Les travaux traditionnels considèrent que le matériel manquant dans le gapping ne peut pas avoir un antécédent pragmatique, extra-linguistique (Hankamer \& \citet{Sag1976}, \citet{Chao1988}). C'est pour cette raison que Hankamer \& \citet{Sag1976} considèrent le gapping (tout comme VPE, le sluicing, le stripping, les anaphores en \textit{so} en anglais) comme un cas d'anaphore {\guillemotleft}~de surface~{\guillemotright} (angl. \textit{surface anaphors}), qu'ils distinguent des anaphores {\guillemotleft}~profondes~{\guillemotright} (angl. \textit{deep anaphors}, p.ex. les expressions pronominales ordinaires, les anaphores de complément nul, les anaphores en \textit{do it}) qui peuvent avoir un antécédent situationnel. Leur exemple est donné en \REF{ex:4:48}. Cependant, on pourrait envisager un contexte comme celui décrit en \REF{ex:4:49} pour le roumain, qui pourrait permettre un antécédent extra-linguistique dans les phrases avec deux éléments résiduels. En l'absence d'exemples attestés, je me limite dans ce travail uniquement aux occurrences de gapping avec antécédents linguistiques et je laisse la question ouverte en ce qui concerne l'antécédent non-linguistique. Il faudrait regarder aussi les titres de journaux avec ellipse multiple, où il n'y a pas d'antécédent explicite \REF{ex:4:50}.\footnote{On doit quand même faire attention à ce type d'exemples, pour bien distinguer les phrases elliptiques des phrases non-elliptiques averbales (en l'occurence, en \REF{ex:4:50} le premier exemple semble être un énoncé elliptique, alors que le deuxième exemple ressemble plutôt à une phrase averbale existentielle).}


\begin{enumerate}
\item \label{bkm:Ref289116304}[Hankamer produces an orange, proceeds to peel it, and just as Sag produces an apple, says :] \#And Ivan, an apple.          (Hankamer \& \citet{Sag1976})

\item \label{bkm:Ref289116322}[C'est la fête de Noël. La mère de Marie et Jean entre dans la pièce avec deux grands paquets joliment décorés, et s'adresse d'abord à Marie et ensuite à Jean :] Tu pachetul roşu. [et après quelques secondes] Iar tu pe cel albastru.


\end{enumerate}
\textit{Toi le paquet rouge. Et toi celui bleu}           


\begin{enumerate}
\item \label{bkm:Ref299473004}a  Ieri pe tron, azi în închisoare.


\end{enumerate}
{\itshape
Hier sur le trône, aujourd'hui en prison  } 

  b  Ploi în vestul țării, caniculă în sud.

    \textstyleapplestylespan{(}www.adevarul.ro/actualitate/eveniment/Ploi\_in\_vestul\_tarii-canicula\_in\_sud\_0\_317368576.html)

{\itshape
Des pluies dans l'ouest du pays, de la canicule dans le sud}

{\bfseries
Gapping et les autres constructions elliptiques}

Le gapping est cooccurrent avec d'autres types d'ellipse, ce qui a suscité un grand débat (surtout dans les années 1970-1980) concernant l'extension ou non de la règle de gapping à d'autres séquences elliptiques. A priori, on n'a pas d'arguments pour distinguer le gapping à l'intérieur d'un énoncé, du gapping qui apparaît dans le dialogue, dans les réponses courtes ou dans les questions courtes. 

Contrairement à ce que soutiennent Hankamer \& \citet{Sag1976}, \citet{Williams1977}\footnote{A partir de cette observation (à laquelle s'ajoute la contrainte sur les syntagmes nominaux complexes), \citet{Williams1977} distingue les ellipses bornées (p.ex. le gapping) des ellipses non-bornées (p.ex. l'ellipse du VP) et il propose une règle spécifique à chaque type: les ellipses bornées relèvent de la grammaire de la phrase, alors que les autres doivent être décrites dans une grammaire du discours (\textit{Sentence Grammar} vs. \textit{Discourse Grammar}).}, \citet{Lobeck1995}, etc., on constate que le gapping peut apparaître aussi dans le dialogue, donc il peut intervenir au-delà des limites d'un énoncé, si les locuteurs des énoncés en question communiquent de façon coopérative (\citet{Merchant2001}, en reprenant Sag \textit{et al.} (1985 : 160)).


\begin{enumerate}
\item a  A : Eu vreau să merg la mare.


\end{enumerate}
{\itshape
Je veux aller à la mer } 

    B : Iar eu la munte.

{\itshape
Et moi à la montagne}

  b  A : Trebuie să lucrez, deşi nu prea am spor după o masă copioasă.

{\itshape
Je dois travailler, bien que je ne sois pas très productif après un repas copieux } 

    B : Nici eu după o întâlnire cu Otilia.

{\itshape
Moi non plus après une rencontre avec Otilia} 

Un type d'ellipse proche du gapping est le stripping (appelé aussi BAE, cf. angl. \textit{Bare Argument Ellipsis}). Bien que le terme \textit{stripping} regroupe plusieurs sous-types d'ellipse avec des propriétés complètement différentes (voir, par exemple, Abeillé (2006)), les constructions qui m'intéressent ici sont surtout les ellipses polaires, qui présentent habituellement un élément résiduel, accompagné d'un adverbe {\guillemotleft}~polaire~{\guillemotright} comme les adverbiaux \textit{şi}\footnote{La conjonction \textit{şi} est distincte de l'adverbe additif \textit{şi}, cf. chapitre 2, section \ref{sec:2.3.1} et 2.7.1.} `aussi' et \textit{nici} `non plus' en roumain, et\textit{ aussi} et\textit{ non plus} en français), auxquelles j'ajoute les exemples avec \textit{şi/dar nu} `et/mais pas' en roumain, et \textit{mais pas} en français. 


\begin{enumerate}
\item a  Binele va ieşi biruitor, [şi, odată cu el, \textbf{şi} Dreptatea].


\end{enumerate}
le-Bien va sortir vainqueur, et, une fois avec lui, aussi la-Justice 

{\itshape
Le Bien triomphera et, avec lui, la Justice aussi  } 

  b  Ion nu vrea să aibă copii, [şi \textbf{nici} nevastă-sa].

Ion \textsc{neg} veut avoir.\textsc{subj} enfants, et non-plus sa-femme

{\itshape
Ion ne veut pas avoir d'enfants, et sa femme non plus     } 

  c  Libertate, egalitate, fraternitate -- [dar \textbf{nu [219?]i} pentru romi].

liberté, égalité, fraternité -- mais pas aussi pour les Rroms

{\itshape
Liberté, égalité, fraternité -- mais pas pour les Rroms}


\begin{enumerate}
\item a   Jean viendra à la fête [et Marie \textbf{aussi}].  


\end{enumerate}
  b  Jean n'est pas venu à la fête [et Marie \textbf{non plus}].

  c  Jean est venu hier [mais \textbf{pas} Marie].

Ce genre d'exemples est analysé comme un sous-type de gapping par Hankamer \& \citet{Sag1976}, \citet{Williams1977}, \citet{Chao1988}, \citet{Gardent1991}, \citet{Lobeck1995}, \citet{Hendriks1995}, \citet{Hartmann2000}, \citet{Toosarvandani2011}. En l'absence d'une description adéquate de ces constructions, je ne peux pas me prononcer sur leur lien avec le gapping. En revanche, les cas intermédiaires, avec deux éléments résiduels, ne semblent pas poser de problème particulier pour une analyse à gapping. Je ferai référence à ces exemples à plusieurs reprises dans ce chapitre.


\begin{enumerate}
\item a  Ne e greu, pretențiile sunt mari, [dar \textbf{şi} răsplata pe măsură].


\end{enumerate}
nous est difficile, les-prétentions sont grandes, mais aussi la-recompense sur mesure 

{\itshape
C'est difficile pour nous, les exigences sont grandes, mais la recompense aussi sera à la hauteur      } 

  b  Ion nu merge la film [şi \textbf{nici} Maria la teatru].

Ion \textsc{neg} va à film et ni Maria à théâtre

{\itshape
Ion n'ira pas au cinéma, ni Maria au théâtre}

  c  MaRIa\textsubscript{i} îl     loveşte pe  Ion\textsubscript{j}, [şi \textbf{nu}  el\textsubscript{j} pe  ea\textsubscript{i}].

Maria \textsc{cl.acc} frappe  \textsc{mrq} Ion, et \textsc{neg} lui \textsc{mrq} elle 

{\itshape
  C'est Maria qui frappe Ion, et pas l'inverse     } 

Les structures comparatives constituent un autre type d'ellipse qui permet des séquences qui ressemblent au gapping (cf. Zribi-\citet{Hertz1986}, Culicover \& \citet{Jackendoff2005}, Amsili \& \citet{Desmets2008}). Assimiler les séquences comparatives avec deux éléments résiduels sur les occurrences typiques de gapping est un choix controversé, car on considère souvent que le gapping est compatible uniquement avec les coordonnants ; or, les marqueurs comparatifs ne sont pas habituellement inclus dans la liste des conjonctions. Une solution serait d'élargir l'inventaire des marqueurs de coordination, afin d'inclure aussi les marqueurs comparatifs (Matos \& \citet{Brito2008}). Cependant, cette solution est problématique, car, contrairement aux conjonctions ordinaires qui ne peuvent pas se combiner entre elles, les marqueurs comparatifs sont compatibles avec une conjonction (p.ex. \textit{plus que Paul et plus que Jean}). De plus, même si l'on inclut les marqueurs comparatifs dans la classe des conjonctions, il reste néanmoins des différences notables entre le gapping dans la coordination et le gapping dans les comparatives. \citet{Jackendoff1971}~est le premier qui affirme cela, en se basant sur les différences observées entre \REF{ex:4:55}a et \REF{ex:4:55}b, qui montrent que les constructions comparatives permettent beaucoup plus de types d'ellipse que les constructions coordonnées.


\begin{enumerate}
\item \label{bkm:Ref289120242}a   Bill ate more peaches \textbf{than} \{Harry / Harry did / Harry did grapes / Harry grapes / Harry will grapes\}.  


\end{enumerate}
  b  Bill ate the peaches \textbf{and} \{*Harry / *Harry did / *Harry did the grapes / Harry the grapes / *Harry will the grapes\}.

Il y a d'autres éléments qui suggèrent la souplesse des contraintes sur les constructions comparatives, par rapport à celles agissant dans une structure coordonnée. Si la coordination ne permet que l'analepse \REF{ex:4:56}, l'ellipse peut présenter les deux directions (analepse et catalepse) dans les comparatives \REF{ex:4:57}.


\begin{enumerate}
\item \label{bkm:Ref289120961}a  Ion mănâncă un măr, \textbf{iar} Maria o pară.


\end{enumerate}
{\itshape
Ion mange une pomme, et Maria une poire } 

  b  *Ion un măr, \textbf{iar} Maria o pară mănâncă.

    Ion une pomme, et Maria une poire mange

{\itshape
Ion mange une pomme, et Maria une poire}


\begin{enumerate}
\item \label{bkm:Ref289120973}a  Ion s-a bagat şi el în discuție \textbf{ca} musca în lapte.


\end{enumerate}
{\itshape
Ion s'est mêlé dans la discussion comme une mouche dans le lait } 

  b  Exact \textbf{ca} o muscă în lapte, Ion s-a băgat în discuție pe nepusă masă.

{\itshape
Tout comme une mouche dans le lait, Ion s'est mêlé à la discussion d'une manière imprévisible}

Contrairement aux structures coordonnées, dans une construction comparative on peut avoir une discordance en ce qui concerne le temps, l'aspect ou le mode~ (Mc\citet{Shane2005}) :


\begin{enumerate}
\item a  *Ion pleacă azi, \textbf{iar} Maria ieri.


\end{enumerate}
{\itshape
Ion part aujourd'hui, et Maria hier } 

  b  Ion se comportă cu mine acum \textbf{ca} Maria ieri.

{\itshape
Ion se comporte avec moi maintenant comme Maria hier } 

  c  Maria se uita la mine \textbf{precum} câinele la stăpân.

{\itshape
Maria me regardait comme le chien son maître}

De plus, la contrainte de parallélisme sémantique et contraste est moins stricte dans les comparatives. Les éléments d'une paire contrastive peuvent appartenir à des domaines assez éloignés, p.ex. la paire contrastive \textit{Maria, câinele}, ou encore ils peuvent avoir un statut syntaxique différent, p.ex. affixe pronominal \textit{mă} vs. syntagme nominal.


\begin{enumerate}
\item a  \#Maria ascultă de profesor, \textbf{iar} câinele de stăpân.


\end{enumerate}
{\itshape
Maria écoute son professeur, et le chien son maître } 

  b  Maria ascultă de profesor, \textbf{precum} câinele de stăpân.

{\itshape
Maria écoute son professeur, comme le chien son maître } 


\begin{enumerate}
\item a  *Paul mă iubeşte \textbf{şi} Dan pe Ioana.


\end{enumerate}
{\itshape
Paul m'aime et Dan Ioana } 

  b  Paul mă educă \textbf{precum} dascălul pe elevii lui.

{\itshape
Paul m'éduque comme un maître ses disciples } 

Finalement, on observe qu'une structure comparative elliptique peut être contenue dans son antécédent (\textit{Antecedent Contained Ellipsis}) :


\begin{enumerate}
\item Pentru omenire, zâmbetele sunt [\textbf{precum} soarele pentru flori].


\end{enumerate}
  \textit{Pour les humains, les sourires sont comme le soleil pour les fleurs}

Certes, une étude détaillée devrait être faite pour voir si l'on peut envisager une analyse uniforme pour ces deux structures. Dans cette thèse, je ne me prononce pas sur l'une ou l'autre des approches. Cependant, je considère que la reconstruction syntaxique ne marche pour aucune des deux constructions (voir Amsili \& \citet{Desmets2008} pour une approche similaire des comparatives en français). 

D'autres constructions qui permettent des séquences à deux éléments résiduels en dehors du gapping dans la coordination sont les ellipses appelées sluicing \REF{ex:4:62}a et certaines subordonnées ayant la fonction ajout : ajouts circonstanciels \REF{ex:4:62}b, ajouts additifs \REF{ex:4:62}c, ajouts exceptifs \REF{ex:4:62}d et ajouts relatifs averbaux \REF{ex:4:62}e. 


\begin{enumerate}
\item \label{bkm:Ref289276295}a  Cineva a sărutat pe cineva, dar nu ştiu [cine pe cine].


\end{enumerate}
quelqu'un a embrassé \textsc{mrq} quelqu'un, mais je ne sais pas qui \textsc{mrq} qui

  b  [Deşi pentru prima oară în străinătate], nu-i era deloc dor de țară.

{\itshape
Quoique pour la première fois à l'étranger, son pays ne lui manquait pas du tout}

  c  Toți copiii au adus câte ceva, [inclusiv Maria o prăjitură].

{\itshape
Tous les enfants ont apporté quelque chose, y compris Maria un gâteau}

  d  Niciun elev nu-şi făcuse temele, [mai puțin Ion tema la engleză].

{\itshape
Aucun élève n'avait fait ses devoirs, mis à part Ion le devoir d'anglais } 

  e  Mai mulți prieteni au plecat în străinătate, [dintre care 2 la Roma].

    \textit{Plusieurs amis sont partis à l'étranger, dont 2 à Rome}

Pour conclure, on doit dire que les phrases fragmentaires avec au moins deux éléments résiduels ne sont pas restreintes à la coordination. Cependant, leur occurrence dans la coordination (et juxtaposition) est conditionnée par des contraintes particulières, qui ne s'appliquent pas en dehors de la coordination standard. En particulier, toutes les constructions discutées dans cette sous-section peuvent ne comporter qu'un seul élément résiduel, ce qui n'est pas le cas du gapping dans la coordination. Par conséquent, dans ce chapitre je me limite à une description de ce type d'ellipse en prenant en compte les structures coordonnées. Je reviendrai néanmoins sur un des types d'ajouts mentionnés ci-dessus dans le chapitre suivant (chapitre 5). 

{\bfseries
Contraintes générales sur le matériel manquant}

Minimalement, le matériel manquant doit obligatoirement inclure le verbe tête de la phrase, qu'il s'agisse d'un auxiliaire ou non. Ce qui inclut le gapping dans le groupe des ellipses sans tête, selon la typologie de \citet{Chao1988}.


\begin{enumerate}
\item a  Maria mănâncă un măr, iar Ion o pară.


\end{enumerate}
{\itshape
Maria mange une pomme, et Ion une poire } 

  b  Maria a mâncat un măr, iar Ion o pară.

    \textit{Maria a mangé une pomme, et Ion une poire}

On explique dès lors immédiatement l'impossibilité de garder le complémenteur dans une phrase trouée \REF{ex:4:31}, si l'on admet que les complémenteurs (au moins roum. \textit{că} `que', fr. \textit{que} complétif) héritent du statut verbal fini de la phrase avec laquelle il se combine.

En dehors des formes verbales finies standard, on peut avoir comme matériel manquant un verbe au participe présent (qui se comporte comme une forme finie, cf. la possibilité d'être hôte des clitiques pronominaux).  


\begin{enumerate}
\item Monitorizarea face parte din contractul de consultanță [...], care prevede trei astfel de analize lingvistice, două dintre ele fiind efectuate în martie şi mai, iar următoarea în septembrie, informează NewsIn. (\url{http://www.cna.ro/Revista-Presei-CNA-20-iunie-2008.html})


\end{enumerate}
{\itshape
Le monitorage fait partie du contrat de consultance [...], qui prévoit trois analyses linguistiques de ce type, deux étant effectuées en mars et mai, et la suivante en septembre, informe NewsIn } 

Si le verbe est à un temps composé et contient un auxiliaire, le gapping opère nécessairement sur les deux éléments en roumain \REF{ex:4:65}, ce qui l'oppose à des langues comme le français \REF{ex:4:66}a ou l'anglais \REF{ex:4:66}b, dans lesquelles le trou peut correspondre seulement à un auxiliaire. Cette différence peut être due au statut différent des auxiliaires dans ces langues : statut lexical en français et anglais, statut de clitique en roumain (cf. chapitre 1, section \ref{sec:1.3.1.1}).


\begin{enumerate}
\item \label{bkm:Ref289278270}a  *Maria va citi o poveste, iar Ion recita o poezie.


\end{enumerate}
{\itshape
Maria va lire une histoire, et Ion réciter un poème } 

  b  *Dan a mâncat un sandviş, iar Maria băut o bere.

{\itshape
Dan a mangé un sandwich, et Maria bu une bière } 


\begin{enumerate}
\item \label{bkm:Ref289278283}a   Paul a écrit un roman et Marie fini sa thèse.          


\end{enumerate}
b  Kim will lead the party and Pat bring up the rear.         

En roumain \REF{ex:4:67}, comme en français \REF{ex:4:68}a, les formes verbales composées exigent l'élision de l'auxiliaire, si le participe passé ou l'infinitif est élidé. La situation est différente en anglais \REF{ex:4:68}b, ce qui explique la possibilité du pseudogapping en anglais, mais pas dans les deux langues romanes.


\begin{enumerate}
\item \label{bkm:Ref289279540}a  Ion a mâncat mere, iar Maria (*a) banane.


\end{enumerate}
{\itshape
Ion a mangé des pommes, et Maria (a) des bananes } 

  b  Ion va mânca mere, iar Maria (*va) banane.

    \textit{Ion va manger des pommes, et Maria (va) des bananes}


\begin{enumerate}
\item \label{bkm:Ref289279575}a   Jean a mangé des pommes et Marie (*a) des bananes.          


\end{enumerate}
  b  John will vote for Bush and Mary (will) for Nader.

Le trou peut correspondre à une expression idiomatique\footnote{Si le matériel manquant contient toute l'expression idiomatique, le gapping est parfaitement acceptable. En revanche, si le matériel manquant ne contient qu'une partie de l'expression idiomatique, le gapping est moins acceptable, voire agrammatical, cf. les exemples (i) en français~(\textit{donner lieu} vs. \textit{donner naissance}) signalés par Olivier Bonami.
(i)  a  *La mise en cause de Paul donne \textbf{lieu} à une polémique et ses commentaires \textbf{naissance} à un scandale.
  b   La mise en cause de Paul donne lieu à une polémique et ses commentaires à un scandale.} : \textit{a da foc} litt. `donner feu' pour `mettre le feu' \REF{ex:4:69}a, \textit{a-şi bate joc} litt. `se battre jeu' pour `se moquer' \REF{ex:4:69}b.


\begin{enumerate}
\item \label{bkm:Ref289332014}a  Ca să se amuze în lipsa părinților, băiatul a dat foc grajdului, iar Maria grămezii de coceni.


\end{enumerate}
{\itshape
Pour s'amuser en l'absence de leurs parents, le garçon a mis le feu à l'écurie, et Maria aux tas d'épis  } 

  b  Guvernul îşi bate joc de munca Senatului, iar Senatul de munca Guvernului.

{\itshape
Le Gouvernement se moque du travail du Sénat, et le Sénat du travail du Gouvernement}

A part la tête verbale, on peut omettre d'autres éléments~(sujets \REF{ex:4:70}a, compléments \REF{ex:4:70}b, ajouts \REF{ex:4:70}c) qui peuvent être de même niveau \REF{ex:4:71}a ou enchâssés \REF{ex:4:71}b. Certains ajouts (ou compléments optionnels) présents uniquement dans la phrase source peuvent s'interpréter dans les deux phrases (et dans ce cas ils font partie du matériel manquant) ou bien uniquement dans la phrase source \REF{ex:4:72}. 


\begin{enumerate}
\item \label{bkm:Ref289286824}a  La Valea Leurzii, întemeietorul de şcoală a fost C. Ionescu, iar la Buciumeni, Ion Apostolescu, fiu al satului.


\end{enumerate}
{\itshape
A Valea Leurzii, le fondateur de l'école a été C. Ionescu, et à Buciumeni, Ion Apostolescu, fils du village}

  b  Ion îşi face temele cu mama, iar Maria cu sora ei mai mare.

{\itshape
Ion fait ses devoirs avec sa mère, et Maria avec sa s{\oe}ur aînée  } 

  c  Ion aleargă în parc dimineața, iar Maria seara.

    \textit{Ion court dans le parc le matin, et Maria le soir}


\begin{enumerate}
\item \label{bkm:Ref289286797}a  Mama i-a făcut un cadou Mariei de Paşte, iar tata de Crăciun.


\end{enumerate}
{\itshape
La mère a fait un cadeau à Maria à Pâques, et le père à Noël}

  b  Dan şi-a dorit să înceapă să scrie o nuvelă, iar Maria o piesă de teatru.

    \textit{Dan a voulu commencer à écrire une nouvelle, et Maria une pièce de théâtre}


\begin{enumerate}
\item \label{bkm:Ref289287997}a  Ion merge \textbf{cu familia} la munte, iar Dan la mare.


\end{enumerate}
{\itshape
Ion va avec sa famille à la montagne, et Dan à la mer}

  b  Marin merge \textbf{la schi}\textbf{} în Austria, iar Dan în Germania.

    \textit{Marin va au ski en Autriche, et Dan en Allemagne}

  c  Ion \textbf{întotdeauna} merge la film, iar Maria la teatru.

{\itshape
Ion va toujours au cinéma, et Maria au théâtre  } 

  d  Ion \textbf{tocmai} a sosit acum 5 minute, iar Maria azi-dimineață.

    \textit{Ion vient justement d'arriver il y a 5 minutes, et Maria ce matin}

Le trou ne correspond pas nécessairement à un constituant. Les éléments manquants peuvent être discontinus \REF{ex:4:73} ou en position finale \REF{ex:4:74}. Contrairement au roumain, le français ne permet pas facilement le positionnement final du matériel manquant, ayant une préférence pour une construction (pseudo)clivée \REF{ex:4:75}. L'acceptabilité de ces exemples en roumain s'explique par trois aspects discutés dans le chapitre 1 : i) l'ordre de mots assez libre, ii) le manque d'isomorphisme entre la fonction syntaxique et la position dans l'arbre, et iii) le marquage prosodique des éléments focalisés.


\begin{enumerate}
\item \label{bkm:Ref289289227}a  Ion merge S\^AMbăta la piață, iar Maria duMInica.


\end{enumerate}
{\itshape
Ion va le samedi au marché, et Maria le dimanche}

  b  Ion crede că FRANța va câştiga, iar Maria ArgenTIna.

    \textit{Ion croit que la France va gagner, et Maria l'Argentine}


\begin{enumerate}
\item \label{bkm:Ref289289250}a  In sectorul 4, PoPEScu are şanse să câştige, iar în sectorul 1, PăuNEScu.


\end{enumerate}
\textit{Dans le 4}\textit{\textsuperscript{ème}}\textit{ arrondissement, Popescu a une chance de gagner (les élections), et dans le 1}\textit{\textsuperscript{er}}\textit{, Păunescu}

  b  La noi în casă, păRINții iau deciziile, dar la voi, coPIii.

    \textit{Chez nous, les parents prennent les décisions, mais chez vous, les enfants}


\begin{enumerate}
\item \label{bkm:Ref298089714}a   ??Chez nous, les parents décident et chez vous les enfants.


\end{enumerate}
  b  Chez nous, les parents décident et chez vous ce sont les enfants.

  c  Chez nous, ce sont les parents qui décident et chez vous les enfants.  

On a longtemps considéré que la règle du gapping était bloquée lorsque la phrase source comportait une négation (\citet{Ross1967}). La conclusion des premiers travaux (\citet{Ross1967}, \citet{Jackendoff1971}, Zribi-\citet{Hertz1986}, etc.)~est que la négation ne peut pas être élidée, à moins que la phrase trouée contienne la conjonction \textit{nor} ou \textit{or} \REF{ex:4:76}. 


\begin{enumerate}
\item \label{bkm:Ref289290807}a   *I didn't eat fish and Bill ice-cream.          


\end{enumerate}
b  I didn't eat fish nor/or Bill ice-cream.           (\citet{Sag1976})

\citet{Repp2009} fait une analyse approfondie du comportement de la négation dans les constructions à gapping, en montrant non seulement que la négation est tout à fait acceptable dans ces contextes, mais surtout qu'il y a plusieurs interprétations possibles si on a une négation dans la phrase source. Ainsi, on obtient trois lectures : i) négation distribuée sur les deux conjoints, ii) négation avec portée étroite et iii) négation avec portée large. Les facteurs qui jouent sur l'interprétation de la négation dans le gapping seraient, selon elle,~l'intonation, le type de conjonction utilisée, le type de négation (propositionnelle, pour les deux premières lectures, ou illocutoire, pour la portée large), la présence de certains opérateurs sémantiques dans la phrase trouée ou bien la forme du trou verbal (verbe fini, auxiliaire avec ou sans verbe fini, modal, item à polarité négative). Je montre brièvement la présence de ces trois interprétations de la négation dans le gapping en roumain.

Dans le premier cas, la négation est distribuée sur les deux conjoints : ({\textlnot}A) ${?}$({\textlnot}B). La phrase trouée, tout comme la phrase source, est négative. Selon \citet{Repp2009}, c'est la lecture par défaut dans les constructions à gapping. Au niveau prosodique, chaque conjoint constitue une unité prosodique autonome, et le verbe antécédent ne reçoit pas d'accent prosodique particulier. 


\begin{enumerate}
\item a  La nunta Anei, lui Ion \textbf{nu} i-a plăcut muzica, iar Mariei mâncarea.


\end{enumerate}
{\itshape
Au mariage d'Ana, Ion n'a pas aimé la musique, et Maria la nourriture}

  b  ({\textlnot}A) ${?}$({\textlnot}B) = [Ce n'est pas le cas que Ion ait aimé la musique] et [ce n'est pas le cas que Maria ait aimé la nourriture].

La deuxième interprétation se résume à une portée étroite de la négation : ({\textlnot}A) ${?}$(B).~La négation s'interprète uniquement dans la phrase source, tandis que la phrase trouée est positive. Chaque conjoint est une unité prosodique autonome (cf. \citet{Oehrle1987}). La négation ainsi que les éléments contrastés sont marqués prosodiquement. Cette interprétation est disponible en roumain au moins dans trois contextes : i) de manière générale, dans tous~les emplois de la conjonction corrective \textit{ci} `mais' \REF{ex:4:78}, ii) si la phrase trouée contient un adverbe associatif, comme le restrictif \textit{doar} `seulement' \REF{ex:4:79}a, et iii) si la phrase trouée est introduite par un connecteur adversatif/argumentatif, p.ex. \textit{însă, dar} `mais' \REF{ex:4:79}b.


\begin{enumerate}
\item \label{bkm:Ref302299588}a  \textbf{Nu} Ion\textsubscript{i} o loveşte pe Maria, \textbf{ci} Maria pe el\textsubscript{i}.


\end{enumerate}
{\itshape
Ce n'est pas Ion qui frappe Maria, mais Maria Ion}

  b  ({\textlnot}A) ${?}$(B) = [Ce n'est pas le cas que Ion frappe Maria], mais [c'est le cas que Maria frappe Ion].


\begin{enumerate}
\item \label{bkm:Ref289301786}a  A : - Ce-au cumpărat Ion şi Maria de la târg ? B : - Ion \textbf{n}-a cumpărat mai nimic, iar Maria \textbf{doar} o pereche de papuci.


\end{enumerate}
{\itshape
A : - Qu'est-ce que Ion et Maria ont acheté au foire ? B : - Ion n'a pas acheté grand-chose, et Maria seulement une paire de chaussons } 

  b  A : - La câte întrebări au răspuns Ion şi Maria ? B : - Ion \textbf{n}-a răspuns la aproape nicio întrebare, \textbf{însă} Maria la toate, şi încă fără greşeală.

{\itshape
A : - A combien de questions ont répondu Ion et Maria ? B : - Ion n'a répondu à presque aucune question, mais Maria à toutes, et sans faute}

Enfin, il y a des constructions à gapping avec une {\guillemotleft}~montée~{\guillemotright} sémantique de la négation. Dans ces contextes, la négation (avec modal parfois) dans la phrase source prend portée large sur la coordination dans son ensemble (une seule négation qui porte sur les deux conjoints) : {\textlnot}(A ${?}$B). \citet{Repp2009} considère que la négation dans ce contexte n'est pas une négation propositionnelle (comme dans les premiers deux cas), mais plutôt une négation au niveau illocutoire. De plus, contrairement aux deux autres lectures, la portée large de la négation est corrélée au niveau prosodique avec le fait que les deux conjoints forment une seule unité prosodique (pas de pause possible entre les conjoints), cf. \citet{Oehrle1987}. La négation (plus le modal ou l'auxiliaire en anglais) est prosodiquement proéminente, cf. \citet{Winkler2005}. Cette interprétation est plus difficile à obtenir hors contexte, c'est pour cela que je reprends le modèle de \citet{Repp2009} et je fabrique des contextes pour forcer cette interprétation en roumain.


\begin{enumerate}
\item Kim DIDn't play bingo and Sandy sit at home all night. I am sure Sandy went to a club herself. That's what she always does when Kim plays bingo.  (\citet[171]{Repp2009})  

\item [\textit{Contexte : Ion et Maria sont frère et s{\oe}ur. Il est devenu très riche, mais sa s{\oe}ur est restée très pauvre. Les gens commentent~le fait que Ion n'aide pas sa s{\oe}ur.}]


\end{enumerate}
  a  Ion \textbf{nu poate} locui într-un palat şi Maria într-o cocioabă. Trebuie să facă ceva să-şi ajute sora !

{\itshape
Ion ne peut pas habiter dans un palais et Maria dans une baraque. Il doit faire quelque chose pour aider sa s{\oe}ur } 

  b  {\textlnot}(A ${?}$B) = Ce n'est pas le cas que [Ion habite dans un palais et Maria dans une baraque].

\paragraph[Degré d'identité entre le trou et son antécédent]{Degré d'identité entre le trou et son antécédent}
Avant d'en terminer avec les contraintes générales sur le matériel manquant, on doit préciser le degré d'identité qui s'établit entre le matériel antécédent et le matériel manquant, en relevant d'abord les ressemblances et ensuite les différences. 

\textbf{Ressemblances}~

En ce qui concerne les ressemblances, deux aspects sont très importants. Premièrement, le verbe manquant doit appartenir au même paradigme de flexion que le verbe antécédent et avoir le même sens. Généralement il s'agit du même lexème, sauf dans certains exemples avec des formes homonymes ou avec des zeugmes sémantiques (ou attelages), présents dans quelques citations littéraires, mais rejetés dans l'usage ordinaire, qui ont une lecture ironique incitée par le jeu de mot (voir les exemples français en \REF{ex:4:82}). Dans ces occurrences inattendues, le matériel manquant et le verbe antécédent appartiennent au même paradigme flexionnel, mais avec deux acceptions différentes d'un même terme (ils n'ont pas le même sens et constituent donc deux lexèmes différents). Leur emploi reste cependant très marginal.\footnote{Pour plus de détails sur la description et l'analyse de ces exemples, voir Clément (2011).} 


\begin{enumerate}
\item \label{bkm:Ref302302078}a   Son corps nageait dans l'eau verte, et son esprit dans l'opulence.


\end{enumerate}
                 (Troyat cité par Clément (2011 : 233))   b  La pie vole des bijoux et l'oiseau vers son nid.

On rend ainsi compte de l'inacceptabilité de l'exemple en \REF{ex:4:83}a, où le verbe antécédent et le matériel manquant n'appartiennent pas au même paradigme de flexion (voir la flexion différente \textit{acord} vs. \textit{acordez} au présent en \REF{ex:4:83}b et \REF{ex:4:83}c) et n'ont pas le même sens (\textit{a acorda}\textsubscript{1~}: octroyer une aide à quelqu'un \textit{vs. a acorda}\textsubscript{2~}: régler un instrument musical). 


\begin{enumerate}
\item \label{bkm:Ref299532241}a  \#Pe perioada concediului, eu am acordat ajutoare săracilor, iar soțul meu piane.


\end{enumerate}
{\itshape
Pendant nos congés d'été, j'ai accordé des aides aux pauvres, et mon mari des pianos} 

  b  Eu \textbf{acord} ajutoare săracilor.

{\itshape
J'accorde des aides aux pauvres} 

  c  Eu \textbf{acordez} piane.

{\itshape
J'accorde des pianos}

Les mêmes contraintes s'appliquent aux autres éléments faisant partie du matériel manquant. Ainsi, en \REF{ex:4:84}, où le matériel manquant contient une copule suivie d'un nom prédicatif, il faut que le nom prédicatif du matériel manquant soit du même lexème que le nom prédicatif antécédent.


\begin{enumerate}
\item \label{bkm:Ref299484419}a  *Filip e frate cu directorul, iar Maria cu secretara.


\end{enumerate}
Filip est frère avec le-directeur, et Maria avec la-secrétaire

{\itshape
Filip est le frère du directeur et Maria (est la s{\oe}ur) de la secrétaire}

  b  *Maria e soră cu secretara, iar Filip cu directorul.

    Maria est s{\oe}ur avec la-secrétaire, et Filip avec le-directeur

{\itshape
Maria est la s{\oe}ur de la secrétaire, et Filip (est le frère) du directeur} 

  c  Filip e frate cu directorul, iar Florin (e frate) cu secretara.

    Filip est frère avec le-directeur, et Filip (est frère) avec la-secrétaire

{\itshape
Filip est le frère du directeur et Florin (est le frère) de la secrétaire}

  d  Maria e soră cu secretara, iar Ioana (e soră) cu directorul.

    Maria est s{\oe}ur avec la-secrétaire, et Ioana (est s{\oe}ur) avec le-directeur

{\itshape
  Maria est la s{\oe}ur de la secrétaire, et Ioana (est la s{\oe}ur) du directeur} 

Deuxièmement, ils doivent partager les mêmes propriétés de temps\footnote{On parle ici du gapping typique de la coordination. On a vu précédemment que les comparatives autorisaient une différence de temps entre les événements des deux conjoints.}, mode, voix\footnote{En ce qui concerne la voix, le gapping se distingue de VPE (en anglais), le dernier type d'ellipse permettant l'alternance passive-active ou active-passive, cf. \citet[31]{Merchant2009}.
(i)  a  This problem was to have been looked into, but obviously nobody did {\textless}look into this problem{\textgreater}.   b  The janitor should remove the trash whenever it is apparent that it needs to be {\textless}removed{\textgreater}. }  et aspect. Ainsi, en \REF{ex:4:85}, on ne peut pas avoir de temps ou de modes différents. De même, les exemples en \REF{ex:4:86} montrent qu'une discordance de voix (passive-active ou bien active-passive) est possible s'il n'y a pas ellipse ; en revanche, l'emploi du gapping impose une identité de voix entre les deux conjoints.


\begin{enumerate}
\item \label{bkm:Ref289297914}a  *Ion a sosit ieri, iar Maria mâine.


\end{enumerate}
{\itshape
Ion est arrivé hier, et Maria demain}

  b  *Ion ar merge la film azi, iar Maria ieri.

    \textit{Ion irait au cinéma aujourd'hui, et Maria hier}


\begin{enumerate}
\item \label{bkm:Ref289298087}a  Ion a fost muşcat de un câine, iar pe Maria *(a muşcat-o) o şopârlă.


\end{enumerate}
Ion a été mordu par un chien, et \textsc{mrq} Maria a mordu-\textsc{cl} un lézard

{\itshape
Ion a été mordu par un chien, et Maria par un lézard}

  b  Pe Maria a muşcat-o o şopârlă, iar Ion *(a fost muşcat) de un câine.

    \textsc{mrq} Maria a mordu-\textsc{cl} un lézard, et Ion a été mordu par un chien 

    \textit{Maria a été mordue par un lézard, et Ion par un chien}

Pour les langues qui présentent des marques aspectuelles, comme le russe, on observe qu'on doit avoir le même aspect dans une construction à gapping : en russe, le gapping n'accepte pas une discordance imperfectif vs. perfectif dans les deux conjoints.


\begin{enumerate}
\item *Wtchera ja pisala    pismo dwa tchasa, a ty napisala pismo  za dri tchasa. 


\end{enumerate}
  hier     je écrire.\textsc{past.dur} lettre 2 heures,  et tu écrire.\textsc{past.perf} lettre en 3 heures

  \textit{Hier j'ai écrit une lettre pendant 2 heures et toi en 3 heures}        (\citet[9]{Repp2009})

Selon \citet{Repp2009}, les propriétés TAM (temps-aspect-mode) ont une fonction d'ancrage référentiel dans le monde factuel. La phrase trouée est une phrase non-ancrée, qui a besoin d'emprunter son ancrage à la phrase source. Ce rôle d'ancrage empêche ainsi les TAM d'avoir des valeurs différentes dans la phrase trouée par rapport à la phrase source.

\textbf{Différences} 

Du côté des différences, on observe que le matériel manquant ne possède pas nécessairement les mêmes propriétés de personne \REF{ex:4:89}a ou de nombre \REF{ex:4:89}b.


\begin{enumerate}
\item a  Eu vreau un ceai, iar Ioana (vrea) o cafea.


\end{enumerate}
Je veux un thé, et Ioana (veut) un café

{\itshape
Je veux un thé, et Ioana un café}

  b  Noi citim o carte, iar tu (citeşti) un ziar.

    nous lisons un livre, et tu (lis) un journal

{\itshape
Nous lisons un livre, et toi un journal}

Quant au genre, on observe une certaine asymétrie en fonction de la catégorie qui apparaît dans la composition du matériel manquant, et cela dans les deux langues (roumain et français).\footnote{L'asymétrie en genre est par ailleurs possible dans d'autres constructions elliptiques. Voir \citet{Merchant2011b} pour une discussion sur ce type d'asymétrie avec l'ellipse nominale en grec.} S'il s'agit d'un verbe au participe (\REF{ex:4:89}a et \REF{ex:4:90}a) ou d'un adjectif prédicatif (\REF{ex:4:89}b et \REF{ex:4:90}b), on n'a pas nécessairement le même genre. En revanche, si le matériel manquant contient un nom prédicatif, il doit généralement avoir le même genre que le nom prédicatif dans la phrase source. Les asymétries de genre qui sont permises concernent uniquement les formes syncrétiques \REF{ex:4:91}a ou les formes homophones \REF{ex:4:91}b-c en français, ou bien les noms de métier qui permettent l'emploi du masculin pour les deux genres (\REF{ex:4:92}a et \REF{ex:4:93}a). Si le matériel antécédent contient un nom prédicatif sous sa forme au féminin, le matériel manquant ne peut pas correspondre à un nom prédicatif au masculin (\REF{ex:4:92}b et \REF{ex:4:93}b). 


\begin{enumerate}
\item \label{bkm:Ref289299506}a  Fata e iubită de toți, dar băiatul (nu e iubit) de nimeni.


\end{enumerate}
la-fille est aimée par tous, mais le-garçon (\textsc{neg} est aimé) par personne

{\itshape
La fille est aimée par tous, mais le garçon par personne}

  b  Maria e încântată de noua ei rochie, iar Ion (e încântat) de noua lui maşină.

{\itshape
Maria est enchantée de sa nouvelle robe, et Ion de sa nouvelle voiture}


\begin{enumerate}
\item \label{bkm:Ref298175054}a   La lettre a été écrite par la secrétaire et le mail (a été écrit) par le directeur. 


\end{enumerate}
  b  Jean est content de son travail et Marie (est contente) de ses vacances.


\begin{enumerate}
\item \label{bkm:Ref298175121}a   Jean est secrétaire dans un garage BMW à Paris et Maria (est secrétaire) dans un collège à Lyon. 


\end{enumerate}
  b  Jean est ami avec Paul et Marie (est amie) avec Sophie.

  c  Marie est amie avec Sophie et Jean (est ami) avec Paul.


\begin{enumerate}
\item \label{bkm:Ref298175408}a  Doru e profesor la un liceu teoretic din Cluj, iar Ana la o şcoală generală din Iaşi.


\end{enumerate}
{\itshape
Doru est professeur dans un lycée théorique de Cluj, et Ana dans une école élémentaire de Iaşi} 

  b  ??Ana e profesoară la o şcoală generală din Iaşi, iar Doru la un liceu teoretic din Cluj.

{\itshape
Ana est professeur dans une école élémentaire de Iaşi, et Doru dans un lycée théorique de Cluj} 

  c  Doru e \{profesor {\textbar} *profesoară\}.

    Doru est \{professeur {\textbar} professeur\textsc{.fem\}}

{\itshape
Doru est professeur} 

  d  Ana e \{profesor {\textbar} profesoară\}.

    Ana est \{professeur {\textbar} professeur\textsc{.fem\}}

{\itshape
  Ana est professeur} 


\begin{enumerate}
\item \label{bkm:Ref298175412}a   Patrick est directeur de l'UFR et Marie du laboratoire. 


\end{enumerate}
  b  *Marie est directrice du laboratoire et Patrick de l'UFR.

  c  Patrick est \{directeur {\textbar} *directrice\}.

  d  Marie est \{directrice {\textbar}~directeur\}.

Ce comportement rappelle la distinction entre flexion inhérente et flexion contextuelle, due à Booij (1994, 1996, 2007).\footnote{Merci à Olivier Bonami pour ce commentaire.} Le premier terme fait référence à la flexion d'un mot qui n'est pas demandée par le contexte syntaxique (p.ex. le genre sur les noms, le pluriel des noms, les marques de temps sur un verbe), alors que le deuxième terme caractérise toute flexion qui est dictée par le contexte syntaxique dans lequel un mot apparaît (p.ex. les marques d'accord en genre sur un adjectif, l'accord entre un verbe et un sujet, le marquage casuel). Pour résumer les différences liées aux marques de personne, nombre et genre dans les constructions à gapping, on peut donc dire que le gapping maintient la flexion inhérente, mais pas nécessairement la flexion contextuelle.

Deuxièmement, le matériel manquant ne prend pas nécessairement les mêmes affixes pronominaux ou adverbiaux. On a deux cas de figure : soit les affixes en question, s'ils sont reconstruits avec un verbe dans la phrase trouée, n'ont pas la même forme que les affixes dans la phrase source \REF{ex:4:94}, soit ces affixes ne sont présents que dans la phrase source \REF{ex:4:95}a, ou bien dans la phrase trouée \REF{ex:4:95}b.


\begin{enumerate}
\item \label{bkm:Ref289300406}a  Ion \textbf{l}-a văzut pe Dan, iar Ana (a văzut-\textbf{o}) pe Maria.


\end{enumerate}
Ion \textsc{cl.masc-}a vu \textsc{mrq} Dan, et Ana (a vu-\textsc{cl.fem) mrq} Maria

{\itshape
Ion a vu Dan, et Ana Maria}

  b  Eu \textbf{i}-am văzut pe [Ion şi Maria], iar Ana (\textbf{l}-a văzut) pe Paul.

    je \textsc{cl.pl-}ai vu \textsc{mrq} Ion et Maria, et Ana (\textsc{cl.sg}-a vu\textsc{) mrq} Paul

{\itshape
J'ai vu Ion et Maria, et Ana Paul}


\begin{enumerate}
\item \label{bkm:Ref289301071}a  Maria \textbf{le}-a citit pe toate, dar Ana ((*\textbf{le}-)a citit) doar câteva.


\end{enumerate}
Maria \textsc{cl.acc-}a lu \textsc{mrq} toutes, mais Ana (\textsc{cl.acc-}a lu) seulement quelques-unes

{\itshape
Maria les a tous lus, mais Ana seulement quelques-uns}

  b  Ion a citit câteva dintre ele, dar Maria (*(\textbf{nu}) a citit) absolut niciuna.

    Ion a lu quelques-unes parmi elles, mais Maria (\textsc{neg} a lu) absolument aucune

{\itshape
Ion en a lu quelques-uns, mais Maria absolument aucun} 

Les affixes ont le même comportement en français (cf. Abeillé, Bîlbîie \& Mouret \textit{à paraître}) :


\begin{enumerate}
\item a   Paul \textbf{en} a lu seulement certains, mais Marie (\textbf{les} a) presque tous (lus).      


\end{enumerate}
b  Paul \textbf{les} a lus, vos livres, et Marie (\textbf{en} a lu) seulement certains.      


\begin{enumerate}
\item a   Paul a lu tous vos livres et Marie (\textbf{en} a lu) quelques-uns.      


\end{enumerate}
  b  Paul en a lu certains, et Marie (*(\textbf{n}')en a lu) absolument aucun.

Troisièmement, la polarité dans les deux conjoints n'est pas toujours la même : la phrase source peut être positive, alors que la phrase trouée est négative, ou vice-versa (avec la portée étroite de la négation uniquement dans la phrase source). Pour le deuxième cas de figure, voir les exemples mentionnés ci-dessus en \REF{ex:4:79}.


\begin{enumerate}
\item a  De ce unii au totul, iar eu (\textbf{nu} am) nimic ?


\end{enumerate}
{\itshape
Pourquoi certains ont tout, et moi (je n'ai) rien}

  b  Eram la un metru depărtare de un vampir care ştia prea multe despre mine, \emph{\textup{iar eu (}}\emph{\textbf{\textup{nu}}}\emph{\textup{ ştiam) nimic}} despre el.

{\itshape
J'étais à un mètre d'un vampire qui savait trop de choses sur moi, et moi (je ne savais) rien sur lui} 

  c  Tu ai primit mereu cadouri de ziua ta, dar eu (\textbf{nu} am primit) nimic.

{\itshape
Tu as toujours reçu des cadeaux pour ton anniversaire, mais moi (je n'ai) rien (reçu)}

  d  Emil a învățat enorm pentru admitere, dar Maria (\textbf{nu} a învățat) mai nimic pentru permisul de conducere.

{\itshape
Emil a appris énormément de choses pour son admission, mais Maria (n'a) pas (appris) grand-chose pour son permis de conduire} 

  e  Mariei îi plac doar merele şi perele, însă Ioanei (\textbf{nu}-i plac) nici măcar astea.

\textit{Maria aime seulement les pommes et les poires, mais Ioana (n'aime) même pas celles-ci} 

{\bfseries
\label{bkm:Ref299907500}Contraintes générales sur les éléments résiduels}

Contrairement à d'autres types d'ellipse, la séquence trouée dans les coordinations à gapping doit comporter au moins deux éléments résiduels\footnote{A ne pas confondre l'exemple \REF{ex:4:99} avec l'exemple (i) qui contient un seul élément résiduel accompagné obligatoirement d'un adverbial \textit{şi} (et de la conjonction \textit{şi}). A priori, les propriétés (distribution, intonation, etc.) ne semblent pas être les mêmes dans les deux cas, d'où l'hypothèse selon laquelle dans l'exemple (i) on a une ellipse polaire plutôt que du gapping.
(i)  Ioana  mănâncă  un  măr,  şi  şi  Maria.
  Ioana  mange  une  pomme,  et  aussi  Maria
\textit{Ioana mange une pomme et Maria aussi}.} , mis en correspondance avec des éléments parallèles dans la phrase source.  


\begin{enumerate}
\item \label{bkm:Ref298090742}[Ioana] mănâncă [un măr], \{iar {\textbar} şi\} [Maria] [*(o pară)].


\end{enumerate}
{\itshape
Ioana mange une pomme, et Maria une poire } 

La séquence trouée peut contenir plus de deux éléments résiduels \REF{ex:4:100}. Certains travaux sur le gapping en anglais (\citet{Jackendoff1971}, \citet{Kuno1976}, \citet{Haspelmath2007}, etc.) considèrent que le gapping permet strictement deux éléments résiduels. Cependant, on observe que, même en anglais, la présence des éléments résiduels multiples n'est pas bloquée par une contrainte grammaticale (cf. \citet{Kuno1976}, Sag \textit{et al.} (1985), Zribi-\citet{Hertz1986}, \citet{Steedman1990}).


\begin{enumerate}
\item \label{bkm:Ref289338286}a  De Paşte, părinții au mers la mare cu bunicii, iar copiii la munte cu prietenii.


\end{enumerate}
{\itshape
A Pâques, les parents sont allés à la mer avec les grand-parents, et les enfants à la montagne avec leurs amis}

  b  Seara, Ion vorbeşte cu prietena lui pe Skype, iar Maria cu amantul pe Messenger.

    \textit{Le soir, Ion parle avec sa copine sur Skype, et Maria avec son amant sur Messenger} 

Les jugements d'acceptabilité sont sensibles à des contraintes psycholinguistiques sur le traitement de l'information. En particulier, selon Zribi-\citet{Hertz1986}, la restriction sur le nombre de constituants relève de la performance. De manière générale, les exemples qui posent un problème d'acceptabilité ont des séquences elliptiques composées uniquement de syntagmes nominaux.\footnote{Voir, dans ce sens, Sag \textit{et al.} (1985 : 157) :~{\guillemotleft}~processing difficulty associated with sequences of NPs found in ellipsis contexts~{\guillemotright}.} Leur traitement est difficile sous deux aspects. D'abord, une séquence elliptique composée uniquement de syntagmes nominaux est beaucoup plus difficile à traiter qu'une séquence où les éléments résiduels reçoivent un marquage morpho-syntaxique (cas, préposition, etc.). C'est ce qui expliquerait les différences dans l'acceptabilité des exemples \REF{ex:4:101}a-b en anglais : \REF{ex:4:101}b est meilleur, car les trois éléments résiduels ont chacun un marquage différent~(aucun marquage pour le premier, la préposition \textit{with} pour le deuxième, la préposition\textit{ about} pour le troisième). Comme le roumain est une langue à marquage casuel et prépositionnel, les séquences à trois éléments résiduels ne posent pas de problème particulier \REF{ex:4:102}. 


\begin{enumerate}
\item \label{bkm:Ref289336107}a   *Millie will send the President an obscene telegram, and [Paul] [the Queen] [a pregnant duck].              (\citet[25]{Jackendoff1971})


\end{enumerate}
  b  Some talked with you about politics and [others] [\textbf{with} me] [\textbf{about} music].

{\raggedleft
                    (\citet[193]{Winkler2005}) 
}


\begin{enumerate}
\item \label{bkm:Ref289338424}a  Ion i-a dat Mari\textbf{ei} o carte, iar Dan Ioan\textbf{ei} un buchet de flori.


\end{enumerate}
Ion \textsc{cl.dat}-a donné Maria.\textsc{dat} un livre, et Dan Ioana.\textsc{dat} un bouquet de fleurs 

{\itshape
Ion a offert à Maria un livre, et Dan à Ioana un bouquet de fleurs}

  b  Eu am vorbit cu Ion despre Maria, iar [tu] [\textbf{cu} Dan] [\textbf{despre} Ana].

\textit{J'ai parlé avec Ion au sujet de Maria, et toi avec Dan au sujet de Ana}  

Il semble aussi que les séquences avec des éléments résiduels de même type sémantique (p.ex. individus) peuvent poser plus de problèmes d'acceptabilité que les séquences avec des éléments résiduels de type différent. Ainsi, l'acceptabilité de l'exemple \REF{ex:4:103}a est dégradée à cause d'un nombre important de noms d'individus. Les deux aspects que je viens de mentionner montrent l'importance des facteurs non-syntaxiques dans l'acceptabilité des constructions à gapping avec plus de deux éléments résiduels. Ces facteurs sont responsables aussi d'autres {\guillemotleft}~violations~{\guillemotright} qu'on observe dans certains contextes de gapping (en particulier, les contraintes de localité discutées plus loin), ce qui nous oblige à réfuter une approche purement syntaxique de ce type d'ellipse.  


\begin{enumerate}
\item \label{bkm:Ref289336111}a   *Arizona elected Goldwater Senator, and Massachussets McCormack Congressman.


\end{enumerate}
{\raggedleft
       (\citet[25]{Jackendoff1971})
}

  b  John calls Mary `une imbécile', and Peter Ann `une cruche'.

La phrase trouée prototypique contient généralement un élément résiduel correspondant à un sujet dans la phrase source, donc la séquence typique est sujet-complément ou sujet-ajout. Mais, étant donné qu'on peut omettre d'autres éléments, en dehors de la tête verbale (c.-à-d. sujets, compléments, ajouts) et que le roumain a un ordre de mots assez libre, les éléments résiduels peuvent avoir des fonctions~différentes par rapport au verbe antécédent. Ainsi, la phrase trouée peut être une séquence complément-sujet \REF{ex:4:104}a, ajout-sujet \REF{ex:4:104}b, complément-complément \REF{ex:4:104}c, ajout-complément \REF{ex:4:104}d, complément-ajout \REF{ex:4:104}e.\footnote{Pour une discussion sur la position préverbale dans la phrase simple en roumain, voir le chapitre 1. Je considère que la position préverbale dans les exemples \REF{ex:4:104} ne correspond pas à une fonction syntaxique spécifique (p.ex. \textit{antéposé/extrait}, voir l'exemple (i) en français), mais plutôt à une fonction discursive (c.-à-d. en l'absence d'un marquage prosodique particulier, le premier élément est interprété généralement comme un topique au niveau discursif). 
(i)  A Pierre, Paul a promis d'apporter un disque et à Marie un livre.} 


\begin{enumerate}
\item \label{bkm:Ref289342410}a  Mariei îi plac fructele, iar Ioanei prăjiturile.


\end{enumerate}
Maria.\textsc{dat cl.dat} plaisent les-fruits, et Ioana.\textsc{dat} les-gâteaux

{\itshape
Maria aime les fruits, et Ioana les gâteaux}

  b  Ieri a venit Ion, iar azi Maria.

    hier est venu Ion, et aujourd'hui Maria

{\itshape
Hier, c'est Ion qui est venu, et aujourd'hui, c'est Maria  } 

  c  Mariei i-am dat o carte, iar Ioanei un stilou.

    Maria.\textsc{dat cl.dat}-ai donné un livre, et Ioana.\textsc{dat} un stylo

    \textit{A Marie je lui ai donné un livre, et à Ioana un stylo}

  d  Dimineața, mănânc cereale, iar seara fructe.

    le-matin, mange.\textsc{1sg} céréales, et le-soir fruits

{\itshape
Le matin, je mange des céréales, et le soir des fruits  } 

  e  Engleza o învăț la şcoală, iar franceza acasă.

    l'anglais \textsc{cl.acc} apprends.\textsc{1sg} à école, et le-français à-la-maison

    \textit{L'anglais je l'apprends à l'école, et le français à la maison}

En dehors des deux éléments résiduels, la séquence trouée peut contenir un adverbe de phrase, qui modifie la séquence en entier et qui sémantiquement prend comme argument une proposition ou un acte illocutoire, dans le cas des énonciatifs et de certains connecteurs (Bonami \& \citet{Godard2005}). Ainsi, en roumain, la séquence trouée peut comporter un connecteur \REF{ex:4:105}a, un adverbe modal \REF{ex:4:105}b, évaluatif \REF{ex:4:105}c ou énonciatif \REF{ex:4:105}d. Cela est un argument pour considérer la séquence trouée comme ayant un contenu propositionnel. 


\begin{enumerate}
\item \label{bkm:Ref289361673}a  Ion va interpreta o piesă la pian, iar \textbf{apoi} Maria la vioară.


\end{enumerate}
{\itshape
Ion va interpréter une pièce au piano, et ensuite Maria au violon}

 b  Ion vine azi, iar Maria \textbf{probabil} mâine.

{\itshape
Ion vient aujourd'hui, et Maria probablement demain}

  c  La examen, Ion a luat nota 10, iar Maria, \textbf{din nefericire}, nota 2.

    à examen Ion a pris la-note 10, et Maria, par malheur, la-note 2

{\itshape
Ion a eu 10/10 à l'examen et Maria malheureusement 2/10 } 

  d  Cu Ion am reuşit să discut ceva, dar cu Maria, \textbf{sincer}, nimic.

    avec Ion j'ai réussi discuter.\textsc{subj} quelque-chose, mais avec Maria, franchement, rien

    \textit{J'ai réussi à discuter quelque chose avec Ion, mais avec Maria, franchement, rien}

{\bfseries
Les éléments résiduels sont des constituants majeurs}

Minimalement les éléments résiduels sont des projections maximales (\citet{Sag1976}, \citet{Hartmann2000}, etc.), ce qui explique l'agrammaticalité des exemples suivants, où le deuxième élément résiduel ne peut pas former de syntagme à lui tout seul (un déterminant sans nom \REF{ex:4:106}a ou une préposition transitive sans son complément \REF{ex:4:106}b). En revanche, si l'on substitue aux éléments en question un pronom \REF{ex:4:107}a ou une préposition utilisée intransitivement \REF{ex:4:107}b (qui chacun forme un syntagme unaire), les exemples ne posent aucun problème de grammaticalité.\footnote{\citet{Hartmann2000} et \citet{Repp2009} considèrent que les prépositions ne peuvent pas être des constituants majeurs en allemand. Mais il faudrait faire la différence entre les prépositions toujours transitives et les prépositions qui peuvent avoir un emploi intransitif.}  


\begin{enumerate}
\item \label{bkm:Ref298091401}a  *Lui Ion îi place acest costum, iar Mariei acel.


\end{enumerate}
Ion.\textsc{dat cl.dat} plaît ce costume, et Maria.\textsc{dat det.dem}

{\itshape
Ion aime ce costume-ci, et Maria celui-là      } 

  b  *Maria îşi pune geanta sub masă, iar Ion pe.

    Maria \textsc{cl.refl} pose la-serviette sous la-table, et Ion sur

{\itshape
Maria pose sa serviette sous la table, et Ion au-dessus}


\begin{enumerate}
\item \label{bkm:Ref298091439}a  Lui Ion îi place acest costum, iar Mariei, acela.


\end{enumerate}
Ion.\textsc{dat cl.dat} plaît ce costume, et Maria.\textsc{dat, pro.dem}

{\itshape
Ion aime ce costume-ci, et Maria, celui-là      } 

  b  Maria îşi pune geanta sub masă, iar Ion, deasupra.

    Maria \textsc{cl.refl} pose la-serviette sous la-table, et Ion, au-dessus

{\itshape
Maria pose sa serviette sous la table, et Ion, au-dessus}

Essentiellement, ces éléments résiduels doivent être des \textbf{constituants majeurs}, c.-à-d. des arguments ou ajouts d'une tête verbale (ou prédicative) dans la phrase source. Par conséquent, on ne peut pas avoir comme élément résiduel une sous-partie d'un constituant majeur (Hankamer (1971, 1973), \citet{Neijt1979}, \citet{Gardent1991}, \citet{Hartmann2000}, etc.), dépendant d'une tête non-verbale et non-prédicative, p.ex. un nom sans son déterminant \REF{ex:4:108}a, un syntagme nominal sans sa tête prépositionnelle\footnote{A priori, certains locuteurs acceptent les exemples sans préposition en anglais (p.ex. Culicover \& \citet{Jackendoff2005}), alors que la plupart les considèrent agrammaticaux (\citet{Gardent1991}). \citet{Chaves2005} considère qu'il y a une gradience dans l'acceptabilité de ces exemples. 
(i)  a  *Jim reads a book to Fred, and Mary, Peter. 
  b  ?John is going to Japan, and his sister, Australia.
  c  Jim reads to his brother, and Mary, our kids.        (\citet[11]{Chaves2005}) }  \REF{ex:4:108}b, un syntagme nominal (ou un syntagme prépositionnel) complément d'un autre syntagme nominal \REF{ex:4:108}c, un syntagme adjectival ajout à un syntagme nominal \REF{ex:4:108}d, etc. 


\begin{enumerate}
\item \label{bkm:Ref289362271}a  Paul a mâncat \textbf{o} portocală, iar Maria *(\textbf{o}) banană.


\end{enumerate}
{\itshape
Paul a mangé une orange, et Maria une banane  } 

  b  Maria vorbeşte \textbf{cu} un avocat, iar Ion *(\textbf{cu}) o actriță.

{\itshape
Ion parle avec un avocat, et Ion avec une actrice      } 

  c  Ion citeşte \textbf{introducerea} unui roman, iar Ana *(\textbf{introducerea}) unui eseu.

{\itshape
Ion lit l'introduction d'un roman, et Ana l'introduction d'un essai    } 

  d  Ion şi-a vândut \textbf{maşina} albastră, iar Maria *(\textbf{maşina}) roşie.

    Ion \textsc{cl.refl-aux} vendu la-voiture bleue, et Maria (la voiture) rouge

{\itshape
Ion a vendu sa voiture bleue et Maria sa voiture rouge} 

Les exemples qui semblent violer cette contrainte sont en fait de faux contre-exemples, car les éléments résiduels qui à première vue semblent être des dépendants d'une tête nominale et non-prédicative peuvent être en fait réanalysés comme des compléments du verbe tête. Ainsi, en \REF{ex:4:109}a, le syntagme adjectival \textit{dulce} `douce' n'est pas un ajout à la tête nominale \textit{ciorba} `la soupe' dans la phrase source, mais un attribut de l'objet et du verbe.


\begin{enumerate}
\item \label{bkm:Ref288742730}a  Mariei îi place ciorba acră, iar lui Ion, dulce.


\end{enumerate}
Maria.\textsc{dat cl.dat} plaît la-soupe aigre, et Ion.\textsc{dat,} douce

{\itshape
Maria aime la soupe aigre, et Ion douce      } 

  b  Mie îmi place ciocolata cu mentă, iar lui, cu stafide.

    moi\textsc{.dat} \textsc{cl.dat} plaît le-chocolat avec menthe, et lui\textsc{.dat,} avec raisins-secs

{\itshape
Moi, j'aime le chocolat à la menthe, et lui aux raisins secs}

Dans la plupart des cas, tous les éléments résiduels dépendent de la même tête qui est à la fois verbale et prédicative, c.-à-d. le verbe racine. Cependant, il y a des cas où l'un des éléments résiduels dépend (i) d'une tête non-verbale mais prédicative, ou bien (ii) d'une tête verbale mais enchâssée. 

(i) Le premier cas est illustré par les constructions à prédicat complexe. Le deuxième élément résiduel peut dépendre d'un complément du verbe racine, si celui-ci est un verbe attributif \REF{ex:4:110} ou un verbe support \REF{ex:4:111}. Ces exemples ne posent pas de problème pour la condition de constituant majeur, car l'élément résiduel est (ré)analysé dans ces cas comme un complément du verbe tête, via l'héritage ou la {\guillemotleft}~composition d'arguments~{\guillemotright} (cf. Abeillé \& \citet{Godard2003}). La condition générale qui doit être remplie est que les éléments résiduels doivent être légitimés par \textbf{une} des têtes prédicatives de la phrase source, tout en respectant l'ordre licite de mots dans la grammaire.


\begin{enumerate}
\item \label{bkm:Ref288745854}a  Tatăl e foarte mândru de fiul lui, iar mama, (foarte mândră) de fiica ei.


\end{enumerate}
{\itshape
Le père est très fier de son fils, et Marie (très fière) de sa fille  } 

  b  Cel din stânga mea e premierul pentru criză, iar cel din dreapta, (premierul) pentru haos. (\textit{Dilema veche,} VII\REF{ex:4:341})

{\itshape
Celui à ma gauche est le ministre pour la crise, et celui à ma droite (le ministre) pour le chaos}

  c  Unii devin dependenți de exercițiile fizice, alții, (dependenți) de curele de slăbire.

{\itshape
Certains deviennent dépendants aux exercices physiques, d'autres (dépendents) aux cures d'amincissement} 


\begin{enumerate}
\item \label{bkm:Ref288745870}a  Hasan a făcut o călătorie la Mecca, iar Elena, (o călătorie) la muntele Athos.


\end{enumerate}
{\itshape
Hasan a fait un voyage à la Mecque, et Elena, (un voyage) au Mont Athos  } 

  b  Autostrada face o curbă la stânga, iar drumul național, (o curbă) la dreapta.

{\itshape
L'autoroute fait un virage à gauche, et la route nationale (un virage) à droite}

  c  Decebal a dus lupte aprige cu romanii, iar Ștefan cel Mare cu turcii.

{\itshape
Decebal a mené de vives luttes contre les Romains, et Stefan le Grand contre les Turcs } 

  d  Băiatului i-a pus numele Flavius, iar fetei Dorina.

{\itshape
Au garçon on lui a donné le nom (de) Flavius, et à la fille Dorina}

(ii) Un deuxième cas est illustré par les exemples dans lesquels un des éléments résiduels ne dépend pas directement du verbe racine, mais d'un verbe enchâssé. L'exemple classique est celui de \citet{Ross1970} en \REF{ex:4:112}, avec plusieurs infinitifs enchâssés en anglais. En roumain, l'exemple typique de verbe enchâssé est le subjonctif, qui, dans l'état actuel de la langue, prend de plus en plus la place de l'infinitif (\textit{GALR} (2005 : 392)). Le gapping opère donc facilement à travers une complétive non-marquée par un complémenteur, comme c'est le cas des subordonnées au subjonctif introduites par \textit{să} en roumain (on a justifié dans le chapitre 1, section \ref{sec:1.3.1.2}, que le roumain \textit{să} n'est pas un complémenteur, mais une marque flexionnelle).


\begin{enumerate}
\item \label{bkm:Ref289362573}I want to try to begin to write a novel and Mary a play.      (\citet[250]{Ross1970})

\item a  Ion încearcă să intre la drept, iar Maria la medicină.


\end{enumerate}
 Ion essaie \textsc{mrq} entrer\textsc{.subj} à droit, et Maria à médecine

{\itshape
Ion essaie de se faire admettre à la faculté de droit, et Maria à la faculté de médecine } 

  b  Ion pare să fie bolnav, iar Maria obosită.

    Ion semble \textsc{mrq} être\textsc{.subj} malade, et Maria fatiguée

{\itshape
Ion semble être malade, et Maria fatiguée}

  c  Dan şi-a dorit să înceapă să scrie o nuvelă, iar Maria un poem.

Dan \textsc{cl.refl} a désiré \textsc{mrq} commencer.\textsc{subj} \textsc{mrq} écrire.\textsc{subj} une nouvelle, et Maria un poème

{\itshape
Dan a voulu commencer à écrire une nouvelle, et Maria un poème}

Un cas plus complexe toujours en lien avec les verbes enchâssés concerne les vraies subordonnées, introduites par un complémenteur. Si l'on accepte comme élément résiduel un dépendant d'un verbe enchâssé à l'infinitif (ou au subjonctif en roumain), les choses ne sont pas claires avec les autres types d'enchâssement, qui sont plus complexes. Ainsi, \citet{Koutsoudas1971}, \citet{Hankamer1979}, \citet{Wilder1994}, Johnson (1996/2004), \citet{Williams1997}, etc. considèrent qu'on ne peut pas avoir d'élément résiduel enchâssé dans une complétive en \textit{that} (\textit{que} en français, \textit{că} `que' en roumain). \citet{Gardent1991} donne des exemples acceptables en anglais \REF{ex:4:114}, mais elle considère que la grammaticalité est liée au statut syntaxique de la subordonnée par rapport au verbe racine : les complétives en \textit{that} vs. les circonstancielles ajouts (comparer \REF{ex:4:114}a-b et \REF{ex:4:114}c). Cependant, un regard attentif des données laisse entrevoir des exemples qui sont acceptables en roumain (et en français).  


\begin{enumerate}
\item \label{bkm:Ref288750752}a   This doctor said that I should eat salmon and that doctor tuna.


\end{enumerate}
  b  The child insisted that she wanted chips and the mother salad.

c  *John left without telling his boss and Bill his colleagues.  (\citet{Gardent1991})

Je me limite ici aux discussions sur les complétives en angl. \textit{that}, fr. \textit{que} et roum. \textit{că}. Les autres exemples d'enchâssement seront mentionnés dans la section \ref{sec:4.4.3.1}, où je discute les contraintes de localité. On trouve un exemple classique dans l'{\oe}uvre de Diderot, avec un gapping multiple, où le deuxième élément résiduel est toujours un dépendant d'un verbe enchâssé dans une complétive en \textit{que}. 


\begin{enumerate}
\item Et les voilà embarqués dans une querelle interminable sur les femmes ; l'un prétendant qu'elles étaient bonnes, l'autre méchantes : et ils avaient tous deux raison ; l'un sottes, l'autre pleines d'esprit : et ils avaient tous deux raison ; l'un fausses, l'autre vraies : et ils avaient tous deux raison ; [...] l'un folles, l'autre sensées, l'un grandes, l'autre petites : et ils avaient tous deux raison. (Diderot, \textit{Jacques le fataliste et son maître})


\end{enumerate}
Il est vrai que les jugements d'acceptabilité ne sont pas clairs. Mais ce qui a été observé et en anglais et en français laisse penser que ce n'est pas une contrainte syntaxique qui rend compte de ce genre d'exemples, mais des facteurs sémantiques et psycholinguistiques. En français, le gapping à travers les complétives en \textit{que} est tout à fait acceptable si le sujet enchâssé est un clitique explétif \REF{ex:4:116}a, ou un clitique référentiel \REF{ex:4:116}b, et assez dégradé si le sujet enchâssé est un syntagme nominal \REF{ex:4:116}c. De même, \citet[113]{Merchant2001}, \citet{Lasnik2006} et \citet{Repp2009}~observent qu'en anglais l'élément résiduel peut être dépendant d'un verbe enchâssé dans une complétive en \textit{that} si le sujet racine et le sujet enchâssé sont coréférents (voir les données en \REF{ex:4:117} et \REF{ex:4:118}).


\begin{enumerate}
\item \label{bkm:Ref288750980}a   Paul \{dit / pense\} qu'il faut aller à Rome et Marie à Florence.


\end{enumerate}
b  Paul dit qu'il est allé à Rome et Marie à Florence.

c  ??Paul dit que l'orage a détruit la culture de seigle et Marie la culture de blé.    


\begin{enumerate}
\item \label{bkm:Ref288751323}a   Jim said that he called his mum and John his dad.


\end{enumerate}
b  *Jim claimed that Alan went to the ballgame and John to the movies. 

                        (\citet[12]{Repp2009})  


\begin{enumerate}
\item \label{bkm:Ref288751334}a   John\textsubscript{i} thinks that he\textsubscript{i} will see Susan and Harry Mary.


\end{enumerate}
b  *John said you kissed Mary, and Bill Mary.        (\citet{Lasnik2006})

Quant au roumain, on peut avoir des enchâssées dont le sujet n'est pas coreférent au sujet racine, mais dans la plupart de ces exemples, il s'agit d'un sujet {\guillemotleft}~inclus~{\guillemotright} dans la flexion verbale (c.-à-d. pro-drop). Le pro-drop du sujet joue probablement un rôle dans leur acceptabilité. Le seul problème est qu'en roumain ces exemples, en l'absence d'un contexte spécifique, posent un problème d'ambiguïté concernant le niveau auquel opère le gapping : au niveau de la phrase racine ou bien au niveau des phrases subordonnées. Hors contexte particulier, on a une préférence pour un gapping plus {\guillemotleft}~bas~{\guillemotright}, dans la subordonnée uniquement, mais il reste à vérifier quel type de facteur joue sur cette préférence (en particulier, s'il s'agit d'une contrainte grammaticale ou si c'est lié à la facilité du processing). Si on arrive à avoir un bon contexte, avec une intonation particulière, corrélée parfois avec la juxtaposition, on arrive à établir le parallélisme avec la phrase racine, comme en \REF{ex:4:119}.


\begin{enumerate}
\item \label{bkm:Ref288752581}(Am vorbit şi cu medicul cardiolog şi cu cel nutriționist.) Cardiologul mi-a spus că ar trebui să mănânc mai multe lipide, nutriționistul, mai multe glucide. (Nu mai înțeleg nimic de la medicii ăştia !)


\end{enumerate}
Le-cardiologue m'a dit que devoir.\textsc{cond mrq} manger\textsc{.subj} plus de lipides, le nutritionniste, plus de glucides

{\itshape
(J'ai discuté et avec le cardiologue et avec le nutritionniste.) Le cardiologue m'a dit que je devrais manger plus de lipides, le nutritionniste, plus de glucides. (Je ne comprends plus rien à cette histoire)}

Les mêmes contraintes non-syntaxiques qu'on avait invoquées pour expliquer les différences d'acceptabilité pour le gapping avec plus de deux éléments résiduels, peuvent rendre compte de l'acceptabilité des exemples en \REF{ex:4:120} : l'exemple \REF{ex:4:120}a est meilleur que l'exemple \REF{ex:4:120}b, car les éléments résiduels ne sont sémantiquement pas de même nature (humain \textit{vs}. pays) et, de plus, il met en jeu les connaissances encyclopédiques du locuteur, ce qui facilite l'organisation des éléments résiduels et corrélats en paires contrastives.  


\begin{enumerate}
\item \label{bkm:Ref289364183}a  Ion crede că Franța va câştiga, iar Maria Argentina.


\end{enumerate}
{\itshape
Ion pense que la France va gagner, et Maria l'Argentine}

  b  ??Ion crede că Ana va câştiga, iar Maria Ioana.

    \textit{Ion pense que Ana va gagner, et Maria Ioana} 

Pour conclure, les éléments résiduels doivent pouvoir être mis en correspondance avec des constituants majeurs dans la phrase source, et en particulier avec les dépendants d'un verbe racine ou enchâssé. A priori, tous les problèmes qu'on rencontre avec l'enchâssement ne sont pas d'ordre syntaxique, mais plutôt sémantique et/ou psycholinguistique, ce qu'on observera aussi avec les contraintes de localité.

{\bfseries
\label{bkm:Ref289439628}Contraintes de parallélisme}

\label{bkm:Ref289631781}Syntaxe

Un des arguments majeurs qu'on mentionne habituellement en faveur d'une reconstruction syntaxique dans les constructions elliptiques est la présence des effets de~{\guillemotleft}~connectivité~{\guillemotright} discutés dans le chapitre 3, section \ref{sec:3.5.1.1}, c.-à-d. un parallélisme structural entre la phrase trouée et la phrase source, en ce qui concerne les propriétés morpho-syntaxiques des éléments résiduels (marquage casuel, marquage prépositionnel, catégorie et fonction syntaxique, nombre des éléments, ordre des mots, etc.), cf. \citet{Hartmann2000}, Culicover \& \citet{Jackendoff2005}, \citet{Culicover2009}.

Dans les constructions à gapping, on observe qu'un élément résiduel doit avoir la même fonction syntaxique et généralement la même marque casuelle. L'identité fonctionnelle est une conséquence de la contrainte d'identité sémantique qu'on discutera dans la section  : ainsi, en \REF{ex:4:121}a, on ne peut avoir une paire contrastive {\textless}\textit{cartea, noaptea}{\textgreater}, où \textit{cartea} `le livre' est un complément, alors que \textit{noaptea} `la nuit' est un ajout, par rapport au verbe antécédent \textit{a citi} `lire'. De même, une paire contrastive comporte généralement la même marque casuelle, à l'exception de quelques idiosyncrasies casuelles en roumain, comme c'est l'exemple du datif en \REF{ex:4:121}b, où le syntagme contenant un élément quantitatif (p.ex. \textit{la} \textbf{\textit{trei} }\textit{dintre copii} `à trois parmi les enfants') reçoit la marque prépositionnelle \textit{la} (demandant une forme d'accusatif), alors que son correspondant résiduel dans la séquence trouée comporte la marque synthétique (affixée) habituelle (p.ex. \textit{tutu}\textbf{\textit{ror}}\textit{ copii}\textbf{\textit{lor}}\textbf{} `à tous les enfants).  


\begin{enumerate}
\item \label{bkm:Ref289428025}a  \#Ion citeşte cartea, iar Maria noaptea.


\end{enumerate}
{\itshape
Ioana lit le livre, et Maria (pendant) la nuit} 

  b  Ion oferă mere [\textbf{la} trei dintre copii], iar Maria [tutu\textbf{ror} copii\textbf{lor}].

    Ion donne pommes à trois des enfants, et Maria tous.\textsc{dat} enfants.\textsc{def.dat} 

    \textit{Ion offre des pommes à trois des enfants, et Maria à tous les enfants} 

En revanche, on observe que le parallélisme structural n'est pas strict en ce qui concerne la catégorie grammaticale, le nombre de dépendants réalisés, ainsi que l'ordre dans lequel apparaissent les éléments résiduels par rapport à leurs corrélats dans la phrase source. 

Ainsi, comme Sag \textit{et al.} (1985) l'observent pour l'anglais, un élément résiduel et son corrélat n'ont pas nécessairement la même catégorie syntaxique, p.ex. syntagme nominal \textit{vs.} syntagme prépositionnel en \REF{ex:4:122}a ou bien syntagme nominal \textit{vs.} phrase en \REF{ex:4:122}b, à condition que chacune de ces catégories constitue un dépendant possible du prédicat antécédent.


\begin{enumerate}
\item \label{bkm:Ref289380971}a  Ioana citeşte [\textsubscript{NP} ziua], iar Maria [\textsubscript{PP} pe-ntuneric].


\end{enumerate}
{\itshape
Ioana lit pendant la journée, et Maria dans l'obscurité} 

  b  Mie îmi place [\textsubscript{NP} muzica], iar prietenului meu [\textsubscript{S} să facă sport].

    moi.\textsc{dat cl.dat} plaît la-musique, et ami.\textsc{def.dat} \textsc{poss} faire\textsc{.subj} sport 

    \textit{Moi j'aime la musique, et mon ami faire du sport} 

La phrase trouée peut comporter un nombre de dépendants différent du nombre de dépendants dans la phrase source. On a deux cas de figure : (i) un élément résiduel des deux paires contrastives n'a pas de corrélat lexical, ou bien (ii) en dehors des deux paires contrastives, la phrase trouée contient un élément supplémentaire. Le premier cas est illustré par le phénomène du pro-drop, où l'élément résiduel sujet n'a pas de corrélat lexical dans la phrase source \REF{ex:4:123}, mais aussi par les exemples dans lesquels un des corrélats dans la phrase source correspond à un élément {\guillemotleft}~faible~{\guillemotright}, p.ex. un clitique adverbial \REF{ex:4:124}a ou pronominal \REF{ex:4:124}b.\footnote{On pourrait rajouter ici les exemples (i)-(ii), dans lesquels la séquence qui suit la conjonction \textit{iar} est analysée comme une phrase trouée (voir section \ref{sec:4.6} de ce chapitre). Les éléments mis en gras n'ont pas de correspondant explicite dans le premier conjoint, mais leur présence est obligatoire pour la grammaticalité des exemples (cf. la contrainte du double contraste avec la conjonction \textit{iar}).
(i)  Nu am nicio legătură cu biserica, sunt [un simplu credincios], iar *(\textbf{de meserie}) [şofer].
  \textit{Je n'ai aucun lien avec l'église, je suis un simple croyant, et quant à mon métier, chauffeur}
(ii)  Ioana mănâncă [un măr], iar *(\textbf{apoi}) [o pară].
  \textit{Ioana mange une pomme, et ensuite une poire}}


\begin{enumerate}
\item \label{bkm:Ref289416846}a  Lunea merg la film, iar \textbf{sora mea} la muzeu.


\end{enumerate}
le-lundi aller.\textsc{ind.1sg} à film, et s{\oe}ur.\textsc{def poss} à musée 

    \textit{Le lundi, je vais au cinéma, et ma s{\oe}ur au musée } 

  b  O vorbă n-ai scos, de parcă ai fi văduv, iar \textbf{eu} menajera ta.

    un mot \textsc{neg-}as sorti, comme si être.\textsc{cond.2sg} veuf, et moi domestique.\textsc{def poss}

    (\href{http://www.scribd.com/doc/19489125/Familie-casnicie1}{{www.scribd.com/doc/19489125/Familie-casnicie1}}) 

    \textit{Tu n'as pas dit un mot, comme si tu étais veuf, et moi, ta domestique} 

  c  (Marin e doar cu câțiva ani mai mare decât noi) : pe vremea aceea să fi avut vreo 35-37 de ani, \textbf{Petre şi Toader} vreo 30-31, iar \textbf{eu} 27-28.

à cette époque avoir.\textsc{subj.passe.3sg} environ 35-37 ans, Petre et Toader environ 30-31, et moi 27-28 

{\itshape
Marin est un petit peu plus âgé que nous : à cette époque, il aurait dû avoir environ 35-37 ans, Petre et Toader environ 30-31, et moi 27-28} 


\begin{enumerate}
\item \label{bkm:Ref289416863}a  Marian \textbf{tot} \textbf{mai} citeşte, dar prietena lui \textbf{absolut nimic}.


\end{enumerate}
Marian \textsc{cl.adv cl.adv} lit, mais copine.\textsc{def poss} absolument rien 

    \textit{Marian lit un peu, mais sa copine absolument rien } 

  b  Ion \textbf{mi}-e prieten, iar \textbf{ție} duşman.

    Ion \textsc{cl.1sg.dat-}est ami, et toi\textsc{.2sg.dat} ennemi 

    \textit{Ion est mon ami, et ton ennemi} 

Le deuxième cas est illustré par les exemples en \REF{ex:4:125}, où la séquence trouée contient un élément de plus par rapport aux deux paires contrastives obligatoires, n'ayant pas de corrélat dans la phrase source. Cet élément {\guillemotleft}~solitaire~{\guillemotright} peut être réinterprété\footnote{Cette observation reprend la note 4 (page186) dans l'article de Abeillé \& \citet{Mouret2010}.} , en considérant que la deuxième paire contrastive (p.ex. {\textless}\textit{un ziar}, \textit{o jucărie (pentru fetița ei)}{\textgreater} en \REF{ex:4:125}a) met en jeu deux propriétés (et non deux entités), la deuxième propriété étant obtenue par montée de type du syntagme nominal résiduel \textit{o jucărie} `un jouet' et composition fonctionnelle avec le terme {\guillemotleft}~solitaire~{\guillemotright} \textit{pentru fetița ei} `pour sa fille', en suivant l'analyse proposée en Grammaire Catégorielle par \citet{Steedman2000}. 


\begin{enumerate}
\item \label{bkm:Ref289418814}a  [Ion] a cumpărat [un ziar], iar [Maria] [o jucărie] \textbf{pentru fetița ei}.


\end{enumerate}
    \textit{Ion a acheté un journal, et Maria un jouet pour sa fille } 

  b  [Dan] merge [la munte], iar [Maria] \textbf{probabil} [la mare]. 

    \textit{Dan va à la montagne, et Maria probablement à la mer} 

Enfin, on observe que l'ordre des éléments résiduels n'est pas nécessairement le même que l'ordre des corrélats dans la phrase source, à condition que l'ordre en question soit possible par ailleurs dans la grammaire. Cela s'explique en roumain par les faits suivants : (i) l'ordre relativement libre des mots, (ii) le marquage prosodique (c.-à-d. focus prosodique, marqué dans les exemples suivants par des majuscules), et (iii) la présence d'une conjonction, surtout la conjonction \textit{iar}, qui impose des contraintes discursives facilitant l'organisation des paires contrastives. Le manque de parallélisme en ce qui concerne l'ordre des éléments résiduels et corrélats semble être moins acceptable avec la juxtaposition \REF{ex:4:127}b.  


\begin{enumerate}
\item a  Dimineața (EU) spăl (EU) vesela (EU), iar seara IOAna.


\end{enumerate}
le-matin (moi) lave.\textsc{1sg} (moi) la-vaisselle (moi), et le-soir Ioana 

\textit{Le matin c'est moi qui lave la vaisselle, et le soir (c'est) Ioana}  

  b  EU spăl vesela dimineața, iar seara IOAna.

moi lave.\textsc{1sg} la-vaisselle le-matin, et le-soir Ioana

{\itshape
Le matin c'est moi qui lave la vaisselle, et le soir (c'est) Ioana}

  c  Eu spăl vesela dimiNEAța, iar Ioana SEAra.

moi lave.\textsc{1sg} la-vaisselle le-matin, et Ioana le-soir

{\itshape
Je lave la vaisselle le matin, et Ioana le soir } 

  d  DimiNEAța spăl eu vesela, iar Ioana SEAra.

    le-matin lave.\textsc{1sg} moi la-vaisselle, et Ioana le-soir

{\itshape
C'est le matin que je lave la vaisselle, et Ioana (c'est) le soir} 


\begin{enumerate}
\item \label{bkm:Ref289427373}a  Fiul lor studiază DREPtul, fiica mea mediCIna.


\end{enumerate}
fils.\textsc{det poss} étudie le-droit, fille\textsc{.def poss} la médecine 

    \textit{Mon fils étudie le droit, ma fille la médecine } 

  b  Fiul LOR studiază dreptul, *(\textbf{iar}) medicina, fiica MEA.

    fils.\textsc{det poss} étudie le-droit, (et) la-médecine fille\textsc{.def poss}

    \textit{C'est leur fils qui étudie le droit, et la médecine, (c'est) ma fille} 

Sur la base de ces observations, on doit conclure que le parallélisme syntaxique n'opère pas au niveau de la catégorie ou encore au niveau de l'ordre des mots (\textit{contra} Culicover \& \citet{Jackendoff2005}, \citet{Culicover2009}), mais plutôt au niveau de la structure argumentale du prédicat antécédent. Ce parallélisme syntaxique {\guillemotleft}~relâché~{\guillemotright} exige simplement que les éléments résiduels remplissent les conditions de sélection du prédicat antécédent dans la phrase source (cf. la généralisation de Wasow, discutée dans le chapitre 2, section \ref{sec:2.4.1}). Ainsi, les éléments résiduels et leurs corrélats dans les constructions à gapping obéissent aux mêmes contraintes syntaxiques que les coordinations ordinaires. Par exemple, dans la coordination à gapping en \REF{ex:4:128}, le verbe \textit{a cere} `demander' est compatible à la fois avec un syntagme nominal et une phrase au subjonctif, mais moins avec un syntagme verbal à l'infinitif ; il aura le même comportement dans une coordination ordinaire comme en \REF{ex:4:129}.  


\begin{enumerate}
\item \label{bkm:Ref290554010}a  La meeting-ul de azi, unii cereau [\textsubscript{NP} demisia Preşedintelui], alții [\textsubscript{S} să li se mărească salariile]... Era tot o hărmălaie !


\end{enumerate}
{\itshape
Au meeting d'aujourd'hui, les uns demandaient la démission du Président, les autres qu'on leur augmente les salaires... C'était un bazar } 

  b  ??La meeting-ul de azi, unii cereau [\textsubscript{NP} demisia Preşedintelui], alții [\textsubscript{S} a avea salarii mai mari].

{\itshape
Au meeting d'aujourd'hui, les uns demandaient la démission du Président, les autres avoir de meilleurs salaires } 


\begin{enumerate}
\item \label{bkm:Ref290554068}a  La meeting-ul de azi, oamenii cereau [\textsubscript{NP} demisia Preşedintelui] şi [\textsubscript{S} să li se mărească salariile].


\end{enumerate}
{\itshape
Au meeting d'aujourd'hui, les gens demandaient la démission du Président et qu'on leur augmente les salaires}

  b  ??La meeting-ul de azi, oamenii cereau [\textsubscript{NP} demisia Preşedintelui] şi [\textsubscript{VP} a avea salarii mai mari].

{\itshape
Au meeting d'aujourd'hui, les gens demandaient la démission du Président et d'avoir de meilleurs salaires} 

\label{bkm:Ref289107440}Sémantique

Au niveau sémantique, chaque élément résiduel doit être mis en correspondance avec un corrélat dans la phrase source. Pour le type d'ellipse qui nous intéresse ici, cette correspondance se traduit par une relation de \textit{contraste}.\footnote{La notion de \textit{contraste} doit être prise en compte pour d'autres types d'ellipse aussi, p.ex. \textit{Bare Argument Ellipsis} (BAE). A la différence du gapping, BAE exige (au moins) \textbf{une} paire contrastive. Voir les détails dans Konietzko \& \citet{Winkler2010}. }  On peut donc dire qu'une coordination à gapping doit contenir au moins deux \textit{paires contrastives} (\citet{Kuno1976}, \citet{Sag1976}, \citet{Hartmann2000}, Féry \& \citet{Hartmann2005}, \citet{Winkler2005}, \citet{Repp2009}, etc.).\footnote{On utilise parfois le terme de \textit{contrastive foci}, mais on évite le terme de \textit{focus} ici. On l'utilisera uniquement dans son sens strict, d'information nouvelle (voir section \ref{sec:4.3.1.3}).} Ainsi, en \REF{ex:4:130}, les deux paires mises en jeu sont {\textless}\textit{Ioana, Maria}{\textgreater} et {\textless}\textit{un măr, o pară}{\textgreater}.


\begin{enumerate}
\item \label{bkm:Ref299640651}Ioana a mâncat un măr şi Maria o pară.


\end{enumerate}
\textit{  Ioana a mangé une pomme et Maria une poire }  

Le roumain se distingue des autres langues romanes (et se rapproche des langues slaves) par le fait qu'il comporte une conjonction spécialisée pour le \textit{double contraste} (voir section \ref{sec:2.9} du chapitre 2 et Bîlbîie \& \citet{Winterstein2011}). Il s'agit de la conjonction \textit{iar}, qui est la conjonction la plus fréquente dans les coordinations à gapping \REF{ex:4:131}a, étant exclue dans les contextes à une seule paire contrastive \REF{ex:4:131}b-c. Dans la plupart des cas, les corrélats dans la phrase source sont lexicalisés. Cependant, il y a certains cas où le corrélat dans une paire contrastive est implicite, p.ex. le pro-drop en \REF{ex:4:123} ou bien les exemples mentionnés en-dessous de la note \pageref{fnt:ftn0}. 


\begin{enumerate}
\item \label{bkm:Ref289438493}a  Ioana a mâncat un măr, iar Maria o pară.


\end{enumerate}
{\itshape
Ioana a mangé une pomme, et Maria une poire } 

  b  *Ioana a mâncat un măr, iar Maria. 

    \textit{Ioana a mangé une pomme, et Maria}

  c  Ioana a mâncat un măr \{şi / *iar\} o pară. 

    \textit{Ioana a mangé une pomme, et une poire } 

Une paire contrastive se définit essentiellement par deux aspects concomitants : (i) l'appartenance à un même \textit{ensemble d'alternatives}, et (ii) la présence d'une \textit{opposition sémantique} entre ses éléments. Ainsi, le contraste suppose à la fois une relation de ressemblance et dissemblance (\citet{Sag1976}, \citet{Rooth1992}, Vallduví \& \citet{Vilkuna1998}). La même idée apparaît chez \citet{Umbach2005}, qui définit le parallélisme sémantique en utilisant la terminologie de \citet{Lang1984} : la ressemblance des éléments d'une paire s'établit grâce à la présence implicite d'un \textit{intégrateur commun}, c.-à-d. un concept qui subsume les deux conjoints ; tandis que la dissemblance est garantie par la condition d'\textit{indépendance sémantique} ou la propriété d'être \textit{distincts}, c.-à-d. les éléments en question ne doivent pas se subsumer (voir aussi \citet{Zeevat2004}). C'est ce qui explique les différences d'acceptabilité pour les exemples en \REF{ex:4:132}. Selon \citet{Umbach2005}, dans le premier exemple, on viole le principe de l'indépendance sémantique, car \textit{a drink} subsume \textit{a martini}. En revanche, dans le deuxième exemple, la condition d'avoir un intégrateur commun détermine le choix d'un certain sens (compatible avec \textit{the beer}) pour le mot \textit{the port}.  


\begin{enumerate}
\item \label{bkm:Ref289436982}a   \#John had a drink, and Mary a martini.


\end{enumerate}
b  John bought the beer, and Mary the port.    (exemple adapté d'\citet{Umbach2005})

Dans cette thèse, j'utiliserai plutôt les notions d'\textit{ensemble d'alternatives} et \textit{opposition sémantique}. Dans une coordination à gapping, on a donc pour chaque paire un ensemble restreint d'alternatives identifiables et explicites. Les alternatives de chaque ensemble doivent avoir un type approprié. Ainsi, appartiennent au même ensemble d'alternatives des éléments dénotant différents agents, différentes indications temporelles, différentes indications spatiales, différents objets, etc. L'énoncé devient inacceptable si la paire contient des éléments qui ne font pas partie du même ensemble d'alternatives ou qui n'ont pas le même type sémantique \REF{ex:4:133} : on ne peut pas avoir comme paire contrastive un objet et une indication temporelle \REF{ex:4:133}b, ou bien un agent et un thème \REF{ex:4:133}c, etc.


\begin{enumerate}
\item \label{bkm:Ref289441864}a  \#Maria cântă \textbf{la pian}, iar Ioana \textbf{arii de Chopin}.


\end{enumerate}
{\itshape
Maria joue du piano, et Ioana des airs de Chopin}

  b  \#Ioana mănâncă \textbf{mere}, iar Maria \textbf{la miezul nopții}. 

    \textit{Ioana mange des pommes, et Maria à minuit}

  c  \# Sunt păzit \textbf{de Cel-de-Sus}, iar tu \textbf{de orice rău}. 

    \textit{Je suis protégé par le bon Dieu, et toi de tout mal}

  d  \#Ion merge la film \textbf{cu Maria}, iar Dan \textbf{cu maşina}.\footnote{Un exemple comme celui-ci peut être amélioré si le syntagme prépositionnel \textit{cu Maria} `avec Maria' désigne un moyen de transport possédé par Maria (la voiture de Maria) : 
(i)  Ion merge la lucru cu Maria, iar Dan cu trenul.
  \textit{Ion va au travail avec Maria, et Dan en train}} 

    \textit{Ion va au cinéma avec Maria, et Dan en voiture}

Selon la première contrainte pesant sur le contraste (c.-à-d. la ressemblance des éléments mis en contraste, cf. \citet{Umbach2005}), les éléments doivent appartenir au même ensemble d'alternatives (ayant en commun un type sémantique et une archi-propriété), mais ils ne doivent pas se subsumer. Ce qui explique l'inacceptabilité de l'exemple \REF{ex:4:134}a, où l'on a comme deuxième paire contrastive {\textless}\textit{un măr, un fruct}{\textgreater} définie comme {\textless}hyponyme, hyperonyme{\textgreater}. Mais comment expliquer l'acceptabilité de l'exemple \REF{ex:4:134}b, qui semblent contredire la contrainte qu'on vient de préciser ? Selon Bîlbîie \& \citet{Winterstein2011}, dans la paire {\textless}\textit{câteva, toate}{\textgreater} la subsomption est fausse si l'on établit la liste exhaustive des membres associés à \textit{quelques}-\textit{unes}, c.-à-d. si on interprète \textit{quelques-unes} comme \textit{quelques-unes, mais pas toutes}. Or, cette lecture exhaustive est fortement préférée par les locuteurs dans cet exemple. En revanche, on ne peut pas exhaustifier un hyperonyme, p.ex. \textit{fruit} = \textit{fruit, mais pas pomme}. Ainsi, il est impossible de dériver une inférence scalaire à partir de l'assertion d'un hyperonyme (\textit{Maria a mangé un fruit} n'implique pas \textit{Maria n'a pas mangé une pomme}), ce qui n'est pas le cas avec les implicatures quantitatives en \REF{ex:4:134}b (\textit{Maria a répondu à quelques questions} implique \textit{Maria n'a pas répondu à toutes les questions}).


\begin{enumerate}
\item \label{bkm:Ref289597573}a  \#Ioana a mâncat un\textbf{ măr}, iar Maria un\textbf{ fruct}.


\end{enumerate}
    \textit{Ioana a mangé une pomme, et Maria un fruit } 

  b  Ioana a răspuns la \textbf{toate} întrebările, iar Maria la \textbf{câteva}. 

    \textit{Ioana a répondu à toutes les questions, et Maria à quelques-unes}

Selon la deuxième contrainte pesant sur le contraste (c.-à-d. la dissemblance, dans \citet{Umbach2005}), il faut qu'il y ait une distinction entre les éléments mis en contraste. Tournons-nous à présent vers d'autres cas complexes, qui nécessitent des discussions sur l'identité lexématique (appelée {\guillemotleft}~identité de surface~{\guillemotright} dans Hinterwimmer \& \citet{Repp2008}), ainsi que sur l'éventuelle co-indiciation des éléments qui forment une paire contrastive. La deuxième contrainte d'une paire contrastive exige que les éléments soient distincts, d'où résulterait le fait trivial selon lequel une paire contrastive ne peut pas contenir des éléments qui présentent une identité lexématique. 


\begin{enumerate}
\item \label{bkm:Ref289551900}a  \#\textbf{Ioana} plăteşte chiria şi \textbf{Ioana} impozitele.


\end{enumerate}
{\itshape
Ioana paie le loyer et Ioana les impôts } 

  b  \#Ion vorbeşte \textbf{cu Maria} şi Dan \textbf{cu Maria}. 

    \textit{Ion parle avec Maria et Dan avec Maria}

  c  \#Ion a cumpărat (nişte) \textbf{flori} şi Maria (nişte) \textbf{flori}. 

    \textit{Ion a acheté des fleurs et Maria des fleurs } 

Cependant, il y a au moins trois types de contextes qui contredisent cette contrainte liée à la non-identité lexématique : les pronoms à interprétation déictique \REF{ex:4:136}, les éléments interrogatifs \REF{ex:4:137} ou encore les numéraux\footnote{Voir Hinterwimmer \& \citet{Repp2008} et \citet{Repp2009} pour plus de détails sur ce point en anglais. Ils observent que seuls certains types de syntagmes quantifiés (c.-à-d. ce qu'ils appellent les \textit{indéfinis} \textit{spécifiques} ou \textit{indéfinis} \textit{topiques}) permettent l'identité {\guillemotleft}~de surface~{\guillemotright} en anglais (voir le contraste entre (i)a et (i)b) et cela, uniquement en position préverbale (voir le contraste entre (i)a et (i)c). De plus, ils observent des différences entre les numéraux simples comme \textit{three} (cf. (ii)a) et les numéraux dans des quantifieurs plus complexes comme \textit{less than three}, les derniers demandant a priori une identité {\guillemotleft}~de surface~{\guillemotright} en anglais (comparer (ii)b-c).
(i)  a  One student called the director and one student the dean.  
  b  *A student called the director and a student the dean.
  c  *The director called one student and the dean one student.          (\citet[9]{Repp2009}
(ii)  a  Three children chose the book and three (children) the CD.  
  b  *Less than three children chose the book and less than three (children) the CD.
  c  Less than three children chose the book and less than four (children) the CD. 
(Hinterwimmer \& \citet[244]{Repp2008}} \REF{ex:4:138}. Crucialement, on observe que les éléments {\guillemotleft}~identiques~{\guillemotright} dans ces paires ne renvoient pas au même référent.


\begin{enumerate}
\item \label{bkm:Ref299618472}a  Uite cele două rochii pe care le-am cumpărat ieri : \textbf{pe-asta} am cumpărat-o pentru cununia civilă, iar \textbf{pe-asta} pentru cununia religioasă.~ 


\end{enumerate}
{\itshape
Voici les deux robes que j'ai achetées hier : celle-là je~l'ai achetée pour la cérémonie civile, et celle-là pour la cérémonie religieuse } 

  b  Priveşte-i pe cei doi colegi ai mei pe scenă : \textbf{el} [\textit{arătând cu degetul spre dreapta}] a făcut medicina, iar \textbf{el} [\textit{arătând cu degetul spre stânga}] dreptul.

\textit{Regarde mes deux collègues sur l'estrade : lui [en pointant du doigt vers la droite] a fait la médecine, et lui [en pointant du doigt vers la gauche] a fait le droit}  


\begin{enumerate}
\item \label{bkm:Ref289452061}a  \textbf{Cine} vine azi şi \textbf{cine} mâine ?


\end{enumerate}
{\itshape
Qui vient aujourd'hui et qui demain } 

  b  \textbf{De când} te-ai sculat tu şi \textbf{de când} eu ?  [question de reproche]

    \textit{Depuis quand tu t'es réveillé et depuis quand moi}


\begin{enumerate}
\item \label{bkm:Ref289452076}a  Dintre cei şase copii selecționați, \textbf{trei} vin azi şi \textbf{trei} mâine.


\end{enumerate}
{\itshape
Parmi les six enfants sélectionnés, trois viennent aujourd'hui et trois demain} 

  b  Dintre cele patru mere rămase, Ioana a luat \textbf{două} şi Maria \textbf{două}.

    \textit{Parmi les quatre pommes restées, Ioana en a pris deux, et Maria deux}

Ainsi, \citet{Hartmann2000} et \citet{Repp2009} vont plus loin et postulent la condition d'un contraste \textit{référentiel} entre un élément résiduel et son corrélat. Par conséquent, une paire contrastive ne peut comporter des éléments qui renvoient au même référent. On explique ainsi l'inacceptabilité des exemples en \REF{ex:4:139}, dans lesquels la première paire contrastive contient des éléments qui sont co-indicés.\footnote{On note certains exemples marginaux dans lesquels une paire contrastive contient des syntagmes nominaux renvoyant au même référent, mais avec des interprétations différentes (intensionnelle vs. extensionnelle). 
(i)  [Le président de la République]\textsubscript{i} est agnostique, mais [l'homme Sarkozy]\textsubscript{i} catholique.}


\begin{enumerate}
\item \label{bkm:Ref299633369}a  \#Maria\textsubscript{i} participă la concursul de fotografie şi Maria\textsubscript{i} la festivalul de muzică.


\end{enumerate}
{\itshape
Maria participe au concours de photographie et Maria au festival de musique } 

  b  \#Maria\textsubscript{i} participă la concursul de fotografie şi [proasta asta]\textsubscript{i} la festivalul de muzică. 

    \textit{Maria participe au concours de photographie et Maria au festival de musique}

Néanmoins, les exemples inacceptables avec identité lexématique et co-indiciation en \REF{ex:4:135} peuvent être améliorés (au moins pour la paire contrastive ne contenant pas le résiduel en première position dans la phrase trouée\footnote{Il reste à expliquer pourquoi \REF{ex:4:140}a et \REF{ex:4:141}b, c.-à-d. les exemples dans lesquels la paire contrastive établie par identité lexématique et co-indiciation contient des éléments apparaissant en première position dans la phrase (ici, des sujets), ne peuvent pas être améliorés par la présence d'un adverbe comme \textit{tot} `aussi' ou \textit{aussi}. Une explication possible serait liée aux différences d'association qu'engendre la position de l'adverbe. Si l'adverbe \textit{aussi} est en position préverbale, on a nécessairement une association étroite de l'adverbe, qui met en parallèle uniquement l'associé de \textit{aussi} dans la séquence trouée et son corrélat dans la phrase source. En revanche, en position finale, l'adverbe additif peut avoir une association large, donc il peut avoir comme associé toute la séquence trouée dans les constructions à gapping. Dans ce dernier cas, on obtient un contraste plus large entre les événements pris dans leur totalité, ce qui fournit (en plus du parallélisme) l'opposition sémantique dont on a besoin dans une construction à gapping. Une autre explication serait liée au statut informationnel de l'associé de \textit{aussi}. Il a été remarqué que l'associé de \textit{aussi} est prosodiquement distingué, c.-à-d. il est un focus prosodique, et donc, au niveau discursif, un focus informationnel (\citet{Jackendoff1972}). Or, le prototype d'une séquence trouée dans une construction à gapping est une séquence contenant un topique contrastif (en première position) et un focus contrastif (en deuxième position), cf. \citet{Winkler2005}.} ) si l'on emploie des adverbes additifs comme \textit{tot} `aussi' en roumain \REF{ex:4:140} ou \textit{aussi} en français \REF{ex:4:141}. Konietzko \& \citet{Winkler2010} remarquent eux aussi que, dans certains cas, le dispositif permettant le contraste dans les ellipses {\guillemotleft}~contrastives~{\guillemotright} est l'emploi de ce qu'ils appellent une particule discursive (p.ex. \textit{aussi}).


\begin{enumerate}
\item \label{bkm:Ref299633645}a  ?\textbf{Ioana} plăteşte chiria şi \textbf{tot Ioana} impozitele ; prietenul ei nu plăteşte niciodată nimic.


\end{enumerate}
{\itshape
Ioana paie le loyer et toujours Ioana les impôts ; son ami ne paie jamais rien } 

  b  Ion vorbeşte \textbf{cu Maria}, iar Dan \textbf{tot} \textbf{cu Maria}. 

    \textit{Ion parle avec Maria et Dan avec Maria aussi}

  c  Ion a cumpărat (nişte) \textbf{flori} şi Maria \textbf{tot} (nişte) \textbf{flori}. 

    \textit{Ion a acheté des fleurs et Maria des fleurs aussi } 


\begin{enumerate}
\item \label{bkm:Ref299633649}a   \#\textbf{Marie} paie le loyer et \textbf{Marie} les impôts.


\end{enumerate}
b  ??\textbf{Marie} paie le loyer et puis \textbf{Marie aussi} les impôts ; son ami ne paie jamais rien.

c  \#Jean est arrivé \textbf{aujourd'hui} et Marie \textbf{aujourd'hui}.

d  Jean est arrivé \textbf{aujourd'hui} et Marie \textbf{aujourd'hui} \textbf{aussi}.

e  \#Jean parle \textbf{avec Marie} et Pierre \textbf{avec Marie}.

f  Jean parle \textbf{avec Marie} et Pierre \textbf{avec Marie} \textbf{aussi}.

Dans ces cas-là, le lien entre la relation de contraste (exigée dans une paire contrastive) et la relation de parallélisme (caractérisant l'emploi de \textit{aussi}) est compliqué à établir. D'une part, l'absence du marqueur de parallélisme rend la phrase dégradée, car \textit{aussi} est obligatoire (cf. S{\ae}b{\o} (2004), Amsili \& \citet{Beyssade2009}, etc.). Le caractère obligatoire de \textit{aussi} est demandé par la relation de similarité qui doit exister entre l'associé de \textit{aussi} et un élément appartenant à l'ensemble d'alternatives de l'associé. D'autre part, pour obtenir le double contraste dont on a besoin dans une construction à gapping, on est obligé de postuler que, en dehors du contraste local qui s'établit entre deux éléments formant une paire contrastive, on a parfois un contraste plus large entre les événements pris dans leur totalité, grâce à la présence d'un adverbial comme \textit{aussi} dont l'associé est la séquence trouée dans son ensemble. Un argument qui pourrait être donné en faveur de cette hypothèse est la préférence pour le positionnement de l'adverbe additif sur le deuxième élément résiduel : en français, \textit{aussi} en position postverbale peut avoir une association large sur toute la phrase (pour plus de détails sur le fonctionnement de \textit{aussi} en français, voir \citet{Winterstein2010}).

Parfois, l'opposition sémantique à l'intérieur d'une paire contrastive est lexicalisée : l'élément résiduel peut être une expression désignant par elle-même l'idée de contraste, comme le syntagme nominal \textit{contrariul} `le contraire'. Ainsi, la deuxième paire contrastive dans les exemples suivants est construite d'un syntagme nominal comme \textit{ceva} `quelque chose', \textit{singurătatea} `la solitude' ou encore une phrase \textit{că pilula scade riscul de cancer} `que la pilule réduit le risque de cancer', et le syntagme nominal \textit{contrariul} `le contraire'.  


\begin{enumerate}
\item a  \textstyleapplestylespan{Eşti bombardat zilnic cu tot felul de informații, unii} \textstyleapplestylespan{susțin}\textstyleapplestylespan{} \textstyleapplestylespan{\textbf{ceva}}\textstyleapplestylespan{,}{~}\emph{\textup{alții}}\emph{\textbf{\textup{ contrariul}}}\textstyleapplestylespan{, chiar nu mai ştii ce să mai crezi}.


\end{enumerate}
{\itshape
On est bombardé chaque jour avec toutes sortes d'informations, certains soutiennent quelque chose, d'autres le contraire, on ne sait plus quoi croire } 

  b  \textstyleapplestylespan{Sunt oameni} \textstyleapplestylespan{care preferă}\textstyleapplestylespan{} \textstyleapplestylespan{\textbf{singurătatea}}\textstyleapplestylespan{, iar}{~}\emph{\textup{alții,}}\emph{\textbf{\textup{ contrariul}}}. 

    \textit{Il y a des gens qui préfèrent la solitude, et d'autres, le contraire}

  c  Unii spun \textbf{că pilula scade riscul de cancer}, iar alții \textbf{contrariul}. 

    \textit{Certains disent que la pilule réduit le risque de cancer, et d'autres le contraire } 

Les deux facettes du contraste dans les paires rendent aussi compte de l'inacceptabilité des exemples en \REF{ex:4:143}\footnote{Ces exemples sont acceptés uniquement s'ils sont interprétés comme des occurrences de zeugmes sémantiques, ayant une lecture ironique.}, où la deuxième paire est construite avec des portions d'expressions idiomatiques qui ne peuvent pas réaliser un contraste approprié, bien que le matériel manquant ait la même forme que le matériel antécédent.


\begin{enumerate}
\item \label{bkm:Ref289603838}a  \#Mie îmi arde \textbf{sufletul} de durere, iar ție \textbf{călcâiele} să mergi la discotecă.


\end{enumerate}
moi.\textsc{dat} brûle l'âme de douleur, et toi.\textsc{dat} les-talons aller\textsc{.subj.2sg} à discothèque

{\itshape
Mon âme brûle de douleur, et tes talons d'impatience pour aller à la discothèque}

  b  \textstyleapplestylespan{\#De când stă beat prin şanțuri, nevastă-sa} \textstyleapplestylespan{îşi duce}\textstyleapplestylespan{} \textstyleapplestylespan{\textbf{crucea}}\textstyleapplestylespan{ fără să crâcnească, iar el} \textstyleapplestylespan{\textbf{zilele}}\textstyleapplestylespan{ de azi pe mâine}. 

    depuis quand reste\textsc{.3sg} ivre dans fossés, sa-femme \textsc{cl.refl} porte la-croix sans broncher, et lui les-jours d'aujourd'hui à demain

{\itshape
Depuis qu'il est toujours ivre au bord de la route, sa femme porte sa croix sans broncher, et lui vit au jour le jour} 

  c  \textstyleapplestylespan{\#După întrevederea de ieri cu Ion, eu} \textstyleapplestylespan{am ajuns cu el}\textstyleapplestylespan{} \textstyleapplestylespan{\textbf{la o înțelegere}}\textstyleapplestylespan{, iar Maria} \textstyleapplestylespan{\textbf{la cuțite}}. 

{\itshape
Après la rencontre d'hier avec Ion, je suis arrivé avec lui à un accord, et Maria à couteaux tirés}

  d  \textstyleapplestylespan{\#Față de incidentul produs în firmă, cei mai mulți} \textstyleapplestylespan{păstrează}\textstyleapplestylespan{} \textstyleapplestylespan{\textbf{tăcerea}}\textstyleapplestylespan{, iar alții} \textstyleapplestylespan{\textbf{amintiri de neuitat}}. 

{\itshape
Face à l'incident survenu dans l'entreprise, la plupart garde le silence, et d'autres des souvenirs inoubliables}

  e  \textstyleapplestylespan{\#Nu reuşesc să rezolv situația în niciun fel : eu} \textstyleapplestylespan{pun}\textstyleapplestylespan{} \textstyleapplestylespan{\textbf{o vorbă bună}}\textstyleapplestylespan{ pentru el, iar el (în schimb)} \textstyleapplestylespan{\textbf{paie pe foc}}. 

    \textsc{neg} réussis.\textsc{2sg} résoudre.\textsc{subj} la-situation en aucune manière : je mets un mot bon pour lui, et lui (en revanche) paille sur feu

{\itshape
Je n'arrive à résoudre la situation d'aucune manière : j'interviens en sa faveur, et lui (en revanche) attise la querelle } 

Pour conclure, les coordinations à gapping mettent en jeu un parallélisme sémantique~fort, c.-à-d. il doit y avoir au moins deux contrastes sémantiques entre les éléments résiduels et les corrélats. Les paires contrastives exploitent chacune un ensemble d'alternatives qui fournit les éléments qui vont être mis en contraste. 

\paragraph[Relations discursives]{Relations discursives}
Au niveau du discours, on considère généralement que les relations qui s'établissent entre les phrases liées par coordination appartiennent à l'un des deux grands types suivants : relations symétriques vs. relations asymétriques\footnote{\citet{Asher1993} et Asher \& \citet{Lascarides2003} utilisent plutôt la distinction \textit{relations de coordination} vs. \textit{relations de subordination}. Ces termes peuvent créer une confusion avec les termes utilisés habituellement pour les deux types de phrases liées. Par conséquent, j'éviterai ces termes dans cette thèse. Voir une discussion plus détaillée dans le chapitre 2, sections 2.2 et 2.4.2. }  (\citet{Asher1993}, Asher \& \citet{Lascarides2003}, Kehler (1996, 2000, 2002). Dans les phrases entretenant une relation symétrique, les événements coordonnés sont indépendants l'un par rapport à l'autre, ce qui explique la possibilité d'inverser l'ordre des conjoints sans changer d'interprétation. En revanche, une relation asymétrique place les conjoints dans une relation hiérarchique en quelque sorte, dans le sens où le deuxième événement dépend du premier, p.ex. dans une relation de type cause-effet. Par conséquent, tout changement dans l'ordre des conjoints entraîne des différences pour leur interprétation.  

Levin \& \citet{Prince1986} sont les premiers à observer une différence discursive entre les coordinations standard et les coordinations à gapping. Si une coordination simple comme celle en \REF{ex:4:144}a est compatible avec les deux types de relations, le gapping en \REF{ex:4:144}b impose une lecture symétrique. On ne peut donc avoir l'interprétation selon laquelle Nan devient furieux à cause du fait que Sue était fâchée. 


\begin{enumerate}
\item \label{bkm:Ref289606047}a   Sue became upset and Nan became downright angry.


\end{enumerate}
b  Sue became upset and Nan downright angry.        (\citet[83]{Kehler2002}) 

\citet{Kehler2002} reprend l'observation de Levin \& \citet{Prince1986} et l'applique non seulement à la conjonction \textit{and}, mais aussi aux conjonctions \textit{or} et \textit{but}. A l'instar de \citet{Kehler2002}, on peut dire que c'est le facteur discursif qui explique l'impossibilité du gapping avec un marqueur de subordination comme \textit{because, even though, despite the fact that, although}, etc. (voir les exemples mentionnés auparavant en \REF{ex:4:32} pour l'anglais et \REF{ex:4:34} pour le roumain) ou encore avec une conjonction comme fr. \textit{or} ou \textit{car}, et roum. \textit{or}, car tous ces éléments indiquent une relation asymétrique de type cause-effet. Par conséquent, le gapping sera compatible uniquement avec les marqueurs qui entretiennent des relations symétriques entre les phrases, en particulier avec tout élément lexical qui n'est pas contradictoire avec la notion de contraste : ce sont le plus souvent les conjonctions (de coordination), mais aussi certains marqueurs {\guillemotleft}~hybrides~{\guillemotright} comme \textit{în timp ce} `alors que' en roumain (cf. les données en \REF{ex:4:35} discutées dans la section ), certains connecteurs adverbiaux \REF{ex:4:47} ou encore certains adverbes additifs (comme \textit{aussi, de même} en français). On explique aussi pourquoi les premiers travaux excluaient le connecteur adversatif \textit{but} de la liste des conjonctions possibles avec le gapping, car ce type de connecteur (en particulier, dans son usage argumentatif) demande un conjoint droit argumentativement plus fort que celui de gauche (voir \citet{Winterstein2010} pour \textit{mais} en français), par conséquent les conjoints n'ont pas le même type de contribution et entretiennent en quelque sorte une asymétrie discursive. Mais, comme le note \citet{Winterstein2010}, l'adversatif \textit{mais} en français (comme d'ailleurs la conjonction \textit{but} en anglais) a plusieurs emplois, dont un usage contrastif (\textit{opposition sémantique}, dans \citet{Lakoff1971}) en \REF{ex:4:145}a. Cet usage se distingue des autres emplois de \textit{mais} par le fait que la coordination met en jeu deux paires contrastives, avec un élément provenant de chacun des conjoints dans chacune des paires (p.ex. {\textless}\textit{Lemmy, Ritchie}{\textgreater} et {\textless}\textit{basse, guitare}{\textgreater}). De plus, l'interprétation dans ce type d'emploi contrastif est symétrique, ce qui nous permet d'inverser l'ordre des conjoints en \REF{ex:4:145}b sans modification de sens, contrairement à l'usage argumentatif (\textit{déni d'attente}, dans \citet{Lakoff1971}) en \REF{ex:4:146}a, dont l'interprétation n'est pas symétrique (\REF{ex:4:146}a et \REF{ex:4:146}b ne sont donc pas équivalents). Enfin, dans son emploi contrastif, la conjonction \textit{mais} peut être facilement remplacée par la conjonction \textit{et} en \REF{ex:4:145}c, avec un changement de sens à peine perceptible ; en revanche, la substitution de \textit{mais} par \textit{et} dans l'usage argumentatif change le sens de l'énoncé global (\REF{ex:4:146}a et \REF{ex:4:146}c ont ainsi des interprétations différentes). On observe donc que les contraintes liées à l'usage contrastif de \textit{mais} sont proches de celles du gapping, ce qui explique l'occurrence de la conjonction \textit{mais} dans ce type de constructions elliptiques. 


\begin{enumerate}
\item \label{bkm:Ref302318675}a   Lemmy joue de la basse, \textbf{mais} Ritchie de la guitare.


\end{enumerate}
  b  Ritchie joue de la guitare, mais Lemmy de la basse.  

c  Lemmy joue de la basse, et Ritchie de la guitare.      (\citet[42]{Winterstein2010}) 


\begin{enumerate}
\item \label{bkm:Ref302319520}a   Lemmy fume, \textbf{mais} il est en bonne santé.


\end{enumerate}
  b  Lemmy est en bonne santé, mais il fume.  

c  Lemmy fume, et il est en bonne santé.           (\citet[43]{Winterstein2010}) 

Les relations discursives, selon Kehler (2000, 2002), s'organisent en trois types majeurs. D'une part, on a les relations de ressemblance (dont le prototype est la relation de parallélisme, paraphrasée par \textit{and similarly}), qui caractérise toute connexion de deux ou plusieurs séquences dans laquelle on met l'accent sur les similarités ou les contrastes qui s'établissent entre les entités ou les événements en question (p.ex. le parallélisme, le contraste, la généralisation, l'exemplification, l'exception, l'élaboration). D'autre part, on a les relations de type cause-effet (dont la relation canonique est de type résultat, paraphrasée par \textit{and therefore}), dans lesquelles on doit repérer une sorte d'implication entre les propositions dénotées par les énoncés (p.ex. le résultat, l'explication, le déni d'attente, la concession). Enfin, on a les relations de contiguïté (dont le prototype est la relation de narration, paraphrasée par \textit{and then}), qui impliquent le plus souvent une séquence d'événements. 

Parmi les trois types de relations mentionnés ci-dessus, le gapping est très naturel avec les relations discursives de ressemblance (où les événements sont interprétés comme étant indépendants l'un par rapport à l'autre), et en particulier avec les relations de parallélisme et contraste. Pourquoi ces deux relations ? Parce qu'elles explicitent les deux conditions du contraste sémantique discutées dans la section  : d'une part, la relation de contraste est rendue explicite par la constitution même des paires contrastives ; d'autre part, le parallélisme est licite avec le gapping, car les éléments de chaque paire contrastive doivent être parallèles, c.-à-d. appartenir au même ensemble d'alternatives. Cette contrainte liée au type de relation discursive explique aussi pourquoi les coordinations à gapping obéissent toujours à la Contrainte sur les Structures Coordonnées (CSC), car on a établi dans le chapitre 2 que la CSC était une contrainte liée à des facteurs discursifs plutôt que syntaxiques.\footnote{Contrairement aux constructions à gapping, BAE (\textit{Bare Argument Ellipsis}) permet l'extraction asymétrique, seulement d'un des conjoints, cf. l'exemple (i), ce qui est en argument pour l'analyser comme ajout.
(i)  Iată o carte al cărei editor este necunoscut, nu însă şi autorul ei.
voici un livre duquel éditeur est inconnu, pas cependant aussi l'auteur son
Voici un livre dont l'éditeur est inconnu, mais pas son auteur} 


\begin{enumerate}
\item a  Acesta e primul \textbf{film} în care Ion joacă rolul principal, iar Maria un rol secundar (*în film).


\end{enumerate}
Celui-ci est le-premier film dans lequel Ion joue le-rôle principal, et Maria un rôle secondaire (dans film)

{\itshape
C'est le premier film où Ion joue le rôle principal, et Maria un rôle secondaire } 

  b  Acesta e \textbf{filmul} pe care Ion vrea să-l vadă, iar Maria să-l cumpere (*filmul). 

celui-ci est le-film \textsc{mrq} lequel Ion veut \textsc{mrq.cl} voir\textsc{.subj,} et Maria \textsc{mrq.cl} acheter\textsc{.subj} (le-film)

    \textit{C'est le film que Ion veut voir, et Maria acheter } 

Cependant, le gapping n'est pas très naturel avec les autres relations de ressemblance, comme l'exemplification, la généralisation, l'exception ou encore l'élaboration, bien qu'elles soient symétriques. Ces relations ne réalisent pas un contraste approprié entre les éléments parallèles dans une paire contrastive (Kehler (2000, 2002)). Ainsi, dans le cas de l'exemplification en \REF{ex:4:148}a et de la généralisation \REF{ex:4:148}b, on a des paires dans lesquelles un élément subsume un autre : dans les paires {\textless}\textit{plantele medicinale, sunătoarea}{\textgreater}, {\textless}\textit{anumite boli, durerile de stomac}{\textgreater}, le deuxième élément de chaque paire est l'hyponyme du premier ; dans les paires {\textless}\textit{Cristea, oamenii politici{\textgreater}}, \textit{{\textless}pe țărani, pe cei neinstruiți}{\textgreater}, le premier élément est une instance de la classe dénotée par le deuxième élément ; or, cela contredit la première condition observée pour le contraste sémantique. 


\begin{enumerate}
\item \label{bkm:Ref289627548}a  ??Plantele medicinale pot trata anumite boli, \textbf{spre exemplu} sunătoarea durerile de stomac.


\end{enumerate}
{\itshape
Les plantes médicinales peuvent guérir certaines maladies, par exemple la verveine les maux d'estomac } 

  b  ??Cristea îi manipulează pe țărani şi, \textbf{în general}, oamenii politici pe cei neinstruiți. 

{\itshape
Cristea manipule les paysans et, en général, les hommes politiques les gens sans instruction } 

Bien que ce ne soit pas la relation privilégiée, le gapping peut apparaître avec une relation de contiguïté, c.-à-d. narration qui présente une séquence d'événements dans une progression temporelle (\citet{Hendriks2004}). 


\begin{enumerate}
\item Paul va cânta la pian, iar \textbf{apoi} Maria la vioară.


\end{enumerate}
{\itshape
  Paul va jouer du piano, et ensuite Maria au violon } 

En revanche, on observe que le gapping n'est pas préféré avec les relations de type cause-effet : résultat \REF{ex:4:150}a, concession \REF{ex:4:150}b, condition \REF{ex:4:150}c. 


\begin{enumerate}
\item \label{bkm:Ref289623742}a  \#Copilul era grav bolnav şi, \textbf{prin urmare}, părinții lui extrem de nefericiți.


\end{enumerate}
{\itshape
L'enfant était gravement malade, et, par conséquent, ses parents extrêmement malheureux } 

  b  \#Alex era supărat şi \textbf{totuşi} prietena lui extrem de voioasă. 

    \textit{Alex était fâché et cependant sa copine extrêmement gaie}

  c  \#Ion va pleca la Paris sau, \textbf{în caz contrar}, Maria la Roma. 

    \textit{Ion va partir à Paris ou sinon Maria à Rome } 

Selon le principe du contraste symétrique/équilibré (\textit{principle of balanced contrast}) de \citet{Repp2009}, les conjoints dans une construction à gapping doivent avoir le même type de contribution par rapport à un topique discursif. Or, comme le remarque \citet{Hendriks2004}, les relations de type cause-effet, contrairement aux relations de parallélisme et contraste, construisent un topique non-contrastif. Les conjoints dans une relation cause-effet ne peuvent pas être tous des réponses adéquates à une question multiple implicite, comme c'est le cas des relations symétriques (cf. section suivante 4.3.1.3). 

Par conséquent, on doit maintenir l'idée d'un parallélisme discursif fort dans les constructions à gapping, qui privilégie les relations symétriques de parallélisme et contraste. 

\paragraph[Structure informationnelle]{Structure informationnelle}
\label{bkm:Ref289439451}On a vu dans la section  que le roumain n'exigeait pas un parallélisme syntaxique strict en ce qui concerne l'ordre des éléments résiduels et corrélats. Mais comment expliquer l'inacceptabilité de l'exemple en \REF{ex:4:151} ? Le but de cette section est de montrer l'importance de la structure informationnelle pour la légitimation d'une coordination à gapping, en particulier le gapping avec la conjonction \textit{iar}, qui est de loin la conjonction la plus fréquente avec ce type d'ellipse. 


\begin{enumerate}
\item \label{bkm:Ref289631912}\#Ioana mănâncă un măr, iar o pară Maria.


\end{enumerate}
{\itshape
Ioana mange une pomme, et une poire Maria } 

Le parallélisme sémantique fort, qui exige des paires contrastives, est corrélé à un parallélisme informationnel (\citet{Winkler2005}). Deux aspects sont à discuter ici : (i) le rapprochement avec les questions multiples, et (ii) les notions de \textit{topique (contrastif)} et \textit{focus (informationnel)}.

On a proposé de rapprocher les constructions à gapping des couples question-réponse, en particulier de la notion de \textit{congruence} des réponses (\citet{Kuno1982}, \citet{Steedman2000}, \citet{Reich2006}, \citet{Winkler2005}, \citet{Hoyt2008}, \citet{Repp2009}). La motivation pour une telle approche vient du fait que l'acceptabilité des phrases trouées semble être fortement dépendante du contexte discursif, comme l'avait suggéré \citet{Kuno1976}, \citet{Prince1986}, \citet{Steedman2000}, etc. Ainsi, les constructions à gapping sont acceptables lorsqu'on présuppose une proposition ouverte (\textit{open proposition}, cf. \citet{Prince1986}), qui a la forme d'une question multiple. Je reprends l'affirmation de \citet[212]{Prince1986} : {\guillemotleft}~gappings are felicitous just in case they can be taken to instantiate an OP [=open proposition] corresponding to the full conjunct, where the leftmost constituents bear the same sort of anaphoric (set) relation to something in the prior context found in Topicalization and where the rightmost constituents instantiate the variable in the OP.~{\guillemotright}

Dans cette perspective, le prototype discursif dans le gapping est une réponse en liste de paires à une question multiple implicite \REF{ex:4:152}. Les conjoints (source et troué) dans une construction à gapping mettent en valeur une même question en discussion (QUD, c.-à-d. \textit{Question Under Discussion}), cf. \citet{Reich2006}. Voir dans ce sens la remarque de \citet[248]{Steedman1990} : {\guillemotleft}~even the most basic gapped sentence, like \textit{Fred ate bread, and Harry, bananas}, is only really felicitous in contexts which support (or can accomodate) the presupposition that the topic under discussion is \textit{Who ate what}.~{\guillemotright} 


\begin{enumerate}
\item \label{bkm:Ref299699277}A : - Who ate what ?


\end{enumerate}
  B : - Fred ate bread, and Harry bananas.

Cette hypothèse se justifie empiriquement dans les constructions à gapping avec \textit{iar} en roumain \REF{ex:4:153}b (et avec \textit{a} en russe \REF{ex:4:153}c, cf. \citet{Kazenin2001}, Jasinskaja \& \citet{Zeevat2009}). Cette conjonction spécialisée pour le double contraste au niveau de la phrase et la plus fréquente avec~le gapping~relie des phrases qui répondent implicitement à une question multiple, cf. Bîlbîie \& \citet{Winterstein2011} (pour le russe \textit{a}, voir Jasinskaja \& \citet{Zeevat2010}). La description synthétique des deux conjonctions figure en \REF{ex:4:154}. La spécification la plus importante concerne le premier trait ${\lnot}$SINGLE : cela rend compte du fait que les deux conjonctions ne peuvent pas lier des phrases qui répondent à une question contenant un seul mot \textit{qu-} (cf. le trait ${\lnot}$SINGLE) ; chaque conjoint doit être une réponse à une question contenant au moins deux éléments \textit{qu-}. Les autres traits sont moins importants pour la discussion de cette section (ils sont expliqués dans le chapitre 2 de cette thèse, section \ref{sec:2.9.5} et dans Bîlbîie \& \citet{Winterstein2011}). 


\begin{enumerate}
\item \label{bkm:Ref289640885}a  \textbf{Qui} aime \textbf{quoi} ? 


\end{enumerate}
  b  Lui Ion îi place fotbalul, iar Mariei baschetul. 

    Ion\textsc{.dat cl} plaît le-football, \textsc{conj} Maria\textsc{.dat} le-basketball

{\itshape
Ion aime le football, et Maria le basketball}

  c  Oleg ljubit futbol \{\textbf{a /} ??i\} Roma basketbol. 

{\itshape
Oleg aime le football, et Roma le basketball } 


\begin{enumerate}
\item \label{bkm:Ref289640714}Conjonctions {\guillemotleft}~contrastives~{\guillemotright} en roumain et russe


\end{enumerate}

\begin{table}


\begin{tabular}{lll}

roum. \textit{iar} &  & ${\lnot}$SINGLE,${\lnot}$CORRECTION,${\lnot}$(WHETHER,2\textsuperscript{nd})\\
  russe \textit{a}\par &  & ${\lnot}$SINGLE, ${\lnot}$(WHETHER,2\textsuperscript{nd},WHY)\\
\end{tabular}

\caption{}
%\label{}
\end{table}

Si les paires contrastives fournissent les réponses à une question multiple (implicite), est-ce que leur contribution est identique du point de vue informationnel ? En particulier, quel est le statut informationnel des éléments résiduels ? Dans la littérature, on trouve deux analyses possibles pour le statut informationnel des éléments résiduels : (i) tous les éléments résiduels sont des focus (\citet{Kuno1976}, \citet{Hartmann2000}), ou (ii) un des éléments résiduels est un topique (\citet{Winkler2005}, \citet{Repp2009}, Konietzko \& \citet{Winkler2010}). Dans toutes ces analyses, le topique et/ou le focus en question sont contrastifs, c.-à-d. ils sont associés à des alternatives.

La première analyse ne peut pas tenir pour le roumain (cf. Bîlbîie \& \citet{Winterstein2011}) et le russe (cf. Jasinskaja \& \citet{Zeevat2009}), au moins pour le gapping avec \textit{iar} (et respectivement \textit{a} en russe). Je discute par la suite les éléments qui justifient l'analyse de la phrase trouée introduite par \textit{iar} en roumain comme une séquence topique-focus.

De manière générale, on observe qu'en roumain l'ordre naturel des éléments suit la structuration du discours (section \ref{sec:1.3.4} du chapitre 1). Ainsi, la différence majeure entre \REF{ex:4:155} et \REF{ex:4:156} consiste dans l'ordre relatif des personnes (\textit{Ioana, Maria}) et des activités (\textit{cinéma, théâtre}), en fonction du type de question posée. Le placement naturel de l'élément qui résout la question est à la fin du conjoint, alors que l'élément distingué pour répondre à une question (c.-à-d. celui qui donne une indication sur la manière de résoudre la question, cf. la notion de \textit{sorting key} de \citet{Kuno1982}) apparaît en position initiale.  


\begin{enumerate}
\item \label{bkm:Ref289643790}A :  Cu cine ieşi la film şi cu cine la teatru ?


\end{enumerate}
{\itshape
Avec qui tu sors au cinéma et avec qui au théâtre}

  B :  La film, ies cu Ioana, iar la teatru cu Maria. 

{\itshape
Au cinéma je sors avec Ioana, et au théâtre avec Maria}

  C :  \#Cu Ioana ies la film, iar cu Maria la teatru. 

{\itshape
    Avec Ioana je sors au cinéma, et avec Maria au théâtre } 


\begin{enumerate}
\item \label{bkm:Ref289643809}A :  Unde ieşi cu fetele weekendul ăsta ?


\end{enumerate}
{\itshape
Où est-ce que tu sors avec tes filles ce weekend}

  B :  Cu Ioana ies la film, iar cu Maria la teatru. 

{\itshape
Avec Ioana je sors au cinéma, et avec Maria au théâtre } 

  C :  \#La film ies cu Ioana, iar la teatru cu Maria. 

{\itshape
Au cinéma je sors avec Ioana, et au théâtre avec Maria } 

On peut expliquer ces différences si on fait appel aux notions de \textit{topique contrastif} et \textit{focus informationnel}, telles que définies par Büring (2003). Je dois préciser que la notion de \textit{topique contrastif} de Büring (2003) correspond à la notion de {\guillemotleft}~clé de tri~{\guillemotright} (angl. \textit{sorting key}) de \citet{Kuno1982} ; la seule différence entre les deux notions concerne le domaine d'application : \textit{sorting key} est un terme utilisé par Kuno exclusivement dans les couples question-réponse, alors que le \textit{topique contrastif} est un terme utilisé de manière plus générale dans le modèle de Büring. Les phrases présentent deux valeurs sémantiques associées à un topique (contrastif) et respectivement à un focus (informationnel). Le topique contrastif est inclus dans la question à laquelle répond l'énoncé et, d'une manière générale, est l'élément qui est saillant dans le discours (dans beaucoup de cas, il reprend un élément déjà mentionné dans le discours). En revanche, le focus informationnel est l'élément qui répond à la question, indiquant l'information nouvelle. Généralement, celui-ci est marqué par un contour prosodique spécifique. 

A travers la littérature, la plupart des exemples avec gapping ont comme première paire contrastive un ensemble de noms propres. Si l'on veut observer le comportement informationnel des éléments résiduels, il s'avère difficile de le faire en se limitant à ce type d'expressions, car ils sont discursivement neutres, c.-à-d. ils peuvent être utilisés en toutes circonstances, l'utilisation d'un nom propre demandant simplement que les interlocuteurs sachent de qui ils parlent. Par conséquent, j'utilise deux moyens spécifiques, afin de tester le statut informationnel des éléments résiduels introduits par \textit{iar~}: (i) la réalisation emphatique d'un accent lexical (c.-à-d. saillance prosodique), et (ii) la variation entre les syntagmes nominaux spécifiés par un déterminant indéfini vs. déterminant défini. 

L'identification du focus informationnel peut être forcée par la présence d'une saillance prosodique sur l'élément en question. Comme notation dans les exemples suivants, j'utilise le gras pour indiquer quelle est la conjonction la plus naturelle et préférée par les locuteurs ; je marque la saillance prosodique en utilisant simultanément les majuscules et le gras. Si l'on regarde le premier conjoint, on observe que le focus informationnel n'a pas un ordre contraint (il peut être en position finale -- et c'est l'ordre habituel --, ou bien en position initiale, et cela uniquement s'il reçoit une saillance prosodique). En revanche, si l'on regarde le conjoint introduit par la conjonction \textit{iar}, on observe que, indépendamment de l'ordre des éléments dans la phrase source, les locuteurs ne préfèrent pas avoir en position initiale un élément résiduel distingué prosodiquement. Par conséquent, dans le dernier exemple de \REF{ex:4:157} et \REF{ex:4:158}, le parallélisme syntaxique est violé, afin d'éviter un focus informationnel en première position dans la phrase trouée. On voit donc que l'ordre dans lequel ils apparaissent ne correspond pas nécessairement à l'ordre de leurs corrélats dans la phrase source (\textit{contra} Konietzko \& \citet{Winkler2010}).  


\begin{enumerate}
\item \label{bkm:Ref289683478}A :  La film, ies cu I\textbf{OA}na, \{\textbf{iar} / şi\} la teatru cu Ma\textbf{RI}a.


\end{enumerate}
{\itshape
Au cinéma je vais avec Ioana, et au théâtre avec Maria}

  B :  Cu I\textbf{OA}na ies la film, \{\#iar / ?şi\} cu Ma\textbf{RI}a la teatru. 

{\itshape
Avec Ioana je vais au cinéma, et avec Maria au théâtre}

  C :  Cu I\textbf{OA}na ies la film, \{\textbf{iar} /~şi\} la teatru cu Ma\textbf{RI}a. 

{\itshape
    Avec Ioana je vais au cinéma, et au théâtre avec Maria } 


\begin{enumerate}
\item \label{bkm:Ref289683484}A :  Cu Ioana ies la \textbf{FILM}, \{\textbf{iar} / şi\} cu Maria la \textbf{TEA}tru.


\end{enumerate}
{\itshape
Avec Ioana je vais au cinéma, et avec Maria au théâtre}

  B :  La \textbf{FILM} ies cu Ioana, \{\#iar / ??şi\} la \textbf{TEA}tru cu Maria. 

{\itshape
Au cinéma je vais avec Ioana, et au théâtre avec Maria}

  C :  La \textbf{FILM} ies cu Ioana, \{\textbf{iar} /~şi\} cu Maria la \textbf{TEA}tru. 

{\itshape
    Au cinéma je vais avec Ioana, et avec Maria au théâtre } 

Un argument supplémentaire justifiant la partition topique-focus dans la phrase trouée introduite par \textit{iar} vient des différences qu'on observe avec les syntagmes nominaux accompagnés d'un déterminant. Ainsi, on préfère avoir comme premier élément résiduel un syntagme nominal spécifié par un déterminant défini (\textit{stiloul} `le stylo') plutôt qu'un syntagme nominal avec un déterminant indéfini (\textit{un stilou} `un stylo'). Or, on suppose habituellement qu'une expression référentielle définie introduit un référent \textit{connu/identifiable} dans le discours, alors qu'une expression référentielle indéfinie doit désigner un référent non préalablement identifié, c.-à-d. information \textit{nouvelle}. Il faudrait vérifier pourquoi les indéfinis génériques sont acceptables en \REF{ex:4:159}c ; une explication est qu'ils se comportent différemment des autres indéfinis et qu'ils se rapprochent en quelque sorte du fonctionnement d'un syntagme nominal défini ou un nom propre. 


\begin{enumerate}
\item \label{bkm:Ref289685475}a  \#Mariei i-am oferit o carte, iar \textbf{un stilou} Ioanei.


\end{enumerate}
{\itshape
A Marie j'ai offert un livre, et un stylo à Ioana } 

  b  MaRIei i-am oferit cartea, iar \textbf{stiloul} IOAnei. 

{\itshape
A Marie j'ai offert le livre, et le stylo à Ioana } 

  c  O casă costă 200.000 de euro, iar \textbf{o maşină} 20.000. 

{\itshape
Une maison coûte 200.000 euros, et une voiture 20.000 } 

L'analyse des constructions avec \textit{iar} montre ainsi que le premier élément résiduel de la phrase trouée est un topique contrastif. Selon \citet{Winkler2005}, un topique contrastif présente trois propriétés : (i) il a une intonation montante (mais cette propriété varie à travers les langues), (ii) il occupe une position initiale dans la phrase, et (iii) cf. Molnàr (1998), il exige la présence dans le même conjoint d'un focus contrastif. Un focus contrastif est différent d'un focus non-contrastif (cf. \citet{Repp2010}) : il appartient à un ensemble d'alternatives fermé/restreint, dont les alternatives sont identifiables dans le discours ; ce qu'on dit sur le focus contrastif ne peut pas s'appliquer à un autre élément du même ensemble, p.ex. à son corrélat.

On arrive ainsi à distinguer entre le gapping \REF{ex:4:160}a, dont la phrase trouée est une séquence topique contrastif -- focus contrastif, et les BAE \REF{ex:4:160}b-c, dont la séquence elliptique contient simplement un focus contrastif.\footnote{A priori, les cas de stripping avec un adverbe propositionnel seraient différents des BAE. Voir les données de Konietzko \& \citet{Winkler2010}, où le premier élément précédant l'adverbe propositionnel est interprété comme un topique contrastif. A priori, cette hypothèse semble être correcte pour le roumain, vu la possibilité d'employer la conjonction \textit{iar} dans ces contextes, alors que cela n'est pas possible pour les BAE. 
(i)  a  Ioana joacă [volei]\textsubscript{CT}, iar [tenis]\textsubscript{CT} de asemenea. 
    \textit{Ioana joue voleyball, et tennis de même}
  b  Ioana joacă [volei]\textsubscript{CT}, dar [tenis]\textsubscript{CT} nu.
    \textit{Ioana joue voleyball, mais tennis non}}  


\begin{enumerate}
\item \label{bkm:Ref289688955}a  [Ioana]\textsubscript{CT} joacă [volei]\textsubscript{CF}, iar [Maria]\textsubscript{CT} [tenis]\textsubscript{CF}.


\end{enumerate}
    \textit{Ioana joue volleyball, et Maria tennis}

  b  Ioana joacă [volei]\textsubscript{CF}, \{şi / ??iar\} [nu tenis]\textsubscript{CF}. 

{\itshape
Ioana joue voleyball, et Maria tennis } 

  c  Ioana joacă [volei]\textsubscript{CF}, \{dar /~*iar\} [şi tenis]\textsubscript{CF}. 

{\itshape
Ioana joue voleyball, mais aussi tennis}

\citet{Schwabe2000}~va plus loin et considère que la structure informationnelle joue un rôle très important dans l'identification d'une structure syntaxique ou sémantique pour les constructions elliptiques, en particulier, c'est le deuxième conjoint qui détermine la structure informationnelle de l'antécédent (cf. Rooth (1992, 1996)). Cela se rapproche de l'hypothèse de \citet{Kuno1982} qui considère le premier élément dans une construction à gapping comme une {\guillemotleft}~clé de tri~{\guillemotright} (\textit{sorting key}) : {\guillemotleft}~in a multiple \textit{wh-}word question, the fronted \textit{wh-}word represents the key for sorting relevant pieces of information in the answer. {\guillemotright}  \citet[141]{Kuno1982}. Par conséquent, en l'absence d'un verbe, on arrive à avoir une bonne interprétation des éléments résiduels grâce à la structure informationnelle aussi : on identifie le topique contrastif dans la phrase trouée, on lui trouve le corrélat dans la phrase source ; l'autre élément résiduel sera un focus contrastif, corrélé à un autre focus contrastif dans la phrase source. 

Je finis cette section en précisant que le matérial manquant doit être \textit{donné} dans le discours. A part le verbe antécédent, tout autre matériel qui fait partie du fond (dans la phrase source) ne peut être répété dans la phrase trouée \REF{ex:4:161}a, sauf s'il facilite le processing \REF{ex:4:161}b (pour les résiduels qui ne sont pas des dépendants directs par rapport au verbe antécédent). 


\begin{enumerate}
\item \label{bkm:Ref289690938}a  Ioana vine mâine cu trenul, iar Maria (\#mâine) cu autobuzul.


\end{enumerate}
{\itshape
Ioana vient demain en train, et Maria (demain) en bus} 

  b  Maria a luat trenul care merge la Briançon, iar Ion ??(trenul care merge) la Saint-Gervais. 

{\itshape
Maria a pris le train qui va à Briançon, et Ion le train qui va à Saint-Gervais}

Pour conclure, on observe qu'on doit postuler aussi un parallélisme informationnel pour les constructions à gapping~présentant la conjonction \textit{iar} en roumain : le résiduel doit avoir le même statut informationnel que son corrélat. En particulier, une construction à gapping contient au moins une paire contrastive avec des topiques et une paire contrastive avec des focus. Une étude détaillée reste à faire afin de vérifier si cette généralisation s'applique aussi en dehors des coordinations avec \textit{iar}.  

Prosodie

En ce qui concerne la prosodie, on s'est posé la question de savoir si le gapping est associé à une prosodie spéciale. Les principales hypothèses ont été faites sur l'anglais et l'allemand (\citet{Hartmann2000}, \citet{Schwarz2000}, Carlson (2001, 2002), Féry \& \citet{Hartmann2005}, \citet{Winkler2005}). Il y a un travail en cours (basé sur des expériences de production) pour le français (Abeillé \textit{et al.} \textit{en prép.}). Quant au roumain, cela n'a pas été étudié et le sujet ne pourra pas être abordé en détail dans cette thèse. Dans cette section, je veux simplement énumérer les points qui semblent importants à étudier quand on s'intéresse au marquage prosodique des constructions à gapping.

Les recherches faites dans ce sens tournent autour de cinq questions majeures : (i) Est-ce que le verbe antécédent est marqué prosodiquement ? (ii) Est-ce qu'à la place du matériel manquant on a une marque prosodique aussi ? (iii) Est-ce que les éléments contrastifs reçoivent un accent (particulier) ? ou uniquement les éléments résiduels ? (iv) Est-ce qu'il y a une pause entre les conjoints ? (v) Est-ce qu'on observe une compression de registre sur le deuxième conjoint ?

Je me limite ici à simplement résumer les principales hypothèses qui dérivent des travaux faits sur la prosodie de ces constructions.  

(i) On considère que le verbe antécédent est typiquement désaccentué, ce qui permet l'ellipse dans le conjoint troué (\citet{Hartmann2000}, \citet{Schwarz2000}). Cependant, l'étude en cours faite sur le français, qui compare les séquences elliptiques avec leurs contreparties complètes, observe que la désaccentuation du matériel antécédent dans la phrase source ne caractérise pas seulement les constructions elliptiques, mais elle peut apparaître aussi dans une coordination non-elliptique.  

(ii) Dans une approche syntaxique de l'ellipse, on s'attend à ce qu'il y ait toujours une rupture à l'endroit attendu du matériel manquant. Mais il n'y a pas toujours de pause à l'endroit où se trouve le matérial manquant. En français, Abeillé \textit{et al. en prép.} remarquent un enchaînement prosodique entre les éléments résiduels de la séquence trouée. 

(iii) On considère qu'au moins les éléments résiduels reçoivent un accent contrastif, qui est plus fort qu'un accent syntagmatique habituel, ce qui explique l'impossibilité d'avoir des pronoms inaccentués (ou faibles) dans la phrase trouée (cf. Sag \textit{et al.} (1985)) :


\begin{enumerate}
\item *You talked to John's mother, and I him.  [\textit{him} sans accentuation prosodique]


\end{enumerate}
{\raggedleft
 (Sag \textit{et al.} (1985 : 161))
}

\citet{Hartmann2000} et Féry \& \citet{Hartmann2005} considèrent que la prosodie souligne le contraste sémantique : les différents accents sur les éléments résiduels, ainsi que (parfois) sur leurs corrélats facilitent la construction des paires contrastives. Selon Féry \& \citet{Hartmann2005}, les éléments résiduels et corrélats non-finaux reçoivent un pitch accent montant (L*H), alors que les éléments résiduels et corrélats finaux reçoivent un accent descendant (H*L). 

(iv) De manière générale, chaque conjoint constitue une unité prosodique autonome, c.-à-d. il y a une frontière marquée par une pause intonationnelle entre les conjoints. Dans les langues citées plus haut, on observerait ainsi un ton de frontière haut (optionnel) à la fin de la phrase source et un ton de frontière bas dans la phrase trouée. Néanmoins, la pause intonationnelle reste optionnelle. Voir, dans ce sens, les coordinations à gapping avec portée large de la négation, où les conjoints forment une seule unité prosodique (cf. \citet{Oehrle1987}, \citet{Winkler2005}).

(v) Féry \& \citet{Hartmann2000} observent une compression de registre dans la séquence trouée. Cependant, l'étude préliminaire faite sur le français montre que le registre du deuxième conjoint est rarement compressé. 

Les travaux faits sur l'anglais et l'allemand attribuent la spécificité prosodique du gapping au phénomène d'ellipse ; on tirerait donc de l'analyse prosodique des arguments en faveur d'une analyse par ellipse. Cependant, en l'absence d'une description des mêmes contextes sans ellipse, on ne peut pas attribuer une spécificité prosodique aux constructions à gapping.

Une étude détaillée reste à faire pour le roumain, afin de confirmer ou infirmer ces hypothèses avancées sur l'anglais et l'allemand. Pour le français, une partie de ces hypothèses semble s'infirmer, cf. l'étude en cours de Abeillé \textit{et al.} \textit{en prép}.

L'hypothèse qu'on pourrait faire est que l'intonation dans le gapping est plutôt sensible aux aspects sémantiques (parallélisme entre les paires contrastives) et pragmatiques (paire de topiques et paire de focus), et moins aux aspects syntaxiques (\textit{contra} Féry \& \citet{Hartmann2005}), mais cela reste à être vérifié.

\subsection{Les analyses proposées et leurs limites}
\label{bkm:Ref300922524}Les analyses proposées dans la littérature pour traiter les constructions à gapping sont nombreuses et variées. La plupart de ces travaux ont comme cadre théorique le courant dominant de la Grammaire Générative sous ses différentes formes (Théorie Standard étendue, Principes et Paramètres, Programme Minimaliste). Le point commun de toutes ces approches, par-delà leur hétérogénéité, réside dans le rôle très important attribué à la syntaxe pour obtenir l'interprétation de la phrase trouée, la plupart faisant appel à un mécanisme de reconstruction syntaxique. L'idée générale est que le matériel manquant a une certaine structure syntaxique à un certain niveau de la représentation, ce qui justifie l'appellation de \textit{structural approaches}, selon la terminologie proposée par \citet{Merchant2009}. Leur but est de trouver une solution qui aligne la séquence trouée sur un constituant ordinaire. Les approches syntaxiques peuvent être ainsi synthétisées en suivant deux critères : (i) la taille postulée pour la séquence trouée, en fonction du niveau auquel opère la coordination, et (ii) la nature du matériel manquant.  

Cette section a quatre parties. Les deux premières parties développent chacune un des deux critères mentionnés, afin d'avoir une synthèse des propositions faites au sein des approches structurales. La troisième partie montre les problèmes qu'on rencontre avec les deux analyses dominantes, l'effacement et le mouvement du verbe. Finalement, on présente brièvement les analyses alternatives proposées dans une perspective non-structurale, qui me permettront de développer ensuite une analyse constructionnelle en HPSG.

\subsubsection{Taille de la séquence trouée}
Le premier critère qu'on peut utiliser pour résumer~les analyses proposées est le niveau auquel opère la coordination. On arrive ainsi à deux types majeurs : les approches à grand conjoint (\textit{large conjunct approach}) et les approches à petit conjoint (\textit{small conjunct approach}). Dans les deux cas, on part du principe que la coordination opère uniquement entre des constituants de même niveau. Les approches à grand conjoint considèrent que la coordination se passe au niveau supérieur de la phrase (dans les cadres qui postulent des catégories fonctionnelles, cela correspond au CP cf. angl. \textit{Complementizer Phrase}, TP cf. angl. \textit{Tense Phrase}, ou encore IP cf. angl. \textit{Inflexion Phrase}, en fonction des notations choisies). En revanche, le deuxième type d'approche voit la coordination comme ayant lieu à un niveau inférieur, c.-à-d. en-dessous de TP, au niveau VP (\textit{Verb Phrase}) ou \textit{v}P (\textit{Voice Phrase}).  

\paragraph[Coordination au niveau de la phrase]{Coordination au niveau de la phrase}
Les travaux inscrits dans cette perspective débutent avec Ross (1967, 1970) et continuent avec \citet{Jackendoff1971}, Hankamer (1973, 1979), \citet{Stillings1975}, \citet{Sag1976}, \citet{Neijt1979}, van \citet{Oirsouw1987}, Wilder (1994, 1997), Abe \& Hoshi (1997, 1999), \citet{Kim1997}, \citet{Hartmann2000}, etc. Selon eux, dans les constructions à gapping, on coordonne deux phrases. Le conjoint troué a une structure complexe similaire à celle de la phrase source. 

Pour dériver le trou, on fait appel à un certain mécanisme syntaxique de réduction, par lequel le verbe et éventuellement d'autres éléments sont effacés au niveau PF (\textit{Phonological Form}) ou copiés au niveau LF (angl. \textit{Logical Form}), toujours en correspondance avec la phrase source. 

Dans les cadres postulant un homomorphisme entre les relations de constituance syntaxique et les relations de portée sémantique, cette approche rend compte tout de suite des exemples où la négation présente dans la phrase source se distribue sur chaque conjoint, cf. les données de \citet{Repp2009}.


\begin{enumerate}
\item a   Max didn't read the book and Martha the magazine.  


\end{enumerate}
  b  = [It is not the case that Max read the book] and [it is not the case that Martha read the magazine]. 

Elle est justifiée aussi par les coordinations de CPs (dans la terminologie de \citet{Repp2009}, dans lesquelles les éléments résiduels sont des syntagmes \textit{qu-} ou encore des syntagmes antéposés, qui généralement sont considérés comme étant plus haut que le syntagme verbal dans la structure syntaxique. 


\begin{enumerate}
\item a   When did John arrive and when Mary ?          (\citet[34]{Repp2009})


\end{enumerate}
  b  On this table, they put a lamp, and on that table, a radio.     (Sag \textit{et al.} (1985 : 158))

Un argument de plus pour considérer une coordination à un niveau supérieur est fourni par l'impossibilité d'avoir une asymétrie de voix dans le gapping. Merchant (2008a, 2008b) corrèle la {\guillemotleft}~taille~{\guillemotright} de l'ellipse avec la possibilité ou non d'avoir des asymétries de voix entre le syntagme elliptique et la phrase source. On distingue ainsi les ellipses {\guillemotleft}~hautes~{\guillemotright} (élidant plus que le simple syntagme verbal, comme le sluicing -- l'ellipse du TP) des ellipses {\guillemotleft}~basses~{\guillemotright} (p.ex. VPE). Seul le dernier type d'ellipse permet les asymétries de voix. Comme les asymétries ne sont pas permises dans les constructions à gapping, on pourrait considérer, selon~le raisonnement de Merchant, que le gapping est une opération qui a lieu à un niveau supérieur, c.-à-d. celui de la phrase.

Les problèmes de ce type d'approche ont été discutés à plusieurs reprises par Johnson (1996/2004, 2000, 2009) et seront mentionnés dans la section suivante. 

\paragraph[Coordination sous-phrastique]{Coordination sous-phrastique}
\label{bkm:Ref288588249}Le deuxième type d'approche apparaît dans les travaux de Siegel (1984, 1987), Johnson (1996/2004, 2000, 2008, 2009), \citet{Coppock2001}, \citet{Lin2002}, López \& \citet{Winkler2003}, \citet{Winkler2005}, \citet{Hulsey2008}, Agafonova \textit{à paraître}, etc. Dans cette perspective, le conjoint troué est plus petit qu'une phrase, c.-à-d. il est un VP/\textit{v}P (le syntagme verbal dans lequel le sujet est généré). 

Je mentionne par la suite les arguments inventoriés par les adeptes de cette approche contre l'analyse à grand conjoint, tels qu'ils apparaissent dans les discussions de Johnson (1996/2004, 2000, 2009)\footnote{Les exemples qui n'ont aucune indication concernant l'auteur viennent de Johnson (2000, 2009).}. 

Si l'on admet l'homomorphisme syntaxe-sémantique (comme c'est le cas dans les grammaires dérivationnelles), l'approche à grand conjoint prédit le fait qu'aucun élément de la phrase source ne peut lier ou avoir portée sur un élément dans la phrase trouée. Or, cette prédiction s'avère être fausse. Dans les constructions à gapping, il y a des cas de coréférence croisée, ainsi que des situations où un élément prend portée large sur toute la coordination, comme le notent \citet{Oehrle1987} et Mc\citet{Cawley1993} pour le premier cas, et Siegel (1984, 1987) et \citet{Oehrle1987} pour le deuxième. 

Tout d'abord, certaines constructions à gapping permettent la coréférence entre le sujet de la phrase source et une expression anaphorique dans le deuxième sujet, à condition que le verbe ne soit pas répété dans le deuxième conjoint (c.-à-d. à condition qu'il y ait du gapping, cf. les jugements différents d'acceptabilité en \REF{ex:4:165}a et \REF{ex:4:165}b). Johnson explique cette possibilité par le fait que le quantifieur présent dans la phrase source est en dehors du domaine de la coordination et c-commande le pronom en question, ce qui prédit correctement la coréférence. 


\begin{enumerate}
\item \label{bkm:Ref289807670}a   No woman\textsubscript{i} can join the army and her\textsubscript{i} girlfriend the navy.  


\end{enumerate}
  b  *No woman\textsubscript{i} can join the army and her\textsubscript{i} girlfriend can join the navy. 

Ensuite, il y a des exemples comme \REF{ex:4:166}a avec un modal nié qui a portée large sur la coordination, n'ayant pas une lecture distributive dans chaque conjoint. Même observation pour un quantifieur présent dans le syntagme sujet de la phrase source \REF{ex:4:167}a, ou encore un adverbe quantificationnel \REF{ex:4:167}b. L'approche à petit conjoint prédit correctement ces données, car tous ces éléments se situent en dehors du domaine de la coordination. 


\begin{enumerate}
\item \label{bkm:Ref302325523}a  Mrs. Smith \textbf{can't} dance or Mr. Smith sing.  


\end{enumerate}
  b  = Mrs. Smith can't dance and Mr. Smith can't sing.

  c  ${\neq}$ Mrs. Smith can't dance or Mr. Smith can't sing.


\begin{enumerate}
\item \label{bkm:Ref289807683}a   \textbf{Not every} girl ate a GREEN banana and her mother a RIPE one.  


\end{enumerate}
b  A German stepherd is \textbf{rarely} named Kelly or an Irish setter Fritz.

Un argument de plus, invoqué par \citet{Siegel1987}, est constitué par la discordance casuelle qu'on observe en anglais avec les pronoms sujets de la phrase source et respectivement de la séquence trouée. Tandis que le premier sujet est toujours au nominatif, on constate une préférence des locuteurs pour un pronom sujet à l'accusatif dans la séquence trouée, et apparemment c'est ce qui se passe en dehors de l'ellipse aussi. L'analyse à petit conjoint prédit ces cas, si on considère (dans les grammaires dérivationnelles) que le sujet de la séquence trouée ne monte pas vers IP pour vérifier ou recevoir le cas nominatif.\footnote{Le fait d'avoir le sujet à l'accusatif dans la séquence trouée n'est en rien un argument pour motiver le mouvement du verbe. On pourrait considérer l'accusatif ici comme la forme par défaut (comme dans les réponses courtes en (i)) ou, sinon, comme une idiosyncrasie morpho-syntaxique. D'ailleurs \citet{Kim2006} donne plusieurs exemples (ii) avec des coordinations de pronoms marqués différemment :
(i)  A : - Who's entering ? B : - \{Me / *I\}. 
(ii)  a  Them and us are going to the game together.
  b  She and him will drive to the movies.
  c  All debts are cleared between you and I.            (\citet[603]{Kim2006})} 


\begin{enumerate}
\item a   I cooked fish, and \{him / ?he\} rice.        (Zoerner \& \citet{Agbayani2000})


\end{enumerate}
  b  We can't eat caviar and \{him / ??he\} can't eat beans.     (\citet[184]{Winkler2005})

Un autre type de données difficile à concilier avec une analyse postulant une coordination phrastique dans les constructions à gapping est lié aux items à polarité négative qui peuvent apparaître dans la séquence trouée. Ce type d'approche considère que l'item à polarité négative dans le deuxième conjoint se trouve dans une position où la négation ne peut le c-commander, donc son légitimeur devrait être en dehors du conjoint qui le contient.  


\begin{enumerate}
\item During dinner he didn't address his colleagues from Stuttgart or at any time his boss, for that matter.                    (\citet[186]{Winkler2005})


\end{enumerate}
Cependant, cette analyse rencontre des difficultés. Un premier problème est lié à la portée de la négation (\citet{Repp2009}, \citet{Toosarvandani2011}). Cette approche prédit toujours la portée large de tous les éléments, mais elle ne prédit jamais la portée étroite de la négation, où on a une polarité différente dans les deux conjoints, cf. l'exemple de \citet[2]{Repp2009}.


\begin{enumerate}
\item a   Pete wasn't called by Vanessa, but John by Jessie.  


\end{enumerate}
  b  = [It is not the case that Pete was called by Vanessa] but [it is the case that John was called by Vanessa]. 

Un autre problème, relevé par \citet{Ince2009}, est la coréférence qui peut s'établir entre un objet pronominal dans le deuxième conjoint et le sujet de la phrase source. Si on applique l'approche de Johnson à l'exemple \REF{ex:4:171}, \textit{John} c-commande \textit{him} et en même temps il appartient au domaine de liage de \textit{him}. Ce qui a comme résultat une violation du principe B, selon lequel un pronom doit être libre dans son domaine de liage.


\begin{enumerate}
\item \label{bkm:Ref288645246}John\textsubscript{i} will hug Mary and Mary him\textsubscript{i}.            (\citet[205]{Ince2009})


\end{enumerate}
\citet{Ince2009} observe une autre difficulté de cette analyse en turc, où l'ordre des éléments résiduels et corrélats n'est pas le même. Si on a l'ordre SOV-OS comme en \REF{ex:4:172}b, l'objet résiduel est analysé ici comme un syntagme antéposé (TopP), et par conséquent il est nécessairement plus haut que le \textit{v}P. Donc, on ne voit pas comment obtenir une coordination de \textit{v}P, si au moins le deuxième conjoint n'est pas un \textit{v}P.


\begin{enumerate}
\item \label{bkm:Ref288644219}a  Adam kitab-{\i} okudu, çocuk dergi-yi.


\end{enumerate}
homme livre.\textsc{acc} lit, enfant revue.\textsc{acc}

{\itshape
L'homme lit le livre et l'enfant la revue}

 b  Adam kitab-{\i} okudu, dergi-yi çocuk.

homme livre.\textsc{acc} lit, revue.\textsc{acc} enfant

\textit{L'homme lit le livre et l'enfant la revue}      (\citet[199]{Ince2009})

Enfin, une coordination de \textit{v}P dans les constructions à gapping ne rend pas compte de l'occurence des adverbes de phrase dans la séquence trouée ou dans les deux conjoints (cf. \citet{Gardent1991}, \citet{Ince2009}).\footnote{Ce dernier aspect ne constitue pas un problème insurmontable, si l'on considère que les adverbes de phrase peuvent s'attacher à un verbe ou un syntagme verbal et prendre portée sur toute la phrase dans un exemple comme (i).
(i)  Paul necessarily will go to see Mary.}  


\begin{enumerate}
\item John will probably go to see Mary and necessarily Paul, Sarah.  (\citet[49]{Gardent1991})


\end{enumerate}
\subsubsection{Nature du matériel manquant}
Les approches syntaxiques {\guillemotleft}~structurales~{\guillemotright} peuvent être classées encore en fonction d'un autre critère, celui de la nature du matériel manquant. Les différentes solutions proposées pour la reconstruction syntaxique tombent dans une des trois classes majeures : (i) le trou est le résultat d'une ellipse par effacement au niveau PF ; (ii) le trou est une ellipse générée dès le départ sous la forme d'un élément vide qui est reconstruit au niveau LF ; (iii) le trou n'est pas obtenu par l'ellipse, mais il est la trace d'un mouvement. Dans ce dernier cas, le mouvement se prête à deux analyses : soit il opère simultanément dans les deux conjoints, c.-à-d. \textit{Across-the-Board movement}, soit il opère latéralement, c.-à-d. \textit{sideward movement}. Je présente brièvement les trois grandes analyses, mais j'insisterai plus sur les deux analyses {\guillemotleft}~en vogue~{\guillemotright}, effacement du verbe (donc, une ellipse à base de reconstruction syntaxique) vs. mouvement du verbe (donc, pas d'ellipse).

\paragraph[Effacement au niveau PF]{Effacement au niveau PF}
Dans cette approche, la phrase trouée a une structure syntaxique ordinaire, similaire à la phrase source, avec un verbe (et éventuellement d'autres éléments) identique à son antécédent. A un moment donné de la dérivation, lorsque la structure syntaxique est traitée par le module phonologique, le contenu phonologique du verbe / VP / TP est effacé (ou il n'y a pas d'insertion lexicale tardive), ce qui a comme résultat une structure {\guillemotleft}~silencieuse~{\guillemotright}, non-prononcée, en surface. Le matériel manquant est par conséquent présent et articulé au niveau syntaxique, mais non-réalisé au niveau phonologique. L'interprétation de la phrase trouée est obtenue avant l'effacement.

En fonction du type d'effacement qui entre en jeu, on peut avoir deux types d'analyses :

(i) Les analyses qui postulent simplement l'effacement du verbe (avec éventuellement d'autres éléments {\guillemotleft}~non-contrastifs~{\guillemotright}), sans invoquer une opération supplémentaire (Ross (1967, 1970), \citet{Jackendoff1971}, Hankamer (1973, 1979), \citet{Stillings1975}, \citet{Kuno1976}, \citet{Neijt1979}, van \citet{Oirsouw1987}, Wilder (1994, 1997), \citet{Hartmann2000}, Féry \& \citet{Hartmann2005}, etc.). Si le verbe est accompagné d'autres éléments, la règle du gapping opère un effacement de non-constituants.


\begin{enumerate}
\item John likes\textsubscript{} caviar and [\textsubscript{IP} Mary likes beans].  


\end{enumerate}
(ii) Les analyses qui postulent l'effacement d'un constituant syntagmatique incluant le trou, c.-à-d. un VP (dans les approches à petit conjoint) ou bien un TP/IP (dans les approches à grand conjoint)\footnote{Généralement, les adeptes de cette approche considèrent que la coordination dans les constructions à gapping a lieu au niveau de la phrase, mais pas tous (voir, par exemple, \citet{Coppock2001}, \citet{Lin2002}).}. L'effacement dans ces cas a lieu après avoir déplacé les éléments résiduels contrastifs\footnote{On assume généralement que les éléments résiduels se déplacent à gauche. Mais il y a des auteurs qui considèrent que le matériel lourd qui est interne au VP se déplace à droite, via l'opération de \textit{Heavy NP Shift} (\citet{Jayaseelan1990}, Kim (1997, 2006)). Un argument contre ce type d'opération vient du fait que les éléments résiduels peuvent ne pas être des NP lourds, p.ex. les pronoms, certains adverbes. }  (\citet{Sag1976}, \citet{Jayaseelan1990}, Kim (1997, 1998, 2006), \citet{Coppock2001}, \citet{Lin2002}, Konietzko \& \citet{Winkler2010}, Molnàr \& \citet{Winkler2010}, etc.).


\begin{enumerate}
\item John likes\textsubscript{} caviar and [\textsubscript{VP} Mary\textsubscript{1} [\textsubscript{VP} beans\textsubscript{2} [\textsubscript{VP} \textit{t}\textsubscript{1} likes \textit{t}\textsubscript{2}]]].      (\citet{Coppock2001}) 

\item John likes\textsubscript{} caviar and [\textsubscript{TopP} Mary\textsubscript{1} [\textsubscript{FocP} beans\textsubscript{2} [\textsubscript{TP} \textit{t}\textsubscript{1} likes \textit{t}\textsubscript{2}]]].  


\end{enumerate}
Contrairement à l'approche en termes de proforme nulle, le trou {\guillemotleft}~effacé~{\guillemotright} est syntaxiquement structuré, donc des opérations syntaxiques intervenant avant l'effacement phonologique (le mouvement, le liage, etc.) peuvent l'affecter au même titre que dans une phrase ordinaire. Par conséquent, le test utilisé par ces approches est la possibilité ou non d'extraire des éléments enchâssés dans le syntagme contenant le verbe effacé.

Une des analyses les plus récentes dans cette perspective est le travail de \citet{Coppock2001}, qui partage avec les approches à base de mouvement verbal l'idée selon laquelle la coordination dans les constructions à gapping a lieu au niveau \textit{v}P (c.-à-d. approche à petit conjoint). \citet{Coppock2001} essaie de justifier l'effacement en observant les similarités qui existent entre le gapping et les types d'ellipse pour lesquels on assume habituellement une analyse par effacement (en particulier, l'ellipse du VP). Elle mentionne trois aspects qui soutiennent les analyses à effacement, mais qui ne sont pas pris en charge, selon elle, par les analyses en termes de mouvement du verbe. En résumé, ce sont : i) la désambiguïsation de la portée et de l'anaphore, ii) les avantages empiriques de la condition e-GIVENness, et iii) la sélectivité dans l'application des contraintes d'îles. 

En ce qui concerne le premier aspect, \citet{Coppock2001} observe que le gapping, tout comme l'ellipse du VP, désambiguïse la portée des quantifieurs et les interprétations de l'anaphore.\footnote{\citet{Johnson2009} considère qu'en réalité ces deux aspects sémantiques ne posent aucun problème pour une analyse en termes de mouvement ATB. Les deux approches concurrentes en rendent compte.} En conformité avec la condition de parallélisme de portée, formulée par \citet{Fox2000}, le gapping réduit l'ambiguïté liée à la portée des quantifieurs : tandis qu'une phrase simple ordinaire contenant un quantifieur existentiel et un quantifieur universel présente une ambiguïté de portée en \REF{ex:4:177}a ( ${(\forall >\exists ),(\exists >\forall )}$), une phrase plus complexe avec une séquence elliptique en \REF{ex:4:177}b ne présente plus d'ambiguïté, le syntagme nominal résiduel du deuxième conjoint forçant une seule interprétation dans la phrase source aussi, c.-à-d.  ${\exists >\forall} $.  


\begin{enumerate}
\item \label{bkm:Ref288587311}a   A student accompanied every visitor.  ${(\forall >\exists ),(\exists >\forall )}$  


\end{enumerate}
  b  A student accompanied every visitor yesterday, and Mr. Johnson, today. * ${(\forall >\exists ),(\exists >\forall )}$ 

Comme VPE, le gapping élimine aussi l'ambiguïté liée à l'interprétation d'une anaphore dans un contexte avec plusieurs pronoms (angl. \textit{the Many-Pronouns Puzzle}, cf. Fiengo \& \citet{May1994}) : l'exemple non-elliptique en \REF{ex:4:178}a a plus de possibilités de lecture que l'exemple avec ellipse en \REF{ex:4:178}b, qui interdit l'interprétation stricte (angl. \textit{strict reading}) du premier des pronoms si le deuxième pronom a une interprétation relâchée (angl. \textit{sloppy reading}). 


\begin{enumerate}
\item \label{bkm:Ref288559762}a   Max said he gave his mother a bracelet, and Oscar said he gave his mother a watch. (stricte-stricte, relâché-relâché, stricte-relâché, relâché-stricte)


\end{enumerate}
  b  Max said he gave his mother a bracelet, and Oscar a watch. (stricte-stricte, relâché-relâché, *stricte-relâché, relâché-stricte)

\citet{Coppock2001} applique aux constructions à gapping la condition de légitimation de l'ellipse, formulée par \citet{Merchant2001}~pour les cas de Sluicing, c.-à-d. la condition de e-GIVENness. La phrase source $\alpha $ a deux constituants marqués F (focus) qui peuvent être remplacés par des variables liées comme en \REF{ex:4:179}a. La phrase trouée contient elle aussi deux constituants marqués F qui peuvent être remplacés par des variables liées comme en \REF{ex:4:179}b. La clotûre existentielle étant la même dans les deux phrases, elles s'impliquent mutuellement, ce qui fait que la séquence elliptique est donnée (e-GIVEN) et, par conséquent, l'effacement est légitimé. 


\begin{enumerate}
\item \label{bkm:Ref288560299}a   [$\alpha $ John\textsubscript{F} likes caviar\textsubscript{F}] and [$\gamma $ Mary\textsubscript{F} beans\textsubscript{F}].  


\end{enumerate}
  b  F-clo ($\alpha $) =  ${\exists x\text{.}\exists y}$(\textit{x} likes \textit{y})

  c  F-clo ($\gamma $) =  ${\exists x\text{.}\exists y}$(\textit{x} likes \textit{y})

Un premier avantage relevé par \citet{Coppock2001} réside dans le fait que cette condition rendrait immédiatement compte des effets prosodiques, comme le montrent les exemples suivants de \citet{Sag1980}, et respectivement \citet{Hankamer1973}. L'analyse de Johnson n'inclut pas les exemples avec {\guillemotleft}~trou à distance~{\guillemotright} \REF{ex:4:180}a, considérés agrammaticaux, ni l'exemple \REF{ex:4:181}a, considéré ambigu.


\begin{enumerate}
\item \label{bkm:Ref288561510}a   John\textsubscript{F} said he wants caviar\textsubscript{F} for dinner, and Mary\textsubscript{F} beans\textsubscript{F}.  


\end{enumerate}
  b  John said he\textsubscript{F} wants caviar\textsubscript{F} for dinner, and Mary\textsubscript{F} beans\textsubscript{F}.      (\citet{Sag1980})


\begin{enumerate}
\item \label{bkm:Ref288561460}a   Massachussets\textsubscript{F} elected McCormack\textsubscript{F} Congressman, and Pennsylvania\textsubscript{F} Schweicker\textsubscript{F}.  


\end{enumerate}
${\neq}$  b  Massachussets elected [Pennsylvania Schweicker]\textsubscript{F}.       (\citet{Hankamer1973})

Un deuxième avantage en lien avec la condition de \citet{Merchant2001} concerne les antécédents éparpillés (\textit{split antecedents}). Cette contrainte permet le manque d'isomorphisme structural entre la phrase source et la phrase trouée, ce qui rend possibles les cas de gapping où le trou a un antécédent éparpillé (\textit{contra} Hankamer \& \citet{Sag1976})\footnote{\citet[304]{Johnson2009} considère que les exemples en \REF{ex:4:182} sont agrammaticaux.}. Une analyse à base de mouvement du verbe (à la Johnson) échoue, car il n'y a pas de destination unique pour les verbes déplacés ATB, et de plus le trou n'est isomorphe à aucun des antécédents.


\begin{enumerate}
\item \label{bkm:Ref288640715}a   Wendy wants to sail around the world because she loves travel, and Bruce wants to climb Kilimanjaro in order to prove to himself that he can, but neither in order to show off for anyone.  


\end{enumerate}
  b  Fred bought Suzy flowers in order to thank her, and Bob took her out to eat because they both like sushi, but neither because they want to date her.

  c  John calls home on Sundays, and Jill balances her checkbook every other week, but neither very consistently.              (\citet{Coppock2001})

Finalement, \citet{Coppock2001} considère que l'effacement est préférable au mouvement du verbe, car il nous permet de rendre compte des différences observées dans les constructions à gapping par rapport aux contraintes d'îles. Contrairement aux premiers travaux (\citet{Neijt1979}, etc.), \citet{Coppock2001} considère que le gapping est sensible aux contraintes d'îles de manière sélective, ce qui justifierait la dichotomie proposée par \citet{Merchant2001} qui distingue les îles PF et les îles propositionnelles. Elle observe ainsi que le gapping permettrait l'extraction hors de certaines îles PF (p.ex. Contrainte sur la Branche Gauche), mais il obéirait à toutes les contraintes d'îles propositionnelles (contraintes sur les relatives, les sujets phrastiques, les ajouts, les phrases \textit{qu-})\footnote{Dans l'approche de \citet{Coppock2001}, cela est un argument pour le mouvement A' des éléments résiduels.}. Si le gapping est un effacement, on prédirait ces effets, qui sont d'ailleurs observés avec d'autres types d'ellipse aussi. Cependant, cet argument ne peut être tenu, comme on le verra dans la section \ref{sec:4.4.3.1}.


\begin{enumerate}
\item a   I make too strong an espresso, and Fred (*makes) too weak.  


\end{enumerate}
  b  Mary wrote too long a paper, and Suzy (*wrote) too short.     (\citet{Coppock2001})


\begin{enumerate}
\item a   *Suzy doesn't like men who play instruments, and Mary, sports.  


\end{enumerate}
  b  *That John hangs out with Mary is bothersome to Suzy, and Suzy, to Laura.

  c  *John must be a fool to have married Jane, and Bill, Martha

  d  *John wondered what to cook today, and Peter tomorrow.     (\citet{Coppock2001})

\paragraph[Reconstruction au niveau LF]{Reconstruction au niveau LF}
Dans ces approches, le matériel manquant est un élément vide (généré dès le départ). Comme il n'y a pas de matériel lexical, la structure syntaxique n'est pas prononcée. Deux sous-types d'approches peuvent être distingués :

(i) Le trou est une expression anaphorique qui est nulle phonologiquement. Cette proforme nulle est interprétée comme un pronom ordinaire par des moyens purement sémantiques, au niveau LF (\citet{Wasow1972}, Williams (1977, 1997), Zribi-\citet{Hertz1986}, etc.). Le gapping est conçu ainsi comme un mécanisme interprétatif établissant une relation anaphorique entre un élément vide et un antécédent identifié en forme logique. Le trou n'est pas structuré syntaxiquement, c.-à-d. il n'a pas une structure interne, par conséquent, aucune opération syntaxique n'est susceptible de l'affecter. 


\begin{enumerate}
\item Marlon boit\textsubscript{i} du rhum et Raquel {\O}\textsubscript{i} du whisky.      (Zribi-\citet[398]{Hertz1986}) 


\end{enumerate}
(ii) Le trou est une copie de son antécédent au niveau LF. Une fois que l'épel du premier conjoint a eu lieu, l'antécédent dans la phrase source est copié dans le site de l'ellipse de la phrase trouée au niveau LF, ce qui apporte aux éléments vides la bonne interprétation (Abe \& Hoshi (1997, 1999), \citet{Repp2009}). Les travaux de ce type assument généralement le déplacement des éléments résiduels.\footnote{Comme pour les approches à effacement, le déplacement des éléments résiduels (en anglais) a lieu soit dans une seule direction, à gauche (\citet{Repp2009}), soit dans les deux sens, c.-à-d. un élément résiduel à gauche et l'autre à droite (Abe \& \citet{Hoshi1999}).} Selon \citet{Repp2009}, on copie uniquement le matériel nécessaire pour construire une dérivation convergente à partir de la numération appauvrie de la phrase trouée. Cela veut dire qu'on ne copie pas les ajouts.


\begin{enumerate}
\item [\textsubscript{IP} John [\textsubscript{I'} [\textsubscript{I'} talked \textit{t}\textsubscript{1}] about Bill\textsubscript{1}]] and [\textsubscript{IP} Mary [\textsubscript{I'} [\textsubscript{I'} \textit{e} ] about Susan]].  


\end{enumerate}
{\raggedleft
(Abe \& \citet{Hoshi1999})
}

La principale motivation pour postuler une approche en termes d'élément vide~est fondée sur l'observation que le site de l'ellipse se comporte sous certains aspects comme un pronom ordinaire. Ainsi, dans les constructions à gapping le matériel manquant dans la phrase trouée pourrait avoir un antécédent éparpillé (\textit{split antecedent}), comme un pronom ordinaire. Dans ce sens, voir les données de \citet{Coppock2001} en \REF{ex:4:182}. Une deuxième ressemblance avec les pronoms ordinaires consisterait dans le fait que le matériel manquant pourrait correspondre à un antécédent non-linguistique, comme c'est le cas des pronoms déictiques. Si cela semble évident pour VPE, cf. \citet{Lobeck1995}, pour le gapping c'est discutable (Hankamer \& \citet{Sag1976}).

Néanmoins, cette approche est problématique au moins de deux points de vue. Il faut préciser que le matériel manquant dans le gapping a des propriétés qui le distinguent clairement d'une expression pronominale. Il ne peut pas avoir un emploi cataphorique sous enchâssement~\REF{ex:4:187}. En plus, il ne peut pas avoir accès à~l'antécédent d'une phrase source qui ne précède pas immédiatement la phrase trouée \REF{ex:4:188}, cf. \citet{Kehler2002}.


\begin{enumerate}
\item \label{bkm:Ref289775695}*If George the newspaper reporters, Al will make a statement blasting the press. 


\end{enumerate}
{\raggedleft
(\citet[91]{Kehler2002})
}


\begin{enumerate}
\item \label{bkm:Ref289775778}A :   George made a statement blasting the press. He's going to pay a big price for that.  


\end{enumerate}
  B :  \#And Al the newspaper reporters. In his case the fallout will be minimal, however.

{\raggedleft
 (\citet[91]{Kehler2002})
}

\paragraph[Mouvement (parallèle ou latéral) du verbe]{Mouvement (parallèle ou latéral) du verbe}
\label{bkm:Ref289726933}A côté de l'effacement au niveau PF et de la reconstruction au niveau LF, on trouve une troisième approche syntaxique, fondée sur deux points essentiels : d'abord, on considère que la coordination se place non au niveau de la phrase, mais au niveau du \textit{v}P (voir les arguments donnés dans la section \ref{sec:4.4.1.2}) ; ensuite, le trou est analysé comme une trace du mouvement.

Ces approches se divisent en deux sous-types, en fonction du type de mouvement envisagé pour le verbe dans le gapping :

(i) mouvement parallèle (\textit{Across-The-Board Movement}) : le matériel {\guillemotleft}~partagé~{\guillemotright} par les deux conjoints est extrait simultanément des deux conjoints. Ainsi, le verbe du chaque conjoint se déplace dans le Spec,PredP/TP/IP (après avoir déplacé les éléments résiduels contrastifs), cf. Johnson (1996/2004, 2000, 2009), Zoerner \& \citet{Agbayani2000}. Les étapes envisagées~sont : d'abord, l'extraction des éléments résiduels et corrélats vers une position A', à la périphérie gauche du \textit{v}P en question ; ensuite, l'extraction parallèle des deux \textit{v}Ps coordonnés vers la position Spec,PredP, et finalement, le déplacement du sujet du premier conjoint vers Spec,TP. On observe ainsi que ce type d'analyse ne voit pas un phénomène d'ellipse dans les constructions à gapping (voir, dans ce sens, les différences que Johnson établit entre le gapping, d'une part, et l'ellipse du VP et le pseudogapping, d'une autre part).


\begin{enumerate}
\item [\textsubscript{TP} Randy\textsubscript{j} [\textsubscript{T'} drank\textsubscript{i} [\textsubscript{vP} \textit{t}\textsubscript{j}\textit{ t}\textsubscript{i} scotch] and [\textsubscript{vP} Amy \textit{t}\textsubscript{i} rum]]].      (\citet{Johnson2009})~  


\end{enumerate}
(ii) mouvement latéral (\textit{sideward movement}) du \textit{v}P libéré de ses éléments résiduels, qui ont été extraits en préalable à la périphérie gauche du \textit{v}P, cf. López \& \citet{Winkler2003}, Agbayani \& \citet{Zoerner2004}, \citet{Winkler2005}\footnote{López \& \citet{Winkler2003} et \citet{Winkler2005} font plutôt appel à une analyse mixte : mouvement latéral du verbe, ensuite effacement du VP libéré, dans une approche dérivationnelle par phases. Il s'agit d'un phénomène de type minimaliste copier-et-fusionner qui copie des constituants et les fusionne avec des objets syntaxiques qui ne sont pas reliés et qui sont indépendamment fusionnés.}. Les étapes envisagées sont les suivantes. Les éléments résiduels se déplacent vers une position A', à la périphérie gauche du \textit{v}P ; le \textit{v}P qui contient les traces des éléments résiduels est copié, déplacé latéralement vers l'autre \textit{v}P avec lequel il est coordonné, et il fusionne avec les éléments corrélats dans le premier conjoint ; le sujet du premier conjoint se déplace vers la position Spec,TP (pour ainsi satisfaire le Principe de la Projection Etendue) ; le \textit{v}P du premier conjoint est préposé pour des raisons liées à l'ordre des mots ; les deux conjoints fusionnent, et enfin les copies \textit{v}Ps qui sont restées en bas sont effacées. 

Quand je discute l'analyse du mouvement du verbe pour le gapping, je prends en compte essentiellement l'analyse de Johnson. Pour plus d'informations sur les différences entre les deux types de mouvement, voir \citet{Winkler2005}.

En dehors des aspects qu'on a vus dans la section \ref{sec:4.4.1.2}, militant pour une coordination de \textit{v}P, on peut ajouter d'autres éléments qui justifient le mouvement cette fois-ci. Un des avantages majeurs de l'analyse de Johnson, selon \citet{Vicente2010}, réside dans l'explication élégante qu'elle donne pour deux restrictions qu'on observe avec le gapping (mais pas avec l'ellipse du verbe dans VPE) : le fait que le gapping apparaît uniquement avec la coordination et qu'il ne peut être enchâssé.  

Selon Zoerner \& \citet{Agbayani2000}, le mouvement du verbe rend mieux compte des discordances qu'on pourrait trouver entre un trou et son antécédent, en ce qui concerne l'accord par exemple \REF{ex:4:190}. Le verbe {\guillemotleft}~monté~{\guillemotright} s'accorde avec le sujet de la phrase source.  


\begin{enumerate}
\item \label{bkm:Ref306281113}The president approves the education bill, and the senators approve the health bill.~ 


\end{enumerate}
\citet{Repp2009}, dans la synthèse qu'elle fait sur les approches du gapping, présente quelques aspects qui semblent être prédits par le mouvement du verbe. D'abord, ce type d'analyses fait une prédiction sur les langues qui ont ou qui n'ont pas de gapping : les langues qui ne présentent pas le mouvement du verbe, ne peuvent pas avoir de gapping (p.ex. chinois)\footnote{Néanmoins, il faut faire attention à ce type de généralisations, car il semble que le gapping ne soit pas complètement impossible en chinois (cf. \citet{Paul1999}, Ruixi \citet{Ressy2008}).}. Une autre prédiction concerne le fait que le gapping semble être plus contraint dans les langues qui permettent le scrambling que dans les langues qui ne l'ont pas (comparer l'allemand et l'anglais). Selon López \& \citet{Winkler2003}, le mouvement du complexe verbal vers une position flexionnelle rend compte aussi de la directionnalité du gapping dans les langues à tête finale (catalepse en japonais et coréen).  

\subsubsection{Limites des analyses structurales}
\label{bkm:Ref290575994}Le but de cette section est de montrer qu'aucune des analyses structurales majeures, p.ex. effacement au niveau PF (à la \citet{Coppock2001}, ou bien mouvement du matériel manquant, à la Johnson (1996/2004, 2009), n'offre une solution adéquate pour les constructions à gapping.

\paragraph[Problème de l'extraction des éléments résiduels ]{Problème de l'extraction des éléments résiduels} 
\label{bkm:Ref289439559}Toutes les propositions récentes dans les deux types d'approches majeures (effacement au PF~/ reconstruction au LF ou mouvement du matériel manquant) font appel à une opération de déplacement des éléments résiduels (et éventuellement des éléments corrélats dans la phrase source) vers des projections fonctionnelles périphériques. Les motivations d'une telle opération sont plutôt internes aux cadres théoriques respectifs. D'abord, on doit dire que les premiers travaux proposant la reconstruction syntaxique rencontrent des difficultés majeures quant à la notion de constituant : dans les constructions à gapping, on efface un élément ou une série d'éléments qui ne forme pas un syntagme. Pour éviter ce problème de la non-constituance, les grammaires dérivationnelles proposent l'opération de mouvement : on déplace des éléments, afin d'obtenir dans le site de l'ellipse un syntagme. Deuxièmement, dans un cadre théorique basé sur l'idée d'un homomorphisme entre la syntaxe et les autres niveaux linguistiques, postuler une opération de déplacement faciliterait la modélisation de la structure informationnelle : si les éléments résiduels sont des topiques ou des focus contrastifs, ils doivent avoir accès à une position syntaxique spécifique, qui permet ensuite l'effacement (ou la légitimation d'une structure vide) de tout matériel qui est donné (\textit{given}) dans le discours. Finalement, cette opération permet aux cadres théoriques en question d'obtenir un mécanisme uniforme pour d'autres types d'ellipse aussi (p.ex. VPE, sluicing, etc.).

Si l'on suppose un déplacement des éléments résiduels en dehors du site de l'ellipse, on s'attend à ce que cette opération obéisse aux contraintes qui pèsent sur toute extraction, en particulier aux \textit{contraintes de localité}, connues aussi sous le nom d'\textit{effets d'îles} (pour une discussion de ces îlots en lien avec l'ellipse, voir, entre autres, \citet{Ross1967}, Merchant (2001, 2004)). A première vue, la prédiction semble être valable : on affirme souvent que le gapping est une opération locale, qui est contrainte par les {\guillemotleft}~barrières syntaxiques~{\guillemotright} (\citet{Hankamer1973}, \citet{Neijt1979}, \citet{Chao1988}, Johnson (1996/2004), \citet{Coppock2001}, \citet{Winkler2005}). Ceux qui ont une perspective holistique des îles considèrent que le gapping obéit à toutes les contraintes d'îles observées (cf. \citet{Ross1967}, \citet{Neijt1979}), tandis que d'autres (plus récemment) observent que le gapping obéit au moins à certaines contraintes d'îles, et en particulier aux îles {\guillemotleft}~propositionnelles~{\guillemotright}, c.-à-d. îles qui contiennent un domaine propositionnel enchâssé\footnote{\citet{Merchant2001} considère que les îles ne se comportent pas de la même façon par rapport aux ellipses, elles ne sont donc pas homogènes, ce qui l'amène à distinguer entre deux classes : les îles {\guillemotleft}~PF~{\guillemotright} vs. les îles {\guillemotleft}~propositionnelles~{\guillemotright}. Si certaines ellipses violent les contraintes d'îles, il s'agirait toujours des îles {\guillemotleft}~PF~{\guillemotright}. }  (cf. \citet{Hartmann2000}, \citet{Coppock2001}). D'après la typologie de \citet{Merchant2001}, les îles {\guillemotleft}~propositionnelles~{\guillemotright} qui m'intéressent ici sont : l'îlot sujet, l'îlot relatif et l'îlot circonstanciel. Si l'on envisage le déplacement des éléments résiduels à l'extérieur du site elliptique, on s'attend donc à ce que le gapping obéisse au moins à ces îles {\guillemotleft}~propositionnelles~{\guillemotright}, ce qui implique qu'on ne peut avoir comme élément résiduel un constituant enchâssé dans une de ces îles. Selon \citet{Hartmann2000}, \citet{Coppock2001} et en partie \citet{Repp2009}\footnote{Selon \citet{Repp2009}, le gapping ne respecte pas les contraintes de l'îlot relatif et de la Branche Gauche, en revanche il respecte les contraintes de l'îlot sujet phrastique et de l'îlot ajout. Elle finit par dire qu'il faudrait une analyse détaillée du comportement du gapping par rapport aux différents types d'îles.}, cette prédiction est valable dans les constructions à gapping : on ne peut pas extraire un élément résiduel hors d'une phrase sujet \REF{ex:4:191}a, hors d'une phrase relative \REF{ex:4:191}b ou hors d'un circonstanciel \REF{ex:4:191}c.


\begin{enumerate}
\item \label{bkm:Ref289714791}a   *That John hangs out with Mary annoys Suzy, and Suzy Laura.  (\citet{Coppock2001})


\end{enumerate}
  b  *Some wanted to hire the woman who worked on Greek, and others Albanian. 

{\raggedleft
 (\citet{Merchant2009})
}

  c  *John must be a fool to have married Jane, and Bill, Martha.    (\citet{Coppock2001})

Cependant, on observe qu'un élément résiduel peut apparaître dans ce qui devrait être un îlot d'extraction en anglais, cf. Culicover \& \citet{Jackendoff2005}. Ainsi, en \REF{ex:4:192}a et \REF{ex:4:192}c, on a une violation de la contrainte de l'îlot relatif, alors qu'en \REF{ex:4:192}b, on viole la contrainte de l'îlot ajout.


\begin{enumerate}
\item \label{bkm:Ref289716674}a   Robin knows a lot reasons why dogs are good pets, and Leslie, cats.  


\end{enumerate}
  b  Robin believes that everyone pays attention to you when you speak French, and Leslie, German.             (Culicover \& \citet{Jackendoff2005})

  c  Bo decided who is working tomorrow, and Mia, the next day.      (\citet{Chaves2005})

L'élément résiduel peut violer les contraintes d'îles en roumain aussi. Je mentionne quelques exemples dans lesquels un des éléments résiduels peut être le dépendant d'un verbe enchâssé dans un îlot sujet (au subjonctif ou à l'infinitif) en \REF{ex:4:193}, dans un îlot relatif en \REF{ex:4:194} ou dans un îlot circonstanciel en \REF{ex:4:195}.


\begin{enumerate}
\item \label{bkm:Ref289719791}a  Să înveți la pian e greu, dar la vioară şi mai greu.


\end{enumerate}
apprendre.\textsc{subj prep} piano est difficile, mais \textsc{prep} violon \textsc{adv} plus difficile

{\itshape
Apprendre le piano est difficile, mais le violon encore plus difficile}

  b  Mariei îi place să meargă la mare, iar lui Ion la munte.

    Maria.\textsc{dat cl} plaît \textsc{mrq} aller.\textsc{subj} à mer, et Ion\textsc{.dat} à montagne

{\itshape
Marie aime aller à la mer, et Ion à la montagne}

  c  \%A merge la teatru e pasiunea Mariei, iar la film pasiunea lui Ion.

{\itshape
Aller au théâtre est la passion de Maria, et au cinéma la passion de Ion}


\begin{enumerate}
\item \label{bkm:Ref289719799}a  \textstyleapplestylespan{Sunt oameni} \textstyleapplestylespan{care preferă}\textstyleapplestylespan{ singurătatea, iar}{~}\emph{\textup{alții,}}\emph{\textbf{\textup{} }}\emph{\textup{contrariul}}.


\end{enumerate}
{\itshape
Il y a des gens qui préfèrent la solitude, et d'autres, le contraire}

  b  Cel din stânga mea e premierul care e prevăzut pentru criză, iar cel din dreapta pentru haos. 

{\itshape
Celui à ma gauche est le premier qui est prévu pour la crise, et celui à ma droite pour le chaos}


\begin{enumerate}
\item \label{bkm:Ref289719802}a  (In timpul zilei, Ion şi Maria nu se întâlnesc decât la masă.) El mănâncă răsfoind ziarul, iar ea revistele de cosmetică. (Iți dai seama ce mult comunică cei doi parteneri pe parcursul unei zile.)


\end{enumerate}
{\itshape
(Pendant la journée, Ion et Maria ne se voient qu'au déjeuner.) Il mange en feuilletant le journal, et elle, les revues de cosmétiques. (Tu peux t'imaginer combien ils communiquent ces deux partenaires sur une journée) } 

  b  Ion mănâncă uitându-se la documentare, iar Maria la telenovele.

{\itshape
Ion mange en regardant des documentaires, et Maria des feuilletons}

  c  Ion se chinuie încercând să învețe chineza, iar Maria japoneza.

{\itshape
Ion se fait du mal en essayant d'apprendre le chinois, et Maria le japonais}

  d  Ion merge la Paris să vadă Turnul Eiffel, iar Maria, Catedrala Nôtre-Dame.

{\itshape
Ion va à Paris pour voir la Tour Eiffel, et Maria, la Cathédrale Nôtre-Dame}

Les mêmes observations s'appliquent au français, comme on peut voir dans les exemples en \REF{ex:4:196}, tirés de Abeillé, Bîlbîie \& Mouret \textit{à paraître}.


\begin{enumerate}
\item \label{bkm:Ref289720105}a   Comprendre le texte traduit est laborieux et le texte original encore plus laborieux.  


\end{enumerate}
  b  Quand tu parles chinois, tout le monde t'admire, mais anglais personne.

  c  Jean a plein de bonnes raisons pour avoir un chat et Marie un chien.

  d  C'est Paul qui fait la vaisselle et Marie la lessive.

  e  Paul connaît quelqu'un qui a vu un cobra et Marie un ours polaire.

Toutes ces données montrent qu'il y a des îles qui sont violées dans le gapping, et cela indépendamment de la classification proposée par \citet{Merchant2001}. Ainsi, l'extraction des éléments résiduels à la périphérie gauche du conjoint ne se justifie pas empiriquement. De plus, \citet[463]{Culicover2009} ajoute le fait que cette extraction {\guillemotleft}~multiple~{\guillemotright} n'est pas motivée en dehors des constructions elliptiques, car l'anglais ne permet pas la topicalisation multiple. Par conséquent, l'effacement ou toute autre opération supposée affecte un élément ou une suite d'éléments qui ne forment pas un constituant.

Cette discussion sur les effets d'îles dans les structures elliptiques nécessiterait une étude approfondie concernant la nature exacte de ces contraintes, ce qui dépasse largement mon objet d'étude. La motivation pour une telle étude vient des différences qu'on observe quant à l'acceptabilité des violations de ces contraintes : l'extraction hors d'une même île est considérée possible dans certains exemples, mais inacceptable dans d'autres occurrences. Les grammaires dérivationnelles (dans lesquelles s'inscrivent les analyses dominantes proposées pour le gapping) proposent une approche syntaxique des îles, les effets observés étant due à des contraintes de compétence (cf. \citet{Ross1967}). Mais ce type d'approche ne peut pas expliquer les différences d'acceptabilité qu'on observe avec une même île. 

La conclusion de plusieurs travaux (\citet{Hankamer1973}, \citet{Kuno1976}, \citet{Sag1976}, Sag \textit{et al.} (1985), \citet{Gardent1991}, etc.) est que beaucoup des contraintes qui jouent sur l'interprétation des exemples à gapping sont de nature non-syntaxique. Ainsi, le fait que certaines îles soient respectées pourrait ne pas être une question de grammaticalité dans ce type spécifique de constructions, mais plutôt une question d'accessibilité~(c.-à-d. l'extraction hors de ces îles n'est pas agrammaticale, mais simplement elle n'est pas préférée pour d'autres raisons) ; la sensibilité aux îles devrait donc être expliquée en termes de facteurs psycholinguistiques. Notons que, indépendamment de l'étude de l'ellipse, des travaux ont montré que, parmi les facteurs non-syntaxiques qui gèrent l'acceptabilité des exemples avec îles, un rôle très important revient aux facteurs psycholinguistiques (voir Fanselow \& \citet{Frisch2006} : {\guillemotleft}~Processing difficulty can make grammatical sentences unacceptable~{\guillemotright}), mais aussi à d'autres types de facteurs (discursifs, prosodiques, etc.). Pour une approche non-syntaxique des contraintes d'îles, voir \citet{Kluender1998}, Fanselow \& \citet{Frisch2006}, Ambridge \& \citet{Goldberg2008}, Hofmeister \& \citet{Sag2010}.  

\paragraph[Problèmes de l'extraction du matériel manquant]{Problèmes de l'extraction du matériel manquant}
On a vu dans les sections 4.4.1.2 et 4.4.2.3 que la principale motivation pour une approche à la Johnson (1996/2004, 2000, 2009) est d'ordre sémantique. Une analyse qui place la coordination à un niveau sous-phrastique d'où on a extrait le matériel manquant permettrait de rendre compte du fait que certains opérateurs (c.-à-d. la négation, les modaux, certains quantifieurs) dans la phrase source peuvent avoir portée large sur toute la coordination. 

Cependant, ce type d'analyse présente plusieurs difficultés, que j'ai regroupées en deux parties : des problèmes plutôt internes à ce type d'approche, et ensuite des problèmes empiriques, qui sont indépendants de tout cadre théorique. 

Selon \citet{Vicente2010},~le mouvement du verbe est problématique dans le cas des trous complexes, où on devrait déplacer non seulement la tête verbale, mais aussi d'autres constituants. Johnson propose une solution à ce problème (c.-à-d. \textit{remnant predicate movement}), mais elle n'est pas empiriquement adéquate, car on ne peut pas l'appliquer à d'autres types de déplacement enregistrés en anglais. Ainsi, les trous complexes, qui subissent un déplacement à la Johnson, ne peuvent pas être antéposés :


\begin{enumerate}
\item a   Phil read things quickly, and Mike thoroughly.  


\end{enumerate}
  b  *Read things, Mike (did) quickly.     (\citet[510]{Vicente2010})


\begin{enumerate}
\item a   Ready wants to write a novel, and Amy a play.  


\end{enumerate}
  b  *Want to write, Randy (did) a novel.     (\citet[510]{Vicente2010})

Comme \citet{Johnson2009} le précise lui-même, le mouvement des trous complexes nécessite plusieurs déplacements, qui intéragissent avec la linéarisation des éléments d'une manière inattendue : ces opérations permettent des ordres de mots qui ne sont pas attestés dans~la langue.\footnote{\citet{Johnson2009} considère que c'est un problème aussi pour les approches à la \citet{Coppock2001}, qui utilisent l'effacement.}


\begin{enumerate}
\item a   Ice cream gives me brain-freeze if I eat it too fast and beans give me indigestion if I eat them too slow.  


\end{enumerate}
  b  *Ice cream gives me in the morning brain-freeze.  (\citet[314]{Johnson2009})

L'extraction verbale exige que les deux \textit{v}Ps soient identiques. La négation peut être interprétée soit à un niveau supérieur par rapport à la coordination (et on obtient ainsi la portée large), soit à l'intérieur des deux conjoints (et on obtient ainsi la distribution de la négation sur les deux conjoints). Mais on n'arrive toujours pas à obtenir la portée étroite de la négation uniquement dans le premier conjoint (cf. \citet{Repp2009}). 

Ce type d'approche ne se justifie pas empiriquement en roumain. Un premier point faible de cette analyse est le fait qu'elle prédit de manière incorrecte la distribution des items corrélatifs. En roumain, si une coordination de phrases présente des items corrélatifs (conjonctions, p.ex. \textit{fie...fie} `soit...soit' ou adverbes, p.ex. \textit{nici...nici} `ni...ni'), chaque conjoint doit être introduit par un corrélatif, cf. \REF{ex:4:200}a et \REF{ex:4:201}a. Dans l'approche de Johnson, le premier élément corrélat dans la phrase source (habituellement, le sujet) est extrait de manière asymétrique en dehors du premier conjoint. Cela devrait permettre l'occurrence d'un item corrélatif après la tête verbale dans la phrase source, ce qui s'avère être agrammatical en roumain en \REF{ex:4:200}b et \REF{ex:4:201}b.  


\begin{enumerate}
\item \label{bkm:Ref299719793}a  \textbf{Fie} Dan va cânta la vioară, \textbf{fie} Maria (va cânta) la pian.


\end{enumerate}
{\itshape
Soit Dan va jouer du violon, soit Maria (va jouer) du piano  } 

  b  *Dan va cânta \textbf{fie} la vioară, \textbf{fie} Maria la pian.

{\itshape
Dan va jouer soit du violon, soit Maria du piano}


\begin{enumerate}
\item \label{bkm:Ref299719836}a  \textbf{Nici} directorul nu are obligații față de mine, şi \textbf{nici} eu (n-am obligații) față de el.


\end{enumerate}
{\itshape
Ni le directeur n'a d'obligations envers moi, ni moi envers lui } 

  b  *Directorul nu are obligații \textbf{nici} față de mine, şi \textbf{nici} eu față de el.

{\itshape
Le directeur n'a d'obligations ni envers moi, ni moi envers lui}

De plus, cette analyse prédit incorrectement que la conjonction \textit{iar} (utilisée massivement dans les constructions à gapping) relie des éléments sous-phrastiques, alors qu'il est communément admis que cette conjonction coordonne uniquement des contenus propositionnels. 

Par conséquent, ce type d'approche ne peut pas s'appliquer au roumain. Quant à la motivation d'une coordination {\guillemotleft}~basse~{\guillemotright} (c.-à-d. portée large des opérateurs sémantiques), je précise que les problèmes relevés par Johnson sont de nature sémantique et non syntaxique et peuvent trouver une solution convenable dans un cadre théorique qui ne pose pas d'homomorphisme syntaxe-sémantique.\footnote{La solution se baserait sur l'asymétrie qu'on observe entre la coordination ordinaire (sans ellipse) et la coordination à gapping. La coordination ordinaire n'autorise pas à un élément issu d'un des conjoints d'avoir portée large sur toute la coordination, les contraintes de portée sont donc très strictes. En revanche, dans les coordinations à gapping, on peut relâcher ces contraintes et autoriser donc un élément du premier conjoint à prendre portée large, à condition que cela fasse sens sémantiquement: p.ex. un quantifieur qui lie une variable dans le deuxième conjoint ou bien un ajout adverbial comme la négation ou d'autres adverbes. Cela est possible dans un cadre comme le HPSG, qui peut utiliser le langage \textit{Minimal Recursion Semantics} pour la sous-spécification syntaxique de la portée des quantifieurs. }  

\paragraph[Problèmes de l'effacement]{Problèmes de l'effacement}
Dans les analyses qui postulent une reconstruction syntaxique du verbe à l'endroit même du trou, l'effacement ou autre opération envisagée a lieu sous une condition d'identité entre le matériel manquant et~le matériel antécédent. Par conséquent, toute discordance qui apparaît entre les deux entraîne des difficultés supplémentaires qui obligent la théorie en question à faire appel à des stipulations parfois coûteuses.

Dans les constructions à gapping, on observe que le matériel manquant qui doit être reconstruit dans la phrase trouée ne correspond pas toujours à une copie du matériel antécédent dans la phrase source. La condition d'identité doit donc être remaniée afin de prendre en compte les différentes asymétries qu'on observe.

Un fait bien connu et discuté pour l'anglais aussi est l'absence d'identité stricte entre les deux verbes par rapport aux marques d'accord (différence en nombre ou/et en personne). Ces données peuvent trouver néanmoins une solution dans les versions récentes de l'effacement (voir Beavers \& \citet{Sag2004}, Chaves \& \citet{Sag2008}). 


\begin{enumerate}
\item a  Noi citim o carte, iar tu (citeşti) un ziar.


\end{enumerate}
    nous lisons un livre, et tu (lis) un journal

{\itshape
Nous lisons un livre, et toi un journal}

  b  Eu iubesc animalele, iar Ioana (iubeşte) florile.

    \textsc{1sg.nom} aime les animaux, et Ioana (aime) les fleurs

{\itshape
J'aime les animaux et Ioana les fleurs}

Pour le roumain (et le français), le manque d'identité est encore plus aigü quand on prend en compte le comportement des clitiques. Les clitiques pronominaux (ou adverbiaux) affixés au verbe antécédent ne sont pas nécessairement les mêmes que ceux qui seraient affixés au verbe reconstruit dans la phrase trouée \REF{ex:4:203}. De plus, certains clitiques sont interdits dans l'un des conjoints, mais obligatoires dans l'autre \REF{ex:4:204}. 


\begin{enumerate}
\item \label{bkm:Ref289809725}a  Ion \textbf{l}-a văzut pe Dan, iar Ana (a văzut-\textbf{o}) pe Maria.


\end{enumerate}
Ion \textsc{cl.masc-}a vu \textsc{mrq} Dan, et Ana (a vu-\textsc{cl.fem) mrq} Maria

{\itshape
Ion a vu Dan, et Ana Maria}

  b  Eu \textbf{i}-am văzut pe [Ion şi Maria], iar Ana (\textbf{l}-a văzut) pe Paul.

    je \textsc{cl.pl-}ai vu \textsc{mrq} Ion et Maria, et Ana (\textsc{cl.sg}-a vu\textsc{) mrq} Paul

{\itshape
J'ai vu Ion et Maria, et Ana Paul}


\begin{enumerate}
\item \label{bkm:Ref289809793}a  Maria \textbf{le}-a citit pe toate, dar Ana ((*\textbf{le}-)a citit) doar câteva.


\end{enumerate}
Maria \textsc{cl.acc-}a lu \textsc{mrq} toutes, mais Ana (\textsc{cl.acc-}a lu) seulement quelques-unes

{\itshape
Maria les a tous lus, mais Ana seulement quelques-uns}

  b  Ion a citit câteva dintre ele, dar Maria (*(\textbf{nu}) a citit) absolut niciuna.

    Ion a lu quelques-unes parmi elles, mais Maria (\textsc{neg} a lu) absolument aucune

{\itshape
Ion en a lu quelques-uns, mais Maria absolument aucun} 

Les affixes ont le même comportement en français. J'insiste sur l'exemple en \REF{ex:4:206}c, où on observe que l'élément résiduel \textit{moi} ne peut pas fonctionner comme sujet d'un verbe reconstruit ; si l'on reconstruit un verbe, il doit avoir en position sujet le clitique pronominal \textit{je}.


\begin{enumerate}
\item a   Paul \textbf{en} a lu seulement certains, mais Marie (\textbf{les} a) presque tous (lus).      


\end{enumerate}
b  Paul \textbf{les} a lus, vos livres, et Marie (\textbf{en} a lu) seulement certains.      


\begin{enumerate}
\item \label{bkm:Ref289810105}a   Paul a lu tous vos livres et Marie (\textbf{en} a lu) quelques-uns.      


\end{enumerate}
  b  Paul en a lu certains, et Marie (*(\textbf{n}')en a lu) absolument aucun.

  c  Marie aime les pommes et moi (*(\textbf{j}')aime) les oranges

On observe le même problème avec les adverbes restrictifs \textit{decât} `que' et \textit{doar} `seulement' en \REF{ex:4:207}. Pour marquer la restriction, le roumain utilise l'adverbe \textit{decât} dans les contextes négatifs et \textit{doar} dans les contextes positifs. \textit{Decât} est licite uniquement s'il suit un verbe nié. Si l'on assume une théorie à base d'effacement, on devrait pouvoir utiliser \textit{decât} dans la phrase trouée aussi, car son légitimeur (c.-à-d. le verbe nié) serait présent dans la structure syntaxique. Or, les locuteurs ont une préférence pour l'emploi de \textit{doar} (qui ne demande pas de négation et pas de verbe non plus) dans la phrase avec gapping (comparer \REF{ex:4:207}a-b). 


\begin{enumerate}
\item \label{bkm:Ref289810324}a  ??Ion nu ştie \textbf{decât} engleza, iar Maria \textbf{decât} germana.


\end{enumerate}
{\itshape
Ion ne sait que l'anglais, et Maria que l'allemand}

  b  Ion nu ştie \textbf{decât} engleza, iar Maria \textbf{doar} germana.

{\itshape
Ion ne sait que l'anglais, et Maria seulement l'allemand} 

Si ces problèmes de discordance peuvent trouver une solution adéquate (bien que coûteuse), il y a un autre fait qui, à ma connaissance, ne peut pas être pris en compte par une approche syntaxique à base d'effacement. Selon le principe de récupérabilité de l'ellipse (cf. \citet{Chomsky1964}), une forme linguistique non-réalisée phonologiquement doit pouvoir être insérée in situ. Or, dans les constructions à gapping la reconstruction du matériel supposé effacé ne donne pas toujours lieu à une phrase grammaticale. Comme l'ont observé Culicover \& \citet{Jackendoff2005}, il y a des éléments qui apparaissent dans une phrase trouée, mais qui ne se combinent pas avec une phrase finie, ce qui est attendu si le verbe manquant est effectivement absent de la structure. Par exemple, si la phrase trouée est introduite par la conjonction lexicalisée \textit{precum şi} `ainsi que' ou si la conjonction \textit{şi} `and' est immédiatement suivie par la négation de constituant \textit{nu}\footnote{On doit distinguer entre trois \textit{nu} différents en roumain : (i) l'adverbe négation de constituant, (ii) le clitique adverbial négation de phrase, qui est affixé au verbe, et (iii) l'adverbe pro-phrase. Pour une discussion sur leurs propriétés différentes, voir \citet{Barbu2004}, \citet{Ionescu2003}, etc. 
(i)  Lupul îşi schimbă părul, dar \textbf{nu} năravul.
  \textit{Le loup change son pelage, mais non son instinct}
(ii)  Lupul îşi schimbă părul, dar \textbf{nu}-şi schimbă năravul.
  \textit{Le loup change son pelage, mais ne change pas son instinct}
(iii)  Lupul îşi schimbă părul, dar năravul \textbf{nu}.
  \textit{Le loup change son pelage, mais son instinct non}} `non, non pas', on ne peut pas avoir un verbe fini dans la phrase trouée. De même, pour les connecteurs comparatifs\textit{ ca şi,} \textit{la fel ca} `comme', qui n'introduisent jamais une phrase finie. 


\begin{enumerate}
\item a  Istoria veche a egiptenilor mă pasionează dintotdeauna, \textbf{precum şi} cea aztecă (*mă pasionează) de ceva vreme.


\end{enumerate}
{\itshape
L'histoire ancienne des Egyptiens me passionne depuis toujours, ainsi que celle des Aztèques depuis un bon moment } 

  b  DAN va dormi la Maria \textbf{şi nu} ea (*va dormi) la el.

{\itshape
Dan va dormir chez Maria, et non elle chez lui}

  c  Ion se comportă cu Maria \{\textbf{ca şi / la fel ca}\} fratele lui (*se comportă) cu Ana.

{\itshape
Ion se comporte avec Maria comme son frère avec Ana}

Un problème interne aux théories qui présupposent l'homomorphisme syntaxe-sémantique concerne la portée de certains opérateurs sémantiques. Comme le note Johnson, la négation, les modaux ou encore certains quantifieurs peuvent avoir une portée large sur toute la coordination (voir les données en \REF{ex:4:165}-\REF{ex:4:167}). Or, dans une théorie à base d'effacement, ces opérateurs apparaissent dans chaque conjoint, donc la portée large ne peut pas avoir lieu. \citet{Coppock2001} propose une solution à ce problème : la coordination a lieu non au niveau de la phrase, mais au niveau du \textit{v}P. Cependant, les arguments empiriques mentionnés dans la section précédente invalident cette possibilité. Par conséquent, la portée~large de ces opérateurs reste un problème.

Un problème plus général, qui concerne et l'effacement et le mouvement du verbe, est lié à la portée de la négation dans le gapping. Les données sur la négation ont tourné d'une analyse à l'autre, chacune choisissant celles qui étaient en sa faveur : ainsi, l'effacement est adapté pour rendre compte de la portée distribuée de la négation, alors que le mouvement ATB est adapté pour la portée large. Cependant, et l'effacement et le mouvement ATB sont loin d'avoir un traitement complet des trois interprétations de la négation à l'intérieur d'une même théorie, en particulier elles ne captent pas la portée étroite de la négation, quand la polarité est différente dans les deux conjoints : négative vs. positive ou bien positive vs. négative (cf. \citet{Repp2009}).

Vu les problèmes empiriques mentionnés dans cette section, il s'avère très difficile de maintenir une approche en termes d'effacement ou de mouvement du verbe, combiné éventuellement avec une extraction des éléments résiduels. Je me propose d'abandonner ce type d'approches pour les constructions à gapping et je m'oriente vers une solution qui ne postule pas de structure syntaxique pour le matériel manquant. 

\subsubsection{Analyses alternatives : une approche non-structurale}
Les problèmes discutés dans la section précédente peuvent trouver une solution dans une approche à base de constructions, qui postule la récupération du matériel antécédent, sans qu'il soit présent (sous une forme ou autre) dans la structure syntaxique de la phrase trouée. Dans cette perspective, le trou n'a pas de représentation syntaxique (pas d'effacement, pas d'élément vide, pas de mouvement). Par conséquent, la séquence trouée sera une suite de deux ou plusieurs syntagmes sans tête verbale, avec un contenu propositionnel similaire à celui de la phrase source. On appelera cette séquence une \textit{phrase fragmentaire}.

Parmi les analyses proposées dans cette perspective, on peut citer les travaux de Sag \textit{et al.} (1985) en GPSG (\textit{Generalized Phrase Structure Grammar}), Steedman (1990, 2000), \citet{Gardent1991} et \citet{Hoyt2008} en CCG (\textit{Combinatory Categorial Grammar}), Culicover \& \citet{Jackendoff2005} et \citet{Culicover2009} dans leur projet d'une syntaxe plus simple (\textit{Simpler Syntax}).\footnote{A la liste des analyses proposées pour le gapping, je devrais ajouter aussi les analyses à base de {\guillemotleft}~partage~{\guillemotright} (\textit{sharing} \textit{approach}), cf. \citet{Goodall1987}, \citet{Moltmann1992}, etc. L'idée générale est que la phrase source et la phrase trouée sont projetées dans le même arbre (un seul n{\oe}ud S/IP). Dans cet arbre, le matériel qui apparaît uniquement dans la phrase source est partagé littéralement par les deux conjoints. En revanche, les paires contrastives ne sont pas partagées. On distingue trois versions : i) Les paires contrastives apparaissent sous le même n{\oe}ud comme une liste ordonnée (c.-à-d. théorie de la factorisation). ii) Les éléments résiduels apparaissent dans l'arbre comme s'ils créaient une structure bidimensionnelle. iii) Le matériel partagé est présent simultanément dans les deux conjoints, grâce à la possibilité d'avoir un n{\oe}ud dominé par plusieurs n{\oe}uds-mère (c.-à-d. théorie de la dominance multiple).} 

Par la suite, je présente brièvement les analyses de Steedman (1990, 2000), Sag \textit{et al.} (1985), \citet{Gardent1991} et Culicover \& \citet{Jackendoff2005}, vu le fait que l'analyse qu'on retiendra pour les constructions à gapping en roumain et en français (dans la section \ref{sec:4.5.1}) a beaucoup de points en commun avec ces approches. 

Les travaux sur l'ellipse faits en grammaire catégorielle (\citet{Dowty1988}, Steedman (1990, 2000)) ont l'avantage d'offrir une analyse assez aisée des coordinations de non-constituants, car ce type de grammaire permet une extension de la notion de constituance grâce à des règles combinatoires flexibles (p.ex. on peut combiner le verbe soit avec l'objet, soit avec le sujet). Le résultat est que, contrairement aux grammaires syntagmatiques, dans une grammaire catégorielle les règles combinatoires permettent à toute séquence de non-constituants de fonctionner comme un constituant ordinaire, dans la portée syntaxique d'un prédicat extérieur à la structure. Si cela s'applique directement à des constructions comme ACC ou RNR, pour le gapping cela nécessite des modifications de la grammaire. 

Les deux processus majeurs envisagés en grammaire catégorielle sont la montée de type et la composition fonctionnelle. Pour toute séquence de trois constituants, ces règles permettent de déterminer le troisième, à partir des deux premiers. Ce qui est particulier au gapping, par rapport à d'autres coordinations de non-constituants, est l'introduction d'une règle spéciale pour l'ellipse, qui permet la décomposition de la phrase source, afin de pouvoir isoler le prédicat et obtenir une catégorie fonctionnelle similaire à celle obtenue pour la phrase trouée. J'illustre la représentation syntaxique du gapping en \REF{ex:4:209} et \REF{ex:4:210}. On commence par~la composition fonctionnelle de la séquence trouée \REF{ex:4:209}. On obtient une catégorie complexe qu'on ne peut pas combiner telle quelle avec la catégorie phrastique (S) de la phrase source. Ce qui oblige à postuler une règle de décomposition dans la phrase source \REF{ex:4:210} ; c'est une régle {\guillemotleft}~révélatrice~{\guillemotright} qui s'applique au verbe antécédent (ayant le statut discursif de \textit{donné}), afin de récupérer un constituant {\guillemotleft}~révélateur~{\guillemotright} qui va contribuer à déterminer l'interprétation de la séquence trouée. Après avoir mis de côté le verbe {\guillemotleft}~topique~{\guillemotright}, on obtient ainsi un constituant ayant la même catégorie fonctionnelle que celle obtenue par la composition de la séquence trouée. Ayant deux constituants de même catégorie, on peut maintenant les coordonner. Le résultat obtenu est finalement appliqué au verbe {\guillemotleft}~topique~{\guillemotright}, qu'on avait séparé lorsqu'on avait appliqué la première règle.


\begin{enumerate}
\item \label{bkm:Ref289877875}Composition de la phrase trouée  


\end{enumerate}
  Harry eats beans and  Barry    potatoes

          ---------------------------

      conj  S/(S{\textbackslash}NP) (S{\textbackslash}NP){\textbackslash}((S{\textbackslash}NP)/NP)  

 -------------------

 [S/(S{\textbackslash}NP)]\& --------------------------------------------

         [S{\textbackslash}((S{\textbackslash}NP)/NP)]\& \begin{enumerate}
\item \label{bkm:Ref289877878}Décomposition de la phrase source  


\end{enumerate}
  Harry  eats  beans      and Barry potatoes

   ---------------------------    -----------------------

       S          [S{\textbackslash}((S{\textbackslash}NP)/NP)]\& -----------------------------{\textless}decompose

 (S{\textbackslash}NP)/NP S{\textbackslash}((S{\textbackslash}NP)/NP)

         ---------------------------------------------------{\textless}\& S{\textbackslash}((S{\textbackslash}NP)/NP)

  ----------------------------------------------{\textless}

      S

\citet{Gardent1991} considère que la proposition de \citet{Steedman1990} pose quelques problèmes : (i) linguistiquement, elle ne permet pas de prédire quelles combinaisons sont acceptées ou refusées par la grammaire ; (ii) computationnellement, elle permet des structures qui sont distinctes du point de vue de leur dérivation, mais qui sont équivalentes au niveau sémantique ; (iii) empiriquement, la catégorie fonctionnelle obtenue est inadéquate, car elle ne rend pas compte des cas où ce qui manque est, à part le verbe, aussi un syntagme nominal ou un syntagme prépositionnel. 

Par conséquent, \citet{Gardent1991} propose une solution, en se basant sur le travail de Sag \textit{et al.} (1985). Sag \textit{et al.} (1985) analysent la catégorie de la séquence trouée comme une variable sur une ou plusieurs catégories de type X\textsuperscript{2*}. Cette proposition est adoptée par \citet{Gardent1991}, qui y ajoute la notion de \textit{product category}, c.-à-d. une séquence de catégories qui fonctionne comme une catégorie complexe. La légitimation du fragment se fait à l'interface syntaxe-sémantique. Crucialement, la récupération de l'information met en jeu, dans les deux approches, un mécanisme de \textit{substitution}, définie sur les arbres syntaxiques. Grossièrement, une phrase trouée est légitimée dans la grammaire si et seulement si la séquence de catégories formant une catégorie complexe peut être \textit{substituée} dans l'arbre de dérivation de la phrase source. La reconstruction de l'arbre est donc un mécanisme essentiel pour l'analyse du gapping, car elle assure l'interprétabilité d'une phrase et elle vérifie aussi la grammaticalité de la phrase trouée.


\begin{enumerate}
\item   Représentation d'une construction à gapping dans \citet{Gardent1991}


\end{enumerate}
{   [Warning: Image ignored] % Unhandled or unsupported graphics:
%\includegraphics[width=6.2819in,height=2.5244in,width=\textwidth]{fe443409cd384d3fb0f6390ffd77f513-img38}
} 


\begin{enumerate}
\item   Représentation d'une construction à gapping dans Sag \textit{et al.} (1985)[Warning: Draw object ignored]  


\end{enumerate}
{   [Warning: Image ignored] % Unhandled or unsupported graphics:
%\includegraphics[width=5.0035in,height=3.3283in,width=\textwidth]{fe443409cd384d3fb0f6390ffd77f513-img39.jpg}
} 

Ce mécanisme de substitution permet de prendre en compte les informations syntaxiques fournies par la phrase source, qui s'appliquent à la phrase trouée aussi (p.ex. les contraintes de sous-catégorisation en \REF{ex:4:213}). 


\begin{enumerate}
\item \label{bkm:Ref289880599}a   Pat has become crazy, and Chris depressed.  


\end{enumerate}
  b  Pat has become crazy, and Chris an incredible bore.

c  *Pat has become crazy, and Chris in good spirits.     (Sag \textit{et al.} (1985 : 160))

De ce point de vue, Sag \textit{et al.} (1985) considèrent qu'une approche purement sémantique (\citet{Stump1978}) n'est pas adéquate. Un deuxième argument qui montre la supériorité des approches à l'interface syntaxe-sémantique~vient du fait que la substitution permet des discordances interprétatives entre les éléments formant une paire contrastive, p.ex. l'interprétation \textit{de re} vs. \textit{de dicto~}: un élément résiduel peut avoir une interprétation \textit{de re}, alors que son corrélat a une interprétation \textit{de dicto}, et vice-versa. En revanche, dans une approche purement sémantique, l'élément résiduel et son corrélat doivent partager l'interprétation.


\begin{enumerate}
\item Pat is looking for a piece of paper, and Chris, a pencil.      (Sag \textit{et al.} (1985 : 162))


\end{enumerate}
Sag \textit{et al.} (1985) n'ajoutent pas d'autres contraintes syntaxiques, leur hypothèse étant que la plupart des contraintes qui jouent sur l'acceptabilité des constructions à gapping sont de nature extra-syntaxique, cf. \citet{Hankamer1973}, \citet{Kuno1976} et \citet{Sag1976}. En revanche, \citet{Gardent1991} incorpore la contrainte sur la constituance majeure (c.-à-d. l'élément résiduel doit être l'argument ou l'ajout du verbe antécédent ou bien l'argument d'un verbe contenu dans l'argument phrastique du verbe antécédent), afin d'éviter les problèmes de sur-génération qui dérivent de l'analyse de Sag \textit{et al.} (1985), c.-à-d. des phrases qui devraient être agrammaticales sont acceptées par la grammaire, et des phrases qui ne devraient avoir qu'une seule lecture en reçoivent plusieurs.  

Une règle de légitimation de fragments à l'interface syntaxe-sémantique est proposée aussi par Culicover \& \citet{Jackendoff2005}. Le même mécanisme est en place : La phrase trouée est analysée comme une phrase fragmentaire sans tête verbale, dont la bonne formation est déterminée par un principe de substitution (comme pour la coordination en général, cf. la généralisation de Wasow). On doit pouvoir remplacer les éléments corrélats dans la phrase source par les éléments résiduels de la phrase trouée et obtenir une structure qui soit syntaxiquement et sémantiquement bien formée. 

Selon Culicover \& \citet{Jackendoff2005}, les constructions à gapping relèvent d'un mécanisme de {\guillemotleft}~légitimation indirecte~{\guillemotright} (angl. \textit{Indirect Licensing}) d'au moins deux constituants {\guillemotleft}~orphelins~{\guillemotright}. La légitimation indirecte inclut : (i) l'intégration sémantique des constituants orphelins dans une structure propositionnelle P (\~{} reconstruction sémantique), P étant pragmatiquement liée à la phrase source, et (ii) l'intégration syntaxique par substitution en parallèle (\textit{matching}) des paires contrastives. Pour que cela se fasse, les traits morphosyntaxiques des éléments résiduels doivent être compatibles avec ceux imposés aux corrélats par le prédicat antécédent. Les règles du gapping sont données en \REF{ex:4:215}. Une représentation simplifiée de la syntaxe et de la sémantique d'une construction à gapping figure en \REF{ex:4:216} et respectivement \REF{ex:4:217}.


\begin{enumerate}
\item   \label{bkm:Ref300950575}Règles du gapping dans Culicover \& \citet{Jackendoff2005}


\end{enumerate}
  [Warning: Image ignored] % Unhandled or unsupported graphics:
%\includegraphics[width=4.1272in,height=0.6874in,width=\textwidth]{fe443409cd384d3fb0f6390ffd77f513-img40.svm}
 


\begin{enumerate}
\item   \label{bkm:Ref300950610}La syntaxe d'une construction à gapping dans Culicover \& \citet{Jackendoff2005}


\end{enumerate}
{   [Warning: Image ignored] % Unhandled or unsupported graphics:
%\includegraphics[width=5.2055in,height=2.1453in,width=\textwidth]{fe443409cd384d3fb0f6390ffd77f513-img41.svm}
} 


\begin{enumerate}
\item   \label{bkm:Ref300951078}La sémantique d'une construction à gapping dans Culicover \& \citet{Jackendoff2005}


\end{enumerate}
  [Warning: Image ignored] % Unhandled or unsupported graphics:
%\includegraphics[width=4.6075in,height=0.5in,width=\textwidth]{fe443409cd384d3fb0f6390ffd77f513-img42.svm}
 

La différence majeure entre la substitution postulée par Culicover \& \citet{Jackendoff2005} et la substitution proposée par Sag \textit{et al.} (1985) et \citet{Gardent1991} est que la première est définie sur la structure argumentale du prédicat, alors que dans le deuxième cas, elle se définit sur les arbres syntaxiques. 

Mis à part les arguments convaincants qu'ils donnent contre une reconstruction syntaxique du verbe dans la phrase trouée, le reproche qu'on peut faire à l'approche de Culicover \& \citet{Jackendoff2005} est que les contraintes de parallélisme syntaxique sont très strictes ; or, on a montré dans la section  que le roumain permettait une certaine souplesse quant à la catégorie syntaxique, le nombre d'éléments contrastés et l'ordre des mots. Un autre reproche concerne le statut syntaxique qu'ils donnent à la phrase trouée : dans leur vision, la phrase trouée est un ajout à la phrase source. Néanmoins, les propriétés des coordinations canoniques en général et celles à gapping en particulier semblent être différentes des conjoints incidents (cf. Abeillé (2005)) : premièrement, une construction à gapping peut présenter une conjonction {\guillemotleft}~corrélative~{\guillemotright} (et on obtient ainsi une coordination omnisyndétique), alors qu'un conjoint incident ne peut jamais être introduit par une conjonction corrélative ; deuxièmement, si la contrainte sur l'extraction parallèle des deux conjoints peut s'appliquer aux constructions à gapping, elle ne s'applique pas aux conjoints incidents ; troisièmement, le dernier conjoint dans une coordination à gapping manque de mobilité, alors que le conjoint incident est mobile dans la phrase. Pour l'illustration de ces différences en roumain, voir Bîlbîie (2008). 

A l'instar des analyses discutées ici, je présente dans la section suivante une possibilité d'analyse dans le cadre HPSG, tout en restant dans l'esprit de ces approches non-structurales.

\subsection{Une analyse constructionnelle en HPSG}
\label{bkm:Ref290285733}Cette section repose en partie sur l'analyse présentée dans Abeillé, Bîlbîie \& Mouret \textit{à paraître.} Le modèle qu'on retient ici est une version constructionnelle de HPSG (cf. \citet{Sag1997}, Ginzburg \& \citet{Sag2000}, Sag \textit{et al.} (2003), Abeillé (2007)). 

Je ne reprends pas ici l'introduction au modèle HPSG, ni l'analyse formelle des constructions coordonnées. Pour une présentation de l'architecture générale en HPSG, voir section \ref{sec:1.4.2} du chapitre 1. Pour une description de l'analyse syntaxique des coordinations en HPSG, voir section \ref{sec:2.10} du chapitre 2 (et en particulier la sous-section \ref{sec:2.10.3}). 

J'insiste donc par la suite sur l'analyse des coordinations à gapping en termes de \textit{fragments}, comme cela a été proposé par Ginzburg \& \citet{Sag2000} pour les questions et les réponses courtes en anglais, et par Culicover \& \citet{Jackendoff2005} pour plusieurs phénomènes elliptiques.  

\subsubsection{Formalisation des constructions à gapping}
\label{bkm:Ref300002542}D'abord, je présente la formalisation des séquences de constituants appelées \textit{clusters}, qui nous permet de générer toute séquence qu'on peut avoir dans une phrase trouée (section \ref{sec:4.5.1.1}). Ensuite, je montre comment la notion de \textit{fragment} nous permet d'attribuer un contenu propositionnel à la séquence trouée et, en particulier, comment on récupère le contenu de l'antécédent afin d'obtenir la bonne interprétation dans la phrase trouée (section \ref{sec:4.5.1.2}). Finalement, je montre le fonctionnement spécifique des constructions à gapping, en postulant un sous-type de syntagme coordonnée, appelé \textit{gapping-ph} (section \ref{sec:4.5.1.3}).

\paragraph[Une théorie des clusters]{Une théorie des clusters}
\label{bkm:Ref299904255}\label{bkm:Ref299987535}La séquence trouée est composée d'au moins deux éléments résiduels. On a montré dans la section \ref{sec:4.4.3} qu'elle n'a pas toujours la même distribution que la phrase source. En particulier, on observe qu'en français, par exemple, la conjonction \textit{ainsi} \textit{que} peut être suivie d'une séquence de syntagmes ayant un contenu propositionnel, mais elle ne peut pas être suivie d'une phrase \REF{ex:4:218}a. Mouret (2006, 2007), à la suite de Abeillé \& \citet{Godard1996}, observe la même contrainte pour la coordination de séquences (ou ACC) en \REF{ex:4:218}b, ce qui l'amène à analyser ces séquences comme \textit{clusters}.


\begin{enumerate}
\item \label{bkm:Ref290591358}a  Paul a cueilli des framboises, ainsi que Marie (*a cueilli) des fraises. 


\end{enumerate}
  b  Paul a offert un livre à Marie, ainsi qu'(*il a offert) un CD à Anne.  

Mouret (2006, 2007) propose que la notion de \textit{cluster} soit définie indépendamment de la coordination dans la grammaire, car elle est pertinente aussi pour les séquences elliptiques dans le domaine de la subordination \REF{ex:4:219} ou dans le dialogue \REF{ex:4:220}, dans lesquelles les constituants immédiats n'entretiennent pas de relations fonctionnelles.  


\begin{enumerate}
\item \label{bkm:Ref290592974}a  Tout comme {\textless}Marie son thé{\textgreater}, Paul a apprécié son café. 


\end{enumerate}
  b  Plusieurs personnes sont parties à l'étranger, dont {\textless}deux à Rome{\textgreater}.  


\begin{enumerate}
\item \label{bkm:Ref290593112}A :  Je me demande ce que Paul peut bien vendre comme livres et à qui, dans sa librairie miteuse. 


\end{enumerate}
  B :  {\textless}Des livres d'occasion à quelques collectionneurs aventureux{\textgreater}, j'imagine.

Par conséquent, on reprend ici la notion de \textit{cluster} afin de rendre compte de la constituance des séquences trouées dans les constructions à gapping. On propose que la séquence de constituants non-standard, sans tête verbale, dans le gapping soit modélisée comme un sous-type de syntagme sans tête (c.-à-d. \textit{cluster-ph}). Une hiérarchie de syntagmes, incluant le syntagme de type cluster, figure en \REF{ex:4:221}. 


\begin{enumerate}
\item \label{bkm:Ref299877221}Le cluster dans une hiérarchie de syntagmes


\end{enumerate}
{   [Warning: Image ignored] % Unhandled or unsupported graphics:
%\includegraphics[width=6.2161in,height=1.5264in,width=\textwidth]{fe443409cd384d3fb0f6390ffd77f513-img43.svm}
} 

Comme règle syntaxique, on a juste besoin d'une règle qui génère les séquences de syntagmes, similaire à la composition de catégories en grammaire catégorielle (\citet{Steedman1990}). La description formelle d'un syntagme cluster est donnée en \REF{ex:4:222}. Les constituants immédiats du cluster (p.ex. les éléments résiduels dans une séquence trouée) sont enregistrés dans un trait de tête CLUSTER, qui prend comme valeur la liste (non-vide) des descriptions \textit{synsem} de ses branches. Les propriétés syntaxiques et sémantiques des éléments résiduels sont ainsi accessibles au niveau de la construction. En plus, le cluster est un syntagme saturé pour ses traits de valence (cf. la valeur vide des attributs SUBJ, SPR et COMPS). Il amalgame les valeurs SLASH de ses constituants (ce qui nous permet de rendre compte de la contrainte d'extraction parallèle hors d'une coordination de séquences). Les autres propriétés (y compris la catégorie) sont sous-spécifiées, ce qui lui permet la combinaison avec des formes comme \textit{ainsi que} en français, qui n'est jamais compatible avec une phrase finie. 


\begin{enumerate}
\item \label{bkm:Ref299878790}Syntagme de type cluster (cf. Mouret (2006, 2007))


\end{enumerate}
  [Warning: Image ignored] % Unhandled or unsupported graphics:
%\includegraphics[width=4.2417in,height=1.6374in,width=\textwidth]{fe443409cd384d3fb0f6390ffd77f513-img44.svm}
 

Postuler un trait spécifique CLUSTER peut générer a priori toute séquence de syntagmes. Cette sous-spécification massive peut être évitée si on restreint le potentiel combinatoire des clusters. Pour cela, \citet{Mouret2007} propose la contrainte en \REF{ex:4:223}, qui assure que les clusters n'apparaissent jamais dans la structure argumentale d'un prédicat.  


\begin{enumerate}
\item \label{bkm:Ref299888553}\textit{word} ={\textgreater} [ARG-ST \textit{list} ([CLUSTER {\textless} {\textgreater}])]


\end{enumerate}
La contrainte qui décrit les clusters en \REF{ex:4:222} nous permet maintenant de dériver une séquence de deux éléments résiduels dans le gapping \REF{ex:4:224}, comme le montre la représentation simplifiée en \REF{ex:4:225}.


\begin{enumerate}
\item \label{bkm:Ref299889957}Ioanei i-am dat un măr, iar [[Mariei] [o banană]]. 


\end{enumerate}
    Ioana.\textsc{obl cl.obl}-ai donné une pomme, et Maria\textsc{.obl} une banane

     \textit{A Ioana j'ai donné une pomme et à Maria une banane}


\begin{enumerate}
\item \label{bkm:Ref299900245}Représentation simplifiée de \REF{ex:4:224}


\end{enumerate}
{   [Warning: Image ignored] % Unhandled or unsupported graphics:
%\includegraphics[width=3.0543in,height=1.9457in,width=\textwidth]{fe443409cd384d3fb0f6390ffd77f513-img45.svm}
} 

\paragraph[Une théorie des fragments]{Une théorie des fragments}
\label{bkm:Ref302389353}Grâce à la règle syntaxique associée au syntagme cluster \REF{ex:4:222}, la grammaire peut maintenant générer toute séquence de deux éléments résiduels qu'on rencontre dans les constructions à gapping. Il faut montrer toutefois comment on arrive à attribuer un contenu propositionnel à la séquence trouée et, en particulier, comment on récupère le contenu de l'antécédent afin d'obtenir la bonne interprétation dans la phrase trouée. 

Dans la section \ref{sec:4.4.3}, on a observé qu'une séquence trouée n'a pas nécessairement le même comportement syntaxique qu'une phrase ordinaire, bien qu'elle ait le même type de contenu sémantique (c.-à-d. un sous-type de \textit{message}). Cela nous amène à considérer que la reconstruction de l'ellipse se passe plutôt en sémantique qu'en syntaxe. La même idée apparaît dans Ginzburg \& \citet{Sag2000} où on propose la notion de \textit{fragment} pour rendre compte de la structure des questions et des réponses courtes dans le dialogue. Ainsi, dans la réponse/question du locuteur B en \REF{ex:4:226} et respectivement \REF{ex:4:227}, on a un syntagme nominal exhaustivement dominé par une phrase, ayant l'interprétation d'une phrase déclarative (c.-à-d. \textit{John left}) et respectivement interrogative (c.-à-d. \textit{Who called}). On se donne en syntaxe la notion de \textit{fragment} conçue comme une construction à laquelle sont associées des conditions de bonne formation syntaxiques et interprétatives.


\begin{enumerate}
\item \label{bkm:Ref290576352}A :   Who left ? 


\end{enumerate}
  B :  [[John]\textsubscript{NP}]\textsubscript{S}. 


\begin{enumerate}
\item \label{bkm:Ref290576354}A :   Someone called. 


\end{enumerate}
  B :  [[Who]\textsubscript{NP}]\textsubscript{S~}?  

La structure syntaxique des fragments dans les constructions à gapping contient uniquement les éléments résiduels (qui, quant à leur constituance, forment un cluster). Par conséquent, les fragments sont des expressions dont la contribution sémantique n'est donnée que partiellement par leur forme ; ce sont des unités syntaxiques dont l'interprétation nécessite une connaissance de la relation sémantique principale de l'énoncé. Leur contribution sémantique est une fonction de trois éléments : (i) le type de fragment, (ii) l'information contextuelle, et (iii) le contenu littéral du fragment. Par exemple, la contribution sémantique du fragment \textit{who} en \REF{ex:4:227} est une fonction de son type (ici, une question courte, ayant le même contenu qu'une phrase interrogative, c.-à-d. une abstraction propositionnelle), l'information contextuelle (la phrase source \textit{Someone called} fournit l'antécédent nécessaire à la résolution du fragment), et le contenu littéral \textit{who} (qui fournit le paramètre pour l'abstraction propositionnelle). Ainsi, le fragment \textit{who} a un contenu similaire avec la phrase interrogative complète \textit{Who came}.

Ce sont des fragments phrastiques, car (i) leur interprétation est univoque dans le contexte, et (ii) ils ont le même type sémantique que les phrases complètes, c.-à-d. proposition, question, visée.

Les fragments phrastiques ressemblent aux expressions anaphoriques, dans le sens où leur contenu est relié à un antécédent qui est déterminé contextuellement. En particulier, les fragments se rapprochent des anaphores descriptives, qui, contrairement aux anaphores d'instance, ne désignent pas la même instance que l'antécédent ; ce type d'expressions anaphoriques introduit une nouvelle entité sémantique qui partage une partie de sa description avec l'antécédent, mais l'entité elle-même n'est pas partagée, ce qui explique la différence d'indices en \REF{ex:4:228}b.\footnote{Ce n'est pas le cas des anaphores d'instance, qui partagent le même indice avec leur antécédent.}


\begin{enumerate}
\item \label{bkm:Ref290586096}a  [Paul has lost his keys again]\textsubscript{i}. It\textsubscript{i} happened yesterday. 


\end{enumerate}
  b  [Paul has lost his keys again]\textsubscript{i}. It\textsubscript{j} has never happened to me.  

Le fragment, tel qu'il est défini par Ginzburg \& \citet{Sag2000}, est la branche unaire d'un syntagme tête. Il a l'ensemble des propriétés d'une phrase, y compris la catégorie syntaxique VERBAL. La contrainte générale qu'ils donnent pour un fragment phrastique figure en \REF{ex:4:229}.


\begin{enumerate}
\item \label{bkm:Ref299891853}Syntagme de type fragment avec tête dans Ginzburg \& \citet{Sag2000}


\end{enumerate}
  [Warning: Image ignored] % Unhandled or unsupported graphics:
%\includegraphics[width=5.4638in,height=1.2465in,width=\textwidth]{fe443409cd384d3fb0f6390ffd77f513-img46.svm}
 

Elle assure que la catégorie de la branche tête -- restreinte à un sous-type de \textit{nominal} (c.-à-d. un nom ou une préposition) -- est identique à la catégorie de l'élément parallèle (ou corrélat) dans la phrase source, figurant dans le trait contextuel SAL-UTT (abrégé de SALIENT-UTTERANCE). En revanche, la catégorie du syntagme supérieur (c.-à-d. du fragment) a la même catégorie qu'un verbe fini : le fragment peut ainsi fonctionner comme une phrase indépendante ou bien comme le complément d'un verbe qui sélectionne une phrase finie (et non un syntagme nominal). Finalement, cette contrainte co-indicie la branche tête avec l'élément corrélat contenu dans SAL-UTT. Cela permet d'intégrer le contenu de la branche tête dans un contenu obtenu contextuellement. 

Je donne en \REF{ex:4:230} l'analyse que Ginzburg \& \citet{Sag2000} proposent pour une réponse courte, p.ex. \textit{John}, à une question, p.ex. \textit{Who left~}?. 


\begin{enumerate}
\item \label{bkm:Ref299901030}Analyse de la réponse courte \textit{John.} à la question \textit{Who left ?} 


\end{enumerate}
{   [Warning: Image ignored] % Unhandled or unsupported graphics:
%\includegraphics[width=3.5772in,height=5.2591in,width=\textwidth]{fe443409cd384d3fb0f6390ffd77f513-img47.svm}
} 

La catégorie de la branche tête est un syntagme nominal, comme le requiert la contrainte donnée en \REF{ex:4:229}, alors que la catégorie du syntagme supérieur est de type \textit{verbal} (= phrase finie, de type déclarative). Le fragment est une phrase indépendante (cf. [IC +]). Le contenu de la phrase est une proposition. Si dans la plupart des syntagmes avec tête le contenu est fourni en grande partie par la branche tête, dans un syntagme de type fragment avec tête, le contenu est construit essentiellement à partir de la question saillante dans le contexte (c.-à-d. MAX-QUD, abrégé de MAXIMAL-QUESTION-UNDER-DISCUSSION). Le trait MAX-QUD nous donne l'accès au contenu de la phrase source (ici, une question). L'élément parallèle (ou corrélat) se trouvant dans la phrase source est identifié grâce au trait SAL-UTT. La branche tête du fragment et son corrélat dans la phrase source sont co-indicés. Pour les autres détails concernant cette description, voir Ginzburg \& \citet{Sag2000}, chapitre 8, section \ref{sec:8.1.4}.

On observe donc que, dans l'approche de Ginzburg \& \citet{Sag2000}, le fragment est un syntagme qui a le contenu d'une phrase et qui domine exhaustivement une tête de même catégorie et de même indice qu'un élément parallèle saillant dans le contexte. En tant que telle, l'analyse est inadéquate pour les constructions à gapping, qui ont au moins deux éléments résiduels et parfois sans identité catégorielle avec leurs corrélats dans la phrase source (voir l'exemple \REF{ex:4:128} de la section ).  

Par conséquent, on étend l'analyse de Ginzburg \& \citet{Sag2000}, afin d'inclure d'autres constructions elliptiques. Cette notion de \textit{fragment} peut a priori s'appliquer non seulement aux phrases elliptiques indépendantes (dans le dialogue), mais aussi aux constructions elliptiques coordonnées ou subordonnées, contenant (parfois) des séquences avec plus d'un élément résiduel : le gapping \REF{ex:4:231}a, l'ellipse dans les comparatives \REF{ex:4:231}b, le stripping \REF{ex:4:231}c, les ajouts exceptifs \REF{ex:4:231}d, les ajouts relatifs partitifs \REF{ex:4:231}e ou encore les ajouts concessifs \REF{ex:4:231}f. 


\begin{enumerate}
\item \label{bkm:Ref299730672}a  Tudor a cumpărat o carte, [iar Maria *(o păpuşă)].


\end{enumerate}
{\itshape
Tudor a acheté un livre, et Maria une poupée}

  b  Ioana a mâncat mai multe mere [decât Ion (pere)].

{\itshape
Ioana a mangé plus de pommes que Ion (des poires) } 

  c  Toată lumea îl apreciază pe Ion, [chiar şi duşmanii lui].

    tout le-monde \textsc{cl} apprécie \textsc{mrq} Ion, même aussi ennemis \textsc{poss}

{\itshape
Tout le monde apprécie Ion, même ses ennemis } 

  d  Niciun elev nu-şi făcuse temele, [mai puțin Ion (tema la engleză)].

{\itshape
Aucun élève n'avait fait ses devoirs, sauf Ion le devoir d'anglais } 

  e  In România, trăiesc aproximativ 8~000 de evrei, [dintre care jumătate în Bucureşti].

{\itshape
En Roumanie, vivent environ 8~000 juifs, dont la moitié à Bucarest } 

  f  Acoperişurile care sunt ude, [chiar dacă foarte puțin], pot fi foarte alunecoase.

{\itshape
    Les toitures qui sont mouillées, quoique très peu, peuvent être très glissantes } 

On reprend de Ginzburg \& \citet{Sag2000} la hiérarchie de syntagmes, en particulier le sous-type \textit{head-fragment-ph} comme sous-type de \textit{head-only-ph}. 


\begin{enumerate}
\item Le fragment dans une hiérarchie de syntagmes 


\end{enumerate}
{   [Warning: Image ignored] % Unhandled or unsupported graphics:
%\includegraphics[width=5.8402in,height=1.6299in,width=\textwidth]{fe443409cd384d3fb0f6390ffd77f513-img48.svm}
} 

Le syntagme \textit{head-fragment-ph} a une seule branche tête, qui correspond à un cluster tel que défini dans la section précédente (4.5.1.1). La représentation arborescente du fragment et de sa branche cluster est donnée en \REF{ex:4:233}.


\begin{enumerate}
\item \label{bkm:Ref302430578}Représentation arborescente du fragment et de sa branche cluster 


\end{enumerate}
{   [Warning: Image ignored] % Unhandled or unsupported graphics:
%\includegraphics[width=3.102in,height=3.0543in,width=\textwidth]{fe443409cd384d3fb0f6390ffd77f513-img49.svm}
} 

Le fragment hérite de sa branche tête (\textit{cluster-ph}) sa catégorie sous-spécifiée (comme le montre le partage de variables correspondant au trait CAT), ce qui lui permet de se combiner avec des foncteurs sélectionnant des catégories non-finies, comme c'est le cas de la conjonction \textit{ainsi que} en français, discutée dans la section \ref{sec:4.5.1.1} (voir exemple \REF{ex:4:218}). Une représentation simplifiée de la séquence \textit{ainsi que Marie des fraises} en français est donnée en \REF{ex:4:234}. 


\begin{enumerate}
\item \label{bkm:Ref299905788}Arbre simplifié pour la séquence \textit{ainsi que Marie des fraises} en français 


\end{enumerate}
{   [Warning: Image ignored] % Unhandled or unsupported graphics:
%\includegraphics[width=3.7898in,height=3.6874in,width=\textwidth]{fe443409cd384d3fb0f6390ffd77f513-img50.svm}
} 

Le fragment dans le gapping obéit à la contrainte syntaxique décrite en \REF{ex:4:235} : les éléments résiduels (figurant sur la liste du trait CLUSTER) doivent unifier leurs traits de tête avec les traits de tête de leurs corrélats dans la phrase source. Pour cela, on utilise le trait contextuel SAL-UTT de Ginzburg \& \citet{Sag2000}, qui enregistre les éléments corrélats dans la phrase source. De plus, on introduit un trait MAJOR dans la description syntaxique des éléments corrélats dans la phrase source, afin de rendre compte de la contrainte sur les constituants majeurs (discutée dans la section ) : chaque corrélat doit dépendre d'un prédicat verbal dans la phrase source (donc, chaque corrélat doit être [MAJOR +]. On postule ici que les corrélats apparaissent sur la structure argumentale d'un prédicat verbal dans la phrase source \REF{ex:4:236}b, sans qu'ils soient nécessairement réalisés syntaxiquement. S'ils ne sont pas réalisés, ils correspondent à des synsems non-canoniques. Comme je l'ai précisé dans le chapitre 1 (section \ref{sec:1.4.2}), les synsems non-canoniques apparaissent sur la structure argumentale d'un prédicat, mais ils ne figurent pas dans sa valence. Avec ces contraintes, on peut maintenant rendre compte des exemples de gapping (discutés dans la section ) dans lesquels un des corrélats correspond à un pronom nul (cf. le phénomène de pro-drop en roumain \REF{ex:4:237}a) ou bien est un affixe verbal (pronominal \REF{ex:4:237}b ou adverbial \REF{ex:4:237}c).  


\begin{enumerate}
\item \label{bkm:Ref299907665}Contrainte syntaxique du \textit{head-fragment-ph} 


\end{enumerate}
  [Warning: Image ignored] % Unhandled or unsupported graphics:
%\includegraphics[width=5.311in,height=0.7362in,width=\textwidth]{fe443409cd384d3fb0f6390ffd77f513-img51.svm}
 


\begin{enumerate}
\item \label{bkm:Ref299908583}a  \textit{synsem} ={\textgreater}~/ [MAJOR --] 


\end{enumerate}
  b  \textit{verbal-word} ={\textgreater} [ARG-ST \textit{list}([MAJOR +])]  


\begin{enumerate}
\item \label{bkm:Ref299909081}a  Lunea merg la film, iar \textbf{sora mea} la muzeu.


\end{enumerate}
le-lundi aller.\textsc{ind.1sg} à film, et s{\oe}ur.\textsc{def poss} à musée 

    \textit{Le lundi, je vais au cinéma, et ma s{\oe}ur au musée } 

  b  Ion \textbf{mi}-e prieten, iar \textbf{ție} duşman.

    Ion \textsc{cl.1sg.dat-}est ami, et toi\textsc{.2sg.dat} ennemi 

    \textit{Ion est mon ami, et ton ennemi } 

  c  Marian \textbf{tot} \textbf{mai} citeşte, dar prietena lui \textbf{absolut nimic}.

Marian \textsc{cl.adv cl.adv} lit, mais copine.\textsc{def poss} absolument rien 

{\itshape
Marian lit un peu, mais sa copine absolument rien} 

Dans le chapitre 2, section \ref{sec:2.10.3}, on a vu que les entrées lexicales et les syntagmes qu'elles projettent peuvent rester sous-spécifiées quant à leurs traits de tête. Il s'ensuit que les éléments résiduels et leurs corrélats dans les constructions à gapping n'ont pas nécessairement la même catégorie syntaxique, pourvu que le résultat de l'unification de leurs traits de tête soit en accord avec les contraintes de sous-catégorisation du verbe antécédent. Pour illustrer cela, je reprends les deux exemples de coordination de termes dissemblables donnés dans la section mentionnée ci-dessus, en les mettant cette fois-ci dans une construction à gapping \REF{ex:4:238}. Une représentation simplifiée des deux phrases est donnée en \REF{ex:4:239} et respectivement \REF{ex:4:240}.  


\begin{enumerate}
\item \label{bkm:Ref299913361}a  Ion este naiv, iar Gheorghe un imbecil.


\end{enumerate}
    \textit{Ion est naïf, et Gheorghe un imbécile } 

  b  *Ion a întâlnit un imbecil, iar Gheorghe naiv.

Ion a rencontré un imbécile, et Gheorghe naïf 

{\itshape
Ion a rencontré un imbécile, et Gheorghe un naïf} 


\begin{enumerate}
\item \label{bkm:Ref299913495}Représentation simplifiée de la phrase \REF{ex:4:238}a 


\end{enumerate}
{   [Warning: Image ignored] % Unhandled or unsupported graphics:
%\includegraphics[width=5.9689in,height=2.9417in,width=\textwidth]{fe443409cd384d3fb0f6390ffd77f513-img52.svm}
} 

Ainsi, en \REF{ex:4:239} le verbe \textit{a fi} `être' sous-catégorise un complément dont la catégorie est sous-spécifiée (c.-à-d. le super-type \textit{nominal}, qui regroupe les sous-types \textit{adj} et \textit{noun}). Parallèlement, l'unification des traits de tête de la deuxième paire contrastive (c.-à-d. HEAD 3 correspondant au syntagme adjectival \textit{naiv} et HEAD 4 correspondant au syntagme nominal \textit{un imbecil}) réussit, car il existe un super-type \textit{nominal} commun aux deux éléments (4 est donc résolu comme \textit{nominal}). Par conséquent, la coordination des termes dissemblables est possible, car il n'y a aucun désaccord entre les contraintes de sous-catégorisation du verbe attributif et la réalisation effective de ses compléments.

En revanche, l'unification de 2 et 4 échoue en \REF{ex:4:240}, car le verbe \textit{a întâlni} `rencontrer' n'accepte pas un complément sous-spécifié (en particulier, il ne peut pas sélectionner comme complément un syntagme adjectival).


\begin{enumerate}
\item \label{bkm:Ref299913497}Représentation simplifiée de la phrase \REF{ex:4:238}b 


\end{enumerate}
{   [Warning: Image ignored] % Unhandled or unsupported graphics:
%\includegraphics[width=5.8965in,height=3.0854in,width=\textwidth]{fe443409cd384d3fb0f6390ffd77f513-img53.svm}
} 

En ce qui concerne la reconstruction sémantique, il y a plusieurs possibilités. Une possibilité, entre autres, est celle proposée par Dalrymple \textit{et al.} (1991) et \citet{Dalrymple2005}, en termes purement sémantiques, pour l'ellipse du verbe dans VPE. Ils proposent de définir le contenu du fragment par l'application au contenu des éléments résiduels d'une fonction F qui résulte de l'unification d'ordre supérieur (U) de deux lambda termes : (i) la représentation sémantique de la phrase source ; (ii) la représentation sémantique résultant de l'application d'une propriété P au contenu des éléments corrélats dans la phrase source. L'illustration de cette approche est donnée en \REF{ex:4:241}.\footnote{Voir les critiques de Ginzburg \textit{à paraître} par rapport à la couverture empirique de cette approche.} 


\begin{enumerate}
\item \label{bkm:Ref302390864}a  John invited Sue and Bill Jane


\end{enumerate}
  b  John invited Sue = invited'(john', sue')

  c  [F] = U(invited'(john', sue'), P(john', sue')) = ${\lambda}$x. ${\lambda}$y. invited'(x,y)

  d  Bill Jane = [F][(bill', jane')] = ${\lambda}$x. ${\lambda}$y. [invited'(x,y)](bill', jane') 

      = invited'(bill', jane') 

Une autre possibilité serait d'utiliser le langage \textit{Minimal Recursion Semantics} (abrégé MRS), qu'on utilisera pour les ajouts relatifs averbaux dans le chapitre 5. Je ne me prononce pas sur l'une ou l'autre des pistes. On ajoute tout de même une contrainte sémantique à la définition du fragment : le contenu du fragment doit être construit par une relation \textit{R}\textit{\textsubscript{sem}} dont les arguments sont une relation contextuelle figurant dans le BACKGROUND et le contenu des éléments résiduels enregistrés dans le cluster.


\begin{enumerate}
\item Contrainte sémantique du \textit{head-fragment-ph} 


\end{enumerate}
  [Warning: Image ignored] % Unhandled or unsupported graphics:
%\includegraphics[width=6.3189in,height=1.1575in,width=\textwidth]{fe443409cd384d3fb0f6390ffd77f513-img54.svm}
 

\paragraph[La construction à gapping]{La construction à gapping}
\label{bkm:Ref302389376}Les deux contraintes, sémantique et syntaxique, qu'on a postulées dans la section précédente peuvent être utilisées dans diverses constructions elliptiques (p.ex. pour les types d'ellipse exemplifiés \textit{plus haut} en \REF{ex:4:231}). Cependant, on a besoin d'ajouter une contrainte particulière qui décrive le fonctionnement spécifique des constructions à gapping. Parmi les propriétés spécifiques du gapping dans la coordination, on note les aspects suivants : contrairement aux séquences qu'on peut rencontrer, par exemple, dans les ellipses comparatives, la séquence trouée doit suivre la phrase source en roumain et en français ; au niveau discursif, la relation qui s'établit entre les conjoints est toujours une relation symétrique. Afin de rendre compte de ses spécificités par rapport à d'autres types d'ellipse, on postule un sous-type de syntagme coordonné, appelé \textit{gapping-ph} (cf. la hiérarchie donnée en \REF{ex:4:243}). 


\begin{enumerate}
\item \label{bkm:Ref299955971}La construction à gapping dans une hiérarchie de syntagmes 


\end{enumerate}
{   [Warning: Image ignored] % Unhandled or unsupported graphics:
%\includegraphics[width=5.7957in,height=2.5102in,width=\textwidth]{fe443409cd384d3fb0f6390ffd77f513-img55.svm}
} 

Le \textit{gapping-ph} combine une liste non-vide de phrases finies avec une liste non-vide de fragments, chaque fragment contenant au moins deux éléments résiduels. La description de ce nouveau type de syntagme figure en \REF{ex:4:244}. On s'assure ainsi qu'il y a toujours une phrase source complète (qui détermine l'interprétation de la séquence trouée). Cette contrainte a l'avantage de permettre des constructions à gapping avec plusieurs phrases sources et / ou plusieurs fragments. 


\begin{enumerate}
\item \label{bkm:Ref299957003}La construction à gapping


\end{enumerate}
  [Warning: Image ignored] % Unhandled or unsupported graphics:
%\includegraphics[width=6.6453in,height=2.1055in,width=\textwidth]{fe443409cd384d3fb0f6390ffd77f513-img56.svm}
 

Un deuxième aspect important qui découle de cette contrainte concerne la relation discursive qui s'établit entre les conjoints d'une construction à gapping. Dans le trait contextuel BACKGROUND, on spécifie qu'il doit exister entre le contenu du fragment et le contenu de la phrase source une relation discursive symétrique. Cette relation discursive, à laquelle on ajoute le contenu de chaque élément résiduel, nous aide à déterminer le contenu de chaque fragment, via une relation \textit{R}\textit{\textsubscript{sem}}. 

Troisièmement, cette contrainte assure que la catégorie de la coordination dans son ensemble est donnée par ses branches non-elliptiques (c.-à-d. les phrases complètes) et non par ses branches fragmentaires. Ainsi, en \REF{ex:4:244} on observe que le syntagme coordonné et la première liste de branches (qui enregistre les phrases complètes) partagent la même valeur de tête (c.-à-d. [HEAD H]). On transgresse ainsi la règle générale s'appliquant par défaut dans la coordination, qu'on a donnée dans la section \ref{sec:2.10.3} du chapitre 2, afin d'éviter la sous-spécification de la construction à gapping dans son ensemble, vu le fait que, contrairement aux fragments, le syntagme coordonné en entier a clairement la distribution d'une phrase finie. 

A cette analyse, il reste à ajouter une contrainte qui spécifie de manière précise comment on a accès aux éléments corrélats dans la phrase source. 

Les règles postulées jusqu'ici nous permettent maintenant de générer les phrases en \REF{ex:4:245}, dans lesquelles on a trois types d'asymétries : les éléments résiduels n'ont pas la même catégorie, pas le même ordre et pas le même nombre. Les arbres associés sont illustrés en \REF{ex:4:246}, \REF{ex:4:247} et \REF{ex:4:248}. 


\begin{enumerate}
\item \label{bkm:Ref299962029}a  (Lunea) merg la film, (iar) \textbf{sora mea} la muzeu.


\end{enumerate}
(le-lundi) aller.\textsc{ind.1sg} à film, (et) s{\oe}ur.\textsc{def poss} à musée 

    \textit{Le lundi, je vais au cinéma, et ma s{\oe}ur au musée } 

  b  Ion \textbf{mi}-e prieten, (iar) \textbf{ție} duşman.

    Ion \textsc{cl.1sg.dat-}est ami, (et) toi\textsc{.2sg.dat} ennemi 

    \textit{Ion pour moi est un ami, et pour toi un ennemi } 

  c  \textbf{Mai} merg acasă, (dar) la socri \textbf{niciodată}.

\textsc{cl.adv} vais \textsc{adv}, (mais) chez beaux-parents jamais 

{\itshape
Je vais de temps en temps à la maison, mais chez mes beaux-parents jamais} 


\begin{enumerate}
\item \label{bkm:Ref299979855}Arbre simplifié de la phrase \REF{ex:4:245}a


\end{enumerate}
{   [Warning: Image ignored] % Unhandled or unsupported graphics:
%\includegraphics[width=6.2839in,height=4.0417in,width=\textwidth]{fe443409cd384d3fb0f6390ffd77f513-img57.svm}
} 


\begin{enumerate}
\item \label{bkm:Ref299979857}Arbre simplifié de la phrase \REF{ex:4:245}b


\end{enumerate}
{   [Warning: Image ignored] % Unhandled or unsupported graphics:
%\includegraphics[width=7.1772in,height=4.411in,width=\textwidth]{fe443409cd384d3fb0f6390ffd77f513-img58.svm}
} 


\begin{enumerate}
\item \label{bkm:Ref299979859}Arbre simplifié de la phrase \REF{ex:4:245}c


\end{enumerate}
{   [Warning: Image ignored] % Unhandled or unsupported graphics:
%\includegraphics[width=7.1327in,height=4.4756in,width=\textwidth]{fe443409cd384d3fb0f6390ffd77f513-img59.svm}
} 

\subsection{L'ellipse périphérique gauche : gapping ou coordination de séquences ?}
\label{bkm:Ref287994809}Le but de ce chapitre était de décrire les propriétés du gapping en roumain, de démontrer qu'une analyse à base de reconstruction syntaxique n'était pas adéquate et de proposer une solution à l'interface syntaxe-sémantique en termes constructionnels, sans postuler d'effacement, d'élément vide ou de mouvement. 

Dans tous les exemples de gapping observés jusqu'ici, le verbe antécédent dans la phrase source se trouvait en position médiane \REF{ex:4:249}a et parfois en position finale \REF{ex:4:249}b. C'est un choix que j'ai fait, car, comme je l'ai déjà précisé tout au début de ce chapitre dans la section \ref{sec:4.2.2}, ce sont des distributions non-ambiguës quant au type de structure envisagée, c.-à-d. les deux distributions mettent en jeu une coordination de phrases, dont une est complète et l'autre fragmentaire. Par la suite, je ferai référence à ces contextes comme étant du gapping classique, pour le différencier d'une autre occurrence du gapping dans les coordinations {\guillemotleft}~elliptiques~{\guillemotright} ayant le verbe antécédent en position initiale.


\begin{enumerate}
\item \label{bkm:Ref299721530}a  [Ion vine azi], [iar Maria mâine].


\end{enumerate}
Ion vient aujourd'hui, et Maria demain

{\itshape
Ion vient aujourd'hui, et Maria demain}

  b  [Ion \textsc{azi} vine], [iar Maria \textsc{mai}ne].

    Ion aujourd'hui vient, et Maria demain

{\itshape
Ion c'est aujourd'hui qu'il vient, et Maria demain}

Je m'intéresse dans cette section à la distribution restante, c.-à-d. les structures dans lesquelles le verbe antécédent est en position initiale. Toujours dans la section \ref{sec:4.2.2}, j'avais mentionné que, en dehors d'une étude empirique des données, les contextes elliptiques avec un verbe en position initiale, comme en \REF{ex:4:250}, se prêtaient \textit{a priori} à deux analyses. Selon la première possibilité d'analyse \REF{ex:4:250}a, on coordonne deux phrases : une phrase source qui contient le verbe en position initiale, et une phrase trouée. Selon la deuxième interprétation \REF{ex:4:250}b, on coordonne deux séquences de syntagmes dans la portée syntaxique du prédicat verbal, donc il n'y a aucune ellipse dans la structure. Ainsi, on peut étiqueter les deux structures possibles comme du gapping dans le premier cas \REF{ex:4:250}a, ou bien comme une coordination de séquences\footnote{Pour la description des coordinations de {\guillemotleft}~non-constituants~{\guillemotright} se trouvant à droite du verbe tête, on trouve dans la littérature les termes suivants : \textit{Conjunction Reduction}, \textit{Left Peripheral Ellipsis} ou \textit{Argument Cluster Coordination}. Je reprends ici le terme utilisé par Mouret (2007, 2008) : coordination de séquences.}  dans le deuxième cas \REF{ex:4:250}b.


\begin{enumerate}
\item \label{bkm:Ref299721593}a  [Vine Ion azi] [şi Maria mâine].


\end{enumerate}
vient Ion aujourd'hui, et Maria demain

{\itshape
Ion vient aujourd'hui, et Maria demain}

  b  Vine [Ion azi] [şi Maria mâine].

Par ailleurs, la distinction gapping vs. coordination de séquences se fait habituellement en termes de position dans l'arbre syntaxique ; ainsi, en anglais ou en français, on peut dire que les éléments contrastifs dans une séquence à gapping ne sont pas nécessairement au même niveau \REF{ex:4:251}a-\REF{ex:4:252}a, tandis que dans une coordination de séquences, les éléments contrastifs sont nécessairement des constituants s{\oe}urs \REF{ex:4:251}b-\REF{ex:4:252}b. Quant au roumain, j'ai montré dans le chapitre 1 qu'on n'avait pas d'arguments empiriques pour postuler une structure hiérarchique, et en particulier qu'il n'y avait pas de syntagme verbal fini, ce qui implique que tous les dépendants du verbe se trouvent au même niveau (y compris le sujet). Je considère donc que la distinction gapping classique vs. coordination de séquences est plutôt une question d'adjacence, c.-à-d. la linéarisation des éléments contrastifs dans la phrase source, par rapport au verbe tête.


\begin{enumerate}
\item \label{bkm:Ref290053881}a   We play poker at our house, and bridge at Betsy's house. 


\end{enumerate}
  b  At our house we play poker, and at Betsy's house, bridge. 


\begin{enumerate}
\item \label{bkm:Ref290053883}a   Paul apportera un disque à Marie et un livre à Jean. 


\end{enumerate}
  b  A Marie Paul apportera un disque et à Jean un livre. 

Les mêmes possibilités d'ellipse avec le placement médian ou initial du verbe antécédent se retrouvent dans les subordonnées aussi : en \REF{ex:4:253}a, on a un exemple typique de gapping, alors qu'en \REF{ex:4:253}b-c, le verbe est suivi de deux séquences de syntagmes pour lesquelles on peut supposer deux possibilités d'analyse.


\begin{enumerate}
\item \label{bkm:Ref289967144}a  Vreau ca [Ion să vină azi], [iar Maria mâine].


\end{enumerate}
veux.\textsc{1sg} que Ion \textsc{mrq} vienne aujourd'hui, et Maria demain

{\itshape
Je veux que Ion vienne aujourd'hui, et Maria demain}

  b  Vreau [să vină Ion azi], [iar Maria mâine].

    veux.\textsc{1sg} \textsc{mrq} vienne Ion aujourd'hui, et Maria demain

{\itshape
Je veux que Ion vienne aujourd'hui, et Maria demain}

  c  Vreau să vină [Ion azi] [şi Maria mâine].

L'hypothèse selon laquelle il y a deux possibilités d'analyse pour les coordinations de séquences se trouvant à droite du verbe est justifiée, entre autres, par le placement des conjonctions corrélatives dans les coordinations omnisyndétiques. Ainsi, dans un contexte typique de gapping, où le verbe se trouve en position médiane, la conjonction initiale doit obligatoirement précéder la phrase source, comme en \REF{ex:4:254}, alors que, dans le cas où le verbe précède les séquences, la conjonction initiale peut apparaître soit à l'initiale du verbe tête \REF{ex:4:255}a, soit à l'initiale de la coordination de séquences \REF{ex:4:255}.


\begin{enumerate}
\item \label{bkm:Ref289938832}a  \textbf{Fie} Ion vine azi, \textbf{fie} Maria mâine.


\end{enumerate}
{\itshape
Soit Ion vient aujourd'hui, soit Maria demain}

  b  *Ion vine \textbf{fie} azi, \textbf{fie} Maria mâine.

    Ion vient soit aujourd'hui, soit Maria demain

{\itshape
Soit Ion vient aujourd'hui, soit Maria demain}


\begin{enumerate}
\item \label{bkm:Ref289938899}a  \textbf{Fie} vine Ion azi, \textbf{fie} Maria mâine.


\end{enumerate}
Soit vient Ion aujourd'hui, soit Maria demain

{\itshape
Soit Ion vient aujourd'hui, soit Maria demain}

  b  Vine \textbf{fie} Ion azi, \textbf{fie} Maria mâine.

    vient soit Ion aujourd'hui, soit Maria demain

{\itshape
Soit Ion vient aujourd'hui, soit Maria demain}

Les coordinations de séquences (à tête initiale) qui m'intéressent ici concernent non seulement les séquences avec sujet postverbal, mais aussi les séquences typiques de compléments/ajouts. Comme on vient de le voir pour l'anglais en \REF{ex:4:251} et le français en \REF{ex:4:252}, le roumain permet deux linéarisations : soit les deux séquences sont à droite du verbe tête (p.ex. \REF{ex:4:256}a et \REF{ex:4:257}a), soit un des compléments/ajouts est préverbal (p.ex. \REF{ex:4:256}b et \REF{ex:4:257}b).  


\begin{enumerate}
\item \label{bkm:Ref302391672}a  I-am dat Ioanei o carte, iar Mariei un stilou.


\end{enumerate}
\textsc{cl}-ai donné Ioana\textsc{.dat} un livre, et Maria.\textsc{dat} un stylo

{\itshape
J'ai donné à Ioana un livre, et à Maria un stylo}

  b  Ioanei i-am dat o carte, iar Mariei un stilou.

    Ioana.\textsc{dat cl}-ai donné un livre, et Maria.\textsc{dat} un stylo

{\itshape
A Ioana j'ai donné un livre, et à Maria un stylo}


\begin{enumerate}
\item \label{bkm:Ref302391675}a  Am fost în 2004 la Roma, iar în 2005 la Londra.


\end{enumerate}
ai été en 2004 à Rome, et en 2005 à Londres

{\itshape
J'ai été en 2004 à Rome, et en 2005 à Londres}

  b  In 2004, am fost la Roma, iar în 2005 la Londra.

    en 2004, ai été à Rome, et en 2005 à Londres

\textit{En 2004, j'ai été à Rome, et en 2005 à Londres} 

Dans ce qui suit, je m'intéresse donc aux coordinations de séquences (sujet, complément et ajout tout confondu) se trouvant à droite du verbe tête. Le but est de vérifier~si ce type de structures se comporte comme une coordination de phrases (dont une fragmentaire) ou bien comme une coordination (sous-phrastique) de séquences dans la portée syntaxique du verbe tête. 


\begin{enumerate}
\item a  Coordination de phrases       b  Coordination de clusters 


\end{enumerate}
  [Warning: Image ignored] % Unhandled or unsupported graphics:
%\includegraphics[width=2.9972in,height=1.0508in,width=\textwidth]{fe443409cd384d3fb0f6390ffd77f513-img60.svm}
   [Warning: Image ignored] % Unhandled or unsupported graphics:
%\includegraphics[width=3.1236in,height=1.05in,width=\textwidth]{fe443409cd384d3fb0f6390ffd77f513-img61.svm}
 

Je commence par mentionner brièvement les arguments empiriques à l'encontre d'une reconstruction syntaxique dans ce type de structures (section ). Par conséquent, les deux autres possibilités restantes sont exactement celles que je viens de postuler \textit{plus haut~}: (i) une analyse similaire à celle proposée pour les constructions à gapping, c.-à-d. une structure fragmentaire sans tête verbale, dont la bonne formation réside dans un principe de substitution, ou bien (ii) une analyse sans ellipse, c.-à-d. une coordination de séquences (ou clusters), qui satisfait les exigences de sous-catégorisation d'un prédicat comme une suite de constituants ordinaires. La première solution envisageable a été déjà discutée pour le gapping en roumain dans la section \ref{sec:4.5}. C'est pour cela que, dans la deuxième partie de cette section, je discuterai plutôt l'analyse sans ellipse, proposée par Mouret (2006, 2007, 2008) pour la coordination de séquences en français (section ). Ensuite, dans une troisième partie (section ), je motiverai sur une base strictement empirique le besoin de postuler les deux analyses (c.-à-d. gapping \textit{vs.} coordination de séquences) pour rendre compte des différences qui existent entre les coordinations avec \textit{iar} et les coordinations avec la conjonction \textit{şi}. Dans la dernière section, je présente la formalisation de ces données dans le cadre HPSG (section ).  

{\bfseries
\label{bkm:Ref302392620}Pas de reconstruction syntaxique}

Une approche syntaxique de l'ellipse (cf. van \citet{Oirsouw1987}, \citet{Wilder1997}, \citet{Crysmann2003}, Beavers \& \citet{Sag2004}, Chaves \& \citet{Sag2008}, etc.), qui postule la présence (sous une forme ou autre) d'un verbe dans le second conjoint, est inadéquate, tout comme pour les constructions à gapping observées dans les sections précédentes. Contrairement à ce qui est prédit par le principe de récupérabilité de l'ellipse (cf. \citet{Chomsky1964}), on n'arrive pas toujours à restituer une tête verbale dans la coordination de séquences à droite d'un verbe. Les mêmes arguments qu'on a discutés pour le gapping avec le verbe en position non-initiale valent ici. 

La reconstruction du matériel supposé présent dans le second conjoint ne donne pas toujours lieu à une phrase grammaticale. Ainsi, certains connecteurs lexicalisés comme \textit{ca (şi)} `comme'\textit{, precum şi} `ainsi que' ne peuvent jamais se combiner avec un verbe fini.


\begin{enumerate}
\item a  S-a băgat şi Ion în discuție \textbf{ca} (*s-a băgat) musca (*s-a băgat) în lapte.


\end{enumerate}
\textsc{cl.refl-aux} introduit aussi Ion en discussion comme~la-mouche en lait

{\itshape
Ion s'est introduit dans la discussion comme la mouche dans le lait}

  b  I-am dat lui Ion un măr, \textbf{precum şi} (*i-am dat) Mariei (*i-am dat) o pară.

    \textsc{cl.dat-aux} donné Ion.\textsc{dat} une pomme, ainsi que Maria.\textsc{dat} une poire

{\itshape
J'ai donné à Ion une pomme, ainsi qu'à Maria une poire}

Le même contraste est observé avec la négation de constituant \textit{nu} en \REF{ex:4:260}a qui se comporte différemment de la négation propositionnelle \textit{nu} (cf. \citet{Barbu2004}), bien qu'elles soient homonymes en roumain. Contrairement à la négation propositionnelle dans le contexte verbal, la négation de constituant reçoit toujours une accentuation prosodique, ce qui explique les différences observées en \REF{ex:4:260}. En français et anglais, ce contraste est plus évident : des adverbes comme fr. \textit{(non) pas} \REF{ex:4:261} ou angl. \textit{not} \REF{ex:4:262} peuvent introduire une séquence, alors qu'ils sont exclus avec une forme verbale finie (cf. Mouret (2006, 2007, 2008) pour le français, et Culicover \& \citet{Jackendoff2005} pour l'anglais).


\begin{enumerate}
\item \label{bkm:Ref290289547}a  I-am dat MaRIei o pară şi \{\textbf{NU / nicideCUM}\} (*i-am dat) lui Ion un măr.


\end{enumerate}
\textsc{cl.dat-aux} donné Maria.\textsc{dat} une poire et \{non pas / pas du tout\} Ion.\textsc{dat} une pomme

{\itshape
J'ai donné à Maria une poire, et non pas à Ion une pomme}

  b  ??I-am dat Mariei o pară şi \textbf{nu} i-am dat lui Ion un măr.

\textsc{cl.dat-aux} donné Maria.\textsc{dat} une poire et \textsc{neg cl.dat-aux} donné Ion.\textsc{dat} une pomme

{\itshape
J'ai donné à Maria une poire et je n'ai pas donné à Ion une pomme}


\begin{enumerate}
\item \label{bkm:Ref290289841}Paul offrira un disque à Marie et \{\textbf{non / non pas}\} (*offrira) un livre à Jean. 

\item \label{bkm:Ref290289878}Paul gave a record to Mary and \textbf{not} (*gave) a book to Bill.


\end{enumerate}
La reconstruction syntaxique d'un verbe est problématique aussi dans les exemples avec {\guillemotleft}~accord cumulatif~{\guillemotright} dans le listage de paires \REF{ex:4:263}, lorsque le verbe initial reçoit un accord au pluriel, et ce quelle que soit la valeur de nombre de chacun des sujets postverbaux (ici, le premier sujet postverbal est au singulier). 


\begin{enumerate}
\item \label{bkm:Ref290291007}In prima zi \emph{\textup{se iau}}\emph{\textup{: o picătură}} dimineața, două picături la prânz, trei picături seara.


\end{enumerate}
dans le-premier jour \textsc{cl.refl} prennent : une goutte le-matin, deux gouttes à midi, trois gouttes le-soir

{\itshape
Le premier jour, on doit prendre une goutte le matin, deux gouttes à midi, trois gouttes le soir}

Si on assume une reconstruction syntaxique {\guillemotleft}~parallèle~{\guillemotright}, c.-à-d. on reconstruit un verbe dans le second conjont au même endroit que dans le premier conjoint, on n'arrive pas facilement à rendre compte de la distribution idiosyncrasique imposée par la conjonction \textit{iar}, qui ne permet pas une forme verbale finie ayant la fonction tête dans la position initiale du deuxième conjoint en \REF{ex:4:264} (cf. discussion dans la section \ref{sec:2.9.4} du chapitre 2).


\begin{enumerate}
\item \label{bkm:Ref302393109}a  Vreau să vină Ion azi, \textbf{iar} (*să vină) Maria (să vină) mâine.


\end{enumerate}
veux.\textsc{1sg} \textsc{mrq} vienne Ion aujourd'hui, et Maria demain

{\itshape
Je veux que Ion vienne aujourd'hui, et Maria demain}

  b  I-am dat Mariei o carte, \textbf{iar} (*i-am dat) lui Ion (i-am dat) un CD.

    \textsc{cl.dat-aux} donné Maria.\textsc{dat} un livre, et Ion.\textsc{dat} un CD

{\itshape
J'ai donné à Maria un livre, et à Ion un CD}

Un exemple plus délicat est \REF{ex:4:265} où, en dehors du fait que le matériel manquant ne peut être reconstruit qu'en position médiane (à cause des contraintes imposées par \textit{iar} sur le placement du verbe tête dans une phrase), on observe une asymétrie liée à la catégorie syntaxique, au cas et à la fonction syntaxique de la paire {\textless}\textit{pisicii, la pui}{\textgreater} : syntagme nominal vs. syntagme prépositionnel, génitif vs. accusatif avec préposition, complément du nom déverbal vs. ajout phrastique.


\begin{enumerate}
\item \label{bkm:Ref302393291}\textstylestandard{Se recomandă deparazitarea}\textstylestandard{} \textstylestandard{\textbf{pisicii}}\textstylestandard{ de patru ori pe an, iar} \textstylestandard{\textbf{la pui}}\textstylestandard{ (se recomandă deparazitarea) o dată pe lună când este între şase luni şi un an.}


\end{enumerate}
\textstylestandard{\textit{Il est recommandé de déparasiter le chat quatre fois par an, et le chaton une fois par mois quand il est entre six mois et un an}}

Généralement, les approches syntaxiques de l'ellipse postulent un homomorphisme syntaxe-sémantique, c.-à-d. les relations de portée sémantiques dérivent directement de la structure syntaxique. Ce genre d'analyse rencontre des difficultés quant à l'interprétation de certains éléments ou la portée de certains opérateurs sémantiques sur la coordination de séquences. Ainsi, si on postule la reconstruction du matériel manquant dans le deuxième conjoint, on n'arrive pas à dériver la lecture interne ({\guillemotleft}~référentiellement dépendante~{\guillemotright}, cf. Laca \& \citet{Tasmowski2001}) des adjectifs relationnels \textit{acelaşi} `même' et \textit{diferit} `différent'\footnote{Ces adjectifs présentent aussi une lecture anaphorique ou externe, dans laquelle le second terme de la relation d'identité ou de la non-identité est à récupérer dans le contexte gauche, c.-à-d. il a été déjà mentionné dans le discours. Cette lecture ne m'intéresse pas ici.}, dans laquelle les arguments de la relation d'identité ou de la non-identité se trouvent dans la phrase même \REF{ex:4:266}-\REF{ex:4:267}. Cette lecture interne va de pair avec une interprétation à la fois distributive et réciproque (cf. Van \citet{Peteghem2002}), les entités en question fonctionnant comme les arguments d'un même prédicat (dyadique) à sens réciproque. Afin d'obtenir cette lecture interne, les syntagmes nominaux modifiés par ces adjectifs relationnels (p.ex. \textit{acelaşi articol} `le même article' ou \textit{un cadou diferit} `un cadeau différent') doivent pouvoir s'appliquer à une éventualité plurielle (cf. Carlson 1987)). S'ils se combinent avec une éventualité singulière, cette lecture interne devient impossible, ce qui explique l'inacceptabilité des exemples \REF{ex:4:266}b-\REF{ex:4:267}b, où on reconstruit le verbe avec son complément manquant dans le deuxième conjoint.


\begin{enumerate}
\item \label{bkm:Ref290305774}a  Am prezentat \textbf{acelaşi} articol ieri la curs şi azi la seminar.


\end{enumerate}
{\itshape
J'ai présenté le même article hier en cours et aujourd'hui en séminaire}

  b  \#Am prezentat \textbf{acelaşi} articol ieri la curs şi am prezentat \textbf{acelaşi} articol azi la seminar.

{\itshape
J'ai présenté le même article hier en cours et j'ai présenté le même article aujourd'hui en séminaire}


\begin{enumerate}
\item \label{bkm:Ref290305778}a  Voi cumpăra un cadou \textbf{diferit} pentru Ion de Crăciun şi pentru Maria de Paşte.


\end{enumerate}
{\itshape
Je vais acheter un cadeau différent à Ion pour le Noël et à Maria pour Pâques}

  b  \#Voi cumpăra un cadou \textbf{diferit} pentru Ion de Crăciun şi voi cumpăra un cadou \textbf{diferit} pentru Maria de Paşte.

{\itshape
Je vais acheter un cadeau différent à Ion pour le Noël et je vais acheter un cadeau différent à Maria pour Pâques}

Toujours au niveau sémantique, on remarque des non-équivalences entre les exemples {\guillemotleft}~elliptiques~{\guillemotright} et les exemples avec reconstruction syntaxique concernant la portée de la conjonction \textit{şi} `and' \REF{ex:4:268} ou de la disjonction \textit{sau} `ou' \REF{ex:4:269} par rapport à la négation accompagnant le verbe initial (voir Huddleston \& \citet{Pullum2002} pour des données similaires en anglais). Ainsi, dans les contextes {\guillemotleft}~elliptiques~{\guillemotright} la conjonction \textit{şi} et la disjonction \textit{sau} ont une portée étroite par rapport à la négation verbale, qui a portée large, alors que dans les versions reconstruites, la conjonction et la disjonction ont une portée large.  


\begin{enumerate}
\item \label{bkm:Ref302394036}a  Nu pot să-i dau lui Ion o bicicletă \textbf{şi} Mariei doar un stilou.


\end{enumerate}
{\itshape
Je ne peux pas donner à Ion une bicyclette et à Maria seulement un stylo}

${\neq}$  b  Nu pot să-i dau lui Ion o bicicletă \textbf{şi} nu pot să-i dau Mariei doar un stilou.

\textit{Je ne peux pas donner à Ion une bicyclette et je ne peux pas donner à Maria seulement un stylo} 


\begin{enumerate}
\item \label{bkm:Ref302394055}a  Nu i-am dezvăluit nimic lui Paul despre Maria \textbf{sau} Mariei despre Paul.


\end{enumerate}
{\itshape
Je n'ai rien dévoilé à Paul sur Maria ou à Maria sur Paul}

${\neq}$  b  Nu i-am dezvăluit nimic lui Paul despre Maria \textbf{sau} nu i-am dezvăluit nimic Mariei despre Paul.

\textit{Je n'ai rien dévoilé à Paul sur Maria ou je n'ai rien dévoilé à Maria sur Paul} 

Sur la base de ces arguments, on doit conclure que la reconstruction syntaxique n'opère pas non plus dans la coordination de séquences à droite d'un verbe tête, indépendamment de la conjonction utilisée dans ces coordinations.

Il reste maintenant à vérifier si l'on a des arguments empiriques pour désambiguïser ces coordinations avec le verbe en position initiale. En particulier, je veux voir si, dans ce type de configurations, il s'agit d'une coordination d'unités avec contenu propositionnel (donc, une coordination de phrases, dont une fragmentaire, comme dans les constructions à gapping) ou bien s'il s'agit d'une coordination de séquences (non-phrastiques) dans la portée syntaxique d'un prédicat. Avant d'étudier les données du roumain, je présente d'abord l'analyse proposée par Mouret (2006, 2007, 2008) pour la coordination de séquences en français.

{\bfseries
\label{bkm:Ref302392659}La coordination de séquences en français}

Dans cette section, je présente brièvement les arguments mentionnés par Mouret (2006, 2007, 2008) pour une analyse sans ellipse de la coordination de séquences en français, qui pour lui est une coordination sous-phrastique. 

L'argument majeur contre une coordination phrastique (et donc contre une structure fragmentaire du deuxième conjoint) en français est la distribution des conjonctions corrélatives (ou doubles) dans les coordinations omnisyndétiques. Si la coordination de séquences présente des conjonctions doubles comme \textit{et...et} ou \textit{ou bien...ou bien}, la conjonction initiale se place obligatoirement après le prédicat partagé (p.ex. \REF{ex:4:270}a et \REF{ex:4:271}a) et non devant celui-ci (p.ex. \REF{ex:4:270}b-\REF{ex:4:271}b), ce qui s'explique si on admet que le prédicat verbal est extérieur à la structure coordonnée. 


\begin{enumerate}
\item \label{bkm:Ref302394829}a   Paul compte apporter \textbf{et} un disque à Marie \textbf{et} un livre à Jean. 


\end{enumerate}
  b  *Paul compte \textbf{et} apporter un disque à Marie \textbf{et} un livre à Jean.  


\begin{enumerate}
\item \label{bkm:Ref302394834}a   Paul apportera \textbf{ou bien} un disque à Marie \textbf{ou bie}n un livre à Jean. 


\end{enumerate}
  b  *\textbf{Ou bien} Paul apportera un disque à Marie \textbf{ou bien} un livre à Jean. 

Un argument supplémentaire qui confirme cette hypothèse est la distribution et l'interprétation des adverbes restrictifs et additifs. Ces adverbes peuvent introduire une coordination de séquences et s'y associer sémantiquement. On observe ainsi qu'en \REF{ex:4:272} les adverbes \textit{seulement} et \textit{aussi} s'associent à toute la coordination de séquences et non seulement au premier conjoint. Ces phénomènes d'association sont problématiques pour les approches postulant une structure fragmentaire pour le deuxième conjoint, car on n'arrive pas à expliquer comment un adverbe peut prendre la coordination dans son ensemble comme associé sémantique s'il est enchâssé dans le premier conjoint.


\begin{enumerate}
\item \label{bkm:Ref290310729}a   Paul offrira \textbf{seulement} {\textless}un disque à Pierre et un livre à Marie{\textgreater}, alors qu'il aurait pu offrir aussi une bouteille de vin à Jean. 


\end{enumerate}
  b  Paul offrira \textbf{aussi} {\textless}un disque à Pierre et un livre à Marie{\textgreater}, alors qu'il aurait pu offrir seulement une bouteille de vin à Jean.  

Un dernier argument pour une analyse sans ellipse concerne certains phénomènes d'accord observés avec les sujets postverbaux en français. Ainsi, dans les tours narratifs, deux stratégies d'accord sont possibles lorsqu'on coordonne des séquences comportant chacune un sujet postverbal. Si la coordination est interprétée comme la conjonction de deux événements successifs (cf. la présence de l'adverbial \textit{quelques secondes plus tard}), le verbe apparaît au singulier, s'accordant indépendamment avec chacun des sujets postverbaux \REF{ex:4:273}a. En revanche, si la coordination met en jeu une relation symétrique (cf. la présence de l'adverbe \textit{simultanément}), le verbe apparaît obligatoirement au pluriel, peu importe la valeur de nombre de chacun des sujets postverbaux \REF{ex:4:273}b. Cette deuxième stratégie est problématique pour toute structure à ellipse, car on n'arrive pas à expliquer comment un verbe peut apparaître au pluriel si celui-ci appartient uniquement au premier conjoint. En revanche, cela s'explique facilement si on postule une analyse sans ellipse, avec un prédicat verbal à l'extérieur de la séquence coordonnée. 


\begin{enumerate}
\item \label{bkm:Ref290311643}a   Alors \{surgit / *surgirent\} d'un champ un renard et \textbf{quelques secondes plus tard} d'un buisson une biche. 


\end{enumerate}
  b  Alors \{*surgit / surgirent\} \textbf{simultanément} d'un champ un renard et d'un buisson une biche.  

Sur la base de ces propriétés, Mouret (2006, 2007, 2008) conclut que la coordination de séquences en français ne met en jeu aucune ellipse, son analyse se rapprochant de celles proposées en grammaires catégorielles (\citet{Dowty1988}, Steedman (1990, 2000)) ou dans d'autres cadres surfacistes (\citet{Hudson1988}, Maxwell \& \citet{Manning1996}). On aura ainsi une coordination des séquences sans tête qui seront autorisées dans la portée syntaxique d'un verbe tête (plus de détails dans la section ).

{\bfseries
\label{bkm:Ref299980366}Une double analyse en roumain}

On a vu que la reconstruction syntaxique posait des problèmes pour les coordinations de séquences en roumain. En revanche, une analyse sans ellipse se justifie facilement pour les coordinations de séquences en français, où l'on considère que la coordination opère à un niveau sous-phrastique dans ces cas. Qu'en est-il pour le roumain ?

Crucialement, on observe que le roumain permet l'emploi de la conjonction \textit{iar} `et' dans ces contextes, qu'il s'agisse d'une séquence contenant un sujet postverbal \REF{ex:4:274}a ou non \REF{ex:4:274}b. Or, \textit{iar} ne lie que des syntagmes avec contenu propositionnel, c.-à-d. des phrases. Cela semble suggérer que l'analyse proposée pour le français ne peut pas s'appliquer a priori aux coordinations avec \textit{iar}, pour lesquelles une approche similaire à celle proposée pour les constructions typiques de gapping serait plus adéquate.  


\begin{enumerate}
\item \label{bkm:Ref290312953}a  Vreau să vină Ion azi, \textbf{iar} Maria mâine.


\end{enumerate}
veux.\textsc{1sg} \textsc{mrq} vienne Ion aujourd'hui, et Maria demain

{\itshape
Je veux que Ion vienne aujourd'hui, et Maria demain}

  b  I-am dat Mariei o carte, \textbf{iar} lui Ion un CD.

    \textsc{cl.dat-aux} donné Maria.\textsc{dat} un livre, et Ion.\textsc{dat} un CD

{\itshape
J'ai donné à Maria un livre, et à Ion un CD}

En même temps, les coordinations de séquences en roumain peuvent être liées par la conjonction \textit{şi} `et', qui n'est pas contrainte quant au type de catégorie coordonnée. On s'attend donc à ce que les propriétés mentionnées par Mouret (2006, 2007, 2008) pour le français s'appliquent aux coordinations de séquences liées par la conjonction \textit{şi}, mais pas à celles coordonnées par \textit{iar}.

Cette hypothèse est confirmée par les différences de comportement qu'on observe dans les coordinations de séquences liées par \textit{iar}, par rapport à celles liées par \textit{şi}. Par la suite, j'applique une série de tests (y compris ceux proposés par Mouret pour le français) pour montrer qu'une double analyse doit être envisagée pour les coordinations de séquences en roumain.

Bien que ce ne soit pas un test décisif comme en français, je commence par discuter le placement des items corrélatifs\footnote{Je les appelle \textit{items corrélatifs} et non \textit{conjonctions doubles}, car en roumain on a, à côté des conjonctions doubles \textit{fie...fie...} `soit...soit...'\textit{, sau...sau...}, \textit{ori...ori} `ou...ou...', des adverbes corrélatifs \textit{şi...şi} `et...et...' et \textit{nici...nici...} `ni...ni...'. Pour une analyse détaillée de ces constructions, voir le chapitre 2, section \ref{sec:2.7.1}, et Bîlbîie (2008).} . On observe une différence nette entre les coordinations de séquences avec des adverbes corrélatifs (p.ex. \textit{şi...şi...} `et...et...') et les coordinations avec des conjonctions doubles (p.ex. \textit{fie...fie...} `soit...soit...') : l'adverbe corrélatif \textit{şi}\footnote{A ne pas confondre la conjonction \textit{şi} et l'adverbe corrélatif \textit{şi}, qui, malgré l'homonymie de forme, ne partagent pas les mêmes propriétés distributionnelles.} se place obligatoirement après le prédicat \REF{ex:4:275}, alors que la conjonction double \textit{fie} peut apparaître aussi devant le prédicat \REF{ex:4:276}. On ne peut pas tester le placement des items corrélatifs avec la conjonction \textit{iar}, car il n'y a pas de structure corrélative disponible. Mais on peut déjà constater qu'il y a un double comportement des coordinations omnisyndétiques, ce qui va dans le sens de mon hypothèse.


\begin{enumerate}
\item \label{bkm:Ref302401530}a  I-am dat \textbf{şi} Mariei o pară, \{şi / dar\} \textbf{şi} lui Ion un măr.


\end{enumerate}
\textsc{cl.dat-aux} donné \textsc{adv} Maria.\textsc{dat} une poire, \{et / mais\} \textsc{adv} Ion.\textsc{dat} une pomme

{\itshape
J'ai donné et à Maria une poire, et à Ion une pomme}

  b  *\textbf{Și} i-am dat Mariei o pară, \{şi / dar\} \textbf{şi} lui Ion un măr.

    \textsc{adv cl.dat-aux} donné Maria.\textsc{dat} une poire, \{et / mais\} \textsc{adv} Ion.\textsc{dat} une pomme

{\itshape
J'ai donné et à Maria une poire, et à Ion une pomme} 


\begin{enumerate}
\item \label{bkm:Ref302401557}a  Vine \textbf{fie} Ion azi, \textbf{fie} Maria mâine.


\end{enumerate}
vient \textsc{conj} Ion aujourd'hui, \textsc{conj} Maria demain

{\itshape
Soit Ion vient aujourd'hui, soit Maria demain}

  b  \%\textbf{Fie} vine Ion azi, \textbf{fie} Maria mâine.

    \textsc{conj} vient Ion aujourd'hui, \textsc{conj} Maria demain

{\itshape
Soit Ion vient aujourd'hui, soit Maria demain} 

En revanche, le deuxième argument mentionné par Mouret (2006, 2007, 2008) est un test qui montre bien la différence entre les structures avec \textit{şi} et les structures avec \textit{iar}. Ainsi, un adverbe associatif comme les restrictifs \textit{doar / numai / decât} `seulement' prend facilement la coordination dans son ensemble comme associé sémantique si on a la conjonction \textit{şi}, alors que cela est inacceptable avec la conjonction \textit{iar}.


\begin{enumerate}
\item a  Ion \{îi / le\} va da \textbf{doar} [Mariei un stilou \{\textbf{şi / \#iar}\} Elenei o carte], deşi ar fi putut să-i dea şi Danei un caiet.


\end{enumerate}
Ion \{\textsc{cl.sg / cl.pl\}} va donner seulement Maria.\textsc{dat} un stylo, et Elena\textsc{.dat} un livre, même s'il aurait pu donner aussi Dana\textsc{.dat} un cahier

{\itshape
Ion offrira seulement à Maria un stylo et à Elena un livre, bien qu'il ait pu offrir à Dana aussi un cahier}

  b  Voi merge nu\textbf{ numai} [azi la film \{\textbf{şi / \#iar}\} mâine la teatru], ci şi vineri la operă.

{\itshape
Je vais aller non seulement aujourd'hui au cinéma et demain au théâtre, mais aussi vendredi à l'opéra} 

  c  Nu i-am dat \textbf{decât} [Mariei o carte \{\textbf{şi / \#iar}\} lui Ion un stilou], deşi aş fi putut să-i dau şi Danei un caiet.

{\itshape
Je n'ai offert qu'à Maria un livre et à Ion un stylo, bien que j'aie pu offrir à Dana aussi un cahier } 

On remarque encore une différence entre les deux constructions avec \textit{şi} et respectivement \textit{iar} aussi en ce qui concerne l'accord. D'une part, on observe que la coordination de séquences avec la conjonction \textit{şi} permet le redoublement clitique soit sous sa forme au singulier, soit sous une forme au pluriel \REF{ex:4:278}a ; ainsi, l'emploi du clitique pronominal à l'accusatif pluriel (p.ex. \textit{le} `les') redoublant deux compléments nominaux au singulier indique que les compléments redoublés dans les séquences sont sous la portée syntaxique du prédicat verbal qui contient le clitique pronominal au pluriel. En revanche, la coordination de séquences avec \textit{iar} ne permet que l'emploi d'un clitique pronominal singulier dans ces contextes \REF{ex:4:278}b. 


\begin{enumerate}
\item \label{bkm:Ref302401718}a  Ion \{\textbf{îi / le}\} va da Mariei un stilou \textbf{şi} Elenei o carte.


\end{enumerate}
Ion \{\textsc{cl.sg / cl.pl\}} va donner Maria.\textsc{dat} un stylo et Elena.\textsc{dat} un livre

{\itshape
Ion va donner à Maria un stylo et à Elena un livre}

  b  Ion \{\textbf{îi / *le}\} va da Mariei un stilou, \textbf{iar} Elenei o carte.

    Ion \{\textsc{cl.sg / cl.pl\}} va donner Maria.\textsc{dat} un stylo, et Elena.\textsc{dat} un livre

{\itshape
Ion va donner à Maria un stylo et à Elena un livre} 

Une autre différence liée à l'accord concerne le placement postverbal des sujets. On observe deux stratégies d'accord : (i) soit le verbe s'accorde de manière indépendante avec le sujet de chaque séquence \REF{ex:4:279}, l'interprétation étant celle d'une conjonction de deux événements indépendants, et dans ce cas, les deux conjonctions sont possibles ; (ii) soit le verbe reçoit l'accord au pluriel \REF{ex:4:280}, l'interprétation étant plutôt celle d'un événement complexe, et dans ce cas,~les locuteurs préfèrent la conjonction \textit{şi} ou la juxtaposition, à la place de \textit{iar}. 


\begin{enumerate}
\item \label{bkm:Ref290319894}a  La auzul împuşcăturii, \{\textbf{a} / *\textbf{au}\} țâşnit dintr-un tufiş un iepure, \{\textbf{şi / iar}\} \textbf{câteva secunde mai târziu}, dintr-o scorbură o veveriță.


\end{enumerate}
au bruit du coup-de-feu, \{\textsc{aux.sg / aux.pl\}} surgi d'un buisson un lapin, et quelques secondes plus tard, d'une grotte un écureuil

{\itshape
Au bruit du coup de feu, surgit d'un buisson un lapin et quelques secondes plus tard, d'une grotte un écureuil}

  b  \{\textbf{Ai}\textbf{ / *ați}\} intrat \textbf{mai întâi} tu pe banda întâi, (\textbf{şi / iar}) \textbf{apoi} el pe banda a doua.

\{\textsc{aux.sg / aux.pl\}} d'abord toi sur la-voie première, (et) ensuite lui sur la-voie deuxième

{\itshape
D'abord tu es entré sur la première voie, ensuite lui sur la deuxième voie} 


\begin{enumerate}
\item \label{bkm:Ref290319923}a  La auzul împuşcăturii, \{*\textbf{a} / \textbf{au}\textbf{\}} țâşnit \textbf{simultan} dintr-un tufiş un iepure \{\textbf{şi / \#iar}\} dintr-o scorbură o veveriță.


\end{enumerate}
au bruit du coup-de-feu, \{\textsc{aux.sg / aux.pl\}} surgi simultanément d'un buisson un lapin et d'une grotte un écureuil

{\itshape
Au bruit du coup de feu, surgirent simultanément d'un buisson un lapin et d'une grotte un écureuil}

  b  \{*\textbf{Ai /} \textbf{ați}\} intrat \textbf{simultan} tu pe banda întâi, (\textbf{şi / \#iar}) el pe banda a doua.

\{\textsc{aux.sg / aux.pl\}} simultanément toi sur la-voie première, (et) lui sur la-voie deuxième

{\itshape
Vous y êtes entrés simultanément : toi sur la première voie, lui sur la deuxième voie} 

Un fait qui rapproche beaucoup les coordinations de séquences des coordinations à gapping est la contrainte de constituance majeure (discutée dans la section ), c.-à-d. seuls des dépendants d'une tête verbale (racine ou enchâssée) sont légitimés dans la séquence à droite de la conjonction. Ainsi, les exemples en \REF{ex:4:281} sont inacceptables, car un des constituants dans chaque séquence dépend d'une tête non-verbale (en \REF{ex:4:281}, le premier élément de chaque séquence est~le dépendant d'une tête nominale non-prédicative). En revanche, la coordination de deux séquences composées chacune d'un complément du verbe enchâssé et d'un complément du verbe matrice, est acceptable en roumain, au moins avec la conjonction \textit{iar}, comme on l'observe en \REF{ex:4:282}a-b (contrairement à ce que Mouret (2007, 2008) constate pour le français) ; de même, pour les coordinations de séquences dépendant d'une tête non-verbale, mais prédicative dans les structures à prédicat complexe \REF{ex:4:283}a-b. En même temps, pour toutes les occurrences des coordinations de séquences en dehors du domaine verbal, on a une préférence pour la conjonction \textit{şi} plutôt que pour la conjonction \textit{iar} \REF{ex:4:284}a. Cela se vérifie dans l'exemple attesté \REF{ex:4:284}b, où on a deux coordinations de séquences sous la portée syntaxique du verbe \textit{având} `ayant' et respectivement sous la portée syntaxique de la préposition \textit{cu} `avec' : on observe que dans le premier cas, le locuteur a utilisé la conjonction \textit{iar}, tandis que dans le deuxième cas, le locuteur a utilisé la conjonction \textit{şi}.


\begin{enumerate}
\item \label{bkm:Ref290322901}a  *Am cumpărat mere roşii azi \{şi / iar\} verzi ieri.


\end{enumerate}
{\itshape
J'ai acheté des pommes rouges aujourd'hui et vertes hier}

  b  *Maria i-a dat lucrurile de fetiță Ioanei \{şi / iar\} de băiețel lui Dan

{\itshape
Maria a donné les vêtements de fille à Ioana et de garçon à Dan}

  c  *Paul dezaprobă propunerea Mariei de a ieşi în parc şi Ioanei de a merge la film.

    \textit{Paul désapprouve la proposition de Maria de sortir dans le parc et de Ioana d'aller au cinéma}


\begin{enumerate}
\item \label{bkm:Ref290323140}a  Paul i-a recomandat Mariei să meargă la mare, iar Danei la munte.


\end{enumerate}
{\itshape
Paul a recommandé à Maria d'aller à la mer, et à Dana à la montagne}

  b  Ea a oprit păsările cerului să se apropie de ei în timpul zilei, şi fiarele câmpului în timpul nopții.

{\itshape
Elle a empêché les oiseaux du ciel de s'approcher d'eux pendant le jour, et les bêtes des champs pendant la nuit} 


\begin{enumerate}
\item \label{bkm:Ref302402697}a  Rezultatele din județul Cluj sunt inferioare mediei din Iaşi cu 10 la sută, iar mediei din Bucureşti cu 15 la sută.


\end{enumerate}
{\itshape
Les résultats du département de Cluj sont inférieurs à la moyenne de Iaşi de 10 pourcent, et à la moyenne de Bucarest de 15 pourcent} 

  b  Este un tipar sintactic frecvent realizat, iar, semantic, destul de eterogen.

{\itshape
C'est un patron syntaxique fréquemment réalisé, et sémantique(ment) assez hétérogène} 


\begin{enumerate}
\item \label{bkm:Ref300049706}a  Cu Ion director \{şi / ?iar\} Maria secretară, firma nu va merge niciodată bine.


\end{enumerate}
{\itshape
Avec Ion comme directeur et Maria comme secrétaire, l'entreprise n'ira jamais mieux}

  b  Există limbi {\guillemotleft}~head first~{\guillemotright}, având deci capul de grup pe prima poziție, \textbf{iar} determinanții postpuşi, şi limbi {\guillemotleft}~head last~{\guillemotright}, cu regentul pe ultima poziție \textbf{şi} determinanții antepuşi.

{\itshape
Il y a des langues {\guillemotleft}~head first~{\guillemotright}, ayant donc la tête de groupe en première position, et les déterminants postposés, et des langues {\guillemotleft}~head last~{\guillemotright}, avec le régent en position finale et les déterminants antéposés} 

Les coordinations de séquences avec \textit{şi} et avec \textit{iar} se distinguent aussi prosodiquement. Les conjoints coordonnés par \textit{iar} présentent une prosodie incidente (marquée par une pause à l'oral et obligatoirement par une virgule à l'écrit), c.-à-d. chaque conjoint constitue une unité prosodique autonome \REF{ex:4:285}a. Les conjoints coordonnés par \textit{şi}, en dehors d'une intonation particulière, présentent habituellement plutôt une prosodie intégrée des conjoints, c.-à-d. les conjoints forment une seule unité prosodique \REF{ex:4:285}b.


\begin{enumerate}
\item \label{bkm:Ref302402908}a  Am fost ieri la film, {\textbar} \textbf{iar} azi la teatru.


\end{enumerate}
{\itshape
J'ai été hier au cinéma, et aujourd'hui au théâtre}

  b  Am fost ieri la film ({\textbar}) \textbf{şi} azi la teatru.

{\itshape
J'ai été hier au cinéma, et aujourd'hui au théâtre}

Une autre différence entre les deux constructions concerne le nombre d'éléments dans le deuxième conjoint. Pour les coordinations à gapping, on avait postulé le double contraste comme une contrainte sémantique majeure~(section ) ; cela veut dire que le conjoint fragmentaire doit contenir au moins deux constituants qui seront mis en contraste avec des éléments dans le premier conjoint (avec l'observation qu'un élément dans le premier conjoint peut être implicite). On observe la même contrainte avec les coordinations de séquences liées par \textit{iar}, mais pas nécessairement dans les contextes avec la conjonction \textit{şi}. Ainsi, la séquence introduite par la conjonction \textit{şi} peut contenir un seul constituant immédiat \REF{ex:4:286}a, alors que la séquence introduite par \textit{iar} doit contenir au moins deux constituants immédiats (p.ex. \REF{ex:4:286}b et \REF{ex:4:287}). 


\begin{enumerate}
\item \label{bkm:Ref302403853}a  Am cumpărat o jucărie pentru fata mea \{\textbf{şi / *iar}\} un ziar.


\end{enumerate}
{\itshape
J'ai acheté un jouet pour ma fille et un journal}

  b  Am cumpărat un ziar, \{şi\textbf{ / iar}\} pentru fata mea o jucărie.

{\itshape
J'ai acheté un journal, et pour ma fille un jouet}


\begin{enumerate}
\item \label{bkm:Ref302404044}a  Nu am nicio legătură cu biserica, sunt [un simplu credincios], iar *(\textbf{de meserie}) [şofer].


\end{enumerate}
{\itshape
Je n'ai aucun lien avec l'église, je suis un simple croyant, et quant à mon métier, chauffeur}

  b  Ioana mănâncă [un măr], iar *(\textbf{apoi}) [o pară].

{\itshape
Ioana mange une pomme, et ensuite une poire}

\textstyleapplestylespan{Cela s'explique par la contrainte discursive imposée par la conjonction} \textstyleapplestylespan{\textit{iar}}\textstyleapplestylespan{,}\textstyleapplestylespan{ }\textstyleapplestylespan{qui ne peut être immédiatement suivie que par un topique contrastif et non par un focus informationnel (voir section \ref{sec:2.9.3} du chapitre 2). Ce topique contrastif est distingué prosodiquement si} \textstyleapplestylespan{\textit{iar} }\textstyleapplestylespan{est présent, c.-à-d. il forme un syntagme intonationnel à lui tout seul, alors que ce n'est pas le cas avec la conjonction} \textstyleapplestylespan{\textit{şi}}\textstyleapplestylespan{. Cela explique pourquoi la conjonction} \textstyleapplestylespan{\textit{iar} }\textstyleapplestylespan{est préférée quand l'ordre des éléments dans les séquences n'est pas le même : la distinction discursive et prosodique du premier élément suivant} \textstyleapplestylespan{\textit{iar} }\textstyleapplestylespan{permet la mise en parallèle nécessaire pour établir le contraste.}


\begin{enumerate}
\item \textstyleapplestylespan{``}\textstyleapplestylespan{Sunt învinuiți}\textstyleapplestylespan{ pentru ucidere din culpă doamna Florentina Cîrstea şi spitalul, iar [pentru neglijență în serviciu], managerul, şeful secției, un electrician şi directorul administrativ'', a declarat avocatul Florian Șurghie}.


\end{enumerate}
{\itshape
{\guillemotleft}~Sont accusés pour homicide par faute madame Florentina Cârstea et l'hôpital, et pour négligence au travail, le manager, le chef du département, un électricien et le directeur administratif~{\guillemotright}, a déclaré l'avocat Florian Șurghie} 


\begin{enumerate}
\item a  Am cumpărat un ziar, \{\textbf{şi / *iar}\} o jucărie pentru fata mea.


\end{enumerate}
{\itshape
J'ai acheté un journal, et un jouet pour ma fille}

  b  Am cumpărat un ziar, \{şi\textbf{ / iar}\} pentru fata mea o jucărie.

{\itshape
J'ai acheté un journal (pour moi), et pour ma fille un jouet}

Sur la base de toutes ces différences empiriques, on doit distinguer entre les coordinations de séquences liées par \textit{iar} et les coordinations de séquences liées par la conjonction \textit{şi}\footnote{A priori, les séquences juxtaposées ou encore les séquences coordonnées par une autre conjonction que la conjonction \textit{iar} ont un comportement similaire à celles coordonnées par la conjonction \textit{şi}.} . Les coordinations avec \textit{iar} mettent en jeu plutôt une coordination de deux phrases, dont une fragmentaire, alors que celles liées par \textit{şi} restent ambiguës, étant a priori compatibles avec les deux analyses : soit une coordination de phrases dont une fragmentaire, soit une coordination sous-phrastique de séquences sous la portée syntaxique d'un prédicat. Par conséquent, l'analyse proposée par Mouret (2006, 2007, 2008) pour le français ne peut pas être étendue aux coordinations avec \textit{iar}, mais peut être une des solutions possibles pour les coordinations avec \textit{şi}. Par conséquent, les coordinations de séquences avec \textit{iar} reçoivent une analyse commune avec les constructions à gapping discutées dans la section \ref{sec:4.5}. 

Pour conclure, les coordinations {\guillemotleft}~elliptiques~{\guillemotright} avec \textit{iar} reçoivent une seule analyse, indépendamment de la position du verbe. En particulier, la séquence introduite par \textit{iar}, qui contient toujours au moins deux constituants immédiats, est un fragment ayant un contenu propositionnel. En revanche, les coordinations {\guillemotleft}~elliptiques~{\guillemotright} avec la conjonction \textit{şi} restent ambiguës entre une structure à ellipse fragmentaire (coordination phrastique) et une structure sans ellipse (coordination sous-phrastique) dans les configurations à verbe initial en roumain.

D'ailleurs, le français aussi présente des cas ambigüs, qui se prêtent à deux analyses (cf. François Mouret (c.p.)). Ainsi, François Mouret observe une différence entre les séquences \textit{un livre à Jean} vs. \textit{à Jean un livre}. Si la séquence \textit{un livre à Jean} semble privilégier une coordination de séquences en \REF{ex:4:290}b, la séquence \textit{à Jean un livre} est ambiguë en \REF{ex:4:290}c.


\begin{enumerate}
\item \label{bkm:Ref290364017}a   Paul apportera un disque ou bien Jean un livre. (gapping)


\end{enumerate}
  b  Paul apportera un disque à Marie ou un livre à Jean. (Coordination de Séquences)

  c  Paul apportera à Marie un disque ou bien à Jean un livre. (analyse ambiguë)

Cette différence entre les deux séquences est justifiée par le comportement différent qu'elles ont par rapport au placement des conjonctions doubles. Ainsi, la séquence \textit{un livre à Jean} n'est pas compatible avec le placement initial, en début de phrase, de la conjonction corrélative \textit{ou bien} \REF{ex:4:291}a-b, contrairement à ce qu'on observe avec la séquence \textit{à Jean un livre} \REF{ex:4:292}a-b. Cette différence peut être mise en relation avec la possibilité d'antéposer ou non le premier constituant de la séquence~à droite de la conjonction : le syntagme nominal \textit{un livre} ne peut jamais être antéposé en début de phrase \REF{ex:4:291}c, alors que le syntagme prépositionnel \textit{à Jean} le permet \REF{ex:4:292}c.  


\begin{enumerate}
\item \label{bkm:Ref290364941}a   Paul apportera \textbf{ou bien} un disque à Marie \textbf{ou bien} un livre à Jean. 


\end{enumerate}
  b  *\textbf{Ou bien} Paul apportera un disque à Marie \textbf{ou bien} un livre à Jean.

  c  *Un livre, Paul apportera à Jean.


\begin{enumerate}
\item \label{bkm:Ref290364944}a   Paul apportera \textbf{ou bien} à Marie un disque \textbf{ou bien} à Jean un livre. 


\end{enumerate}
  b  \textbf{Ou bien} Paul apportera à Marie un disque \textbf{ou bien} à Jean un livre. 

  c  A Jean, Paul apportera un livre. 

Il y a donc des propriétés syntaxiques qui séparent nettement la coordination phrastique de la coordination sous-phrastique dans les contextes {\guillemotleft}~elliptiques~{\guillemotright} ambigüs, p.ex. contraintes sur les constituants de la séquence, contraintes sur la forme du verbe, distribution des conjonctions doubles, distribution des adverbes associatifs, etc. Une étude empirique s'avère nécessaire avant de postuler l'existence d'un certain type d'ellipse dans une langue.

{\bfseries
\label{bkm:Ref302392744}\label{bkm:Ref290312423}Théorie des coordinations de clusters en HPSG}

Dans la section \ref{sec:4.5}, on a proposé une analyse constructionnelle pour les coordinations à gapping dans lesquelles le verbe antécédent se trouvait en position médiane. Dans la section , j'ai montré qu'en roumain les coordinations de séquences avec la conjonction \textit{iar} sont incontestablement des coordinations de phrases, indépendamment de la position du verbe et indépendamment de la fonction syntaxique des éléments coordonnés. Par conséquent, le problème d'ambiguïté (discuté au début de ce chapitre, dans la section \ref{sec:4.2.2}), qui a priori se pose pour les coordinations de séquences ayant le verbe en position initiale, ne se pose pas pour les coordinations avec \textit{iar}. Ainsi, toutes les configurations décrites en \REF{ex:4:293} se prêtent à une seule analyse. La première configuration \REF{ex:4:293}a, qui constitue d'ailleurs le prototype syntaxique du gapping, a été analysée en détails dans la section \ref{sec:4.5}. J'étends donc la même analyse aux autres configurations en \REF{ex:4:293}b-c-d. Une syntaxe simplifiée de la configuration \REF{ex:4:293}d, exemplifiée en \REF{ex:4:294}, est donnée en \REF{ex:4:295}.  


\begin{enumerate}
\item \label{bkm:Ref300045395}a   sujet \textsc{verbe} complément \textit{iar} sujet complément 


\end{enumerate}
  b  \textsc{verbe} sujet complément \textit{iar} sujet complément

  c  \textsc{verbe} complément sujet \textit{iar} complément sujet

  d  \textsc{verbe} complément complément \textit{iar} complément complément


\begin{enumerate}
\item   \label{bkm:Ref300066204}Ii  dau  Mariei  o  carte  \textbf{iar}  Ioanei  un  stilou.


\end{enumerate}
\textsc{cl } donne.\textsc{1sg}  Maria.\textsc{dat}  un  livre  et  Ioana.\textsc{dat}  un  stylo

{\itshape
Je donne à Marie un livre et à Ioana un stylo}


\begin{enumerate}
\item \label{bkm:Ref300045697}Syntaxe simplifiée des coordinations de séquences avec \textit{iar}


\end{enumerate}
{   [Warning: Image ignored] % Unhandled or unsupported graphics:
%\includegraphics[width=6.1173in,height=2.6181in,width=\textwidth]{fe443409cd384d3fb0f6390ffd77f513-img62.svm}
} 

En revanche, dans les configurations à verbe initial, le problème d'ambiguïté réside pour les coordinations {\guillemotleft}~elliptiques~{\guillemotright} avec la conjonction \textit{şi~}: bien qu'on puisse les analyser comme des coordinations de phrases, elles présentent (dans certains contextes) des propriétés qui nous laissent aussi la possibilité de les analyser comme des coordinations sous-phrastiques dans la portée syntaxique d'un prédicat verbal (c.-à-d. coordinations de clusters). Une représentation simplifiée de la syntaxe de ces coordinations de clusters avec la conjonction \textit{şi} est donnée en \REF{ex:4:296}. 


\begin{enumerate}
\item \label{bkm:Ref300046106}Syntaxe simplifiée des coordinations de clusters avec la conjonction \textit{şi}


\end{enumerate}
{   [Warning: Image ignored] % Unhandled or unsupported graphics:
%\includegraphics[width=5.5862in,height=2.6457in,width=\textwidth]{fe443409cd384d3fb0f6390ffd77f513-img63.svm}
} 

La première possibilité d'analyse a été largement discutée dans la section \ref{sec:4.5}. Je me concentre maintenant plutôt sur la deuxième analyse, proposée par Mouret (2006, 2007) pour les coordinations de séquences en français.

Le point commun entre l'analyse postulée pour les constructions à gapping et l'analyse que je présente ici pour les coordinations de clusters est l'emploi d'un syntagme sans tête, qu'on a appelé \textit{cluster-ph} dans la section \ref{sec:4.5}. Par conséquent, tout ce qu'on avait discuté dans la section dédiée à la théorie des clusters (c.-à-d. section \ref{sec:4.5.1.1}) s'applique aussi à l'analyse des coordinations de clusters. Je ne reprends donc ici que la description du syntagme cluster définie en \REF{ex:4:222} ; pour les autres détails, revoir la section \ref{sec:4.5.1.1}. 


\begin{enumerate}
\item Syntagme de type cluster (cf. Mouret (2006, 2007))


\end{enumerate}
  [Warning: Image ignored] % Unhandled or unsupported graphics:
%\includegraphics[width=4.2417in,height=1.6374in,width=\textwidth]{fe443409cd384d3fb0f6390ffd77f513-img64.svm}
 

Pour pouvoir utiliser le cluster en dehors des structures coordonnées, on se donne dans la grammaire la possibilité qu'un cluster ait un seul constituant immédiat (c.-à-d. un cluster unaire), ce qui est mis en évidence par le fait que la valeur de liste correspondant aux branches non-têtes est \textit{1-to-n-list} (c.-à-d. une liste composée d'au moins un élément). Une hiérarchie générale des valeurs de liste est donnée en \REF{ex:4:298}. Les sous-types proposés peuvent être utilisés pour gérer la coordination de séquences de longueurs différentes (voir exemplification dans \citet[333]{Mouret2007}).


\begin{enumerate}
\item \label{bkm:Ref300047412}\label{bkm:Ref299887457}Hiérarchie de valeurs de liste (cf. \citet{Mouret2007})


\end{enumerate}
{   [Warning: Image ignored] % Unhandled or unsupported graphics:
%\includegraphics[width=4.4728in,height=1.1453in,width=\textwidth]{fe443409cd384d3fb0f6390ffd77f513-img65}
} 

Il nous reste à montrer comment on légitime la coordination de clusters dans la portée syntaxique d'un prédicat (en l'occurrence, un verbe). Pour cela, je reprends la proposition faite par Mouret (2006, 2007) pour le français. Il propose une règle lexicale post-flexionnelle (c.-à-d. qui relie une entrée de type \textit{word} à une nouvelle entrée de type \textit{word}), qui permet à un prédicat donné d'être partiellement saturé par une coordination de clusters, plutôt que par une suite ordinaire de constituants, c'est ce que Mouret appelle une complémentation alternative des prédicats. Les règles lexicales, qui sont utilisées en HPSG pour les changements de valence, sont représentées sous forme de structures de traits (avec deux attributs  INPUT et OUTPUT, cf. Briscoe \& \citet{Copestake1999}). La règle que Mouret (2006, 2007) propose pour la légitimation des coordinations de clusters dans la dépendance d'un prédicat en français figure en \REF{ex:4:299}. 


\begin{enumerate}
\item \label{bkm:Ref300048035}Règle lexicale pour la complémentation alternative des prédicats


\end{enumerate}
  [Warning: Image ignored] % Unhandled or unsupported graphics:
%\includegraphics[width=4.3965in,height=1.9366in,width=\textwidth]{fe443409cd384d3fb0f6390ffd77f513-img66.svm}
 

La règle remplace une sous-liste non-vide de compléments L\textsubscript{2} dans la liste des compléments attendus par l'entrée lexicale INPUT par une coordination de clusters (décrite comme COORD + et ayant une valeur de liste non-vide pour le trait CLUSTER) dans la liste des compléments attendus par l'entrée lexicale OUTPUT. On exclut la récursion à l'infini par la contrainte conjointe à la règle suivant laquelle la liste L\textsubscript{2} remplacée par la coordination de clusters ne peut pas elle-même correspondre à une coordination de clusters.

L'introduction de la description des coordinations de clusters dans la liste COMPS et non dans la liste ARG-ST du prédicat rend compte, d'une part, de l'absence de cliticisation ou d'extraction des coordinations de clusters et, d'autre part, de l'absence de coordinations de clusters de niveaux différents (cf. les données en \REF{ex:4:281}). En HPSG, la liste COMPS ne contient que des synsems canoniques. Donc, elle ne contient pas de gaps (correspondant aux éléments extraits), de pronoms {\guillemotleft}~nuls~{\guillemotright} ou affixes (pronominaux ou adverbiaux). De plus, cette liste enregistre les complèments intrinsèques d'un prédicat ou ceux qu'il hérite de ses arguments, mais non les compléments plus enchâssés. 

La règle proposée en \REF{ex:4:299}, par son trait COMPS, prend en compte toute coordination de clusters en français : clusters contenant un sujet postverbal \REF{ex:4:300}a, les clusters contenant des compléments \REF{ex:4:300}b ou encore les clusters contenant un mélange de modifieurs et d'arguments \REF{ex:4:300}c. Car, en français, le sujet postverbal des constructions inaccusatives (\citet{Marandin1999}), ainsi que les modifieurs postverbaux (\citet{Mouret2007}) sont analysés comme des compléments syntaxiques. 


\begin{enumerate}
\item \label{bkm:Ref300058336}a   Alors surgit d'un champ un renard et quelques secondes plus tard d'un buisson une biche. 


\end{enumerate}
  b  Paul offrira un disque à Marie et un livre à Jean.

  c  Paul joue au tennis le lundi et au football le mardi.

Les contraintes syntaxiques imposées par le prédicat de départ à ses compléments sont préservées dans la liste CLUSTER (cf. le partage de valeur des traits CAT), ce qui prédit correctement la possibilité de coordinations de clusters dissemblables en ce qui concerne leur catégorie \REF{ex:4:301}a ou bien leur nombre \REF{ex:4:301}b. 


\begin{enumerate}
\item \label{bkm:Ref300060544}a   Les enseignants attendent des élèves [qu'ils respectent les règles de l'établissement]\textsubscript{S} et de leur proviseur [un soutien sans faille]\textsubscript{NP}. 


\end{enumerate}
  b  Paul écrira [un petit poème] et [[une lettre] [à sa mère]].    (\citet[337]{Mouret2007})

L'analyse proposée pour le français s'applique aussi aux coordinations de clusters avec la conjonction \textit{şi} en roumain, en particulier aux clusters contenant des syntagmes non-sujets.\footnote{Pour les coordinations de clusters contenant des sujets postverbaux, il reste à voir comment on peut adapter le trait COMPS afin d'intégrer aussi ces configurations en roumain.} J'illustre les conséquences de cette règle lexicale sur un exemple de coordinations de clusters contenant des compléments argumentaux \REF{ex:4:302}. Le verbe \textit{a da}\textsubscript{1} `donner' contient dans sa liste de compléments un syntagme nominal à l'accusatif et un syntagme nominal au datif. La règle lexicale donnée en \REF{ex:4:299} autorise une entrée alternative \textit{a da}\textsubscript{2}, qui permet la combinaison de ce prédicat et d'une coordination de séquences. Les deux entrées lexicales correspondant à ce verbe sont données en \REF{ex:4:303}. Le résultat de l'interaction des propriétés de ce prédicat et des contraintes qui définissent les structures coordonnées est représenté de manière simplifiée en \REF{ex:4:304}.  


\begin{enumerate}
\item   \label{bkm:Ref300066622}Ii  dau  o  carte  Mariei  şi  un  stilou  Ioanei.


\end{enumerate}
\textsc{cl } donne.\textsc{1sg}  un  livre  Maria.\textsc{dat}  et  un  stylo  Ioana.\textsc{dat}

{\itshape
Je donne un livre à Marie et un stylo à Ioana}


\begin{enumerate}
\item \label{bkm:Ref300067000}Entrées lexicales du verbe \textit{a da} `donner'


\end{enumerate}
  [Warning: Image ignored] % Unhandled or unsupported graphics:
%\includegraphics[width=3.7055in,height=0.7909in,width=\textwidth]{fe443409cd384d3fb0f6390ffd77f513-img67.svm}
 


\begin{enumerate}
\item \label{bkm:Ref302431922}Représentation simplifiée de la phrase \REF{ex:4:302}


\end{enumerate}
  [Warning: Image ignored] % Unhandled or unsupported graphics:
%\includegraphics[width=6.3102in,height=5.1217in,width=\textwidth]{fe443409cd384d3fb0f6390ffd77f513-img68.svm}
 

Comme montré par Mouret (2006, 2007), cette approche nous permet d'analyser aussi les coordinations de clusters de longueurs différentes, grâce à~la hiérarchie de valeurs de liste, donnée en \REF{ex:4:298}. Ainsi, un verbe comme \textit{a scrie}\textsubscript{1} `écrire' en \REF{ex:4:305} contient dans sa liste de compléments un syntagme nominal à l'accusatif et optionnellement un syntagme nominal au datif (le fait d'être optionnel est indiqué par \textit{0-to-1-list} dans la règle donnée en \REF{ex:4:306}). La même règle lexicale utilisée plus haut pour le verbe \textit{a da} `donner' autorise l'entrée alternative \textit{a scrie}\textsubscript{2}, qui autorise la coordination de clusters dans la dominance du prédicat.


\begin{enumerate}
\item \label{bkm:Ref300068528}a  Voi  scrie  tema  şi  o  scrisoare  mamei.


\end{enumerate}
vais\textsc{ } écrire.\textsc{1sg}  le-devoir.\textsc{acc}  et  une  lettre  la-mère.\textsc{dat}

{\itshape
Je vais écrire le devoir et une lettre à ma mère}


\begin{enumerate}
\item \label{bkm:Ref302433041}Entrées lexicales du verbe \textit{a scrie} `écrire'


\end{enumerate}
  [Warning: Image ignored] % Unhandled or unsupported graphics:
%\includegraphics[width=4.7492in,height=0.7701in,width=\textwidth]{fe443409cd384d3fb0f6390ffd77f513-img69.svm}
 


\begin{enumerate}
\item Représentation simplifiée de la phrase \REF{ex:4:305}


\end{enumerate}
  [Warning: Image ignored] % Unhandled or unsupported graphics:
%\includegraphics[width=6.9953in,height=5.9937in,width=\textwidth]{fe443409cd384d3fb0f6390ffd77f513-img70.svm}
 

Je ne me prononce pas sur le contenu de ses clusters. Le contenu de l'ensemble peut être calculé au moyen d'une fonction qui compose le contenu des parties (cf. \citet{Steedman2000}).

\subsection{Conclusion}
Dans ce chapitre, j'ai étudié les constructions à gapping, dans lesquelles une séquence de syntagmes sans tête verbale, ayant néanmoins le contenu d'une phrase, se combine avec une phrase complète qui détermine sa forme et son interprétation. La langue sur laquelle porte la description est le roumain, mais la plupart des observations s'appliquent aussi au français (là où j'ai enregistré des différences significatives entre le roumain et le français, je les ai signalées). Comme critères minimaux de définition, je retiens l'absence de la tête verbale (plus éventuellement le sujet ou d'autres dépendants verbaux), ainsi que la présence d'au moins deux éléments résiduels dans la séquence trouée. Contrairement à ce qui est souvent postulé dans la littérature, la position médiane du trou, ainsi que la présence obligatoire d'un élément résiduel sujet ne constituent pas dans cette thèse de vrais critères de définition. Bien que le gapping soit possible en dehors des structures coordonnées, il semble que les propriétés sont différentes d'une construction à l'autre, c'est pour cela que dans cette thèse je me limite au gapping dans la coordination, en laissant de côté d'autres constructions, comme les structures comparatives.

Les deux vrais critères de définition mentionnés ci-dessus permettent de distinguer le gapping du pseudo-gapping ou du stripping, mais ils ne permettent pas toujours de distinguer le gapping d'une coordination de séquences (angl. \textit{Argument Cluster Coordination}, abrégé ACC). En roumain, comme dans plusieurs langues, on a des configurations dans lesquelles on peut avoir descriptivement un flou de constructions. Les configurations qui ne sont pas ambiguës en roumain sont celles dans lesquelles le verbe antécédent se trouve en position médiane \REF{ex:4:308}a-b ou en position finale \REF{ex:4:308}c-d dans la phrase source. En revanche, les coordinations qui sont ambiguës entre une construction à gapping et une coordination de séquences sont celles dans lesquelles le verbe se trouve en position initiale \REF{ex:4:309}.\footnote{Je donne ici uniquement les distributions comportant un sujet et un complément. Mais une liste exhaustive des configurations possibles devrait inclure aussi les combinaisons complément-complément, complément-ajout, ajout-complément, sujet-ajout et ajout-sujet.} 


\begin{enumerate}
\item   \label{bkm:Ref302409749}Configurations non-ambiguës de gapping en roumain 


\end{enumerate}
  a  sujet \textsc{verbe} complément \textsc{conj} sujet complément

  b  complément \textsc{verbe} sujet \textsc{conj} complément sujet

  c  sujet complément \textsc{verbe} \textsc{conj} sujet complément

  d  complément sujet \textsc{verbe} \textsc{conj} complément sujet


\begin{enumerate}
\item   \label{bkm:Ref302410068}Configurations ambiguës en roumain 


\end{enumerate}
  a  \textsc{verbe} sujet complément \textsc{conj} sujet complément

  b  \textsc{verbe} complément sujet \textsc{conj} complément sujet

Afin d'observer les propriétés spécifiques des coordinations à gapping, je me suis concentrée d'abord sur la description des configurations non-ambiguës, en étudiant les contraintes générales pesant sur le matériel manquant et sur les éléments résiduels, ainsi que les contraintes de parallélisme.

Parmi les contraintes générales s'appliquant au matériel manquant, on observe que le trou contient nécessairement le verbe tête de la phrase (y compris l'auxiliaire) et optionnellement d'autres éléments (sujets, compléments, ajouts) ; il peut correspondre à une expression idiomatique (mais pas à une portion d'expression idiomatique) ; il ne correspond pas nécessairement à un constituant ; il peut comporter une négation (qui se prête à plusieurs interprétations). En ce qui concerne le degré d'identité qui s'établit entre le matériel antécédent et le matériel manquant, on note qu'ils doivent appartenir au même paradigme de flexion, avoir le même sens, partager les mêmes propriétés de temps, mode, voix et aspect, et, de manière générale, partager les marques de flexion inhérente. En revanche, ils peuvent différer par les marques de flexion contextuelle, les affixes qu'une forme verbale peut prendre ou encore la polarité. Parmi les contraintes générales s'appliquant aux éléments résiduels, on observe que la séquence trouée doit comporter au moins deux éléments résiduels, mis en correspondance avec des éléments parallèles dans la phrase source.~Les éléments résiduels doivent être des constituants majeurs (c.-à-d. des arguments ou ajouts d'une tête verbale -- racine ou enchâssée -- dans la phrase source). 

J'ai étudié les contraintes de parallélisme au niveau syntaxique, sémantique et discursif. On a observé que le parallélisme le plus strict opère au niveau sémantico-discursif, mais en ne négligeant pas complètement la syntaxe. Au niveau syntaxique, bien que les éléments résiduels puissent différer de leurs corrélats en ce qui concerne la catégorie syntaxique, la position ou encore leur réalisation {\guillemotleft}~de surface~{\guillemotright}, ils doivent chacun correspondre à des arguments ou ajouts possibles du prédicat manquant. Cette généralisation est identique à celle qui caractérise les coordinations de termes dissemblables (cf. la généralisation de Wasow). Au niveau sémantique, il doit y avoir au moins deux paires contrastives, avec un élément provenant de chacun des conjoints dans chacune des paires. Au niveau discursif, entre les phrases reliées on doit avoir une relation symétrique (les relations privilégiées étant le parallélisme et le contraste). Le prototype discursif dans le gapping est une réponse en liste de paires à une question multiple implicite. Du point de vue de la structure informationnelle, il semble qu'au moins les coordinations à gapping avec la conjonction \textit{iar} doivent contenir minimalement une paire contrastive avec des topiques et une paire contrastive avec des focus. 

Ensuite, j'ai présenté les analyses proposées dans la littérature pour rendre compte des constructions à gapping. Elles se regroupent en trois approches majeures : (i) ellipse syntaxique, avec reconstruction \textit{in situ} du matériel manquant, cf. \REF{ex:4:310}a ; (ii) ellipse sémantique, avec légitimation indirecte, cf. \REF{ex:4:310}b, et (iii) pas d'ellipse, avec mouvement du matériel {\guillemotleft}~manquant~{\guillemotright}, cf. \REF{ex:4:310}c.


\begin{enumerate}
\item \label{bkm:Ref302414522}a  Ellipse syntaxique         b  Ellipse sémantique 


\end{enumerate}
    [Warning: Image ignored] % Unhandled or unsupported graphics:
%\includegraphics[width=2.7008in,height=1.5374in,width=\textwidth]{fe443409cd384d3fb0f6390ffd77f513-img71.svm}
 


\begin{table}
 [Warning: Image ignored] % Unhandled or unsupported graphics:
%\includegraphics[width=2.698in,height=1.0835in,width=\textwidth]{fe443409cd384d3fb0f6390ffd77f513-img72.svm}


\caption{}
%\label{}
\end{table}

          c  Pas d'ellipse

{   [Warning: Image ignored] % Unhandled or unsupported graphics:
%\includegraphics[width=2.7in,height=2.0638in,width=\textwidth]{fe443409cd384d3fb0f6390ffd77f513-img73.svm}
} 

J'ai donné des arguments empiriques en faveur d'une approche constructionnelle des coordinations à gapping (avec une reconstruction sémantique de l'ellipse, cf. \REF{ex:4:310}b) et contre les approches alternatives en termes d'ellipse syntaxique ou mouvement. Ainsi, on a vu que les deux processus syntaxiques majeurs envisagés par les approches postulant la reconstruction syntaxique et/ou le mouvement, à savoir l'extraction des éléments résiduels et l'extraction du matériel manquant, ne sont pas justifiés empiriquement (cf. la violation des contraintes de localité, la portée de certains opérateurs sémantiques, la distribution des items corrélatifs dans les coordinations omnisyndétiques, l'absence d'identité stricte entre le matériel antécédent et le matériel manquant, la présence de certains items incompatibles avec une phrase finie, etc.). Par conséquent, une ellipse sémantique (et non syntaxique) doit être envisagée pour les constructions à gapping. Dans cette perspective, la construction à gapping dans son ensemble est une coordination entre une (ou plusieurs) phrase finie non-elliptique, complète, et une (ou plusieurs) phrase fragmentaire non-finie. On a vu comment cette analyse peut être formalisée dans un cadre constructionnel, comme HPSG dans ses versions plus récentes. La phrase fragmentaire hérite à la fois d'un type de \textit{fragment} (utilisé aussi pour les questions et les réponses courtes) pour ce qui est de son interprétation sémantique et de ses contraintes contextuelles, et d'un type de \textit{cluster} (utilisé aussi pour la coordination de séquences dans les constructions ACC) pour ce qui est de sa constituance (c.-à-d. sa structure interne de syntagmes non-reliés en termes de fonctions syntaxiques).

Après avoir décrit et analysé les configurations non-ambiguës de gapping, j'ai consacré une section à l'étude des configurations ambiguës données en \REF{ex:4:309}, dans lesquelles le verbe se trouve en position initiale, configurations qui se prêtent a priori à deux analyses en fonction du niveau auquel opère la coordination : coordination phrastique ou bien coordination sous-phrastique dans la portée syntaxique d'un prédicat verbal. Après avoir invalidé l'hypothèse d'une reconstruction syntaxique, j'ai donné des arguments empiriques pour distinguer les coordinations de séquences avec la conjonction \textit{iar} et les coordinations de séquences avec la conjonction \textit{şi}. Les configurations a priori ambiguës se désambiguïsent si les séquences sont coordonnées par la conjonction \textit{iar}. Dans ce cas, il s'agit toujours d'une coordination de phrases, dont une fragmentaire, ce qui nous permet d'aligner ces coordinations avec \textit{iar} sur les cas standard de gapping discutés dans les sections précédentes. En revanche, les configurations avec la conjonction \textit{şi} restent ambiguës~entre les deux structures, à savoir une coordination phrastique (comme pour les constructions à gapping) ou bien une coordination sous-phrastique (comme pour les coordinations de clusters en français), cf. la possibilité d'avoir une association large vs. étroite de certains adverbes, un accord au pluriel vs. un accord au singulier sur le verbe initial, etc. Ainsi, dans les coordinations avec \textit{şi} où l'on a une association large des adverbes associatifs ou bien un accord au pluriel, l'analyse qui semble la plus adéquate est celle d'une coordination de clusters (donc, on a une construction de type ACC). Si, en revanche, dans les coordinations avec \textit{şi}, on a une association étroite des adverbes associatifs ou un accord au singulier, l'analyse qui semble la plus adéquate est celle d'une coordination de phrases (on a donc une construction à gapping). L'étude des configurations ambiguës en roumain nous montre qu'on peut avoir deux structures différentes en fonction de la conjonction (\textit{iar} vs. \textit{şi}) et que, de plus, les deux structures sont parfois nécessaires pour analyser un même exemple (en particulier, pour les coordinations avec la conjonction \textit{şi}). Une étude empirique s'avère donc nécessaire avant de postuler l'existence d'un certain type d'ellipse dans une langue.

