%3

\section{Les phrases elliptiques}
\subsection{La notion générale d'ellipse}
Selon la définition saussurienne du signe linguistique, chaque unité linguistique se définit par l'association d'une forme (le signifiant) et d'un contenu (le signifié). Ainsi,~à une séquence de sons on attribue un certain sens, ce qui rend possible la communication dans les langues naturelles. Cependant, toute langue naturelle présente des situations dans lesquelles cette correspondance manque et où l'on observe une discordance entre les deux éléments constitutifs du signe linguistique : on arrive à avoir une interprétation en l'absence d'une forme phonologique. L'ellipse est par conséquent le grand défi de la dichotomie classique mentionnée ci-dessus, en ce qu'elle est \textit{significatio ex nihilo} (cf. \citet{Merchant2006}).

La propriété la plus remarquable de l'ellipse n'est pas le fait qu'une partie de la phrase n'est pas prononcée, mais le fait qu'on n'a pas de problème à interpréter le matériel qui manque. A première vue, cela semble être un défi pour le principe de compositionalité de Frege, selon lequel l'interprétation d'une expression complexe est une fonction de l'interprétation de ses parties et de la manière dont elles sont assemblées. 

On définit l'ellipse~comme une relation entre une séquence de constituants dont l'interprétation requiert plus que ce qui est donné par les mots qui la composent et une expression inférée à partir du contexte extra-linguistique ou bien présente dans le contexte linguistique, qui fournit à cette séquence le matériel dont elle a besoin pour être interprétée. 

Les critères essentiels que je retiens ici pour identifier une structure elliptique sont donc les suivants : (i) dans une structure elliptique, une partie du matériel nécessaire à l'interprétation de la structure en question manque, ce qui fait que la syntaxe est apparemment incomplète ; (ii) les éléments réalisés dans la structure elliptique doivent pouvoir être analysés comme argument, ajout ou prédicat du matériel manquant, et (ii) l'interprétation d'une structure dite elliptique est toujours obtenue contextuellement, grâce à la présence d'un antécédent (linguistique ou non-linguistique, explicite ou implicite). Sur la base de ces critères, je ne considère pas comme ellipses les formules de salutation (p.ex. \textit{Salut}, \textit{Au revoir} en français, ou \textit{Hello}, \textit{Bye} en anglais), car leur contenu descriptif n'a pas besoin d'être résolu contextuellement, ni les styles appartenant à des registres spécifiques (p.ex. télégrammes, titres, étiquettes, etc.), qui contiennent des expressions sans antécédent.\footnote{Pour une liste détaillée des formes qui ne doivent pas être analysées comme elliptiques, voir Merchant (2006, 2011).} 

La notion d'ellipse n'est pas réservée à la phrase et concerne tous les types de syntagme. Dans cette thèse, je me limite aux phrases elliptiques, en laissant de côté d'autres occurrences de l'ellipse, comme par exemple l'ellipse nominale\footnote{Pour une description de l'ellipse nominale en roumain, voir le travail de \citet{Giurgea2010} et Cornilescu \& \citet{Nicolae2010}.} . 

{\bfseries
Pourquoi ellipse ?} 

L'ellipse est l'un des phénomènes les plus présents dans les langues naturelles et en même temps l'un des phénomènes les moins compris dans la grammaire. On considère souvent que l'emploi de l'ellipse s'explique par le principe du {\guillemotleft}~moindre effort~{\guillemotright} et qu'une séquence elliptique est toujours en distribution libre avec sa contrepartie non-elliptique. Voir dans ce sens ce que dit Zribi-\citet[377]{Hertz1986} : {\guillemotleft}~l'ellipse apparaît donc comme un moyen dont disposent les usagers d'une langue pour abréger la forme des énoncés~{\guillemotright} et {\guillemotleft}~l'ellipse correspond toujours à un choix pour l'usager~{\guillemotright}. Un autre aspect qu'on relève souvent au sujet de l'ellipse est le fait que l'emploi de l'ellipse serait une des principales raisons pour lesquelles les langues naturelles sont si ambiguës. 

En guise de réponse, je m'appuie sur le papier de Hendriks \& \citet{Spenader2005}, qui discute la loi du moindre effort, ainsi que d'autres fonctions qu'on peut attribuer à l'ellipse. La loi dite du moindre effort dans le phénomène d'ellipse dérive essentiellement de l'interaction entre deux principes antinomiques, identifiés par \citet{Grice1968} dans sa Maxime de Quantité : l'un vise à économiser les efforts de production du locuteur (c.-à-d. la contribution du locuteur ne doit pas contenir plus d'information qu'il n'est requis), alors que l'autre principe vise à économiser les efforts d'interprétation de l'interlocuteur (c.-à-d. la contribution du locuteur doit contenir autant d'information qu'il est requis).\footnote{Dans l'approche néo-gricéenne de \citet{Horn1993}, ces deux forces sont rebaptisées Principe-R (ne pas en dire plus que nécessaire) et  Principe-Q (en dire autant que possible).} Appliqué à l'ellipse, cela veut dire que le locuteur peut faire appel à une économie d'efforts de production, à condition que l'interlocuteur soit capable de reconstruire le matériel manquant. On arrive ainsi à omettre l'information qui est redondante et qui peut être facilement récupérable du contexte immédiat. Si beaucoup de phénomènes elliptiques se laissent expliquer tout simplement par la loi du moindre effort, car il n'y a aucune différence entre l'emploi d'une version elliptique et l'emploi d'une version non elliptique, il y a toutefois des emplois de l'ellipse qui doivent trouver une autre explication, vu les différences qui existent entre les deux. 

Crucialement, dans certains contextes, l'emploi de l'ellipse est le seul moyen dont on dispose pour construire une séquence grammaticale ou bien pour obtenir une certaine interprétation. 

Du point de vue syntaxique, on observe des distributions dans lesquelles la version non-elliptique d'une phrase la rend agrammaticale à cause de la violation des contraintes syntaxiques, alors que l'emploi de l'ellipse ne pose aucun problème d'acceptabilité. Ainsi, il y a des formes qui peuvent se combiner à une séquence de constituants, mais qui sont incompatibles avec une forme verbale finie. C'est le cas de la conjonction lexicalisée \textit{ainsi que} \REF{ex:3:1}a et de la négation de constituant \textit{non} et \textit{non pas} \REF{ex:3:1}b en français (cf. Abeillé \& \citet{Godard1996}, \citet{Mouret2007}), les connecteurs \textit{as well as} \REF{ex:3:96}a, \textit{and not} et \textit{but not} \REF{ex:3:96}b en anglais (cf. Culicover \& \citet{Jackendoff2005}), ou encore le marqueur comparatif \textit{ca} `like' \REF{ex:3:97} en roumain. Toutes ces formes apparaissent dans une phrase elliptique, comme ici une phrase trouée (dans les constructions à gapping), mais ne peuvent jamais apparaître dans une phrase verbale finie.  


\begin{enumerate}
\item \label{bkm:Ref302495468}a  Jean parle le japonais, \textbf{ainsi que} Marie (*parle) le coréen. 


\end{enumerate}
  b  Paul dormira chez Marie, et \textbf{non pas} Marie (*dormira) chez Paul.


\begin{enumerate}
\item \label{bkm:Ref302495505}a  Robin speaks French, \textbf{as well as} Leslie (*speaks) German. 


\end{enumerate}
  b  Robin speaks French, \textbf{and not} Leslie (*speaks) German.


\begin{enumerate}
\item \label{bkm:Ref302495539}Te comporți acum cu mine exact \textbf{ca} Maria (*s-a comportat) ieri.


\end{enumerate}
{\itshape
  Tu te comportes maintenant avec moi exactement comme Maria (s'est comportée) hier}

Dans d'autres contextes, la version non-elliptique est agrammaticale, car elle viole certaines contraintes syntaxiques, comme les contraintes dites d'îles. Ainsi, dans les exemples avec sluicing en \REF{ex:3:98}, la présence d'une phrase subordonnée complète après la forme \textit{qu-} \textit{which} viole la contrainte de localité s'appliquant aux relatives (c.-à-d. l'extraction d'un mot \textit{qu-} en dehors d'une phrase relative), alors que l'emploi de l'ellipse dans ce contexte ne pose aucun problème particulier.


\begin{enumerate}
\item \label{bkm:Ref302543968}a  They want to hire someone who speaks a Balkan language but I don't remember which (*they want to hire someone who speaks).      (\citet[5]{Merchant2001})


\end{enumerate}
b  We will discover something that solves a well-known problem, but I won't divulge which (*we will discover something that solves).      (\citet[2]{Johnson2008})

c  Ben will be mad if Abby talks to one of the teachers, but she couldn't remember which (*Ben will be mad if she talks to).        (\citet[136]{Merchant2008})

Du point de vue sémantique, on observe que certains phénomènes, comme la coréférence, se produisent uniquement dans la version elliptique. Ainsi, dans les exemples anglais cités par Johnson (1996/2004, 2000, 2009) que j'ai repris en \REF{ex:3:99}, on a une co-indiciation du pronom dans le deuxième conjoint avec le sujet quantifié dans le premier conjoint uniquement s'il y a ellipse.


\begin{enumerate}
\item \label{bkm:Ref286421489}a  [Not every girl]\textsubscript{i} ate a green banana and her\textsubscript{i} mother (*ate) a ripe one. 


\end{enumerate}
  b  [No woman]\textsubscript{i} can join the army and her\textsubscript{i} girlfriend (*can join) the navy. 

Parfois, les interprétations associées à une phrase elliptique et respectivement non-elliptique sont en distribution complémentaire : ainsi, dans l'exemple \REF{ex:3:100}a repris de Siegel (1984, 1987), le verbe modal répété dans les deux conjoints a nécessairement une portée étroite, alors que l'ellipse du modal nié dans l'exemple \REF{ex:3:100}b entraîne une portée large de celui-ci sur les deux conjoints.\footnote{Siegel (1984, 1987) distingue les cas de gapping du modal comme en \REF{ex:3:100}b des cas où l'on a un trou complexe qui contient le modal, ainsi que le verbe lexical du syntagme verbal. Selon lui, l'exemple (i) ci-dessous a les deux lectures possibles.
(i)  John can't eat caviar and Mary beans.
a  John can't eat caviar and Mary can't eat beans.
b  It can't be that John eats caviar and Mary beans.} 


\begin{enumerate}
\item \label{bkm:Ref286420421}a  John can't eat caviar and Mary can't eat beans.  = (({\textlnot}${\lozenge}$p)  ${?}$ ({\textlnot}${\lozenge}$q))


\end{enumerate}
b  John can't eat caviar and Mary eat beans.    = ({\textlnot}${\lozenge}$(p ${?}$q))

Dans d'autres contextes, l'emploi de l'ellipse restreint le nombre d'interprétations possibles en dehors de l'ellipse. L'ellipse peut ainsi désambiguïser un énoncé qui, dans sa version non-elliptique, est ambigu. Un exemple bien connu dans la littérature est celui cité par Levin \& \citet{Prince1986} et repris par \citet{Kehler2002}, figurant en \REF{ex:3:101}. La coordination de deux phrases complètes en \REF{ex:3:101}a est compatible avec deux types de relations discursives : une relation symétrique, si les deux événements sont interprétés comme étant indépendants l'un par rapport à l'autre (et, dans ce cas, on a une relation discursive de ressemblance et en particulier de contraste, dans les termes de \citet{Kehler2002}), ou bien une relation asymétrique, si le premier événement est présenté comme étant la cause du deuxième événement (et, cette fois-ci, la relation discursive est de type cause-effet). En revanche, la coordination elliptique en \REF{ex:3:101}b n'est compatible qu'avec une lecture symétrique. 


\begin{enumerate}
\item \label{bkm:Ref286308655}a  Sue became upset and Nan became downright angry.  


\end{enumerate}
  b  Sue became upset and Nan downright angry.  

Un autre exemple qui indique la restriction du nombre d'interprétations possibles avec l'ellipse est le fameux puzzle de Dahl (cf. Fiengo \& \citet{May1994}), exemplifié pour le gapping dans \citet{Coppock2001} : la présence de plusieurs pronoms dans l'exemple non-elliptique en \REF{ex:3:102} entraîne, dans la deuxième phrase, quatre interprétations possibles, dont deux mixtes. Si la deuxième phrase est elliptique \REF{ex:3:103}, on perd la dernière interprétation mixte \REF{ex:3:103}d (c.-à-d. strict/sloppy reading).  


\begin{enumerate}
\item \label{bkm:Ref286356267}Max said \textbf{he} gave \textbf{his} mother a bracelet, and Luc said \textbf{he} gave \textbf{his} mother a watch.  


\end{enumerate}
a  Max\textsubscript{i} said he\textsubscript{i} gave his\textsubscript{i} mother a bracelet, and Luc\textsubscript{j} said he\textsubscript{i} gave his\textsubscript{i} mother a watch. 

b  Max\textsubscript{i} said he\textsubscript{i} gave his\textsubscript{i} mother a bracelet, and Luc\textsubscript{j} said he\textsubscript{j} gave his\textsubscript{j} mother a watch.

c  Max\textsubscript{i} said he\textsubscript{i} gave his\textsubscript{i} mother a bracelet, and Luc\textsubscript{j} said he\textsubscript{j} gave his\textsubscript{i} mother a watch.

d  Max\textsubscript{i} said he\textsubscript{i} gave his\textsubscript{i} mother a bracelet, and Luc\textsubscript{j} said he\textsubscript{i} gave his\textsubscript{j} mother a watch.


\begin{enumerate}
\item \label{bkm:Ref286356626}Max said \textbf{he} gave \textbf{his} mother a bracelet, and Luc a watch.  


\end{enumerate}
a  Max\textsubscript{i} said he\textsubscript{i} gave his\textsubscript{i} mother a bracelet, and Luc\textsubscript{j} said he\textsubscript{i} gave his\textsubscript{i} mother a watch. 

b  Max\textsubscript{i} said he\textsubscript{i} gave his\textsubscript{i} mother a bracelet, and Luc\textsubscript{j} said he\textsubscript{j} gave his\textsubscript{j} mother a watch.

c  Max\textsubscript{i} said he\textsubscript{i} gave his\textsubscript{i} mother a bracelet, and Luc\textsubscript{j} said he\textsubscript{j} gave his\textsubscript{i} mother a watch.

d    *Max\textsubscript{i} said he\textsubscript{i} gave his\textsubscript{i} mother a bracelet, and Luc\textsubscript{j} said he\textsubscript{i} gave his\textsubscript{j} mother a watch.

Comme signalé par plusieurs auteurs (Grinder \& \citet{Postal1971}, Hankamer \& \citet{Sag1976}, Zribi-\citet{Hertz1986}, \citet{Gardent1991}, etc.), la relation mise en jeu par l'ellipse se rapproche d'une relation anaphorique ordinaire entre une expression pronominale et son antécédent. De manière générale, l'ellipse établit la cohérence discursive de la même façon que toute relation anaphorique. L'emploi du pronom \textit{il} coréférent au nom propre \textit{Jean} rend l'énoncé en \REF{ex:3:104}a plus acceptable en termes de cohérence discursive que l'énoncé en \REF{ex:3:104}b, où on répète le nom propre. On observe les mêmes jugements dans des structures elliptiques comme l'ellipse polaire\footnote{Le terme employé dans la littérature est \textit{stripping}, mais il regroupe plusieurs constructions hétérogènes. Dans cette thèse, je suis Abeillé (2005, 2006) et je fais la distinction entre les ellipses {\guillemotleft}~polaires~{\guillemotright} (incluant un adverbe à polarité, tel qu'\textit{aussi} ou \textit{non plus}), les conjoints différés et les phrases adverbiales.} en \REF{ex:3:105} ou l'ellipse du syntagme verbal (angl. \textit{Verb Phrase Ellipsis}, abrégé VPE) en \REF{ex:3:106}.  


\begin{enumerate}
\item \label{bkm:Ref286951586}a  \textbf{Jean} n'est pas venu à l'école. \textbf{Il} est malade.  


\end{enumerate}
b  \textbf{Jean} n'est pas venu à l'école. \textbf{Jean} est malade. 


\begin{enumerate}
\item \label{bkm:Ref286951914}a  Jean aime les pommes et Marie aussi (aime les pommes).  


\end{enumerate}
b  \#Jean aime les pommes et Marie aime les pommes. 


\begin{enumerate}
\item \label{bkm:Ref286951937}a  John saw the flying saucer, and Bill did too. (\citet{Dalrymple2005})


\end{enumerate}
  b  \#John saw the flying saucer, and Bill saw the flying saucer. 

Avant de finir cette section, je voudrais revenir à un aspect que j'ai mentionné dans l'introduction de cette sous-section, à savoir le lien entre l'ellipse et l'ambiguïté. On a vu précédemment que dans certains contextes la version elliptique est moins ambiguë que la version non-elliptique. Cependant, on pense souvent que l'ellipse est un des facteurs responsables de l'existence de l'ambiguïté dans les langues naturelles. Dans la littérature, on trouve généralement au moins deux types d'ambiguïtés systématiques liés, d'une part, à l'interprétation des pronoms et, d'autre part, à l'interprétation d'une expression nominale non-marquée. En ce qui concerne le premier type d'ambiguïté, on observe que l'élision d'une expression pronominale possessive de troisième personne (p.ex.\textit{ his} en \REF{ex:3:107}) induit dans le second conjoint (elliptique) une double interprétation, à savoir une interprétation stricte (angl. \textit{strict reading}) ou bien une interprétation relâchée (angl. \textit{sloppy reading}). Quant au deuxième type d'ambiguïté, on observe que, si le premier élément résiduel dans le conjoint elliptique est une expression nominale non-marquée, il peut être interprété syntaxiquement comme sujet ou bien comme objet, comme illustré dans l'exemple \REF{ex:3:108} repris de Hendriks \& \citet{Spenader2005} ou encore dans les exemples \REF{ex:3:109}-\REF{ex:3:111}, repris de Carlson \textit{et al.} (2005). 


\begin{enumerate}
\item \label{bkm:Ref286955860}John\textsubscript{i} loves \textbf{his}\textbf{\textsubscript{i}} wife, and Bill\textsubscript{j} does too.  


\end{enumerate}
a  Bill\textsubscript{j} loves his\textsubscript{i} wife. (interprétation stricte)

  b  Bill\textsubscript{j} loves his\textsubscript{j} wife. (interprétation relâchée)    


\begin{enumerate}
\item \label{bkm:Ref286958052}Mary likes John, and \textbf{Bill} too.  


\end{enumerate}
a  Bill likes John. (\textit{Bill} = sujet)

  b  Mary likes Bill. (\textit{Bill} = objet)


\begin{enumerate}
\item \label{bkm:Ref286957630}Bob insulted the guests during dinner and \textbf{Sam} during the dance.


\end{enumerate}
a  Sam insulted the guests during the dance. (\textit{Sam} = sujet) 

  b  Bob insulted Sam during the dance. (\textit{Sam} = objet)


\begin{enumerate}
\item Tasha called Bella more often than \textbf{the doctor}.


\end{enumerate}
a  ... more often than the doctor called Bella. (\textit{the doctor} = sujet) 

  b  ... more often than Tasha called the doctor. (\textit{the doctor} = objet)


\begin{enumerate}
\item \label{bkm:Ref286957659}A friend called Marcus for advice, (but) not \textbf{a relative}.


\end{enumerate}
a  A relative didn't call Marcus for advice. (\textit{a relative} = sujet) 

  b  A friend didn't call a relative for advice. (\textit{a relative} = objet)

Cependant, le problème de l'ambiguïté liée à l'ellipse se pose beaucoup moins dans les langues avec plus de marquage morpho-syntaxique, lexical ou prosodique. En roumain par exemple, l'ambiguïté liée à l'interprétation des pronoms est levée (au moins pour l'interprétation relâchée) par l'emploi de formes pronominales différentes : ainsi, un clitique réfléchi au datif (à interprétation possessive) permet uniquement l'interprétation relâchée en \REF{ex:3:112}. De même, l'ambiguïté liée à l'interprétation syntaxique (sujet vs. objet) est résolue en roumain grâce au marquage casuel : le sujet (au nominatif) n'a pas de marque spécifique en \REF{ex:3:113}a et \REF{ex:3:114}a, alors que l'objet (à l'accusatif) reçoit la marque \textit{pe} en \REF{ex:3:113}b et \REF{ex:3:114}b. Par conséquent, les exemples qui étaient ambigus en anglais ne le sont pas en roumain. 


\begin{enumerate}
\item \label{bkm:Ref286961281}a  Ion\textsubscript{i}  \textbf{îşi}\textbf{\textsubscript{i}}  iubeşte  soția,  iar  Dan\textsubscript{j}  de asemenea. 


\end{enumerate}
Ion  \textsc{cl.refl.dat}  aime  l'épouse,  et  Dan  de même

{\itshape
Ion aime sa femme et Dan aussi}

b  Dan\textsubscript{j} îşi\textsubscript{j} iubeşte soția. 

  \textit{Dan}\textsubscript{j}\textit{ aime sa}\textsubscript{j}\textit{ femme} 

c  *Dan\textsubscript{j} îşi\textsubscript{i} iubeşte soția. 

  \textit{Dan}\textsubscript{j}\textit{ aime sa}\textsubscript{i}\textit{ femme}


\begin{enumerate}
\item \label{bkm:Ref302553088}a  \textbf{Maria}  îl  iubeşte  pe  Ion,  iar  \textbf{Dan}  de asemenea. 


\end{enumerate}
Maria  \textsc{cl.acc}  aime  \textsc{mrq}  Ion,  et  Dan  de même

{\itshape
Maria aime Ion, et Dan l'aime aussi} 

b  Maria  îl  iubeşte  \textbf{pe}  \textbf{Ion},  iar  \textbf{pe  Dan}  de asemenea. 

Maria  \textsc{cl.acc}  aime  \textsc{mrq}  Ion,  et  \textsc{mrq}  Dan  de même

{\itshape
    Maria aime Ion, et elle aime Dan aussi} 


\begin{enumerate}
\item \label{bkm:Ref302553090}a  \textbf{Bob}  i-a  insultat  pe  invitați  la  masă,  iar  \textbf{Sam}  la  sfârşitul  petrecerii. 


\end{enumerate}
Bob  \textsc{cl.acc}-a  insulté  \textsc{mrq}  invités  à  table,  et  Sam  à  la-fin  la-fête.\textsc{gen}

{\itshape
Bob a insulté les invités pendant le repas, et Sam les a insultés à~la fin de la fête}

b  Bob  i-a  insultat  \textbf{pe}  \textbf{invitați}  la  masă  şi  \textbf{pe}  \textbf{Sam}  la  petrecere.

Bob  \textsc{cl.acc}-a  insulté  \textsc{mrq}  invités  à  table  et  \textsc{mrq}  Sam  à  fête

{\itshape
    Bob a insulté les invités pendant le repas et il a insulté Sam pendant la fête}

En même temps, les travaux montrent que, même dans les langues qui ne possèdent pas de marques casuelles ou de formes pronominales spécifiques, l'ambiguïté n'est pas un problème réel, une fois qu'on a le contexte approprié. Par conséquent, il ne s'agit pas d'une contrainte grammaticale, mais plutôt d'un système de préférences relativement complexe qui met en jeu des contraintes informationnelles, prosodiques et psycholinguistiques (voir \citet{Hankamer1973}, \citet{Kuno1976}, \citet{Carlson2002} ; voir le rôle de l'intonation et la facilité du processing qui désambiguïse entre le gapping et la coordination de séquences en anglais, dans les travaux de Carlson (2001, 2002)). 

Pour conclure, on observe qu'en réduisant toutes les occurrences de l'ellipse au principe du moindre effort on ne peut pas rendre compte des différentes discordances observées entre une séquence elliptique et sa contrepartie non-elliptique. Il faut donc considérer l'ellipse comme un phénomène en soi, qui doit avoir sa place dans la grammaire.

\subsection{La phrase et la notion de tête}
Dans le chapitre 1, on a défini la phrase comme un syntagme maximal saturé à tête prédicative, dénotant une situation et ayant un contenu de type \textit{message} (proposition ou abstraction propositionnelle). Il s'agit d'un prédicat saturé, soit parce qu'il a trouvé tous ses arguments (y compris le sujet) soit parce qu'il est lexicalement/intrinsèquement saturé (et donc il n'attend pas de sujet). 

Cette définition nous permet d'inclure parmi les occurrences phrastiques non seulement les phrases verbales finies, mais aussi les phrases verbales non-finies, ainsi que les phrases averbales. 

\subsubsection{Phrases verbales non-finies} 
Les phrases verbales non-finies ont toujours comme tête un verbe non-fini : en emploi racine, on retrouve l'infinitif \REF{ex:3:115}a, le supin \REF{ex:3:115}b ou le participe \REF{ex:3:115}c, alors qu'en emploi subordonné, on retrouve l'infinitif \REF{ex:3:116}a, le gérondif\footnote{Ou participe présent en français.} \REF{ex:3:116}b ou le participe\footnote{Le participe est limité dans ces situations aux verbes transitifs et inaccusatifs. \citet{Mouret2011} distingue entre les phrases verbales au participe présent (ou gérondif en roumain) et les phrases {\guillemotleft}~averbales~{\guillemotright} au participe passé. Les phrases avec participes passés sont analysées comme des phrases averbales, en vertu du statut catégoriel mixte du participe passé (entre verbe et adjectif) : d'une part, il a un comportement verbal, car il se construit directement avec un complément ; d'autre part, il a un comportement adjectival, car il présente l'accord en genre et en nombre et, de plus, il ne peut pas être hôte d'un affixe pronominal. Cette distinction, bien que convaincante, n'est pas pertinente pour mon étude ici.}  \REF{ex:3:116}c.\footnote{La plupart de ces phrases verbales non-finies en emploi subordonné ont la fonction d'ajout par rapport à la phrase racine (cf. le terme \textit{constructions absolues circonstancielles}, retenu par la tradition grammaticale). Comme le note \citet{Laurens2007} et \citet{Mouret2011} pour le français, leur distribution peut varier en fonction de la présence ou non d'un introducteur : les phrases qui ont un marqueur subordonnnant (p.ex. \textit{o dată} `une fois' en (i)) ont une distribution assez libre par rapport à la phrase racine (c.-à-d. position initiale, médiane ou finale), alors que celles sans marqueur sont naturelles uniquement en position initiale (ii).
(i)  a  [\textbf{O dată} mama ajunsă acasă], am început să-mi fac temele.
    \textit{Une fois ma mère arrivée à la maison, j'ai commencé à faire mes devoirs}
  b  Am început, [\textbf{o dată} mama ajunsă acasă], să-mi fac temele.
    \textit{J'ai commencé, une fois ma mère arrivée à la maison, à faire mes devoirs}
  c  Am început să-mi fac temele, [\textbf{o dată} mama ajunsă acasă]. 
    \textit{J'ai commencé à faire mes devoirs, une fois ma mère arrivée à la maison}
(ii)  a  [Mama ajunsă acasă], am început să-mi fac temele.
    \textit{Ma mère arrivée à la maison, j'ai commencé à faire mes devoirs}
  b  *Am început, [mama ajunsă acasă], să-mi fac temele.
    \textit{J'ai commencé, ma mère arrivée à la maison, à faire mes devoirs}
  c  *Am început să-mi fac temele, [mama ajunsă acasă]. 
    \textit{J'ai commencé à faire mes devoirs, ma mère arrivée à la maison}}


\begin{enumerate}
\item \label{bkm:Ref302561879}a  \textbf{A nu se folosi} acest medicament fără prescripția medicului. 


\end{enumerate}
{\itshape
Ne pas prendre ce médicament sans l'avis d'un médecin}

b  \textbf{De rezolvat} primele două exerciții.  

  résoudre.\textsc{supin} les-premiers deux exercices 

{\itshape
A résoudre les deux premiers exercices}

c  Oprirea \textbf{interzisă}.  

  l'arrêt interdit 

{\itshape
Interdit de stationner}


\begin{enumerate}
\item \label{bkm:Ref302561915}a  Am  cerut  [\textbf{a  se  înscrie}  imobilul  la  oficiul  de  cadastru]. 


\end{enumerate}
ai  demandé  \textsc{mrq  cl  inf}  l'immeuble  à  l'office  de  cadastre

{\itshape
J'ai demandé à ce que l'immeuble soit inscrit à l'office de cadastre}

b  [\textbf{Date  fiind}  rezultatele  nesatisfăcătoare],  directorul  a  renunțat  la  proiect. 

donnés  étant  les-résultats  insatisfaisants,  le-directeur  a  renoncé  à  projet

  \textit{Etant donné les résultats insatisfaisants, le directeur a renoncé au projet} 

c  [\textbf{Ajuns}  Ion  acasă],  a  început  ploaia. 

arrivé  Ion  à-maison,  a  commencé  la-pluie

  \textit{Ion arrivé à la maison, la pluie commença}

\subsubsection{Phrases averbales} 
On regroupe sous le nom de \textit{phrase} \textit{averbale} toute phrase dont la tête prédicative appartient à une autre catégorie que le verbe. La tête d'une phrase averbale peut appartenir à une catégorie adjectivale \REF{ex:3:117}a, nominale \REF{ex:3:117}b, prépositionnelle \REF{ex:3:117}c, adverbiale \REF{ex:3:117}d ou interjectionnelle \REF{ex:3:117}e, à condition que la catégorie en question permette un emploi prédicatif. Les phrases averbales peuvent avoir un seul constituant immédiat ou deux (exemples).


\begin{enumerate}
\item \label{bkm:Ref305619139}a  \textbf{Foarte reuşită}, prăjitura asta !


\end{enumerate}
\textit{  Trés réussi, ce gâteau}  

  b  \textbf{Multe salutări} prietenei tale !

    \textit{Meilleures salutations à ton amie}  

c  Toată lumea \textbf{la masă~}!

  \textit{Tout le monde à table}

d  \textbf{Gata} (cu) joaca !  

{\itshape
Fini (avec) le jeu}

  e  \textbf{Stop} accidentelor rutiere, viața are prioritate !

\textit{    Stop aux accidents routiers, la vie a la priorité}  

Leur statut est problématique pour certains qui essayent de les aligner sur les phrases verbales. Traditionnellement, les phrases averbales sont analysées comme des phrases elliptiques (on considère donc qu'il y a un verbe présent syntaxiquement et effacé en phonologie). Je suis la perspective de Laurens (2007, 2008) et \textit{GGF en prép.} qui considèrent que les phrases averbales ne posent pas de problème particulier et que leur structure s'intègre très naturellement dans la grammaire de la langue. Par conséquent, ces phrases ne doivent donc pas être assimilées aux phrases verbales. 

Si les phrases averbales étaient elliptiques, on devrait pouvoir reconstruire le verbe manquant de manière systématique. Or, pour le français, on a observé qu'il n'y a pas, en général, de phrase verbale qui correspond exactement à ces séquences (cf. Laurens (2007, 2008), \textit{GGF} \textit{en prép.}). On ne peut pas trouver un mécanisme de reconstruction d'une phrase verbale ordinaire qui s'applique à toutes les phrases averbales. 

Ainsi, bien qu'on puisse reconstruire dans beaucoup de phrases averbales un verbe comme \textit{être} \REF{ex:3:118}a, les phrases averbales exclamatives ne le permettent pas en français \REF{ex:3:118}b (cf. \citet{Laurens2008}) :


\begin{enumerate}
\item \label{bkm:Ref302645069}a  Il est très intéressant, ce livre.  


\end{enumerate}
b  *Quel dommage c'est, qu'il ne vienne pas.

En roumain aussi, on a des cas dans lesquels la reconstruction d'un verbe produit un énoncé inacceptable, comme \REF{ex:3:119}b :


\begin{enumerate}
\item \label{bkm:Ref283990469}a  Gata (cu) joaca !


\end{enumerate}
{\itshape
Fini (avec) le jeu}

b  \#(E)  gata  (e)  joaca.

  (est)  fini  (est)  le-jeu

    \textit{C'est fini le jeu}

Selon \textit{GGF en prép.}, en français il y a au moins trois possibilités de reconstruction, qui font appel d'ailleurs à des mécanismes supplémentaires et côuteux : (i) dans certains contextes avec ordre tête-sujet \REF{ex:3:120}a, il faut reconstruire un ordre sujet-verbe \REF{ex:3:120}b ; (ii) parfois, il faut reconstruire une structure plus complexe, avec dislocation à droite du sujet (comparer \REF{ex:3:121}a et \REF{ex:3:121}b) ; (iii) dans d'autres contextes avec un syntagme exclamatif contenant une forme \textit{qu-} \REF{ex:3:122}a, il faut à la fois une dislocation à droite du sujet et un syntagme antéposé \REF{ex:3:122}b.


\begin{enumerate}
\item \label{bkm:Ref302631645}a  Agréable, ce petit vin nouveau !  


\end{enumerate}
b  Ce petit vin nouveau est agréable.


\begin{enumerate}
\item \label{bkm:Ref302631695}a  Finies, les vacances !  


\end{enumerate}
b  Elles sont finies, les vacances !


\begin{enumerate}
\item \label{bkm:Ref302631718}a  Quelle poisse, cette pluie !  


\end{enumerate}
b  Quelle poisse c'est, cette pluie !

L'ordre des éléments dans une phrase averbale est souvent contraint en roumain aussi, contrairement à ce qui se passe dans une phrase verbale, où l'ordre est relativement libre (cf. chapitre 1, section \ref{sec:1.3.4}). Ainsi, dans les exemples \REF{ex:3:123} et \REF{ex:3:124}, le syntagme prépositionnel non-prédicatif doit suivre la tête prédicative, alors que dans une phrase verbale l'ordre du syntagme prépositionnel par rapport à la tête n'est pas contraint \REF{ex:3:125}.


\begin{enumerate}
\item \label{bkm:Ref283990500}a  La moarte cu voi !


\end{enumerate}
{\itshape
A la mort avec vous}

b  ??Cu  voi  la  moarte !

  avec  vous  à  mort 


\begin{enumerate}
\item \label{bkm:Ref283990504}a  Jos  cu  dictatura !


\end{enumerate}
bas  avec  la-dictature

b  *Cu  dictatura  jos !

  avec  la-dictature  bas


\begin{enumerate}
\item \label{bkm:Ref302651444}a  M-am certat cu Maria.


\end{enumerate}
\textsc{cl-aux} fâché contre Maria

{\itshape
Je me suis fâché contre Maria}

b  Cu Maria m-am certat.

  avec Maria \textsc{cl-aux} fâché

    \textit{Je me suis fâché contre Maria}

Comme le note \citet{Laurens2008} pour le français \REF{ex:3:126}, la forme du syntagme prédicatif n'est pas toujours la même dans les phrases averbales et dans leurs contreparties verbales. Ainsi, en roumain le nom avec un modifieur adnominal en emploi prédicatif dans une phrase averbale apparaît sans un déterminant indéfini \REF{ex:3:127}a, contrairement aux occurrences verbales \REF{ex:3:127}b où le même nom, mais cette fois-ci complément du verbe attributif \textit{a fi} `être', reçoit soit un déterminant défini \REF{ex:3:127}c soit un déterminant indéfini \REF{ex:3:127}d.


\begin{enumerate}
\item \label{bkm:Ref302651020}a  Très bel animal, ce chat.  


\end{enumerate}
b  C'est *(un) très bel animal, ce chat.


\begin{enumerate}
\item \label{bkm:Ref283990465}a  Foarte bună idee să mergem pe jos ! 


\end{enumerate}
{\itshape
Très bonne idée de marcher à pied}

b  *\textbf{E}  foarte  bună  idee  să  mergem  pe jos !

  est  très  bonne  idée  \textsc{mrq}  marcher.\textsc{subj } à pied 

{\itshape
C'est une très bonne idée de marcher à pied}

c  E  foarte  bună  idee\textbf{a}  să  mergem  pe jos !

  est  très  bonne  idée.\textsc{def}  \textsc{mrq}  marcher.\textsc{subj } à pied

{\itshape
C'est une très bonne idée de marcher à pied}

d  E  \textbf{o } foarte  bună  idee  să  mergem  pe jos ! 

  est  une  très  bonne  idée  \textsc{mrq}  marcher.\textsc{subj } à pied

{\itshape
    C'est une très bonne idée de marcher à pied}

Dans les phrases verbales qui permettent la reconstruction d'un verbe, on observe que les verbes reconstruits sont très variés :


\begin{enumerate}
\item a  E  foarte  reuşită,  prăjitura  asta !


\end{enumerate}
    est  très   réussi,  gâteau  \textsc{dem}

\textit{  Il est trés réussi, ce gâteau}  

  b  Transmite  multe  salutări  prietenei  tale !

    transmets  beaucoup  salutations  amie  \textsc{poss}

    \textit{Transmets les meilleures salutations à ton amie}  

c  Toată  lumea  să  vină  la  masă\textbf{~}!

  tout  le-monde  \textsc{mrq}  venir.\textsc{subj}  à  table

  \textit{Que tout le monde vienne à table}

  d  Spune stop accidentelor rutiere, viața are prioritate !

{\itshape
    Dis stop aux accidents routiers, la vie a la priorité}

Crucialement, les phrases averbales peuvent être énoncées hors de tout contexte, ce qui les distingue des phrases elliptiques, dont l'interprétation est déterminée contextuellement (cf. discussion plus bas).

Dans le chapitre 1, section \ref{sec:1.2.2}, on a vu qu'à chaque type phrastique on associe un certain type de contenu sémantique. De ce point de vue, les phrases averbales se comportent comme des phrases ordinaires, étant compatibles avec tout type de phrase (c.-à-d. déclaratif, exclamatif, interrogatif et désidératif), tout type de contenu sémantique (c.-à-d. proposition, question, visée) et, pour les phrases averbales en emploi racine, tout type d'acte illocutoire (c.-à-d. assertion, exclamation\footnote{L'acte illocutoire le plus fréquent dans les phrases averbales est l'exclamation.}, interrogation, injonction). Pour tester le contenu d'une phrase, on fait appel aux adverbes évaluatifs et/ou aux ajouts illocutoires (Bonami \& \citet{Godard2005}, Beyssade \& \citet{Marandin2006}), soulignés dans les exemples en \REF{ex:3:129} : une phrase déclarative dont le contenu est une proposition en \REF{ex:3:129}a ; une phrase exclamative, introduite par un syntagme exclamatif contenant une forme \textit{qu-} en \REF{ex:3:129}b, ayant comme type de contenu toujours une proposition ; une phrase interrogative, introduite par un syntagme interrogatif contenant une forme \textit{qu-} en \REF{ex:3:129}c, dont le contenu est une question, comme indiqué par l'emploi de l'ajout illocutoire \textit{oare} (cf. section \ref{sec:1.2.3} du chapitre 1) ; enfin, une phrase désidérative en \REF{ex:3:129}d, dont le contenu est de type visée, cf. l'emploi de l'ajout illocutoire \textit{vă rog} `s'il vous plaît'. 


\begin{enumerate}
\item \label{bkm:Ref302653181}a  Din nefericire, nimeni la orizont.


\end{enumerate}
\textit{  (Malheureusement), personne à l'horizon}  

  b  Câtă dezordine în casa asta, din păcate !

    \textit{Quel désordre dans cette maison, malheureusement}  

  c  Pe când oare nunta ? 

{\itshape
A quand le mariage}

\textbf{ } d  Gata cu joaca, vă rog !

    \textit{Stop aux accidents routiers, la vie a la priorité}

La plupart des phrases averbales apparaissent en emploi racine, mais certaines peuvent être liées à une phrase verbale par coordination \REF{ex:3:130} ou enchâssement \REF{ex:3:131}. Parmi les phrases averbales enchâssées, on mentionne les phrases à tête adverbiale, dont la tête est un adverbe prophrase (p.ex. \textit{da} `oui'\textit{, nu~}`non'\textit{, ba da} `si'\textit{, ba nu} `non'), comme \REF{ex:3:131}c. 


\begin{enumerate}
\item \label{bkm:Ref283996605}a  [Foarte frumos tablou], \textbf{dar} din păcate nu am bani să-l cumpăr. 


\end{enumerate}
{\itshape
Très joli tableau, mais malheureusement je n'ai pas d'argent pour l'acheter}

b  M-a acuzat că i-am furat banii, \textbf{dar} [nici pomeneală de aşa ceva]. 

{\itshape
Il m'a accusé d'avoir volé son argent, mais pas question d'avoir fait cela}

c  Nu ştiu cine a scris articolul, \textbf{dar} [halal exprimare] !

  est très bonne idée.\textsc{def} marcher.\textsc{subj} à pied

{\itshape
Je ne sais pas qui a écrit l'article, mais quelle drôle d'expression}


\begin{enumerate}
\item \label{bkm:Ref283996624}a  Stau şi mă întreb uneori [\textbf{de ce} atâta suferință în viață].


\end{enumerate}
{\itshape
Je me demande parfois pourquoi tant de souffrances dans cette vie}

b  Medicul care m-a consultat mi-a spus [\textbf{că} nici vorbă de cancer]. 

{\itshape
Le médecin qui m'a consulté m'a dit que pas question de cancer} 

c  [Sigur [\textbf{că} \{da / nu\}]].

{\itshape
Assûrement qu'oui / non}

D'autres phrases à tête adverbiale contiennent comme tête prédicative un adverbe qui peut sélectionner un complément phrastique (p.ex. \textit{bine} `bien'\textit{, evident} `évidemment', \textit{poate} `peut-être', \textit{probabil} `probablement', \textit{sigur} `assûrement', \textit{fără} \textit{îndoială} `sans doute', \textit{cu} \textit{siguranță} `assûrement', \textit{bine} \textit{înțeles} `bien sûr'), comme en \REF{ex:3:132}.


\begin{enumerate}
\item \label{bkm:Ref302655172}a  \textbf{Bine} că nu ninge.


\end{enumerate}
  bien qu'il \textsc{neg} neige 

\textit{  C'est bien qu'il ne neige pas}  

  b  \textbf{Probabil} că va ninge.

    probablement que va neiger

    \textit{Il est probable qu'il neige} 

Pour conclure, les phrases averbales ne sont pas elliptiques. Elles ont toujours une catégorie prédicative présente. Leur contenu propositionnel est construit sans faire appel au contexte, ce qui permet leur énonciation hors de tout contexte. 

\subsubsection{Phrases elliptiques (et fragmentaires)}
\label{bkm:Ref306038431}On a vu dans la section précédente que les phrases averbales ne sont pas des phrases elliptiques. Il nous reste maintenant à définir ce que c'est une phrase elliptique. Afin de capter la distinction phrase complète (verbale ou averbale) vs. phrase elliptique, j'utilise une hiérarchie des types de syntagmes légèrement modifiée (cf. \citet{Laurens2008}) par rapport à la hiérarchie proposée par Ginzburg \& \citet{Sag2000}, que j'ai donnée dans le chapitre 1, section \ref{sec:1.4.2}. La hiérarchie en \REF{ex:3:133} contient cette fois-ci trois dimensions de classification, et non deux. La première dimension, étiquetée HEADEDNESS, est utilisée pour distinguer les syntagmes avec tête (angl. \textit{headed-phrases}) des syntagmes sans tête (angl. \textit{non-headed-phrases}). La deuxième dimension, appelée CONTENT-TYPE, sert à distinguer les syntagmes ayant un contenu de type \textit{message} (angl. \textit{message-denoting-ph}) des syntagmes qui n'ont pas un contenu propositionnel (angl. \textit{non-message denoting-ph}). Enfin, la troisième dimension, étiquetée AUTONOMY, distingue les syntagmes dont le contenu n'est pas sensible au contexte (angl. \textit{autonomous}) des syntagmes dont le contenu est dépendant du contexte (angl. \textit{non-autonomous}). 


\begin{enumerate}
\item \label{bkm:Ref302663728}Hiérarchie des types de syntagmes (cf. \citet{Laurens2008})


\end{enumerate}
{   [Warning: Image ignored] % Unhandled or unsupported graphics:
%\includegraphics[width=5.8709in,height=1.6098in,width=\textwidth]{fe443409cd384d3fb0f6390ffd77f513-img35.svm}
} 

Selon cette hiérarchie, les phrases étudiées jusqu'à maintenant (c.-à-d. les phrases verbales et les phrases averbales) ont toutes une tête (verbale ou non-verbale) prédicative (donc, elles sont de type \textit{headed-ph}), ont toutes un contenu de type \textit{message} (elles sont donc de type \textit{message-denoting-ph}) et, enfin, le message véhiculé est donné par la phrase elle-même et ne varie pas en fonction du contexte discursif (elles sont donc \textit{autonomous-ph}) ; on construit le contenu de la phrase sans faire appel au contexte. En vertu de ces propriétés, auxquelles on ajoute le fait qu'il s'agit toujours dans ces cas d'un syntagme syntaxiquement saturé, on peut parler de \textit{phrase complète}.

Comment peut-on maintenant définir la \textit{phrase elliptique} par rapport à la \textit{phrase complète~}? Je commence par l'aspect que les phrases elliptiques et les phrases complètes ont en commun : les deux ont un contenu sémantique de type \textit{message} (donc, les phrases elliptiques sont, elles aussi, de type \textit{message-denoting-ph}). Cependant, une différence majeure entre les phrases complètes et les phrases elliptiques réside dans le fait que les dernières sont syntaxiquement incomplètes : un (ou plusieurs) des constituants de la phrase manque. Ce constituant peut être la tête de la phrase ou bien un dépendant de la tête (plus de détails dans la section \ref{sec:3.3.1}). Cette incomplétude syntaxique entraîne une autre différence majeure, cette fois-ci en plan interprétatif. Crucialement, le contenu principal d'une phrase elliptique dépend du contexte (c.-à-d. \textit{non-autonomous-ph}) : le matériel manquant doit être récupéré dans le contexte, à partir d'un antécédent dans le discours. De ce point de vue, l'ellipse ressemble aux anaphores.\footnote{Plus précisément, on peut dire, en suivant la distinction faite par Grinder \& \citet{Postal1971} entre \textit{Identity of Reference Anaphora} et \textit{Identity of Sense Anaphora}, que l'ellipse est un type d'anaphore de sens (car pas de coréférence, pas de dénotation).}

Les phrases elliptiques sont donc des phrases avec une constituance a priori incomplète, qui n'ont pas d'autonomie discursive, mais qui ont un contenu de type \textit{message}. 

Parmi les phrases elliptiques, je distingue les \textit{phrases fragmentaires}. Comme toute phrase, elles ont un contenu de type \textit{message}. Comme dans toute phrase elliptique, le message véhiculé n'est pas donné par la phrase elle-même, mais varie en fonction du contexte discursif (donc, elle n'a pas d'autonomie discursive). Sur le plan syntaxique, les phrases fragmentaires n'ont pas de catégorie prédicative présente ; il manque la tête et on ne peut pas, du point de vue formel, reconstituer une phrase complète à un mode personnel (cf. Abeillé \textit{en prép.}). Une phrase fragmentaire n'a pas donc une structure endocentrique ordinaire. 

Selon cette dernière définition, les phrases fragmentaires posent deux défis à toute théorie linguistique : d'une part, elles ont une constituance fragmentaire, qui ne se laisse pas facilement décrite par les règles habituelles d'une structure endocentrique classique ; d'autre part, comme la relation sémantique principale de ces phrases dépend du contexte, on a besoin d'un mécanisme interprétatif complexe.

\subsection{Typologie des phrases elliptiques} 
Avant de présenter les principales constructions mettant en jeu des phrases elliptiques, je veux introduire les termes minimaux dont on a besoin. L'élément qui manque dans une phrase elliptique sera désigné comme \textit{matériel manquant}. Les éléments réalisés lexicalement dans la phrase elliptique sont appelés \textit{éléments résiduels} (angl. \textit{remnants}). La phrase complète qui fournit le matériel nécessaire à l'interprétation de la phrase elliptique est la\textit{ phrase source}. La phrase source contient l'\textit{antécédent} du matériel manquant, ainsi que les \textit{éléments corrélats} qui sont les constituants parallèles aux éléments résiduels dans la phrase complète.  

Je m'intéresse par la suite aux phrases elliptiques qui correspondent à des constructions spécifiques, c.-à-d. des structures syntaxiques auxquelles manquent la tête et/ou des dépendants, dont l'interprétation, et dans une certaine mesure la syntaxe, sont fixées par le contexte. 

Trois critères peuvent être utilisés pour le classement des constructions elliptiques phrastiques : (i) la nature du matériel manquant (c.-à-d. son statut syntaxique : tête ou dépendant) ; (ii) le type de contexte syntaxique dans lequel une construction elliptique peut apparaître, et (iii) la directionnalité~de l'ellipse (c.-à-d. le matériel manquant apparaît avant ou après l'antécédent).

\subsubsection{Nature du matériel manquant}
\label{bkm:Ref305101923}Selon ce critère, on arrive à trois classes majeures : (i) ellipse de la tête, si le matériel manquant dans une phrase elliptique correspond à la tête de la phrase ; (ii) ellipse des dépendants, si le matériel manquant ne correspond pas à la tête, mais à un dépendant de la phrase (argument ou ajout), et (iii) ellipse non-sélective ou indifférenciée, si le matériel manquant peut correspondre à une tête ou bien à un dépendant de la phrase. 

Pour faciliter la lecture et la compréhension des phénomènes, je garde la terminologie anglo-saxonne pour les constructions qui n'ont pas un équivalent propre en français. L'inventaire des constructions elliptiques présenté par la suite ne tient pas compte de la perspective typologique. Comme l'anglais possède tous les types inventoriés, j'illustre chaque construction avec des exemples en anglais, pour mieux observer les différences d'une construction à l'autre. Dans les exemples qui suivent, le matériel qui est souligné correspond à l'antécédent du matériel manquant.

Ellipse de la tête

Parmi les phrases elliptiques dans lesquelles manque la tête (verbale), on distingue habituellement les constructions suivantes :

\textbf{\textit{Gapping} }(Ross (1967, 1970)), ou {\guillemotleft}~phrase trouée~{\guillemotright} (en français), désigne toute phrase elliptique qui compte au moins deux éléments résiduels (dont un est généralement -- mais pas de façon obligatoire -- le sujet)  et où manque au moins le verbe principal (qui se trouve habituellement en position médiane), cf. l'exemple \REF{ex:3:134}. Les travaux dédiés au gapping sont ceux de \citet{Neijt1979}, \citet{Gardent1991}, \citet{Hartmann2000}, \citet{Repp2009}, etc.


\begin{enumerate}
\item \label{bkm:Ref302724139}John drinks scotch [and Bill bourbon].  


\end{enumerate}
\textbf{\textit{Conjunction Reduction} }(\citet{Jackendoff1971}), ou {\guillemotleft}~réduction de conjoints~{\guillemotright} (en français), connue aussi sous le nom de \textit{Left Peripheral Ellipsis} ou encore \textit{Left-node raising}, est exemplifiée en \REF{ex:3:135}a.\footnote{Historiquement, la {\guillemotleft}~réduction de conjoints {\guillemotright} incluait les constructions dans lesquelles le matériel manquant apparaissait en début ou en fin de phrase. Donc, ce terme couvrait à la fois \textit{Left Peripheral Ellipsis} et \textit{Right-Node Raising}. Voir dans ce sens \citet{Jackendoff1971}.} Cette construction est très proche des constructions à gapping. Traditionnellement, pour des langues à tête non-finale et à ordre fixe comme l'anglais, la distinction entre les deux constructions réside dans la position du matériel manquant : le matériel manquant est en position médiane dans le gapping, alors qu'il est dans une position périphérique dans les réductions de conjoints. Cependant, certains travaux (\citet{Dowty1988}, \citet{Hudson1988}, Maxwell \& \citet{Manning1996}, \citet{Steedman2000}, Mouret (2006, 2007), etc.) montrent que la réduction de conjoints ne met pas en jeu une ellipse, mais plutôt une coordination de séquences de syntagmes (qu'on peut appeler pseudo-constituants ou clusters) dans la portée syntaxique d'un prédicat verbal, cf. \REF{ex:3:135}b. Dans cette perspective, le terme qu'on trouve dans la littérature est \textbf{\textit{Argument Cluster Coordination}} (\citet{Steedman2000}). La distinction entre les deux constructions, qu'il s'agisse d'une ellipse ou pas, reste un problème si on se place dans une perspective typologique (plus de détails, dans la section \ref{sec:4.2} du chapitre 4).\footnote{Selon \citet{Haspelmath2007}, le fait que certaines langues emploient des coordonnants différents dans le gapping et respectivement la réduction de conjoints, à savoir un coordonnant phrastique dans le premier cas et un coordonnant sous-phrastique dans le deuxième cas, est un argument solide pour un traitement différent des deux constructions.}  


\begin{enumerate}
\item \label{bkm:Ref302724517}a  John went to Paris on Monday [and to Rome on Friday]. 


\end{enumerate}
  b  John went [to Paris on Monday] [and [to Rome on Friday]].  

\textbf{\textit{Sluicing} }(\citet{Ross1969}),\textbf{} ou {\guillemotleft}~interrogative fragmentaire~{\guillemotright} (en français), définit toute phrase elliptique dans laquelle est réalisé un syntagme interrogatif et où manque le reste, cf. \REF{ex:3:136}. L'élément résiduel en question correspond habituellement à un argument \REF{ex:3:136}a ou ajout \REF{ex:3:136}b dans le contexte. Une étude détaillée du sluicing apparaît dans Chung \textit{et al.} (1995), Ginzburg \& \citet{Sag2000}, Merchant (2001, 2006), \citet{Chung2005}, etc.


\begin{enumerate}
\item \label{bkm:Ref306093193}a  John drinks something, but I don't know [what]. 


\end{enumerate}
  b  John will go to Paris, but I don't know [when].  

\textbf{\textit{Stripping}}\textbf{} (\citet{Ross1969}, Hankamer \& \citet{Sag1976}) caractérise toute phrase elliptique ayant un seul élément résiduel, accompagné souvent d'un adverbe. Habituellement, cet élément résiduel est un valent du verbe prédicat dans la phrase source. La construction \textit{stripping} est connue dans la littérature sous différents noms : \textit{end-attachment coordination} (cf. Huddleston \& \citet{Pullum2002}), \textit{Bare Argument Ellipsis} (\citet{Wilder1997}, Culicover \& \citet{Jackendoff2005}, etc.). Cette étiquette regroupe des constructions assez hétérogènes, plus ou moins elliptiques, plus ou moins coordonnées (voir Abeillé (2005, 2006) pour les détails).


\begin{enumerate}
\item a  John drinks scotch, [and Bill too]. 


\end{enumerate}
  b  John drinks scotch, [but not Bill].  

Ellipse des dépendants

Les constructions qui mettent en jeu une ellipse de dépendants sont toutes des variantes d'une ellipse {\guillemotleft}~post-auxiliaire~{\guillemotright}.

\textbf{\textit{Verb Phrase Ellipsis}} (dorénavant VPE), ou ellipse du syntagme verbal (en français), est de loin la construction la plus étudiée (\citet{Sag1976}, \citet{Hardt1993}, \citet{Johnson2001}, \citet{Dalrymple2005}, etc.). Dans ce type de constructions, il s'agit généralement d'une ellipse du syntagme verbal en entier, excepté le verbe auxiliaire \REF{ex:3:138}a-b ou le verbe modal \REF{ex:3:138}c. 


\begin{enumerate}
\item \label{bkm:Ref306099250}a  John drinks scotch, [but Bill doesn't].


\end{enumerate}
  b  John drinks scotch, [and Bill does too]. 

  b  John cannot drink scotch, [but Bill can].  

\textbf{\textit{Pseudo-gapping}} (\citet{Levin1986}) caractérise toute phrase elliptique dans laquelle manque une partie du syntagme verbal, mais dans laquelle on a comme éléments résiduels au moins deux dépendants plus l'auxiliaire ou le modal. Le pseudo-gapping se distingue ainsi de l'ellipse du syntagme verbal par le fait qu'il peut comporter plusieurs éléments résiduels (comme dans les constructions à gapping)\footnote{Pour les différences entre le gapping et le pseudo-gapping, voir \citet{Johnson2009}.}. Voir des détails dans \citet{Jayaseelan1990}, \citet{Lasnik1999}, etc.


\begin{enumerate}
\item a  John drinks scotch, [and Bill does bourbon]. 


\end{enumerate}
  b  John can drink scotch, [and Bill can bourbon].  

\textbf{\textit{Null Complement Anaphora} }(Hankamer \& \citet{Sag1976}),\textbf{ }ou anaphore de complément nul (en français), caractérise toute phrase elliptique dans laquelle manque le complément d'un prédicat modal, aspectuel, verbe d'attitude, etc. Une monographie de cette construction est faite dans \citet{Depiante2000}.\textbf{ }


\begin{enumerate}
\item a  I asked Bill to leave, [but he refused]. 


\end{enumerate}
  b  John could have come, [but Mary disapproved].  

\textbf{\textit{Antecedent Contained Ellipsis} }(\citet{Bouton1970}) caractérise l'ellipse qui apparaît dans une relative restrictive dont l'antécédent est dans la portée d'un quantifieur fort ou d'un défini, cf. \REF{ex:3:141}. Plus précisément, dans ce type d'ellipse, le matériel manquant est contenu dans le syntagme verbal qui sert d'antécédent. Pour plus de détails sur cette construction, voir \citet{Baltin1987}, \citet{Kennedy1997}, \citet{Lappin1999}, etc.


\begin{enumerate}
\item \label{bkm:Ref302739948}a  John tried to read everything [he could]. 


\end{enumerate}
  b  Alicia visited every town [that Beatrix did].  

Une autre variante, moins étudiée, est représentée par \textbf{\textit{Argument Contained Ellipsis} }(\citet{Kennedy1994}), caractérisant une relation elliptique entre deux syntagmes verbaux dont un est enchâssé dans l'argument de l'autre, à condition que les arguments soient identiques.


\begin{enumerate}
\item History suggests [that a proof [that God exists] never will].  


\end{enumerate}
Ellipse non-sélective

La seule construction elliptique majeure mettant en jeu une ellipse non-sélective est \textbf{\textit{Right-Node Raising}} (\citet{Postal1974}), connue également sous le nom de \textit{Right Peripheral Ellipsis} (Höhle (1991)), ou phrase à {\guillemotleft}~factorisation droite~{\guillemotright} (en français), dorénavant RNR. Ce type d'ellipse définit une relation entre deux phrases dont une elliptique à laquelle manque un dépendant ou la tête (en position finale), qui précède l'autre phrase (complète) qui détermine son interprétation. Pour plus de détails, voir \citet{Abbot1976}, \citet{Hartmann2000}, Chaves \& \citet{Sag2008}, etc.\footnote{Il faut noter que RNR se rencontre avec tous les types de syntagmes et pas seulement avec les phrases, cf. Chaves \& \citet{Sag2008} et Mouret \& Abeillé (2011).} 


\begin{enumerate}
\item a  [John made], and Mary sold a piece of furniture. 


\end{enumerate}
  b  [If the President], and if Congress act under a letter of attorney from the people, so do the judges.  

\subsubsection{Contextes syntaxiques}
\label{bkm:Ref305101970}Selon ce critère, on veut savoir quel est le domaine d'occurrence des différents types d'ellipse mentionnés plus haut, en fonction de la classification des phrases mentionnée dans le chapitre 1 (section \ref{sec:1.1}) et 2 (section \ref{sec:2.1}), à savoir phrase indépendante, phrase coordonnée ou phrase subordonnée.

A part \textit{Antecedent Contained Ellipsis} et \textit{Argument Contained Ellipsis}, toutes les autres constructions ont été largement étudiées en lien avec la coordination, car leur contexte syntaxique privilégié (voire le seul possible pour certaines d'entre elles) est la coordination.

Si certains types d'ellipse semblent être restreints à la coordination (p.ex. le gapping, cf. exemples en \REF{ex:3:144}), d'autres constructions (p.ex. VPE \REF{ex:3:145}, RNR\footnote{Contrairement à ce qu'on dit souvent (p.ex. \citet{Haspelmath2007}), RNR peut apparaître sous l'enchâssement, cf. Chaves \& \citet{Sag2008}.} \REF{ex:3:146}) peuvent apparaître en dehors des phrases coordonnées. ~


\begin{enumerate}
\item \label{bkm:Ref306114635}a  Robert cooked the first course, \textbf{and} Marie the dessert. 


\end{enumerate}
  b  *Robert cooked the first course, \textbf{because} Marie the dessert.


\begin{enumerate}
\item \label{bkm:Ref306115008}a  Joan write a novel, \textbf{and} Marvin did too. 


\end{enumerate}
  b  Joan write a novel \textbf{after} Marvin did too.  


\begin{enumerate}
\item \label{bkm:Ref306123939}a  You know a man who sells, \textbf{and} I know a man who buys, pictures of Elvis Presley. 


\end{enumerate}
  b  It seemed likely to me, \textbf{though} it seemed unlikely to everyone else, that he would be impeached.          (exemples cités par Chaves \& \citet{Sag2008})

Dans le domaine de la subordination, les constructions qu'on retient, en suivant Abeillé \textit{en prép.} pour le français, sont les comparatives elliptiques \REF{ex:3:147}a, les relatives elliptiques \REF{ex:3:147}b et les circonstancielles (additives \REF{ex:3:147}c, exceptives \REF{ex:3:147}d, concessives \REF{ex:3:147}e, conditionnelles \REF{ex:3:147}f) elliptiques. 


\begin{enumerate}
\item \label{bkm:Ref306117171}a  Jean s'est comporté [\textbf{comme} Marie (hier)]. 


\end{enumerate}
  b  Plusieurs personnes sont venues, [\textbf{dont} Paul (hier)]. 

  c  Tout le monde viendra, [\textbf{y compris} Paul].

  d  Tout le monde est venu, [\textbf{sauf} Paul].

  e  Paul participe aux réunions, [\textbf{bien que} rarement].

  f  {\guillemotleft}~Si j'épouse une femme avare, elle ne me ruinera pas ; [\textbf{si} une joueuse], elle pourra s'enrichir ; [\textbf{si} une prude], elle ne sera pas emportée ; [\textbf{si} une emportée], elle exercera ma patience ; [\textbf{si} une coquette], elle voudra me plaire ; [\textbf{si} une galante], elle le sera peut-être jusqu'à m'aimer...~{\guillemotright}

{\raggedleft
    (La Bruyère, \textit{Les caractères}, cité dans \citet{Piot1988})  
}

Il reste à décider si l'ellipse dans ces subordonnées a des propriétés spécifiques par rapport aux autres contextes. Un aspect particulièrement problématique est le rapprochement qu'on peut faire entre le gapping et les subordonnées elliptiques contenant deux éléments résiduels (p.ex. les comparatives), car le gapping est généralement considéré comme un phénomène spécifique à la coordination. Je reviendrai à ce problème dans le chapitre 4, section \ref{sec:4.3.1}.  

Enfin, les phrases indépendantes peuvent elles aussi être elliptiques. Les structures les plus étudiées sont les \textit{fragments dialogiques} (c.-à-d. questions courtes \REF{ex:3:148} -- angl. \textit{short questions} --, ou les réponses courtes \REF{ex:3:149} -- angl. \textit{short answers} -- dans le dialogue), cf. \citet{Barton1990}, Ginzburg \& \citet{Sag2000}, \citet{Schlangen2003}, \citet{Merchant2004}, Fernàndez \citet{Rovira2006}, etc.  


\begin{enumerate}
\item \label{bkm:Ref306121981}a  A : - A friend of mine will arrive tomorrow. 


\end{enumerate}
  b  B : - [Who] ?  


\begin{enumerate}
\item \label{bkm:Ref306122023}a  A : - Who attended the meeting~? 


\end{enumerate}
  b  B : - [John].  

\subsubsection{Directionnalité~}
Selon ce dernier critère, on veut voir si l'antécédent se trouve avant ou après la forme elliptique. De ce point de vue, la littérature enregistre deux types d'ellipse majeurs : (i) ellipse progressive ou analepse (angl. \textit{forward ellipsis}) si la phrase source précède la séquence elliptique, ou bien (ii) ellipse régressive ou catalepse (angl. \textit{backward ellipsis}) si la phrase source suit la séquence elliptique.

Les deux directions de l'ellipse peuvent être observées dans une même langue (à travers les différentes constructions mentionnées dans la section \ref{sec:3.3.1} ou encore à travers les différents contextes syntaxiques mentionnés dans la section \ref{sec:3.3.2}), ou bien dans une perspective typologique, à travers les langues. 

Ainsi, à l'intérieur d'une même langue, on observe que certains types d'ellipse ne peuvent apparaître que dans une des deux directions : en anglais (comme en français, roumain, etc.), le gapping dans la coordination a toujours lieu dans la phrase qui suit une phrase complète (c.-à. ellipse progressive) comme en \REF{ex:3:150}a, alors que RNR dans le même contexte syntaxique est toujours une ellipse régressive \REF{ex:3:150}b. En revanche, une construction comme VPE dans la subordination permet les deux ordres \REF{ex:3:151}. De manière générale, on peut dire que la direction de l'ellipse est déterminée de façon plus stricte dans la coordination que dans la subordination. 


\begin{enumerate}
\item \label{bkm:Ref306122971}a  John loves apples, and Mary bananas. 


\end{enumerate}
  b  Birds eat, and flies avoid long-legged spiders.  


\begin{enumerate}
\item \label{bkm:Ref306123001}a  Bill will make a statement blasting the press, if Hillary will. 


\end{enumerate}
  b  If Hillary will, Bill will make a statement blasting the press.  

Dans une perspective typologique, on considère souvent que la directionnalité de l'ellipse est liée à l'ordre des mots dans une langue. Par exemple, pour les constructions à gapping, \citet{Ross1970} établit que les langues à tête initiale présentent plutôt l'analepse, alors que les langues à tête finale présentent plutôt la catalepse. Mais cette généralisation est loin de capter tous les faits empiriques qu'on observe à travers les langues. Plus de détails dans la section \ref{sec:4.2.1} du chapitre 4.

\subsubsection{L'ellipse dans une approche typologique}
Les critères présentés précédemment nous permettent de faire les généralisations suivantes : (i) l'ellipse peut toucher tout constituant (argument, ajout ou prédicat) d'une phrase ; (ii) l'ellipse peut apparaître dans n'importe quelle position syntaxique (initiale, médiane ou finale), et (iii) l'ellipse peut apparaître soit après l'antécédent (c.-à-d. analepse), soit avant l'antécédent (c.-à-d. catalepse).

Dans une approche typologique, observer le fonctionnement de l'ellipse en suivant ces généralisations s'avère un travail difficile. Les constructions elliptiques, telles que listées dans la section \ref{sec:3.3.1}, se rencontrent dans la plupart des langues sous des formes variées (\citet{Sanders1977}, Harries-\citet{Delisle1978}, Mallinson \& \citet{Blake1981}, \citet{Haspelmath2007}). Cependant, deux aspects semblent problématiques dans une étude typologique de l'ellipse : (i) la disponibilité des formes, et (ii) les problèmes d'identification. 

\paragraph[Disponibilité des formes]{Disponibilité des formes}
En ce qui concerne la disponibilité des formes, on admet que certaines constructions ne se rencontrent pas dans certaines langues. Par exemple, il n'y aurait pas de RNR en hausa et une dizaine de langues apparentées (cf. \citet{Koutsoudas1971}), ni de gapping en thai, mandarin ou maltais (cf. \citet{Rosenbaum1977}, Mallinson \& \citet{Blake1981}), ni de VPE dans la plupart des langues romanes, contrairement aux langues germaniques et celtiques (cf. \citet{Chao1988})\footnote{Voir exceptions: VPE est possible en portugais, mais impossible en allemand ou néerlandais.}. 

Certains auteurs ont essayé d'établir des règles de prédictibilité pour la disponibilité des formes elliptiques à travers les langues.\footnote{Je retiens ici comme illustration uniquement les hypothèses de \citet{Sanders1977}. Voir aussi Zoerner \& \citet{Agbayani2000} pour une règle alternative de prédictibilité des formes elliptiques à travers les langues, liée aux différents types de mouvement du verbe disponibles dans les langues.} Ainsi, \citet{Sanders1977} arrive à six types de constructions elliptiques (dans la coordination), en fonction de la position du matériel manquant dans la phrase : trois types de catalepse, avec le matériel manquant dans différentes positions de la première phrase \REF{ex:3:152}, et trois types d'analepse, avec le matériel manquant dans la deuxième phrase \REF{ex:3:153}.


\begin{enumerate}
\item \label{bkm:Ref305327151}a  \textbf{(A)}BC \& DEF  ellipse de A    catalepse initiale  


\end{enumerate}
  b  A\textbf{(B)}C \& DEF  ellipse de B    catalepse médiane

  c  AB\textbf{(C)} \& DEF  ellipse de C    catalepse finale  


\begin{enumerate}
\item \label{bkm:Ref305327155}a  ABC \& \textbf{(D)}EF  ellipse de D    analepse initiale  


\end{enumerate}
  b  ABC \& D\textbf{(E)}F  ellipse de E    analepse médiane

  c  ABC \& DE\textbf{(F)}  ellipse de F    analepse finale  

Sur la base de cette typologie, \citet{Sanders1977} établit une hiérarchie d'accessibilité \REF{ex:3:154}, selon laquelle si un type T est accessible dans une langue, alors tous les types à sa droite le sont aussi.


\begin{enumerate}
\item \label{bkm:Ref305704096}Hiérarchie d'accessibilité pour les types d'ellipse dans la coordination  


\end{enumerate}

C

F>E

A>B>???>D


Selon \citet{Sanders1977}, cette hiérarchie rendrait compte des disponibilités des formes elliptiques dans les langues, comme illustrées en \REF{ex:3:155} :


\begin{enumerate}
\item \label{bkm:Ref305328194}a  chinois :  AB(C) -- (D)EF  


\end{enumerate}
  b  anglais :  AB(C) -- (D)(E)F

  c  quechua :  ABC -- (D)(E)(F)

  d  russe :  AB(C) -- (D)(E)(F)

  e  hindi :  A(B)(C) -- (D)(E)(F)

  c  tojolabal :  (A)(B)(C) -- (D)(E)(F)       

Son hypothèse générale est que les types elliptiques les moins accessibles sont plus difficiles à décoder. Cette hypothèse est basée sur deux observations prises en compte par des facteurs différents : 

(i) On considère que les langues utilisent l'analepse beaucoup plus que la catalepse (voir aussi \citet{Haspelmath2007}). L'explication donnée par \citet{Sanders1977} dans ce sens est liée à un facteur temporel : la catalepse est plus difficile que l'analepse dans le processing de l'énoncé si l'antécédent du matériel élidé n'a pas été encore présenté. D'ailleurs, le facteur du processing apparaît aussi chez d'autres auteurs (\citet{Ramat1987}, \citet{Hawkins1988}, Gaeta \& \citet{Luraghi2001}). 

(ii) En même temps, toujours selon \citet{Sanders1977}, les positions A et F seraient les meilleurs antécédents, alors que les positions C et D seraient les pires, d'où résulte le fait que les ellipses de D et de C seraient les plus préférées, alors que les ellipses de A et F seraient les moins privilégiées. Il explique ces préférences en termes de {\guillemotleft}~proéminence de l'antécédent~{\guillemotright} : les antécédents en début et fin de séquence sont appris plus vite et gardés en mémoire de façon plus exacte que les antécédents en position médiane. 

Cependant, la question qui reste à résoudre est par quelles propriétés de ces langues s'explique cette disponibilité des formes.

\paragraph[Problèmes d'identification]{Problèmes d'identification}
Un aspect particulièrement problématique dans une approche typologique résulte du fait que certaines constructions elliptiques sont difficiles à identifier dans certaines langues.

Pour illustrer ce problème, je reprends en \REF{ex:3:156} une version modifiée d'un tableau d'\citet{Haspelmath2007}, qui présente les configurations elliptiques les plus fréquentes dans les coordinations phrastiques à travers les langues.\footnote{Les cases vides ne signifient pas que les structures en question n'existent pas, mais simplement qu'elles ne sont pas fréquentes.} Chaque configuration est illustrée par un exemple, repris pour la plupart d'\citet{Haspelmath2007}.


\begin{enumerate}
\item \label{bkm:Ref305339477}Les types d'ellipse les plus fréquents en fonction de l'ordre des mots dans les langues  


\end{enumerate}




\begin{tabular}{llllll}

\multicolumn{2}{l}{} & SVO\par & SOV\par & VSO\par & OSV/OVS\par\\
ellipse de la tête verbale & analepse & SVO + SO \REF{ex:3:157}\par & SOV + SO \REF{ex:3:158}\par & VSO + SO\par

 \REF{ex:3:159}\par & \\
 & catalepse &  & SO + SOV\par

 \REF{ex:3:160}\par &  & \\
ellipse d'un argument & analepse &  & SOV + SV\par

 \REF{ex:3:161}\par &  & OSV + SV \REF{ex:3:162}\par

 OVS + VS \REF{ex:3:163}\par\\
 & catalepse & SV + SVO \REF{ex:3:164}\par &  & VS + VSO\par

 \REF{ex:3:165}\par & \\
\hhline{~-----}

\end{tabular}


{\raggedleft
Tableau adapté d'\citet{Haspelmath2007}
}


\begin{enumerate}
\item \label{bkm:Ref305341376}John drinks scotch and Bill bourbon.  (anglais)

\item \label{bkm:Ref305341717}Vanja  vodu  pil,  a  Masha  vino.  (russe)


\end{enumerate}
Vanja  eau  a-bu,  et  Masha  vin

\textit{Vanja a bu de l'eau, et Masha du vin} 


\begin{enumerate}
\item \label{bkm:Ref305342075}Liebt  Julia  Romeo  und  Kleopatra  Cäsar ?  (allemand)


\end{enumerate}
aime  Julia  Romeo  et  Kleopatra  Caesar

{\itshape
Est-ce que Julia aime Romeo et Kleopatra Caesar ?}


\begin{enumerate}
\item \label{bkm:Ref305342478}Ich  glaube  dass  Peter  Kartoffeln  und  Maria  Brod  ass ?  (allemand)


\end{enumerate}
je  crois  que  Peter  pommes-de-terre  et  Maria  pain  a-mangé

{\itshape
Je crois que Peter a mangé des pommes de terre et Maria du pain} 


\begin{enumerate}
\item \label{bkm:Ref305343073}Sonyen-i  swuley-lul  kul-ko  sonye-ka  mile-ss-ta.  (coréen)


\end{enumerate}
garçon.\textsc{nom}  carriole.\textsc{acc}  tire-et  fille.\textsc{nom}  pousse.\textsc{passe.decl}

{\itshape
Le garçon tire, et la fille pousse la carriole}


\begin{enumerate}
\item \label{bkm:Ref305344064}Pustakam  Raamu  vaa{\ng}{\ng}i  pak[1E63?]e  K[1E5B?][1E63?][1E47?]an  vaayiccu.  (malayalam)


\end{enumerate}
livre  Ramu  a-acheté  mais  Krishnan  a-lu

{\itshape
Ramu a acheté, mais Krishnan a lu le livre}


\begin{enumerate}
\item \label{bkm:Ref305344415}Das  Buch  kaufte  mein  Vater  und  las  meine  Mutter ?  (allemand)


\end{enumerate}
le  livre  a-acheté  mon  père  et  a-lu  ma  mère

{\itshape
Le livre a été acheté par mon père et lu par ma mère} 


\begin{enumerate}
\item \label{bkm:Ref305344617}Birds eat, and flies avoid long-legged spiders.    (anglais)

\item \label{bkm:Ref305344965}Gwelodd  Gwen,  a  rhybuddiodd  Ifor,  y  dyn.  (gallois)


\end{enumerate}
a-vu  Gwen  et  a-averti  Ifor,  le  homme

{\itshape
Gwen a vu, et Ifor a averti l'homme}

En dehors des configurations SVO + SO en \REF{ex:3:157} et SOV + SO \REF{ex:3:158}, qui sont sans aucun doute des occurrences de gapping, toutes les autres configurations se prêtent à priori à une double analyse. Ainsi, les configurations VSO + SO en \REF{ex:3:159} et SO + SOV en \REF{ex:3:160}, exemplifiées pour l'allemand, sont a priori ambiguës entre une analyse en termes de gapping (c.-à-d. une coordination de phrases, dont une elliptique) et une analyse en termes de coordination de séquences (ou \textit{Argument Cluster Coordination} en anglais, c.-à-d. une coordination sous-phrastique de pseudo-constituants dans la portée syntaxique d'un prédicat verbal). De même, les configurations qui restent, avec l'ellipse d'un argument en \REF{ex:3:161}, \REF{ex:3:162}, \REF{ex:3:163}, \REF{ex:3:164} et \REF{ex:3:165}, sont a priori ambiguës entre une analyse en termes de montée de n{\oe}ud droit/gauche (ou \textit{Right/Left-Node Raising} en anglais) et une construction à complément nul (ou \textit{Null Complement Anaphora} en anglais) ; le matériel souligné en \REF{ex:3:161}, \REF{ex:3:162}, \REF{ex:3:163}, \REF{ex:3:164} et \REF{ex:3:165} peut ainsi être interprété soit comme mis en facteur pour les deux conjoints, soit comme appartenant à un des conjoints.~

Je reviendrai à l'ambiguïté gapping vs. coordination de séquences dans le chapitre 4, section \ref{sec:4.2.2}, où je montre que les langues (en particulier, le roumain) disposent parfois des moyens pour désambiguïser le type de construction elliptique envisagée.

{\bfseries
Théories de l'ellipse~}

Dans la section \ref{sec:3.1} de ce chapitre, j'ai précisé que le phénomène de l'ellipse présente un défi majeur pour toute théorie grammaticale, en ce que la dichotomie classique forme/interprétation n'est pas respectée : dans une construction elliptique, on obtient une interprétation sans avoir de forme. La question générale qui surgit alors est comment on articule cette dichotomie classique dans le cas de l'ellipse ? 

Dans les approches générativistes traditionnelles, la composante syntaxique de la grammaire rend compte de la correspondance entre forme et interprétation. Par conséquent, la question majeure autour de laquelle ont été centrés les débats sur l'ellipse est liée à la structure syntaxique : est-ce que le matériel manquant d'une construction elliptique possède une structure syntaxique non-prononcée ? La réponse donnée à cette question a des implications significatives pour la théorie de la grammaire. Si la réponse est positive, on doit postuler une structure syntaxique plus abstraite, dans une théorie qui permette que certains n{\oe}uds syntaxiques ne correspondent à aucun élément prononcé, qu'il s'agisse d'un mot ou d'un syntagme. En revanche, si la réponse est négative, il n'y a pas de structure syntaxique abstraite, toutes les structures syntagmatiques contiennent uniquement des éléments qui sont prononcés, le principe étant {\guillemotleft}~ce qu'on entend / voit, c'est ce qu'on obtient~{\guillemotright} (angl. \textit{what you hear / see is what you get}). 

Ces deux perspectives donnent lieu à deux types d'approches, appelées par \citet{Merchant2009} approches {\guillemotleft}~structurales~{\guillemotright} vs. {\guillemotleft}~non-structurales~{\guillemotright}.~Les approches structurales considèrent que le matériel manquant a une structure en syntaxe ; la reconstruction du matériel manquant se situe donc au niveau syntaxique. De leur côté, les approches non-structurales n'attribuent pas de structure au matériel manquant, la résolution de l'ellipse se situant uniquement en sémantique ou bien à l'interface syntaxe-sémantique. Une présentation schématique des deux approches est illustrée en \REF{ex:3:166}.


\begin{enumerate}
\item \label{bkm:Ref305706464}Quelle syntaxe pour l'ellipse ? (résumé en anglais, cf. \citet{Merchant2009})  


\end{enumerate}
[Warning: Draw object ignored]

Qu'on se situe dans une approche structurale ou non-structurale, on admet généralement qu'il y a une reconstruction (en syntaxe ou en sémantique), et donc qu'il y a une ellipse (syntaxique ou sémantique). Cependant, je veux préciser qu'il y a une troisième possibilité d'analyse, qui esssaie de réduire le phénomène de l'ellipse à des mécanismes syntaxiques requis (de manière indépendante) pour d'autres phénomènes linguistiques.\footnote{Ce dernier type d'approches est appelé \textit{syntax-first approach} par Schwabe \& \citet{Winkler2003}.} Un exemple bien connu est l'opération de mouvement, qui permettrait de dériver la structure syntaxique d'une construction elliptique comme le gapping, sans postuler qu'il y a ellipse (cf. Johnson (1996/2004, 2000, 2009), Zoerner \& \citet{Agbayani2000}, López \& \citet{Winkler2003}, \citet{Winkler2005}). Ainsi, en \REF{ex:3:167}, le verbe \textit{drank} est extrait simultanément des deux conjoints (cf. angl. \textit{Across-The-Board Movement}, abrégé \textit{ATB-Movement}) et laisse deux traces \textit{t}\textit{\textsubscript{i}} (voir plus de détails dans la section \ref{sec:4.4.2.3} du chapitre 4). La construction gapping est ainsi alignée sur les constructions dites à extraction, ne mettant en jeu aucune ellipse.\footnote{Le mécanisme d'extraction a été repris aussi pour la mise en facteur à droite (abrégé RNR) et certains cas de réduction de conjoints (voir \citet{Ross1967}, \citet{Hudson1976}, \citet{Maling1972}, \citet{Sabbagh2007}, etc.). Dans ce type d'approche sans ellipse, le stripping est analysé comme un cas d'extraposition du syntagme nominal, ou encore l'ellipse comparative comme un cas de complémentation ordinaire.}


\begin{enumerate}
\item \label{bkm:Ref305708886}[\textsubscript{TP} Randy\textsubscript{j} [\textsubscript{T'} drank\textsubscript{i} [\textsubscript{vP} \textit{t}\textsubscript{j}\textit{ t}\textsubscript{i} scotch] and [\textsubscript{vP} Amy \textit{t}\textsubscript{i} rum]]].      (\citet{Johnson2009})  


\end{enumerate}
Cependant, cette approche ne s'applique qu'aux constructions avec un antécédent linguistique, car pour pouvoir extraire l'antécédent, il faut qu'il soit présent. Pour toutes les occurrences elliptiques avec un antécédent non-linguistique (ou pragmatique), cette analyse ne fonctionne pas (p.ex. les fragments dans le dialogue). Comme la notion d'ellipse n'est pas pertinente pour ce type d'approches, j'ai voulu simplement signaler l'existence d'une telle analyse. Par la suite, je me concentre plutôt sur les deux approches mettant en jeu une ellipse syntaxique ou sémantique. 

{\bfseries
Approches structurales ou Ellipse syntaxique}

L'hypothèse majeure faite dans ce type d'approches est la suivante : le matériel manquant a une certaine structure syntaxique à un certain niveau de la représentation. Il est donc présent, mais pas prononcé (on a ainsi une correspondance inhabituelle entre la syntaxe et la phonologie). Par conséquent, on n'a pas besoin d'une règle d'interprétation spécifique ; l'interprétation est dérivée par les mêmes mécanismes agissant dans les contextes non-elliptiques. 

On aligne ainsi les phrases elliptiques sur leurs contreparties non-elliptiques, en considérant que les deux ont la même structure et obéissent aux mêmes règles. Comme le matériel manquant est {\guillemotleft}~non-prononcé~{\guillemotright}, il n'y a pas de moyen direct qui prouve sa présence. Mais on peut détecter sa présence en syntaxe de manière indirecte, en vertu des effets de sa présence sur les éléments résiduels. Si l'on trouve des effets qui semblent être due au matériel manquant, alors cela est un argument pour supposer la présence du matériel manquant dans la structure syntaxique. Si, en revanche, les effets auxquels on s'attend n'apparaissent pas, alors on doit conclure que le matériel manquant n'est pas présent en syntaxe.

\paragraph[Arguments pour la présence d'une structure]{Arguments pour la présence d'une structure}
Les approches structurales sont ainsi basées sur ce qu'on appelle les {\guillemotleft}~effets de connectivité~{\guillemotright} (angl. \textit{connectivity effects}, cf. Merchant (2001, 2004) : la phrase elliptique présente des {\guillemotleft}~connexions~{\guillemotright} grammaticales avec la forme de la phrase source), alors que les approches non-structurales sont fondées sur les effets de non-connectivité. Si les contraintes syntaxiques continuent à s'appliquer aux constructions elliptiques, cela serait un argument pour la résolution de l'ellipse au niveau syntaxique.

Je reprends par la suite certains des effets syntaxiques mentionnés par Merchant (2004, 2009), qui semblent suggérer la présence d'une structure ordinaire, mais non-prononcée, dans les constructions elliptiques.

(i) Un effet syntaxique observé est le marquage casuel des éléments résiduels dans les phrases elliptiques (Ross (1967, 1969), Merchant (2001, 2004)), qui montre que les assigneurs de cas sont syntaxiquement présents. Ainsi, dans l'exemple allemand repris de \citet{Ross1969} en \REF{ex:3:168}, le mot \textit{qu-} apparaissant dans la phrase elliptique avec sluicing reçoit exactement les mêmes marques casuelles que son corrélat dans la phrase source : le cas datif requis par le verbe \textit{schmeicheln} `flatter' en \REF{ex:3:168}a et respectivement le cas accusatif imposé par le verbe \textit{loben} `féliciter' en \REF{ex:3:168}b. 


\begin{enumerate}
\item \label{bkm:Ref305744740}a  Er  will  jemand\textbf{em}  schmeicheln,  aber  sie  wissen  nicht  \{*wer {\textbar} *wen {\textbar} w\textbf{em}\}.


\end{enumerate}
  il  veut  quelqu'un.\textsc{dat}  flatter,  mais  ils  savent  \textsc{neg}  \{qui.\textsc{nom} {\textbar} \textsc{acc {\textbar} dat\}}

\textit{  Il veut flatter quelqu'un, mais ils ne savent pas qui}  

  b  Er  will  jemand\textbf{en}  loben,  aber  sie  wissen  nicht  \{*wer  {\textbar} w\textbf{en}  {\textbar} *wem\}.

    il  veut  quelqu'un.\textsc{acc}  féliciter,  mais  ils  savent  \textsc{neg}  \{qui.\textsc{nom}  {\textbar} \textsc{acc  {\textbar} dat\}}

    \textit{Il veut féliciter quelqu'un, mais ils ne savent pas qui}

(ii) On observe que les contraintes dites de localité (en particulier, les contraintes d'îles) régissent l'extraction des éléments se trouvant à l'intérieur du matériel manquant. Si ces contraintes sont syntaxiques (cf. \citet{Sag1976}, alors leur applicabilité dans le domaine de l'ellipse est un argument en faveur d'une représentation syntaxique non-prononcée. Je reprends en \REF{ex:3:169} deux exemples de \citet{Merchant2009} illustrant l'îlot relatif : l'extraction d'un constituant en dehors d'une phrase relative est interdite avec l'ellipse du syntagme verbal (abrégé VPE) en \REF{ex:3:169}a ou encore le stripping en \REF{ex:3:169}b.


\begin{enumerate}
\item \label{bkm:Ref305742766}a  *Abby knows five people [who have \textbf{dogs}], but \textbf{cats}, she doesn't. 


\end{enumerate}
  b  *They caught the man [who'd stolen \textbf{the car}] after searching for him, but not \textbf{the diamonds}.  

De manière plus générale, dans ce type d'approches, le test de l'extraction est pris comme le diagnostic d'une structure syntaxique {\guillemotleft}~riche~{\guillemotright}, car si l'extraction est permise dans une construction elliptique, le site de l'ellipse doit contenir suffisamment de structure syntaxique pour qu'il y ait une place pour la {\guillemotleft}~trace~{\guillemotright} du mouvement.~ 

(iii) Merchant (2001, 2004, 2009) observe une corrélation typologique entre la possibilité d'avoir une préposition {\guillemotleft}~orpheline~{\guillemotright} (angl. \textit{preposition-stranding}) dans les contextes non-elliptiques avec extraction et aussi dans les contextes elliptiques (avec sluicing ou réponses courtes) d'une part, et l'antéposition obligatoire du syntagme prépositionnel en entier (angl. \textit{pied-piping}) dans les contextes non-elliptiques avec extraction et dans les contextes elliptiques (avec sluicing ou réponses courtes) d'autre part. Ainsi, dans l'exemple anglais repris de \citet{Merchant2009} en \REF{ex:3:170}, on observe qu'une langue comme l'anglais, qui permet une préposition orpheline dans un contexte non-elliptique avec extraction comme en \REF{ex:3:170}b, le permet aussi dans un contexte elliptique avec sluicing \REF{ex:3:170}a ou réponse courte \REF{ex:3:170}c. En revanche, une langue comme le roumain, qui n'accepte pas une préposition orpheline dans les contextes habituels d'extraction \REF{ex:3:171}b, mais qui exige l'antéposition du syntagme prépositionnel en entier \REF{ex:3:171}c, a le même comportement dans les contextes elliptiques avec sluicing \REF{ex:3:171}a ou réponses courtes \REF{ex:3:171}d. Par conséquent, on considère que cette contrainte syntaxique liée à la distribution de la préposition opère de manière uniforme dans les contextes elliptiques et non-elliptiques à travers les langues.


\begin{enumerate}
\item \label{bkm:Ref305743542}a  Peter was talking with someone, but I don't know \textbf{(with) who(m)}. 


\end{enumerate}
  b  A : - \textbf{Who} was he talking \textbf{with} ?

  c  B : - (\textbf{With})\textbf{} Mary.  


\begin{enumerate}
\item \label{bkm:Ref305744086}a  Ion  vorbea  cu  cineva,  dar  nu  ştiu  \textbf{*(cu)  cine}.


\end{enumerate}
  Ion  parlait  avec  quelqu'un,  mais  \textsc{neg}  sais  avec  qui 

\textit{  Ion parlait avec quelqu'un, mais je ne sais pas avec qui}  

  b  *\textbf{Cine } vorbea  Ion  \textbf{cu~}?

    qui  parlait  Ion  avec

    \textit{Avec qui parlait Ion ?} 

  c  A : - \textbf{Cu cine} vorbea Ion ?

    \textit{Avec qui parlait Ion ?}

  d  B : - *(\textbf{Cu})\textbf{} Maria.

    \textit{Avec Maria}

(iv) Une corrélation supplémentaire est observée entre la distribution de l'infinitif dans les contextes non-elliptiques avec clivage d'une part, et son comportement dans les réponses elliptiques d'autre part. Seuls les verbes infinitifs à contrôle peuvent être clivés et c'est le seul type d'infinitifs qui est compatible avec un emploi dans les réponses elliptiques, comme le montrent les exemples en \REF{ex:3:172} et \REF{ex:3:173} repris de \citet{Merchant2009} : un infinitif à montée ne peut ni être clivé \REF{ex:3:172}a, ni apparaître dans une réponse courte \REF{ex:3:172}b, alors qu'un infinitif à contrôle peut apparaître dans les deux contextes \REF{ex:3:173}a-b. 


\begin{enumerate}
\item \label{bkm:Ref305746826}a  *It's [to procrastinate] that people tend.


\end{enumerate}
  b  A : - How do people tend to behave ?

    B : - *To procrastinate.  


\begin{enumerate}
\item \label{bkm:Ref305746829}a  It's [to get a job in Europe] that she really wants. 


\end{enumerate}
  b  A : - What does she really want ?

    B : - To get a job in Europe.  

Les effets de connectivité retenus ici montrent que le matériel manquant obéit dans beaucoup de cas aux mêmes contraintes syntaxiques que son antécédent et/ou que la séquence elliptique se comporte de la même façon qu'une séquence non-elliptique par rapport à certains mécanismes syntaxiques. 

\paragraph[Sous-types d'approches structurales]{Sous-types d'approches structurales}
Comme illustré dans le schéma présenté plus haut en \REF{ex:3:166}, les approches structurales se divisent en deux sous-types majeurs, en fonction du statut du matériel manquant non-prononcé : élément lexical ordinaire ou bien élément vide. Le premier type d'approches est identifié dans la littérature avec l'analyse par effacement phonologique (angl. \textit{PF-deletion approach}), alors que le deuxième est considéré comme étant l'analyse en termes de proforme nulle (avec deux sous-types, que je listerai plus bas).

{\bfseries
Effacement phonologique}

Le premier type mentionné constitue la solution traditionnelle utilisée dans les grammaires génératives, cf. Ross (1967, 1969), \citet{Sag1976}, Hankamer \& \citet{Sag1976}, \citet{Hankamer1979}, \citet{Lasnik1999}, \citet{Hartmann2000}, \citet{Merchant2001}, \citet{Chung2005}, etc.

Une phrase elliptique a la même syntaxe que sa contrepartie complète. La différence apparaît uniquement au niveau de la forme phonologique (angl. \textit{Phonological Form}, abrégé PF), où une partie de la structure est dépourvue de son contenu phonologique, en vertu de la présence d'un antécédent qui permet la reconstruction de ce contenu. Le matériel manquant correspond donc à un élément lexical, qui est ensuite {\guillemotleft}~effacé sous identité~{\guillemotright} (angl. \textit{deletion under identity}).\footnote{La description de ce type d'approches est présentée ici d'une manière simplifiée. En réalité, cette solution syntaxique de l'ellipse a plusieurs versions, que je résume en reprenant \citet{Merchant2009}. La non-prononciation du matériel présent en syntaxe est due : (i) soit à une opération d'effacement (qui se produit (a) en syntaxe avant l'Epel -- angl. \textit{Spell-Out} --, ou bien (b) après l'Epel dans la dérivation vers PF), cf. p.ex. \citet{Sag1976}, (ii) soit à un réflexe phonologique d'un algorithme prosodique, cf. p.ex. \citet{Merchant2001}.}  

La représentation d'une phrase avec sluicing dans une approche par effacement phonologique est donnée en \REF{ex:3:174} : l'élément résiduel correspondant au mot \textit{qu- what} est extrait en périphérie gauche de la phrase (déplacé à l'extérieur du TP, cf. angl. \textit{Tense Phrase}), laissant à sa place une trace \textit{t}\textit{\textsubscript{1}}. Le TP est ensuite effacé sous identité avec l'antécédent \textit{John can play} dans la phrase source. L'arbre simplifié correspondant à la phrase elliptique figure en \REF{ex:3:175}. L'effacement phonologique est déclenché par un trait ellipse E, figurant sur la tête autorisant l'ellipse (p.ex. C pour l'ellipse de TP dans le sluicing, T pour VPE).  


\begin{enumerate}
\item \label{bkm:Ref305877748}John can play something, but I don't know [\textsubscript{CP} what\textit{\textsubscript{1}} [\textsubscript{TP} John can play \textit{t}\textit{\textsubscript{1}}]].  

\item \label{bkm:Ref305948516}Arbre simplifié de la phrase elliptique en \REF{ex:3:174}  C 


\end{enumerate}
{   [Warning: Image ignored] % Unhandled or unsupported graphics:
%\includegraphics[width=1.9862in,height=1.5189in,width=\textwidth]{fe443409cd384d3fb0f6390ffd77f513-img36.svm}
} 

{\bfseries
Proformes nulles}

Le deuxième type d'approche structurale postule en syntaxe des éléments vides (ou proformes nulles), remplacés ou identifiés à un certain niveau de la représentation qui n'est pas pertinent pour la prononciation, p.ex. en forme logique (angl. \textit{Logical Form}, abrégé LF) ou dans une composante sémantique/pragmatique.

L'élément manquant correspond donc en syntaxe à un élément vide. Certains travaux (\citet{Hardt1993}, \citet{Lobeck1995}, etc.) postulent un seul élément vide \textit{e}, indépendamment de la taille du matériel manquant, comme en \REF{ex:3:176}, alors que d'autres (\citet{Wasow1972}, \citet{Ludlow2005}, etc.) postulent une pléthore d'éléments vides, comme en \REF{ex:3:177}.


\begin{enumerate}
\item \label{bkm:Ref305950047}John can play something, but I don't know [\textsubscript{CP} what [\textsubscript{IP} \textit{e}]].  

\item \label{bkm:Ref305950109}John can play something, but I don't know [\textsubscript{CP} what\textit{\textsubscript{4}} [\textsubscript{IP} \textit{e}\textit{\textsubscript{1} }\textit{e}\textit{\textsubscript{2} }\textit{e}\textit{\textsubscript{3} }\textit{t}\textit{\textsubscript{4}}]].  


\end{enumerate}
L'interprétation assignée à cet élément vide est fournie par un antécédent, mais de deux façons différentes : dans certains travaux (\citet{Wasow1972}, \citet{Shopen1972}, \citet{Williams1977}, Zribi-\citet{Hertz1986}, \citet{Hardt1993}, \citet{Lobeck1995}, \citet{Depiante2000}, \citet{Ludlow2005}, etc.)), l'interprétation est obtenue par des algorithmes non-syntaxiques, purement sémantiques, similaires à ceux utilisés pour l'interprétation des anaphores ordinaires, par exemple les pronoms ; dans d'autres travaux (Fiengo \& \citet{May1994}, Chung \textit{et al.} (1995), \citet{Lappin1999}, Beavers \& \citet{Sag2004}, \citet{Fortin2007}, etc.), l'antécédent est copié dans le site de l'ellipse en forme logique (c.-à-d. au niveau LF), assurant à l'élément vide la bonne interprétation. Les deux types de résolution de l'ellipse sont connus dans la littérature sous le nom d'{\guillemotleft}~anaphore nulle~{\guillemotright} (angl. \textit{Null Anaphora}) et respectivement {\guillemotleft}~copie en forme logique~{\guillemotright} (angl. \textit{LF}-\textit{copy}).

L'argument majeur pour une approche en termes de proforme nulle vient des ressemblances qu'on observe entre le site de l'ellipse et le comportement des pronoms, ce qui laisse penser que la relation qui s'établit entre le matériel manquant et son antécédent est une relation anaphorique. Je liste par la suite quatre ressemblances observées entre certaines constructions elliptiques (ici, VPE) et les pronoms.  

(i) Comme les pronoms, le matériel manquant peut avoir un antécédent éparpillé (angl. \textit{split antecedents}), cf. les exemples en \REF{ex:3:178} repris de \citet{Hardt1993} : en \REF{ex:3:178}a, le pronom \textit{they} a un antécédent composé de deux syntagmes nominaux (\textit{Brian} et respectivement \textit{Jill}) se trouvant à des niveaux syntaxiques différents ; de même, dans l'exemple \REF{ex:3:178}b avec VPE, l'antécédent du matériel manquant correspond aux deux séquences soulignées (\textit{walk and ... chew gum}).


\begin{enumerate}
\item \label{bkm:Ref305952265}a  Brian\textsubscript{i} told Jill\textsubscript{j} that they\textsubscript{i+j} could go away together. 


\end{enumerate}
  b  I can walk and I can chew gum. Gerry can too, but not at the same time.  

(ii) Comme les proformes déictiques, le matériel manquant peut parfois avoir un antécédent non-linguistique (ou pragmatique), cf. les exemples \REF{ex:3:179} :


\begin{enumerate}
\item \label{bkm:Ref305952763}a  [\textit{On receiving a present :}] You shouldn't have.


\end{enumerate}
  b  [\textit{At a door :}] May I ?  

(iii) Comme les pronoms, certains types d'ellipse peuvent avoir un emploi cataphorique s'ils apparaissent sous enchâssement. Le matériel manquant peut ainsi précéder son antécédent, à condition qu'il s'agisse d'une phrase elliptique enchâssée (angl. \textit{Backwards Anaphora} \textit{Constraint}), ce qui explique les différences dans les jugements d'acceptabilité pour les exemples en \REF{ex:3:180} et \REF{ex:3:181} repris de \citet{Kehler2000}. 


\begin{enumerate}
\item \label{bkm:Ref305953749}a  \textbf{If} Hillary will, Bill will make a statement blasting the press. 


\end{enumerate}
  b  \textbf{If} he\textit{\textsubscript{i}} makes a statement blasting the press, Bill\textit{\textsubscript{i}} will make a fool of himself.  


\begin{enumerate}
\item \label{bkm:Ref305953751}a  *Bill will, if Hillary will make a statement blasting the press. 


\end{enumerate}
  b  *He\textit{\textsubscript{i}} will make a fool of himself, if Bill\textit{\textsubscript{i}} makes a statement blasting the press.  

(iv) La phrase source contenant l'antécédent d'un pronom ou d'une ellipse de type VPE peut être assez éloignée de la phrase contenant le pronom ou le site de l'ellipse, cf. \citet{Hardt1993}. Cela est illustré dans l'exemple \REF{ex:3:182} repris de \citet{Kehler2000} : l'antécédent (\textit{stifle it}) de l'ellipse verbale (\textit{he couldn't}) ne se trouve pas dans la phrase précédant immédiatement la phrase elliptique ; de même, pour l'antécédent de la dernière proforme \textit{he}.


\begin{enumerate}
\item \label{bkm:Ref305954012}The thought came back, the one nagging at him these past four days. He tried to \textit{stifle it}. But the words were forming. He knew he \textit{couldn't}.  


\end{enumerate}
Dans une approche en termes de proforme nulle, les pronoms, et les anaphores de manière plus générale, sont réécrits en forme logique comme des variables (en utilisant les lambda-calculs), cf. \citet{Williams1977}. Cela permet de dériver directement l'interprétation relâchée (angl. \textit{sloppy reading}) des pronoms se trouvant dans la phrase source, comme c'est le cas des exemples \REF{ex:3:183} et \REF{ex:3:184} repris de \citet{Winkler2006}. Une approche par effacement doit postuler des mécanismes supplémentaires pour dériver l'interprétation relâchée. D'ailleurs, la possibilité d'une telle interprétation est problématique pour les approches postulant une identité syntaxique stricte entre l'antécédent et le matériel manquant.


\begin{enumerate}
\item \label{bkm:Ref305954986}a  A : - Do you think they will like \textbf{me} ?


\end{enumerate}
  b  B : - Yes, I'm sure they will. (like \textbf{you})  


\begin{enumerate}
\item \label{bkm:Ref305955037}John\textit{\textsubscript{i}} visits his\textit{\textsubscript{i}} children on Sunday and Bill does too.  


\end{enumerate}
a  Bill\textit{\textsubscript{k}} visits his\textit{\textsubscript{i}} children on Sunday (interprétation stricte) 

b    Bill\textit{\textsubscript{k}} visits his\textit{\textsubscript{k}} children on Sunday (interprétation relâchée)

Cette analyse offre ainsi une solution simple au puzzle de l'ellipse avec une interprétation relâchée (angl. \textit{sloppy ellipsis puzzle}, cf. \citet{Hardt1999}, \citet{Schwarz2000}). L'exemple \REF{ex:3:185} peut avoir deux interprétations : (i) une interprétation stricte, si le syntagme verbal elliptique reçoit la même interprétation que le syntagme verbal le plus enchâssé dans son antécédent, (ii) une interprétation relâchée, si l'interprétation du syntagme verbal elliptique vient en partie d'un antécédent dans la première phrase (\textit{want me to}) et en partie d'un antécédent dans la deuxième phrase (\textit{kiss you}). L'interprétation relâchée est obtenue uniquement s'il y a ellipse dans la première phrase (comparer \REF{ex:3:185} et \REF{ex:3:186}). Cela montre, selon les deux auteurs, que le syntagme verbal le plus enchâssé se comporte comme une variable en sémantique. 


\begin{enumerate}
\item \label{bkm:Ref305539527}I'll help you if you want me to. I'll kiss you even if you don't {\textless} {\textgreater}.  


\end{enumerate}
a  {\textless} {\textgreater} = {\textless} want me to \textbf{help} you {\textgreater} (interprétation stricte) 

b    {\textless} {\textgreater} = {\textless} want me to \textbf{kiss} you {\textgreater} (interprétation relâchée)


\begin{enumerate}
\item \label{bkm:Ref305539809}I'll help you if you want me to help you. I'll kiss you even if you don't {\textless} {\textgreater}.  


\end{enumerate}
a  {\textless} {\textgreater} = {\textless} want me to \textbf{help} you {\textgreater} (interprétation stricte) 

b    {\textless} {\textgreater} ${\neq}$ {\textless} want me to \textbf{kiss} you {\textgreater} (pas d'interprétation relâchée) 

{\bfseries
Approches non-structurales ou Ellipse sémantique}

Contrairement aux approches structurales faisant appel au mécanisme de reconstruction syntaxique, les approches non-structurales envisagent plutôt une reconstruction sémantique, en enrichissant la théorie des interprétations et en exploitant les moyens qui permettent de générer une interprétation en l'absence d'une structure syntaxique (ce qui justifie le terme d'{\guillemotleft}~interprétation directe~{\guillemotright}, attribué par \citet{Merchant2004} à ce type d'approches). L'ellipse est ainsi expliquée dans une théorie plus générale de la récupération d'information.

Dans ce type d'approches, la structure syntaxique correspond à ce qui est donné par la phonologie, c.-à-d. ce qu'on entend / voit, c'est ce qu'on obtient (angl. \textit{what you hear / see is} \textit{what you get}) ; par conséquent, il n'y a pas de matériel {\guillemotleft}~manquant~{\guillemotright} en syntaxe, qui soit présent sous la forme d'un élément effacé ou encore sous la forme d'un élément vide. Par exemple en \REF{ex:3:187}, le syntagme interrogatif\textit{ what} reçoit une structure syntaxique fragmentaire associée à une règle d'interprétation et à des contraintes de parallélisme.


\begin{enumerate}
\item \label{bkm:Ref305957345}John can play something, but I don't know [\textsubscript{S} what].  


\end{enumerate}
La perspective syntaxique envisagée par ce type d'approches est celle résumée par Culicover \& \citet{Jackendoff2005} en \REF{ex:3:188} :  


\begin{enumerate}
\item \label{bkm:Ref305957581}\textit{Simpler Syntax Hypothesis} (cf. Culicover \& \citet[5]{Jackendoff2005})


\end{enumerate}
The most explanatory syntactic theory is one that imputes the minimum syntactic structure necessary to mediate between phonology and meaning.  

Parmi les travaux qui s'inscrivent dans cette perspective, je cite Sag \textit{et al.} (1985), \citet{Gardent1991}, Dalrymple \textit{et al.} (1991), Ginzburg \& \citet{Sag2000}, \citet{Schlangen2003}, Culicover \& \citet{Jackendoff2005}, \citet{Stainton2006}.

\paragraph[Arguments pour l'absence d'une structure syntaxique]{Arguments pour l'absence d'une structure syntaxique}
Pour argumenter leur position, les approches non-structurales font appel aux effets de non-connectivité : les effets syntaxiques auxquels on s'attend n'apparaissent pas dans certaines constructions elliptiques, ce qui montre que le matériel {\guillemotleft}~manquant~{\guillemotright} dans ces constructions elliptiques n'a pas de structure syntaxique (ou au moins pas de structure avec les mêmes propriétés que la contrepartie non-elliptique).

(i) Un premier argument pour une telle approche vient de l'absence des effets de localité, cf. Culicover \& \citet{Jackendoff2005}, \citet{Stainton2006}. Dans certains contextes elliptiques (sluicing, certaines réponses courtes, certains cas de gapping, certains cas d'ellipse dans les comparatives), on observe des violations des contraintes d'îles. Ainsi, dans les exemples \REF{ex:3:189} repris de Culicover \& \citet{Jackendoff2005}, l'extraction hors d'une subordonnée relative est possible.\footnote{D'ailleurs, la nature syntaxique des contraintes de localité a été récemment contestée. Voir \citet{Kluender1998}, Fanselow \& \citet{Frisch2006}, Ambridge \& \citet{Goldberg2008}, Hofmeister \& \citet{Sag2010}, etc.}


\begin{enumerate}
\item \label{bkm:Ref305362907}a  Bob found a plumber [who fixed the sink], but I'm not sure \textbf{with what}. 


\end{enumerate}
  b  A : - John met a woman [who speaks French]. B : - \textbf{And Bengali} ?

  c  Robin knows a lot reasons [why dogs are good pets], \textbf{and} \textbf{Leslie, cats}.  

(ii) Bien que la généralisation sur les prépositions orphelines, telle qu'illustrée plus haut en \REF{ex:3:170} et \REF{ex:3:171}, s'applique à une variété de langues, on note des exceptions. Ainsi, en italien (cf. \citet{Merchant2001}), certains contextes elliptiques sont compatibles avec une préposition orpheline, alors que cela est impossible dans les contextes non-elliptiques. Ainsi, on peut avoir un syntagme \textit{qu-} sans préposition à la place d'un syntagme prépositionnel dans les ellipses de type sluicing, comme en \REF{ex:3:190}a, alors que l'antéposition du syntagme interrogatif sans préposition est impossible dans les contextes non-elliptiques à extraction, cf. \REF{ex:3:190}b. Le même comportement est observé en français, cf. \REF{ex:3:191}.


\begin{enumerate}
\item \label{bkm:Ref305960276}a  Pietro  ha  parlato  con  qualcuno,  ma  non  so  \textbf{?(con)  chi}.


\end{enumerate}
  Pietro  a  parlé  avec  quelqu'un,  mais  \textsc{neg}  sais  avec  qui

\textit{  Pietro a parlé avec quelqu'un, mais je ne sais pas avec qui}  

  b  *\textbf{Chi}  ha  parlato  Pietro  \textbf{con} ?

    qui  a  parlé  Pietro  avec

    \textit{Avec qui a parlé Pietro ?}


\begin{enumerate}
\item \label{bkm:Ref306366831}a  Pierre a parlé avec quelqu'un, mais je ne sais pas (\textbf{avec}) \textbf{qui}. 


\end{enumerate}
  b  *\textbf{Qui} Paul a-t-il parlé \textbf{avec} ? 

(iii) Parfois, le matériel nécessaire pour la reconstruction syntaxique n'est pas disponible dans la phrase source ; l'antécédent ne peut pas fournir le matériel syntaxique approprié pour la reconstruction. Les deux exemples discutés dans la littérature concernent les asymétries de voix et les asymétries de catégorie syntaxique qu'on peut observer entre le matériel qu'on doit reconstruire dans la phrase elliptique et l'antécédent dans la phrase source. 

Ainsi, on note des exemples dans lesquels le matériel à reconstruire doit contenir un verbe à la voix active, alors que son antécédent est un verbe à la voix passive, comme illustré par \REF{ex:3:192}a repris de \citet{Dalrymple2005}. L'exemple \REF{ex:3:192}b, repris de \citet{Merchant2008}, illustre la situation inverse : le matériel reconstruit doit contenir une forme passive, alors que son antécédent est un verbe à la voix active. 


\begin{enumerate}
\item \label{bkm:Ref305960735}a  In March, four fireworks manufacturers asked that the decision \textbf{be reversed}, and on Monday the ICC did. (\textbf{reverse} the decision) 


\end{enumerate}
  b  The janitor must \textbf{remove} the trash whenever it is apparent that it should be. (\textbf{removed}) 

En ce qui concerne les discordances de catégorie syntaxique, on observe que parfois un nominal fonctionne comme antécédent pour un verbe, comme dans les exemples \REF{ex:3:193}a et \REF{ex:3:193}b, repris de \citet{Dalrymple2005} et respectivement \citet{Kennedy2003}. L'approche non-structurale ne recontre pas de problèmes si elle postule que la nominalisation évoque une représentation événementielle.


\begin{enumerate}
\item \label{bkm:Ref305961105}a  This letter deserves \textbf{a response}, but before you do ... (\textbf{respond}). 


\end{enumerate}
  b   In yesterday's elections, only 43 percent of registered \textbf{voters} did. (\textbf{vote})

(iv) Parfois la reconstruction syntaxique rend les phrases agrammaticales, car elle donne lieu à des violations de contraintes syntaxiques, p.ex. les principes du liage\footnote{Les trois principes du liage, connus comme principes A, B et C, sont (cf. \citet{Lobeck1995}) :
(i)  Principe A : Les anaphores doivent être liées dans leur catégorie gouvernante.
(ii)  Principe B : Les pronoms doivent être libres dans leur catégorie gouvernante.
(iii)  Principe C : Les expressions nominales référentielles doivent être libres.}.

Dans certains contextes (avec des relations discursives de type cause-effet, cf. \citet{Kehler2000}), l'interprétation stricte est disponible même avec des réfléchis, cf. \citet{Hardt1993}, ce qui représente une violation du Principe A. La reconstruction syntaxique de l'antécédent en \REF{ex:3:194}\footnote{Les exemples sont repris de \citet{Kehler2000} et \citet{Dalrymple2005}.} rend la phrase agrammaticale, car le réfléchi présent sur le site de l'ellipse n'est pas lié dans son domaine. 


\begin{enumerate}
\item \label{bkm:Ref305962748}a  Bill\textit{\textsubscript{i}} defended \textbf{himself}\textbf{\textit{\textsubscript{i}}} against the accusations because [his lawyer]\textit{\textsubscript{k}} couldn't. (*defend \textbf{himself}\textbf{\textit{\textsubscript{i}}}) 


\end{enumerate}
  b  John\textit{\textsubscript{i}} voted for \textbf{himself}\textbf{\textit{\textsubscript{i}}} even though [no one else]\textit{\textsubscript{k}} did. (*vote for \textbf{himself}\textbf{\textit{\textsubscript{i}}})

De même, le principe B semble ne pas s'appliquer dans certaines constructions elliptiques si la relation discursive qui s'établit entre la phrase source et la phrase elliptique est de type cause-effet, cf. Kehler (1994, 2000). Dans les exemples \REF{ex:3:195} repris de \citet{Kehler2000}, la reconstruction syntaxique de l'antécédent a comme effet la coréférence entre des arguments pronominaux dans la phrase elliptique, coréférence qui d'ailleurs est interdite dans les phrases ordinaires.


\begin{enumerate}
\item \label{bkm:Ref305962998}a  John's\textit{\textsubscript{i}} mother introduced \textbf{him}\textbf{\textit{\textsubscript{i}}} to everyone because he\textit{\textsubscript{i}} wouldn't. (*introduce \textbf{him}\textbf{\textit{\textsubscript{i}}}) 


\end{enumerate}
  b  John's\textit{\textsubscript{i}} lawyer defended \textbf{him}\textbf{\textit{\textsubscript{i}}} because he\textit{\textsubscript{i}} couldn't. (*defend \textbf{him}\textbf{\textit{\textsubscript{i}}})

Enfin, dans d'autres contextes, la reconstruction syntaxique de l'antécédent implique la présence d'une expression référentielle dans le site de l'ellipse et d'un élément coréférent qui la c-commande, ce qui rendrait la phrase agrammaticale, car il y a violation du principe C~(un pronominal ne peut pas c-commander une expression coréférentielle). Fiengo \& \citet{May1994} ont proposé, comme solution à ce problème, un mécanisme de `vehicle change'~(c.-à-d. une expression référentielle peut avoir le comportement syntaxique d'un pronom dans la phrase elliptique reconstruite), mais cette solution rencontre des difficultés (cf. \citet{Dalrymple2005}). 


\begin{enumerate}
\item \label{bkm:Ref306042682}a  I shaved \textbf{Bill}\textbf{\textit{\textsubscript{i}}} because \textbf{he}\textbf{\textit{\textsubscript{i}}} wouldn't (*shave \textbf{Bill}\textbf{\textit{\textsubscript{i}}}). 


\end{enumerate}
  b  Someone likes \textbf{Janet}\textbf{\textit{\textsubscript{i}}}, but only \textbf{she}\textbf{\textit{\textsubscript{i}}} knows who (*likes \textbf{Janet}\textbf{\textit{\textsubscript{i}}}).

(v) Parmi les effets de connectivité, j'ai mentionné le marquage casuel (voir les exemples en \REF{ex:3:168}). Cependant, \citet{Stainton2006} note le fait qu'en allemand, dans certains contextes (en particulier, les réponses elliptiques dans le dialogue), l'élément résiduel peut recevoir la marque du nominatif (le cas par défaut) à la place de l'accusatif, comme c'est le cas de la réponse courte en \REF{ex:3:197}b, dont l'interprétation est \REF{ex:3:197}c.\footnote{Consulter \citet{Stainton2006} pour avoir les détails des contextes dans lesquels ont lieu les interventions figurant en \REF{ex:3:197}a et \REF{ex:3:197}b. }  


\begin{enumerate}
\item \label{bkm:Ref305964211}a  Die  Lampe  erinnert  mich  an  meinen  Onkel  Wolfram.  


\end{enumerate}
  la  lampe  rappelle  moi  à  mon.\textsc{acc}  oncle  Wolfram

{\itshape
La lampe me rappelle mon oncle Wolfram} 

b  \{Mein  {\textbar} \#Meinen\}  Vater !

  \{mon.\textsc{nom  {\textbar}} mon.\textsc{acc\}}  père  

{\itshape
  Mon père !}

c  Das  erinnert  mich  an  meinen  Vater.

  cela  rappelle  moi  à  mon.\textsc{acc}  père  

  \textit{Cela me rappelle mon père}

(vi) Un autre problème des approches structurales concerne la non-détermination~des ellipses qui tirent leur antécédent directement de la situation en cours ou de la connaissance du monde, sans avoir d'antécédent proprement linguistique. Il s'agit des ellipses avec un antécédent pragmatique (angl. \textit{situation-based ellipsis}, cf. Reich \textit{à paraître}), illustrées dans les exemples \REF{ex:3:198} et \REF{ex:3:199} repris de Reich \textit{à paraître}. Pour un fragment comme en \REF{ex:3:198}, une approche structurale (en termes d'effacement, par exemple) doit supposer plusieurs possibilités de reconstruction (voir \REF{ex:3:198}a et \REF{ex:3:198}b). En \REF{ex:3:199}, l'interprétation du participe \textit{decaf} doit être enrichie avec de l'information pertinente qui est implicite dans le contexte non-linguistique, afin d'obtenir le type sémantique (proposition, question, visée) et/ou l'acte illocutoire (assertion, interrogation, injonction).


\begin{enumerate}
\item   \label{bkm:Ref305675176}Tall decaf cappucino. 


\end{enumerate}
  a  \textit{I'd like to have a tall decaf cappucino}.

  b  \textit{Give me a tall decaf cappucino, please}. 


\begin{enumerate}
\item \label{bkm:Ref305674937}a  \textit{Customer~}: - Decaf ! 


\end{enumerate}
  b  \textit{Barista} : - Decaf ?

  c  \textit{Customer~}: - Yes, decaf. 

(vii) L'ellipse syntaxique, dans ses versions récentes, implique de manière générale le déplacement/l'extraction des éléments résiduels en périphérie gauche de la phrase, afin de pouvoir élider un constituant syntagmatique. Cependant, les éléments résiduels peuvent correspondre à des constituants non-extractibles, comme c'est le cas de l'adjectif \textit{expensive} en \REF{ex:3:200}b ou encore le nom propre \textit{Kim} en \REF{ex:3:201}b, exemples repris de Culicover \& \citet{Jackendoff2005}. 


\begin{enumerate}
\item \label{bkm:Ref306032243}a  A : - What kind of scotch does Harriet drink ? B : - \textbf{Expensive}. 


\end{enumerate}
  b  *It is \textbf{expensive} that Harriet drinks scotch.  


\begin{enumerate}
\item \label{bkm:Ref306032452}a  A : - Did Susan say that she saw PAT Smith ? B : - No, \textbf{Kim}. 


\end{enumerate}
  b  *\textbf{Kim}, Susan said that she saw \_ Smith.  

(viii) Je finis la liste en mentionnant un problème que rencontre l'approche structurale en termes de proforme nulle : le matériel manquant ne se comporte pas exactement de la même façon qu'un pronom. Ainsi, le matériel manquant peut être contenu dans l'antécédent (angl. \textit{antecedent-contained deletion}, cf. \citet{Sag1976}), ce qui n'est pas le cas d'un pronom ordinaire, comme illustré dans les exemples en \REF{ex:3:202} repris d'\citet{Aelbrecht2009}.


\begin{enumerate}
\item \label{bkm:Ref305966424}a  Christina [read every book Hilary did \textit{pro}\textsubscript{i}]\textsubscript{i}.


\end{enumerate}
  b  *Waldo saw [a picture of it\textsubscript{i}]\textsubscript{i}.

Les effets de non-connectivité retenus ici montrent que dans beaucoup de cas la séquence elliptique n'a pas le même comportement syntaxique que sa contrepartie non-elliptique, ce qui justifie le besoin d'une approche non-structurale qui gère ces effets de non-connectivité. De plus, ce type d'aproches semble mieux gérer les interactions de portée dans les contextes elliptiques, ainsi que les interprétations stricte et relâchée des pronoms (voir Dalrymple \textit{et al.} (1991)). 

Les conclusions de Culicover \& \citet{Jackendoff2005} :

- par rapport à la reconstruction syntaxique de manière générale : {\guillemotleft}~A syntactic account needs, in addition to its syntactic machinery, all the machinery of the semantic account.~{\guillemotright}

- par rapport à l'approche en termes de catégories vides : {\guillemotleft}~The real work is done by the interface to semantics, which has to provide interpretations for all the empty nodes by looking at the previous sentence. But notice : the interface could do the same thing without the empty syntactic structure, since all the semantic content assigned to this structure comes from the preceding clause. Why bother assigning it to specific nodes in an empty syntactic structure, especially when these nodes don't always correspond exactly to those in the antecedent ?~{\guillemotright}

\paragraph[Sous-types d'approches non-structurales]{Sous-types d'approches non-structurales}
Les approches non-structurales se divisent en deux sous-types majeurs : (i) reconstruction purement sémantique (p.ex. Dalrymple \textit{et al.} (1991), \citet{Dalrymple2005}), et (ii) reconstruction sémantique avec contraintes de parallélisme (p.ex. Ginzburg \& \citet{Sag2000}, Culicover \& \citet{Jackendoff2005}).

{\bfseries
Reconstruction purement sémantique}

La syntaxe légitime telles quelles toutes les séquences {\guillemotleft}~elliptiques~{\guillemotright} et une règle d'interprétation filtre parmi ces séquences celles qui donnent lieu à une phrase interprétable. Chez Dalrymple \textit{et al.} (1991) et \citet{Dalrymple2005}, cette règle d'interprétation consiste en une unification sur des lambda-termes d'ordre supérieur (angl. \textit{higher-order unification}). L'interprétation du matériel {\guillemotleft}~manquant~{\guillemotright} est fournie en résolvant une égalité sémantique entre la phrase source et la phrase elliptique : une certaine relation P (pour propriété) peut être obtenue à partir de l'interprétation de la phrase source et ensuite utilisée pour obtenir l'interprétation de la phrase elliptique. 

J'illustre ce mécanisme d'unification en \REF{ex:3:203} sur un exemple de VPE, repris de \citet{Kennedy2003}. Il y a essentiellement trois étapes. D'abord, on identifie les éléments sémantiques parallèles dans la phrase source et dans la phrase elliptique (en l'occurrence, le syntagme nominal \textit{Lou} et son corrélat \textit{Sterling}) ; une propriété P est appliquée aux éléments résiduels en question (p.ex. \textbf{\textit{P}}\textit{(Lou)} en \REF{ex:3:203}a). Ensuite, une relation P est dérivée par l'abstraction opérée sur le contenu propositionnel de la phrase source, comme en \REF{ex:3:203}b. Enfin, cette relation P, extraite de la représentation sémantique de la phrase source, est appliquée aux éléments résiduels dans la phrase elliptique, comme en \REF{ex:3:203}c.


\begin{enumerate}
\item \label{bkm:Ref305968647}\label{bkm:Ref306029707}Sterling quit the band because Lou did.            


\end{enumerate}
a  \textit{quit(Sterling, the band)} \textsc{because} \textbf{\textit{P}}\textit{(Lou)} 

b  \textbf{\textit{P}}\textit{ =} \textbf{\textit{$\lambda $x.}}\textit{quit(}\textbf{\textit{x}}\textit{, the band)}

c  \textit{quit(Sterling, the band)} \textsc{because} \textbf{\textit{$\lambda $x.}}\textit{quit(}\textbf{\textit{x}}\textit{, the band)}\textit{(Lou)}

{\bfseries
Reconstruction à l'interface syntaxe-sémantique}

La reconstruction purement sémantique, telle que présentée ci-dessus, permet l'ellipse sans nécessairement avoir un antécédent syntaxiquement approprié et ne rend pas compte du fait que les éléments résiduels doivent généralement respecter les contraintes morpho-syntaxiques qu'ils observeraient si l'on avait une phrase complète. 

Pour pallier ce problème de sur-génération, on se donne en syntaxe la notion de \textit{fragment} (d'où le terme \textit{phrase fragmentaire}\footnote{Cf. Abeillé \textit{en prép.}, tous les fragments ne sont pas des phrases. Il faut que leur interprétation soit univoque (c.-à-d. un contenu bien déterminé) dans le contexte et soit de type phrastique (proposition, question, visée).} dans la section \ref{sec:3.2.3}), conçu comme une construction à laquelle sont associées des conditions de bonne formation syntaxiques et interprétatives (cf. Ginzburg \& \citet{Sag2000}, Culicover \& \citet{Jackendoff2005}). La reconstruction sémantique est donc accompagnée par des contraintes de parallélisme qui s'établissent entre la phrase fragmentaire et la phrase source. 

Par exemple, dans le cas des phrases elliptiques sans tête verbale (p.ex. le stripping, le gapping, etc.), la contrainte de parallélisme est la suivante : Les éléments résiduels d'une phrase fragmentaire doivent être interprétés comme argument ou comme ajout du prédicat antécédent et ils doivent avoir un correspondant (c.-à-d. un corrélat) -- implicite ou explicite -- dans le contexte. La forme de ce correspondant peut déterminer leur propre forme. Dans les exemples en \REF{ex:3:204} repris d'Abeillé \textit{en prép.}, on observe que les deux premiers exemples sont agrammaticaux, car les éléments résiduels en question ne sont pas des valents appropriés pour le prédicat antécédent et on n'arrive pas à avoir les éléments parallèles requis par la contrainte de parallélisme : en \REF{ex:3:204}a, l'élément résiduel \textit{Marie} a comme corrélat un sujet impersonnel et il ne peut pas être l'argument du verbe \textit{pleuvoir~}; en \REF{ex:3:204}b, pour qu'il soit un argument approprié du prédicat antécédent, l'élément résiduel doit recevoir le marquage prépositionnel, ce qui n'est pas le cas (comparer, dans ce sens, \REF{ex:3:204}b et \REF{ex:3:204}c).


\begin{enumerate}
\item \label{bkm:Ref306126652}a  *Il pleut et Marie aussi. 


\end{enumerate}
  b  *Le rouge va à Jean, et je crois que Marie aussi.

  c  Le rouge va \textbf{à Jean} et je crois qu'\textbf{à} \textbf{Marie} aussi. 

La phrase fragmentaire dépend ainsi syntaxiquement et sémantiquement de la phrase source. Pour l'interprétation de la phrase fragmentaire, Culicover \& Jackendoff font appel à un mécanisme de {\guillemotleft}~légitimation indirecte~{\guillemotright} (angl. \textit{indirect licensing}, abrégé IL) d'un ou plusieurs éléments résiduels (appelés {\guillemotleft}~orphelins~{\guillemotright}). Ce mécanisme implique deux étapes essentielles : (i) la reconstruction sémantique, qui consiste dans l'intégration sémantique des constituants orphelins dans une structure propositionnelle, et (ii) l'intégration syntaxique, qui consiste dans une substitution syntaxique en parallèle (c.-à-d. chaque élément résiduel doit avoir un correspondant réalisé ou potentiel dans la phrase source).

J'illustre le fonctionnement du mécanisme proposé par Culicover \& \citet{Jackendoff2005} sur un de leurs exemples dans le dialogue \REF{ex:3:205}. La représentation syntaxique et sémantique (dans leus terms {\guillemotleft}~structure conceptuelle~{\guillemotright}, abrégée CS en anglais) du fragment figure en \REF{ex:3:206} et celle de la phrase source en \REF{ex:3:207}. Ayant les deux représentations, on peut maintenant passer à une procédure de substitution : (i) \textit{scotch} a comme corrélat l'objet direct \textit{something} dans la phrase source ; (ii) \textit{something} a dans sa structure conceptuelle \textsc{theme}\textit{~}: [\textsc{beverage}] ; (iii) de son côté, \textit{scotch} est un [\textsc{beverage}], donc les deux correspondent sémantiquement ; (iv) par conséquent, on substitue la structure conceptuelle du fragment à la place de l'objet direct dans la phrase source et on obtient : scotch[\textsc{beverage}] ${\equiv}$ \textsc{theme}\textit{~}: [\textsc{beverage}] ; (v) enfin, on obtient le résultat attendu : \textbf{drink}(\textsc{agent}\textit{~}: Harriet, \textsc{theme}\textit{~}: scotch[\textsc{beverage}]).


\begin{enumerate}
\item \label{bkm:Ref306135804}a  A : - Harriet was drinking something. 


\end{enumerate}
  b  B : - Yeah, [\textbf{scotch}]. 


\begin{enumerate}
\item \label{bkm:Ref306135893}Le fragment            


\end{enumerate}
  Syntax :    [\textsubscript{NP} scotch] 

  CS :      scotch[\textsc{beverage}]


\begin{enumerate}
\item \label{bkm:Ref306135895}La phrase source            


\end{enumerate}
  Syntax :    [\textsubscript{S} Harriet [\textsubscript{VP} drink [\textsubscript{NP} something] 

  CS :      \textbf{drink}(\textsc{agent}\textit{~}: Harriet, \textsc{theme}\textit{~}: [\textsc{beverage}])

Une proposition similaire apparaît dans Ginzburg \& \citet{Sag2000} dans le cadre d'une grammaire HPSG, pour les questions courtes et les réponses courtes en anglais. Une représentation simplifiée de l'élément résiduel \textit{scotch} est donnée en \REF{ex:3:208}.\footnote{Le fragment, dans l'analyse de Ginzburg \& \citet{Sag2000}, a le comportement d'une phrase finie (cf. \textit{verbal}  et VFORM \textit{finite} dans l'arbre simplifié) et ne peut pas être enchâssé (ce qui explique la valeur positive du trait IC, angl. \textit{Independent Clause}).} Il s'agit d'un syntagme fragmentaire de type déclaratif (angl. \textit{declarative-fragment-phrase}) qui obéit à une contrainte d'{\guillemotleft}~uniformité~{\guillemotright} : il doit dominer un syntagme dont les valeurs CATEGORY et CONTENT sont les mêmes que les valeurs CATEGORY et CONTENT d'un syntagme corrélat dans la phrase source. L'information grammaticale liée à ce corrélat est rendue accessible par un trait contextuel appelé SAL-UTT. 


\begin{enumerate}
\item \label{bkm:Ref306142130}Arbre simplifié de la phrase elliptique en \REF{ex:3:205}b  


\end{enumerate}
{   [Warning: Image ignored] % Unhandled or unsupported graphics:
%\includegraphics[width=2.6146in,height=3.4299in,width=\textwidth]{fe443409cd384d3fb0f6390ffd77f513-img37.svm}
} 

Je reprends ici l'explication fournie par Ginzburg \& \citet[301]{Sag2000} par rapport au trait contextuel SAL-UTT : {\guillemotleft}~In information-structure terms, SAL-UTT can be thought of as means of underspecifying the subsequent focal (sub)utterance or as a potential parallel element in the sense of Dalrymple \textit{et al.} (1991) and Shieber \textit{et al.} (1996). [...] Which constituent of a given utterance will be the SAL-UTT need not be viewed as determined prior to that utterance's taking place. Typically, the determination of SAL-UTT is a consequence of how conversationalist decides to structure her context, depending on which question she decides to make maximal in QUD\footnote{QUD = \textit{Question Under Discussion}} at a given point.~{\guillemotright} 

L'identification de l'élément résiduel avec un corrélat approprié dans la phrase source, grâce au trait SAL-UTT, permet ainsi la reconstruction sémantique du matériel manquant. Pour des détails supplémentaires, voir chapitre 4, section \ref{sec:4.5.1.2}.

\subsubsection{Quel type d'identité ?}
Dans la section \ref{sec:3.2.3}, on a rapproché le phénomène de l'ellipse des relations anaphoriques, car de manière générale le matériel manquant doit être récupéré dans le contexte, à partir d'un antécédent dans le discours, comme c'est le cas des anaphores ordinaires.

La question qui se pose maintenant est quel type de relation s'établit entre le matériel manquant et son antécédent ? Bien que la réponse soit évidente après avoir énuméré les effets de connectivité et surtout les effets de non-connectivité, je synthétise la discussion portée à ce sujet dans la littérature sur l'ellipse. 

Qu'on se situe dans une approche en termes de reconstruction syntaxique ou bien dans une approche par reconstruction sémantique, il semble que l'identité qui s'établit entre le matériel manquant et son antécédent est plutôt de nature sémantique. 

\paragraph[Identité sémantique]{Identité sémantique}
Crucialement, le matériel manquant doit avoir la même interprétation que son antécédent (Dalrymple \textit{et al.} (1991), \citet{Hardt1993}, Ginzburg \& \citet{Sag2000}, \citet{Merchant2001}, Culicover \& \citet{Jackendoff2005}, etc.). 

L'argument majeur pour postuler une identité sémantique regroupe tous les faits empiriques montrant des discordances entre la structure syntaxique de l'antécédent et celle du matériel manquant. La plupart d'entre eux ont été déjà discutés précédemment : les discordances de voix (cf. exemples \REF{ex:3:192}), les violations des principes du liage (cf. exemples \REF{ex:3:194}-\REF{ex:3:195}-\REF{ex:3:196} ci-dessus), l'interprétation relâchée des pronoms \REF{ex:3:184}, et on peut ajouter à cette liste le comportement des expressions de polarité, cf. \citet{Sag1976}. En \REF{ex:3:209}, on observe que l'indéfini se comporte comme un item de polarité dans la phrase source, mais pas dans la phrase reconstruite. 


\begin{enumerate}
\item \label{bkm:Ref306044012}John didn't see \textbf{anyone}, but Mary did.  


\end{enumerate}
a  ... but Mary did see \textbf{someone}. 

b  ... *but Mary did see \textbf{anyone}.

Un autre argument majeur en faveur de l'identité sémantique regroupe des faits liés à l'absence d'ambiguïté dans les contextes avec ellipse, alors que les mêmes structures sont par ailleurs ambiguës. Le premier type de contextes est représenté par les exemples en \REF{ex:3:210}. La séquence \textit{X is ready to eat} est a priori ambiguë entre une interprétation active (p.ex. \textit{X eats}) ou bien passive (p.ex. \textit{X is being eaten}). Pourtant, si cette séquence est suivie par une phrase elliptique, on observe que la phrase source et la phrase elliptique doivent avoir le même type d'interprétation (on obtient ainsi une interprétation parallèle dans les deux conjoints). La généralisation qui découle de ces exemples est la suivante : si plusieurs interprétations existent, la phrase elliptique doit recevoir la même interprétation que la phrase source.


\begin{enumerate}
\item \label{bkm:Ref306044733}The chickens are ready to eat and the children are, too.  


\end{enumerate}
=  a  The chickens eat and the children eat. 

=  b  The chickens are being eaten and the children are being eaten.

${\neq}$  c  The chickens eat and the children are being eaten.

${\neq}$  d  The chickens are being eaten and the children eat.

Le deuxième type d'exemples concerne la portée des quantifieurs. Dans la séquence \textit{Someone hit everyone} que représente la phrase source en \REF{ex:3:211}, le quantifieur \textit{someone} peut avoir une portée large (c.-à-d. quelqu'un a la propriété d'avoir frappé tout le monde) ou bien une portée étroite (c.-à-d. tout le monde a été frappé par quelqu'un). Cependant, on observe que, bien que la phrase source permette les deux portées, la présence de la phrase elliptique enlève l'ambiguïté, le quantifieur \textit{someone} ne pouvant recevoir qu'une portée large, car son corrélat dans la phrase elliptique (en l'occurrence, l'expression référentielle \textit{Bill}) n'a qu'une portée large.


\begin{enumerate}
\item \label{bkm:Ref306045272}Someone hit everyone, and then Bill did. 


\end{enumerate}
Tous les faits mentionnés précédemment suggèrent que l'identité qui s'établit entre le matériel manquant et l'antécédent est essentiellement de nature sémantique. Dans la littérature, cette identité sémantique a été formalisée de différentes manières : (i) identité en forme logique, modulo le lambda-calcul (p.ex. \citet{Sag1976}, \citet{William1977})\footnote{Voir \textit{alphabetic variance} de \citet{Sag1976}.} , ou (ii) identité de sens, modulo les éléments focalisés (p.ex. Merchant (2001, 2004))\footnote{Voir la condition de \textit{e-givenness} de Merchant (2001, 2006) et l'implication mutuelle XP\textsubscript{A} \~{} XP\textsubscript{E}.}.~

\paragraph[Identité syntaxique]{Identité syntaxique}
Certains travaux issus des approches structurales (\citet{Sag1976}, \citet{Williams1977}, Fiengo \& \citet{May1994}, Chung \textit{et al.} (1995), etc.)~postulent qu'entre le matériel manquant et l'antécédent on doit avoir une identité de structure syntaxique. 

Dans ces approches, l'identité syntaxique n'implique pas une identité {\guillemotleft}~superficielle~{\guillemotright} (morpho-phonologique). Ainsi, dans les exemples en \REF{ex:3:212} repris de \citet{Merchant2009}, on observe que les traits flexionnels ne sont pas pertinents : le matériel manquant peut correspondre à une forme verbale infinitive, alors que son antécédent est une forme fléchie au passé.


\begin{enumerate}
\item \label{bkm:Ref306037707}a  Jake \textbf{ate} the sandwich even though his friend told him not \textbf{to} {\textless}\textbf{eat} the sandwich{\textgreater}. 


\end{enumerate}
  b  Emily \textbf{played} beautifully at the recital and her sister \textbf{will} too {\textless}\textbf{play} beautifully at the recital{\textgreater}.  

Parmi les arguments invoqués pour l'identité syntaxique avec l'ellipse figure le comportement spécial de l'auxiliaire \textit{be} en anglais (cf. \citet{Merchant2009}). Contrairement aux autres verbes de l'anglais, l'auxiliaire \textit{be} exige une identité morphologique dans les constructions elliptiques : un exemple avec \textit{be} en \REF{ex:3:213}, bien que construit de la même façon que \REF{ex:3:212}, est agrammatical ; le matériel manquant comportant cet auxiliaire doit avoir la même forme morphologique que son antécédent.


\begin{enumerate}
\item \label{bkm:Ref306037938}*Emily \textbf{was} beautiful at the recital and her sister \textbf{will} too {\textless}\textbf{be} beautiful at the recital{\textgreater}.  


\end{enumerate}
Merchant (2008, 2009) ajoute, comme possible argument, la distribution asymétrique des discordances de voix dans les ellipses {\guillemotleft}~hautes~{\guillemotright} par rapport aux ellipses {\guillemotleft}~basses~{\guillemotright} (angl. \textit{high vs. low ellipsis}). Il observe que~dans les ellipses qu'il appelle {\guillemotleft}~hautes~{\guillemotright} (p.ex. sluicing, gapping, stripping, réponses courtes), l'antécédent et le matériel manquant doivent partager la même voix, ce qui explique l'agrammaticalité des exemples \REF{ex:3:214}a et \REF{ex:3:215}a. En revanche, dans les ellipses {\guillemotleft}~basses~{\guillemotright} (p.ex. VPE ou pseudogapping), on peut avoir des discordances de voix entre le matériel manquant et son antécédent : le matériel manquant peut avoir une forme active et son antécédent une forme passive \REF{ex:3:214}b, et vice-versa \REF{ex:3:215}a. Merchant explique cette distribution irrégulière en termes d'identité syntaxique : les ellipses {\guillemotleft}~hautes~{\guillemotright}, élidant plus que le simple syntagme verbal (angl. VP), sont sensibles à la présence du n{\oe}ud Voix (angl. \textit{Voice}) qu'elles incluent et exigent donc des traits de Voix identiques, alors que dans le cas des ellipses {\guillemotleft}~basses~{\guillemotright}, le n{\oe}ud Voix se trouve à l'extérieur du matériel effacé (car l'ellipse du syntagme verbal ne l'inclut pas), donc la Voix n'a aucune incidence sur les conditions d'identité.\footnote{Il faut noter cependant que ces asymétries de voix ont reçu aussi d'autres explications. Voir l'explication en termes de processing, donnée par Frazier \& Clifton (2005, 2006), ou encore l'explication en termes de relations de cohérence discursive, donnée par Kehler (2000, 2002).} La conclusion de Merchant est la suivante : A chaque fois qu'il y a une discordance apparente, le déclencheur de l'ellipse se situe en dehors du site de l'ellipse, alors que la cible se trouve à l'intérieur.


\begin{enumerate}
\item \label{bkm:Ref306039141}a  *Joe \textbf{was murdered}, but we don't know who. ({\textless}\textbf{murdered} Joe{\textgreater}) 


\end{enumerate}
  b  This problem \textbf{was} \textbf{to have been looked} into, but obviously nobody \textbf{did}. ({\textless}\textbf{look} into this problem{\textgreater}) 


\begin{enumerate}
\item \label{bkm:Ref306039144}a  *Someone \textbf{murdered} Joe, but we don't know who by. ({\textless}Joe \textbf{was murdered}{\textgreater}) 


\end{enumerate}
  b  The janitor should \textbf{remove} the trash whenever it is apparent that it needs to \textbf{be}. ({\textless}\textbf{removed}{\textgreater}) 

Un autre fait soutenant (au moins dans certains contextes) une identité syntaxique est lié au comportement des éléments résiduels qui peuvent être marqués par une préposition dans les constructions avec sluicing. \citet{Chung2005} observe qu'un argument marqué habituellement par une préposition (comme c'est le cas du syntagme prépositionnel \textit{about what} en \REF{ex:3:216}) peut apparaître sans préposition uniquement s'il a un corrélat explicite dans la phrase source \REF{ex:3:216}c ; si son corrélat est implicite, l'élément résiduel doit comporter la préposition (comparer \REF{ex:3:216}a et \REF{ex:3:216}b). \citet{Chung2005} ajoute donc pour le sluicing une contrainte lexico-syntaxique concernant l'identité qui doit s'établir entre le matériel manquant et son antécédent : chaque élément appartenant au matériel manquant doit avoir un corrélat lexical dans l'antécédent de la phrase source (résumée en anglais comme : \textit{no new words}). 


\begin{enumerate}
\item \label{bkm:Ref306040699}a  Bill is upset. Guess about what {\textless}he's upset{\textgreater}. 


\end{enumerate}
  b  Bill is upset. *Guess what {\textless}he's upset about{\textgreater}.

  c  Bill is upset about something. Guess what {\textless}he's upset about{\textgreater}. 

Toujours dans les constructions avec sluicing, on note l'absence des alternances de structure argumentale \REF{ex:3:217}c-d, bien que l'alternance de position des objets soit possible en dehors de l'ellipse \REF{ex:3:217}a-b.


\begin{enumerate}
\item \label{bkm:Ref306041094}a  They embroidered [something] [\textbf{with} peace signs]. 


\end{enumerate}
  b  They embroidered [peace signs] [\textbf{on} something].

  c  *They embroidered [something] [\textbf{with} peace signs], but I don't know what \textbf{on}.

  d  *They embroidered [something] [\textbf{on} their jackets], but I don't know \textbf{with} what. 

Je précise qu'il y a des approches qui postulent une identité hybride (p.ex. \citet{Kehler2002}) : le matériel manquant et l'antécédent sont identiques tant au niveau sémantique, que syntaxique. 

\subsection{Conclusion } 
Dans ce chapitre, j'ai donné un aperçu de la problématique de l'ellipse, phénomène qui sera étudié dans les chapitres suivants à travers deux constructions : (i) le gapping dans la coordination, et (ii) les relatives partitives sans verbe. 

Pour parler d'ellipse dans une structure, il faut~(i) qu'une partie du matériel nécessaire à l'interprétation manque dans la structure syntaxique, et (ii) que le matériel manquant soit récupérable à partir d'un antécédent dans le contexte (linguistique ou extra-linguistique). Contrairement à ce que l'on peut croire, l'ellipse n'est pas toujours facultative ; par conséquent, on ne peut pas réduire tous les emplois de l'ellipse au principe du moindre effort.  

Traditionnellement, la phrase {\guillemotleft}~complète~{\guillemotright} est considérée comme étant la phrase qui contient une tête verbale à un mode personnel. J'ai montré, en m'appuyant sur les données du roumain, qu'on peut avoir des phrases {\guillemotleft}~complètes~{\guillemotright} avec des formes verbales non-finies ou encore des phrases {\guillemotleft}~complètes~{\guillemotright} averbales, dont la tête n'est pas un verbe. Contrairement aux phrases complètes, les phrases elliptiques ont une constituance {\guillemotleft}~incomplète~{\guillemotright} et n'ont pas d'autonomie discursive.

J'ai fait ensuite l'inventaire des constructions elliptiques majeures, en fonction de trois critères : la nature du matériel manquant, le type de contexte syntaxique dans lequel apparaît le type d'ellipse en question et la directionnalité de l'ellipse. En ce qui concerne les conditions de légitimité de l'ellipse, on observe que (i) chaque type d'ellipse est autorisé dans un certain type de configuration syntaxique, et que (ii) tous les types d'ellipse n'apparaissent pas dans toutes les langues. On a vu que l'identification des différentes constructions est un travail difficile si l'on se place dans une perspective typologique.  

Pour ce qui est de la résolution de l'ellipse (c.-à-d. le moyen mis en place pour récupérer l'information qui manque), on a vu que plusieurs possibilités d'analyse se présentent, en fonction du niveau linguistique auquel opère la résolution, c.-à-d. la syntaxe, la sémantique ou bien l'interface syntaxe-sémantique. Les propositions se regroupent en deux approches majeures, que j'ai appelées, en suivant \citet{Merchant2009}, approches structurales vs. approches non-structurales. Le choix entre l'une ou l'autre de ces approches joue autour des effets de connectivité ou de non-connectivité, qu'on observe entre la phrase source et la phrase elliptique : l'ellipse syntaxique semble être justifiée à chaque fois qu'on observe des effets de connectivité, alors que l'ellipse sémantique semble être plus attractive dans les situations qui ne présentent pas ces effets. En dehors de la compétition existant entre ces deux approches pour expliquer un même phénomène elliptique, je considère que dans une grammaire de l'ellipse les deux solutions doivent être disponibles, car on ne peut pas établir d'analyse uniforme pour toutes les constructions elliptiques d'une langue et parfois on ne peut pas avoir une analyse unitaire même pour une même construction elliptique dans des langues différentes. La description et l'analyse de l'ellipse doivent se faire donc construction par construction et langue par langue.

Bien que les constructions elliptiques dans leur hétérogénéité obéissent à des contraintes grammaticales plus ou moins strictes, la contrainte majeure s'appliquant à toutes les constructions elliptiques~concerne l'identité sémantique qui doit caractériser la relation entre le matériel manquant et son antécédent (c.-à-d. ils doivent être équivalents quant à leurs conditions de vérité). 

Dans l'étude du phénomène de l'ellipse, j'ai pris en compte surtout les facteurs syntaxiques et sémantiques. Des travaux récents, que je n'ai pas présentés dans ce chapitre, révèlent l'importance des facteurs d'une autre nature dans le fonctionnement de l'ellipse : la structure informationnelle (p.ex. \citet{Winkler2005}), les relations de cohérence discursive (p.ex. Kehler (1994, 2000, 2002)), les facteurs psycholinguistiques (p.ex. Carlson (2001, 2002), Carlson \textit{et al.} (2005)), etc. 

