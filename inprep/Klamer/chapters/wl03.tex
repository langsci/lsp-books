\begin{abstract}
The Papuan languages of Timor, Alor, Pantar and Kisar have long been thought to be members of a single family. However, their relatedness has not yet been established through the rigorous application of the comparative method. Recent historical work has shown the relatedness of the languages of Alor and Pantar on the one hand \citep{HoltonEtAl2012}, and those of Timor and Kisar on the other \citep{SchapperEtAl2012}. In this chapter, we present a preliminary demonstration of the relatedness of the Timor-Alor-Pantar family based on a comparison of these two reconstructions. We identify a number of regular consonant correspondences across cognate vocabulary between the two groups and reconstruct a list of 89 proto-TAP roots. 
\end{abstract}

\section{Introduction}\label{sec:3:1}
This chapter looks at the historical relationship between the Papuan languages of Alor-Pantar (AP) and those of Timor-Kisar (TK). The TK group of Papuan languages consists of Bunaq, spoken in central Timor, Makasae, Makalero and Fataluku, three languages spoken in a contiguous region of far eastern Timor, and Oirata, spoken on the southern side of Kisar Island to the north of Timor (Map 1). Due to their geographical proximity, AP and TK languages have typically been assumed to be related to one another (e.g., Stokhof 1975; Capell 1975). Together they have been referred to as the Timor-Alor-Pantar (TAP) family. However, there has been no substantive data-driven investigation of the claim.

In this chapter, we test the hypothesis that AP and TK languages are related to one another through the application of the comparative method. Specifically, we compare the results of two recent reconstructions, the one of AP \citep{HoltonEtAl2012} and the other of TK \citep{SchapperEtAl2012}. The sources of the lexical data used are listed in the Appendix. We establish that the AP and TK languages are indeed related by demonstrating that there are regular sound correspondences across cognate vocabulary between the two groups. 

In comparing \citet{HoltonEtAl2012} and \citet{SchapperEtAl2012} in this chapter, we assume the existence of two nodes in the TAP tree, namely Proto-AP (pAP) and Proto-TIM (pTIM). Whilst pAP appears to be a robust node, the existence of pTIM is less secure. As \citet[227-228]{SchapperEtAl2012} point out, it is possible that Bunaq and the Eastern Timor languages (reconstructed as Proto-ET in Schapper et al. 2012) both form their own separate primary subgroups within TAP. Our aim here is not to make claims about the high-level subgrouping of the AP and TK languages, and we do not presume to definitively determine the constituency of the TK-AP tree at this stage, but merely seek to show that TK and AP languages are related. Conclusive evidence of innovations shared by Bunaq and ET languages to the exclusion of AP languages is the subject of ongoing research. 

\begin{figure}
\includegraphics[width=\textwidth]{figures/ch3_map.png}
\caption[The Papuan languages of Timor and Kisar]{The Papuan languages of Timor and Kisar. Hatching marks areas where Papuan languages are found. Only Timor-Kisar languages are marked by name.}
\end{figure}

Section \ref{sec:3:2} presents the sound correspondences we find in cognate vocabulary between pAP and pTIM. Section \ref{sec:3:3} summarises our preliminary findings and discusses issues arising out of them. An Appendix is included with supporting language data for any reconstructions that do not appear in \citet{HoltonEtAl2012} or \citet{SchapperEtAl2012}, as well as a list of pTAP forms that can be reconstructed on the basis of the sound correspondences identified in this chapter. New, additional reconstructions have in some cases been necessary since the two articles each reconstruct only a small number of lexemes with only partial overlap between them. We also throw out several cognate sets from the AP reconstruction as they reflect borrowing from Austronesian languages. 

\section{Sound correspondences}\label{sec:3:2}
In this section, we describe the consonant correspondences that we have identified between AP and TK languages. We do draw on vowel correspondences where they condition particular sound changes in consonants, but otherwise do not deal with vowels in this preliminary demonstration of relatedness. We chiefly draw attention to the correspondences in cognate vocabulary between pAP and pTIM. However, we provide the reader also with the forms of the lexemes in the TK languages as they are not available elsewhere in this volume. The argumentation and underpinning data for pAP is given in \citet{HoltonRobinsonTV} and is based on \citet{HoltonEtAl2012}.

In the subsections that follow, transcription of language data adheres to IPA conventions. Long vowels are indicated with a length mark `{\textlengthmark}'. Bracketed segments `( )' are those deemed to be non-etymological, that is, typically reflecting some morpheme which has fossilised on a root. In the correspondence tables, square brackets `[]' are used where an item is cognate but doesn't reflect the segment in question. The inverted question mark `?`' is used where a cognate shows an unexpected reflex of the segment in question. Grammatical items are glossed in small caps. Reconstructions marked with `!!' are new reconstructions not found in \citet{HoltonEtAl2012} or \citet{SchapperEtAl2012}, or are revised from \citet{HoltonEtAl2012}. The symbol `!!' signals that the full data set on which the reconstruction in question is based is given in the appendices. AP data supporting the additional pAP reconstructions is given in Appendix I and TK data in Appendix II. In the text of the chapter itself, for reasons of compactness, we only give simple one word glosses which reflect the presumed meaning of the proto-lexeme. Should the reader need more information, he can refer to the appendix. We also do not provide information on irregular changes, such as metathesis or apocope, in the correspondence tables, except where directly relevant to the reconstruction of the segment in question. The appendix provides the reader with fuller information on any irregularities in form or meaning in individual languages. 

\subsection{Reconstruction of bilabial stops}
We identify two robust correspondent sets for bilabial plosives, reconstructing to pTAP *p and *b. Note that in \citet{SchapperEtAl2012}, we reconstruct a three-way distinction (*p, *b, and *f) for bilabial obstruents in pTIM, despite the fact that it is not maintained in any of the modern TK languages: Bunaq, Makasae and Fataluku have merged reflexes of pTIM *p and *f, whereas in Fataluku and Oirata, *f and *b are merged. We find no evidence to support a three-way split in pTAP; instead, it looks like pTIM underwent a conditioned phoneme split, with distinct reflexes of pTAP *b in initial and non-initial positions, respectively. 

Table \ref{tab:3:1} and Table \ref{tab:3:2} present the forms for these two correspondence sets respectively. In the first, pAP *p corresponds to pTIM *f in all positions. In the second, pTAP *b was retained as *b in pAP, but split to pTIM *b initially and pTIM *p non-initially. In these sets, there are three notable irregularities: (i) pAP *tiara `expel' lost the medial bilabial that is retained in pTIM *tifar `run'; (ii) pAP *siba(r) `new' and pTIM *(t,s)ifa(r) `new' show an irregular correspondence of pAP *b with pTIM *f; and (iii) pAP *karab `scratch' and pTIM *gabar `scratch', which show an irregular correspondence of pAP *b with pTIM *b. 
 


\begin{sidewaystable}
\caption{Correspondence sets for pTAP *p}
\label{tab:3:1}  
\begin{tabular}{llllllll}
\mytoprule
 & pAP & pTIM & Bunaq & Makasae & Makalero & Fataluku & Oirata\\
\midrule
{\bfseries initial *p} & {\bfseries *p} & {\bfseries *f} & {\bfseries p, w} & {\bfseries f} & {\bfseries f} & {\bfseries f} & {\bfseries p}\\
spit & *purVn !! & *fulu(k, n) !! & {\itshape puluk} & -- & {\itshape fulun} & {\itshape fulu} & --\\
taboo & *palol !! & *falu(n) & por & falun & falun & {\itshape falu} & --\\
1PI & *pi- & *fi & {}-{}- & fi & fi & afa & ap-\\
\textsc{low} \textsuperscript{1} & *po !! & *ufe !! & {}-{}- & he- ?` & ufe- & [ua] & [ua]\\
girl & *pon !! & *fana \textsuperscript{2} & pana & fana(rae) & fana(r) & fana(r) & pana(rai)\\
scorpion & *pVr & *fe(r, R)e !! & wele & {}-{}- & {}-{}- & {}-{}- & {}-{}-\\
{\bfseries medial *p} & {\bfseries *p} & {\bfseries *f} & \textbf{w,} \textbf{{\O}} & {\bfseries f} & {\bfseries f} & {\bfseries f} & {\bfseries p}\\
face & *-pona !! & *-fanu !! & {}-ewen & fanu & fanu & fanu & panu\\
dream & *hipar & *ufar(ana) !! & waen & ufarena & ofarana & ufarana & upar(a)\\
run & [*tiara] & *tifar & t{\textesh}iwal & [ditar] & [titar] & tifar(e) & tipar(e)\\
pound & *tapai & *tafa & tao \textsuperscript{3} & {}-{}- & tafa & tafa & tapa\\
\mybottomrule
\end{tabular}
 
\textsuperscript{1} This item is a deictic marker for items at lower elevation than the deictic center. See \citet{SchapperTV} for more information on this deictic distinction.

\textsuperscript{2} The bracketed \textit{rae/r/rai} element appears to be an innovation in the Eastern Timor languages, presumably a lexical doublet or a derivational morpheme related to the nominalizing \textit{{}-r} formative found in Makalero. We have no evidence for reconstructing this element higher than Proto-Eastern Timor.

\textsuperscript{3} This would have originally been *tawo in pre-Bunaq, but in the modern language medial /w/ is not preserved before back vowels. 
\end{sidewaystable}

 

\begin{sidewaystable} 
\caption{Correspondence sets for pTAP *b}
\label{tab:3:2} 
\begin{tabular}{llllllll}
\mytoprule
 & pAP & pTIM & Bunaq & Makasae & Makalero & Fataluku & Oirata\\
\midrule
{\bfseries inital *b} & {\bfseries *b} & {\bfseries *b} & {\bfseries b} & {\bfseries b} & {\bfseries p} & {\bfseries p} & {\bfseries h}\\
pig & *baj & *baj & {}-{}- & baj & paj & paj & haj\\
price & *bol !! & *bura & bol & bura & pura & pura & hura\\
mat & *bis & *biti !! & {}-{}- & {}-{}- & piti & pet(u) & het(e)\\
leg & *-bat !! & *-buta !! & {}-but & {}-{}- & {}-{}- & {}-{}- & {}-{}-\\
mountain & *buku !! & *bugu !! & -- & {\itshape bu{\textglotstop}u} & {\itshape pu{\textglotstop}u} & -- & --\\
{\bfseries non-initial *b} & {\bfseries *b} & {\bfseries *p} & {\bfseries p, w} & {\bfseries f} & {\bfseries f} & {\bfseries p} & {\bfseries h}\\
fish & *habi !! & *hapi !! & {}-{}- & afi & afi & api & ahi\\
star & *jibV \textsuperscript{1} & *ipi(-bere) & [bi] \textsuperscript{2} & ifi-bere & ifi & ipi(naka) & ihi\\
shark & * sib(a ,i)r !! \textsuperscript{3} & *supor !! & {}-{}- & {}-{}- & [su] \textsuperscript{4} & hopor(u) & {}-{}-\\
sugarcane & *hu:ba !! & *upa & up & ufa & ufa & upa & uha\\
tongue & *-lebur !! & *-ipul & {}-up & ifi & ifil & epul(u) & uhul(u)\\
dog & *jibar !! \textsuperscript{5} & *Depar & zap & defa & sefar & ipar(u) & ihar(a)\\
other & *aben(VC) !! & *epi !! & ewi & {}-{}- & {}-{}- & {}-{}- & {}-{}-\\
scratch & *karab !! & *gabar ?` !! \textsuperscript{6} & {}-{}- & {}-{}- & kapar & kafur(e) & {}-{}-\\
new & *siba(r) !! & *(t,s)ipa(r) ?` !! & tip & sufa & hofar & -- & {}-{}-\\
\mybottomrule
\end{tabular}
 \textsuperscript{1} Several AP languages have a compound for `star', although the second element does not appear to be cognate to that reconstructed for pTIM. Note also that \citet{HoltonEtAl2012} gave this item as *jibC; this error has been corrected for \citet{HoltonRobinsonTV}.

\textsuperscript{2} The Bunaq form reflects the second half of the pTIM doublet that is not found in AP languages.

\textsuperscript{3} The cognate set for this item is given in \citet{HoltonEtAl2012}, but no pAP reconstruction is given.

\textsuperscript{4} The reflex of the relevant bilabial has been lost in Makalero due to apocope.

\textsuperscript{5} The cognate set for this item is given in \citet{HoltonEtAl2012}, but no pAP reconstruction is given.

\textsuperscript{6} This form shows liquid-stop metathesis. There is no evidence of *b occurring word-finally in pTIM.
\end{sidewaystable}



At this stage, we have no evidence for the reconstruction of a third bilabial obstruent to pTAP, as is found in pTIM (*p, *b and *f), but not pAP (*p and*b). Based on the current correspondence sets, the three-way distinction appears to have arisen due to pTAP medial *b changing to pTIM *p, while pTAP *b stayed *b initially in pTIM. We are yet to find any AP cognates for words reconstructing with initial *p in pTIM.

\subsection{Reconstruction of coronal stops}
There are two coronal stops, *t and *d, reconstructed to pAP, and four, *t, *d, *T and *D to pTIM. \citet{SchapperEtAl2012} note the uncertainty of pTIM *d, which is supported by three cognate sets only, all of which are in initial position. This is played out also when comparing coronals between AP and TK languages. We can reconstruct the pTAP coronal stops *t with relative certainty, and *d, albeit with less security. The latter segment split in pTIM to *T and *D. At present, we cannot reconstruct pTIM *d to pTAP. There are, however, a substantial number of coronal correspondences which remain unexplained. 

Our most consistent correspondence is pTIM *t to pAP *t and *s (Table \ref{tab:3:3}). Initially, we find a steady and unchanging correspondence of pAP *t and pTIM *t, supported by a sizeable number of cognates. Only Bunaq shows a change of *t to /t{\textesh}/ before a high front vowel. Non-initially, we find fewer cognates, but nevertheless a steady and unchanging correspondence. In two cognate sets (`sit' and `mat'), pAP final *s preceded by *i corresponds to pTIM *t.
 

\begin{sidewaystable}
\caption{Correspondence sets for pTAP *t}
\label{tab:3:3}  
\begin{tabular}{llllllll}
\mytoprule
 & pAP & pTIM & Bunaq & Makasae & Makalero & Fataluku & Oirata\\
\midrule
{\bfseries initial *t} & {\bfseries *t} & {\bfseries *t} & \textbf{t, t}\textbf{{\textesh}} & {\bfseries t} & {\bfseries t} & {\bfseries t} & {\bfseries t}\\
hand & *-tan & *-tana & {}-ton & tana & tana & tana & tana\\
sea & *tam & *mata & [mo] & {}-{}- & {}-{}- & mata & mata\\
six & *talam & *tamal !! \textsuperscript{1} & tomol & {}-{}- & {}-{}- & {}-{}- & {}-{}-\\
pound & *tapai & *tafa & tao & {}-{}- & tafa & tafa & tapa\\
run & *tiara & *tifar & t{\textesh}iwal & ditar ?` & titar & tifar(e) & tipar(e)\\
sleep & *tia & *tia(r) & t{\textesh}ier & ta{\textglotstop}e & tia & taia & taja\\
{\bfseries non-initial *t} & {\bfseries *t, *s} & {\bfseries *t} & {\bfseries t} & {\bfseries t} & {\bfseries t} & {\bfseries t} & {\bfseries t}\\
tree & *tei & *hate !! & hotel & ate & ate & ete & ete\\
stand & *nate(r) !! & *nat & {\itshape net} & {\itshape [na] ?`} & {\itshape nat} & {\itshape (a)nat(e)} & {\itshape nat(e)}\\
clew & *maita !! & *matar & {\itshape mot} & {}-{}- & {}-{}- & matar(u) & matar(a)\\
flat & *tatok !! & *tetok !! & {\itshape toi{\textglotstop}} & -- & {\itshape tetu{\textglotstop}} & -- & --\\
leg & *-bat !! & *-buta !! & {\itshape {}-but} & -- & -- & -- & --\\
sit & *mis & *mit & mit & mit\~{}[mi] & mit & [(i)mir(e)] ?` & [mir(e)] ?`\\
mat & *bis & *biti !! & {}-{}- & {}-{}- & piti & pet(u) & het(e)\\
\mybottomrule
\end{tabular}
 
\textsuperscript{1} Bunaq /o/ is a regular reflex of pTIM *a, as seen, for instance, from the `hand', `sea', `wake', `tree' and `clew' sets. 
\end{sidewaystable}


The reconstruction of pTAP *d is supported by only a small number of cognate sets (Table \ref{tab:3:4}) and therefore still needs confirmation. In these sets, initial pAP *d corresponds to pTIM *D, while non-initial pAP *d corresponds with pTIM *T. This is consistent with what we observed with the bilabial stops, where a medial voiced stop in pAP corresponds to a voiceless stop in pTIM. Note that the cognate set for `bird' is listed under the heading of initial *d, even though its pTIM and (arguably) pAP reflexes are in medial position. We place it there due to the fact that the sound correspondence is parallel to that for `rat'. However, more sets supporting this reconstruction are clearly needed before we can be certain of it. 
 
\begin{sidewaystable}
\caption{Correspondence sets for pTAP *d}
\label{tab:3:4}  
\begin{tabular}{llllllll}
\mytoprule
 & pAP & pTIM & Bunaq & Makasae & Makalero & Fataluku & Oirata\\
\midrule
{\bfseries initial *d} & {\bfseries *d} & {\bfseries *D} & {\bfseries z, s} & {\bfseries d, s} & {\bfseries s} & {\bfseries c} & {\bfseries {\textrtailt}, s}\\
rat & *dur & *Dura & zul & dura & sura & cura & {\textrtailt}ura\\
dog & *jibar ?` !! \textsuperscript{1} & *Depar & zap & defa & sefar & [ipar(u)] ?` & [ihar(a)] ?`\\
bird & *(a)dVl !! & *haDa & hos & asa & asa & aca & asa\\
{\bfseries medial *d} & {\bfseries *d} & {\bfseries *T} & {\bfseries t} & {\bfseries t} & {\bfseries t} & {\bfseries c } & {\bfseries {\textrtailt}}\\
bat & *madel & *maTa !! \textsuperscript{2} & {}-{}- & {}-{}- & {}-{}- & maca & ma{\textrtailt}a\\
fire & *hada !! & *haTa & hoto & ata & ata & aca & a{\textrtailt}a\\
far & *lete !! & *eTar !! & ate & {}-{}- & {}-{}- & icar & {}-{}-\\
sun & *wadi !! & *waTu & hot & watu & watu & wacu & wa{\textrtailt}u\\
garden & *magad(a) & [*(u, a)mar] \textsuperscript{3} & mar & ama & ama & {}-{}- & uma\\
\mybottomrule
\end{tabular}

\textsuperscript{1} We note the irregularity of pAP *jibar `dog' where we would expect pAP *dipar `dog'. This is likely the result of a change pre-pAP *d {\textgreater} *j

\textsuperscript{2} The cognate set for this item is given in \citet{SchapperEtAl2012}, but no pTIM reconstruction is given.

\textsuperscript{3} This form shows metathesis with associated loss of the syllable with pTAP *g, thus: pTAP *magad {\textgreater} *madag {\textgreater} *amar. 
\end{sidewaystable}


Furthermore, there are a range of cognate sets which show as yet unexplained correspondences (Table \ref{tab:3:5}). In these, we find coronal correspondences between pAP and pTIM and between TK languages (especially in Bunaq and Fataluku) that don't fit well in the above given sets. More work is needed to clarify the history of the coronals in TAP.

 

\begin{sidewaystable}
\caption{Problematic coronal cognate sets}
\label{tab:3:5}  
\begin{tabular}{llllllll}
\mytoprule
 & pAP & pTIM & Bunaq & Makasae & Makalero & Fataluku & Oirata\\
\midrule
grandparent \textsuperscript{1} & *tam(a, u) !! & *moTo & mata(s) & mata & mata & moco & mo{\textrtailt}o\\
far & *lete !! & *eTar !! & ate & {}-{}- & {}-{}- & icar & {}-{}-\\
wake & *-ten & *Tani & otin & tane & tane & {\itshape tani \~{} cani} & {}-{}-\\
coconut & *wata !! & *wa(t, D)a & hoza & wata & wata & $\beta $ata & wata\\
{\itshape P. indicus} & *matar !! & *ma(t, D)ar & {\itshape mazo{\textglotstop}} & {\itshape mater} & {\itshape mater} & {\itshape matar(ia)} & --\\
excrement & *has & *a(t, D)u !! & ozo & atu(-gu{\textglotstop}u) & atu & atu & atu\\
\mybottomrule
\end{tabular} 
\textsuperscript{1} This is a reciprocal kinship term, denoting either `grandparent' or `grandchild'. PTIM *moTo means `child'.
\end{sidewaystable}


\subsection{Reconstruction of velar stops}
We reconstruct two velar stops for pTAP, *k and *g. We find insufficient evidence, however, for the uvular stop reconstructed for pAP in \citet{HoltonEtAl2012} and \citet{HoltonRobinsonTV}. 

PTAP *k and *g are retained as *k and *g in pAP, but merged to *g in pTIM. Note that, based on the comparative TAP evidence and the additional pTAP reconstructions in this chapter, we have to substantially revise Schapper et al.'s (2012) pTIM reconstructions with regard to velar stops. Concretely, we can trace only one pTIM velar back to pTAP. We find no pAP reflexes for any of the small sets of roots reconstructed for pTIM with initial *k and medial *g; those for pTIM medial *g, in particular, are rather tenuous, as noted in \citet[212]{SchapperEtAl2012}. The cognate sets that we can trace back to pTAP involve Schapper et al.'s initial *g and medial *k, and the comparative evidence is consistent with these being differential realisations of a single pTIM segment *g: initially, pTIM *g is reflected as /g/ in Bunaq and Makasae, and as /k/ in Makalero and Fataluku. We currently only have no evidence for Oirata. In non-initial position, *g is reflected in Bunaq as /g/ medially and as /k/ finally, consistent with Bunaq phonotactic rules, which prohibit voiced stops from codas; in Makasae, Makalero and Fataluku, *g is reflected in non-initial position as /{\textglotstop}/, and variably as /{\textglotstop}/ and {\O} in Oirata. 

The cognate sets that support the reconstruction of pTAP *k are given in Table \ref{tab:3:6}. 
 

\begin{sidewaystable}
\caption{Correspondence set for pTAP *k}
\label{tab:3:6}  
\begin{tabular}{llllllll}
\mytoprule
 & pAP & pTIM & Bunaq & Makasae & Makalero & Fataluku & Oirata\\
\midrule
 & {\bfseries *k} & {\bfseries *g} & {\bfseries g (k)} & {\bfseries (g) {\textglotstop}} & \textbf{k,} \textbf{{\textglotstop}}\textbf{} & \textbf{k,} \textbf{{\textglotstop}}\textbf{} \textbf{} \textsuperscript{1} & \textbf{({\textglotstop})} \textbf{{\O}} \textsuperscript{1}\\
scratch & *karab !! & *gabar !! \textsuperscript{2} & -- & -- & {\itshape kapar} & {\itshape kafur(e)} & --\\
bite & *(ta)ki !! \textsuperscript{3} & *(ga)gel !! \textsuperscript{3} & {\itshape gagil} & {\itshape ga{\textglotstop}el} & {\itshape ka{\textglotstop}el} & {\itshape (ki)ki{\textglotstop}(e) \textsuperscript{4}} & --\\
dirty & *karok !! & *gari !! & {\itshape gar} & {\itshape ra{\textglotstop}i} & {\itshape ra{\textglotstop}i} & {\itshape ra{\textglotstop}e(ne)} & --\\
walk 1 & *laka !! & *lagar !! & {\itshape lagor} & {\itshape la{\textglotstop}a} & {\itshape la{\textglotstop}a} & {\itshape la{\textglotstop}a} & {\itshape [lare] ?`}\\
itchy & *(i)ruk !!\textsuperscript{ 5} & *ilag !! & -- & {\itshape ila{\textglotstop}} & {\itshape ile{\textglotstop}} & -- & --\\
mountain & *buku !! & *bugu !! & -- & {\itshape bu{\textglotstop}u} & {\itshape pu{\textglotstop}u} & -- & --\\
\mybottomrule
\end{tabular} 
\textsuperscript{1} See \citet[211-212]{SchapperEtAl2012} for more Fataluku and Oirata correspondences.

\textsuperscript{2} This form shows liquid-stop metathesis.

\textsuperscript{3} The bracketed initial segments in these forms reflect different inflectional prefixes which have fossilized on these verbs.

\textsuperscript{4} The initial bracketed syllable is a fossilized reduplicated CV. This item also has the variant pronunciation \textit{ciki{\textglotstop}e}.

\textsuperscript{5} This form represents a different root from the `itchy' root given in \citet{HoltonRobinsonTV}. See Appendix I for supporting AP forms.
\end{sidewaystable}


As in both pTIM \citep[213-214]{SchapperEtAl2012} and pAP \citep[98]{HoltonEtAl2012}, the reconstruction of initial *g in pTAP hinges on third person markers. Two forms are reconstructable (Table \ref{tab:3:7}): a prefix *g(a,i)- `3INLN' occurring on verbs and inalienably possessed nouns, and a free form *gie `3ALN' encoding 3\textsuperscript{rd} person alienable possessors. Number marking was lost in TK languages, so the correspondence we observe is between pAP third person singular forms and pTIM third person forms which are unmarked for number (i.e., can be used in singular and plural contexts). The zero correspondence that we observe in Fataluku and Oirata is the result of the stripping off of the *g marking 3\textsuperscript{rd} person (as set out in \citealt[214]{SchapperEtAl2012}). In the case of the alienable possessive marker, this means we are left with the possessive root pTIM *-ie `ALN' alone.
 

\begin{sidewaystable}
\caption{Correspondence set for pTAP 3\textsuperscript{rd} person prefixes}
\label{tab:3:7}  
\begin{tabular}{llllllll}
\mytoprule
 & pAP & pTIM & Bunaq & Makasae & Makalero & Fataluku & Oirata\\
\midrule
 & {\bfseries *g} & {\bfseries *g} & {\bfseries g} & {\bfseries g} & {\bfseries k} & {\bfseries {\O}} & {\bfseries {\O}}\\
3INLN & *ga- & *g- & {\itshape g-} & {\itshape g-} & {\itshape k-} & -- & --\\
3ALN & *ge \textsuperscript{1} & *gie & {\itshape gie} & {\itshape gi} & {\itshape ki} & {\itshape i} & {\itshape ue}\\
\mybottomrule
\end{tabular} 

\textsuperscript{1} We reconstruct this as a free form on account of the existence of free reflexes in at least two AP languages (Blagar and Adang); morphologisation must thus post-date the break-up of pAP. 

\end{sidewaystable}
In non-initial positions, we find numerous cognates reflecting pTAP *g, corresponding to pAP *g and pTIM *g as set out in Table \ref{tab:3:8}. 
 

\begin{sidewaystable}
\caption{Correspondence set for pTAP *g}
\label{tab:3:8}  
\begin{tabular}{llllllll}
\mytoprule
 & pAP & pTIM & Bunaq & Makasae & Makalero & Fataluku & Oirata\\
\midrule
 & {\bfseries *g} & {\bfseries *g} & {\bfseries g, k} & {\bfseries g, {\textglotstop}} & {\bfseries (k) {\textglotstop}} & {\bfseries (k) {\textglotstop}} & {\bfseries {\textglotstop}, {\O}}\\
yellow & *bagori !! \textsuperscript{1} & *gabar !! \textsuperscript{2} & -- & {\itshape gabar} & -- & -- & --\\
green & *(wa)logar !! & *ugar & {\itshape ugar} & {\itshape (h)u{\textglotstop}ur} & {\itshape (h)u{\textglotstop}ur} & {\itshape u{\textglotstop}ur(eke)} & {\itshape u{\textglotstop}ul(e)}\\
laugh & *jagir !! & *jiger !! & {\itshape higal} & {\itshape hi{\textglotstop}a} & {\itshape hi{\textglotstop}e} & {\itshape he{\textglotstop}e} & --\\
path & *jega !! & *jiga !! & {\itshape hik} & {\itshape hi{\textglotstop}a} & {\itshape hi{\textglotstop}a} & {\itshape i{\textglotstop}a} & {\itshape ia(ra)}\\
banana & *mogol & *mugu !! & {\itshape mok} & {\itshape mu{\textglotstop}u} & {\itshape mu{\textglotstop}u} & {\itshape mu{\textglotstop}u} & {\itshape mu{\textlengthmark}}\\
hear & *magi !! \textsuperscript{3} & *mage(n) !! & {\itshape mak} & {\itshape ma{\textglotstop}en} & {\itshape ma{\textglotstop}en} & -- & --\\
garden & *magad(a) & [*(u, a)mar] !! \textsuperscript{4} & {\itshape mar} & {\itshape ama} & {\itshape ama} & -- & {\itshape uma}\\
\mybottomrule
\end{tabular}
\textsuperscript{1} The cognate set for this item is given in \citet{HoltonEtAl2012}, but no pAP reconstruction is given.

\textsuperscript{2} This form is apparently metathesized from pTAP *bagori `yellow'.

\textsuperscript{3} The cognate set for this item is given in \citet{HoltonEtAl2012}, but no pAP reconstruction is given.

\textsuperscript{4} This form shows metathesis with associated loss of the syllable with pTAP *g, thus: pTAP *magad {\textgreater} *madag {\textgreater} *amar. Loss of *g is found occasionally in AP languages (e.g. `laugh', see Appendix \ref{sec:3:app:1}), suggesting a certain degree of instability for this segment.
 
\end{sidewaystable}


Finally, there is as yet an insufficient number of reconstructions of pAP *q with cognates in TK languages to allow for a higher-level pTAP reconstruction. Currently, we have only Bunaq \textit{{}-ol} `child' (presumably reflecting pTIM *-al) as cognate with pAP *-uaqal `child'. We await further reconstructions with TK cognates for the determination of the pTAP form.

\subsection{Reconstruction of fricatives}
Two fricatives *s and *h can be reconstructed to pTAP. The number of cognates is still small for both phonemes, but the correspondences are relatively well-behaved. 

Table \ref{tab:3:9} sets out the cognate sets for pTAP *s. Initial pTAP *s is supported by several cognate sets and has been maintained without change in pAP and pTIM. Non-initial cognates of pAP *s are difficult to find in TK languages, as many instances of reconstructed word-final *s in pAP correspond to pTIM *t (e.g., pAP *mis `sit', *bis `mat' and *has `excrement'  ---  see also our discussion of these in Section \ref{sec:3:4}).
 

\begin{sidewaystable}
\caption{Correspondences of pTAP *s}
\label{tab:3:9}  
\begin{tabular}{llllllll}
\mytoprule
 & pAP & pTIM & Bunaq & Makasae & Makalero & Fataluku & Oirata\textsuperscript{1}\\
\midrule
{\bfseries initial *s} & {\bfseries *s} & {\bfseries *s} & {\bfseries s} & {\bfseries s} & \textbf{h, s} \textsuperscript{2} & {\bfseries h} & {\bfseries s} \\
bone & *ser !! & *(se)sa(r, R) !! & sesal & {}-{}- & {}-{}- & {}-{}- & {}-{}-\\
shark & * sib(a ,i)r

!! & *supor !! & {}-{}- & {}-{}- & su- & hopor(u) & {}-{}-\\
spoon & *surV !! & *sula & sulu & sulu & hulu & hula & sulu \\
weave & *sine(N) !! & *sina & sien & sina & hina & hina & hina(na) ?`\\
new & *siba(r) !! \textsuperscript{3} & *(t, s)ipa(r) !! \textsuperscript{3} & tip ?` & sufa & hofar & {}-{}- & {}-{}-\\
{\bfseries non-initial *s} & {\bfseries *s} & {\bfseries *s} & {\bfseries s} & {\bfseries s} & {\bfseries s} & {\bfseries h} & {\bfseries {\O}}\\
meat & *iser !! \textsuperscript{4} & *seor & {\itshape sael} & {\itshape seu} & {\itshape seur} & {\itshape [leura] ?`} & {\itshape [leura] ?`}\\
tooth & *-uasin !! & *-wasin !! & [(-e)we] ?` & wasi & wasi & {\ss}ahin(u) & wain(i)\\
\mybottomrule
\end{tabular} 
\textsuperscript{1} See \citet[209]{SchapperEtAl2012} for more instances of Oirata cognates.

\textsuperscript{2}\textbf{} Makalero seems to be part-way through a sound change s {\textgreater} h. See \citet[209-211]{SchapperEtAl2012} for more cognates showing the variable s \~{} h reflexes in Makalero. 

\textsuperscript{3} Cognates for these reconstructions show a relatively high degree of irregularity in both AP and TK indicating that there may have been variable realisations in not only pAP and pTIM, but also pTAP.

\textsuperscript{4} Denotes `meat' or `game'
\end{sidewaystable}


PTAP *h can be reconstructed as a word-initial segment, but not in other positions. The segment corresponds to pTIM *h and pAP *h except before back vowels (Table \ref{tab:3:10}). Based on the cognate sets available, pAP *h did not occur before back vowels. In this environment, pTAP *h changed either to *w (as in pAP *wur `moon') or was lost (as in pAP *tei `tree') in pAP (cf. Table \ref{tab:3:11} for the items and vocalic environments in which pAP *w is attested). The reconstruction pTIM *h hinges on Bunaq, which retains it as /h/, while the eastern Timor languages have all lost pTIM *h (which, in turn, reflects pTAP *h). This means that where we have no Bunaq reflex (as in the `fish', `breast' and `dream' sets) we have no modern language attesting pTIM *h, and the presence of the phoneme can only be inferred from the fact that *h is reconstructed for the pAP cognate. 

 
\begin{sidewaystable}
\caption{Correspondence set for *h}
\label{tab:3:10}  
\begin{tabular}{llllllll}
\mytoprule
 & pAP & pTIM & Bunaq & Makasae & Makalero & Fataluku & Oirata\\
\midrule
 & {\bfseries *h (*w/{\O})} & {\bfseries *h} & {\bfseries h} & {\bfseries {\O}} & {\bfseries {\O}} & {\bfseries {\O}} & {\bfseries {\O}}\\
fire & *hada !! & *haTa & {\itshape hoto} & {\itshape ata} & {\itshape ata} & {\itshape aca} & {\itshape a{\textrtailt}a}\\
fish & *habi !! & *hapi !! & -- & {\itshape afi} & {\itshape afi} & {\itshape api} & {\itshape ahi}\\
breast & *hami & *hami !! & {}-{}- & ami & {}-{}- & ami(-tapunu) & {}-{}-\\
moon & *wur & *huru & {\itshape hul} & {\itshape uru} & {\itshape uru} & {\itshape uru} & {\itshape uru}\\
tree & *tei \textsuperscript{1} & *hate !! & {\itshape hotel} & {\itshape ate} & {\itshape ate} & {\itshape ete} & {\itshape ete}\\
\mybottomrule
\end{tabular}
 
\textsuperscript{1} The loss of initial syllable may have to do with the fact that stress was apparently based on syllable weight. See also `dog' in Appendix \ref{sec:3:app:1} and \citet{HoltonRobinsonTV}. 
\end{sidewaystable}


\subsection{Reconstruction of glides}
Two glides can be reconstructed to pTAP, *w and *j. Both appear to have only occurred in initial position. It is unclear whether the reconstructed glides could occur before all vowel qualities. Nevertheless, the cognate sets supporting these proto-phonemes are robust and show little irregularity.

The pTAP glide *w shows a stable and unchanging correspondence of *w in pAP and pTIM for the most part (Table \ref{tab:3:11}). The major change is that pTAP *w is vocalised in pAP to *u root-initially on inalienably possessed nouns. In TK languages, Bunaq shows conditioned reflexes of pTAP *w, maintaining it as /w/ before front vowels, but changing it to /h/ before non-back vowels. Fataluku shows the change of *w to /{\ss}/, though we note that this is an allophone of /w/ in many languages.
 

\begin{sidewaystable}
\caption{Correspondence set for pTAP *w}
\label{tab:3:11} 
\begin{tabular}{llllllll}
\mytoprule
 & pAP & pTIM & Bunaq & Makasae & Makalero & Fataluku & Oirata\\
\midrule
 & {\bfseries *w, *u} & {\bfseries *w} & {\bfseries h, w} & {\bfseries w} & {\bfseries w} & {\bfseries {\ss}} & {\bfseries w}\\
blood & *wai & *waj & {\itshape ho} & {\itshape waj} & {\itshape wej} & {\itshape {\ss}ehe} & {\itshape we}\\
coconut & *wata !! & *wa(t, D)a & {\itshape hoza} & {\itshape wata} & {\itshape wata} & {\itshape {\ss}aca} & {\itshape wata}\\
stone & *war & *war & hol & {}-{}- & war & {}-{}- & war(aha)\\
sun & *wadi !! & *waTu & {\itshape hot} & {\itshape watu} & {\itshape watu} & {\itshape {\ss}acu} & {\itshape wa{\textrtailt}u}\\
bathe & *weli & *weru & {\itshape wer} & {\itshape waru{\textglotstop}} & {\itshape waro{\textglotstop}} & {\itshape {\ss}ahu} & {\itshape wau}\\
ear & *-uari !! & *-wali & -- & {\itshape wala(ku{\textlengthmark})} & {\itshape wali} & {\itshape {\ss}ali} & {\itshape wali}\\
tooth & *-uasin !! & *-wasin !! & {}-(e)we & wasi & wasi & {\ss}ahin(u) & wain(i)\\
\mybottomrule
\end{tabular} 
\end{sidewaystable}

Table \ref{tab:3:12} gives the four clear cognate sets that we have across TAP languages for pTAP *j. We see that pTAP *j is maintained as *j in pAP, but is variably lost or maintained as *j in pTIM. It may be that differing vocalic environments in pTAP conditioned the different reflexes in pTIM, but we don't have enough understanding of the history of vowels yet to determine this. There is no direct evidence for pTIM *j, that is, no TK language still reflects the proto-phoneme as /j/, but the sound correspondences between TK languages make it differentiable form sets reflecting pTIM *h (see Table \ref{tab:3:10}).
 

\begin{sidewaystable}
\caption{Correspondence set for pTAP *j}
\label{tab:3:12}  
\begin{tabular}{llllllll}
\mytoprule
 & pAP & pTIM & Bunaq & Makasae & Makalero & Fataluku & Oirata\\
\midrule
 & {\bfseries *j} & {\bfseries j, {\O}} & {\bfseries h, {\O}} & {\bfseries h, {\O}} & {\bfseries h, {\O}} & {\bfseries {\O}} & {\bfseries {\O}}\\
star & *jibV & *ipi(-bere) & {\itshape [bi]} & {\itshape ifi-bere} & {\itshape ifi} & {\itshape ipi(-naka)} & {\itshape ihi}\\
water & *jira & *ira & il & ira & ira & ira & ira\\
laugh & *jagir !! & *jiger !! & higal & {\itshape hi{\textglotstop}a} & {\itshape hi{\textglotstop}e} & he{\textglotstop}e ?` & {}-{}-\\
path & *jega !! & *jiga !! & {\itshape hik} & {\itshape hi{\textglotstop}a} & {\itshape hi{\textglotstop}a} & {\itshape i{\textglotstop}a} & {\itshape ia(ra)}\\
\mybottomrule
\end{tabular} 
\end{sidewaystable}

\subsection{Reconstruction of liquids}
We identify three robust liquid correspondence sets between pAP and pTIM and as such reconstruct three pTAP liquids: *r, *R, and *l.

The most robust set is that for pTAP *r, which is reflected as *r in both pAP and pTIM (Table \ref{tab:3:13}). PTAP *r is only found in non-initial positions, as are its reflexes in the daughter languages pAP and pTK. Word-finally in polysyllabic words pTAP *r is particularly susceptible to sporadic loss, as is attested by the various irregular forms in Table \ref{tab:3:13}. In one instance (pTAP *sibar `new'), the occurrence of a reflex of final *r is so erratic in both primary subgroups that we perhaps must consider it already partly lost in pTAP's daughter languages.

 
\begin{sidewaystable}
\caption{Correspondence set for pTAP *r}
\label{tab:3:13}  
\begin{tabular}{lllllllllll}
\mytoprule
 & \multicolumn{2}{l}{pAP

} & \multicolumn{2}{l}{pTIM

} & \multicolumn{2}{l}{Bunaq

} & Makasae & Makalero & Fataluku & Oirata\\
\midrule
 & \multicolumn{2}{l}{{\bfseries *r}

} & \multicolumn{2}{l}{{\bfseries *r}

} & \multicolumn{2}{l}{l

} & r & r & r & r\\
run & \multicolumn{2}{l}{*tiara

} & \multicolumn{2}{l}{*tifar

} & \multicolumn{2}{l}{{\itshape t{\textesh}iwal}

} & {\itshape ditar} & {\itshape titar} & {\itshape tifar(e)} & {\itshape tipar(e)}\\
moon & \multicolumn{2}{l}{*wur

} & \multicolumn{2}{l}{*huru

} & \multicolumn{2}{l}{{\itshape hul}

} & {\itshape uru} & {\itshape uru} & {\itshape uru} & {\itshape uru}\\
rat & \multicolumn{2}{l}{*dur

} & \multicolumn{2}{l}{*Dura

} & \multicolumn{2}{l}{{\itshape zul}

} & {\itshape dura} & {\itshape sura} & {\itshape cura} & {\itshape {\textrtailt}ura}\\
stone & \multicolumn{2}{l}{*war

} & \multicolumn{2}{l}{*war

} & \multicolumn{2}{l}{{\itshape hol}

} & -- & {\itshape war} & -- & {\itshape war(aha)}\\
vagina & \multicolumn{2}{l}{*-ar

} & \multicolumn{2}{l}{*-aru

} & \multicolumn{2}{l}{--

} & {\itshape aru} & {\itshape aru} & {\itshape aru} & {\itshape aru}\\
water & \multicolumn{2}{l}{*jira

} & \multicolumn{2}{l}{*ira

} & \multicolumn{2}{l}{{\itshape il}

} & {\itshape ira} & {\itshape ira} & {\itshape ira} & {\itshape ira}\\
crawl & \multicolumn{2}{l}{*er !!

} & \multicolumn{2}{l}{*er !!

} & \multicolumn{2}{l}{{\itshape el}

} & -- & -- & {\itshape er(eke)} & --\\
dream & \multicolumn{2}{l}{*hipar

} & \multicolumn{2}{l}{*ufar(ana) !!

} & \multicolumn{2}{l}{{\itshape [waen] ?`}

} & {\itshape ufarena} & {\itshape ofarana} & {\itshape ufar(e)} & {\itshape upar(a)}\\
meat & *iser !! & \multicolumn{2}{l}{*seor

} & \multicolumn{2}{l}{{\itshape sael}

} & \multicolumn{2}{l}{{\itshape [seu] ?`}

} & {\itshape seur} & {\itshape leura} & {\itshape leura}\\
dog & \multicolumn{2}{l}{*jibar !!

} & \multicolumn{2}{l}{*Depar

} & \multicolumn{2}{l}{[zap] ?`

} & [defa] ?` & sefar & ipar(u) & ihar(a)\\
bamboo & \multicolumn{2}{l}{*mari

} & \multicolumn{2}{l}{*mari

} & \multicolumn{2}{l}{{\itshape [ma] ?`}

} & {\itshape maeri} & {\itshape mar} & -- & --\\
{\itshape P. indicus} & \multicolumn{2}{l}{*matar !!

} & \multicolumn{2}{l}{*ma(t, D)er

} & \multicolumn{2}{l}{{\itshape [mazo{\textglotstop}] ?`}

} & {\itshape mater} & {\itshape mater} & {\itshape matar(ia)} & --\\
shark & \multicolumn{2}{l}{*sib(a, i)r !! 

} & \multicolumn{2}{l}{*supor !!

} & \multicolumn{2}{l}{{}-{}-

} & {}-{}- & [su] ?` & hopor(u) & {}-{}-\\
new & \multicolumn{2}{l}{*siba(r) !! 

} & \multicolumn{2}{l}{*(t, s)ipa(r) !!

} & \multicolumn{2}{l}{[tip] ?`

} & [sufa] ?` & hofar & {}-{}- & {}-{}-\\
\mybottomrule
\end{tabular} 
\end{sidewaystable}

PTAP *R is reflected in pAP as *r and in pTIM as *l. Like pTAP *r, *R does not appear in word-initial positions and is sporadically lost word-finally in polysyllabic words. The sets supporting the reconstruction of *R (Table \ref{tab:3:14}) are also fewer and less robust than for pTAP *r. 
 

\begin{sidewaystable}
\caption{Correspondence set for pTAP *R}
\label{tab:3:14}  
\begin{tabular}{llllllll}
\mytoprule
 & pAP & pTIM & Bunaq & Makasae & Makalero & Fataluku & Oirata\\
\midrule
 & {\bfseries *r} & {\bfseries *l} & {\bfseries l} & {\bfseries l} & {\bfseries l} & {\bfseries l} & {\bfseries l}\\
spoon & *surV !! & *sula & sulu & sulu & hulu & hula & {}-{}-\\
tail & *-ora !! & *-ula({\textglotstop}) & {\itshape {}-ulo({\textglotstop})} & {\itshape ula} & {\itshape ula} & {\itshape ula(fuka)} & {\itshape ula(pua)}\\
tongue & *-lebur !! & *-ipul & {\itshape [-up] ?`} & {\itshape [ifi] ?`} & {\itshape ifil} & {\itshape epul(u)} & {\itshape uhul(u)}\\
laugh & *jagir !! & *jiger & higal & [hi{\textglotstop}a] ?` & [hi{\textglotstop}a] ?` & [he{\textglotstop}e] ?` & {}-{}-\\
spit & *purVn !! & *fulu(k, n) !! & {\itshape puluk} & -- & {\itshape fulun} & {\itshape fulun} & --\\
ear & *-uari !! & *-wali & -- & {\itshape wala(ku{\textlengthmark})} & {\itshape wali} & {\itshape {\ss}ali} & {\itshape wali}\\
\mybottomrule
\end{tabular} 
\end{sidewaystable}

The three pTIM cognates listed in Table \ref{tab:3:15} are based on Bunaq only, in which pTIM *r and *R are merged. We have thus no means of determining whether these forms are to be reconstructed to pTAP with *r or with *R.
 


\begin{sidewaystable}
\caption{Cognate sets reconstructable to either pTAP *r or *R}
\label{tab:3:15}  
\begin{tabular}{llllllll}
\mytoprule
 & pAP & PTIM & Bunaq & Makasae & Makalero & Fataluku & Oirata\\
\midrule
 & {\bfseries *r} & {\bfseries *(r, R)} & {\bfseries l} & {\bfseries {}-{}-} & {\bfseries {}-{}-} & {\bfseries {}-{}-} & {\bfseries {}-{}-}\\
bone & *ser !! & *(se)sa(r, R) !! & {\itshape sesal} & -- & -- & -- & --\\
scorpion & *pVr & *fe(r, R)e !! & {\itshape wele} & -- & -- & -- & --\\
rain & *anur !! & *ine(r, R) !! & {\itshape inel} & -- & -- & -- & --\\
\mybottomrule
\end{tabular} 
\end{sidewaystable}

Cognate sets for pTAP *l are relatively infrequent in both pAP and pTIM (Table \ref{tab:3:16}).\footnote{\citet{HoltonRobinsonTV} remark that, even though correspondences appear relatively regular for initial and medial *l in pAP, they can identify only a few cognates that are widely distributed across the AP subgroup. Similarly, \citet{SchapperEtAl2012} caution that their reconstruction for pTIM *l cannot yet be called secure due to the small number of cognate sets identified.} Cognates reflecting initial pTAP *l with pAP *l and pTIM *l (i.e., `bark', `new place' and `crouch') have only a low degree of certainty. Based on the data available, there also appears to be a tendency to lose pTAP initial *l in pTIM, as in `far', `tongue' and `green', but a clear conditioning environment for this is not yet obvious. Word-finally in polysyllabic words, pTAP *l is regularly lost in pTIM, as in `banana', `bat', `bird' and `taboo', However, it is retained in `walk 2' and `six', apparently due to nasal-liquid metathesis, and in `child' due to the loss of the item's medial syllable with *q prior to the application of the final polysyllabic deletion rule in pAP.
 

\begin{sidewaystable}
\caption{Correspondence set for pTAP *l}
\label{tab:3:16}  
\begin{tabular}{llllllll}
\mytoprule
 & pAP & pTIM & Bunaq & Makasae & Makalero & Fataluku & Oirata\\
\midrule
 & {\bfseries *l} & {\bfseries *l, {\O}} & {\bfseries l ({\O})} & {\bfseries l ({\O})} & {\bfseries l ({\O})} & {\bfseries l ({\O})} & {\bfseries l ({\O})}\\
bark & *lVu & *le(k)u(l) !! & -- & {\itshape leu} & {\itshape leu} & {\itshape le{\textglotstop}ul(e)} & {\itshape leul(e)}\\
new place & *lan !! & *lan !! & {\itshape lon} & -- & -- & -- & --\\
crouch & *luk(V) & *luk !! & {\itshape lu{\textglotstop}(-lu{\textglotstop})} & -- & -- & -- & --\\
far & *lete !! & [*eTar] !! & ate & {}-{}- & {}-{}- & icar & {}-{}-\\
tongue & *-lebur !! & [*-ipul] & {}-up & ifi & ifil & epul(u) & uhul(u)\\
green & *(wa)logar !! & [*ugar] & {\itshape ugar} & {\itshape hu{\textglotstop}ur} & {\itshape (h)u{\textglotstop}ur} & {\itshape u{\textglotstop}ur(eke)} & {\itshape u{\textglotstop}ul(e)}\\
banana & *mogol & [*mugu] !! & {\itshape mok} & {\itshape mu{\textglotstop}u} & {\itshape mu{\textglotstop}u} & {\itshape mu{\textglotstop}u} & {\itshape mu{\textlengthmark}}\\
bat & *madel & [*maTa] !! & -- & -- & -- & {\itshape maca} & {\itshape ma{\textrtailt}a}\\
bird & *(a)dVl !! & [*haDa] & {\itshape hos} & {\itshape asa} & {\itshape asa} & {\itshape aca} & {\itshape asa}\\
taboo & *palol !! & [*falu(n)] & {\itshape por} & {\itshape falun} & {\itshape falun} & {\itshape falu} & --\\
walk 2 & *lam(ar) !! & *male !! & {\itshape mele} & {}-{}- & {}-{}- & {}-{}- & {}-{}-\\
six & *talam & *tamal !! & tomol & {}-{}- & {}-{}- & {}-{}- & {}-{}-\\
child & *-uaqal & *-al !! & {\itshape {}-ol} & -- & -- & -- & --\\
\mybottomrule
\end{tabular} 
\end{sidewaystable}

Finally, there are several cases in which the appearance of liquids in AP and TK languages can be reconciled with none of the three sets we have identified here. Table \ref{tab:3:17} lists these problematic instances (the relevant segments are bolded). These sets pointedly express that we are still a long way away from a complete understanding of liquids in pTAP. 
 
\begin{sidewaystable}
\caption{Problematic liquid cognate sets}
\label{tab:3:17}  
\begin{tabular}{llllllll}
\mytoprule
 & pAP & pTIM & Bunaq & Makasae & Makalero & Fataluku & Oirata\\
\midrule
price & *bo\textbf{l} !! & *bu\textbf{r}a & {\itshape bo\textbf{l}} & {\itshape bu\textbf{r}a} & {\itshape pu\textbf{r}a} & {\itshape pu\textbf{r}a} & {\itshape hu\textbf{r}a}\\
bathe & *we\textbf{l}i & *we\textbf{r}u & {\itshape we\textbf{r}} & {\itshape wa\textbf{r}u{\textglotstop}} & {\itshape wa\textbf{r}o{\textglotstop}} & {\itshape vahu ?`} & {\itshape wau ?`}\\
garden & *maga\textbf{d}(a) & *(u, a)ma\textbf{r} & {\itshape ma\textbf{r}} & {\itshape [ama]} & {\itshape [ama]} & -- & {\itshape [uma]}\\
green & *(wa)loga\textbf{r} !! & *uga\textbf{r} & {\itshape uga\textbf{r}} & {\itshape (h)u{\textglotstop}u\textbf{r}} & {\itshape (h)u{\textglotstop}u\textbf{r}} & {\itshape u{\textglotstop}u\textbf{r}(eke)} & {\itshape u{\textglotstop}u\textbf{l}(e) ?`}\\
Taboo & *pa\textbf{l}ol !! & *fa\textbf{l}u(n) & po\textbf{r} & fa\textbf{l}un & fa\textbf{l}un & fa\textbf{l}u & {}-{}-\\
\mybottomrule
\end{tabular} 
\end{sidewaystable}

\subsection{Reconstruction of nasals}
Two nasals can be reconstructed to pTAP, *m and *n. For the most part, they are relatively stable and unchanging in both pAP and pTIM.

Table \ref{tab:3:18} presents a selection of the many cognate sets for pTAP *m. In word-initial position, pTAP *m corresponds unproblematically to pAP *m and pTIM *m. Identifying non-initial instances of pTAP *m is somewhat more difficult, with *hami `breast' being the only straightforward case. Word-final *m in pAP has only non-final reflexes in pTIM, apparently because, as in the modern TK languages, word-final *m was not permitted. This issue is resolved in pTIM through metathesis of the nasal out of the final position, as in `sea' and `six'. Other instances of medial pTIM *m correspond to root-initial *m in pAP (as in `garden' and `die'). 
 
\begin{sidewaystable}
\caption{Correspondence sets for pTAP *m}
\label{tab:3:18}  
\begin{tabular}{llllllll}
\mytoprule
 & pAP & pTIM & Bunaq & Makasae & Makalero & Fataluku & Oirata\\
\midrule
{\bfseries initial *m} & {\bfseries *m} & {\bfseries *m} & {\bfseries m} & {\bfseries m} & {\bfseries m} & {\bfseries m} & {\bfseries m}\\
bamboo & *mari & *mari & {\itshape ma} & {\itshape maeri} & {\itshape mar} & -- & --\\
banana & *mogol & *mugu !! & {\itshape mok} & {\itshape mu{\textglotstop}u} & {\itshape mu{\textglotstop}u} & {\itshape mu{\textglotstop}u} & {\itshape mu{\textlengthmark}}\\
sit & *mis & *mit & {\itshape mit} & {\itshape mit \~{} mi} & {\itshape mit} & {\itshape (i)mir(e)} & {\itshape mir(e)}\\
bat & *madel & *maTa !! & -- & -- & -- & {\itshape maca} & {\itshape ma{\textrtailt}a}\\
inside & *mi & *mi & mi(l) & mu(tu) & mu(tu-) & mu(cu) & mu({\textrtailt}u)\\
hear & *magi !! & *mage(n) !! & mak & ma{\textglotstop}en & ma{\textglotstop}en & {}-{}- & {}-{}-\\
{\bfseries non-initial *m} & {\bfseries *m} & {\bfseries *m} & m & m & m & m & m\\
breast & *hami & *hami !! & -- & {\itshape ami} & -- & {\itshape ami(-tapunu)} & --\\
sea & *tam & *mata & {\itshape mo} & -- & -- & {\itshape mata} & {\itshape mata}\\
six & *talam & *tamal !! & {\itshape tomol} & -- & -- & -- & --\\
garden & *magad(a) & *(u, a)mar !! & {\itshape mar} & {\itshape ama} & {\itshape ama} & -- & {\itshape uma}\\
die & *min(a) & *-umV & {\itshape {}-ume} & {\itshape umu} & {\itshape (k)umu} & {\itshape umu} & {\itshape umu}\\
nose & *-mim & *-muni !! & {\itshape [-inup] ?`} & {\itshape muni(kai)} & {\itshape mini} & {\itshape mini(ku)} & --\\
\mybottomrule
\end{tabular} 
\end{sidewaystable}

Table \ref{tab:3:19} presents the many cognate sets for pTAP *n. Initial and medial correspondences are abundant, but final correspondences are difficult to identify. pTIM *n did not appear to occur in final position; all instances of pAP final *n are either followed by a vowel or are lost in pTIM.
 

\begin{sidewaystable}
\caption{Correspondence sets for *n}
\label{tab:3:19}  
\begin{tabular}{llllllll}
\mytoprule
 & pAP & pTIM & Bunaq & Makasae & Makalero & Fataluku & Oirata\\
\midrule
{\bfseries initial *n} & {\bfseries *n} & {\bfseries *n} & {\bfseries n} & {\bfseries n} & {\bfseries n} & {\bfseries n} & {\bfseries n}\\
stand & *nate(r) !! & *nat & {\itshape net} & {\itshape nat \~{} na} & {\itshape nat} & {\itshape (a)nat(e)} & {\itshape nat(e)}\\
{\scshape 1sg} & *na- & *n-~!! & {\itshape n-} & -- & -- & -- & --\\
eat & *nai & *nua !! & {\itshape [a \~{} -ia]} & {\itshape nawa} & {\itshape nua} & {\itshape una, na$\beta $a} & {\itshape una, nawa}\\
one & *nuk & *uneki~!! & {\itshape uen, en} & {\itshape [u]} & {\itshape [u] \~{} un} & {\itshape ukani} & {\itshape a{\textglotstop}uni}\\
{\bfseries non-initial *n} & {\bfseries *n} & {\bfseries *n} & {\bfseries n} & {\bfseries n} & {\bfseries n} & {\bfseries n} & {\bfseries n}\\
face & *-pona !! & *-fanu !! & {}-(e)wen & fanu & fanu & fanu & panu\\
ripe & *tena !! & *tena !! & ten & tina & tina & {}-{}- & {}-{}-\\
name & *-en(i, u) !! & *-nej & {\itshape {}-ini(l)} & {\itshape naj} & {\itshape nej} & {\itshape ne} & {\itshape ne{\textlengthmark}(ne)}\\
give & *-ena & *-inV & {\itshape {}-ini} & {\itshape (g)ini} & {\itshape (k-)ini} & {\itshape ina} & {\itshape ina}\\
wake & *-ten & *Tani & otin & tane & tane & tani \~{} cani & {}-{}-\\
girl & *pon !! & *fana & pana & fana(rae) & fana(r) & fana(r) & pana(rai)\\
person & *anin !! & *anu~!! & {\itshape en} & {\itshape anu} & {\itshape anu} & -- & --\\
other & *aben(VC) !! & *epi !! & [ewi] & {}-{}- & {}-{}- & {}-{}- & {}-{}-\\
\mybottomrule
\end{tabular} 
\end{sidewaystable}

\section{Summary of correspondences and reconstructed phonemes}\label{sec:3:3}
For the first time since the start of TAP studies some sixty years ago (see Schapper and Huber forthcoming for a historical perspective on TAP studies), we have rigorously shown in this chapter that the TAP languages form a family: the regularity of sound correspondences in cognate vocabulary demonstrates that the AP and TK Papuan languages are indeed genetically related to one another.

In Table \ref{tab:3:20}, we provide an overview of the consonant correspondences we observed in cognate vocabulary between pAP and pTIM and their reconstruction in their ancestral language pTAP. In this table, we indicate whether the correspondence applies in initial (\#\_ ), medial (V\_V), or final ( \_\#) position. An empty slot means that there is no particular conditioning environment for the correspondence. The symbol `{\O}' in a column indicates that a pTAP sound is lost in the daughter language in question. 
 
\begin{table}\centering 
\begin{tabular}{llll}
\mytopline 
pTAP\ilt{proto-Timor-Alor-Pantar}&environment&pAP\ilt{proto-Alor-Pantar}&pTIM\\\midrule  
*p&&*p&*f\\
*b&\#\_&*b&*b\\[.4em]
\multirow{3}{*}{*t}  &V\_V&*b&*p\\
 &\#\_&*t&*t\\
  &V\_V, \_\#&*t, *s&*t\\[.4em]
\multirow{2}{*}{*d}&\#\_&*d&*D\\
&V\_V&*d&*T\\[.4em]
*k&&*k&*k\\
*g&&*g&*g\\
*s&&*s&*s\\
*h&&*h (*w/{\O})&*h\\
*w&&*w, *u&*w\\
*j&&*j&*j, {\O}\\
*r&&*r&*r\\
*R&&*r&*l\\
*l&&*l&*l, {\O}\\
*m&&*m&*m\\
*n&&*n&*n\\
\mybottomline
\end{tabular}
\caption{No caption!}
\label{tab:3:20}
\end{table}

\section{Discussion}\label{sec:3:4}
Whilst we have been able to show clearly that AP and TK languages are related to one another, the comparative data presented here draws into question a number of aspects of the existing reconstructions of pAP and pTIM and necessitates revisions to these. In this final section, we will draw attention to the issues, provide a general discussion of them and suggest some possible solutions to them. 

A major issue for the current pAP reconstruction is the apparent invalidity of many word- final consonant reconstructions. It is argued in \citet[95]{HoltonEtAl2012} that the gemination of medial stops in modern Western Pantar can be used as a diagnostic for determining whether a given pAP root was consonant-final or vowel-final. Specifically, the authors claim that geminate medial stops in modern Western Pantar reflect pAP medial stops, whereas non-geminate medial stops in Western Pantar reflect an original consonant-final form, or perhaps a borrowing from another AP language. However, this argument cannot be sustained on closer inspection of the comparative evidence. Consider the items in Table \ref{tab:3:21} that are reconstructed as basically consonant final in pAP, because of the lack of stop gemination in WP. In each case, we have between three and nine reflexes in modern AP languages with a V(C) following the supposed historically final consonant. We must ask ourselves where so many additional final segments came from in so many of these languages. \citet{HoltonEtAl2012} seek to explain these appearances with vowel epenthesis. Yet, under this scenario, we would expect to be able to predict the type of the epenthetic vowel from the shape of the root, but this is not the case; instead, the epenthetic vowels are of all different values from one item to the next and bear no apparent relationship to the vowel of the root (as defined by \citealt{HoltonEtAl2012}). What is more, the final V(C) elements we observe in AP languages are not erratic, rather they in general adhere to correspondences observed elsewhere. This suggests that these final V(C) elements were not epenthetic to the items after the break-up of pAP, but have been inherited from pAP. This is further supported by the fact that we find clearly corresponding V(C) segments on cognate vocabulary in TK languages, meaning that the segments reconstruct to pTAP and that they were inherited into pAP. The alternative leaves us without explanation for the cognacy of the final segments in these (and other items) across the family. 
 

\begin{sidewaystable}
\caption{Dubious consonant-final reconstructions in AP and beyond}
\label{tab:3:21} 
\begin{tabular}{lllllll}
\mytoprule
 & `fish' & `sun' & `fire' & `coconut' & `tongue' & `ripe'\\
\midrule
pTAP & {\bfseries *habi} & {\bfseries *wad(u, i)} & {\bfseries *hadi} & {\bfseries *wata} & {\bfseries *(l)ebur} & {\bfseries *tena}\\
pTIM & *hapi & *waTu & *haTa & *wa(t, D)a & *-ipul & *tena\\
pAP original & *hab(i) & *wad(i) & *had(a) & *wat(a) & *-leb(ur) & *ten\\
Teiwa & {\itshape {\pharfric}a{\textphi}} & {\itshape war} & {\itshape {\pharfric}ar} & {\itshape wat} & {\itshape {}-livi} & {\itshape tanan}\\
Nedebang & {\itshape a{\textlengthmark}fi} & {\itshape (get)} & {\itshape ar} & {\itshape wata} & {\itshape {}-lefu} & --\\
Kaera & {\itshape ab} & {\itshape wer} & {\itshape ad} & {\itshape wat} & {\itshape {}-leb} & {\itshape ten} \\
WPantar & {\itshape hap} & {\itshape wer} & {\itshape a{\textlengthmark}d} & {\itshape wata} & {\itshape {}-lebu} & {\itshape ta{\ng}}\\
Blagar & {\itshape a{\textlengthmark}b} & {\itshape war} & -- & {\itshape vet} & {\itshape {}-lebul} & {\itshape tena}\\
Adang & {\itshape a{\textlengthmark}b} & {\itshape ved} &  & {\itshape fa{\textglotstop}} & {\itshape {}-lib(u{\ng})} & {\itshape tene}\\
Klon & {\itshape {\textschwa}bi} & {\itshape f{\textepsilon}d} & {\itshape {\textschwa}da} & -- & {\itshape {}-l{\textepsilon}b} & {\itshape {\textschwa}t{\textepsilon}n}\\
Kui & {\itshape eb} & -- & {\itshape ar} & {\itshape bat} & {\itshape {}-liber} & {\itshape tain}\\
Abui & {\itshape afu} & {\itshape wari} & {\itshape ara} & {\itshape wata} & {\itshape {}-lifi} & --\\
Kamang & {\itshape api} & {\itshape wati} & {\itshape ati} & {\itshape wate} & {\itshape {}-opui \textsuperscript{1}} & {\itshape iten\~{}iton}\\
Sawila & {\itshape api} & {\itshape wadi} & {\itshape ada} & {\itshape wata} & {\itshape {}-li(m)puru} & {\itshape iti{\textlengthmark}na}\\
Wersing & {\itshape api} & {\itshape widi} & {\itshape ada} & {\itshape wata} & {\itshape {}-jebur} & --\\
\mybottomrule
\end{tabular} 

\textsuperscript{1} \citet{HoltonEtAl2012} state that these and other Kamang forms missing pAP *l medially are irregular. However, pAP *l is regularly lost in Kamang between non-front vowels, e.g., pAP *talam `six' {\textgreater} Kamang \textit{ta{\textlengthmark}m}, pAP *palol `taboo'{\textgreater} Kamang \textit{fo{\textlengthmark}i} etc. The vowel of the inalienable possessive prefix is /a/, thus providing the right environment for the loss in \textit{{}-opui} `tongue' of the root-initial /l/.  
\end{sidewaystable}


The problem then is how to explain medial geminate and non-geminate stops in WP. One answer would to be maintain that the difference in stop gemination was still due to a final- non-final distinction. For example, it could be said that the loss of the final vowel occurred after the breakup of pAP but prior to the application of the gemination rule. This cannot, however, be fully sustained as WP has in some cases final vowels which clearly reflect pTAP and pAP (e.g., `tongue'). A more attractive explanation is presented by stress-induced gemination. Although little is known about the historical prosody of TAP, it seems a good possibility that WP gemination may have been a result of final stress. That is, we suggest roots of the shape /(C)V{\textprimstress}CV(C)/ surfaced as [(C)V{\textprimstress}C{\textlengthmark}V(C)], while roots of the shape /{\textprimstress}(C)VCV(C)/ surfaced as [{\textprimstress}(C)VCV(C)]. While this scenario remains to be confirmed by a more detailed study, discarding the final/non-final explanation for geminates in WP allows for a more satisfactory account of final segments in TAP.

A second issue for the pAP reconstruction is the presence of many unexplained phonemes in a range of environments in different languages. Velars, post-velar and laryngeal consonants are a case in point. Most of the complexity in this domain is found in the languages of Pantar and the Pantar Straits, whose phoneme inventories generally include not only velar and glottal stops, but also uvular ones, as well as a velar or pharyngeal fricative next to the glottal fricative /h/. This contrasts with the situation as found in most of Alor and the TK languages, which tend to be rather simpler. Table \ref{tab:3:22} exemplifies the velar and post-velar plosives and fricatives in a language of Pantar (Teiwa), Alor (Kamang), and Timor (Bunaq).



\begin{table} 
\begin{tabular*}{.7\textwidth}{@{\extracolsep{\fill}}rcccc}
\mytopline
         & \multicolumn{4}{c}{Teiwa\ilt{Teiwa}}                   \\ 
         & velar& uvular& pharyngeal  & glottal        \\
\midrule
plosive  &  k g &   q   &             & {\textglotstop}\\
fricative&      &       &{\pharfric}  & h              \\
\\
\mybottomline
\end{tabular*}
 
\begin{tabular*}{\textwidth}{@{\extracolsep{\fill}}rcccc}
\mytopline
 &         \multicolumn{2}{c}{Kamang\ilt{Kamang}}& \multicolumn{2}{c}{Bunaq\ilt{Bunaq}}\\
         & velar& glottal& velar & glottal\\
\midrule
plosive  & k g& ({\textglotstop})& k g & {\textglotstop} \\
fricative&    &  (h)    &        & h\\
\mybottomline
\end{tabular*}
\caption{Velar and post-velar phonemes in TAP languages}
\label{tab:3:22} 
\end{table} 

The existing pAP reconstruction leaves a significant part of the complexity in the (post\nobreakdash-)velar domain in the Pantar languages unexplained; for instance, it does not account for /g/ in Blagar and the relation between the various (post-)velar phonemes such as /q/ and /x/ found in different dialects of Blagar.\footnote{See \citet{Steinhauer1995}.} It also does not explain the origin of /{\textglotstop}/ in languages other than Blagar and Adang, and does not give reflexes for pAP medial *k in Teiwa and pAP final *k in Sawila, leaving the field in question blank in the table summarising the correspondences \citep{HoltonRobinsonTV}. Finally, note a variety of irregularities in the reconstructions involving velars in Appendix \ref{sec:3:app:1}, especially in the Pantar languages. In short, the frequency of irregularities and unexplained occurrences of (post-)velar phonemes shows how limited our understanding of this domain in AP still is, and serves as a reminder that much more extensive reconstruction work needs to be undertaken.

A similar issue is presented by the phonemic velar nasal /{\ng}/ in many AP languages. This phoneme is not reconstructed for pAP, and is also absent in all of the TK languages. According to \citet{HoltonRobinsonTV}, pAP *n became /{\ng}/ in word-final position in all AP languages except Teiwa, where it was retained as /n/. This historical scenario does work well for some languages, for instance, Wersing, where [{\ng}] is synchronically a word-final allophone of /n/. However, in other languages, questions remain. For instance, Kamang has an unexplained contrast between /{\ng}/ and /n/ in codas (e.g., \textit{ee{\ng}} `\textsc{2sg.poss'} versus \textit{een} `\textsc{2sg.foc'}). Similarly, the existence of /{\ng}/ in coda and medial position in Teiwa is unexplained, as well as the occurrence of /{\ng}/ in other positions than the final one in various languages (e.g. Sar \textit{la{\ng}ja} `digging stick' and Kula \textit{{\ng}apa} `father'). 

Vowels also present a major challenge to the reconstruction of the ancestral TAP language. The various vowel systems as illustrated in Table \ref{tab:3:23} are yet to be historically reconciled with one another. Most AP languages have a length distinction in their vowels: the most common system is 5 short and 5 long cardinal vowels (Kaera, Blagar, Abui and Kamang), though matching long vowels may be missing in the mid-vowel range (Teiwa and Klon). Blagar has a marginal length distinction with only a small number of items occurring with long vowels (Steinhauer forthcoming), while it is Klon's short mid-vowels that are marginal. A length distinction is entirely absent from WP's and Wersing's five vowel system and Adang's seven vowel system. A relationship, if any, between the mid-vowels in Adang and length distinctions in other languages remains to be established. Non-cardinal vowels are found in Sawila /y, y{\textlengthmark}/ and in Klon /{\textschwa}/. TK languages all have simple five cardinal vowels and there is a marginal length distinction in only one language, Makalero. Stress in conjunction with length appears to have played an important role in vowel histories. For instance, Klon /{\textschwa}/ seems to originate in a short, unstressed pAP *a (e.g., Klon \textit{{\textschwa}bi} appears to go back to pAP *ha{\textprimstress}bi `fish'). In Wersing, historically short unstressed vowels are lost in words with long vowels, which in turn become short stressed vowels (e.g., Wersing \textit{tlam} appears to go back to pAP *tala{\textlengthmark}m `six', cf. Abui \textit{tala{\textlengthmark}ma}). In short, much careful bottom-up reconstructive work needs to be done in order to reconcile these different systems to a single ancestral system.
 

\begin{table}\centering
\caption{TAP vowel systems}
\label{tab:3:23}


\setlength{\tabcolsep}{0pt} 
\begin{tabular}{cp{2cm}c}  
\mytoprule 
\begin{tabular}{p{.7cm}p{.7cm}p{.7cm}p{.7cm}p{.7cm}}
  \multicolumn{5}{c}{Western Pantar\ilt{Western Pantar}}\\
\midrule
  i &      &      &      &  u\\
    &   e  &      &   o   &  \\ 
    &      &   a  &      &   \\ 
\end{tabular}
  && 
\begin{tabular}{p{.7cm}p{.7cm}p{.7cm}p{.7cm}p{.7cm}}
  \multicolumn{5}{c}{Teiwa\ilt{Teiwa}}\\
\midrule

  i i:&     &      &      &  u u:\\
    &   e  &      &   o   &  \\ 
    &      &   a a:  &      &   \\ 
\end{tabular}
 \\\\ 
\begin{tabular}{p{.7cm}p{.7cm}p{.7cm}p{.7cm}p{.7cm}}
  \multicolumn{5}{c}{Kaera\ilt{Kaera}}\\
\midrule

  i i:&      &      &      &  u u:\\
    & e  e: &      &   o o:  &  \\ 
    &      &   a a: &      &   \\
  
\end{tabular}
&& 
\begin{tabular}{p{.7cm}p{.7cm}p{.7cm}p{.7cm}p{.7cm}}
  \multicolumn{5}{c}{Blagar\ilt{Blagar}}\\
\midrule

  i i:&      &      &      & u u:\\
    & e  e: &      &   o o:     &  \\ 
    &      &   a a: &      &   \\
  
\end{tabular}
\\\\ 
\begin{tabular}{p{.7cm}p{.5cm}p{.3cm}p{.3cm}p{.3cm}p{.7cm}p{.7cm}}
  \multicolumn{5}{c}{Adang\ilt{Adang}}\\
\midrule

  i &      &   &&   &      & u \\
    &  e   &   &&   &~~  o   &   \\
    &      &{\textepsilon}  &      &  {\textopeno} &   &   \\
    &      &  & ~a&   &      &   \\
  
\end{tabular}
  && 
\begin{tabular}{p{.7cm}p{.7cm}p{.7cm}p{.7cm}p{.7cm}}
  \multicolumn{5}{c}{Klon\ilt{Klon}}\\
\midrule

  i i: &      &      &      & u u: \\
    &  e   &      &  o o:  &   \\
    &  {\textepsilon}  {\textepsilon}: & ~~~{\textschwa}    &  {\textopeno}   &   \\
    &      &  a a:  &      &   \\
  
\end{tabular}
\\\\ 
\begin{tabular}{p{.7cm}p{.7cm}p{.7cm}p{.7cm}p{.7cm}}
  \multicolumn{5}{c}{Abui\ilt{Abui}}\\
\midrule

  i i: &      &      &      &  u u:\\
    &   e e:  &      &   o o:  &  \\ 
    &      &   a a:  &      &   \\
  
\end{tabular}
  && 
\begin{tabular}{p{.7cm}p{.7cm}p{.7cm}p{.7cm}p{.7cm}}
  \multicolumn{5}{c}{Kamang\ilt{Kamang}}\\
\midrule

  i i: &      &      &      &  u u:\\
    &   e e:  &      &   o o:  &  \\ 
    &      &   a a:  &      &   \\
  
\end{tabular}
\\\\ 
\begin{tabular}{p{.7cm}p{.7cm}p{.7cm}p{.7cm}p{.7cm}}
  \multicolumn{5}{c}{Sawila\ilt{Sawila}}\\
\midrule

    i\,i:\,y\,y:&      &      &      & u u:  \\
    &   e e:   &      & o o:     &   \\
    &      & a a:     &      &   \\
  
\end{tabular}
&& 
\begin{tabular}{p{.7cm}p{.7cm}p{.7cm}p{.7cm}p{.7cm}}
  \multicolumn{5}{c}{Wersing\ilt{Wersing}}\\
\midrule

  i &      &      &      &  u\\
    &   e  &      &   o   &  \\ 
    &      &   a  &      &   \\
  
\end{tabular}
\\\\ 
\begin{tabular}{p{.7cm}p{.7cm}p{.7cm}p{.7cm}p{.7cm}}
  \multicolumn{5}{c}{Bunaq\ilt{Bunaq}}\\
\midrule

  i &      &      &      &  u\\
    &   e  &      &   o   &  \\ 
    &      &   a  &      &   \\
  
\end{tabular}
  && 
\begin{tabular}{p{.7cm}p{.7cm}p{.7cm}p{.7cm}p{.7cm}}
  \multicolumn{5}{c}{Makalero\ilt{Makalero}}\\
\midrule

  i &      &      &      &  u\\
    &   e  &      &   o   &  \\ 
    &      &   a  &      &   \\
  
\end{tabular}\\


\end{tabular} 

{\scriptsize The data in these tables are from \citet{Holtonta} for Western Pantar\ilt{Western Pantar}, \citet{Klamer2010} for Teiwa\ilt{Teiwa}, \citet{Klamerta} for Kaera\ilt{Kaera}, \citet{Steinhauerta} for Blagar\ilt{Blagar}, \citet{Haan2001} for Adang\ilt{Adang}, \citet{Baird2008} for Klon\ilt{Klon}, \citet{Kratochvil2007} for Abui\ilt{Abui}, \citet{Schapperndb} for Kamang\ilt{Kamang}, \citet{Kratochvilta} for Sawila\ilt{Sawila}, \citet{SchapperEtAlta} for Wersing\ilt{Wersing}, \citet{Schapper2010} for Bunaq\ilt{Bunaq}, and \citet{Huber2011} for Makalero\ilt{Makalero}.
}
\setlength{\tabcolsep}{6pt}
\end{table} 

 

In sum, with the positive establishment of the relatedness of the Papuan languages scattered across the islands of Timor, Kisar, Alor, Pantar and the Pantar Straits, a start has been made towards a history of the TAP languages. However, we are still a long way off a complete and nuanced understanding of the family and its development (cf. Schapper and Huber's (forthcoming) statement of prospective research questions). It will be the task of future reconstructive historical work to definitively solve remaining issues in the comparative data. 


\section*{Sources}



 
\begin{tabular}{lp{7.5cm}}
Abui (\textsc{Abu}) & Kratochv\'il 2007, Kratochv\'il and Delpada 2008, Schapper fieldnotes 2010\\
Adang (\textsc{Adg}) & Robinson fieldnotes 2010\\
Blagar (\textsc{Blg}) & Robinson fieldnotes 2010\\
Bunaq (Lamaknen) & Schapper n.d. a, Schapper 2010\\
Deing (\textsc{Dei}) & Robinson fieldnotes 2010\\
Fataluku & Fataluku online dictionary\footnotemark{}, van Engelenhoven fieldnotes\\
Hamap (H\textsc{mp}) & Robinson fieldnotes 2010\\
Kamang (\textsc{Kmg}) & Schapper n.d. b, Schapper and Manimau 2011\\
Kabola (K\textsc{ab}) & Robinson fieldnotes 2010\\
Kaera (K\textsc{ae}) & Klamer Kaera corpus 2005-2007\\
Kafoa (\textsc{Kaf}) & Baird fieldnotes 2003\\
Klon (\textsc{Kln}) & Baird fieldnotes 2003\\
Kui (\textsc{Kui}) & Holton fieldnotes 2010\\
Kula (\textsc{Kul}) & Holton fieldnotes 2010, Nicholas Williams p.c. 2011\\
Makalero & Huber 2011, Huber fieldnotes 2007-2013\\
Makasae & Brotherson 2003, Carr 2004, Huber 2008, Huber fieldnotes 2005, 2012-2013, Language Documentation Training Center of the University of Hawaii\footnotemark{}\\
Nedebang (N\textsc{ed}) & Robinson fieldnotes 2010\\
Oirata & Josselin de Jong 1937, van Engelenhoven fieldnotes\\
Reta (\textsc{Ret}) & Robinson fieldnotes 2010\\
Sar (S\textsc{ar}) & Robinson fieldnotes 2010\\
Sawila (S\textsc{wl}) & Kratochv\'il n.d.\\
Teiwa (\textsc{Tew}) & Klamer Teiwa corpus, Klamer and Sir 2011, Robinson fieldnotes 2010\\
Wersing (\textsc{Wer}) & Schapper and Hendery fieldnotes 2012, Holton fieldnotes 2010\\
Western Pantar (WP) & Holton and Lamma Koly 2008, Holton fieldnotes 2010\\
\mybottomrule
\end{tabular}


\addtocounter{footnote}{-2}
\stepcounter{footnote}\footnotetext{Online at www.fataluku.com}
\stepcounter{footnote}\footnotetext{Online at http://www.ling.hawaii.edu/ldtc/languages/makasae\_fatum/ and http://www.ling.hawaii.edu/ldtc/languages/makasae\_osor/.}

The orthographic conventions used in the Appendices are the following: `\emph{\textup{\~{}' joins morphological variants of the same lexeme. In Appendix \ref{sec:3:app:1} and Appendix  \ref{sec:3:app:2}, material given in round brackets `( )' represents fossilized morphology or other unetymological material. In Appendix  \ref{sec:3:app:3}, round brackets indicate that a given phoneme cannot be reconstructed with abosolute certainty. Furthermore, `N' is used to represent an unspecified nasal; `L' an unspecified liquid, and `Q' a putative postvelar stop for which we have only very weak evidence. An empty slot in the pTAP column means that the reconstructed pAP and pTIM forms, although clearly cognate, are too different to allow for a secure pTAP reconstruction.} }


\startappendix
\subsection{Data supporting the additional pAP reconstructions}
\label{sec:3:app:1}
 



\noindent
\begin{tabular*}{\textwidth}{@{\extracolsep{\fill}}lllllll}
\mytoprule
{\bfseries gloss} & bark & bird & bite & bone & \parbox{2cm}{clew, stone circle\textsuperscript{3}} & coconut\\
\midrule
{\bfseries pAP original} & {}-{}- & *dVl & {}-{}- & {}-{}- & {}-{}- & *wat(a)\\
{\bfseries pAP new} & *lVu & *(a)dVl & *(ta)ki & *ser & *maita & *wata\\
{\scshape Sar} & -- & {\itshape dal} & -- & -- & -- & {\itshape wat}\\
{\scshape Dei} & -- & {\itshape dal} & -- & -- & -- & {\itshape wat}\\
{\scshape Tew} & -- & {\itshape dai} & -- & -- & -- & {\itshape wat}\\
{\scshape Ned} & -- & {\itshape daya} & -- & -- & -- & {\itshape wata}\\
{\scshape Kae} & -- & -- & -- & -- & -- & {\itshape wat}\\
{\scshape WP} & {\itshape lau} & -- & -- & -- & -- & {\itshape hatua}\\
{\scshape Blg} & {\itshape olovi} & -- & {\itshape (ga)ki} & -- & -- & {\itshape vet}\\
{\scshape Ret} & {\itshape lu} & -- & {\itshape ki(-ki)} & -- & -- & {\itshape vat}\\
{\scshape Adg} & {\itshape lowo{\textglotstop}} & -- & -- & -- & -- & {\itshape fa{\textglotstop}}\\
{\scshape Hmp} & -- & -- & -- & -- & -- & --\\
{\scshape Kab} & {\itshape olowo} & -- & -- & -- & -- & {\itshape wa{\textglotstop}}\\
{\scshape Kui} & -- & {\itshape adol} & -- & -- & -- & {\itshape bat}\\
{\scshape Kaf} & -- & -- & -- & -- & -- & --\\
{\scshape Kln} & -- & -- & -- & -- & -- & --\\
{\scshape Abu} & {\itshape lou} & -- & {\itshape (ta)kai} & -- & {\itshape masa{\ng} ?` \textsuperscript{4}} & {\itshape wata}\\
{\scshape Kmg} & -- & {\itshape atul} & {\itshape ka(te) \textsuperscript{1}} & {\itshape s{\textepsilon}l ?` \textsuperscript{2}} & {\itshape maita} & --\\
{\scshape Kul} & {\itshape leloja} & -- & -- & {\itshape (gi)saja} & -- & {\itshape g\textsuperscript{w}ata}\\
{\scshape Swl} & -- & {\itshape adala} & -- & {\itshape sara} & -- & {\itshape wata}\\
{\scshape Wer} & {\itshape aloi} & {\itshape adol} & {\itshape (mi)kik} & {\itshape (ge)seri} & -- & {\itshape wata}\\
\mybottomrule
\end{tabular*}


\textsuperscript{1} Metathesised form; denotes `eat'.  \textsuperscript{2} Kamang normally\textsuperscript{} reflects pAP *r as as i in final position.  \textsuperscript{3} See \citet{Rodemeier1992} on clews in Alor.  \textsuperscript{4} Abui normally reflects pAP *t as t.  \textsuperscript{5} This reconstruction must be viewed as tentative, since Kabola does not make part of the existing pAP reconstruction.  \textsuperscript{6} Note the loss of the initial syllable in several of the daughter languages. According to \citet{HoltonEtAl2012} and \citet{HoltonRobinsonTV}, this has to do with stress being based on syllable weight. The heavy *bar syllable attracts stress, which leads to the loss of the initial syllable. A similar case is, possibly, pAP *tei `tree'.





\noindent
\begin{tabular*}{\textwidth}{@{\extracolsep{\fill}}llllll}
\mytoprule
{\bfseries gloss} & crawl & die & dirty & dog & ear\\
\midrule
{\bfseries pAP original} & {}-{}- & *minV & {}-{}- & {}-{}- & *-uar(i)\\
{\bfseries pAP new} & *er & *min(a) & *karok \textsuperscript{5} & *jibar \textsuperscript{6} & *-uari\\
{\scshape Sar} & -- & {\itshape min} & -- & {\itshape jifar} & --\\
{\scshape Dei} & -- & {\itshape mi{\ng}} & -- & {\itshape jewar} & {\itshape {}-war}\\
{\scshape Tew} & -- & {\itshape min} & -- & {\itshape jifar} & {\itshape {}-uar}\\
{\scshape Ned} & -- & {\itshape min{\textlengthmark}a} & -- & {\itshape bar} & {\itshape {}-ow}\\
{\scshape Kae} & -- & {\itshape min} & -- & {\itshape ibar} & {\itshape {}-uar}\\
{\scshape WP} & -- & -- & -- & {\itshape jab{\textlengthmark}e} & {\itshape {}-ue}\\
{\scshape Blg} & -- & {\itshape (i)mina} & -- & {\itshape jabar} & {\itshape {}-veli}\\
{\scshape Ret} & -- & {\itshape (a)mina} & -- & {\itshape jobal} & --\\
{\scshape Adg} & -- & {\itshape mini{\textglotstop}} & {\itshape karo{\textglotstop}o} & {\itshape bel} & --\\
{\scshape Hmp} & -- & {\itshape min} & -- & {\itshape b{\o}l} & --\\
{\scshape Kab} & -- & {\itshape mini} & {\itshape (na)karo{\textglotstop}o} & {\itshape bel} & --\\
{\scshape Kui} & -- & {\itshape min} & -- & -- & {\itshape {}-uel}\\
{\scshape Kaf} & -- & {\itshape (i)mon} & -- & -- & --\\
{\scshape Kln} & -- & -- & -- & -- & {\itshape {}-u{\textepsilon}r}\\
{\scshape Abu} & -- & {\itshape mo{\ng}} & -- & -- & {\itshape {}-uei}\\
{\scshape Kmg} &  \textit{eei\~{}eel} & -- & -- & -- & {\itshape {}-uai}\\
{\scshape Kul} & -- & -- & -- & -- & --\\
{\scshape Swl} & -- & -- & -- & -- & {\itshape uari}\\
{\scshape Wer} & {\itshape er} & -- & -- & -- & {\itshape {}-ueri}\\
\mybottomrule
\end{tabular*}


\textsuperscript{1} This is a reciprocal term. The reflexes in the modern languages denote either `grandparent' or `grandchild'.  \textsuperscript{2} Semantic shift to `child'.  \textsuperscript{3} Denotes `grandchild'.  \textsuperscript{4} Denotes `grandparent'.  \textsuperscript{5} While clearly cognate, the forms in this set show a variety of unexpected or irregular sound changes: Teiwa, Nedebang and Kaera normally reflect pAP *l as l in initial and medial position, rather than j; Teiwa and Nedebang normally reflect pAP *g as {\pharfric} and x, respectively, in medial position, rather than g; pAP *g is normally reflected as g in Klon and j in Sawila; and finally, initial h in Western Pantar is usually a reflex of pAP *h, rather than *w. The pAP reconstruction must thus be seen as somewhat tentative.






\noindent
\begin{tabular*}{\textwidth}{@{\extracolsep{\fill}}llllll}
\mytoprule
{\bfseries gloss} & face & far & fire & fish & flat\\
\midrule
{\bfseries pAP original} & {}-{}- & {}-{}- & *had(a) & *hab(i) & {}-{}-\\
{\bfseries pAP new} & *-pona & *lete & *hada & *habi & *tatok\\
{\scshape Sar} & -- & -- & -- & -- & --\\
{\scshape Dei} & -- & -- & -- & -- & --\\
{\scshape Tew} & -- & -- & {\itshape {\pharfric}ar} & {\itshape {\pharfric}af} & --\\
{\scshape Ned} & -- & -- & {\itshape ar} & {\itshape a{\textlengthmark}fi} & --\\
{\scshape Kae} & -- & -- & {\itshape ad} & {\itshape ab} & --\\
{\scshape WP} & -- & -- & -- & {\itshape hap} & --\\
{\scshape Blg} & -- & -- & {\itshape a{\textlengthmark}d} & {\itshape a{\textlengthmark}b} & --\\
{\scshape Ret} & -- & -- & -- & -- & --\\
{\scshape Adg} & -- & -- & -- & {\itshape a{\textlengthmark}b} & --\\
{\scshape Hmp} & -- & -- & -- & -- & --\\
{\scshape Kab} & -- & -- & -- & -- & --\\
{\scshape Kui} & -- & -- & {\itshape ar} & {\itshape eb} & --\\
{\scshape Kaf} & -- & -- & -- & -- & --\\
{\scshape Kln} & -- & -- & {\itshape {\textschwa}d{\textscripta}} & {\itshape {\textschwa}bi} & --\\
{\scshape Abu} & {\itshape {}-po{\ng}} & -- & {\itshape ara} & {\itshape afu} & --\\
{\scshape Kmg} & {\itshape {}-funa{\textlengthmark}} & {\itshape letei} & {\itshape ati} & {\itshape api} & {\itshape tatok}\\
{\scshape Kul} & -- & -- & -- & -- & --\\
{\scshape Swl} & -- & -- & {\itshape ada} & {\itshape api} & --\\
{\scshape Wer} & -- & -- & {\itshape ada} & {\itshape api} & --\\
\mybottomrule
\end{tabular*}


\textsuperscript{1} \citet{HoltonEtAl2012} reconstruct *jari for `laugh'. We revise this form on the basis of the clear presence of a medial velar in the reflexes of many AP languages. Note, however, the irregular loss of reflexes of pAP *g in Western Pantar, Kui, Sawila and Wersing.  \textsuperscript{2} See \citet{SchapperTV} for details on this reconstruction.  \textsuperscript{3} The reflexes of this form denote `game' or `meat'. Note that there are several irregularities in this set: Adang normally reflects pAP *r as l, rather than r; and Sawila and Wersing normally reflect *s as t, rather than s.  \textsuperscript{4} Abui normally reflects pAP *b as f, rather than b, and pAP *b is usually reflected in Kamang as p, rather than b.  \textsuperscript{5} Denotes `coast'. The relationship between the two senses is explained by the typical settlement patterns in the region: older settlements are located in high places, often on top of knolls or ridges, whilst newer settlements are downhill towards the coast.




\noindent
\begin{tabular*}{\textwidth}{@{\extracolsep{\fill}}llllll}
\mytoprule
{\bfseries gloss} & girl & \parbox{2cm}{grandparent\\grandchild} & green & hear & itchy\\
\midrule
{\bfseries pAP original} & {}-{}- & {}-{}- & {}-{}- & {}-{}- & {}-{}-\\
{\bfseries pAP new} & *pon & *tam(a, u) \textsuperscript{1} & *(wa)logar \textsuperscript{5} & *magi & *(i)ruk\\
{\scshape Sar} & -- & -- & {\itshape logar} & -- & --\\
{\scshape Dei} & -- & -- & {\itshape alogur} & -- & --\\
{\scshape Tew} & -- & -- & {\itshape ajogar ?`} & -- & --\\
{\scshape Ned} & -- & -- & {\itshape aejaga ?`} & -- & --\\
{\scshape Kae} & -- & -- & {\itshape ojogi ?`} & -- & --\\
{\scshape WP} & -- & -- & {\itshape haluaga} & -- & --\\
{\scshape Blg} & -- & -- & -- & {\itshape m{\textepsilon}{\textglotstop}{\textepsilon}} & --\\
{\scshape Ret} & -- & -- & -- & -- & --\\
{\scshape Adg} & -- & -- & -- & {\itshape ma{\textglotstop}eh} & --\\
{\scshape Hmp} & -- & -- & -- & -- & --\\
{\scshape Kab} & -- & -- & -- & {\itshape me{\textglotstop}ehe} & --\\
{\scshape Kui} & -- & -- & -- & {\itshape magi} & {\itshape rok}\\
{\scshape Kaf} & -- & -- & -- & -- & --\\
{\scshape Kln} & -- & -- & {\itshape w{\textschwa}w{\textepsilon}l{\textepsilon}{\ng} ?`} & {\itshape m{\textschwa}gih} & --\\
{\scshape Abu} & -- & -- & {\itshape wala{\ng}aj} & {\itshape mahi} & {\itshape joku{\ng}}\\
{\scshape Kmg} & {\itshape fon} & {\itshape dum \textsuperscript{2}} & -- & {\itshape {}-mai} & {\itshape joku{\ng}}\\
{\scshape Kul} & -- & {\itshape atamu \textsuperscript{3}} & {\itshape wala{\ng}ka} & {\itshape magin} & {\itshape joka}\\
{\scshape Swl} & -- & {\itshape (ga)ta{\textlengthmark}mu \textsuperscript{3}} & {\itshape wala{\ng}ara ?`} & {\itshape maji{\textlengthmark}{\ng}} & --\\
{\scshape Wer} & -- & {\itshape (ne)tamu \textsuperscript{4}} & {\itshape walar} & -- & {\itshape iruk}\\
\mybottomrule
\end{tabular*}





\noindent
\begin{tabular*}{\textwidth}{@{\extracolsep{\fill}}lllllll}
\mytoprule
{\bfseries gloss} & laugh & leg & LOW & meat & mountain & name\\
\midrule
{\bfseries pAP original} & *jari & {}-{}- & {}-{}- & {}-{}- & {}-{}- & *-ain(i, u)\\
{\bfseries pAP new} & *jagir \textsuperscript{1} & *-bat & *po \textsuperscript{2} & *iser \textsuperscript{3} & *buku & *-en(i, u)\\
{\scshape Sar} & {\itshape jehar} & {\itshape {}-fat} & -- & -- & -- & --\\
{\scshape Dei} & {\itshape jaxar} & {\itshape {}-wat} & -- & -- & -- & --\\
{\scshape Tew} & {\itshape j{\textschwa}{\pharfric}ar} & {\itshape {}-fat} & -- & -- & -- & --\\
{\scshape Ned} & {\itshape gela} & -- & -- & -- & -- & {\itshape {}-einu}\\
{\scshape Kae} & {\itshape agar} & {\itshape at} & -- & -- & {\itshape buku{\textlengthmark}} & {\itshape {}-en}\\
{\scshape WP} & {\itshape jali ?`} & -- & -- & -- & -- & {\itshape {}-in{\textlengthmark}u}\\
{\scshape Blg} & {\itshape iriga} & -- & {\itshape po} & -- & {\itshape buku} & {\itshape {}-ene}\\
{\scshape Ret} & {\itshape agala} & -- & -- & -- & -- & --\\
{\scshape Adg} & -- & -- & {\itshape p{\textopeno}} & {\itshape hiri ?`} & -- & {\itshape {}-ani{\ng}}\\
{\scshape Hmp} & -- & -- & -- & {\itshape (ma)hil} & -- & {\itshape an{\textepsilon}}\\
{\scshape Kab} & {\itshape ja{\textlengthmark}la} & -- & -- & -- & -- & --\\
{\scshape Kui} & {\itshape jeri ?`} & -- & -- & {\itshape Is} & -- & {\itshape {}-enei}\\
{\scshape Kaf} & -- & -- & -- & {\itshape (ma)he:l} & -- & {\itshape {}-n{\textepsilon}i}\\
{\scshape Kln} & {\itshape {\textschwa}gar} & -- & -- & {\itshape (m{\textschwa})h{\textepsilon}l} & -- & {\itshape {}-{\textschwa}n{\textepsilon}{\textglotstop}}\\
{\scshape Abu} & -- & -- & {\itshape pa} & {\itshape mahiti{\ng}} & {\itshape buku ?` \textsuperscript{4}} & {\itshape {}-ane}\\
{\scshape Kmg} & -- & -- & {\itshape fe} & {\itshape isei} & {\itshape buk ?` \textsuperscript{4}} & {\itshape {}-nei}\\
{\scshape Kul} & {\itshape geja} & -- & -- & -- & -- & --\\
{\scshape Swl} & {\itshape jara ?`} & -- & -- & {\itshape isi ?`} & -- & {\itshape {}-ani}\\
{\scshape Wer} & {\itshape jer ?`} & -- & -- & {\itshape (ge)is ?`} & -- & --\\
\mybottomrule
\end{tabular*}






\begin{tabular*}{\textwidth}{@{\extracolsep{\fill}}llllll}
\mytoprule


{\bfseries gloss} & new & new place & other & path & person\\
\midrule
{\bfseries pAP original} & *siba & {}-{}- & {}-{}- & {}-{}- & {}-{}-\\
{\bfseries pAP new} & *siba(r) & *lan & *abenVC & *jega \textsuperscript{1} & *anin\\
{\scshape Sar} & -- & -- & -- & -- & --\\
{\scshape Dei} & {\itshape sib} & -- & -- & -- & --\\
{\scshape Tew} & {\itshape sib} & -- & -- & -- & --\\
{\scshape Ned} & {\itshape sava({\textglotstop}a)} & -- & -- & {\itshape ji{\textlengthmark}ja ?`} & --\\
{\scshape Kae} & {\itshape sib-} & -- & {\itshape bani{\ng}} & -- & --\\
{\scshape WP} & {\itshape sab{\textlengthmark}a} & -- & -- & {\itshape ja ?`} & --\\
{\scshape Blg} & {\itshape hiba} & -- & \emph{abeu}\textit{{\ng}}\emph{ \~{} ebeu}\textit{{\ng}} & {\itshape iga ?`} & --\\
{\scshape Ret} & {\itshape haba} & -- & -- & {\itshape viag} & --\\
{\scshape Adg} & {\itshape habar} & -- & -- & {\itshape se{\textglotstop}} & --\\
{\scshape Hmp} & {\itshape habar} & -- & -- & {\itshape se{\textglotstop}} & --\\
{\scshape Kab} & -- & -- & -- & {\itshape je{\textglotstop}} & --\\
{\scshape Kui} & {\itshape saba} & -- & {\itshape aba{\ng}an} & -- & {\itshape anin(ou)}\\
{\scshape Kaf} & {\itshape hifa} & -- & {\itshape afenaj} & {\itshape {\textglotstop}ij{\textepsilon}} & --\\
{\scshape Kln} & {\itshape h{\textschwa}b{\textscripta}{\textlengthmark}} & -- & {\itshape ebe{\ng}} & {\itshape {\textepsilon}g{\textepsilon}{\textglotstop}} & {\itshape {\textscripta}n{\textsci}n(ok)}\\
{\scshape Abu} & {\itshape t{\i}f{\textscripta}} & -- & -- & -- & --\\
{\scshape Kmg} & {\itshape supa(ka)} & {\itshape la{\ng}} & -- & -- & --\\
{\scshape Kul} & {\itshape tupa} & -- & -- & -- & {\itshape ani{\ng}(na)}\\
{\scshape Swl} & {\itshape tipea} & {\itshape la{\textlengthmark}{\ng} \textsuperscript{5}} & -- & -- & {\itshape ani{\ng}(ka{\textlengthmark})}\\
{\scshape Wer} & {\itshape t{\textschwa}pa} & {\itshape la{\ng} \textsuperscript{5}} & -- & -- & {\itshape ani{\ng}}\\
\mybottomrule
\end{tabular*}





\noindent
\begin{tabular*}{\textwidth}{@{\extracolsep{\fill}}llllll}
\mytoprule


{\bfseries gloss} & price & {\itshape P.} 

\textit{indicus} \textsuperscript{4} & rain & ripe & scratch\\
\midrule
{\bfseries pAP original} & {}-{}- & {}-{}- & {}-{}- & *ten & {}-{}-\\
{\bfseries pAP new} & *bol \textsuperscript{2} & *matar & *anur & *tena & *karab\\
{\scshape Sar} & -- & -- & -- & -- & {\itshape k{\textschwa}ra{\textlengthmark}b}\\
{\scshape Dei} & -- & -- & -- & {\itshape ten{\textlengthmark}a{\ng}} & {\itshape krab}\\
{\scshape Tew} & -- & -- & -- & -- & --\\
{\scshape Ned} & -- & -- & -- & {\itshape tanan} & {\itshape (ki)kar ?` \textsuperscript{5}}\\
{\scshape Kae} & -- & -- & -- & {\itshape ten-} & {\itshape krabis ?` \textsuperscript{6}}\\
{\scshape WP} & -- & {\itshape mat{\textlengthmark}e} & -- & {\itshape ta{\ng}} & {\itshape karasi ?` \textsuperscript{7}}\\
{\scshape Blg} & -- & -- & {\itshape onor} & {\itshape tena} & --\\
{\scshape Ret} & {\itshape (ta){\texthtb}eli \textsuperscript{3}} & -- & -- & -- & --\\
{\scshape Adg} & -- & -- & {\itshape nui} & {\itshape tene} & --\\
{\scshape Hmp} & -- & -- & -- & {\itshape t{\textepsilon}n} & --\\
{\scshape Kab} & {\itshape ({\textglotstop}o)wol \textsuperscript{3}} & -- & {\itshape nui} & {\itshape tena{\ng}} & --\\
{\scshape Kui} & -- & -- & {\itshape anor} & {\itshape tain} & {\itshape ukuberi}\\
{\scshape Kaf} & -- & -- & -- & -- & {\itshape ukafi}\\
{\scshape Kln} & -- & {\itshape mtar} & -- & {\itshape {\textschwa}t{\textepsilon}n} & {\itshape k{\textschwa}r{\textopeno}b}\\
{\scshape Abu} & {\itshape (he)bel \textsuperscript{3}} & {\itshape mitai} & {\itshape anui} & -- & {\itshape kafi}\\
{\scshape Kmg} & {\itshape bol \textsuperscript{3}} & -- & -- & {\itshape iten \~{} iton} & --\\
{\scshape Kul} & -- & -- & -- & -- & {\itshape kapi}\\
{\scshape Swl} & -- & {\itshape mata{\textlengthmark}ri} & -- & {\itshape iti{\textlengthmark}na} & {\itshape kapari}\\
{\scshape Wer} & -- & -- & -- & -- & {\itshape k{\textschwa}pir}\\
\mybottomrule
\end{tabular*}


\textsuperscript{1} There are a number of irregularities in this set: Nedebang normally reflects medial *g as x, Western Pantar as g{\textlengthmark}, and Blagar as either {\O} or {\textglotstop}.\textsuperscript{  2} This root is likely an Austronesian loan: PMP *b{\textschwa}li `price', bride price'.  \textsuperscript{3} Denotes `bride price'.  \textsuperscript{4} New Guinea rosewood (Petrocapus indicus), typically referred to in eastern Malay as \textit{kayu merah}.  \textsuperscript{5} Note the irregular loss of the final syllable.  \textsuperscript{6} Semantic shift to `claw'. Also, note the unetymological s, present in both Kaera and Western Pantar.  \textsuperscript{7} While this form is very likely related, it includes several irregularities: the expected reflex of pAP *r in medial position is l in Western Pantar; there is no reflex of pAP *b, which is normally reflected as b; and there is an unetymological s.  \textsuperscript{8} Blagar normally reflects pAP *s as h in word-initial position.  \textsuperscript{9} This set shows a variety of irregularities: Adang normally reflects pAP *r as l or I, rather than r; pAP *r is normally reflected as i in final position in both Abui and Kamang; and Wersing normally reflects pAP *s as t, rather than s.




\noindent
\begin{tabular*}{\textwidth}{@{\extracolsep{\fill}}llllll}
\mytoprule


{\bfseries gloss} & shark & spit & spoon & stand & sugarcane\\
\midrule
{\bfseries pAP original} & {}-{}- & *purVN & {}-{}- & {}-{}- & *u:b\\
{\bfseries pAP new} & *sib(a, i)r & *purVn & *surV \textsuperscript{9} & *nate(r) \textsuperscript{1} & *hu:ba\\
{\scshape Sar} & {\itshape sifir} & -- & -- & -- & --\\
{\scshape Dei} & {\itshape sib{\textlengthmark}ir} & -- & -- & -- & --\\
{\scshape Tew} & {\itshape sifar} & {\itshape puran} & -- & -- & --\\
{\scshape Ned} & -- & -- & -- & -- & {\itshape u{\textlengthmark}fa}\\
{\scshape Kae} & {\itshape sibar} & {\itshape pura{\ng}} & -- & -- & {\itshape u{\textlengthmark}b}\\
{\scshape WP} & {\itshape sib{\textlengthmark}u} & -- & -- & {\itshape natar ?` \textsuperscript{2}} & --\\
{\scshape Blg} & {\itshape sibir \textsuperscript{8}} & {\itshape puru{\ng}} & -- & -- & {\itshape ub}\\
{\scshape Ret} & {\itshape hibil} & {\itshape puru{\ng}} & -- & -- & {\itshape juwab}\\
{\scshape Adg} & -- & -- & {\itshape hur} & -- & {\itshape so{\textlengthmark}b}\\
{\scshape Hmp} & -- & -- & -- & -- & --\\
{\scshape Kab} & -- & {\itshape para{\ng}} & -- & -- & {\itshape job}\\
{\scshape Kui} & {\itshape sobor} & {\itshape puri{\ng}} & -- & -- & {\itshape u{\textlengthmark}b}\\
{\scshape Kaf} & -- & -- & -- & {\itshape natei} & --\\
{\scshape Kln} & -- & {\itshape p{\textschwa}r{\textupsilon}in} & -- & -- & --\\
{\scshape Abu} & -- & {\itshape puina} & {\itshape tur} & {\itshape nate} & {\itshape fa}\\
{\scshape Kmg} & -- & -- & {\itshape su{\textlengthmark}t} & -- & --\\
{\scshape Kul} & -- & -- & -- & -- & {\itshape p\textsuperscript{w}a}\\
{\scshape Swl} & -- & -- & -- & -- & --\\
{\scshape Wer} & -- & -- & {\itshape sire} & -- & {\itshape upa}\\
\mybottomrule
\end{tabular*}





\noindent
\begin{tabular*}{\textwidth}{@{\extracolsep{\fill}}llllll}
\mytoprule


{\bfseries gloss} & sun & taboo & tail & tongue & tooth\\
\midrule
{\bfseries pAP original} & *wad(i) & {}-{}- & *-or & *-leb(ur) & *-uas\\
{\bfseries pAP new} & *wadi & *palol & *-ora & *-lebur & *-uasin\\
{\scshape Sar} & {\itshape war} & -- & {\itshape {}-or} & -- & --\\
{\scshape Dei} & -- & -- & {\itshape {}-or} & -- & --\\
{\scshape Tew} & {\itshape war (get)} & -- & {\itshape {}-or} & {\itshape {}-livi} & {\itshape {}-usan}\\
{\scshape Ned} & {\itshape weri} & -- & {\itshape {}-ola} & {\itshape {}-lefu} & {\itshape {}-usi{\ng}}\\
{\scshape Kae} & {\itshape wer} & -- & {\itshape {}-or} & {\itshape {}-le{\textlengthmark}b} & {\itshape {}-uasi{\ng}}\\
{\scshape WP} & {\itshape war} & -- & -- & {\itshape {}-lebu} & {\itshape {}-wasi{\ng}}\\
{\scshape Blg} & {\itshape ved} & -- & {\itshape ora} & {\itshape {}-d{\textyogh}ebur} & {\itshape {}-ve{\ng}}\\
{\scshape Ret} & {\itshape vid} & -- &  & {\itshape {}-lebul} & --\\
{\scshape Adg} & {\itshape f{\textepsilon}d} & -- & {\itshape olo{\textglotstop}} & {\itshape {}-l{\textepsilon}b} & {\itshape {}-w{\textepsilon}h{\textepsilon}{\ng}}\\
{\scshape Hmp} & {\itshape f{\o}d} & -- & {\itshape ol} & -- & {\itshape {}-fi{\textglotstop}i{\ng}}\\
{\scshape Kab} & {\itshape wer} & -- & {\itshape {\textglotstop}ol} & {\itshape {}-leb} & --\\
{\scshape Kui} & {\itshape ber} & -- & {\itshape {}-or} & {\itshape {}-liber} & {\itshape {}-wes}\\
{\scshape Kaf} & {\itshape uru} & -- & -- & {\itshape {}-lip} & {\itshape {}-wehe{\ng}}\\
{\scshape Kln} & -- & -- & {\itshape {}-or} & {\itshape {}-l{\textepsilon}b} & {\itshape {}-w{\textepsilon}h}\\
{\scshape Abu} & {\itshape wari} & {\itshape palol} & -- & {\itshape {}-lifi} & {\itshape {}-weiti}\\
{\scshape Kmg} & {\itshape wati} & {\itshape fo{\textlengthmark}i} & {\itshape {}-(w)ui} & {\itshape {}-opei} & {\itshape {}-weh}\\
{\scshape Kul} & {\itshape wad} & -- & -- & {\itshape il{\i}p} & --\\
{\scshape Swl} & {\itshape wadi} & -- & {\itshape {}-(w)o{\textlengthmark}ra} & -- & {\itshape {}-wa}\\
{\scshape Wer} & {\itshape widi} & -- & {\itshape wori} & {\itshape {}-jebur} & {\itshape {}-wesi}\\
\mybottomrule
\end{tabular*}


\textsuperscript{1} There is a competing and morphologically unrelated form *tas `stand', which is more widely distributed across modern AP languages (see \citealt{HoltonRobinsonTV}).  \textsuperscript{2} Western Pantar normally reflects pAP *r as {\O} in word-final position.  \textsuperscript{3} This root is possibly an Austronesian loan: PMP *lakaj `stride, take a step'.  \textsuperscript{4} Kamang normally reflects pAP *k as k.  \textsuperscript{5} Semantic shift to `follow'.  \textsuperscript{6} Kaera normally reflects pAP *l as l in word-initial position.




\noindent
\begin{tabular*}{\textwidth}{@{\extracolsep{\fill}}lllll}
\mytoprule
{\bfseries gloss} & walk 1 & walk 2 & weave & yellow\\
\midrule
{\bfseries pAP original} & {}-{}- & {}-{}- & {}-{}- & {}-{}-\\
{\bfseries pAP new} & *laka \textsuperscript{3} & *lam(ar) & *sine(N) & *bagori\\
{\scshape Sar} & -- & -- & -- & {\itshape bahar}\\
{\scshape Dei} & -- & -- & -- & {\itshape bug}\\
{\scshape Tew} & -- & {\itshape lam \textsuperscript{5}} & -- & {\itshape ba{\pharfric}ari}\\
{\scshape Ned} & -- & -- & -- & {\itshape baxori}\\
{\scshape Kae} & -- & {\itshape amar ?` \textsuperscript{6}} & -- & {\itshape bagari}\\
{\scshape WP} & -- & {\itshape lama} & {\itshape sin{\textlengthmark}a{\ng}} & {\itshape bug{\textlengthmark}a}\\
{\scshape Blg} & -- & {\itshape lamar} & -- & {\itshape bagori ?` \textsuperscript{1}}\\
{\scshape Ret} & -- & {\itshape lamal} & -- & {\itshape bagori}\\
{\scshape Adg} & -- & {\itshape lami} & -- & {\itshape ba{\textglotstop}oi}\\
{\scshape Hmp} & -- & {\itshape lam{\textepsilon}} & -- & {\itshape ba{\textglotstop}oil}\\
{\scshape Kab} & {\itshape la{\textglotstop}aw} & -- & -- & {\itshape ba{\textglotstop}oil}\\
{\scshape Kui} & {\itshape lak} & -- & -- & {\itshape bagura}\\
{\scshape Kaf} & {\itshape la{\textlengthmark}ka} & -- & -- & {\itshape fij{\textupsilon}i}\\
{\scshape Kln} & -- & {\itshape (g{\textepsilon}pun)lam} & {\itshape hnan} & {\itshape b{\textupsilon}b{\textupsilon}g{\textopeno}r}\\
{\scshape Abu} & {\itshape la{\textlengthmark}k} & -- & {\itshape tinei} & --\\
{\scshape Kmg} & {\itshape lo{\textlengthmark} ?` \textsuperscript{4}} & -- & {\itshape sine} & --\\
{\scshape Kul} & -- & -- & -- & --\\
{\scshape Swl} & -- & -- & -- & --\\
{\scshape Wer} & -- & -- & -- & --\\
\mybottomrule
\end{tabular*}


\textsuperscript{1} Blagar normally reflects pAP *g as {\O} or {\textglotstop} in medial position.

\newpage
\subsection{Data supporting the additional pTIM reconstructions}
\label{sec:3:app:2}
 
\begin{tabular*}{\textwidth}{@{\extracolsep{\fill}}llllll}
\mytoprule


{\bfseries gloss} & banana & bark & bat & bite & bone\\
\midrule
{\bfseries pTIM original} & *muku & {}-{}- & {}-{}- & *gakel & {}-{}-\\
{\bfseries pTIM new} & *mugu & *le(k)u(l) & *maTa & *(ga)gel & *(se)sa(r, R)\\
{\scshape Bunaq} & {\itshape mok} & -- & -- & {\itshape gagil} & {\itshape sesal}\\
{\scshape Makasae} & {\itshape mu{\textglotstop}u} & {\itshape leu \textsuperscript{1}} & -- & {\itshape ga{\textglotstop}el} & --\\
{\scshape Makalero} & {\itshape mu{\textglotstop}u} & {\itshape leu \textsuperscript{1}} & -- & {\itshape ka{\textglotstop}el} & --\\
{\scshape Fataluku} & {\itshape mu{\textglotstop}u} & {\itshape le{\textglotstop}ul(e) \textsuperscript{2}} & {\itshape maca} & {\itshape (ki)ki{\textglotstop}(e)} & --\\
{\scshape Oirata} & {\itshape mu{\textlengthmark}} & {\itshape leule \textsuperscript{2}} & {\itshape ma{\textrtailt}a} & -- & --\\
\mybottomrule
\end{tabular*}
\\
\begin{tabular*}{\textwidth}{@{\extracolsep{\fill}}llllll}
\mytoprule
{\bfseries gloss} & breast & child & crawl & crouch & dirty\\
\midrule
{\bfseries pTIM original} & {}-{}- & {}-{}- & *er(ek) & {}-{}- & {}-{}-\\
{\bfseries pTIM new} & *hami & *-al & *er & *luk & *gari\\
{\scshape Bunaq} & -- & {\itshape {}-ol} & {\itshape el} & {\itshape lu{\textglotstop}(-lu{\textglotstop}) \textsuperscript{4}} & {\itshape gar}\\
{\scshape Makasae} & {\itshape ami} & -- & -- & -- & {\itshape ra{\textglotstop}i \textsuperscript{5}}\\
{\scshape Makalero} & -- & -- & -- & -- & {\itshape ra{\textglotstop}i \textsuperscript{5}}\\
{\scshape Fataluku} & {\itshape ami(-tapunu) \textsuperscript{3}} & -- & {\itshape er(eke)} & -- & {\itshape ra{\textglotstop}e(ne) \textsuperscript{5, 6}}\\
{\scshape Oirata} & -- & -- & -- & -- & --\\
\mybottomrule
\end{tabular*}
\textsuperscript{1} Semantic shift to `call'.  \textsuperscript{2} Semantic shift to `sing'.  \textsuperscript{3} This lexeme is a lexical doublet, i.e. originally a compound or a lexicalized parallel expression (see \citet[224]{SchapperEtAl2012}).  \textsuperscript{4} Semantic shift to `bent over (as with age)'.  \textsuperscript{5} This form shows metathesis in Proto-Eastern Timor: *kari {\textgreater} *raki {\textgreater} ra{\textglotstop}i / ra{\textglotstop}e(ne).  \textsuperscript{6} Semantic shift to `littered with stones'.

\newpage\noindent
\begin{tabular*}{\textwidth}{@{\extracolsep{\fill}}llllll}
\mytoprule
{\bfseries gloss} & dream & eat & excrement & face & far\\
\midrule
{\bfseries pTIM original} & {}-{}- & {}-{}- & {}-{}- & *fenu & {}-{}-\\
{\bfseries pTIM new} & *ufar(ana) & *nua & *a(t, D)u & *-fanu & *eTar\\
{\bfseries Bunaq} & {\itshape waen \textsuperscript{1}} & {\itshape a \~{} -ia} & {\itshape ozo} & {\itshape {}-ewen} & {\itshape ate}\\
{\bfseries Makasae} & {\itshape ufarena} & {\itshape nawa} & {\itshape atu[-gu{\textglotstop}u]\textsuperscript{2}} & {\itshape fanu} & --\\
{\bfseries Makalero} & {\itshape ofarana} & {\itshape nua} & {\itshape atu} & {\itshape fanu} & --\\
{\bfseries Fataluku} & {\itshape ufarana} & {\itshape una \~{} na$\beta $a} & {\itshape atu \textsuperscript{3}} & {\itshape fanu} & {\itshape icar}\\
{\bfseries Oirata} & upar(a) & {\itshape una \~{} nawa} & {\itshape atu \textsuperscript{3}} & {\itshape panu} & --\\
\mybottomrule
\end{tabular*}
\\
\begin{tabular*}{\textwidth}{@{\extracolsep{\fill}}llllll}
\mytoprule
{\bfseries gloss} & fish & flat & garden & hear & itchy\\
\midrule
{\bfseries pTIM original} & *api & {}-{}- & *(u)mar & *make(n) & {}-{}-\\
{\bfseries pTIM new} & *hapi & *tetok & *(u, a)mar & *mage(n) & *ilag\\
{\bfseries Bunaq} & -- & {\itshape toi{\textglotstop} \textsuperscript{4}} & {\itshape mar} & {\itshape mak} & --\\
{\bfseries Makasae} & {\itshape afi} & -- & {\itshape ama} & {\itshape ma{\textglotstop}en} & {\itshape ila{\textglotstop}}\\
{\bfseries Makalero} & {\itshape afi} & {\itshape tetu{\textglotstop}} & {\itshape ama} & {\itshape ma{\textglotstop}en} & {\itshape ile{\textglotstop}}\\
{\bfseries Fataluku} & {\itshape api} & -- & -- & -- & --\\
{\bfseries Oirata} & {\itshape ahi} & -- & {\itshape uma} & -- & --\\
\mybottomrule
\end{tabular*}

1 This item shows metathesis: waen {\textless} *awen following on fusion from the two halves of the reconstructed doublet.  2 The Bunaq cognate for the second half of this lexical doublet is \textit{g-io} `3AN-faeces', but it doesn't appear in a doublet with \textit{ozo} `faeces'.  3 Semantic shift to `belly'.  4 The final glottal stop in Bunaq is likely a reflex of final *k in pTIM. However, more evidence is needed to substantiate this claim.


\newpage\noindent
\begin{tabular*}{\textwidth}{@{\extracolsep{\fill}}llllll}
\mytoprule
{\bfseries gloss} & laugh & leg & LOW & mat & mountain\\
\midrule
{\bfseries pTIM original} & *hika & {}-{}- & {}-{}- & {}-{}- & {}-{}-\\
{\bfseries pTIM new} & *jiger & *-buta & *ufe & *biti & *bugu\\
{\bfseries Bunaq} & {\itshape higal} & {\itshape {}-but \textsuperscript{1}} & -- & -- & --\\
{\bfseries Makasae} & {\itshape hi{\textglotstop}a} & -- & {\itshape he- ?` \textsuperscript{2}} & -- & {\itshape bu{\textglotstop}u}\\
{\bfseries Makalero} & {\itshape hi{\textglotstop}e} & -- & {\itshape ufe-} & {\itshape piti} & {\itshape pu{\textglotstop}u \textsuperscript{3}}\\
{\bfseries Fataluku} & {\itshape he{\textglotstop}e} & -- & {\itshape ua- ?` \textsuperscript{2}} & {\itshape pet(u)} & --\\
{\bfseries Oirata} & -- & -- & {\itshape ua ?` \textsuperscript{2}} & {\itshape het(e)} & --\\
\mybottomrule
\end{tabular*}
\\
\begin{tabular*}{\textwidth}{@{\extracolsep{\fill}}llllll}
\mytoprule
{\bfseries gloss} & new & new place & nose & one & other\\
\midrule
{\bfseries pTIM original} & *(t, s)ifa & {}-{}- & {}-{}- & {}-{}- & {}-{}-\\
{\bfseries pTIM new} & *(t, s)ipa(r) & *lan & *-muni & *uneki & *epi\\
{\bfseries Bunaq} & {\itshape tip} & {\itshape lon} & {\itshape {}-inup ?` \textsuperscript{4}} & {\itshape uen \~{} en} & {\itshape ewi \textsuperscript{6}}\\
{\bfseries Makasae} & {\itshape sufa} & -- & {\itshape muni(kai) \textsuperscript{5}} & {\itshape u} & --\\
{\bfseries Makalero} & {\itshape hofar} & -- & {\itshape mini} & {\itshape u \~{} un} & --\\
{\bfseries Fataluku} & -- & -- & {\itshape mini(ku)} & {\itshape ukani} & --\\
{\bfseries Oirata} & -- & -- & -- & {\itshape a{\textglotstop}uni} & --\\
\mybottomrule
\end{tabular*}

\textsuperscript{1} Semantic shift to mean `knee'.  \textsuperscript{2} The reflex of pTIM *f as /h/ in Makasae and {\O} in Fataluku and Oirata is irregular; /f/ is expected for Makasae and Fataluku, and /p/ for Oirata.  \textsuperscript{3} Semantic shift to `gable, top of house'.  \textsuperscript{4} This item appears to show metathesis in the following stages: pTIM *-muni {\textgreater} *-minu {\textgreater} *-imun {\textgreater} *-inum {\textgreater} Bunaq \textit{{}-inup} `nose'. The change of *m to Bunaq p is explainable as the result of m being prohibited from codas in Bunaq.  \textsuperscript{5} The suffix --\textit{kai} is frequently found in body part terms in Makasae.  \textsuperscript{6} It seems likely that medial *p changes to /w/ in Bunaq. However, we currently lack sufficient data to support this conclusion. There has also been a semantic shift to `foreigner'.


\newpage\noindent
\begin{tabular*}{\textwidth}{@{\extracolsep{\fill}}llllll}
\mytoprule
{\bfseries gloss} & path & person & rain & ripe & scorpion\\
\midrule
{\bfseries pTIM original} & *hika & {}-{}- & {}-{}- & *tina(k) & {}-{}-\\
{\bfseries pTIM new} & *jiga & *anu & *ine(r, R) & *tena & *fe(r, R)e\\
{\bfseries Bunaq} & {\itshape hik} & {\itshape en} & {\itshape inel} & {\itshape ten \textsuperscript{1}} & {\itshape wele \textsuperscript{4}}\\
{\bfseries Makasae} & {\itshape hi{\textglotstop}a} & {\itshape anu} & -- & {\itshape tina \textsuperscript{2}} & --\\
{\bfseries Makalero} & {\itshape hi{\textglotstop}a} & {\itshape anu} & -- & {\itshape tina \~{}} 

{\itshape dina \textsuperscript{2}} & --\\
{\bfseries Fataluku} & {\itshape i{\textglotstop}a} & -- & -- & {\itshape tina \textsuperscript{3}} & --\\
{\bfseries Oirata} & {\itshape ia(ra)} & -- & -- & -- & --\\
\mybottomrule
\end{tabular*}
\\
\begin{tabular*}{\textwidth}{@{\extracolsep{\fill}}llllll}
\mytoprule
{\bfseries gloss} & scratch & shark & six & spit & tooth\\
\midrule
{\bfseries pTIM original} & {}-{}- & {}-{}- & {}-{}- & {}-{}- & *wasi\\
{\bfseries pTIM new} & *gabar & *supor & *tamal & *fulu(k, n) & *-wasin\\
{\bfseries Bunaq} & -- & -- & {\itshape tomol} & {\itshape puluk} & {\itshape (-e)we}\\
{\bfseries Makasae} & -- & -- & -- & -- & {\itshape wasi}\\
{\bfseries Makalero} & {\itshape kapar} & {\itshape su(-amulafu) \textsuperscript{5}} & -- & {\itshape fulun} & {\itshape wasi}\\
{\bfseries Fataluku} & {\itshape kafur(e)} & {\itshape hopor(u) \textsuperscript{6}} & -- & {\itshape fulu} & {\ss}ahin(u)\\
{\bfseries Oirata} & -- & -- & -- & -- & wain(i)\\
\mybottomrule
\end{tabular*}

\textsuperscript{1} Semantic shift to `be cooked, ready'.  \textsuperscript{2} Semantic shift to `cook'.  \textsuperscript{3} Semantic shift to `set alight'.  \textsuperscript{4} It seems likely that initially before front vowels *f changes to /w/ in Bunaq. However, we currently lack sufficient data to support this conclusion.  \textsuperscript{5} The meaning of the compound \textit{su-amulafu} is not quite clear. It seems to refer to a large sea creature, possibly a dolphin or a dugong. The second element, \textit{amulafu}, translates as `human being, person'.  \textsuperscript{6} This form is glossed in various ways in the different Fataluku sources either as `shark' or `dugong'


\newpage\noindent
\begin{tabular*}{\textwidth}{@{\extracolsep{\fill}}lllllll}
\mytoprule 
{\bfseries gloss} & tree & walk 1 & walk 2 & yellow & 1SG & 1PI\\
\midrule
{\bfseries pTIM original} & *hote & *lakor & {}-{}- & {}-{}- & {}-{}- & {}-{}-\\
{\bfseries pTIM new} & *hate & *lagar \textsuperscript{1} & *male & *gabar & *n- & *fi\\
{\bfseries Bunaq} & {\itshape hotel} & {\itshape lagor} & {\itshape mele} & -- & {\itshape n-} & --\\
{\bfseries Makasae} & {\itshape ate} & {\itshape la{\textglotstop}a} & -- & {\itshape gabar} & -- & {\itshape fi}\\
{\bfseries Makalero} & {\itshape ate} & {\itshape la{\textglotstop}a} & -- & -- & -- & {\itshape fi}\\
{\bfseries Fataluku} & {\itshape ete} & {\itshape la{\textglotstop}a} & -- & -- & -- & {\itshape afi}\\
{\bfseries Oirata} & {\itshape ete} & {\itshape lare} & -- & -- & -- & {\itshape ap-}\\
\mybottomrule
\end{tabular*}

\textsuperscript{1} This root is possibly an Austronesian loan: PMP *lakaj `stride, take a step'.


\subsection{List of cognates and pTAP\ilt{proto-Timor-Alor-Pantar} reconstruction}
\label{sec:3:app:2}

 \begin{center}
\begin{tabular*}{.7\textwidth}{@{\extracolsep{\fill}}llll}
\mytopline 
{gloss}&\sc {pTAP\ilt{proto-Timor-Alor-Pantar}}&\sc {pAP\ilt{proto-Alor-Pantar}}&\sc {pTIM\ilt{proto-Timor}}\\
\midrule 
bamboo&\rm *mari&\rm *mari&\rm *mari\\
banana&\rm *mugul&\rm *mogol&\rm *mugu\\
bark, call&&\rm *lVu&\rm *le(k)u(l)\\
bat&\rm *madel&\rm *madel&\rm *maTa\\
bathe&\rm *weLi&\rm *weli&\rm *weru\\
bird&\rm *(h)adul&\rm *(a)dVl&\rm *haDa\\
bite&\rm *ki(l)&\rm *(ta)ki&\rm *(ga)gel\\
blood&\rm *waj&\rm *wai&\rm *waj\\
bone&\rm *se(r, R)&\rm *ser&\rm *(se)sa(r, R)\\
breast&\rm *hami&\rm *hami&\rm *hami\\
grandparent&\rm *(t, d)ama&\rm *tam(a, u)&\rm *moTo\\
child&\rm *-uaQal&\rm *-uaqal&\rm *-al\\
clew&\rm *ma(i)ta(r)&\rm *maita&\rm *matar\\
coconut&\rm *wata&\rm *wata&\rm *wa(t, D)a\\
crawl&\rm *er&\rm *er&\rm *er\\
crouch&\rm *luk(V)&\rm *luk(V)&\rm *luk\\
die&\rm *mV(n)&\rm *min(a)&\rm *-umV\\
dirty&\rm *karV(k)&\rm *karok&\rm *gari\\
dog&\rm *dibar&\rm *jibar&\rm *Depar\\
dream&\rm *(h)ipar&\rm *hipar&\rm *ufar(ana)\\
ear&\rm *-waRi&\rm *-uari&\rm *-wali\\
eat&\rm *nVa&\rm *nai&\rm *nua\\
excrement&\rm *(h)at(V)&\rm *has&\rm *a(t, D)u\\
face&\rm *panu&\rm *-pona&\rm *-fanu\\
far&\rm *le(t, d)e&\rm *lete&\rm *eTar\\
fire&\rm *hada&\rm *hada&\rm *haTa\\
fish&\rm *habi&\rm *habi&\rm *hapi\\
flat&\rm *tatok&\rm *tatok&\rm *tetok\\
garden&\rm *magad&\rm *magad(a)&\rm *(u, a)mar\\
girl&\rm *pan(a)&\rm *pon&\rm *fana\\
give&\rm *-(e, i)na&\rm *-ena&\rm *-inV\\
green&\rm *lugar&\rm *(wa)logar&\rm *ugar\\
\mybottomline
\end{tabular*} 
 

\begin{tabular*}{.7\textwidth}{@{\extracolsep{\fill}}llll}
\mytopline
{gloss}&\sc {pTAP\ilt{proto-Timor-Alor-Pantar}}&\sc {pAP\ilt{proto-Alor-Pantar}}&\sc {pTIM\ilt{proto-Timor}}\\
\midrule 
hand&\rm *-tan(a)&\rm *-tan&\rm *-tana\\
hear&\rm *ma(g, k)e(n)&\rm *magi&\rm *mage(n)\\
inside&\rm *mi&\rm *mi&\rm *mi\\
itchy&\rm *iRak&\rm *(i)ruk&\rm *ilag\\
laugh&\rm *jagir&\rm *jagir&\rm *jiger\\
leg&\rm *buta&\rm *-bat&\rm *-buta\\
\textsc{low}&\rm *po&\rm *po&\rm *ufe\\
mat&\rm *bi(s, t)&\rm *bis&\rm *biti\\
meat&\rm *isor&\rm *iser&\rm *seor\\
moon&\rm *hur(u)&\rm *wur&\rm *huru\\
mountain&\rm *buku&\rm *buku&\rm *bugu\\
name&&\rm *-en(i, u)&\rm *-nej\\
new&\rm *(t, s)iba(r)&\rm *siba(r)&\rm *(t, s)ipa(r)\\
new place&\rm *lan&\rm *lan&\rm *lan\\
nose&\rm *-mVN&\rm *-mim&\rm *-muni\\
one&\rm *nukV&\rm *nuk&\rm *uneki\\
other&\rm *abe(nVC)&\rm *aben(VC)&\rm *epi\\
{\itshape P. indicus}&\rm *matar&\rm *matar&\rm *ma(t, D)ar\\
path&\rm *jega&\rm *jega&\rm *jiga\\
person&\rm *anV(N)&\rm *anin&\rm *anu\\
pig&\rm *baj&\rm *baj&\rm *baj\\
pound&\rm *tapa(i)&\rm *tapai&\rm *tafa\\
price&\rm *boL&\rm *bol&\rm *bura\\
rain&\rm *anu(r, R)&\rm *anur&\rm *ine(r, R)\\
rat&\rm *dur(a)&\rm *dur&\rm *Dura\\
ripe&\rm *tena&\rm *tena&\rm *tena\\
run&\rm *tipar&\rm *tiara&\rm *tifar\\
\mybottomline
\end{tabular*}
 

\begin{tabular*}{.7\textwidth}{@{\extracolsep{\fill}}llll}
\mytopline
{gloss}&\sc {pTAP\ilt{proto-Timor-Alor-Pantar}}&\sc {pAP\ilt{proto-Alor-Pantar}}&\sc {pTIM\ilt{proto-Timor}}\\
\midrule 
scorpion&\rm *pV(r, R)&\rm *pVr&\rm *fe(r, R)e\\
scratch&\rm *karab&\rm *karab&\rm *gabar\\
sea&\rm *tam(a)&\rm *tam&\rm *mata\\
shark&\rm *sibar&\rm *sib(a ,i)r&\rm *supor\\
sit&\rm *mit&\rm *mis&\rm *mit\\
six&\rm *talam&\rm *talam&\rm *tamal\\
sleep&\rm *tia(r)&\rm *tia&\rm *tia(r)\\
spit&\rm *puRV(n)&\rm *purVn&\rm *fulu(k, n)\\
spoon&\rm *suRa&\rm *surV&\rm *sula\\
stand&\rm *nat(er)&\rm *nate(r)&\rm *nat\\
star&\rm *jibV&\rm *jibV&\rm *ipi(-bere)\\
stone&\rm *war&\rm *war&\rm *war\\
sugarcane&\rm *ub(a)&\rm *hu{\textlengthmark}ba&\rm *upa\\
sun&\rm *wad(i, u)&\rm *wadi&\rm *waTu\\
taboo&\rm *palu(l, n)&\rm *palol&\rm *falu(n)\\
tail&\rm *-oRa&\rm *-ora&\rm *-ula({\textglotstop})\\
tongue&\rm *-lebuR&\rm *-lebur&\rm *-ipul\\
tooth&\rm *-wasin&\rm *-uasin&\rm *-wasin\\
tree&\rm *hate&\rm *tei&\rm *hate\\
vagina&\rm *-ar(u)&\rm *-ar&\rm *-aru\\
wake&\rm *tan(i)&\rm *-ten&\rm *Tani\\
walk 1&\rm *lak(Vr)&\rm *laka&\rm *lagar\\
walk 2&\rm *lamV&\rm *lam(ar)&\rm *male\\
water&\rm *jira&\rm *jira&\rm *ira\\
weave&\rm *sine(N)&\rm *sine(N)&\rm *sina\\
yellow&\rm *bagur(V)&\rm *bagori&\rm *gabar\\
1PI&\rm *pi&\rm *pi-&\rm *fi\\
1SG&\rm *na-&\rm *na-&\rm *n-\\
3\textsc{aln}&\rm *gie&\rm *ge&\rm *gie\\
3\textsc{inal}&\rm *g(a, i)-&\rm *ga-&\rm *g-\\
\mybottomline
\end{tabular*}
 
 \end{center}
 
\section*{Abbreviations}
\begin{tabular}{>{\sc}ll}
1 & 1st person\\
2 & 2nd person\\
3 & 3rd person\\
alien & alienable\\
AP & Alor-Pantar\\
foc & focus\\
inal & inalienable\\
low & refers to any location down(ward of) the deictic centre\\
pAP & proto-Alor-Pantar\\
pl & plural\\
poss & possessive\\
pTAP & proto-Timor Alor Pantar\\
pTIM & proto-Timor\\
pTK & proto-Timor Kisar\\
sg & singular\\
TAP & Timor Alor Pantar\\
TK & Timor Kisar\\
WP & Western Pantar\\
\end{tabular}

