%7
%bib
%tables
%examples

 
\section{Introduction}  

Elevation\is{elevation} in a spatial deictic\is{deixis} system is where a referent's location or trajectory is identified as being at a certain elevation\is{elevation} relative to the deictic\is{deixis} centre (abbreviated as `\textsc{dc}'). Elevation\is{elevation} is a common component of systems of spatial reference in several language areas: it is pervasive in Tibeto-Burman\il{Tibeto-Burman language(s)} \citep{Bickel2001,Cheung2007,Post2011} and New Guinea\il{New Guinea language(s)} \citep{Senft1997,Senft2004,Diessel1999,Levinson1983} areas, and less common but recurrent in pockets of the Americas (e.g., Uto-Aztecan languages\il{Uto-Aztecan language(s)} such as Guarj\'io\il{Guarj\'io}, \citealt{Miller1996}), Australia (e.g., Dyirbal\il{Dyirbal}, \citealt[48]{Dixon1972}) and the Caucasus (e.g., East Caucasian languages\il{East Caucasian language(s)}, \citealt{Schulze2003}). In the typological and descriptive literature, many terms have been used to describe elevation\is{elevation} components in spatial deictic\is{deixis} systems, including: ``environmental space deixis\is{deixis}'' \citep{Bickel2001}, ``altitudinal case markers'' \citep{Ebert2003}, ``height'' \citep{Dixon2003}, ``vertical case'' \citep{Noonan2006}, ``spatial coordinate systems'' \citep{Burenhult2008}, and ``topographical deixis\is{deixis}'' \citep{Post2011}. 

In this chapter I further the typological study of spatial deictic\is{deixis} systems with an elevation\is{elevation} component by surveying the elevation\is{elevation} expressing terms in Alor-Pantar (AP) languages. Every AP language possesses elevation\is{elevation} expressing terms in at least two domains: (i) set of motion\is{motion} verbs (labelled here ``elevational\is{elevation} motion\is{motion} verbs'') expressing that a trajectory is at a certain elevation\is{elevation} relative to the deictic\is{deixis} centre (go up, come down, go across, etc.), and (ii) set of non-verbal items (generically referred to here as ``elevationals\is{elevation}'') expressing that a location is at a certain elevation\is{elevation} relative to the deictic\is{deixis} centre. This synchronic part of this chapter focuses on the use and function of the second of these sets and any additional elevational\is{elevation} sets a language might have. These items show much morphosyntactic variation, in contrast to elevational\is{elevation} verbs which have near-identical distributions across the AP languages.\footnote{Note that I do not deal with how elevation\is{elevation} terms are influenced by pragmatic and other contextual factors or by ultimate orientation effects (see \citealt{Schapper2012} for discussion of some of these effects in two Timor-Alor-Pantar languages).} I further consider the history and reconstructability of an elevational\is{elevation} system to proto-Alor-Pantar\il{proto-Alor-Pantar}, observing that the elevation\is{elevation} distinction itself is very stable in the deictic\is{deixis} systems of the AP languages, but that the terms of the systems are not always stable and that the systems are often subject to elaboration. 

The chapter is structured as follows. In {\S} \ref{sec:7:2}, I set out the terminology and conventions that I will use in describing the elevational\is{elevation} systems. In {\S} \ref{sec:7:3}, I describe the elevational\is{elevation} systems of seven AP languages. For each language I discuss the number of elevation terms in the system, both within and across paradigms which contain elevation\is{elevation} marked terms. I highlight the variation that exists in the elaboration of the systems as well as in the morpho-syntactic behaviour of the items in the individual systems. In {\S} \ref{sec:7:4}, I turn to the history of AP elevational\is{elevation} systems. Using data from eleven AP languages, I reconstruct the proto-AP\il{proto-Alor-Pantar} elevational\is{elevation} system and look at how different languages have expanded and complicated the inherited system. {\S} \ref{sec:7:5} concludes the discussion and considers briefly the potential typological significance of AP elevational\is{elevation} systems.

All data is cited in a unified transcription in order to avoid confusion due to different orthographic practices for different languages. The sources for the data cited are given throughout the text of the chapter, but are also summarised in the `Sources' section before the References. 

\section{Terminological preliminaries}\label{sec:7:2}
The various labels that we saw in the previous section are indicative of the lack of standardised terminology to describe deictic\is{deixis} systems with an elevational\is{elevation} component. In this section, I define the terminology for the different categories we encounter to be used throughout this chapter. 

Of primary importance are the labels given to elevational\is{elevation} heights. I distinguish three heights of elevation\is{elevation} in basic glosses, as set in \REF{ex:7:1}. I avoid terms such as ``below'', ``above'', etc. as used by other authors, since these are typically relational terms whose locative reference does not hinge on a speech participant (speaker and/or addressee). For instance, in the sentence \textit{The cat is below the chair}, the position of speech participants does not have any impact of locative relation between the cat and the chair. 




\ea \label{ex:7:1}
\upshape
`\textsc{high}':   refers to any location situated \textit{up(ward of)} the deictic\ist{deixis} centre;\\
`\textsc{low}':   refers to any location situated \textit{down(ward of)} the deictic\ist{deixis} centre;\\
`\textsc{level}':  refers to any location situated \textit{level with} the deictic\ist{deixis} centre.\\ 
\z


There are very different ways in which an entity can be `\textsc{high}', `\textsc{low}' or `\textsc{level}' relative to the deictic\ist{deixis} centre. The most sophisticated typology of this is set out in \citet{Burenhult2008}. He identifies three kinds of systems \citep[110-111]{Burenhult2008} (see Table~\ref{tab:7:elevsystems}).



\begin{table}
\begin{tabular}{p{2cm}p{8cm}}
\mytoprule
Global \newline elevation\ist{elevation} & projects general search domains above or below the level of the deictic\ist{deixis} centre, with an axis from the deictic\ist{deixis} centre to the referent can but need not be strictly vertical (e.g., there anywhere above, below, etc.)\\
\\
Verticality & projects very narrow search domains along a truly vertical axis running at a right angle through the deictic\ist{deixis} centre, invoking a sense of exactly above/overhead or below/underneath (e.g., there straight up, there directly below, etc.)\\
\\
Geophysical elevation\ist{elevation} & projects search domains which restrict themselves to elevation\ist{elevation} as manifested in features of the geophysical environment and are not used to refer to the vertical dimension in general (e.g., there uphill, there downstream, etc.)\\
\mybottomrule
\end{tabular}
\caption{Types of spatial coordinate system \citep[110-111]{Burenhult2008}.}
\label{tab:7:elevsystems}
\end{table}

The AP languages have, for the most part, systems of global elevation\is{elevation}. There are languages in which geophysics plays a role in mapping the elevation\is{elevation} system onto the landscape, but this does not limit the systems from referring to locations as, for instance, only uphill or downhill. An example of this comes from Wersing\il{Wersing}: in this, elevational\is{elevation} motion\is{motion} verbs -\textit{a} `go.\textsc{low'} and -\textit{mid} `go.\textsc{high'} are often translated by speakers as `go towards the sea' and `go towards the mountains'. However, it does not follow that this is a geophysical system, since when we move speakers to a non-coastal environment, the verbs can still be applied despite the absence of the sea-landwards dichotomy. In addition, AP languages may also incorporate elements of {other elevational\is{elevation} types into otherwise globally elevated\is{elevation} systems. In {\S} \ref{sec:7:3.5}, we will see that, whilst Adang\il{Adang} marks only global elevation\is{elevation} in its elevationals\is{elevation}, demonstratives\is{demonstrative} and elevational\is{elevation} motion\is{motion} verbs, it also has a special set of directional\is{direction} elevationals\is{elevation} containing dedicated geophysical elevation\is{elevation} terms as well as extra elevation\is{elevation} terms in the} \textsc{high} {domain marked for different degrees of verticality. Two languages, Western Pantar\il{Western Pantar} and Kamang\il{Kamang}, also incorporate the steepness of the slope into their elevational\is{elevation} systems, which in essence is also a means of distinguishing greater or lesser degrees of verticality in elevational\is{elevation} deixis\is{deixis}.}

In several AP languages which I will discuss, elevation-marked\is{elevation} terms occur in paradigms with terms that are not marked for elevation\is{elevation}. To refer to any term in a paradigm with elevation-marked\is{elevation} terms which is not marked for elevation\is{elevation} as `\textsc{unelevated}' and for those that are elevation-marked\is{elevation}, I use the label `\textsc{elevated'.} Note that I avoid describing \textsc{elevated} terms as ``distal'' as compared to the \textsc{unelevated} terms with which they occur in paradigms. E\textsc{levated} terms, in many cases, seem to form a separate system that contrasts with their \textsc{unelevated} counterparts in terms of speech participant-anchoring. This means that, whereas \textsc{unelevated} terms take one of their speech participants (speaker or addressee) as the deictic\is{deixis} centre, \textsc{elevated} terms refer to locations relative to the speech situation as a whole. However, on account of their only vague locational reference, they are not typically used in relation to items that are very close to a speaker. Labels such as ``distal'' (\textsc{dist}) and ``proximal'' (\textsc{prox}) as well as ``addressee-anchored'' (\textsc{addr)} and ``speaker-anchored'' (\textsc{spkr)} will be used only in reference to \textsc{unelevated} terms.\footnote{This glossing of demonstratives\ist{demonstrative} is taken from  \citet{SchapperEtAl2011demnom}. See their article for discussion and illustration of the meanings and uses of such demonstratives\ist{demonstrative}.} The terms `\textsc{near}' and `\textsc{far'} are used instead for the few occasions in which we find distance-related distinctions between \textsc{elevated} terms.

Finally, I use the term ``elevational''\is{elevation} to refer to the sets of non-verbal items denoting a location that is at a certain elevation\is{elevation} relative to the deictic\is{deixis} centre. I use the term ``locational'' to refer to paradigms of \textsc{elevated} and \textsc{unelevated} terms referring to locations. This means, \textsc{elevated} locationals are ``elevationals\is{elevation}'', while \textsc{unelevated} locationals are functional equivalents to such items as English ``here'' and ``there''. However, I avoid the common label given to these (``demonstrative\is{demonstrative} adverbs'', as, e.g., in \citealt{Diessel1999}) since locationals in AP languages are not typically restricted to adverbial positions, but can often also occur as predicates and in NPs. I reserve the term ``demonstrative\is{demonstrative}'' for an NP constituent that refers to an entity by locating it in space. By contrast, locationals, including elevationals\is{elevation}, denote a location relative to which a referent can be identified in space. The morpho-syntax of elevationals in individual languages will be described in {\S} \ref{sec:7:3}.

\section{Alor-Pantar elevational\ist{elevation} systems}\label{sec:7:3}
The expression of elevation\is{elevation} is considered in seven AP languages from across the archipelago. I discuss languages in order of the complexity of their elevational\is{elevation} systems. Complexity here is calculated looking at both the number of elevation\is{elevation} marked terms and the number of semantic components within the different elevational\is{elevation} domains. The relative complexity of the different AP systems is discussed at the end of this section ({\S} \ref{sec:7:3.8}).

\subsection{Wersing\ilt{Wersing}} 
Wersing\il{Wersing} has one of the simpler elevational\is{elevation} systems, with a total of nine elevation\is{elevation} marked terms. There are three elevationals\is{elevation} for the three elevational\is{elevation} heights, each matched with motion\is{motion} verbs to and from the deictic\is{deixis} centre (Table~\ref{tab:7:wersing}). No additional semantic distinctions are made in the elevational\is{elevation} or verbal paradigm.\footnote{This section is based on \citet[457-458]{SchapperEtAltawersing}.}

\begin{table}\centering
\begin{tabular}{l>{\it}l>{\it}p{2cm}>{\it}l}
\mytopline
 & \rm Elevationals & \multicolumn{2}{c}{\rm Elevational\ist{elevation} motion\is{motion} verb}\\
 &  & \rm From \textsc{dc} & \rm To \textsc{dc}\\
\midrule
{\scshape level} & mona & -wai & -mai \\
{\scshape high} & tona &  -mid & -dai\\
{\scshape low} &  yona &   -a  & -sir\\
\mybottomline
\end{tabular}
\caption{Wersing\ilt{Western Pantar} elevation\ist{elevation} terms}
\label{tab:7:wersing}
\end{table}

Wersing\ilt{Western Pantar} elevationals\is{elevation} can be used as one-place predicates encoding the location of a NP referent at an elevation\is{elevation} relative to the speaker. Example \REF{ex:7:3} illustrates this predicative use. 

\xbox{\textwidth}{
\ea%3
\label{ex:7:3}
\langinfo{Wersing}{}{\citealt[457]{SchapperEtAltawersing}} \\
\gll Sobo  ba  \textbf{tona}.\\
        house  \textsc{art} \textbf{\textsc{high}} \\
\glt`The house is up there.' 
\z
}
  



   

The elevationals\is{elevation} also have non-predicative uses where they locate an action or an entity by their elevation. In these contexts the elevational\is{elevation} follows the clausal element(s) over which it has scope. In \REF{ex:7:4} the elevational\is{elevation} \textit{mona} follows the NP headed by \textit{pei} `pig' and denotes the elevation\is{elevation} of the pig at the time of its still breathing. In \REF{ex:7:5} \textit{yona} follows the verbal predicate \textit{aki} `call' and denotes the elevation\is{elevation} at which the calling takes place.
\todo{is indeed 258 intended in these examples, and not 458?}


\ea%4
\label{ex:7:4}
\upshape
 NP scope
\langinfo{Wersing}{}{\citet[258]{SchapperEtAltawersing}} \\
\gll   Pei   ba            \textbf{{mona}}      de                       geki{\ng}  sesai.\\
       pig  \textsc{art} \textbf{\textsc{down}} \textsc{ipfv} \textsc{3-}breathe  breath \\
\glt  `The pig (that is) over there is still breathing.'
\z

 

 



\ea%5
\label{ex:7:5}
\upshape
Predicate scope    
\langinfo{Wersing}{}{\citet[258]{SchapperEtAltawersing}} \\
\gll David   aki  \textbf{{yona}}{.} \\
       David  call  \textbf{\textsc{down}} \\
\glt `David calls (from) down there.'
\z

  

 

 

\subsection{Teiwa\ilt{Teiwa}}
Teiwa\il{Teiwa} also has a simple 9-term elevational\is{elevation} system (Table~\ref{tab:7:teiwa}). Like Wersing\il{Wersing}, elevationals\is{elevation} and elevation\is{elevation} marked motion\is{motion} verbs distinguish the three elevational\is{elevation} heights and no additional semantic distinctions are made.

 



\begin{table}\centering
\begin{tabular}{l>{\it}l>{\it}l>{\it}l}
\mytopline
 & \rm Elevationals & \multicolumn{2}{c}{\rm Elevational\ist{elevation} motion\is{motion} verb}\\
 &  & \rm From \textsc{dc} & \rm To \textsc{dc}\\
\midrule
{\scshape level} & {wunaxai} & {wa} & ma\\
{\scshape high} & {maraqai} & {mir} & daa\\
{\scshape low} & {yaqai} & {yix} & yaa\\
\mybottomline
\end{tabular}

\caption{Teiwa\ilt{Teiwa} elevation\ist{elevation} terms}
\label{tab:7:teiwa}
\end{table}

Teiwa\ilt{Teiwa} elevationals\is{elevation} occur predicatively, where they indicate the elevational\is{elevation} height of the NP referent, as in \REF{ex:7:6}. 

 

\ea%6
\label{ex:7:6}
\langinfo{Teiwa}{}{\citealt[142]{Klamer2010grammar}} \\
\gll  Uy  nuk   un  \textbf{maraqai}. \\
       \textsc{3sg} one  \textsc{cont} \textbf{\textsc{high}} \\
\glt `Is a person up there?'
\z



 

 

Elevationals\is{elevation} in Teiwa\il{Teiwa} can also occur in positions both before and after predicates. In \REF{ex:7:7} \textit{maraqai} precedes the postpositional\is{adposition} predicate \textit{uyan me{\textglotstop}}, and locates it as at a higher elevation\is{elevation} than the speaker. In \REF{ex:7:8} \textit{yaqai} after the verb \textit{yix} denotes the location resulting from the motion\is{motion} as at a lower elevation\is{elevation} than the speaker.



\ea%7
\label{ex:7:7}
\langinfo{Teiwa}{}{\citealt[141]{Klamer2010grammar}} \\
\gll A  \textbf{{maraqai}} {uyan}  me{\textglotstop}{.} \\
      \textsc{3sg} \textbf{\textsc{high}} mountain  in    \\
\glt `He's in the mountains up there.'  
\z

 

   

 



\ea%8
\label{ex:7:8}
\langinfo{Teiwa}{}{Klamer, fieldnotes} \\
\gll Iman     yix-in \textbf{yaqai}. \\
       \textsc{3pl} go.\textsc{low}{}-\textsc{real} \textbf{\textsc{low}} \\
\glt  `They went down there.'
\z

 

 



\subsection{Abui\ilt{Abui}}
In Abui\il{Abui} elevational\is{elevation} motion\is{motion} verbs maintain the simple three-way distinction already observed in Wersing\il{Wersing} and Teiwa\il{Teiwa}. However, the elevationals\is{elevation} show an extra degree of elaboration in the \textsc{high} and \textsc{low} spheres, with a distance contrast being added between \textsc{near} and \textsc{far} locations. The \textsc{level} sphere does not show this extra semantic component.



\begin{table}\centering
\caption{Abui\ilt{Abui} elevation\ist{elevation} terms}
\label{tab:7:abui}
\begin{tabular}{>{\sc}l>{\sc}r>{\it}c>{\it}c>{\it}c}
\mytopline
& 		& \rm Elevationals 		& \multicolumn{2}{c}{\rm Elevational\ist{elevation} motion\is{motion} verb}\\ 
		& 		&              		& \rm From \textsc{dc} & \rm To \textsc{dc}\\
\midrule
{level}	&  		& {oro}			& we 		& me \\
\\
\multirow{2}{*}{high}	& {near}& \textit{\'o} {\rm\dag} 	& \multirow{2}{*}{marei} 	& \multirow{2}{*}{mara{\ng}}\\
		& {far} & {w\'o} 		&        	& 			\\
\\
\multirow{2}{*}{low}  	& {near}& {\`o} 		& \multirow{2}{*}{pa}		&    \multirow{2}{*}{sei}     \\
		& {far} & {w\`o} 		&  		&				\\
\mybottomline
\end{tabular}

{\dag} Accents mark tone. The rising accent marks high tone, while the grave accent marks low tone. See \citet[60]{Kratochvil2007}
\end{table}

\footnotetext{The syntactic classification of the elevationals\ist{elevation} is that of the present author. \citet{Kratochvil2007} includes elevationals\ist{elevation} in a single class with the demonstratives\is{demonstrative} \textit{do, o, to, yo}, and the articles \textit{hu} and \textit{nu}. These two sets have different syntactic distributions from the set of elevationals\ist{elevation} I identify. See \citet{SchapperEtAl2011demnom} for details on the morphosyntactic properties of Abui\il{Abui} demonstratives\is{demonstrative}. The distributional characteristics of Abui\il{Abui} elevations\ist{elevation} are set out in the main text here.}


Abui\il{Abui} elevationals\is{elevation} can be predicates, as for instance in \REF{ex:7:9} where \textit{oro} denotes elevation\is{elevation} of the branch in relation to the speaker. Where they indicate the elevation\is{elevation} at which an action takes place, elevations\is{elevation} occur directly before a verb, as with the predicative verb \textit{burok} `move' in \REF{ex:7:10}.



\ea%9
\label{ex:7:9}
\langinfo{Abui}{}{Kratochv\'il, Abui corpus} \\
\gll  Bataa   ha-ta{\ng} dara  \textbf{oro}. \\
       tree  3.\textsc{poss}{}-arm   still  \textbf{\textsc{level}} \\
\glt `The tree branch is still over there.'
\z



 

\ea
\label{ex:7:10}
\langinfo{Abui}{}{Kratochv\'il, Abui corpus} \\
\gll Bataa   ha-ta{\ng} dara \textbf{oro} burok.  \\
tree  3.\textsc{poss}{}-arm   still  \textbf{\textsc{level}} move  \\
\glt   `The tree branch is still moving over there.'  
\z


 

  



Abui\il{Abui} elevationals\is{elevation} can also occur in NPs. In an NP headed by a noun the elevational\is{elevation} follows the head, but to the left of any article or demonstrative\is{demonstrative} marking the right periphery of the NP. For instance, in \REF{ex:7:11} the \textsc{level} elevational\is{elevation} \textit{oro} follows the NP head \textit{fu} `betel' but precedes the demonstrative\is{demonstrative} \textit{do}. It indicates the elevation\is{elevation} at which the betel palm is found. An elevational\is{elevation} can also occur in an NP without a head noun. In this case the elevational\is{elevation} is the head of the NP and the referent of the NP is the location indicated by the elevational\is{elevation}. In \REF{ex:7:12} the \textsc{low} elevational\is{elevation}\textit {\`o} heads the NP marked by the article \textit{nu} and the demonstrative\is{demonstrative} \textit{do}. This NP occurs in the postpositional\is{adposition} phrase headed by \textit{={\ng}} and denotes the goal location for the motion\is{motion} dignified by the elevational\is{elevation} verb \textit{pa} `go down'.



\ea%11
\label{ex:7:11}
\langinfo{Abui}{}{Kratochv\'il, Abui corpus} \\
\gll Di  yaa  {\ob}fu \textbf{oro} do{\cb}\textsubscript{NP} mia. \\
  \textsc{3} go   betel  \textbf{\textsc{level}} \textsc{dem} in  \\
\glt   `He went to this betel (palm) (which is) over here.'  
\z

  

 



 

\ea%12
\label{ex:7:12}
\langinfo{Abui}{}{Kratochv\'il, Abui corpus} \\
\gll {\dots} ha-buka{\ng} dika{\ng} mi     {\ob}\textbf{{\`o}} \textbf{nu}    do{\cb}\textsubscript{NP}={\ng}    pa. \\
  {\dots} 3.\textsc{poss}{}-thimble  again  take  \textbf{\textsc{low.far}} \textsc{art}  \textsc{dem=loc} go.\textsc{low}  \\
\glt `{\dots} (he) again goes to take his thimble to down there.' 
\z

 

   

  

\subsection{Blagar\ilt{Blagar}}
Blagar\il{Blagar} has a plethora of elevation\is{elevation} terms, with a total of 32 elevation-marked\is{elevation} forms. These occur in paradigms with speech participant-anchored terms (Table~\ref{tab:7:blagar}). Blagar\il{Blagar} has five locationals. These appear both as independent words and as constituents of multiple sets of derived items (bolded in Table~\ref{tab:7:blagar}). These particles consist of the three elevationals\is{elevation}, \textit{mo} `\textsc{level',} \textit{do} `\textsc{high'} and \textit{po} `\textsc{low',} plus two \textsc{unelevated} speech participant-anchored locationals, \textit{{\textglotstop}}\textit{a} `\textsc{prox.spkr'} and \textit{{\textglotstop}}\textit{u} `\textsc{prox.addr'.} Only the elevational\is{elevation} motion\is{motion} verbs, which have different etymologies, do not include the locational forms in their paradigms. 



\setlength{\tabcolsep}{3pt}
\begin{table}\centering
\small
\caption{Blagar\ilt{Blagar} elevation\ist{elevation} terms}
\label{tab:7:blagar}
\begin{tabular*}{.5\textwidth}{@{\extracolsep{\fill}}>{\sc}l>{\sc}l>{\it}c} 
\mytopline
                & &               \rm Locationals    \\
\midrule 
{level}  & &               \textbf{mo} \\
{high}	 & &              \textbf{do}\\
{low}	 & &                 \textbf{po}\\
\multirow{2}{*}{unelevated} &
	  \textsc{  prox.spkr} &           {\textglotstop}a     \\
	& \textsc{  prox.addr} &         {\textglotstop}u        \\
\mybottomline
\end{tabular*} 
 
\begin{tabular*}{\textwidth}{@{\extracolsep{\fill}}>{\sc}l>{\sc}l>{\it}c>{\it}c>{\it}c>{\it}c>{\it}c} 
\mytopline
                &               & \multicolumn{5}{c}{\rm Stative verbs} \\
                &               & \rm be as much as & \rm be as big as & \rm be as high as & \rm be at & \rm be at vis \\
\midrule 
{level}  &               & \textbf{mo}noaŋ & \textbf{mo}vaŋ & \textbf{mo}hukaŋ & \textbf{mo}{\textglotstop}e & \textbf{momo} \\
{high}	                 &             & \textbf{do}noaŋ & \textbf{do}vaŋ & \textbf{do}hukaŋ & \textbf{do}{\textglotstop}e & \textbf{dodo} \\
{low}	                 &             & \textbf{po}noaŋ & \textbf{po}vaŋ & \textbf{po}hukaŋ & \textbf{po}{\textglotstop}e & \textbf{popo} \\
\multirow{2}{*}{unelevated}     &  
	\textsc{prox.spkr}       & {\textglotstop}anoaŋ & {\textglotstop}avaŋ & {\textglotstop}ahukaŋ & {\textglotstop}a{\textglotstop}e & {\textglotstop}a{\textglotstop}a\\
	      &\textsc{prox.addr} & {\textglotstop}unoaŋ & {\textglotstop}uvaŋ & {\textglotstop}uhukaŋ & {\textglotstop}u{\textglotstop}e & {\textglotstop}u{\textglotstop}u\\

\mybottomline
\end{tabular*} 
 
\begin{tabular*}{\textwidth}{@{\extracolsep{\fill}}>{\sc}l>{\sc}l>{\it}c>{\it}c>{\it}c}
\mytopline
              &            & \multicolumn{2}{c}{\rm Demonstratives\ist{demonstrative}}                                      & \rm Manner   adverbs    \\   
              &            &\rm  Basic                              & \rm   Collective                       &                  \\              
\midrule 
{level}       &            & {{\textglotstop}}{a}\textbf{{mo}} & {{\textglotstop}}{ana}\textbf{{mo}} & \textbf{{mo}}{la{\ng}}\\ 
{high}        &            & {{\textglotstop}}{a}\textbf{{do}} & {{\textglotstop}}{ana}\textbf{{do}} & \textbf{{do}}{la{\ng}} \\       
{low}         &            & {{\textglotstop}}{a}\textbf{{po}} & {{\textglotstop}}{ana}\textbf{{po}} & \textbf{{po}}{la{\ng}}\\         
\multirow{2}{*}{unelevated}  
             & \textsc{prox.spkr} & {{\textglotstop}}{a{\ng}a}        & {{\textglotstop}}{ana{\ng}a}        & {{\textglotstop}}{ala{\ng}} \\   
             & \textsc{prox.addr} & {{\textglotstop}}{a{\ng}u}        & {{\textglotstop}}{ana{\ng}u}        & {{\textglotstop}}{ula{\ng}} \\   
\mybottomline
\end{tabular*} 

\begin{tabular*}{.5\textwidth}{@{\extracolsep{\fill}}>{\sc}l>{\it}c>{\it}c}
\mytopline
               & \multicolumn{2}{c}{\rm Elevational\ist{elevation} motion\is{motion} verbs} \\
               &\rm  From \textsc{dc}& \rm To \textsc{dc}\\ 
\midrule 
{level}         & va & ma \\
{high}        & {mida} & da  \\
{low}         & {{\textglotstop}}{ipa} & ya\\ 
{unelevated}  & {{\textglotstop}}{ila} & ho{\textglotstop}a\\
\mybottomline
\end{tabular*}
\end{table}

\setlength{\tabcolsep}{6pt} 
\normalsize


The elevationals\ist{elevation} occur in two positions: between subject and predicate, as in \REF{ex:7:13}, and following a predicate, as in \REF{ex:7:14}. The different positions are associated with different epistemic values. The clause-medial position connotes epistemic certainty on the part of the speaker, while the clause-final position connotes epistemic accessibility to the addressee, that is, that the addressee is or could be aware of the situation described in the clause (Hein Steinhauer, p.c.).\footnote{\citet{SchapperEtAl2011demnom} describe similar epistemic uses of demonstratives\is{demonstrative} in TAP languages. Blagar\il{Blagar} appears to be unique in its use of different syntactic positions of deictic\is{deixis} particles to denote different levels of epistemic accessibility.}  



\ea%13
\label{ex:7:13}
\langinfo{Blagar}{}{Steinhauer, p.c.} \\
\gll {{\textglotstop}}{ana} \textbf{{po}} {ab}    {na.} \\
    \textsc{3sg.subj} \textbf{\textsc{low}} fish  eat\\
\glt  `S/he eats fish down there (for sure).'
\z

 

 
\ea%14
\label{ex:7:14}
\langinfo{Blagar}{}{Steinhauer, p.c.} \\
\gll {{\textglotstop}}{ana}  ab    na \textbf{{po}}{.}\\
  \textsc{3sg.subj} fish  eat  \textbf{\textsc{low}}   \\
\glt   `S/he eats fish down there (as you may know).'
\z
 


  





The derived demonstratives\is{demonstrative} (basic and collective) occur marking the right-hand periphery of the NP either with \REF{ex:7:15} or without a noun head \REF{ex:7:16}.



\ea%15
\label{ex:7:15}
\langinfo{Blagar}{}{\citealt{Steinhauer2012}} \\
\gll {\ob}Hava  kiki  {\textglotstop}a-na-\textbf{po}{\cb}\textsubscript{NP} ka{\textglotstop}ana.\\
    house  little  \textsc{dem-coll}\textbf{\textsc{-low}} black \\
\glt `That group of little houses down there is black.'
\z

  



  



\ea%16
\label{ex:7:16}
\langinfo{Blagar}{}{\citealt{Steinhauer2012}} \\
\gll {\textglotstop}ini {\ob}{\textglotstop}{a}\textbf{-mo}{\cb}\textsubscript{NP} mi   {mihi}.\\
   \textsc{3pl.subj} \textsc{dem-level} \textsc{loc} sit \\
\glt `They live in that (place) over there.'
\z

  

 

  

The derived manner adverbs occur in one of two positions: (i) preceding the subject \REF{ex:7:17}, or (ii) following the predicate \REF{ex:7:18}.



\ea%17
\label{ex:7:17}
\langinfo{Blagar}{}{\citealt{Steinhauer2012}} \\
\gll  {{\textglotstop}}{u-la}{{\ng}}    ana  tia. \\
   \textsc{prox.addr}{}-like  \textsc{2sg.subj} sleep \\
\glt   `That is how you sleep.'
\z



 \ea%18
\label{ex:7:18}
\langinfo{Blagar}{}{\citealt{Steinhauer2012}} \\
\gll Ana  tia-t  {{\textglotstop}}{a-la}{{\ng}}{.} \\
  \textsc{2sg.subj} sleep-\textsc{mnr} \textsc{prox.spkr}{}-like  \\
\glt   `You sleep like this.'
\z 
 
 



Derived stative verbs refer to measurement \REF{ex:7:19}, and static location \REF{ex:7:20}. 



\ea%19
\label{ex:7:19}
\langinfo{Blagar}{}{\citealt{Steinhauer2012}} \\
\gll  Ne   hava  \textbf{{do-}}{va{\ng}}\textbf{{.}} \\
   \textsc{1sg.poss} house  \textbf{\textsc{high}}\textsc{{}-}big.as  \\
\glt   `My house is as big as the one up there.'
\z









\ea%20
\label{ex:7:20}
\langinfo{Blagar}{}{\citealt{Steinhauer2012}} \\
\gll  {\textglotstop}ana  mida  \textbf{do}-{\textglotstop}e.  \\
  \textsc{3sg.subj} go.\textsc{high} \textbf{\textsc{high}}-be.at     \\
\glt   `S/he went up and is up there.'
\z







\subsection{Adang}\label{sec:7:3.5}

\begin{table}
\begin{tabular}{>{\sc}l>{\sc}l>{\it}l>{\it}l>{\it}l}
\mytopline

        &         & \multicolumn{2}{c}{\rm Locationals}                                                   & \rm Demonstratives\ist{demonstrative}                          \\      
        &         &  \rm Basic                                  &  \rm Directional\ist{direction}                              &                                         \\      
\midrule 
\multirow{2}{*}{level} 
        &         & {m}{{\textopeno}}{{\ng}}               & \multirow{2}{*}{{fal}{{\textepsilon}}}                    & \multirow{2}{*}{{h}{{\textepsilon}}{m}{{\textopeno}}}       \\   
        &         & {mal}{{\textepsilon}}                  & \\
\\
\multirow{5}{*}{high}  
        &         & {t}{{\textopeno}}{{\ng}}                & \multirow{5}{*}{{midl}{{\textepsilon}}}                  & \multirow{5}{*}{{h}{{\textepsilon}}{t}{{\textopeno}}}       \\   
        &         & {mad}{{\textopeno}}{{\ng}l}{{\textepsilon}}&\\
        &         & {ada{\ng}l}{{\textepsilon}}             & \\
        &         & {ta}{{\textglotstop}}{l}{{\textepsilon}}& \\
        &         & {tal}{{\textepsilon}}                   & \\
\\
\multirow{3}{*}{low}   
        &         & \multirow{3}{*}{{p}{{\textopeno}}{{\ng}}}& {ipl}{{\textepsilon}}\textbf{{}}        & \multirow{3}{*}{{h}{{\textepsilon}}{p}{{\textopeno}}}   \\     
        &         &                                          & {h}{{\textepsilon}l}{l}{{\textepsilon}} &                                         \\     
        &         &                                          & {lifa{\ng}l}{{\textepsilon}}            &                                         \\     
\multirow{2}{*}{unelevated}
    & {prox.spkr} & \multirow{2}{*}{{{\textopeno}}{{\ng}}}  &                                          & {h}{{\textopeno}{\textglotstop}{\textopeno}} \\
  &  {prox.addr}   &                                         &                                          & ho                                           \\
\mybottomline
\end{tabular}

 \begin{tabular}{>{\sc}l>{\it}l>{\it}l}
\mytopline
               & \multicolumn{2}{c}{\rm Elevational\ist{elevation} motion\ist{motion} verbs} \\
               &\rm  From \textsc{dc}& \rm To \textsc{dc}\\ 
\midrule 
{level}         & fa & ma \\
{high}        &  mid & mad{\textopeno}{\ng}  \\
{low}         & ip & h{\textepsilon}l\\ 
{unelevated}  & sam & ho{\textglotstop}\\
\mybottomline
\end{tabular}

\caption{Adang\ilt{Adang} elevation\ist{elevation} terms \citep[reanalysed from][]{Haan2001}}
\label{tab:7:adang}
\end{table}

Adang\il{Adang} has 22 elevation\is{elevation} marked terms occurring in a paradigm with \textsc{unelevated} terms (Table~\ref{tab:7:adang}). Elevation\is{elevation} terms are divided across three word classes: locationals, demonstratives\is{demonstrative} and elevational\is{elevation} motion\is{motion} verbs. These are described below.

Elevational\is{elevation} motion\is{motion} verbs follow the simple 6-term pattern that we have seen for all AP discussed thus far. E\textsc{levated} demonstratives\is{demonstrative} have a three way elevational\is{elevation} contrast marked by \textit{m}\textit{{\textopeno}} `\textsc{level}', \textit{t}\textit{{\textopeno}} `\textsc{high}' and \textit{p}\textit{{\textopeno}} `\textsc{low}', while their \textsc{unelevated} counterparts are essentially characterisable by the absence of these morphemes. The largest elevational\is{elevation} word class is the \textsc{elevated} locationals, or elevationals\is{elevation}. These divide into two sets, basic and directional\is{direction}, that are distinguished from one another both formally and semantically. The basic set has the elevational\is{elevation} marking morphemes we saw in the \textsc{elevated} demonstratives\is{demonstrative} marked with \textit{{}-}\textit{{\ng}} and occurs in a paradigm with an \textsc{unelevated} term. The directional\is{direction} set of elevationals differs from the basic set in that they are derived from other roots with the suffix \textit{{}-l}\textit{{\textepsilon}} and do not have \textsc{unelevated} counterparts.
 

Semantically, the contrast between the basic and directional\is{direction} elevationals\is{elevation} is in the first place the type of elevation\is{elevation} they reference. Basic elevationals\is{elevation} refer to global elevation\is{elevation}. In the directional\is{direction} set, different terms have different elevational\is{elevation} reference. In Table~\ref{tab:7:ex21} I set out the elevational\is{elevation} reference and the sources of roots of the directional\is{direction} elevationals\is{elevation}. The two geophysical elevationals\is{elevation} in Adang\il{Adang} reference a trajectory between the inland mountains where Adang\il{Adang} villages are traditionally located and the coastal lowlands away from Adang\il{Adang} villages. The two vertical elevationals\is{elevation} reference a location that is vertically \textsc{high} in relation to the \textsc{dc}. The difference between \textit{ta}\textit{{\textglotstop}}\textit{l}\textit{{\textepsilon}} and \textit{tal}\textit{{\textepsilon}} appears to be one of the contact relationship between the \textsc{dc} and the referent location. \textit{Ta}\textit{{\textglotstop}}\textit{l}\textit{{\textepsilon}} references a location straight up from the \textsc{dc} without being in contact with the \textsc{dc}, while \textit{tal}\textit{{\textepsilon}} references a location that is directly on top of and in contact with the \textsc{dc}. Finally, the directional\is{direction} elevationals\is{elevation} with global elevational\is{elevation} reference are built from elevation\is{elevation} marked motion\is{motion} verbs. They differ referentially from the basic set which also refers to elevation\is{elevation} globally with reference to location as being towards or away from the \textsc{dc}, according to what elevational\is{elevation} motion\is{motion} verb is the root (see Table~\ref{tab:7:ex21}).  


\begin{table}[h]
\caption{Sources of Adang directional elevationals.}
\label{tab:7:ex21}
\begin{tabular}{l>{\it}llll}
\mytopline 
Geophysical: & \textit{ada{\ng}l}\textit{{\textepsilon}} & {\scshape mountain.wards} & {\textless}   {\itshape ada{\ng}} & `mountain'\\
 & \textit{lifa{\ng}l}\textit{{\textepsilon}} & {\scshape coast.wards} & {\textless}   {\itshape lifa{\ng}} & `anchor'\\
\\
Vertical: & \textit{ta}\textit{{\textglotstop}}\textit{l}\textit{{\textepsilon}} & {\scshape high.vertical} & {\textless}   {\itshape ta} & `(put) on'\\
 & \textit{tal}\textit{{\textepsilon}} & {\scshape on.vertical}  &  & \\
\\
Global: & \textit{midl}\textit{{\textepsilon}} & {\scshape high.away.dc} & {\textless}   {\itshape mid} & `go\textsc{.high}'\\
 & \textit{mad}\textit{{\textopeno}}\textit{{\ng}l}\textit{{\textepsilon}}\textbf{\textit{}} & {\scshape high.wards.dc} & {\textless}   \textit{mad}\textit{{\textopeno}}\textit{{\ng}} & `come\textsc{.high'}\\
 & \textit{ipl}\textit{{\textepsilon}}\textbf{\textit{}} & {\scshape low.away.dc} & {\textless}   {\itshape ip} & `go\textsc{.low'}\\
 & \textit{h}\textit{{\textepsilon}l}\textit{l}\textit{{\textepsilon}} & {\scshape low.wards.dc} & {\textless}   \textit{h}\textit{{\textepsilon}l} & `come\textsc{.low'}\\
 & \textit{fal}\textit{{\textepsilon}} & {\scshape level.away.dc} & {\textless}   {\itshape fa} & `go\textsc{.level'}\\
 & \textit{mal}\textit{{\textepsilon}} & {\scshape level.wards.dc} & {\textless}   \textit{ma}\textbf{\textit{}} & `come\textsc{.level'}\\
\mybottomline
\end{tabular}
\end{table}

Despite the formal ands semantic differences between basic and directional\is{direction} elevationals\is{elevation}, they have the same syntactic distributional properties and cannot cooccur in the clause. This indicates that they are of one and the same word class. They occur in three positions. 

First, an elevational\is{elevation} can occur as an independent clausal predicate. This is seen in \REF{ex:7:22} with the basic elevational\is{elevation} \textit{t}\textit{{\textopeno}}\textit{{\ng}} `\textsc{high}' and in \REF{ex:7:23} with the directional\is{direction} elevational\is{elevation} \textit{ipl}\textit{{\textepsilon}} \textsc{`low.dc-away'.} 



\ea%22
\label{ex:7:22}
\langinfo{Adang}{}{\citealt[192]{Haan2001}} \\
\gll  Aru   nu  \textbf{{t}}\textbf{{{\textopeno}}}\textbf{{{\ng}}}{.}  \\
   deer  one  \textsc{high}   \\
\glt   `There is a deer up there.'
\z









\ea%23
\label{ex:7:23}
\langinfo{Adang}{}{\citealt[192]{Haan2001}} \\
\gll  Bel  \textbf{{ipl}}\textbf{{{\textepsilon}}}{.}  \\
  dog  \textsc{low.away.dc}    \\
\glt   `There are dogs down there (away from the speaker).'
\z







Second, elevationals\is{elevation} can occur adverbially before a predicate and its (if any) adjunct. In \REF{ex:7:24} the basic elevational\is{elevation} \textit{m}\textit{{\textopeno}}\textit{{\ng}} `\textsc{level}' indicates the locational setting for the verbal predicate \textit{tuf} `stand' and its adjunct \textit{bana} \textit{mi} `in the forest'. In \REF{ex:7:25} the directional\is{direction} elevational\is{elevation} \textit{ipl}\textit{{\textepsilon}} \textsc{`low.dc-away'} modifies the simple verbal predicate \textit{tar} `lie down'. 



\ea%24
\label{ex:7:24}
\langinfo{Adang}{}{\citealt[237]{RobinsonEtAltaadang}} \\
\gll  Ti  ta{{\textglotstop}}{at}  ho  \textbf{{m}{{\textopeno}}}\textbf{{{\ng}}} {bana}  mi  {tuf=eh}. \\
    tree  dry  \textsc{dem} \textsc{level} forest  \textsc{in} stand=\textsc{prog} \\
\glt   `The dry stick is standing over there in the forest.'
\z









\ea%25
\label{ex:7:25}
\langinfo{Adang}{}{\citealt[191]{Haan2001}} \\
\gll   Bel  \textbf{ipl}\textbf{\textepsilon} tar=eh. \\
   dog  \textsc{low.away.dc} lie.down=\textsc{prog}   \\
\glt   `There are dogs lying down down there (in a direction\is{direction} away from the speaker).' 
\z







Finally, elevationals\is{elevation} can also occur with an NP. Where an elevated\is{elevation} demonstrative\is{demonstrative} also occurs in the NP, then the elevational\is{elevation} and demonstrative\is{demonstrative} must match in elevational marking. The NP headed by \textit{bel} `dog' is modified by the basic \textsc{level} elevational\is{elevation} and the \textsc{level} demonstrative\is{demonstrative} in \REF{ex:7:26} and by a directional\is{direction} \textsc{low} elevational\is{elevation} and the \textsc{low} demonstrative\is{demonstrative} in \REF{ex:7:27}.  



\ea%26
\label{ex:7:26}
\langinfo{Adang}{}{\citealt[188]{Haan2001}} \\
\gll Bel  \textbf{m{\textopeno}{\ng}}  \textbf{h{\textepsilon}m{\textopeno}} mat{\textepsilon}.  \\
  dog  \textsc{level} \textsc{dem.level} big      \\
\glt   `That dog over there is large.'
\z

 
 





\ea%27
\label{ex:7:27}
\langinfo{Adang}{}{\citealt[188]{Haan2001}} \\
\gll   Bel  \textbf{ipl{\textepsilon}} \textbf{h{\textepsilon}p{\textopeno}} mat{\textepsilon}. \\
    dog  \textsc{low.away.dc} \textsc{dem.low} big    \\
\glt   `That dog over there is large.'
\z



   



Table~\ref{tab:7:combinations} summarises the permitted combinations of demonstrative\is{demonstrative} and elevationals\is{elevation}. Note that the only exception to the matching of elevations\is{elevation} between demonstratives\is{demonstrative} and elevationals\is{elevation} within an NP is with \textit{tal}\textit{{\textepsilon}} `\textsc{on.vertical'}. This elevational\is{elevation} refers to the location of \textit{another entity on} the NP referent. Thus, the NP referent may be specified with a demonstrative\is{demonstrative} as being \textsc{high}, \textsc{low} or \textsc{level} in relation to the speaker as \textsc{dc}, and then also be located on another entity by means of \textit{tal}\textit{{\textepsilon}}. The possibility of these combinatorics is illustrated in \REF{ex:7:28} and \REF{ex:7:29}.

 

\begin{table}[hbt]
\centering
\begin{tabular}{>{\sc}l>{\it}l>{\it}l}
\mytopline
 & \rm Demonstrative\ist{demonstrative}   & \rm Elevational\ist{elevation}\\
\midrule  
\multirow{6}{*}{high} & \multirow{6}{*}{{h{\textepsilon}t{\textopeno}}}   & m{\textopeno}{\ng}\\
 &  &  midl{\textepsilon}\\
 &  &  mad{\textopeno}{\ng}l{\textepsilon}\\
 &  &  ada{\ng}l{\textepsilon} \\
 &  &  ta{\textglotstop}l{\textepsilon}\\
 &  &  tal{\textepsilon}  \\
 \\
\multirow{5}{*}{low} & \multirow{5}{*}{h{\textepsilon}p{\textopeno}}   & p{\textopeno}{\ng}\\
 &  &   ipl{\textepsilon}\\
 &  &   h{\textepsilon}ll{\textepsilon}  \\
 &  &   lifa{\ng}l{\textepsilon}\\
 &  &   tal{\textepsilon}  \\
 \\
\multirow{4}{*}{level} & \multirow{4}{*}{h{\textepsilon}m{\textopeno}}   & m{\textopeno}{\ng}\\
 &  &   fal{\textepsilon}  \\
 &  &   mal{\textepsilon}\\
 &  &   tal{\textepsilon} \\
\mybottomline
\end{tabular}
\caption{Combinations of \textsc{elevated} demonstratives\ist{demonstrative} and elevationals\ist{elevation} \citep[adapted from][188]{Haan2001}}
\label{tab:7:combinations}
\end{table}

 

\ea%28
\label{ex:7:28}
\langinfo{Adang}{}{\citealt[188]{Haan2001}} \\
\gll   Nam{\textepsilon} be  \textbf{tal{\textepsilon}} \textbf{h{\textepsilon}m{\textopeno}} fail. \\
   person  mango  \textsc{on.vertical} \textsc{dem.level} sell      \\
\glt   `Someone is selling those mangoes on the others mangoes (the upper group of mangoes) over there.'
\z







    

\ea%29
\label{ex:7:29}
\langinfo{Adang}{}{\citealt[188]{Haan2001}} \\
\gll  Bel  \textbf{tal{\textepsilon}} \textbf{h{\textepsilon}p{\textopeno}} mat{\textepsilon}.  \\
   person  \textsc{on.vertical} \textsc{dem.low} big    \\
\glt   `That dog up here from the others down there is big.'
\z



    

\clearpage

\subsection{Western Pantar\ilt{Western Pantar}}
Western Pantar\il{Western Pantar} has a total of 26 elevation\is{elevation} marked terms, occurring in paradigms with speech participant-anchored terms (Table~\ref{tab:7:wp}). As in Blagar\il{Blagar} and Adang\il{Adang}, elevation\is{elevation} marking is repeated across multiple paradigms of different word classes in Western Pantar\il{Western Pantar}. These are: locationals, demonstratives\is{demonstrative} and elevational\is{elevation} motion\is{motion} verbs. 


\begin{table}[htb]
\caption[Western Pantar elevation terms]{Western Pantar\ilt{Western Pantar} elevation\is{elevation} terms (adapted from \citealt{Holton2007} and \citealt{Holtontawesternpantar})}
\small
\label{tab:7:wp}

\begin{tabular}{>{\sc}l>{\sc}l>{\it}l>{\it}l>{\it}l>{\it}l>{\it}l>{\it}l}
\mytopline
       &           & \multicolumn{2}{c}{\rm Locationals}   & \multicolumn{4}{c}{\rm Demonstratives\ist{demonstrative}}                \\                                
       &           &                &                  & \multicolumn{2}{c}{\rm visible}             & \multicolumn{2}{c}{\rm invisible}      \\
       &           & {\scshape nspec}        & {\scshape spec}           & {\scshape nspec}           & {\scshape spec}              & {\scshape nspec}   & {\scshape spec}\\ 
\midrule 
{level}&           & {\bfseries mau}& {s}\textbf{{mau}}& \textbf{{mau}}{gu}& {s}\textbf{{mau}}{gu}& \textbf{{mau}}{me}& {s}\textbf{{mau}}{me}\\
{high} &           & {\bfseries dau}& {s}\textbf{{rau}}& \textbf{{dau}}{gu}& {s}\textbf{{rau}}{gu}& \textbf{{dau}}{me}& {s}\textbf{{rau}}{me}\\
{low}  &           & {\bfseries pau}& {s}\textbf{{pau}}& \textbf{{pau}}{gu}& {s}\textbf{{pau}}{gu}& \textbf{{pau}}{me}& {s}\textbf{{pau}}{me}\\
{unelevated}
& {prox.spkr}      & {iga}& {siga}& {aiga}& {saiga}& {igamme}& {sigamme}\\ 
& {dist.spkr{\dag}}& {ina}& {sina}& {aina}& {saina}& {inamme}& {sinamme}\\ 
\mybottomline
\end{tabular}

 \begin{tabular}{>{\sc}l>{\sc}l>{\it}l>{\it}l} 
\mytopline
             &  & \multicolumn{2}{c}{\rm Elevational\ist{elevation} motion\ist{motion} verbs} \\
             &  &\rm  From \textsc{dc}& \rm To \textsc{dc}\\ 
\midrule 
{level}       &        & wa & ma\\
\multirow{2}{*}{high}        
              & steep  & mia   & \multirow{2}{*}{midda{\ng}} \\
              & nsteep & rau{\ng} &         \\
\multirow{2}{*}{low}
              & steep  & pia   & \multirow{2}{*}{ya{\ng}} \\
              & nsteep & daka{\ng}&      \\
\mybottomline
\end{tabular}

{\textsc{{\dag}} The distal means away from speaker or other established deictic\is{deixis} centre. Not necessarily close to hearer.}

\end{table}



The number of elevational\is{elevation} motion\is{motion} verbs is higher than in the AP languages looked at thus far. This is due to an extra distinction between steep versus non-steep appearing in the verbs denoting motion\is{motion} away from the deictic\is{deixis} centre. The high number of elevation-marked\is{elevation} terms found in the three elevation-marked\is{elevation} word classes is, however, chiefly due to the existence of multiple paradigms of locationals and demonstratives\is{demonstrative} in Western Pantar\il{Western Pantar}. Locationals and demonstratives\is{demonstrative} have distinct paradigms for specific versus non-specific reference, and demonstratives\is{demonstrative} further have separate paradigms for visible versus non-visible referents. Across the locational and demonstrative\is{demonstrative} paradigms, marking for location has the same forms derived from the basic (i.e., non-specific) locationals. These are the three elevationals\is{elevation}, \textit{mau} `\textsc{level',} \textit{dau} `\textsc{high'} and \textit{pau} `\textsc{low'} (bolded in Table~\ref{tab:7:wp}), plus the two \textsc{unelevated} speech participant-anchored locationals, \textit{iga} `\textsc{prox.spkr'} and \textit{ina} `\textsc{dist.spkr'.} Specific-marked forms of locationals and demonstratives\is{demonstrative} are derived by means of \textit{s-} prefixed onto the basic locationals \REF{ex:7:30}. Demonstratives\is{demonstrative} are derived from the elevationals\is{elevation} by \textit{{}-gu} for the visible paradigm and \textit{{}-me} for the non-visible paradigm \REF{ex:7:31}.


\enlargethispage{1em}
\ea%30
\label{ex:7:30}
\langinfo{Western Pantar}{}{\citealt{Holton2011}} \\
\gll   {\ob}Ging  s\textbf{pau}gu{\cb}\textsubscript{NP} kua{\ng}  i-pari{\ng}.\\
    \textsc{3pl} \textsc{spec.low.vis:dem} moko.drum  \textsc{3pl-}surrender\\
\glt `Those who are the ones visible down there will hand over the moko drums.'
\z



 
  



\ea%31
\label{ex:7:31}
\langinfo{Western Pantar}{}{\citealt{Holton2011}} \\
\gll {\ob}Aname  ye  \textbf{dau}m{\cb}\textsubscript{NP} is   ta{\ng} ti{\textglotstop}a{\ng} kor  id-dia.\\
    person  one  \textsc{high.}\textsc{nvis:dem} banyan  on  sleep  snore  \textsc{prog}{}-go\\
\glt `Someone who is up there in a banyan tree sleeping and snoring away.'
\z

  
 

  

Western Pantar\il{Western Pantar} elevationals\is{elevation} occur as predicates denoting the location of a NP referent at an elevation\is{elevation} relative to the speaker. Example \REF{ex:7:32} illustrates this predicative use. 

 

\ea%32
\label{ex:7:32}
\langinfo{Western Pantar}{}{Holton, p.c.} \\
\gll  Hinani=b  \textbf{{srau}}{?} \\
   what=\textsc{foc} \textsc {spec.high}  \\
\glt `What is up there?'  
\z

 

 

  

Within the clause, elevationals\is{elevation} follow the element whose location they denote, and thus may appear clause-medially or -finally. For instance, in \REF{ex:7:33} the \textsc{low} elevational\is{elevation} \textit{pau} follows the subject \textit{eu} `girl' and denotes the location of her at the time of calling. In \REF{ex:7:34} \textit{pau} denotes the location of the object \textit{habba{\ng}} `village' which it follows, while in \REF{ex:7:35} \textit{mau} denotes the location of the pre-subject locative adjunct \textit{habba{\ng}} `village' which it follows.

 



\ea%33
\label{ex:7:33}
\langinfo{Western Pantar}{}{Holton, Western Pantar corpus} \\
\gll   Eu  \textbf{pau} asa{\ng} ,...\\
   girl  \textsc{low} say ... \\
\glt   `The girl down there says, ...' 
\z









\ea%34
\label{ex:7:34}
\langinfo{Western Pantar}{}{Holton, Western Pantar corpus} \\
\gll  Sinam   bila    ta{\ng}    misi{\ng}  i  habba{\ng} \textbf{pau}   ya     sauka{\ng}  pia. \\
    \textsc{spec.nvis:dem} hill  top  sit  \textsc{3pl.rflx.poss} village  \textsc{low}   toward  watch  go\textsc{.low.steep} \\
\glt   `(They) sat on the top of the mountain there and looked down at their village.'
\z

\ea%35
\label{ex:7:35}
\langinfo{Western Pantar}{}{\citealt[97]{HoltonEtAl2008}} \\
\gll  {Habba{\ng}} \textbf{mau} aname  hora{\ng}  sauke-yabe.\\
  village  \textsc{level} person  make.noise  women.dance \\
\glt  `Over in the village people are making noise dancing lego-lego.' 
\z


Elevationals\is{elevation} in clause-final position indicate the location at which the preceding predicate takes place. For instance, final \textit{pau} in \REF{ex:7:36} denotes that the event of \textit{teri} `anchoring' is at lower elevation\is{elevation} than the deictic\is{deixis} centre. Similarly, in \REF{ex:7:37} final \textit{dau} signals that the motion\is{motion} denoted by \textit{mia} `go.\textsc{high.steep}' is higher in elevation\is{elevation} than the deictic\is{deixis} centre. 



\ea%36
\label{ex:7:36}
\langinfo{Western Pantar}{}{Holton, Western Pantar corpus} \\
\gll  Asa{\ng}  siba{\ng}  tukka  yallu  paum  i-teri \textbf{pau}.\\
  say  driftwood  short  one  \textsc{low.nvisible:dem} \textsc{prog}{}-anchor  \textsc{low} \\
\glt  `Apparently, there's a short (piece of) driftwood caught down there.'
\z
     

\ea%37
\label{ex:7:37}
\langinfo{Western Pantar}{}{Holton, Western Pantar corpus} \\
\gll Manne   ga{\ng}  a-wake  si{\ng}  usi{\ng}  ga-r  halli  wa  im-mia  \textbf{dau}. \\
   woman  \textsc{3sg} 4-child  this    cradle  3-with  cry  go.\textsc{level}   \textsc{prog}{}-go.\textsc{high.steep} \textsc {high   } \\
\glt  `His wife cradled her child while crying over him going back up there.'
\z


 

\subsection{Kamang\ilt{Kamang}}
Kamang\il{Kamang} elevation\is{elevation} terms are given in Table~\ref{tab:7:kamang}. The Kamang\il{Kamang} elevational\is{elevation} paradigms have more terms than most other AP languages due to the presence of two additional semantic components in the \textsc{high} and \textsc{low} domains, namely, direction\is{direction} and distance. Direction\is{direction} has to do with the angle of the path taken or referenced location relative to the angle of the slope. Using a \textsc{direct} elevation\is{elevation} term means that the path taken follows the angle of the slope directly (i.e., at its steepest), whilst an \textsc{indirect} elevation\is{elevation} term means that the path traverses across the angle of the slope or that the referenced location is off to the side of angle of the slope. Distance is only marked in the \textsc{indirect} domain, and is concerned with whether the path taken is short or long or the referenced location is near or far. Thus, using a \textsc{near} elevation\is{elevation} term means traversing across a slope for a short distance, while using a \textsc{far} one traversing across a slope for a long distance.

 

\begin{table}


\begin{tabular}{>{\sc}l>{\sc}l>{\sc}l>{\it}l>{\it}l>{\it}l}
\mytopline
        &  &  & \rm Elevationals\ist{elevation} & \multicolumn{2}{c}{\rm Elevational\ist{elevation} motion\ist{motion} verbs}\\ 
 &  &  &              &\rm  From \textsc{dc} &  \rm To \textsc{dc}  \\
\midrule 
{level} &         &        & {mu{\ng}}   & we & me \\
\\
\multirow{3}{*}{high} & {direct} &        & {tu{\ng}}   & te & taa{\ng}\\
       & \multirow{2}{*}{indirect}& {near}& {mutu{\ng}} & {wete} & metaa{\ng}\\
       &           & {far} & {tumu{\ng}} & {tewe} & {taa{\ng}me}\\
\\
\multirow{3}{*}{low} &   {direct} &       & {fu{\ng}}   & fe & yaa{\ng}\\
      & \multirow{2}{*}{indirect} & {near} & {muhu{\ng}} & {wehe} & \multirow{2}{*}{yaa{\ng}me}\\
      &            &  {far} & {fumu{\ng}} & {fewe} & \\


\mybottomline
\end{tabular}

\caption{Kamang\ilt{Kamang} elevation\ist{elevation} terms}
\label{tab:7:kamang}
\end{table}

Kamang\il{Kamang} elevationals\is{elevation} occur adverbially, directly before a predicate or a predicate and its object. For instance, in \REF{ex:7:38} \textit{mutu{\ng}} denotes the location from which the calling takes place, and in \REF{ex:7:39} \textit{tu{\ng}} gives the location on the slope where the stumbling takes place. An elevational\is{elevation} may also occur following a motion\is{motion} verb specifying the resultant location of the motion\is{motion}, as in \REF{ex:7:40} where the elevational\is{elevation} \textit{tu{\ng}} follows its corresponding elevational\is{elevation} verb \textit{te}.



\ea%38
\label{ex:7:38}
\langinfo{Kamang}{}{\citealt[306]{Schapperta}} \\
\gll   Nok   sue  koo \textbf{mutu{\ng}}    wo-iti-si.\\
  one  come  stay  \textsc{level} \textsc {3.loc}{}-call-\textsc{ipfv}  \\
\glt  `Somebody was calling him from over there.'
\z



\ea%39
\label{ex:7:39}
\langinfo{Kamang}{}{\citealt[306]{Schapperta}} \\
\gll Markus  \textbf{tu{\ng}} wuleh  sama  kawaila-ma.\\
 Markus  \textsc{high.drct} slope  middle  stumble-\textsc{pfv}  \\
\glt `Markus stumbled on the slope up (which is) up there.'
\z

  

  

  

 

\ea%40
\label{ex:7:40}
\langinfo{Kamang}{}{\citealt[306]{Schapperta}} \\
\gll Nal   te  \textbf{tu{\ng}}.\\
 \textsc{1sg} go.\textsc{high.drct} \textsc {high.drct}   \\
\glt  `I go up top.'
\z

  

 

 

\subsection{Summary}\label{sec:7:3.8}
AP languages invariably have elevation\is{elevation} marking in a set of non-verbal elevationals\is{elevation} and in a paradigm of elevational\is{elevation} verbs. In the preceding sections, we have seen some of the variety that elevational\is{elevation} systems contain. 

AP languages vary significantly in the number of elevation\is{elevation} terms, the number of paradigms over which they occur and the extra semantic components that are added within the three elevational\is{elevation} heights (summarised in Table~\ref{tab:7:summary}). 

 


\begin{table}[ht]
\caption{Overview of elaboration of elevational\ist{elevation} systems in AP languages}
\label{tab:7:summary}
\begin{tabular}{rrrr}
\mytopline
 & 
 \parbox{2cm}{\flushright elevation\\ marked \\forms} & 
 \parbox{2cm}{\flushright paradigms\\ with\\ elevationals\ist{elevation}} & 
 \parbox{2cm}{\flushright extra\\ semantic\\ features}\\
\midrule 
Wersing\ilt{Wersing} & 9& 2& 0\\ 
Teiwa\ilt{Teiwa} & 9& 2& 0\\ 
Abui\ilt{Abui} & 11& 2& 1\\ 
Blagar\ilt{Blagar} & 32& 10& 0\\ 
Adang\ilt{Adang} & 22& 4& 3\\ 
Western Pantar\ilt{Western Pantar} & 26& 8& 1\\ 
Kamang\ilt{Kamang} & 20& 2& 2\\ 
\mybottomline
\end{tabular}
\end{table}

Minimally, AP languages have 9 elevation\is{elevation} terms, with three elevationals\is{elevation} and six elevational\is{elevation} motion\is{motion} verbs distinguishing three elevations\is{elevation}. A much higher number of terms are found in languages such as Blagar\il{Blagar}, Adang\il{Adang} and Western Pantar\il{Western Pantar}, which have elevation\is{elevation} marking morphology reiterated over multiple paradigms of different word classes, including in particular demonstratives\is{demonstrative} (one extra paradigm in Adang\il{Adang}, two in Blagar\il{Blagar} and four in Western Pantar\il{Western Pantar}), verbs (six extra paradigms in Blagar\il{Blagar}) and adverbs (one extra paradigm in Blagar\il{Blagar}).

The number of elevation\is{elevation} marked terms has also been increased by adding semantic distinctions within the three elevational\is{elevation} heights. Adang\il{Adang} has the greatest number of semantic elaborations, with geophysical, vertical and directional\is{direction} terms being added in the elevationals\is{elevation} to the standard global elevationals\is{elevation}. Kamang\il{Kamang} adds two new semantic components to its elevation\is{elevation} marked terms, directionality\is{direction} and distance. Western Pantar\il{Western Pantar} and Abui\il{Abui} add one extra semantic distinction, steepness and distance respectively. 

Added semantic components are typically limited to either particular elevational\is{elevation} domains or to particular paradigms of elevation-marked\is{elevation} terms. Table~\ref{tab:7:elevdomain} presents an overview of the distribution of these across AP languages. A cell with `1' represents a domain without semantic elaboration, whilst higher numerals (bolded) indicate the presence of semantic elaborations. 

We see that it is not typical to elaborate in the \textsc{level} domain. Only Adang\il{Adang} has more than one \textsc{level} term in its elevationals\is{elevation}, due to the regular derivation of directional\is{direction} elevationals\is{elevation} from elevation\is{elevation} marked verbs (\textit{fal}\textit{{\textepsilon}} {\textless} \textit{fa} `go\textsc{.level',} \textit{mal}\textit{{\textepsilon}} {\textless} \textit{ma}  `come\textsc{.level'}). All other languages restrict their elaborations to the \textsc{high} and \textsc{low} domains. Semantic elaborations are typically also limited to one paradigm and are not elaborated over all paradigms. Abui\il{Abui} and Adang\il{Adang} limit their extra distinctions to elevationals\is{elevation}, while Western Pantar\il{Western Pantar} limits it to elevational\is{elevation} motion\is{motion} verbs denoting movement away from the \textsc{dc}. Kamang\il{Kamang} is unusual in that it has almost the same semantic elaborations in both its elevationals\is{elevation} and elevational\is{elevation} verbs. Asymmetries in the number of extra distinctions are present in Adang\il{Adang} and Kamang\il{Kamang}, while Abui\il{Abui} and Western Pantar\il{Western Pantar} apply the semantic elaboration to all parts of the paradigm.
 

\begin{table}
 
\begin{tabular*}{\textwidth}{@{\extracolsep{\fill}}p{1.4cm}rclrclrcl}
\mytopline
 & \multicolumn{3}{c}{Elevationals\ist{elevation}} & \multicolumn{6}{c}{Elevational\ist{elevation} motion\is{motion} verbs} \\
		&       &      &     &     \multicolumn{3}{c}{From \textsc{dc}} &  \multicolumn{3}{c}{To \textsc{dc}} \\
		& \textsc{level} & \textsc{high} & \textsc{low} & \textsc{level} & \textsc{high} & \textsc{low} & \textsc{level} & \textsc{high} & \textsc{low}\\
\midrule 
Wersing\ilt{Wersing} 		& 1 & 1 & 1 & 1 & 1 & 1 & 1 & 1 & 1 \\ 
Teiwa\ilt{Teiwa} 		& 1 & 1 & 1 & 1 & 1 & 1 & 1 & 1 & 1 \\ 
Abui\ilt{Abui} 		& 1 & \textbf{2} & \textbf{2} & 1 & 1 & 1 & 1 & 1 & 1 \\
Blagar\ilt{Blagar} 		& 1 & 1 & 1 & 1 & 1 & 1 & 1 & 1 & 1 \\
Adang\ilt{Adang} 		& \textbf{3} & \textbf{6} & \textbf{4} & 1 & 1 & 1 & 1 & 1 & 1 \\
\multirow{2}{*}{\parbox{1.4cm}{Western  Pantar\ilt{Western Pantar}}} 	& \multirow{2}{*}{1} & \multirow{2}{*}{1} & \multirow{2}{*}{1} & \multirow{2}{*}{1} & \multirow{2}{*}{\textbf{2}} & \multirow{2}{*}{\textbf{2}} & \multirow{2}{*}{1} & \multirow{2}{*}{1} & \multirow{2}{*}{1}  \\
\\
Kamang\ilt{Kamang} 		& 1 & \textbf{3} & \textbf{3} & 1 & \textbf{3} & \textbf{3} & 1 & \textbf{3} & \textbf{2}  \\

\mybottomline
\end{tabular*}
\caption{Number of elevation-marked\ist{elevation} terms by elevational\ist{elevation} domain and word class}
\label{tab:7:elevdomain}
\end{table}

The syntax of elevation-marked\is{elevation} terms also shows variation between languages. Focusing on the elevationals\is{elevation} (or ``\textsc{elevated} locationals'', items referring to a location at a specified elevation\is{elevation}), we observed a range of syntactic differences from one language to the next. In Table~\ref{tab:7:syntaxAP}, I summarise the ability of AP elevationals\is{elevation} to occur predicatively, adverbially and within the NP.  

 


\begin{table}\centering
\begin{tabular*}{\textwidth}{@{\extracolsep{\fill}}rccccc}
\mytopline
	&	& \multicolumn{2}{c}{Adverbial} & \multicolumn{2}{c}{(Ad-)Nominal}\\  
	& predicative & medial &  final & \parbox{1cm}{with noun head} & \parbox{1cm}{without noun head} \\
\midrule 
Wersing\ilt{Wersing} & yes & yes & no & no & no\\
Teiwa\ilt{Teiwa} & yes & yes & yes & no & no\\
Abui\ilt{Abui} & yes & yes & no & yes & yes\\
Blagar\ilt{Blagar} & no & yes & yes & no & no\\
Adang\ilt{Adang} & yes & yes & no & yes & no\\
Western Pantar\ilt{Western Pantar} & yes & yes & yes & no & no\\
Kamang\ilt{Kamang} & no & yes & yes & no & no\\
\mybottomline
\end{tabular*}
\caption{Overview of syntax of elevationals\ist{elevation} in AP languages}
\label{tab:7:syntaxAP}
\end{table}

In all but two languages (Kamang\il{Kamang} and Blagar\il{Blagar}), elevationals\is{elevation} occur as independent clausal predicates indicating the elevation\is{elevation} at which the subject was to be located. Blagar\il{Blagar} does not allow elevationals\is{elevation} predicatively, and instead has a derived paradigm of stative elevational\is{elevation} verbs which fulfill the same function as predicative elevationals\is{elevation} in other AP languages. 

All languages allow their elevationals\is{elevation} to occur clause-medially, when adverbial. However, only four languages (Teiwa\il{Teiwa}, Blagar\il{Blagar}, Western Pantar\il{Western Pantar} and Kamang\il{Kamang}) allow elevationals\is{elevation} to occur clause-finally. Yet, even where the clausal position was the same, there were, however, differences from language to language in the function and constituency of elevationals\is{elevation} in adverbial use. The most common clause-medial function of an elevational\is{elevation} was to mark that the situation or event denoted by the following predicate took place at a certain elevation\is{elevation}. This was found for Wersing\il{Wersing}, Teiwa\il{Teiwa}, Abui\il{Abui}, Adang\il{Adang} and Kamang\il{Kamang} clause-medial adverbial elevationals\is{elevation}, but not in Blagar\il{Blagar} and Western Pantar\il{Western Pantar}. In Blagar\il{Blagar} the choice of clausal position of an elevational\is{elevation} reflected not spatial but epistemic differences, with clause-medial position signalling certainty on the part of the speaker and clause-final position signalling epistemic accessibility of knowledge of the event to the addressee. In Kamang\il{Kamang}, by contrast, the clausal position of an elevational\is{elevation} reflects a different kind of location: clause-medially an elevational\is{elevation} denotes the location at which the following predicate take places, whereas clause-finally an elevational\is{elevation} denotes a location resulting from the predicate. In Western Pantar\il{Western Pantar}, making a clause-final versus clause{}-medial distinction is misleading because the constituency of an elevational\is{elevation} is the same in both positions: Western Pantar\il{Western Pantar} elevationals\is{elevation} follow the element whose location they denote, medially these are NPs and finally these are verbs. 

In the nominal domain, we also observed variation in how individual languages could use elevationals\is{elevation}. All but Abui\il{Abui} and Adang\il{Adang} did not allow elevationals\is{elevation} to occur in the NP. Abui\il{Abui} allowed elevationals\is{elevation} not only to occur within an NP alongside a head noun, but also to head the NP itself, while Adang\il{Adang} only allowed elevationals\is{elevation} to occur within a head noun.

In short, elevation-marking\is{elevation} in AP languages is characterised by diversity not only in the sheer number of terms that systems contain, but also in the semantic components and syntactic behaviour of those terms. 

\section{History of AP elevation\ist{elevation} terms}\label{sec:7:4}
\begin{table}[b]
\centering


\begin{tabular}{>{\sc}l>{\it}l>{\it}c>{\it}c}
\mytopline
               & \rm Elevationals & \multicolumn{2}{c}{\rm Elevational\ist{elevation} motion\is{motion} verbs} \\
               &  &\rm  From \textsc{dc}& \rm To \textsc{dc}\\ 
\midrule 
{level}       &  *mo & *wai & *mai\\
 {high}       & *(d,t)o &  *mid(a) & *medai({\ng})\\
\multirow{2}{*}{low}        & *po  & \multirow{2}{*}{*pia} & *seri\\
              &*yo   &       &  *ya({\ng})\\
\mybottomline
\end{tabular}

\caption {pAP\ilt{proto-Alor-Pantar} elevation\ist{elevation} terms}
\label{tab:7:pap}
\end{table}


Thus far our explorations of AP elevational\is{elevation} systems have been synchronic, describing the internal structures of the systems one language at a time. Today, even if the majority of elevational\is{elevation} systems in AP languages are little explored, the quantity and quality of existing information is sufficient for the formulation of historical hypotheses about the elevational\is{elevation} system of their common ancestor, proto-AP\il{proto-Alor-Pantar} (pAP).



In Table~\ref{tab:7:pap}, I present the reconstructable elevational\is{elevation} forms of pAP\il{proto-Alor-Pantar}. These reconstructions are made by comparing the terms in the systems found in modern AP languages. The one peculiarity of this reconstructed system is that the \textsc{low} elevational\is{elevation} domain has two competing reconstructions in the elevational\is{elevation} particles (*po versus *yo) and in the elevational\is{elevation} verbs denoting motion\is{motion} towards the \textsc{dc} (*seri versus *ya({\ng})). The evidence for these will be discussed in subsequent sections.



 


In Sections \ref{sec:7:4.1} and \ref{sec:7:4.2}, I look at the evidence for the different forms in the reconstructed proto-paradigms of elevationals\is{elevation}  and elevational\is{elevation} verbs respectively. Finally, in {\S} \ref{sec:7:4.3}, I consider the mechanisms by which the proto-system has been complicated and additional distinctions have been built up. In the following subsections, I draw on data not only from the seven languages already discussed in {\S} \ref{sec:7:3}, but also from an additional four languages, Kaera\il{Kaera}, Klon\il{Klon}, Kui\il{Kui}, and Sawila\il{Sawila}. In these languages, individual basic elevation\is{elevation} terms are known but the semantics and morpho-syntax of the elevation\is{elevation} system are not fully understood or described.\footnote{The following language abbreviations are used in tables in subsequent sections: \textsc{Tw} Teiwa\il{Teiwa}, \textsc{Ke} Kaera\il{Kaera}, \textsc{WP} Western Pantar\il{Western Pantar}, \textsc{Bl} Blagar\il{Blagar}, \textsc{Ad} Adang\il{Adang}, \textsc{Kl} Klon\il{Klon}, \textsc{Ki} Kui\il{Kui},\textsc Ab  Abui\il{Abui}, \textsc{Km} Kamang\il{Kamang}, \textsc{Sw} Sawila\il{Sawila}, and \textsc{We} Wersing\il{Wersing}.} 

\subsection{Proto-elevationals\ist{elevation}}\label{sec:7:4.1}
Table~\ref{tab:7:reflexes} presents pAP elevationals\is{elevation} and their reflexes in the eleven modern AP languages for which we have data. Bolding in the table selects the cognate parts of the modern reflexes.

 


\begin{table}\centering

\begin{tabular}{l>{\it}ll>{\it}l}
\mytopline
 &  \rm p\textsc{AP} & Lg & \rm Reflexes\\
\midrule 
{\scshape level} &
 *mo &
      \textsc{WP\ilt{Western Pantar}} & \textbf{{mau}}\\ 
 & &    \textsc{Bl\ilt{Blagar}} & \textbf{{mo}}\\
 & &    \textsc{Ad\ilt{Adang}} & \textbf{{m}}\textbf{{{\textopeno}}}{{\ng}}\\ 
 & &    \textsc{Km\ilt{Kamang}} & \textbf{{mu}}{{\ng}}\\ 
 & &    \textsc{We\ilt{Wersing}} & \textbf{{mo}}{na}\\
 & &    \textsc{Sw\ilt{Sawila}} & \textbf{{ma}}{na}  \\
{\scshape high} &
 *(d,t)o &
	\textsc{WP\ilt{Western Pantar}} & \textbf{{dau}}\\
&&	\textsc{Ke\ilt{Kaera}} & \textbf{{de}}\\
&&	\textsc{Bl\ilt{Blagar}} & \textbf{{do}}\\
&&	\textsc{Ad\ilt{Adang}} & \textbf{{t}}\textbf{{{\textopeno}}}{{\ng}}\\
&&	\textsc{Kl\ilt{Klon}} & \textbf{{ta}}\\
&&	\textsc{Km\ilt{Kamang}} & \textbf{{tu}}{{\ng}}\\
&&	\textsc{We\ilt{Wersing}} & \textbf{{to}}{na}\\
&&	\textsc{Sw\ilt{Sawila}} & \textbf{{ta}}{na}  \\
	{\scshape low} &
 *po &
      \textsc{WP\ilt{Western Pantar}} & \textbf{{pau}}\\
& &     \textsc{Ke\ilt{Kaera}} & \textbf{{pe}}\\
&  &    \textsc{Bl\ilt{Blagar}} & \textbf{{po}}\\
&   &   \textsc{Ad\ilt{Adang}} & \textbf{{p}}\textbf{{{\textopeno}}}{{\ng}}\\
&    &  \textsc{Km\ilt{Kamang}} & \textbf{{fu}}{{\ng}}\\
& *yo &
     \textsc{Tw\ilt{Teiwa}} & \textbf{{ya}}{qai}\\
& &     \textsc{Ki\ilt{Kui}} & {i}\textbf{{yo}}\\
&  &    \textsc{We\ilt{Wersing}} & \textbf{{yo}}{na}\\
&   &   \textsc{Sw\ilt{Sawila}} & \textbf{{ya}}{na}\\
\mybottomline
\end{tabular}

\caption{Reflexes of pAP\ilt{proto-Alor-Pantar} elevationals\ist{elevation}}
\label{tab:7:reflexes}
\end{table}


Reflexes of all four morphemes are found in non-contiguous areas of both Alor and Pantar. The distribution also does not conform to any known subgroups of the AP languages, thus justifying the reconstruction of the four morphemes to the highest level, pAP\il{proto-Alor-Pantar}.

 We see from Table~\ref{tab:7:reflexes} that ``bare'', that is unaffixed, reflexes of the proto-elevationals\is{elevation} are found in Western Pantar\il{Western Pantar}, Kaera\il{Kaera} and Blagar\il{Blagar}. In West Pantar, Blagar\il{Blagar} and  Adang\il{Adang}, these morphemes are found across multiple paradigms of different word classes. Notably, several modern AP languages have reflexes suffixed with a nasal segment. This, I suggest, traces back to an enclitic postposition, pAP\il{proto-Alor-Pantar} *={\ng} `\textsc{loc}'.\footnote{I give this morpheme its phonetic rather than phonemic value for ease of explication. It seems likely that, as in many modern AP languages, in pAP\il{proto-Alor-Pantar} the velar nasal was a word-final allophone of pAP\il{proto-Alor-Pantar} *n.}  Abui\il{Abui} reflects the proto-morpheme as \textit{=}\textit{{\ng}} `\textsc{loc}' (see example \ref{ex:7:12}), \todo{check if this is the right example} an enclitic postposition closely resembling the probable original function of *={\ng}. In other AP languages, *={\ng} is preserved fused onto a range of location signifying words. Many AP languages have postpositions marked with *={\ng}, for instance: on Blagar\il{Blagar} \textit{ta}\textit{{\ng}} `on top of', but not on Kamang\il{Kamang} \textit{taa} and Abui\il{Abui} \textit{taha}, possibly {\textless} *tas `stand', or on Wersing\il{Wersing} \textit{ming} `in', but not on Kamang\il{Kamang} \textit{mi}, Klon\il{Klon} \textit{mi} and many more {\textless} *mi `in(side)'. 

The four languages for which we have reflexes of proto-elevationals\is{elevation} marked with *={\ng} `\textsc{loc}' are Adang\il{Adang}, Kamang\il{Kamang}, Wersing\il{Wersing} and Sawila\il{Sawila}. In the latter three the morpheme is fused on, whilst in Adang\il{Adang} reflexes of *={\ng} only occur on one paradigm and the basic elevational\is{elevation} forms combine with other affixes in other paradigms (e.g., \textit{h}\textit{{\textepsilon}}\textit{{}-} in the demonstratives\is{demonstrative}, or \textit{{}-l}\textit{{\textepsilon}} in directional\is{direction} elevationals\is{elevation}). In East Alor languages, Wersing\il{Wersing} and Sawila\il{Sawila}, the forms have further fossilised suffixed with \textit{{}-a}, a morpheme of unknown significance at this stage.\footnote{Wersing\il{Wersing} has an enclitic article \textit{=a} `\textsc{art}' which marks NPs for specificity, and a suffix \textit{{}-a} which marks realis mood on verbs. Note there is some evidence for the existence of elevationals\is{elevation} in Wersing without \textit{{}-a}. In Schapper and Hendery, Wersing\il{Wersing} corpus., there are two instances of \textit{yo{\ng}} that were said by an informant to have the same meaning as \textit{yona}.} It appears that *={\ng} was used originally on the elevationals\is{elevation} to make them into locative predicates. This is seen in that, whilst Blagar\il{Blagar} and Western Pantar\il{Western Pantar} cannot use their ``bare'' elevationals\is{elevation} as predicates, the elevationals\is{elevation} marked with *={\ng} as in Adang\il{Adang},  Wersing\il{Wersing} and Sawila\il{Sawila} can be predicates. From there, *={\ng} would have become fixed on the elevationals\is{elevation}, even in adverbial function where it would not have been needed originally in pAP\il{proto-Alor-Pantar}, as is suggested by the adverbial use of ``bare'' elevationals\is{elevation} in Blagar\il{Blagar} and Western Pantar\il{Western Pantar}.

\subsection{Proto-elevational\ist{elevation} verbs}\label{sec:7:4.2}
Table~\ref{tab:7:reflexes2} presents pAP\il{proto-Alor-Pantar} elevational\is{elevation} verbs and their reflexes in the 12 modern AP languages for which we have data. Differences between the reconstructed meaning and the modern meaning of the verbs are given below the table. 

The reconstruction of the paradigm with proto-forms of the verbs in the \textsc{level} and \textsc{high} domains is robust and well-supported. Reflexes of these are found throughout the Alor-Pantar area with consistent form-meaning pairings. Some small irregularity is observed in the sound correspondences of reflexes, particularly amongst the reflexes of *medai({\ng}) `come.\textsc{low}'. Teiwa\il{Teiwa} \textit{daa}, Kaera\il{Kaera} and Blagar\il{Blagar} \textit{da} and Wersing\il{Wersing} \textit{dai} all show loss of the initial syllable of *medai({\ng}). It is likely that the initial syllable of the verb was unstressed (i.e., *me{\textprimstress}dai({\ng})), as is often found in Alor-Pantar roots made up of a light-heavy syllable series. Historical loss of initial unstressed syllables has been observed repeatedly in AP languages \citep[93, 111]{HoltonEtAl2012}.

 

\begin{table}\centering
\caption[Reflexes of pAP\ilt{proto-Alor-Pantar} elevational verbs]{Reflexes of pAP\ilt{proto-Alor-Pantar} elevational\ist{elevation} verbs.}

\label{tab:7:reflexes2}
\begin{tabular}{>{\sc}l>{\sc}l>{\it}l>{\sc}l>{\it}l>{\sc}l>{\it}l}
\mytopline
{level} & \multicolumn{3}{l}{\it *wai} \\
& {WP\ilt{Western Pantar}}& {wa} & {Ad\ilt{Adang}}& {fa} & {Km\ilt{Kamang}}& {we}\\ 
& {Tw\ilt{Teiwa}}& {wa} & {Kl\ilt{Klon}}& {wa} & {We\ilt{Wersing}}& {wai}\\ 
& {Ke\ilt{Kaera}}& {wa} & {Ki\ilt{Kui}}& {bai} & {Sw\ilt{Sawila}}& {we}\\ 
& {Bl\ilt{Blagar}}& {va} & {Ab\ilt{Abui}}& {we} \\ 
 & \multicolumn{3}{l}{\it *mai} \\
& {WP\ilt{Western Pantar}} &{ma}& {Ad\ilt{Adang}}& {ma}& {We\ilt{Wersing}}& {mai}\\
& {Tw\ilt{Teiwa}}& {ma}& {Kl\ilt{Klon}}& {ma}& {Ki\ilt{Kui}}& {mai}\\
& {Ke\ilt{Kaera}}& {ma}& {Ab\ilt{Abui}}& {me}\\ 
& {Bl\ilt{Blagar}}& {ma}& {Km\ilt{Kamang}}& {me}\\ 
{high} & \multicolumn{3}{l}{\it *mid(a)} \\
& {WP}& {mia} & {Ad\ilt{Adang}}& {mid}& {We\ilt{Wersing}}& {mid}\\ 
& {Tw}& {mir}& {Kl}& {mid}& {Sw}& {mide}\\ 
& {Ke}& {mid}& {Ki}& {mira}\\ 
& {Bl}& {mida}& {Ab}&{marei}\\ 
 & \multicolumn{3}{l}{\it *medai({\ng})} \\
& {WP}& {midda}{{\ng}} & {Ad}& {mad{\textopeno}{\ng}} & {We}& {dai}\\ 
& {Tw}& {daa} & {Kl}& {mde} & {Sw}& {made}\\
& {Ke}& {da} & {Ki}& {maran}\\ 
& {Bl}& {da} & {Ab}& {mara}{{\ng}}\\
{low} & \multicolumn{3}{l}{\it *pia} \\
& {WP}& {pia} & {Ad}& {ip} & {Ab}& {pa}\\
& {Ke}& {ip} & {Kl}& {ip}& {Km}& {fe}\\
& {Bl}& {{\textglotstop}ipa} & {Ki}& {pa}{\dag}\\
 & \multicolumn{3}{l}{\it *seri} \\
& {Ad}& {h}{{\textepsilon}}{l} & {Ki}& {sei} & {We}& {sir} \\
& {Kl}& {her} & {Ab}& {sei}& {Sw}& {sire} \\ 
 & \multicolumn{3}{l}{\it *ya({\ng})} \\
& {WP}& {ya}{{\ng}} &{Bl}& {ya} & {Sw}& {yaa}{\ddag}\\
& {Tw}& {yaa} & {Km}& {yaa}{{\ng}} \\
& {Ke}& {ya} & {We}& {a}{\ddag}\\ 
\mybottomline
\end{tabular}

{\dag} This term in Kui\ilt{Kui} has shifted meaning in Kui to mean `go west', instead of `go.\textsc{low}'. This new meaning makes sense as a conventionalisation due to the local geography whereby west Alor is significantly less mountainous and overall at a much lower elevation\is{elevation} than east Alor, as per \citet{Windschuttel2013}.

{\ddag} Means `go.\textsc{low}' instead of expected `come.\textsc{low}'.
\end{table}

The reconstruction of proto-forms of elevational\is{elevation} verbs in the \textsc{low} domain is more complex due to the existence of two competing `come.\textsc{low}' forms, *seri and *ya({\ng}). The majority of AP languages have a reflex of only one of these two. Typically, Pantar languages have reflexes of *ya({\ng}) for `come.\textsc{low}', while west Alor languages have reflexes of *seri for `come.\textsc{low}'. Only east Alor languages have reflexes of both, with a reflex of *seri for `come.\textsc{low}' and a reflex of *ya({\ng}) for `go.\textsc{low}', while no reflexes of *pia are found, as would be expected for `go.\textsc{low}'.\footnote{East Alor forms a well-defined low-level subgroup and it is reasonable to assume that this shared characteristic among the languages goes back to their common ancestor, proto-East Alor\ilt{proto-East Alor}.} At this stage, both *ya({\ng}) and *seri are reconstructed to pAP\il{proto-Alor-Pantar}, because evidence for reconstructing one over the other is thin. The slightly wider distribution of reflexes of *ya({\ng}) might be taken to indicate that this was the earlier term, and that *seri was introduced into the elevational\is{elevation} verb paradigm soon after the breakup of the proto-language. One potential source for this introduction would be verbs such as Kamang\il{Kamang} \textit{sila}\textit{{\ng}} `descend', a verb which is not part of the elevation\is{elevation} paradigm proper as it is not anchored to a deictic\is{deixis} centre as elevational\is{elevation} verbs are. 

\subsection{Elaborations of the proto-system}\label{sec:7:4.3}
Having reconstructed the elevational\is{elevation} system of pAP\il{proto-Alor-Pantar}, we are now in a position to investigate changes to pAP\il{proto-Alor-Pantar} elevational\is{elevation} system and establish various developmental paths that have been taken by individual languages or groups of languages since the breakup of the pAP\il{proto-Alor-Pantar}. Note that I am concerned here not with adding further elevation-marked\is{elevation} terms to the set through innovative morphology (e.g., Adang\il{Adang} directional\is{direction} elevationals\is{elevation} marked with \textit{{}-}\textit{l}\textit{{\textepsilon}}), so much as with the processes by which distinctions within the elevational\is{elevation} system are elaborated. 

The first observation to be made is that the pAP\il{proto-Alor-Pantar} elevational\is{elevation} system has often altered where new elevation\is{elevation} terms (i.e., not reflecting the proto-terms) have emerged. Abui\il{Abui} elevationals\is{elevation} are an example of this, since reflexes of pAP\il{proto-Alor-Pantar} elevationals\is{elevation} are entirely absent in this language (see Table~\ref{tab:7:abui}). Abui\il{Abui} has innovated new terms with a tonal distinction between \textsc{high} and \textsc{low} elevations\is{elevation}, with a further distance contrast being added between near and far locations, the latter marked by /w/, the former by its absence. Western Pantar\il{Western Pantar} complicates its system of elevational\is{elevation} motion\is{motion} verbs towards the \textsc{dc} by incorporating the innovative verbs \textit{diaka{\ng}} and \textit{rau{\ng}} into the paradigm alongside \textit{mia} and \textit{pia}, reflexes of the pAP elevational\is{elevation} motion\is{motion} verbs *mid(a) `go.\textsc{high'} and *pia `go.\textsc{low'.} \textit{Diaka{\ng}} and \textit{rau{\ng}} have been incorporated into the paradigm for motion\is{motion} along gentle slopes, thereby causing the restriction of meaning of the inherited verbs to be for steeper slopes. Holton (p.c.) notes that for some speakers the innovative steep terms, \textit{diaka{\ng}} and \textit{rau{\ng}}, have even largely replaced the inherited gentle slope terms, \textit{mia} and \textit{pia}, in casual speech. 

The second mechanism of elaboration of sets of elevation-marked\is{elevation} terms is compounding basic terms together to create ``mediated'' distinctions. Consider the forms of the Sawila\il{Sawila} elevational\is{elevation} motion\is{motion} verbs presented in Table~\ref{tab:7:sawila}. 

 


\begin{table}
\caption{Sawila\ilt{Sawila} elevational\ist{elevation} motion\ist{motion} verbs (\citealt{Kratochvilta} and Kratochv\'il, Sawila corpus)}
\label{tab:7:sawila}
\centering
\begin{tabular}{>{\sc}l>{\sc}l>{\it}l>{\it}l>{\it}l}
\mytopline
 &  & \rm Elevationals\ist{elevation} & \multicolumn{2}{c}{\rm Elevational\ist{elevation} motion\ist{motion} verbs}\\
 &  &  & \rm From \textsc{dc} & \rm To \textsc{dc}\\
\midrule 
{level} &           & {mana}& {we} & me\\
\\
\multirow{3}{*}{high} & {direct}  & \multirow{3}{*}{anna}& {midde} & made\\
       & \multirow{2}{*}{indirect} &      & \multirow{2}{*}{waamide}& {mamade}\\
       &            &      &          & {madaame}  \\
       \\
\multirow{3}{*}{low} &  {direct}  & \multirow{3}{*}{yana}& yaa & sire \\ 
      & \multirow{2}{*}{indirect} &       & \multirow{2}{*}{wayaa} &  masire\\
      &            &       &         & {mayaa}\\
\mybottomline
\end{tabular}

\end{table}

In the \textsc{high} and \textsc{low} domains we see that there are not the expected two terms each, but instead five each. The direct terms denoting movement along an axis following the line of a slope straight up or straight down reflect individual pAP\ilt{proto-Alor-Pantar} elevation\is{elevation} terms. The indirect terms denote a movement that traverses across the slope diagonally and are formed by compounding different proto-terms together. The compounding process is not completely regular: there is some inconsistency in the terms that are compounded together in the verbs denoting motion\is{motion} toward the \textsc{dc}.\footnote{The difference between high indirect terms denoting motion\is{motion} towards the \textsc{dc} is not understood (Franti\v{s}ek Kratochv\'il, p.c.). As such I have not attempted to supply any more precise characterisation of these. Kula\il{Kula} has a similar system to Sawila\il{Sawila}, but the meanings of all compound elevational\is{elevation} terms are also not yet well understood (Nicholas Williams, p.c..).} Nevertheless, the etymologies for the terms are clear, as set out in Table\enlargethispage{3em}  \ref{tab:7:4142}.

\begin{table}
\begin{tabular}{ll@{\textless}l} 
\mytoprule
\textsc{high} domain  \\ 
\textsc{direct} 
&\textit{midde} & *mid(a) `go.{high'} \\
& \textit{made} & *medai({\ng}) `come.{high'} \\
\\
\textsc{indirect}
& \textit{waamidde} & *wai `go.{level'} + *mid(a) `go.{high'}\\
& \textit{mamade} & *mai `come.{level'} + *medai{\ng} `come.{high'}\\
& \textit{madaame} & *medai({\ng}) `come.{high'} + *mai `come.{level'}\\
\\
\midrule
\textsc{low} domain \\
\textsc{direct}
&\textit{yaa} & *ya({\ng}) `come.{low'} \\
& \textit{sire} & *sire `come.{low'} \\
\\
\textsc{indirect} 
& \textit{wayaa} & *wai `go.{level'} + *ya({\ng}) `come.{low'}\\
& \textit{masire} & *mai `come.{level'} + *sire `come.{low'}\\
& \textit{mayaa} & *mai `come.{level'} + *ya({\ng}) `come.{low'}\\
\mybottomrule
\end{tabular}
\caption{Sources of Sawila elevational motion verbs}
\label{tab:7:4142}
\end{table}

Kamang\il{Kamang} presents a more complex example of system elaboration, involving compounding of terms across all elevational\is{elevation} word classes not just verbs, as well as paradigm regularisation. Looking at the forms of Kamang\il{Kamang} elevation-marked\is{elevation} terms in Table~\ref{tab:7:kamang}, we see particular morphemic ``atoms'' are used to build up the elaborated terms in a semi-regular manner. ``\textsc{direct}'' terms are simplest, being built thus: (i) the elevational\is{elevation} domain is marked by a single consonant \textit{t-} for \textsc{high}, either \textit{f-} or \textit{y-} for \textsc{low} and either\textit m-  or \textit{w-} for \textsc{level}, and (ii) the word class is marked by \textit{{}-u-{\ng}} for elevationals\is{elevation}, by \textit{{}-e} for elevational\is{elevation} motion\is{motion} verb from \textsc{dc} and by \textit{{}-aa{\ng}} for elevational\is{elevation} motion\is{motion} verb from the \textsc{dc}. This pattern is perfectly illustrated by Kamang's\il{Kamang} \textsc{high} \textsc{direct} terms: \textit{tu{\ng}} `\textsc{high.drct'}, \textit{te} `go.\textsc{high.drct'} and \textit{taa{\ng}} `come.\textsc{high.drct'}. Of these, only \textit{tu{\ng}} is inherited from pAP, while \textit{te} and \textit{taa{\ng}} are Kamang\il{Kamang} innovations\is{innovation} following the pattern of morphemic atoms. 

Irregularities in the formation of non-compounded elevation\is{elevation} terms in Kamang\il{Kamang} stem from cases in which the morphemic atoms have not been fully applied (as explained further below), and instead there is retention of etymological forms. Table~\ref{tab:7:kamangncemt} presents an overview of the non-compounded elevation\is{elevation} terms in Kamang\il{Kamang}, followed by their expected but non-occurring form (marked with **) if they were formed on the morphemic atom pattern, and their relationship to pAP\il{proto-Alor-Pantar} terms.
 

\begin{table}[h]
\centering
\caption{Kamang\ilt{Kamang} non-compounded elevation-marked\ist{elevation} terms and their etymologies}
\label{tab:7:kamangncemt}
\begin{tabular}{>{\sc}lccc}
\mytopline
 & Elevationals\ist{elevation} & \multicolumn{2}{c}{Elevational\ist{elevation} motion\ist{motion} verbs}\\
      &     & From \textsc{dc} & To \textsc{dc} \\
\midrule 
\multirow{3}{*}{level} & \textit{muŋ} & we      & \textit{me} \\
      &     & **\textit{me}    & **\textit{maaŋ} \\
      &{\textless} pAP\ilt{proto-Alor-Pantar}  *mo-ŋ  & {\textless}  pAP  *wai & {\textless}  pAP  *mai \\
      \\
\multirow{3}{*}{high} & tuŋ & te & taaŋ\\
\\
     &{\textless} pAP\ilt{proto-Alor-Pantar} *(d,t)o-ŋ & {\textless} pAP *mid(a) & {\textless}  pAP *medai(ŋ)\\
     \\
\multirow{3}{*}{low} & \textit{fuŋ} & \textit{fe} & \textit{yaaŋ}\\
  &  &  &  **\textit{faaŋ}\\
  & {\textless} pAP\ilt{proto-Alor-Pantar}  *po-ŋ & {\textless} pAP  *pia & {\textless}  pAP  *yaa(ŋ)\\

\mybottomline
\end{tabular}

\end{table}

In the Table~\ref{tab:7:kamangncemt}, we see that the appearance of both \textit{m-} and \textit{w-} in the formation of \textsc{level} motion\is{motion} verbs is a result of the retention of reflexes of pAP\il{proto-Alor-Pantar} *wai `go.\textsc{level'} alongside *mai `come.\textsc{level'}. If the formation of these terms were to conform to the atomic pattern, we would find the forms **me and **maa{\ng} instead. In the \textsc{low} domain, \textit{fu{\ng}} and \textit{fe} are inherited terms that follow the morphemic atom pattern, while \textit{yaa{\ng}} is a retention of a reflex of pAP\il{proto-Alor-Pantar} *yaa({\ng}) that does not conform to the pattern expected when using the morphemic atoms.

These basic forms that are established by this set in Kamang\il{Kamang} are then compounded together to create complex indirect terms in the \textsc{high} and \textsc{low} domains. N\textsc{ear} \textsc{indirect} terms are built by prefixing the \textsc{level} morpheme onto the \textsc{direct} term of the corresponding word class, while \textsc{far} \textsc{indirect} terms are built by prefixing the \textsc{direct} morpheme onto the \textsc{level} morpheme of the corresponding word class. The composition of these terms is set out in Table~\ref{tab:7:ex42}. Also, in this set of compounds, we find irregularity: the expected form **meyaa{\ng} for `come.\textsc{low.indrct.near}'  does not appear, instead \textit{yaa{\ng}me} is used for near and far indirect motion\is{motion}. This gap in the Kamang\il{Kamang} paradigm shows that the elaboration of such systems is not as regular as we might anticipate for a process in which morphemes are so transparent. 



\begin{table}
\caption{Sources of Kamang \textsc{indirect}  elevation terms.}
\label{tab:7:ex42}
\begin{tabular}{ll@{$<$}lll}
\mytoprule
   \textsc{High} domain   \\ 
\midrule   
\textsc{near} \textsc{indirect} terms   \\
 {\scshape Elevational\ist{elevation}:} & {\itshape mutu{\ng}} & {\itshape mu} `{\scshape level}'+ {\itshape tu} `{\scshape high.drt}'+\textit{{\ng}}\\  Motion\ist{motion} verb from  \textsc{dc}& {\itshape wete} & {\itshape we} `go.\textsc{level}'+{\itshape te}  `go.\textsc{high.drt}'   \\
  Motion\ist{motion} verb to  \textsc{dc}& {\itshape metaa{\ng}} & {\itshape me}  `come\textsc{.level}'+\textit{taa}\textit{{\ng}}  `come.\textsc{high.drt}'  \\
\\
\textsc{Far} \textsc{indirect} terms  \\
  {\scshape Elevational\ist{elevation}:} & {\itshape tumung} & {\itshape tu} `{\scshape high.drt}'+{\itshape mu} `{\scshape level}'+\textit{{\ng}}\\
  Motion\ist{motion} verb from \textsc{dc}& {\itshape tewe} & {\itshape te} `go.\textsc{high.drt}'  {\itshape we} `go.\textsc{level}'   \\
  Motion\ist{motion} verb to \textsc{dc}& {\itshape taa{\ng}me} & \textit{taa}\textit{{\ng}}  `come.\textsc{high.drt}'+{\itshape me} `come'\textsc{.level}   \\ 
  \\
  \textsc{Low} domain   \\
\midrule   
  \textsc{near} \textsc{indirect} terms \\ 
  {\scshape Elevational\ist{elevation}:} & \textit{muhu}\textit{{\ng}} & {\itshape mu} `{\scshape level}'+{\itshape fu} `{\scshape high.drt}'+\textit{{\ng}}\\
  Motion\ist{motion} verb from \textsc{dc}& {\itshape wehe} & {\itshape we} `go.\textsc{level}'+ {\itshape fe} `go.\textsc{low.drt}'   \\
  Motion\ist{motion} verb to \textsc{dc}& {\itshape yaa{\ng}me} & \textit{yaa}\textit{{\ng}} `come\textsc{.low.drt}'+{\itshape me} `come\textsc{.level}'   \\
\\
\textsc{Far} \textsc{indirect} terms  \\
  {\scshape Elevational\ist{elevation}:} & \textit{fumu}\textit{{\ng}} & {\itshape fu} `{\scshape high.drt}'+{\itshape mu} `{\scshape level}'+\textit{{\ng}}\\
  Motion\ist{motion} verb from  \textsc{dc}& {\itshape fewe} & {\itshape fe} `go.\textsc{low.drt}'+ {\itshape we} `go.\textsc{level}'   \\
  Motion\ist{motion} verb to  \textsc{dc}& {\itshape yaa{\ng}me} & \textit{yaa}\textit{{\ng}} `come\textsc{.low.drt}'+{\itshape me} `come\textsc{.level}'\\ 
  \mybottomrule
\end{tabular}
\end{table}
 

In sum, AP languages have elaborated the inherited elevational\is{elevation} system by bringing innovative\is{innovation} new terms often alongside reflexes of terms from the proto-system and/or by combining reflexes of the original system together to create complex forms with ``mediated'' (i.e., \textsc{indirect} or diagonal directions\is{direction}) semantics.

\section{Conclusion}\label{sec:7:5}
All AP languages have rich systems of spatial deixis\is{deixis} with elevation\is{elevation} components. The languages show significant similarity in the basic, core system in which elevation\is{elevation} terms occur, namely, in both a verbal and non-verbal domain consistently contrasting \textsc{level,} \textsc{high,} and \textsc{low} elevations\is{elevation}. The shared characteristics of the systems can be traced back to a paradigm of elevationals\is{elevation} and a paradigm of elevational\is{elevation} motion\is{motion} verbs in the ancestral language, pAP\il{proto-Alor-Pantar}. Despite their common origin, modern AP elevational\is{elevation} systems display noteworthy differences in the number of terms, paradigms and semantic features they have. Individual languages have complicated the basic system by: (i) reiterating the elevational\is{elevation} distinction in multiple, additional domains (e.g., Blagar\il{Blagar}, Western Pantar\il{Western Pantar}), (ii) adding additional terms through innovative\is{innovation} morphology (e.g., Adang\il{Adang} \textit{l}\textit{{\textepsilon}}\textit{{}-} elevationals\is{elevation}), or (iii) compounding basic terms together to create more distinctions (e.g., Kamang\il{Kamang}, Sawila\il{Sawila}). The result is that the AP languages today display the kind of diversity in the details of their morphology, syntax and semantics of their elevational\is{elevation} systems that is typical of the family.

Typologically, the AP systems are remarkable for their complexity, which is much greater than that found in Papuan languages elsewhere for which deictic\is{deixis} systems with elevational\is{elevation} components have been described (see, e.g., \citealt{Heeschen1982,Heeschen1987}). Other Papuan languages only ever have three terms for the three elevational\is{elevation} heights and do not reiterate the elevational\is{elevation} distinctions across multiple parts of the lexicon. We might conjecture that the semantic elaborations of elevational\is{elevation} domains with features such as distance, steepness and directionality\is{direction} that we have observed in AP languages are rare cross-linguistically, and parallels remain to be identified in a world-wide survey of elevational\is{elevation} systems. 

The persistent occurrence of elevational\is{elevation} distinctions across word classes in AP languages can be usefully understood in terms of the preexisting concept of ``semplates'' \citep{LevinsonEtAl2009}. A semplate is defined as ``a configuration consisting of distinct sets or layers of lexemes, drawn from different semantic subdomains or different word classes, mapped onto the same abstract semantic template''  \citep[154]{LevinsonEtAl2009}. This fits well with the basic AP pattern in which locationals and motion\is{motion} verbs are organised by a semantic template differentiating the three elevational\is{elevation} domains. The interesting feature of AP elevational\is{elevation} semplates is their overtness in many instances: Adang\il{Adang}, Blagar\il{Blagar}, Western Pantar\il{Western Pantar} use the same morphemes to reiterate the elevational\is{elevation} semplate across word classes, while, as we saw in {\S} \ref{sec:7:4.2}, Kamang\il{Kamang} has in part discarded inherited lexemes and developed a system of morphemic atoms used to form complex subnodes in the elevational\is{elevation} semplate. Thus, the AP elevational\is{elevation} systems studied here not only present new evidence for the existence of \citeauthor{LevinsonEtAl2009}'s (2009) templates, but also have the potential to illuminate the diachronic processes by which abstract semplates may become productive and increasingly overt in their marking. 


\section*{Sources} 
\begin{tabular}{lp{8cm}}
Abui\ilt{Abui}  		&  Kratochv\'il Abui corpus, \citet{Kratochvil2007} \\			
Adang\ilt{Adang} 		&  \citet{Haan2001,RobinsonEtAltaadang} 	\\		
Blagar\ilt{Blagar}  		&  \citet{Steinhauer1977,Steinhauer1991,Steinhauer2012,Steinhauerta}, p.c. 	\\
Kaera\ilt{Kaera}  		&  \citet{Klamertakaera} 		\\				
Kamang\ilt{Kamang}  		&  \citet{Schapper2012,Schapperta}	\\				
Klon\ilt{Klon}  		&  \citet{Baird2008}, Baird Klon corpus 	\\				
Kui\ilt{Kui}  		&  \citet{Windschuttel2013}\\
Sawila\ilt{Sawila} 	 	&  \citet{Kratochvilta}, Sawila Toolbox dictionary, p.c.\\
Teiwa\ilt{Teiwa}  		&  \citet{Klamer2010grammar}, fieldnotes, Teiwa corpus\\
Wersing\ilt{Wersing}  	&  \citet{SchapperEtAltawersing}\\
Western Pantar\ilt{Western Pantar}  	&  \citet{Holton2007,Holton2011,Holtontawesternpantar}, Western Pantar corpus, p.c., \citet{HoltonEtAl2008}\\ 
\end{tabular}



\section*{Acknowledgments}
Thanks go to Hein Steinhauer, Gary Holton and Franti\v{s}ek Kratochv\'il for answering my many questions on the elevation\is{elevation} terms in the languages of their expertise. I would also like to thank Juliette Huber for the very useful discussions on elevation\is{elevation} terms in Makasae\il{Makasae} and Makalero\il{Makalero}, related languages spoken on Timor. Information on these languages does not appear in this chapter, but comparison with them informed some of the reconstructions made here. Any errors are, of course, my own.
 
\section*{Abbreviations}
\begin{tabular}{p{6cm}p{6cm}}
\begin{tabular}{>{\sc}p{1cm}p{4cm}}
2 & 2nd person\\
3 & 3rd person\\
4 &  4th person\\
Ab &  Abui\ilt{Abui}\\
Ad &  Adang\ilt{Adang}\\
addr & Addressee-anchored\\
AP & Alor-Pantar\\
art & Article\\
Bl &  Blagar\ilt{Blagar}\\
dc & Deictic\ist{deixis} Centre\\
dem & Demonstrative\ist{demonstrative}\\
dist & Distal\\ 
drct & Direct\\
high & refers to any location situated up(ward of) the deictic\ist{deixis} centre\\
indrct & Indirect\\
ipfv & Imperfective\\
Ke &  Kaera\ilt{Kaera}\\
Ki &  Kui\ilt{Kui}\\
Km &  Kamang\ilt{Kamang}\\
level & refers to any location situated level with the deictic\ist{deixis} centre\\
\\
\end{tabular}
&
\begin{tabular}{>{\sc}p{1cm}p{4cm}}
loc & Locative\\
low & refers to any location situated down(ward of) the deictic\ist{deixis} centre\\
NP & Noun phrase\\
nspec &  non-specific\\
nsteep &  non-steep\\
nvis & Non-visible\\
pAP & proto-Alor-Pantar\ilt{proto-Alor-Pantar}\\
pl & Plural\\
poss & Possessive\ist{possession}\\
prog & Progressive\\
prox & Proximal\\
real & Realis\\
rflx & Reflexive\\
sg & Singular\\
spec & Specifier\\
spkr &  Speaker-anchored\\
steep &  steep\\
subj & Subject\\
Sw &  Sawila\ilt{Sawila}\\
Tw &  Teiwa\ilt{Teiwa}\\
vis & Visible\\
We &  Wersing\ilt{Wersing}\\
WP & Western Pantar\ilt{Western Pantar}\\  
\end{tabular}
\end{tabular}