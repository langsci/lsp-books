
\addchap{Abbreviations and notational conventions}

\section*{Morphological glosses}

The following list includes only abbreviations for glossing of linguistic examples not defined by the Leipzig Glossing Rules.\footnote{\url{http://www.eva.mpg.de/lingua/resources/glossing-rules.php} 16.02.2014}

\newcommand{\TABh}{\hspace{50pt}}%sets spacing for first tab column in all tabbing environments below

\begin{tabbing}
\TABh \= \kill
\textsc{abess} \> abessive\\
\textsc{adjz} \> adjectivizer, adjectivization\\
\textsc{agr} \> (any kind of) agreement\\
\textsc{attr} \> or (attr.); attribution, attributive\\%!!check or.
\textsc{anr} \> action nominal\\
\textsc{compar} \> comparative (adjective derivation)\\
\textsc{contr} \> contrastive focus\\
\textsc{crs} \> currently relevant state\\
\textsc{deriv} \> derivative, derivation (unspecified)\\
\textsc{dim} \> diminutive\\
\textsc{ess} \> essive\\
\textsc{hum} \> human (gender)\\
\textsc{ill} \> illative\\
\textsc{infl} \> (any) inflexion\\
\textsc{mod} \> modification\\
\textsc{nar} \> narrative (case)\\
\textsc{nonfut} \> non-future\\
\textsc{nonhum} \> non-human (gender)\\
\textsc{pfct} \> perfective (verb derivation)\\
\textsc{pred} \> or (pred.); predication, predicative\\%!!check or.
\textsc{prepos} \> prepositional\\
\textsc{real} \> realis\\
\textsc{stat} \> stative (verb derivation)\\
\textsc{super} \> superlative\\
\textsc{utr} \> utrum, common (gender)\\
\end{tabbing}

\section*{Syntactic classes and phrase constituents}

\begin{tabbing}
\TABh \= \kill
{A} \> adjective\\
{AdP} \> adpositional phrase\\
{AP} \> adjective phrase\\
{ART} \> (attributive) article\\
{DEF} \> definite article\\
{Deg} \> degree word\\
{HEAD} \> phrase head\\
{INDEF} \> indefinite article\\
{N} \> noun\\
{NP} \> noun phrase\\
{PSD} \> possessed (head in possessive noun phrase)\\ 
{PSR} \> possessor (dependent in possessive noun phrase)\\
{Rel} \> relative clause\\
{V} \> verb\\
\end{tabbing}

\section*{Geographic abbreviations}

%$^a$In linguistic examples, the lowest genus of the mentioned language and the family it belongs to are indicated as in \textsc{German} (W-Germanic<Indoeuropean; own knowledge).

\begin{multicols}{2}
\begin{tabbing}
\TABh \= \kill
{C} \> Central\\
{E} \> East(ern)\\
{N} \> North(ern)\\
{NE} \> North-East(ern)\\
{NW} \> North-West(ern)\\
{S} \> South(ern)\\
{SE} \> South-East(ern)\\
{SW} \> South-West(ern)\\
{W} \> West(ern)\\
\end{tabbing}
\end{multicols}

\section*{Other symbols}

The following symbols are used for the illustration of linguistic changes.

\begin{tabbing}
long gloss \= \kill
$\leftarrow$  \> derivation or other synchronic process\\
$\Leftarrow$  \> grammaticalization\is{grammaticalization} or other diachronic process\\
<  \> borrowing\\
\end{tabbing}

Note that the term \emph{grammaticalization}\is{grammaticalization} is used for different types of linguistic changes leading to re-analysis of a given construction's grammatical meaning. A prototypical instance in this rather broad sense of grammaticalization\is{grammaticalization} is the morphologization of a formerly lexical morpheme to a grammatical (free or bound) morpheme, as the development of definite markers from anaphoric pronouns in Germanic\il{Germanic} languages, e.g.~English\il{English} \textit{the house} (\textit{the} $\Leftarrow$ Old English\il{Old English} \textit{þæt}) and Swedish\il{Swedish} \textit{hus-et} (\textit{-et} $\Leftarrow$ Old Norse\il{Old Norse} \textit{hið}).
