
\chapter{Typologizing noun phrase structure}

The goal of the following chapters is to typologize noun phrases and to present a comprehensive ontology of different syntactic, morpho-syntactic, and morpho-semantico-syntactic attribution marking devices attested in the languages of northern Eurasia and beyond. 

In order to illustrate the different noun phrase types to which these devices belong, data from several languages both within and outside the geographic area of investigation are taken into consideration. The focus, however, will be on constructions and features especially relevant to adjective attribution in the northern Eurasian area.

The term \emph{noun phrase type} used here denotes the specific syntactic or morpho-syntactic structure type of a noun phrase. This term is thus superordinate and belongs to noun phrase structure in general. Since the present study is restricted to a rather small subset of noun phrases, namely noun phrases with adjectival modifiers, the subordinated term \emph{adjective attribution marking device} (instead of \emph{adjective attribution marking type}) will be used to cover all grammatical operations which license the syntactic relation of adjective attribution.

\paragraph{Attribution marking} Minimally, an attribution marking device will simply license the syntactic structure without ranking single constituents, i.e.~without licensing any of the constituents as head or dependent. This is the case for the pure syntactic devices \emph{juxtaposition} and \emph{incorporation}.

The syntactic relation of attribution can also be licensed by a device linking the modifying and the modified constituents morphologically to each other, namely in the case of agreement marking. The morphological device of \emph{agreement marking} is characterized by the assignment of an inherent (i.e.~true morphological) feature from one constituent to another through morpho-syntactic government.

A different instance of “indirect” licensing of attribution is the marking of a semantic relation between the modifier and the modified, as with possessor case (genitive) marking.

It is not at all unusual that the syntactic, morphological, and/\-or semantic relations between noun phrase constituents are marked simultaneously. If, for instance, an attributive construct marker is attached to a modifier which additionally inflects for head-driven agreement, both the syntactic and the morphological relation between the noun phrase constituents are marked. Another example for simultaneously marked syntactic and semantic relations is a noun phrase with a case marked possessor noun (e.g.~in genitive case) and a head noun which is additionally marked for dependent-driven agreement (e.g.~with a cross-referencing possessive affix).

\paragraph{Typological parameters} Noun phrase types with formally distinct characteristics can be defined according to several parameters. Such parameters are, for example, the order of constituents inside the noun phrase (e.g., attribute-head order, head-attribute order, free order), the attribution marker's locus (e.g., on-head, on-dependent), the marker's behavior relative to the whole phrase (e.g.~clitic), its phonological fusion (e.g., free, bound, non-linear), or its position relative to the word host (e.g., pre, post, circum).\footnote{These parameters, adapted from Croft's (\citeyear[93–94]{croft1995}) typological classification of genitive constructions, are applied for a general typology of noun phrase structure in the noun phrase structure module of AUTOTYP (cf.~\citealt{AUTOTYP-NP}).}

Examples for a variety of phonologically, morphologically, syntactically, and semantically distinct types of attribution marking devices will be given in the following chapter. The focus of the ontology presented here is on morphological and morpho-syntactic parameters, especially with regard to the absence or presence of additional attribution marking morphemes, as well as to their kind and behavior. An overall picture of the ontology of attribution devices relevant to this study is given in Figure \ref{tree ontology} at the end of Chapter \ref{ontology}.

Noun phrase types can also be defined on a polyfunctionality scale with regard to the class of modifying elements: Attributive adjectives and other adnominal modifiers (demonstratives, bare nouns, noun phrases, adpositional phrases, clauses, etc.) may or may not occur in similar noun phrase types. The polyfunctionality parameter even takes the content of certain devices beyond attribution marking into consideration. Since the present study investigates adjective attribution marking, the polyfunctionality of attribution marking devices will be dealt with in less detail (see Chapter \ref{polyfunctionality}). 

\paragraph{How many noun phrase types does a language exhibit?} Most languages exhibit more than one distinct noun phrase type because different classes of attributed elements may occur in noun phrase structures which behave differently in their syntax or morpho-syntax. In English, for instance, adjectives and clauses are attributed by means of different devices: Whereas attributive clauses can be attributed by means of relativization (\textit{the dog \textbf{which is nice}}), adjectives are normally juxtaposed (\textit{the \textbf{nice} dog}). Since the present study is devoted to the morpho-syntax of one single class of adnominal modifiers, namely adjectives, variation in attribution marking devices across different classes of attributed elements is of minor importance. 

Nonetheless, attributed elements belonging to one and the same class may also occur in noun phrases which are marked differently: Possessive pronouns in English, for example, can be attributed either by means of juxtaposition (\textit{\textbf{her} dog}) or by using them in a prepositional construction (\textit{the dog \textbf{of hers}}). Even attributive adjectives may occur in two formally distinct noun phrase types. In Turkish, for instance, attributive adjectives are unmarked (\textit{\textbf{kara} kalem} ‘black pencil’); in headless noun phrases marked as direct objects, however, adjectives must be nominalized by means of the 3\textsuperscript{rd} person singular possessive suffix (\textit{\textbf{kara-sını}} ‘the black one (=pencil)’ [\textsc{poss:3sg.acc}]; see also below Section \ref{turkish synchr}). 

Prototypically, the use of different devices for licensing one and the same class of attributed elements is not arbitrary but governed by constraints. Nominalization of adjectives in Turkish, for instance, is due to a syntactic subset constraint affecting those phrases in direct object position and without a lexical head noun. In other languages, the occurrence of a given noun phrase type may also be constrained lexically and/\-or semantically by subsets of either attributes or heads. A well-known example beyond adjective attribution comes from languages in which the choice of possession marking devices is determined semantically by the animate or inanimate subset of the head noun (i.e.~the possessed). Even other subsets of head nouns are known to constrain the choice of possession marking in some languages, such as kinship terms, (non-) referential nouns, etc.

Similarly, languages may exhibit subset constraints on the semantic class of heads modified by adjectives. The epithet-construction marked with an attributive article in English (or other Germanic languages, cf.~\textit{Frederick \textbf{the Great}, Friedrich \textbf{der Große}}; see also below Section \ref{attr nmlz}) may serve as an example. In English, this special noun phrase type only occurs if the head noun belongs to the semantic subclass of proper nouns. 

Examples of a semantic subset of attributes governing a special attribution marking device are commonly found in languages with contrastive focus marking of adjectives. In Rumanian, for instance, adjective attribution marking is usually characterized by a noun phrase type with head-initial constituent order. A different noun phrase type, formally distinguished by the reversed order of constituents, occurs if the adjective bears contrastive focus (see the Rumanian example (\ref{rumanian wo}) on page \pageref{rumanian wo} below).

Finally, many languages exhibit lexically defined subclasses of adjectives (or other adnominal modifiers) which are sensitive with regard to the required attributive marking. In Albanian, for instance, the members of one adjective class are regularly marked by head-driven agreement whereas the members of another adjective class require an additional agreement marker (see the Albanian example (\ref{albanian ex}) on page \pageref{albanian synchr}).

In many languages these lexical subclasses seem marginal and are thus often mentioned merely \emph{en passant} (if at all) in grammatical descriptions. The adjective \textit{pikku} ‘little’ in Finnish is an example for such a marginal subclass: \textit{pikku} is juxtaposed to the modified noun while other adjectives in Finnish show number and case agreement as a rule \citep[75]{karlsson1999}. Similarly in German a few adjectives like the colors \textit{lila} ‘lavender’ and \textit{rosa} ‘pink’ behave morpho-syntactically different and do not agree with the modified noun. Another example for a marginal subclass of adjectives comes from Itelmen, where attributive adjectives are regularly marked with a special attributive suffix (see the Itelmen example (\ref{itelmen ex}) on page \pageref{itelmen synchr}). Only a few loan adjectives from Russian occur in juxtaposition \citep[60–71]{volodin1997}.

These marginal adjective classes are often hard to come across in a rather broad typological survey. It seems to be one disadvantage of the typological method (i.e.~sampling and coding a huge amount of different languages on the basis of qualitatively highly diverse grammatical descriptions) that interesting cases are often missed due to limited knowledge or understanding of the structure of all particular languages. From a diachronic perspective, however, “irregular” linguistic structures are very important because they often reflect innovative tendencies or archaic features, i.e.~features which are due to language change. Marginal noun phrase types should thus be included in typological surveys if they are discovered.
