%
%\documentclass[ number=2
%			   ,series=sidl
%			   ,isbn=xxx-x-xxxxxx-xx-x
%			   ,url=http://langsci-press.org/catalog/book/19
%			   ,output=long   % long|short|inprep              
%			   %,blackandwhite
%			   %,smallfont
%			   ,draftmode   
%			  ]{LSP/langsci}                          
%          
%\usepackage{LSP/lsp-styles/lsp-gb4e}
%
%% just for the XeLaTeX logo
%\usepackage{xspace}\newcommand{\latex}{\LaTeX\xspace}
%\usepackage{dtklogos}\newcommand{\xelatex}{\XeLaTeX\xspace}
%
%%for better hyphenation
%\usepackage{german}\selectlanguage{USenglish}
%\usepackage{hyphenat}
%
%\usepackage{multicol}
%
%\usepackage{lscape}
%\usepackage{longtable,tabularx}
%
%\usepackage{tikz-qtree}
%%% has strange side effects
%%%\tikzset{every tree node/.style={align=left, anchor=north}}
%%\tikzset{every roof node/.append style={inner sep=0.1pt,text height=2ex,text depth=0.3ex}}
%
%\usepackage[textsize=tiny,textwidth=1.5cm]{todonotes}
%
%
\hyphenation{ 
ac-cu-sa-tive 
ad-jec-tives 
Ame-ri-can 
con-tras-tive 
de-crea-sing 
lan-gu-a-ges 
for-med 
mark-ed 
no-mi-na-tive 
ty-po-lo-gy 
} 
%          
% \iflsDraft
% \proofmodetrue
% \fi
%
%\hypersetup{bookmarksopenlevel=0}
%
%\begin{document}
%%%%%%%%%%%%%%%%%%%%%%%%%%%%%%%%%%%%%%%%%%%%%%%%%%%%%%%%%

\chapter{The syntax-morphology interface} \label{syntax-morphology-interface}

\section{Morpho-syntax}

An inventory of grammatical features relevant to morphology and its interfaces with semantics and syntax has recently been systematized and presented on-line by Kibort (\citeyear{kibort2008a}; grounded in other work, for instance by \citealt{aronoff1994,corbett1987,carstairs-mccarthy1999,corbett2006,corbett-etal2006,bickel-etal2007})\footnote{A volume on the same topic edited by Anna Kibort and Greville Corbett (\emph{Features: perspectives on a key notion in linguistics}) was announced by Oxford University Press but unfortunately has not yet been published before completion of this thesis in July 2010.}\todo{This book is published and needs to be included now} Kibort's typology of morpho-syntactic features will be evaluated in the following sections. It will be shown that true morpho-syntactic features (i.e.~features not interfacing with semantics) relevant to noun phrase structure are missing but have to be added to such an inventory.

Note that the term “morpho-syntax” is sometimes exaggeratedly used for any type of syntactic construction in which morphological processes take place. It is also commonly used as a homonym for “grammar” thus subsuming all kinds of morphological and syntactic structure of a language. For the present study, however the scopes of syntactic and morphological processes are differentiated from each other. Consequently morpho-syntax is here understood as the interface between syntax and morphology, i.e.~syntactic structure assigning morphology on one or more of its constituents.

\paragraph{Morphological features} True morphological features have only inherent values, i.e.~the assignment of these values is not sensitive to syntax. Morphological features include values which are either fixed, i.e.~supplied on the lexical level, or selected from a range of values. The selection of these values is based only on formal criteria. A prototypical example of a purely morphological feature is inflection class.

\paragraph{Morpho-semantic features} Morpho-semantic features also only have inherent values whose assignment is not sensitive to syntax. The values of morpho"~semantic\todo{ist global für die Silbentrennung als mor-pho-se-man-tic definiert, trennt hier aber nicht. Warum?} features are selected from a range of values. However, unlike purely morphological features, the selection is based on semantic criteria. A prototypical example of the assignment of a morpho-semantic feature is definite marking.

\paragraph{Morpho-syntactic features} Morpho-syntactic features are sensitive to syntax because either agreement or government is involved in the assignment of their values. In the case of agreement, however, a morpho-syntactic feature belongs per definition both to morpho-syntax – due to its contextual assignment to the agreement target – and to pure morphology (or morpho-semantics) – due to its inherent status to the agreement trigger.

The difference between morpho-syntactic and purely morphological (or mor\-pho-semantic) features can be illustrated by definiteness marking in Albanian\il{Albanian}, Bulgarian\il{Bulgarian} and Rumanian\il{Rumanian}. The definite markers in these three Balkan\il{Balkan languages} languages are bound morphemes in postposition, cf.~(\ref{definfl alb}) (\ref{definfl rum}) (\ref{definfl bg}). The syntactic behavior of the definite marker in all three languages is also similar: In noun phrases with modifying adjectives the marker attaches enclitically to the first constituent. 
%%%
\begin{exe}
\ex {\rm Albanian\il{Albanian} \citep{buchholz-etal1987}} 
\begin{xlist}
\ex \label{definfl alb}
\gll	djal=i\\
	boy(\textsc{m})=\textsc{def:m.sg}\\
\glt	‘the boy’
\ex \label{encl alb a}
\gll	djal=i 				i 			mire\\
	boy(\textsc{m})=\textsc{def:m.sg} 	\textsc{attr:def.m.sg}	good.\textsc{m.sg}\\
\glt	‘the good boy’
\ex \label{encl alb b}
\gll	i 			mir=i 			djalë\\
	\textsc{attr:def.m.sg} 	good=\textsc{def:m.sg} 	boy(\textsc{m})\\
\glt	‘the GOOD boy’ 
\end{xlist}
\ex {\rm Rumanian\il{Rumanian} \citep{beyer-etal1987}}
\begin{xlist}
\ex \label{definfl rum}
\gll	băiat=ul\\
	boy(\textsc{m})=\textsc{def.m.sg}\\
\glt	‘the boy’
\ex \label{encl rum a}
\gll	băiat=ul 				bun\\
	boy(\textsc{m})=\textsc{def.m.sg} 	good.\textsc{m.sg}\\
\glt	‘the good boy’
\ex \label{encl rum b}
\gll	bun=ul 					băiat\\
	good=(\textsc{m})-\textsc{def.m.sg} 	boy(\textsc{m})\\
\glt	‘the GOOD boy’
\end{xlist}
\ex {\rm Bulgarian\il{Bulgarian} (own knowledge)}
\begin{xlist}
\ex \label{definfl bg}
\gll	momče=to\\
	boy(\textsc{n})=\textsc{def.n.sg}\\
\glt	‘the boy’
\ex \label{encl bg}
\gll	dobro=to 		momče\\
	good=\textsc{def.m.sg}	boy(\textsc{n})\\
\glt	‘the good boy’
\end{xlist}	
\end{exe}
%%%
The feature \textsc{species},\footnote{Typical values of \textsc{species} are, for instance, \textsc{definite, indefinite} or \textsc{specific}. The use of the term \textsc{species} (from Latin\il{Latin} ‘appearance, form’) is borrowed from Swedish\il{Swedish} and Finnish\il{Finnish} grammatical terminology, \cite[cf., e.g.,][]{holm-etal1970,itkonen-t1980a}. It will be used throughout this investigation instead of the commonly known “definiteness” because it seems terminologically odd to have a feature \textsc{definiteness} exhibiting a value with the similar label \textsc{definite}.} however, does not belong to morpho-syntax in all of these three languages. Even though the definite marker shows the same syntactic behavior (i.e.~attaching in second-position), the morphological feature \textsc{species} is sensitive to syntactic government only in Albanian\il{Albanian}. Whereas definiteness is a purely morpho-semantic feature not involved in any syntactic triggering in Bulgarian\il{Bulgarian} and Rumanian\il{Rumanian}, in Albanian\il{Albanian} a second marker of definiteness occurs on the adjective. This marker is required by syntax through the mechanism of agreement. Hence, only definiteness in the Albanian\il{Albanian} example is morpho-syntactic. In Bulgarian\il{Bulgarian} and Rumanian\il{Rumanian} definiteness is purely morphological.

\section{Morpho-syntactic features} \label{crit eval}

As shown in the previous section, \emph{morpho-syntactic marking} can basically be defined as \emph{morphological marking relevant to syntax}. According to \cite{kibort2008a}, the syntactic relevance of a certain morphological marker is determined by the involvement of this marker in either agreement or government. Kibort's view of morpho-syntax, however, is based on definitions of {agreement} and {government} which imply obligatory interfacing of the respective grammatical features with all three components: morphology, syntax and semantics. Hence, a “more accurate term [\dots] would be ‘morpho-semantico-syntactic’ features” \citep{kibort2008a}. 

Both agreement and government require a syntactic constituent as trigger and another constituent as target of morpho-syntactic marking. Kibort's terms \emph{trigger} and \emph{target} are used in the case of agreement marking, whereas \emph{governor} and \emph{governee} are the respective labels in the cases of government. Consequently, Kibort's \emph{government} covers only morpho-syntactic marking assigned by triggers (governors) which are constituents – like a head noun marked for certain gender and number values triggering gender and number \emph{agreement} on the modifier.

Instances of morphological marking triggered not by constituents but by the syntactic structure as such seem to fall outside the range of Kibort's typology of morpho-syntactic features. A prototypical example of morpho-syntactic marking without a trigger inside the noun phrase is attributive state marking in Persian\il{Persian}.
%%%
\begin{exe}
\ex {\rm Persian\il{Persian} \citep{mahootian1997}}
\label{persian state}
\begin{xlist}
\ex “Construct state” (i.e.~attributive state)
\gll 	xane-ye bozorg\\
	house-\textsc{construct} big\\
\glt 	‘large house’
\ex “Absolute state” (i.e.~predicative state)
\gll	in xane bozorg ast\\
	\textsc{dem} house(\textsc{absolute}) big is\\
\glt	‘the house is large’
\end{xlist}
\end{exe}
%%%
In Persian\il{Persian}, a nominal head is obligatorily inflected in the construct state if an adjective is present. The trigger of the head-marking attributive suffix \textit{-ye} in Persian\il{Persian} is the syntactic structure alone. Since no other value than [+construct] is assigned, semantics cannot be involved. It could be argued that semantics is relevant to the choice of whether to use the adjective as attribute or as predicate and that the attributive inflection on the head noun is inherent (i.e.~morpho-semantically assigned). Semantics (or pragmatics) is of course relevant to the speaker's decision to utter a noun phrase instead of a predication. Semantics is, however, irrelevant to the argumentation about the syntactic structure requiring certain morphological marking: Once the speaker has made her or his decision, it is the syntactic structure alone which is involved in the assignment of the relevant morphological marking. Consequently, attributive construct state in Persian\il{Persian} is an example of true morpho-syntactic marking.

Attributive construct state marking morpho-syntactically similar to the Persian\il{Persian} construct state marking occurs in many other languages. In Bulgarian\il{Bulgarian}, for instance, some nouns require a special inflection after numerals.
%%%
\begin{exe}
\ex {\rm Bulgarian\il{Bulgarian} (own knowledge)}
\label{bg state}
\gll 	dva 	stol-a\\
	two	chair{\textsc{(m)-construct}}\\
\glt 	‘two chairs’
\end{exe}
%%%
Unlike attributive construct state marking in Persian\il{Persian} which occurs obligatorily in noun phrases with different types of modifiers (adjectives, nouns, and some other), attributive construct state marking in Bulgarian\il{Bulgarian} is restricted with regard to both dependent and head. Thus, it occurs only in noun phrases in which the modifier is a numeral higher than ‘one’ and in which the head noun belongs to the class of non-human masculines. In Bulgarian\il{Bulgarian} grammatical tradition this inflectional marking is called the “counting form”.\footnote{Bulgarian\il{Bulgarian} \emph{brojna forma}} The marker originates historically from the genitive singular inflection of masculines. The diachrony, however, does not affect the analysis of this marker as belonging to the morpho-syntactic feature \textsc{state} from a synchronic-typological point of view. Even though attributive construct state marking in Bulgarian\il{Bulgarian} is much more restricted than in Persian\il{Persian}, it clearly belongs to the same type of syntactically assigned inflection on the head noun.

The term \emph{state} here is adapted from \citet[114–116]{melcuk2006} who defines it as an inflectional category of nouns heading a noun phrase. According to Mel'čuk, the function of morphological state marking is licensing the syntactic relationship between the phrase constituents. In the case of head-marking state, as in Persian\il{Persian} and Bulgarian\il{Bulgarian} (\ref{persian state}+\ref{bg state}), the head noun is inflected and shows the morphological value [+construct] if it is the governing member in the present syntactic relation (i.e.~the noun phrase). 

Even though \emph{state} in Mel'čuk's (and others') terms is usually associated with head-marking constructions of the Persian\il{Persian} type (cf.~example \ref{persian state}), a similar morpho-syntactic mechanism applies to dependent marking construct states in other languages. Consider, for example, Kildin\il{Saami, Kildin} Saami in which the dependent noun phrase of a postposition is obligatorily inflected in the genitive case.
%%%
\begin{exe}
\ex {\rm Kildin\il{Saami, Kildin} Saami (own knowledge)}
\label{state ap kildin}
\gll 	tuel'		al'n\\
	chair\textbackslash\textsc{gen}	on\\
\glt 	‘on the chair’
\end{exe}
%%%
It could be argued that the genitive inflection of ‘chair’ in example \ref{state ap kildin} is a morphological value of the feature \textsc{case} assigned to the dependent noun phase by the mechanism of \emph{government}. But since genitive is the obligatory and only possible marker in postpositional phrases in Kildin\il{Saami, Kildin} Saami, there is no motivation for assuming that any case value is marked here. There is no semantic connection to a genitive case which marks a possessor noun in Kildin\il{Saami, Kildin} Saami either.\footnote{This is true from a synchronic point of view. Historically, the origin of the genitive marking in adpositional phrases is easily accounted for and goes back to possessor marking in noun phrases with relational head nouns. But again, the diachrony of a certain marker is not relevant to its synchronic-typological categorization.} Since this modification marker is assigned by the syntax of the specific construction alone, and since the only function of this marker is licensing the given syntactic relation (i.e.~an adpositional phrase), a more appropriate gloss could be \textsc{construct}.

Several languages also exhibit dependent marking construct state in noun phrases. The matching value is usually glossed as \textsc{attributive}. In Kildin\il{Saami, Kildin} Saami, for example, members of one (lexically defined) subclass of adjectives are obligatorily inflected for attributive state if they are used as modifiers in a noun phrase.
%%%
\eal {\rm \textsc{Kildin\il{Saami, Kildin} Saami} (own knowledge)}
\label{state np kildin}
\ex	Attributive adjective (cf.~“attributive state”)\\
\gll 	vīl'k-es'		puaz\\
	white-\textsc{attr}	reindeer\\
\glt 	‘white reindeer’
%%%
\ex	Predicative adjective (cf.~“absolute state”)\\
\gll	puaz lī vīll'k-e\\
	reindeer is white-\textsc{pred}\\
\glt	‘the reindeer is white’
\zl
%%%
The assignment of attributive inflection on (adjectival) modifiers of nouns as well as the assignment of genitive inflection on (nominal) modifiers of adpositions thus follow a similar syntactic mechanism in Kildin\il{Saami, Kildin} Saami: A certain syntactic relationship (i.e.~an adpositional phrase or a noun phrase, respectively) is licensed by marking the dependent phrase constituent with the feature \textsc{state}.

Finally, the feature \textsc{state} may not only be dependent-marked, as in Kildin\il{Saami, Kildin} Saami, but can even interfere with other features. Whereas attributive state marking is invariable in Kildin\il{Saami, Kildin} Saami, in other languages it shows interference with semantic values assigned through the mechanism of agreement. The agreement inflection of attributive adjectives in Russian\il{Russian}, for instance, marks the syntactically governed feature \textsc{state} simultaneously with the morpho-syn\-tacti\-cally governed features \textsc{number/\-gender/\-case}.
%%%
\begin{exe}
\ex {\rm Russian\il{Russian} (own knowledge)}
\label{state np russian}
\begin{xlist}
\ex Attributive adjective inflection (cf.~“attributive state”)
\gll 	belyj	olen'\\
	white:\textsc{attr:m.sg}	reindeer\\
\glt 	‘the white reindeer’
\ex	Predicative adjective inflection (cf.“absolute state”)
\gll	olen' bel\\
	reindeer white:\textsc{pred:m.sg}\\
\glt	‘the reindeer is white’
\end{xlist}
\end{exe}

\section{An ontology of morpho-syntactic features}

Besides introducing a few very basic notions connected to noun phrase structure and adjectival modification, the syntax-morphology interface has been discussed in the theoretical sections above. In particular, Kibort's (\citeyear{kibort2008a}) inventory of grammatical features relevant to morphology and its interfaces with semantics and syntax have been critically evaluated. True morpho-syntactic features (i.e.~features not interfacing semantics) are not yet included in her inventory of grammatical features. The argumentation in the present chapter aims at establishing a new feature \textsc{state}, which according to Kibort's own definitions must be regarded as a true morpho-syntactic feature and which should definitely be added to Kibort's (\citeyear{kibort2008a}) list. 

Figure \ref{features figure} shows the morpho-syntactic features relevant to the present inventory of noun phrase types. Note that only the rightmost feature (6) in that figure can be characterized as being of true \emph{morpho-syntactic} nature. The group of features under (5) must be characterized as \emph{morpho-semantico-syntactic} because the syntactic assignment of these features on the agreement target requires their semantically based assignment on the agreement trigger as well. The group of features under (2–4) are \emph{morpho-semantic} features. The group (1) is purely \emph{morphological}. Note also that the feature \textsc{case} shows up in several leaves because  it can be assigned both in morpho-syntax (through agreement on adjectives) or in morphology (through the assignment of either grammatical or semantic cases on head nouns). 
%%%%
%\begin{figure}[htbp]%!!??check
%\centerline{
%\Tree [.{Morphological\\marking} [.{Inherently\\assigned} [.{Fixed\\(lexically\\supplied)} [.{Based on\\formal\\criteria} [.{e.g.\\\textsc{class}} {1} ] ] [.{Based on\\semantic\\criteria} [.{e.g.\\\textsc{gender}} {2} ] ] ] [.{Selected} [.{Based on\\formal\\criteria} [.{e.g.\\\textsc{number},\\\textsc{case} (gram-\\matical)} {3} ] ] [.{Based on\\semantic\\criteria} [.{e.g.\\\textsc{species},\\\textsc{case} (se-\\mantic)} {4} ] ] ] ] [.{Contextually\\assigned} [.{Determined\\through} [.{Agreement} [.{e.g.\\\textsc{gender},\\\textsc{number},\\\textsc{case},\\\textsc{species}} {5} ] ] ] [.{Determined\\through} [.{(Syntactic)\\Government} [.{e.g.\\\textsc{state}} {6} ] ] ] ] ]\\
%}
%\caption[Ontology of morpho-syntactic features]{An ontology of morpho-syntactic features relevant to the present inventory of noun phrase types (adapted from Kibort \citeyear{kibort2008a} and extended with the feature \textsc{state})}
%\label{features figure}
%\end{figure}
%%%%
In the next part of this investigation, dependent marking \emph{state} will be dealt with in more detail since this type occurs in several languages of the geographic area under investigation.

%\end{document}