%ISBN  978-952-10-6416-6  (PDF) \textbf{(NB. this is dissertation ISBN; manuscript has been edited since)}

\section*{Acknowledgements}

There are several people without whom this description of Mauwake grammar would not have become a reality, and whom I want to thank from my heart.

My colleague for 25 years and a close friend since 1977, Kwan Poh San shared the joys and burdens of life and work with me during the whole of the Mauwake project. We learnt the language together and analysed it together, and although I have written this grammar, she has also contributed significantly towards it.  There are many sections where some of the analysis was done by her and some by myself, but since we worked together it is sometimes hard to distinguish who did which part. Her oral command of the language is better than mine, and I have benefitted from her insights and comments during the writing process.  

The Mauwake people welcomed Kwan Poh San and myself to live with them in Moro village and the family of Leo Magidar adopted us as their daughters. The people built our house, and brought us food. They taught us their customs and shared their everyday lives with us for those over 20 years that we lived in Moro. Although we naturally had more contact with the people in Moro village, the inhabitants of the other Mauwake villages also showed their hospitality and friendship to us. For this I thank them all. 

Several people helped us with language learning and analysis. Saror Aduna first became the main language teacher and later the co-translator for the whole time we were working with the Mauwake people. Others who gave us texts include Kalina Sarak, Balthasar Saakawa, Kululu Sarak, Albert Kiramaten, Alois Amdara, Charles Matuwina, John Meldia, Aduna, Kedem Saror, Bang, Kuumu, Komori, Darawin, and Muandilam.The New Testament checking committee members Balthasar Saakawa, Lukas Miime, Charles Matuwina, Lawrence Alinaw and Leo Nimbulel helped us to understand the language better during our long checking sessions.

Without the encouragement of Professor Fred Karlsson I would have given up many times. He believed in writing descriptive grammars even when it was not fashionable in linguistics, and gave me his unwavering support during the many years that this grammar was in the making.

In my early years in Papua New Guinea, when I was not yet excited about grammar,  Bob Litteral and Ger Reesink encouraged me to take up a study program. The numerous discussions with many SIL-PNG colleagues, Cynthia Farr, Larry Lovell, Robert Bugenhagen, Dorothy James, John Roberts, Carl Whitehead, Eileen Gasaway, Ren\'e van den Berg, Catherine McGuckin and several others, helped me to gain a better understanding both of  PNG languages and of linguistics. Betty Keneqa, the librarian in the SIL linguistic library, was always helpful in locating material and making photocopies. The instruction received from the international linguistics consultants Thomas Payne and David Weber was inspiring and practical.

Others whose teaching, writings and/or personal interaction have shaped my thinking and writing include the late John Verhaar, Bernard Comrie, Talmy Giv\'on, Malcolm Ross and Andrew Pawley, and many others.  

The comments from the external readers of the dissertation, Malcolm Ross and Ger Reesink, helped me to clarify, modify and reorganise the text, and even re-analyse some of the data for the final version.

The financial assistance through scholarships from the Finnish Cultural Foundation, SIL International and SIL-PNG are gratefully acknowledged. My employer, the Finnish Evangelical Lutheran Mission, allowed some working time to be used for studies, and some of the writing was done as my work assignment. 

I thank my husband Jouko for his love and encouragement, and for being there to remind me that there is a lot more to life besides language work.

My greatest thanks go to God, who in his Word gives life, and love, and hope. 



\section*{Abbreviations and symbols}
\begin{tabular}{ll}
ACC & accusative nn, \\
ADD & additive connective \\
ADJ & adjective \\
ADV & adverb(ial) \\
AdvP & adverbial phrase \\
AP & adjective phrase \\
APP & apposition(al) \\
ASP & aspect \\
ASSOC & associative \\
AUX & auxiliary \\
BEN & benefactive \\
BNFY1 & beneficiary 1/2singular \\
BNFY2 & beneficiary non-1/2singular \\
BPx & bring-prefix \\
CAUS & causative \\
CC & complement clause \\
CF & contrastive focus \\
CL & clause \\
CNJ & connective \\
CNTF & counterfactual \\
COM & comitative \\
CMPL & completive aspect \\
CONT & continuous aspect \\
COORD & coordinate \\
CTV & complement-taking verb \\
\end{tabular}
\begin{tabular}{ll}
DAT & dative \\
DEM & demonstrative deictic \\
DISTR/A & distributive: ``all'' \\
DISTR/PL & distributive: ``many'' \\
d & dual \\
DS & different subject following \\
EXC & exclamation, interjection \\
FC & focal (pronoun) \\
FU & future tense \\
GEN & genitive \\
HAB & habitual \\
HN & head noun \\
IMP & imperative \\
INAL & inalienably possessed noun \\
INC & inceptive \\
INCH & inchoative \\
INSTR & instrument \\
INTJ & interjection \\
ISOL & isolative \\
LIM & limiter \\
LOC & locative \\
MAN & manner \\
NEG & negation \\
NF & neutral focus \\
NMZ & nominaliser \\
N & noun \\
\end{tabular}

\begin{tabular}{ll}
NP & noun phrase \\
Np & non-past \\
NVP & non-verbal predicate \\
O & object \\
p, & pl plural \\
P & phrase \\
PA & past tense \\
PAT & patient \\
POSS & possessive \\
PR & present tense \\
QM & question marker \\
QP & quantifier phrase \\
RC & relative clause \\
RDP & reduplication \\
REC & recipient \\
REFL & reflexive \\
s & singular \\
S & subject \\
SEQ & sequential action \\
SIM & simultaneous action \\
SPEC & specifier \\
SR & switch-reference \\
SS & same subject following \\
TH & theme \\
TNG & Trans-New Guinea \\
TP & topic \\
\end{tabular}
\begin{tabular}{ll}
T.P. & Tok Pisin \\
UNM & unmarked (pronoun) \\
v, & V verb \\
1 & first person \\
2 & second person \\
3 & third person \\
* & ungrammatical \\
? & questionable \\
/ & or \\
/ & / phonemic transcription \\
 {} [ & ] phonetic transcription \\
( & ) variant; optional \\
. & syllable break \\
- & morpheme break \\
= & clitic break \\
{\O} & zero morpheme \\
{{\textprimstress}} & primary stress \\
{{\textprimstress}}{{\textprimstress}} & secondary stress \\
\end{tabular}



\section{Introduction}
\hypertarget{RefHeading18281935131865}{}
\subsection{Background}
\hypertarget{RefHeading18301935131865}{}
``Mauwake used to be a big language. The neighbours knew it too, and it was used as a trade language in the area. But today it is not so important any more.'' This is what my colleague Kwan Poh San and I heard when we settled among the Mauwake people in the late 1970's to do linguistic and Bible translation work. Especially before the second World War everybody, including the Mauwake speakers themselves, knew their neighbours' languages better than nowadays, but it may also be true that Mauwake did have a stronger position among the languages in the area. And it is certainly true that the language is fast losing ground to Tok Pisin (also called Melanesian Pidgin), the trade language \textit{par excellence} in Papua New Guinea today.  The process is so strong that Mauwake can be considered an endangered language.

\subsection{Purpose and theoretical orientation of the study}
\hypertarget{RefHeading18321935131865}{}
\subsubsection{Purpose}
\hypertarget{RefHeading18341935131865}{}
My aim is to give a synchronic description of grammatical structures and their functions in Mauwake. Occasionally some attention is given to diachronic aspects as well, when that is considered interesting or helpful for understanding the system at present \citep[20]{EvansEtAl2006}%Dench
. 

This grammar covers mainly morphology and syntax, but a brief overview of phonology is also given, and some pragmatic features are discussed very briefly at the end. A short introduction to typological features and the basic clause structure is given in the introduction to familiarise especially those readers a little with the language who are not reading the grammar from the beginning to the end. The description proper of  the morphology and clausal syntax starts from the structures and describes their functions, as the basic structural features need to be understood first to get a good idea of the language \citep[59]{Mosel2006}. But since functional domains increase in importance when one moves higher up in the unit hierarchy, this is reflected in the arrangement of the grammar: the syntax above the clause level starts from functions and describes different structures used for those functions. Another reason for this switch from an analytic (form-based) to a synthetic (function-based)\footnote{The analytic approach is also called \textit{onomasiological} and synthetic approach \textit{semasiological}.}  approach is the desire to make the grammar more useful for typologists (Cristofaro 2006, Evans and Dench 2006:15). 

The size of the grammar presents a challenge as to the relative amount of documentation vs. analysis.  While documentation is the main purpose of this work, I have attempted to present enough of the analysis to show the reader reasons for certain choices,\footnote{E.g. the status of adjectives as a separate word class and the question of serial verbs have received more discussion than some other topics.} even if I may not meet Dixon's (1997:132) requirement of justifying all my choices ``with a full train of argumentation''. 

This grammar does not include a vocabulary, as a Mauwake dictionary (J\"arvinen, Kwan and Aduna 2001) is available electronically.

\subsubsection{Theoretical considerations}
\hypertarget{RefHeading18361935131865}{}
In the analysis and writing I have been following the informal descriptive theory that was only recently given the name Basic Linguistic Theory (\textstyleAcronymallcaps{BLT}) by \citet{Dixon1997} and elaborated by him (2010-2012) and by Dryer (2006a, 2006b), who also defends its status as a legitimate linguistic theory. \textstyleAcronymallcaps{BLT} makes use of the cumulative knowledge acquired during decades, and centuries -- even millennia -- of grammatical studies, and in the writing of descriptive grammars \citep[3]{Dixon2010}. It is largely based on traditional grammar, but in contrast to traditional grammar it aims to describe the `essential nature' of each language rather than fitting the language into a pre-determined formal model\footnote{These models are often called ``theories''.  For a comment on this, see \citet[131]{Dixon1997}. \citet{Dryer2006a} prefers to call them theoretical frameworks.} \citep[211]{Dryer2006a}. Each language is seen ``as a complete linguistic system'' on its own \citep[4]{Dixon2010}. The theory has been modified over time, and is continually being modified, by developments in typological and formal linguistics (Evans and Dench 2006:6, Rice 2006a and 2006b, Dixon 2010:3).  

Even if \textstyleAcronymallcaps{BLT} does not try to fit languages into any predetermined formal model, it may borrow formalisms from various models as far as they are appropriate and helpful for the description of a particular language \citep[128-135]{Dixon1997}.\footnote{Among others promoting the use of BLT, whether they use the name or not, are \citet[354]{Noonan2006}, Rice (2006a and 2006b), Evans and \citet{Dench2006}, and Payne (1997, 2006).}  \textstyleAcronymallcaps{BLT} is closely linked with language typology, ``[setting] out a typological paradigm, by inductive generalization from reliable grammars'' \citep[205]{Dixon2010}. Evans and \citet[6]{Dench2006} note that grammars written in this framework tend to stand the test of time better than those following strict formal models. Formal theories have been and are useful in providing useful research questions -- both bringing up completely new ones and deepening old ones -- and in forcing the descriptions to be more rigorous. In Rice's (2006b:262) words, ``[t]he theory informs and shapes, but does not control''. 

My own dislike of formalisms is certainly one reason why they are used so little in this grammar.  A more important factor is my desire to make the grammar readable to as many people as possible regardless of their linguistic background. This is also reflected in the use of terminology. I have tried to use widely accepted and transparent terminology as much as possible, to avoid technical terms specific to some particular formalism, and to explain my terminology where necessary (cf. Cristofaro 2006). 

For the description of Mauwake the following basic concepts familiar from traditional grammar are assumed as given:

Word classes like noun, verb, pronoun, adverb (the status of adjective as a class of its own is discussed separately);

Morphological cases like nominative, accusative, genitive and dative;

Syntactic roles of subject and object;

Semantic/case roles like agent, patient, recipient and beneficiary;

Phrases like \textstyleAcronymallcaps{NP}, \textstyleAcronymallcaps{AP}, \textstyleAcronymallcaps{AdvP};

Clause as a separate level from sentence. 

The concept of medial verbs, as against final/finite verbs, which is generally accepted in Papuan linguistics, is also presupposed. 

Since frequency of occurrence is an important and interesting characteristic of grammatical usage, I initially planned to do a fair amount of quantification and frequency counting.  But to do an adequate job would have required a much larger corpus, as well as better computer programs and knowledge of corpus linguistics, and much more time, than I had at my disposal. Even though actual percentages are seldom mentioned in the final product, I have occasionally included frequency statements based on whatever frequency counts I have made during the course of the work and on my personal experience with Mauwake. 

To my knowledge there are no trained linguists among the Mauwake speakers, so the kind of cooperation between a native and a non-native speaker linguist, together with native speaker non-linguists, that \citet{Ameka2006} advocates, was not possible.  Even though I have aimed at checking the material as carefully as possible, there are bound to be mistakes both in the data and in the interpretation. It is necessary to heed Ameka's (ibid. 92) warning that ``[o]ne of the most dangerous things about authoritative and influential foundation records ... is that their misanalyses which pertain to some theoretical or typological point are repeated over and over again in the literature.  What is even worse is that the theories and generalizations are built on such mistakes''. 

\subsubsection{Audience} 
\hypertarget{RefHeading18381935131865}{}
One can anticipate the readership for a reference grammar of a previously undocumented and endangered language spoken by a couple of thousand speakers to consist mainly of linguists.  I especially hope this grammar to be useful for those linguists who work on language typologies and typology-related questions. Naturally the material is also available for those interested in more formal models.

A grammar is expected to describe features that exist in a language, rather than those that do not exist. But for the benefit of typologists I have at times mentioned the non-existence of certain features that they might be looking for and wondering about, if there is no mention at all \citep{Cristofaro2006}. 

Another readership I want to address are those people particularly in Papua New Guinea who are linguistically somewhat less trained, yet are vitally interested in language development and translation.  If this grammar helps any of them to study and understand a language better, or encourages someone to write a grammar of yet another undocumented language, my work has been worthwhile.

It is unlikely that many outsiders would use this grammar to learn Mauwake.  It may also be unrealistic to wish that many Mauwake speakers would become familiar with it. Yet it is my desire that it would help the Mauwake speakers in at least two ways: by preserving their language and giving them more pride in it as they realise that it does have a real grammar \citep[255]{Kadanya2006}, and also by providing some help for those interested and involved in teaching vernacular literacy.

\subsubsection{On the data and examples} 
\hypertarget{RefHeading18401935131865}{}
The bulk of the text data used for this grammar were collected between 1979 and 1985, with some later additions. The basic data of 19 spoken and 7 written texts contain over 8300 words in all (200+ KB in plain text), edited by a native speaker. They consist mainly of narratives, also including traditional stories (60\%), but descriptive texts (15\%), process descriptions (14\%) and one long hortatory text (11\%) are included as well, from different speakers and authors.\footnote{Appendix 1 provides a list of the texts used.} Many syntactic features were further checked against another set of texts about the same size. 

When choosing examples, I have taken as many from text material as possible, especially when the examples consist of a clause or a sentence.  Elicited examples were checked for correctness with native speakers.

In the examples the present orthography is used, but with morpheme breaks added. There is no gender distinction in Mauwake pronouns, so in the free translation the third person singular pronoun and verbal suffix are translated as either `he' or `she' whenever justified by either textual or cultural context, otherwise as `(s)he'. 

Regarding the glosses, the reader will be wise to remember Mosel's (2006:50) caution that the interlinear glossing is \textstyleEmphasizedWords{not} ``an accurate form-meaning relationship {\dots} The meaning of words and larger units of grammatical analysis does not equal the sum of the meanings of their component parts {\dots} but results from the interaction of the meaning of the construction as such and the meanings of its parts.  Thus interlinear glossing should only be seen as a tool to help the reader to understand the examples''. 

\subsection{The Mauwake people, their environment and culture}
\hypertarget{RefHeading18421935131865}{}
The Mauwake language is spoken along the North coast of Madang province, about 120 km northwest of Madang town. The area comprises about 100 square kilometres, and there are 15 villages where Mauwake is the main language, seven of them along or near the coast along a stretch of 15 km between the Kumil and Nemuru rivers, and up to 12 km inland from the coast.

\subsubsection{Geography and administration} 
\hypertarget{RefHeading18441935131865}{}
The Mauwake area is typical of the Madang North coast: coral reefs off the coast, white sand beaches,\footnote{White and black sand beaches alternate on the coast, depending on the existence of coral reefs off the coast and on the closeness of the two of volcanic islands of Karkar and Manam.} a narrow belt of coastal plain, and hills about 200 to 400 feet in height. The soil is mostly coral limestone, with shallow alluvial soil.  The lower hills close to the coast are covered by \textstyleForeignWords{kunai} grass (\textstyleEmphasizedWords{Imperata cylindrica}), the higher ones deeper inland by rainforest, some of which is garden regrowth (Haantjens et al. 1976:22). 

The climate is lowland tropical climate with temperatures varying between 20{\textdegree} and 32{\textdegree} centigrade. Humidity is high, especially during the wet season.  The dry season is between May and October with average monthly rainfall of 40 mm, the wet season is between November and April and with average rainfall of 250 mm. The dry season is longer and drier in this area than in many other parts of the country apart from the Port Moresby area. During the last two decades there have been significant climate changes, and the weather patterns are less predictable than they used to be.

The North Coast Highway that was completed in 1973-74 and sealed in 1999 passes close to all the coastal Mauwake villages.  Almost all the inland villages are also accessible by a road of some kind. 

The two main centres in the area are Ulingan, where there is a Roman Catholic mission station and community school, and Malala, where there is a high school and a community school, a sub-health centre sponsored by the high school, a reasonably well stocked store, and a market.


\begin{figure}
%\includegraphics[width=\textwidth]{a2baed703b528405dbd2e29ffc5720026-img1.jpg}
\caption{Mauwake language area (non-Mauwake speaking villages are in brackets)}
\label{fig:map:1}
\end{figure}

Administratively the Mauwake people belong to the Bogia sub-province and the Almami (derived from the language names Alam--Mauwake--Miani) local level government area. 

There are four primary schools in the area, and one high school.  In all of these schools there are students from more than one language area. The Roman Catholic Church was instrumental in getting the schools started, and is still administering the Malala High School.  Nearly all of the children go to primary school, but the number of Mauwake students in the high school is not very high. Vernacular preschools were started in the whole Mauwake area in the early 1990s, but many of them have since changed into Tok Pisin preschools.

\subsubsection{On the history of the Mauwake people}
\hypertarget{RefHeading18461935131865}{}
Until fairly recently, little was known about the pre-history of the Papuan-speaking people in Near Oceania (including New Guinea island, Bismarck Archipelago and the Solomon Islands), compared with the archaeological information available on the Austronesian-speaking people in the area. By the late 1990's it was established that human occupation on the northern coast of New Guinea island dated back to at least 40 000 years. There are signs of semi-domestication of some tree crops from 20 000 to 10 000 years ago, and of agriculture from about 10 000 years ago, roughly the same time that the Highlands valleys became more habitable after the end of the Ice Age (Pawley 2005a:xi-xvii). 

From the great diversity of the languages around Cape Croisilles area across Karkar Island, \citet[27]{Ross1996} hypothesizes that this probably is where the Croisilles linkage languages, including Mauwake (or its parent language), started spreading from.\footnote{See \sectref{sec:1.4.1} for a description of the genealogical affiliation. }  He does not provide any dates for the migrations.

Besides some traditional myths we have not been able to obtain stories telling about life earlier than the first half of the 20th century.  The majority of the Mauwake people agree that the language group has spread to the coast from inland, and they specify Aketa village as their place of origin. It is commonly believed that long ago the people of the Amiten village, now considered the ``heart area'' of the language by many speakers,spoke a different language, which has since disappeared. 

The hypothesis that the Mauwake people came from inland would at least partially explain the present language situation on the coast, where there are many languages scattered in a small geographical area. If at some point in history the coast did not have permanent inhabitants to defend it from intruders, it would have been easy for people migrating from various directions and speaking different languages to settle there. One cultural trait that points towards an earlier home area inland is that among the Mauwake speakers fishing is not as important as it is for some other language groups. The coastal villagers mainly catch fish for their own needs, and only occasionally take it to the local market if they happen to have surplus. Gardening, rather than fishing, is the important activity for them.

Possibly the first mention of the larger area where the Mauwake people live is given by the German \citet[338]{Hollrung1888}, who mentions ``the Tsimbin tribe'', meaning the people of Simbine village,\footnote{Situated 8 km from Moro village, and 5 km from the closest Mauwake village.} speaking the Maiani language which borders the Mauwake language area. H\"oltker (1937:964) calls Maiani and the related languages by the name ``M\'oando languages'' based on the word \textstyleEmphasizedWords{man} in those languages.  He also mentions Mauwake as ``Moro-Sapara-Ulingan'' -- picking names of three coastal villages -- as a language deviating from the M\'oando languages (ibid.). 

The written history of the Mauwake area itself began during the German colonial era (ca. 1884-1921) with the report of the killing of two Lutheran missionaries\footnote{The Rhenish Mission had planned to start the work in the area for some time, but it was blocked by the Neu-Guinea-Compagnie. The reason for the killing of the two missionaries, Wilhelm Scheidt and Friedrich B\"osch, was never found out, but it is likely that the local people associated them with the Compagnie and feared that they were in fact planters coming to start plantations in their area (Wagner and Reiner 1986:106-107).}  and an officer of the Neu-Guinea-Compagnie\footnote{The company had established a big coconut plantation further northwest on an island off Hatzfeldthafen in 1885. It developed quickly despite  various problems, but had to be abandoned completely in 1891 because of the hostility of the inhabitants of the area. Within 20 years the site was again covered by rainforest \citep[450-51]{Tranel1952}.} , as well as 14 accompanying native people, in Malala Bay in May 1891 (Tranel 1952:454, Wagner and Reiner 1986:106-109).  After this the Lutheran church abandoned the plan to establish a mission station in the area, and founded one further southeast in the Bunabun area instead.

The Roman Catholic mission was then given the authority in 1891 to search the area between Ulingan and Bogia for suitable places for the mission \citep[8]{Duamba1996}.  The Ulingan-Sapara mission station was established in 1926, and a church big enough for a thousand people was built in Sapara village the same year \citep[21]{BrummEtAl1995}%Mihalic
.  A tsunami struck the coast in the morning of Christmas Eve, 1930, killing five people and destroying the new church and the priest's house.\footnote{Presumably the rest of the Sapara village was destroyed as well, as the church was probably the strongest building in the whole village.}  The mission station was moved to the Ulingan village and a new church was built on top of a hill there \citep[20-21]{Davies1999}. The Malala church was built in 1958 on land owned by the Moro villagers, and a high school started on the same compound in 1966 \citep[45]{BrummEtAl1995}%Mihalic
. Both the high school and all the community schools in the area were established by the Catholic Church.  Because of the many missionaries engaged in the work there the local people had a fair amount of contact with Westerners.

In the early years the priests were expected to learn the local language and to become familiar with the culture, especially religious beliefs \citep[25]{BrummEtAl1995}%Mihalic
. The liturgy and some preaching were done in Mauwake too, and a few hymns and prayers were composed in it. But whatever written materials there may have been, they were all lost in the Second World War (Z'Graggen 1971:3-4).  And already in the 1930s Tok Pisin had started to replace the local languages as the official language for evangelization in the Catholic Church \citep[179]{BrummEtAl1995}%Mihalic
.  Especially in an area where five different languages are spoken along a 20 km stretch of the road, this is understandable.  

The Second World War had a profound influence on the area.  In December 1942 thousands of Japanese soldiers landed in Madang and Wewak (ibid. 37).  From Wewak the troops marched down towards Madang, and some of them settled in Ulingan.  They required the local men to help build bridges, and asked the people for food.  The women and also many men from the coastal villages fled to inland villages and to the rainforest, because they were scared of the soldiers.  They were suffering from a shortage of food, as they were not able to do their gardening in a normal way.  The Japanese apparently did not commit cruelties, as was the case in some other areas, and the relationship between them and the local people was uneasy but not hostile. When the Allied forces started to bomb the Japanese-occupied areas, the people had to keep hiding even more and were not even able to cook, as they were afraid that the smoke from their cooking fires might attract the pilots' attention and cause the area to be bombed.  A number of bombs were dropped in the Mauwake language area, and a few people died. 

Before the war, the missionaries were almost the only outsiders that the local people met, but during the war they had contact especially with Japanese but also with Allied soldiers.  After the war a number of young men went to work on plantations in different parts of the country or had other employment outside their home area, thus gaining knowledge of the wider world. The founding of Malala High School in 1966 and the completing of the North Coast Highway in the mid-70's further widened the people's horizons.

\subsubsection{Demography}\footnotemark{}
\hypertarget{RefHeading18481935131865}{}
\footnotetext{ Much of the contents of the sections 1.3.3-1.3.6  is based on the Mauwake background study written by Kwan Poh San in 1988.}
The inhabitants in the 15 Mauwake-speaking villages number about 4000;  the number is based on the census figures in 2000.  Not all of them speak the language, however, as most of the children now learn Tok Pisin as their first language.

The Mauwake speakers are not a uniform group socially or politically. The basic political unit is a village made up of a few clans. There is usually a main village, with some hamlets attached to it.  Recently there has been a tendency towards moving away from the main village and building small hamlets near the family's garden or coconut plot. 

A person's main responsibility is towards one's own family and clan. The basic unit is a nuclear family: parents and their children, either their own or adopted.  The society is patrilineal: kinship is traced through, and the inheritance handed down from, the father. Adoption is widespread and always takes place within extended family, usually the husband's side of the family.  Members of an extended family are expected to assist each other in various ways: providing food at feasts, helping to pay a debt, bride price or some other obligation, and looking after each other in general. The responsibilities towards one's clan are also strong but not quite as strong as to one's extended family.  Traditionally the clans used to own all the land, but planting coconuts, and later cocoa, changed the situation.  The use of garden land is still decided by the headmen (leaders) of each clan, but now there is rivalry even between members of the same clan about the existing coconut trees and about land where new coconut or cocoa trees can be planted.

Every clan has its own headman, and in earlier times the headman of the most prestigious clan also used to be the headman of the whole village. Decisions were based on consensus after discussions in the village meetings, but the final authority rested on the headmen.

After the establishment of the local level government system the authority of the headmen partly transferred to the local government member (\textstyleForeignWords{kaunsil}), to the magistrate and to the leader of the community work (\textstyleForeignWords{komiti}). The traditional authority structure has more or less broken down and since it has not been completely replaced by the new structure, this has given way to individualism and even disregard of any authority, especially among the young people. The Catholic Church is a somewhat cohesive force, but it has lost some of its authority with the social breakdown and also with the coming of other churches.

Each village has social ties with other, usually closely situated villages regardless of the language.  Many of the Mauwake villages have close interaction with non-Mauwake-speaking villages.  This has also resulted in extensive intermarrying between different language groups, which in the earlier times encouraged bilingualism or trilingualism, but which nowadays strengthens the use of Tok Pisin.

The six languages either bordering the Mauwake area, or inside it, are the Kaukombar\footnote{I am utilising Ross' 2006 grouping here. For a discussion on the classification of the Madang languages, see \sectref{sec:1.4.1} below.} languages Maiani, Miani (Tani)\footnote{The names without parentheses are what the speakers prefer to use for their languages, the ones in parentheses are those used in linguistic literature especially by Z'Graggen and those utilising his data. Maiani and Miani are mentioned here as separate languages, but they can also be considered different dialects of one language.}  and Mala (Pay)\footnote{Mala has two distinct dialects, Mala and Alam. The latter is spoken in the two villages that have close contact with the Mauwake area.}, the Tibor language Mawak, the Korak-Waskia group language Amako (Korak), all of which are Trans New Guinea languages; the only Austronesian language is Beteka (Medebur), closely related to Manam language. None of these languages is dominant compared with the others.  The Mauwake speakers say that it used to be a prestigious language in the area, but I have not been able to confirm this with speakers of the other languages. Bi- and trilingualism used to be extensive in the whole area especially before the arrival of Tok Pisin. 

\subsubsection{Economy}
\hypertarget{RefHeading18501935131865}{}
Subsistence farming is the main activity of the Mauwake people. They get most of their food and building materials from their own land.  Traditionally the main staple was taro, supplemented with yam, sweet potato and cooking bananas; sago was used particularly when little other food was available. Especially on the coast yam has recently been replacing taro as the main staple, because there is not enough land for slash-and-burn gardening required by taro. The traditional diet was very balanced, the basic meal including staples, vegetables and some smoked fish or meat, all cooked in coconut milk. Fruit eaten as snacks provided extra vitamins. Nowadays store-bought foods give variety to the diet but do not add much nutritional value, apart from tinned fish and meat, which provide some extra protein.

Hunting and fishing used to be important activities especially for men, but their significance has decreased. Wild pigs are getting scarce, and bandicoots are mainly hunted during the dry season.  As the Mauwake people have probably come from further inland, fishing has not been as important for them as for some other groups on the coast.  Both men and women do some fishing, but mainly for their own family's needs. 

Any garden produce, fish or bandicoots not needed by the family may be sold at the Malala market, which is the biggest one between Madang and Bogia, or at the smaller Ulingan market.  

For a long time coconut has been the main cash crop, but with the falling copra prices the people have diversified into growing cocoa, coffee\footnote{Growing coffee was given up later, because it is very labour-intensive and the \textstyleForeignWords{robusta} coffee grown in the lowlands fetches a very low market price.} and recently also vanilla.  The cash crops are transported to Madang to sell.  During the German colonial era tobacco was introduced in the area, and still in the 1930s Malala area was famous for its tobacco \citep[454]{Tranel1952}. Nowadays the people mainly grow it for their own use, and sell any extra at the local market.

The high school and a logging company provide employment for a few local men.  In the area where logging is done landowners also get some royalties from it.  Logging has caused controversy among the people.\footnote{The first logging company in the 1980s went bankrupt and the landowners received very little money for their timber.  Even with subsequent logging the benefits for the local people have been rather modest.} Many of the more educated men, and some women, now in their 40s and 50s have migrated into towns where they work as tradesmen, teachers, or in other occupations. 

\subsubsection{Cultural notes} 
\hypertarget{RefHeading18521935131865}{}
In the traditional worldview the seen and the unseen are both important parts of the same universe.  The unseen world consists of different kinds of spirits: clan spirits and other spirits in nature (\textstyleStyleVernacularWordsItalic{inasina}), spirits of the recently dead (\textstyleStyleVernacularWordsItalic{kukusa}) and spirits of those who have died a long time ago (\textstyleStyleVernacularWordsItalic{sawur}). The spirits need to be treated with respect so that they will not harm but rather help the people.  Although the reliance on the spirits has decreased with the coming of Christianity, various rituals are still fairly widely practised to ensure the benevolence of the spirits, especially in connection with birth, death, sickness, hunting and gardening. 

Sickness is normally attributed to one's bad relations with other people or disregard of the spirits, the work of a sorcerer, or in some cases to ``natural causes''. Death is still commonly believed to be caused by sorcery.

Name taboos are a typical feature of the cultures in Oceania.  It is forbidden to call one's in-laws by name, or call anyone else by name who has the same name as the in-laws.  In the Mauwake culture both of the parents give a child the name of one of his or her own relatives, which the other parent naturally may not pronounce.  In addition to these two names, a child also receives a Christian name at baptism, and may be given other names as well.  Thus a person can have even five or six names, which are used by different people to call him or her.  And when the person gets married, all those names are forbidden for the in-laws to use.  They may use a kinship term or invent a nickname by which to address the person. In general, kinship terms are used widely both to address people and to refer to them.

Passing on the traditional culture and customs is hampered by the lessening use of the vernacular as well as the lack of interest especially among many young people. Grown-ups may deplore the situation, but there is little attempt to actively pass on the cultural heritage or to help the young generation to evaluate, appreciate and renew their own culture. 





\begin{table}

%\includegraphics[width=\textwidth]{a2baed703b528405dbd2e29ffc5720026-img2.jpg}


\caption{}
%\label{}
\end{table}


\subsubsection{Mauwake kinship system}
\hypertarget{RefHeading18541935131865}{}
The kinship system of the Mauwake people is a slightly modified Iroquis system. Both gender and generation are important, but also the distinction of parental siblings of the opposite sex (Chart 1). One's father's brother is also called \textstyleStyleVernacularWordsItalic{auwa} `father' and his wife is \textstyleStyleVernacularWordsItalic{aite} `mother'; likewise one's mother's sister is also `mother' and her husband is `father'. But mother's brother is called \textstyleStyleVernacularWordsItalic{yaaya} `uncle', and his wife is \textstyleStyleVernacularWordsItalic{paapan} `aunt'; father's sister is also `aunt' and her husband is `uncle'. The term `father' is used for the following as well: one's own father's cross-cousins, one's father-in law and, for a female, elder sister's husband. Two generations up from self the grandparents are distinguished by gender: \textstyleStyleVernacularWordsItalic{kae} `grandfather' and \textstyleStyleVernacularWordsItalic{kome} `grandmother', but two generations down all the grandchildren are called \textstyleStyleVernacularWordsItalic{iimasip} `grandchild'. 

In one's own generation there are two sets of terms for brothers and sisters. Their use  depends on whether relative age or gender is in focus: \textstyleStyleVernacularWordsItalic{paapa} `older sibling' and \textstyleStyleVernacularWordsItalic{aamun} `younger sibling' are used for siblings of either sex, whereas \textstyleStyleVernacularWordsItalic{yomokowa} `brother'  and \textstyleStyleVernacularWordsItalic{ekera} `sister' are gender-bound terms. The latter are more commonly used by siblings of the opposite sex than by those of the same sex. All the parallel cousins are also considered one's siblings, whereas one's cross-cousins, the children of the `uncles' and `aunts', are called \textstyleStyleVernacularWordsItalic{yomar/emar} `cousin', a term used for either sex.

One generation down from self, one's children include not only one's own sons (\textstyleStyleVernacularWordsItalic{muuka}) and daughters (\textstyleStyleVernacularWordsItalic{wiipa}), but also those of one's siblings of the same sex, \textstyleEmphasizedWords{and} those of one's cross-cousins. For the sons and daughters of one's siblings of the opposite sex there is a single term, \textstyleStyleVernacularWordsItalic{eremena} `nephew/niece'. Most of the terms for kin relations are inalienably possessed nouns (\sectref{sec:3.2.4}). 

Mother's brother is a particularly important relative for performing rites of passage like initiation, marriage and funeral. When a person dies, his/her maternal uncle, together with the deceased person's male cross-cousins, is responsible for burying him/her and distributing his/her possessions.\footnote{For an older person whose uncles have already died, nephews (= sons of the siblings of opposite sex) take their place among these men..} These men are called \textstyleStyleVernacularWordsItalic{weria} men.  \textit{Weria} means `planting stick', and the term is used as a metaphor for burial.\footnote{It is not unusual to have the same verb for `burying' and `planting' in Papuan languages, but in Mauwake they are different.} An uncle also has an important function as a mediator, if his nephew or niece has serious problems with his/her nuclear family. Although father's sister's husband is also called an `uncle', he does not have a similar role to that of mother's brother.  

\subsection{The Mauwake language}
\hypertarget{RefHeading18561935131865}{}
\subsubsection{Genealogical affiliation and previous research}
\hypertarget{RefHeading18581935131865}{}
The name Mauwake means `what?'\footnote{Actually it consists of the question word \textit{mauwa} `what' and the contrastive focus clitic -\textit{ke}.}  The Mauwake speakers themselves identify the language by this name, and the speakers of the related Kaukombaran languages use corresponding names to call their own languages.  The people have a myth in which the spirit Turamun gives each group their land area, their main staple as well as their language, and the language name originates in this myth.

Before our taking residence in Moro village in 1978, there was only very sketchy research done on the Mauwake language, just enough to classify it.\footnote{\citet{Capell1952}, and following him Voegelin and \citet{Voegelin1965}, \citet{Greenberg1971}, then Z'Graggen (1971, 1975a, Wurm (1975, 1982) and Wurm and \citet{Hattori1981}.} The name Ulingan was taken from the main mission station in the area, although that is not how the speakers themselves call their language. Sometimes the alternative name Mawake is given in brackets in the earlier language lists.  

Mauwake is a Papuan language. `Papuan' is just a cover term for a number of genetically unrelated language families, which are not Austronesian and are spoken in the New Guinea region.\footnote{The name Papuan has been criticized (Capell 1969, Haiman 1979), but it is widely used instead of its alternative, non-Austronesian.} The Papuan languages consist of several unrelated language families, the biggest of which is the Trans New Guinea (TNG) family.  


\begin{figure}
%\includegraphics[width=\textwidth]{a2baed703b528405dbd2e29ffc5720026-img3.jpg}
\caption{New Guinea island language map \citep[34]{Ross2005}}
\label{fig:map:2}
\end{figure}

The Trans New Guinea hypothesis was originally put forward by McElhanon and \citet{Voorhoeve1970} to account for the similarities between the Finisterre-Huon languages on the one hand, and Central and South New Guinea Stock languages on the other. Later \citet{Wurm1975} argued that a great number of additional languages belong to the phylum.  Much of the work relied on lexico-statistical rather than more rigorous application of the standard comparative method, and because many of the claims are not well substantiated, the whole \textstyleAcronymallcaps{TNG} hypothesis received a fair bit of criticism (Lang 1976, Haiman 1979, Foley 1986, Pawley 1995).

Most of the classificatory work done on the languages of Madang Province  is based on Z'Graggen's (1971, 1975) groundbreaking research.  According to Wurm's (1975) classification following the language family tree model of lexicostatistics, Mauwake\footnote{In Z'Graggen's (1980) listing Mauwake has the code F2, and the ISO 639-3 code for the language is mhl.} belongs to the Madang-Adelbert Range sub-family, Adelbert Range superstock, Pihom stock, and Kumilan\footnote{Z'\citet{Graggen1971} initially called the family Ubean, possibly based on the language names Ulingan and Bepour, but later (1975) changed the name into Kumilan based on the name of the Kumil river.} language family together with two very small languages, Bepour and Moere. 



%\includegraphics[width=\textwidth]{a2baed703b528405dbd2e29ffc5720026-img4.jpg}
\begin{figure}
\caption{Wurm's grouping of Madang-Adelbert languages (Ross 1996:Map 2)}
\label{fig:map:3}
\end{figure}

For nearly two decades there was practically no comparative linguistics done on Papuan languages.  But in the early 1990s more detailed research started on the Madang-Adelbert Range languages, now renamed the Madang group, and later on other \textstyleAcronymallcaps{TNG} languages as well \citep{Pawley1998}. As a result of that research Pawley (1995, 2001) and \citet{Ross1995} came to the conclusion that the Trans New Guinea hypothesis is workable but needs modification. They also concluded that the Madang group definitely is part of the Trans New Guinea language family. According to their new classification Mauwake belongs to the Trans New Guinea family, the Madang group and the Croisilles linkage of languages. \citet[21-25]{Ross1996} also discusses the relationships between the various languages within the Croisilles subgroup, using the term \textstyleEmphasizedWords{Kumil} (Z'Graggen's  \textstyleEmphasizedWords{Kumilan} ) for the family including Mauwake, and \textstyleEmphasizedWords{Kaukombar} (Z'Graggen's \textstyleEmphasizedWords{Kaukombaran} ) for the four languages closest to the Kumil languages. He also does some regrouping within the families based on the pronoun  forms in the languages.  In the Kumil group he includes not only Mauwake, Bepour and Moere, but also the languages Musar and Bunabun. 



%\includegraphics[width=\textwidth]{a2baed703b528405dbd2e29ffc5720026-img5.jpg}
\begin{figure}
\caption{Ross' 1996 grouping of Madang-Adelbert languages (Ross 1996: Map 4)}
\label{fig:map:4}
\end{figure}

Apart from Z'Graggen's survey no other linguistic study of any depth has been carried out on the Mauwake language except what has been done by Kwan Poh San and myself (Kwan 1980, 1983, 1988, 1989, 2002; J\"arvinen 1980, 1988a, 1988b, 1989, 1990, 1991; J\"arvinen, Kwan and Aduna 2001, and Bergh\"all 2006.)  The grammatical work published on related languages includes Reesink's grammar of \citet{Usan1987}, MacDonald's grammar of \citet{Tauya1990} and Ross and Paol's grammar of \citet{Waskia1978}. Two grammars in manuscript form that were also used for reference are Maia grammar by Barbara Hardin and Bargam grammar by Mark Hepner. Both are available electronically and in the SIL-PNG library, Ukarumpa.

The ISO-639 code for Mauwake, based on \citet{Grimes2000}, is mhl, and the Glottolog code is mauw1238 (glottolog.org).

\subsubsection{Typological overview of morphological and syntactic features}
\hypertarget{RefHeading18601935131865}{}
In this section, morphological and syntactic characteristics of the Mauwake language are discussed in relation to the typology of Papuan/Trans New Guinea languages and to the universal word order\footnote{As \citet[72]{Dixon2009} notes, ``word order'' here should be called ``(clausal) constituent order'', as it is the ordering of constituents that the typology is based on rather than that of individual words.}  typology.  To some extent these two overlap, as \textstyleAcronymallcaps{TNG} languages typically are also \textstyleAcronymallcaps{SOV} languages.

\paragraph[Mauwake as a Trans New Guinea language]{Mauwake as a Trans New Guinea language}
\hypertarget{RefHeading18621935131865}{}
Mauwake has many features typical of both Papuan languages in general and Trans New Guinea languages in particular. 

The \textstyleEmphasizedWords{\textsc{phonology}} of the language is simple: there are five vowel and fourteen consonant phonemes, and only a few of them have more than one allophone.  Morphology is quite transparent, so there is very little morphophonology.

The \textstyleEmphasizedWords{\textsc{basic} }\textstyleEmphasizedWords{\textsc{order of clausal constituents}} is verb-final. In neutral clauses with both subject and object the order is \textstyleAcronymallcaps{SOV} (\stepcounter{nx}{\thenx}), but it changes into \textstyleAcronymallcaps{OSV} when the object is fronted (\stepcounter{nx}{\thenx}) as a theme (9.1).  Adverbials are somewhat less constrained in their ordering. It is also very common to have the verb as the only element in a clause (\stepcounter{nx}{\thenx}). 

\ea%x656
\label{ex:x656}
\gll [Ona  emeria  nain=ke]\textsubscript{S}  [maa]\textsubscript{O}  wafur-a-k. \\
      \\
\glt
\z

3s.GEN  woman  that1=CF  thing  trow-PA-3s

`His wife threw things.'

\ea%x657
\label{ex:x657}
\gll [Wiipa  nain]\textsubscript{O}  [eka=ke]\textsubscript{S}  mu-o-k. \\
      \\
\glt
\z

daughter  that1  water=CF  swallow-PA-3s

`The daughter was swallowed by the water.'

\ea%x659
\label{ex:x659}
\gll Uruf-a-m. \\
      \\
\glt
\z

see-PA-1s

`I saw it.'

In \textstyleEmphasizedWords{\textsc{complex sentences}} the subordinate clause usually precedes the main clause.  Thus the reason/cause precedes the result/effect, in conditional sentences protasis precedes the apodosis, and in intention/purpose sentences the intention precedes the expected result. When the reason follows the result, it is a very marked order.

Mauwake is clearly a nominative-accusative type language, rather than ergative-absolutive. The agent of a transitive verb (\stepcounter{nx}{\thenx}) is marked in the same way as the actor of an intransitive verb (\stepcounter{nx}{\thenx}), and most experiential verbs have the experiencer as a nominative subject (\stepcounter{nx}{\thenx}).

\ea%x1523
\label{ex:x1523}
\gll Yo  mauw-owa  nia  asip-i-yem. \\
      \\
\glt
\z

1s.UNM  work-NMZ  2p.ACC  help-Np-PR.1s

`I help you with work.'

\ea%x1524
\label{ex:x1524}
\gll Yo  koka=pa  ik-e-m. \\
      \\
\glt
\z

1s.UNM  jungle=LOC  be-PA-1s

`I was in the jungle.'

\ea%x1525
\label{ex:x1525}
\gll Yo  wailal-i-yem  a. \\
      \\
\glt
\z

1s.UNM  hunger-Np-PR.1s  oh

`Oh, I'm hungry.'

\textstyleEmphasizedWords{\textsc{Verb morphology}} in Mauwake is extensive, even if not as extensive and complex as in some other Papuan languages. The morphology is agglutinative, and  affixation is mostly very transparent.  Suffixes are used for subject, tense and aspect, benefactive, distributive, causative and counterfactual marking. Prefixing is used very little, only for reduplication and to form verbs referring to bringing and taking. It is possible to have several derivational and inflectional affixes in one verb, as shown by the elicited example (\stepcounter{nx}{\thenx}), but in actual usage this is rare.

\ea%x664
\label{ex:x664}
\gll Muuka  wia  \textstyleEmphasizedVernacularWords{arim-ow-omak-om-ek-a-k}. \\
      \\
\glt
\z

son  3p.ACC  grow-CAUS-DISTR/PL-BEN-BNFY1.CNTF-PA-3s

`(S)he would have brought up (many) sons for me.'

Mauwake has a clear three-tense system (\sectref{sec:3.8.3.4}). Even though the tense suffixes only distinguish between past and non-past, the distinction between present and future shows in the subject suffixes, which are different for these two tenses.  Aspect marking is optional (\sectref{sec:3.8.5.1.1}).  The auxiliary follows the main verb. There is no passive form in verbs.

A very typical feature in the Papuan languages is a difference between final (\sectref{sec:3.8.3.4}) and medial verbs (\sectref{sec:3.8.3.5}). The former are finite verbs with full inflection for tense and subject number and person, and the most typical position for them is at the end of a declarative sentence. The medial verbs indicate whether the subject of a clause is the same as (\stepcounter{nx}{\thenx}), or different from (\stepcounter{nx}{\thenx}), that of the following clause. The same-subject forms also indicate whether the action of the second verb is simultaneous with that of the first verb, or sequential (\stepcounter{nx}{\thenx}) in relation to it. Medial clauses (8.2) are coordinate with, but also dependent on, the following clause. Because of the existence and extensive use of medial clauses, temporal subordinate clauses (8.3.3.1) are used very little in Mauwake.\footnote{Medial clauses in Papuan languages are often translated with temporal subordinate clauses in other languages, even if they are not subordinate in the original language.}

\ea%x662
\label{ex:x662}
\gll Owowa  ek-\textstyleEmphasizedVernacularWords{ap},  wailal-\textstyleEmphasizedVernacularWords{ep}  akia  ik-e-k. \\
      \\
\glt
\z

village  go-SS.SEQ  be.hungry-SS.SEQ  banana  roast-PA-3s  

`He went to the village, was hungry and roasted bananas.'

\ea%x663
\label{ex:x663}
\gll Mik-\textstyleEmphasizedVernacularWords{amkun}  me  um-o-k,  wiowa  onaiya  ikiw-em-ik-\textstyleEmphasizedVernacularWords{eya} \\
      \\
\glt
\z

spear-1s/p.DS  not  die-PA-3s  spear  with  go-SS.SIM-be-2/3s.DS  

Olas=ke  war-e-k.

Olas=CF  kill-PA-3s

`When I speared it, it didn't die, (but) as it was going with the spear Olas killed it.'

Medial verbs are also used in tail-head linkage (8.2.3.5), another strategy common in Papuan languages. The last verb of a sentence is repeated in the first clause of the next sentence, but usually in medial form.  In spoken Mauwake this recapitulation device is used to indicate actions that continue on the story line without a major break, but since the development of the written language the tail-head linkage is losing this function and is getting a new function as a  marker of the climax in the story.

Another typical feature of many Papuan languages is the lack of a large inventory of verb stems (\textstyleBibliogBaseChar{Foley 1986}:127). An extreme case is Kalam with its less than 100 verb stems; consequently, Kalam needs to use serial verb and adjunct plus verb constructions for most actions (\textstyleBibliogBaseChar{Pawley 1987}:336-7). Mauwake has a reasonably large verb inventory, but in addition it uses both serial verbs (\sectref{sec:3.8.5.1.2}) and adjunct plus verb constructions (\sectref{sec:3.8.5.2}).

There is no inflection on \textstyleEmphasizedWords{\textsc{nouns}} (\sectref{sec:3.2}) or \textstyleEmphasizedWords{\textsc{adjectives}}\textstyleEmphasizedWords{\textsc{} }(\sectref{sec:3.3}), nor are there gender/noun class distinctions. But Mauwake makes a distinction between alienably and inalienably possessed nouns (\sectref{sec:3.2.4}).  Most kinship terms are inalienably possessed, but body parts are not.

The \textstyleEmphasizedWords{\textsc{noun phrase}} (\sectref{sec:4.1}) most commonly consists of the head noun by itself, or with just one modifier.  In a noun phrase a pluralizing (\stepcounter{nx}{\thenx}) unmarked pronoun, a possessive noun phrase, a temporal phrase, or a qualifier noun phrase may precede the head noun; all the other modifiers follow it. A possessive preceding the head noun and an adjective following it (\stepcounter{nx}{\thenx}) is quite common in Trans New Guinea languages \citep[19]{Reesink1987}. 

\ea%x658
\label{ex:x658}
\gll \textstyleEmphasizedVernacularWords{wi  emeria}  teeria  nain \\
      \\
\glt
\z

3p.UNM  woman  group  that1

`that group of women'

\ea%x660
\label{ex:x660}
\gll \textstyleEmphasizedVernacularWords{yena}  aamun  \textstyleEmphasizedVernacularWords{gelemuta}  kuisow \\
      \\
\glt
\z

1s.GEN  1s/p.younger.sibling  small  one

`my one younger brother' or `one of my younger brothers'

Mauwake exhibits more variation in the \textstyleEmphasizedWords{\textsc{pronoun}} forms (\sectref{sec:3.5}) than many other Papuan languages do.  There is only singular and plural number, and no inclusive-exclusive distinction in the first person plural.  But there are separate sets for unmarked, accusative, dative, genitive, isolative, reflexive-reciprocal and comitative pronouns.  Mauwake is a typical Papuan language in that the subject pronoun may be left out; the third person subject pronoun is overt  mainly when it is used for a re-activating an earlier topic (\sectref{sec:9.2.3}).  But in imperative clauses a subject pronoun is very common, which is \textstyleEmphasizedWords{not} usually mentioned as a typical feature of Papuan languages,\footnote{To my knowledge this particular feature has not been studied much in Papuan languages.}  and is quite rare cross-linguistically.

\paragraph[Mauwake as an SOV language]{Mauwake as an SOV language}
\hypertarget{RefHeading18641935131865}{}
Mauwake conforms very strongly to the typological patterns found to exist in the \textstyleAcronymallcaps{SOV}, or hence, \textstyleAcronymallcaps{OV} languages.  The following discussion on various characteristics in Mauwake that correlate with the \textstyleAcronymallcaps{OV} constituent order is based on \citet{Dryer2007a}.

Concerning the following sentence level features Mauwake shows itself a typical \textstyleAcronymallcaps{OV} language.  The interrogative marker -\textstyleStyleVernacularWordsItalic{i}  always occurs sentence-finally in polar questions (\sectref{sec:7.2.2}).

\ea%x672
\label{ex:x672}
\gll Yo  emeria  efar  uruf-a-man=\textstyleEmphasizedVernacularWords{i}? \\
      \\
\glt
\z

1s.UNM  woman  1s.DAT  see-PA-2p=QM

`Did you see my wife?'

In non-polar, or content questions (7.2.1), the question word or phrase is in the same position that would be occupied by the non-interrogative word or phrase in a statement.

\ea%x673
\label{ex:x673}
\gll Ni  sira  \textstyleEmphasizedVernacularWords{kamenap}  on-a-man? \\
      \\
\glt
\z

2p.UNM  custom  what.like  do-PA-2p

`What did you do?'

In complex sentences (\sectref{sec:8.3}) the subordinate clause usually comes before the main clause.

\ea%x674
\label{ex:x674}
\gll Mua  imen-ap=\textstyleEmphasizedVernacularWords{na}  feeke  wia  p-ekap-eka. \\
      \\
\glt
\z

man  find-SS.SEQ=TP  here.CF  3p.ACC  Bpx-come-IMP.2p

`If you find the men, bring them here.'

Complement clauses (\sectref{sec:8.3.2}) behave like other subordinate clauses, preceding the main clause. 

\ea%x675
\label{ex:x675}
\gll \textstyleEmphasizedVernacularWords{Mukuna  kerer-e-k  nain}  i  me  paayar-e-mik. \\
      \\
\glt
\z

fire  start-PA-3s  that1  1p.UNM  not  understand-PA-1/3p

`We didn't realise that a fire had started.'

The typical \textstyleAcronymallcaps{OV} order for predicate-copula applies only partly in Mauwake, as a copular verb is not used for for the present tense. The \textstyleAcronymallcaps{OV} order does show in the other tenses and the medial forms.

\ea%x676
\label{ex:x676}
\gll O  somek  mua=(pa)  \textstyleEmphasizedVernacularWords{ik-eya}  {\dots} \\
      \\
\glt
\z

3s.UNM  song  man-(LOC)  be-2/3s.DS

`When he was a teacher {\dots}'

Clause and sentence level features that correlate with the \textstyleAcronymallcaps{OV} order are as follows.  The position of a complementiser or a subordinator is clause-final:

\ea%x677
\label{ex:x677}
\gll Yo  emeria  aaw-owa  kookal-ek-a-m=\textstyleEmphasizedVernacularWords{na}  {\dots} \\
      \\
\glt
\z

1s.UNM  woman  get-NMR  like-CNTF-PA-1s=TP

`If I had liked/wanted to get a wife ...  '

Both manner adverbs, postpositional phrases, and non-argument noun phrases precede the verb.

\ea%x678
\label{ex:x678}
\gll Fikera  nain  \textstyleEmphasizedVernacularWords{sira  feenap}  on-a-mik. \\
      \\
\glt
\z

kunai.grass  that1  custom  like.this  do-PA-1/3p

`This is what they did to the \textstyleForeignWords{kunai} grass.'

Typical \textstyleAcronymallcaps{OV} features also manifest themselves in different phrases. In the \textstyleAcronymallcaps{VP}s (or verbal groups, as they are called below in 3.8.5.1), the main verb precedes the auxiliary.

\ea%x679
\label{ex:x679}
\gll Saa=iw  \textstyleEmphasizedVernacularWords{ir-am-ika-i-mik}. \\
      \\
\glt
\z

sand=INST  come-SS.SIM-be-Np-PR.1/3p

`They are coming along the sand/beach.'

In basic noun phrases (\sectref{sec:4.1.1}) the genitive precedes the head noun:

\ea%x680
\label{ex:x680}
\gll yiena  miiwa \\
      \\
\glt
\z

1p.GEN  land

`our land'

Mauwake does not have articles. When the distal-1 deictic \textstyleStyleVernacularWordsItalic{nain} `that' is used,  there is often considerable semantic bleaching, and it seems to be becoming more like a definite article, but in many contexts it still clearly retains its deictic function.  

Mauwake has postpositional phrases (\textstyleAcronymallcaps{PP}), rather than prepositional phrases.

\ea%x681
\label{ex:x681}
\gll koor(a)  kuenuma=pa \\
      \\
\glt
\z

house  underside=LOC

`underneath the house'

An \textstyleAcronymallcaps{OV} feature that shows on word level is that Mauwake has suffixes rather than prefixes in the verbs.

\ea%x682
\label{ex:x682}
\gll Akia  ik-\textstyleEmphasizedVernacularWords{omak-e-mik}. \\
      \\
\glt
\z

banana  roast-DISTR/PL-PA-1/3p

`We roasted many bananas.'

As there are no comparative forms for adjectives in Mauwake, one\textstyleAcronymallcaps{ OV} characteristic that does not apply in Mauwake is the standard of comparison and comparison marker preceding the adjective. 

Case marking of transitive arguments with an affix is more common in \textstyleAcronymallcaps{OV} than in \textstyleAcronymallcaps{VO} languages.  In Mauwake there are no case suffixes on either the subject or the object, but all human objects require an accusative pronoun (\sectref{sec:3.5.3}) to occur preceding the verb.

\subsubsection{Dialects}\footnotemark{}
\hypertarget{RefHeading18661935131865}{}
\footnotetext{ The data for this section is mainly taken from the Mauwake dialect survey report (J\"arvinen 1988, ms.).}
The Mauwake speakers themselves do not identify clearly defined dialects, but they do refer to the speech differences between the inland villages and the coastal villages. Some also separate the Ulingan group from the rest, and the Ulingan group people make a distinction between themselves and those further west along the coast.

The majority of the Mauwake speakers consider Aketa and Amiten as the centre of the language group. People in each village claim that their own way of speaking is the ``true'' way, but at the same time they credit Aketa as the place where the language originated.  The Ulingan and Papur dialect groups do not admit the prestige of Aketa and Amiten quite as willingly.

Comparing the Mauwake data\footnote{The basic 100-word list by \citet[55-59]{Ezard1978} was used with four semantically problematic words deleted and four other words added.} lexicostatistically would indicate that there are no distinct dialects in the language at all. The percentage of cognates between all the villages is 100.  What variation an earlier survey seemed to show, turned out to be multiple cognates. But the phonostatistic method (Grimes and Agard 1959, modified as in Simons 1977:177-178) yields some dialectal differences. There are pronunciation dissimilarities, on the basis of which the language area can be divided into three main dialect areas: Ulingan (Ulingan, Sikor and Meiwok), Papur (Papur, Tarikapa, Yeipamir) and Muaka (Muaka, Moro, Mereman, Sapara, Aketa, Amiten/Susure/Wakoruma\footnote{Susure and Wakoruma were not included in this survey because of their closeness to Amiten both location- and dialectwise.}, and Saramun). 

Of the 100 words in the list, 60\% are pronounced identically in all the villages.  Of the rest, a little over half (i.e. 21\% of the whole data) are cases of non-phonemic variation, namely [w]\~{}[$\beta $], and [j]\~{}[{\textyogh}]. The first one of these the speakers of the language do not even notice, the second one they notice to some extent. 

Map 5 gives the mean degrees\footnote{The mean degree of difference between two sounds was calculated by first counting hypothesized minimal steps from one to another, one minimal step given the value of one. These were added up and divided by the number of words in the data, i.e. 100.} of pronunciation differences between some of the Mauwake villages. 



%\includegraphics[width=\textwidth]{a2baed703b528405dbd2e29ffc5720026-img6.jpg}
\begin{figure}
\caption{Mean degrees of pronunciation difference between some Mauwake villages}
\label{fig:map:5}
\end{figure}

The Ulingan dialect is the most homogeneous, and also most clearly a separate group from the others. The mean degree of pronunciation differences between Ulingan and Sikor, and between Sikor and Meiwok is 0.02, which means that in a hundred-word list there are only two differences of one degree.  The pronunciation difference between Tarikapa and Sikor or Meiwok is the biggest, 0.17 degrees.



\tableref{tab:1} gives the mean degrees of pronunciation differences between all the villages.

Muaka

.08  Saramun

.11  .09  Tarikapa

.10  .12  .11  Papur

.10  .11  .08  .03  Yeipamir

.07  .11  .07  .08  .08  Aketa

.07  .07  .10  .08  .09  .09  Amiten

.03  .07  .11  .08  .12  .07  .09  Moro

.07  .08  .14  .10  .15  .06  .08  .04  Mereman

.08  .05  .10  .11  .12  .05  .10  .06  .04  Sapara

.11  .12  .15  .08  .10  .14  .13  .09  .11  .07  Ulingan

.12  .13  .17  .08  .14  .13  .12  .08  .09  .09  .02  Sikor

.15  .14  .17  .09  .14  .13  .16  .10  .09  .06  .03  .02  Meiwok


\begin{table}
\caption{Mean degrees of pronunciation differences between Mauwake villages}
\label{tab:1}
\end{table}

Indication of a dialect division similar to that mentioned above, especially setting the Muaka group apart from the others, was also provided by morphemes that were not in the 100-word list but which were checked during the survey, because they had been found to occur in a fairly clear pattern across the language area. These morphemes are:

inowa    vs.  unowa      `many'

urup(-iya)  vs.  irip(-iya)      `ascend'

ikiw(-iya)    vs.  itiw(-iya)      `go'

unan    vs.  inuan    vs.  inon  `yesterday'

-era    vs.  -eya/-iya      `2/3 p. medial verb suffix'

The isogloss map 5 shows the distribution of the pronunciation of these morphemes in the various villages.  The only case where the isoglosses would suggest a different dialect grouping from the one presented above is that of Saramun, which would seem to belong more closely to the Papur group than the Muaka group.




%\includegraphics[width=\textwidth]{a2baed703b528405dbd2e29ffc5720026-img7.jpg}
\begin{figure}
\caption{Distribution of some pronunciation differences}
\label{fig:map:6}
\end{figure}

What complicates the dialect division is the fact that sometimes the same pronunciation, deviant from the more common way of pronouncing a word, can be found in villages far apart like Aketa and Meiwok: (\textstyleStyleVernacularWordsItalic{imakuna} rather than \textstyleStyleVernacularWordsItalic{umakuna} `neck'), or Papur,  Moro and Mereman villages and the Ulingan group (\textstyleStyleVernacularWordsItalic{epia} rather than \textstyleStyleVernacularWordsItalic{ipia} `rain'). Also, there is no clear pattern of pronunciation differences between villages; sometimes the differences are opposite in the case of two vocabulary items. The word for `many' in the Muaka dialect\footnote{Excluding Amiten/Susure/Wakoruma} is \textstyleStyleVernacularWordsItalic{inowa}, but the others pronounce it \textstyleStyleVernacularWordsItalic{unowa}, whereas the word for `ascend/go up' in the Muaka dialect is \textstyleStyleVernacularWordsItalic{urupiya} but in the other dialects it is \textstyleStyleVernacularWordsItalic{iripiya}. Likewise, the Ulingan group differs from the rest in the pronunciation of \textstyleStyleVernacularWordsItalic{omaiwia} `tongue' (vs. \textstyleStyleVernacularWordsItalic{omaiwa} in others) and \textstyleStyleVernacularWordsItalic{awulak} `sweet potato' (vs. \textstyleStyleVernacularWordsItalic{awuliak} in others), so the difference is almost exactly the reverse in the two cases.  

No grammatical differences have been found to exist between the dialects.  Neither are there social registers, nor special language for restricted uses like rituals. 

\section{Phonology: a brief overview}
\hypertarget{RefHeading18681935131865}{}
\subsection{Phonemes}
\hypertarget{RefHeading18701935131865}{}
The phonological system in Mauwake is quite regular and straightforward, even if not one of the very simplest found in Papuan languages \citep[48-64]{Foley1986}. It has 14 consonants and 5 vowels in its phoneme inventory.  Allophonic variation in Mauwake is very limited, and there is not much morphophonological complexity (\sectref{sec:2.3.3}) either.  In the presentation of the phonology IPA standard phonetic symbols are used.

\subsubsection{Consonants}
\hypertarget{RefHeading18721935131865}{}
The fourteen consonant phonemes in Mauwake are presented in \tableref{tab:1}.  Z'\citet[51]{Graggen1971} also lists the velar nasal /{\ng}/ as a phoneme in Mauwake, but at least synchronically it is not part of the basic inventory.  All the words in Mauwake that have the velar nasal are shared with a neighbouring language, so they are likely to be borrowings. For those words there is also a native synonym, although it may not be as frequently used. It is also possible that Mauwake has earlier had the velar nasal, as it is a very common areal feature in the Madang North Coast area (Z'Graggen 1971). 





\begin{tabular}{lllll}
\mytoprule
 & Bilabial & Alveolar & Palatal & Velar\\
Plosive & p  b & t  d &  & k  g\\
Nasal & m & n &  & \\
Fricative & {\textphi} & s &  & \\
Trill &  & r &  & \\
Lateral &  & l &  & \\
Approximant & w &  & j & \\
\mybottomrule
\end{tabular}



\begin{table}
\caption{Consonant phonemes}
\label{tab:2}
\end{table}

Most of the consonant phonemes in Mauwake have only one extrinsic allophone. 

The voiceless \textstyleEmphasizedWords{\textsc{plosives}} are unaspirated in all the word positions where they occur. They contrast as to bilabial, alveolar and velar points of articulation. Mauwake does not have the glottal stop typical of many Papuan languages. 

/paanek/  [{{\textprimstress}pa:nek}]  `it crashed'

/taanek/  [{{\textprimstress}ta:nek}]  `it is full'

/kaanek(e)/  [{{\textprimstress}ka:nek(e)}]  `where?'

/opa/  [o{{\textprimstress}pa}]  `hold!'

/otal/  [o{{\textprimstress}tal}]  `reef'

/oka/  [o{{\textprimstress}ka}]  `hand drum'

/orop/  [o{{\textprimstress}rop}]  `descend.SS.SEQ'

/rotorot/  [ro{{\textprimstress}torot}]  `painted moray eel'

/orok/  [o{{\textprimstress}rok}]  `he/she descended'

The voiceless plosives occur word-initially, -medially and -finally.

/pepek/  [pe{{\textprimstress}pek} ]  `enough'

/onap/  [o{{\textprimstress}nap}]  `do.SS.SEQ'

/teteke/  [te{{\textprimstress}teke}]  `take apart!'

/menat/  [me{{\textprimstress}nat}]  `tide'

/koka/  [ko{{\textprimstress}ka}]  `bush, jungle'

/onak/  [o{{\textprimstress}nak}]  `his/her mother'

The voiced plosives only occur word-initially or medially. Besides this distributional restriction, their frequency is also markedly lower than that of voiceless plosives. They are not utilised in the derivational or inflectional morphology, except in reduplication. There is ``voicing harmony'' affecting the plosives only: when the first two syllables begin with plosives, both of them are either voiced or voiceless. 

/bebeta/  [be{{\textprimstress}beta}]  `thin'

/pepena/  [pe{{\textprimstress}pena}]  `strange'

/duduwa/  [du{{\textprimstress}duwa}]  `blunt'

/tutupila/  [tu{{\textprimstress}tupila}]  `tadpole'

/googok/  [{{\textprimstress}go:gok}]  `trevally'

/kookalija/  [{{\textprimstress}ko:kalija}]  `he/she likes'

/boga/  [bo{{\textprimstress}ga}]  `barren, empty (land)'

/poka/  [po{{\textprimstress}ka}]  `sit down!'

/dabela/  [da{{\textprimstress}bela}]  `cold'

/tapaka/  [ta{{\textprimstress}paka}]  `cake'

/gubagel/  [gu{{\textprimstress}bagel}]  `lizard sp.' 

/kupakup/  [ku{{\textprimstress}pakup}]  `sago container'

The only exceptions to the voicing harmony are a few words starting with /k/, for instance:

/kadilam/  [ka{{\textprimstress}dilam}]  `leech'

/kibol/  [ki{{\textprimstress}bol}]  `stinging anemone'

/kuben/  [ku{{\textprimstress}ben}]  `prawn trap'

The two \textstyleEmphasizedWords{\textsc{nasals}} occur word-initially, medially and finally, and contrast as to bilabial and alveolar points of articulation. 

/manar/  [ma{{\textprimstress}nar}]  `forehead decoration'

/nanar/  [na{{\textprimstress}nar}]  `story'

/moma/  [mo{{\textprimstress}ma}]  `taro'

/mona/  [mo{{\textprimstress}na}]  `fruit sp.'

/onam/  [o{{\textprimstress}nam}]  `I did'

/onan/  [o{{\textprimstress}nan}]  `you did'

The \textstyleEmphasizedWords{\textsc{fricatives}} contrast as to bilabial/labio-dental and alveolar points of articulation. They are both voiceless.\textstyleStyleVernacularWordsItalic{} The voiceless bilabial fricative /{\textphi}/ [{\textphi}] occurs word-initially and medially, the alveolar grooved fricative /s/ [s] occurs word-initially, -medially and \nobreakdash-finally. 

/{\textphi}ariar-/  [{\textphi}a{{\textprimstress}riar}-]  `abstain'

/sariar-/  [sa{{\textprimstress}riar}-]  `get well'

/kosija/  [ko{{\textprimstress}sija}]  `it comes out of mouth'

/ko{\textphi}ija/  [ko{{\textprimstress}}{\textphi}ija]  `he hammers'

/kawus/  [ka{{\textprimstress}wus}]  `smoke'

A possible reason for the restricted distribution of /{\textphi}/ is that it is a result of a sound change, which is discussed at the end of the consonant section.

The voiced alveolar \textstyleEmphasizedWords{\textsc{trill}} /r/ [r] occurs in free variation with the voiced alveolar flap [{\textfishhookr}] word-initially, -medially and -finally.  

/rowirow/  [ ro{{\textprimstress}wirow}] \~{} [{\textfishhookr}o{{\textprimstress}wi}{\textfishhookr}ow]  `giant clam'

/ewar/  [ e{{\textprimstress}var}] \~{} [e{{\textprimstress}va}{\textfishhookr}]  `west wind'

The voiced alveolar \textstyleEmphasizedWords{\textsc{lateral}} /l/ [l] occurs word-initially, -medially and -finally. 

/lali/  [la{{\textprimstress}li} ]  `small reef fish'

/kaul/  [{{\textprimstress}kaul}]  `hook'

In many Papuan languages [l] and [r] are allophones of the same phoneme, but in Mauwake they are separate phonemes, contrasting with each other:

/liilin-/  [li:lin-]  `sting, smart (v.)'

/riirin-/  [{{\textprimstress}ri:rin}] \~{} [{{\textprimstress}}{\textfishhookr}i:{\textfishhookr}in-]  `quarrel (v.)'

/kalan-/  [ka{{\textprimstress}lan}-]  `have nausea' 

/karan-/  [ka{{\textprimstress}ran}] \~{} [ka{{\textprimstress}}{\textfishhookr}an-]  `shake'

/nanal/  [na{{\textprimstress}nal}]  `tree sp.'

/nanar/  [na{{\textprimstress}nar}]  `story'

Yet in a few words the two fluctuate. This seems to be a dialectal difference.

/eliwa/  [e{{\textprimstress}liva}] \~{} [e{{\textprimstress}riva}]  `good'

/saliwija/  [sa{{\textprimstress}livija}] \~{} [sa{{\textprimstress}rivija}]  `(s)he heals/repairs'

There are two approximants, or \textstyleEmphasizedWords{\textsc{semivowels}}: [w] and [j].  They are interpreted as consonants when occurring in syllable onset or coda, and as vowels when forming part of the syllable nucleus. 

The alveo-palatal semivowel /j/ [j] occurs word-initially and -medially. The voiced alveo-palatal grooved fricative [{\textyogh}] is used instead of [j] in the inland (Papur) and Ulingan dialects.

/jakiya/  [ja{{\textprimstress}kija}] \~{} [{\textyogh}a{{\textprimstress}ki}{\textyogh}a]  `(s)he bathes'

/jaisow/  [{{\textprimstress}jaisow} ] \~{} [{{\textprimstress}}{\textyogh}aisow]  `I alone' 

The bilabial semivowel /w/ has the following allophones:

[w]  \textstyleListBaseChar{voiced bilabial semivowel occurs next to a rounded vowel, fluctuating with [v] when between a preceding unrounded and a following rounded vowel};

[v]  voiced labio-dental frictionless continuant occurs elsewhere;

[$\beta $]  voiced bilabial fricative occurs fluctuating with both [w] and [v] in the inland (Papur) dialect, very strongly in the village of Yeipamir.

/wowosa/  [wo{{\textprimstress}wosa}] \~{} [$\beta $o{{\textprimstress}$\beta $osa}]  `bud'

/now/  [now] \~{} [no$\beta $]  `stonefish'

/kuwiwi/  [ku{{\textprimstress}wiwi}] \~{} [ku{{\textprimstress}$\beta $i$\beta $i}]  `blue-lined surgeonfish'

/iwoka/  [i{{\textprimstress}woka}]\~{}[i{{\textprimstress}voka}]\~{}[i{{\textprimstress}$\beta $oka}]  `yam'

/iwera/  [i{{\textprimstress}vera}] \~{} [i{{\textprimstress}$\beta $era}]  `coconut'

/elew/  [e{{\textprimstress}lev}] \~{} [e{{\textprimstress}le$\beta $}] \footnote{All these different optional allophonic variations of /w/ are not listed in the phonetic representations below, unless relevant to the discussion in the main text of the section. The same applies to the variation of /{\textphi}/, /r/ and /j/.}  `in-law'

The reasons for analyzing the semivowels as consonants are as follows:


\begin{itemize}
\item There are no unambiguous 3-vowel sequences word-initially; 

\item Both the semivowels have a fricative allophone;

\item There are no unambiguous glides starting with a mid vowel; 

\item The geminate non-high vowels only occur in initial syllables;

\item If they were interpreted as vowels, the stress pattern of some words would not follow the otherwise exceptionless stress placement rule.


\end{itemize}
/wiwisa/  [vi{{\textprimstress}visa}] \~{} [$\beta $i{{\textprimstress}$\beta $isa}]  `murky'

/jaisow/  [{{\textprimstress}jaisow}] \~{} [{{\textprimstress}}{\textyogh}aisow]  `I alone'

/marew/  [ma{{\textprimstress}rev}] \~{} [ma{{\textprimstress}}re$\beta $]  `none'

/now/  [now] \~{} [no$\beta $]  `stonefish'

/jakijem/  [ja{{\textprimstress}kijem}] \~{} [{\textyogh}a{{\textprimstress}}ki{\textyogh}em]  `I bathe'

/uruwa/  [u{{\textprimstress}ruwa}] \~{} [u{{\textprimstress}}ru$\beta $a]  `loincloth'

The following sets of examples show clear contrasts between the semivowel /w/ and the vowel /u/, and between the semivowel /j/ and the vowel /i/:

/wulinija/  [wu{{\textprimstress}linija}]  `it shines'

/uusakija/  [{{\textprimstress}u:sakija}]  `he/she roasts'

/wuunija/  [{{\textprimstress}wu:nija}]  `(wind) blows'

/uuwunija/  [{{\textprimstress}u:wunija}]  `he/she talks'

/wuwusirap  [wu{{\textprimstress}wusirap}]  `name of a month'

/lalu/  [la{{\textprimstress}lu}]  `parrotfish'

/diluw/  [di{{\textprimstress}luw}] \~{} [di{{\textprimstress}lu$\beta $}]  `vine sp.'

/jena/  [je{{\textprimstress}na}] \~{} [{\textyogh}e{{\textprimstress}na}]  `my'

/jiena/  [ji{{\textprimstress}ena}] \~{} [{\textyogh}i{{\textprimstress}ena}]  `our'

/iina/  [{{\textprimstress}i:na}]  `mosquito'

/jiija/  [{{\textprimstress}ji:ja}] \~{} [{{\textprimstress}}{\textyogh}i:{\textyogh}a]  `he/she gives to me'

In a few words a semivowel is adjacent to a homorganic vowel, but such a contrast as above is not available, and the regular syllable patterns and the stress placement rule allow for two or more interpretations. Also, the pronunciation varies slightly from village to village and between individuals.  In these cases the decision how to represent the word phonemically is somewhat arbitrary.

/jaamun/  [{{\textprimstress}ja:mun}] \~{} [j{\textsci{{\textprimstress}}}amun]  `my/our younger sibling'

/jaaja/  [{{\textprimstress}ja:ja}] \~{} [{{\textprimstress}jaija}]  `my/our maternal uncle'

/waaja/  [{{\textprimstress}wa:ja}] \~{} [{{\textprimstress}waija}] \~{} [{{\textprimstress}wuaija}]  `pig'

/wuija/  [{{\textprimstress}wuija}] \~{} [{{\textprimstress}waija}] \~{} [{{\textprimstress}wuaija}]  `he/she puts'

The \textstyleEmphasizedWords{\textsc{bilabial}} consonants contrast word-initially and -medially; those consonants that can occur word-finally contrast in this position too. 

/poka/  [po{{\textprimstress}ka}]  `house post'

/boga/  [bo{{\textprimstress}ga}]  `empty, barren (land)'

/moma/  [mo{{\textprimstress}ma}]  `taro'

/{\textphi}oma/  [{\textphi}o{{\textprimstress}ma}]  `ashes'

/womar/  [wo{{\textprimstress}mar}]  `his cousin'

/epa/  [e{{\textprimstress}pa}]  `place'

/bebaura/  [be{{\textprimstress}baura}]  `tree sp.'

/ema/  [e{{\textprimstress}ma}]  `mountain'

/e{\textphi}a/  [e{{\textprimstress}}{\textphi}a]  `me'

/ewar/  [e{{\textprimstress}var}]  `west wind'

/orop/  [o{{\textprimstress}rop}]  `descend.SS.SEQ'

/orom/  [o{{\textprimstress}rom}]  `I descended'

/arow/  [a{{\textprimstress}row}]  `three'

The \textstyleEmphasizedWords{\textsc{alveolar}} consonants contrast in word-initial and -medial positions, and those that can occur in word-final position contrast in that position as well.

/tawowola/  [ta{{\textprimstress}wowola}]  `rubbish'

/dabela/  [da{{\textprimstress}bela}]  `cold'

/nabena/  [na{{\textprimstress}bena}]  `carrying pole'

/sawur/  [sa{{\textprimstress}wur}]  `spirit'    

/raapa/  [{{\textprimstress}ra:pa}]  `bag'

/labuela/  [la{{\textprimstress}buela}]  `pawpaw'

/otal/  [o{{\textprimstress}tal}]  `reef'

/odaweleka/  [o{{\textprimstress}daweleka}]  `gill'

/onam/  [o{{\textprimstress}nam}]  `I did'

/osaiwa/  [o{{\textprimstress}saiva}]  `bird of paradise'

/oraija/  [o{{\textprimstress}raija}]  `he/she descends'

/olal/  [o{{\textprimstress}lal}]  `fish species'

/menat/  [me{{\textprimstress}nat}]  `tide'

/konan/  [ko{{\textprimstress}nan}]  `garfish'

/oras/  [o{{\textprimstress}ras}]  `spinefoot (fish)'

/nanar/  [na{{\textprimstress}nar}]  `story'

/nanal/  [na{{\textprimstress}nal}]  `tree sp.'

The \textstyleEmphasizedWords{\textsc{velar}}\textstyleEmphasizedWords{} and\textstyleEmphasizedWords{} \textstyleEmphasizedWords{\textsc{alveo-palatal}} consonants contrast word-initially and -medially. Word-finally /j/ does not occur at all, and /g/ is extremely rare.\footnote{There are only 4 occurrences of word-final /g/ in the lexicon of over 3600 words. Those may all be loans from neighbouring languages.}

/kia/  [k{\textsci{{\textprimstress}}}a]  `white'

/gia/  [g{\textsci{{\textprimstress}}}a]  `baby'

/jia/  [j{\textsci{{\textprimstress}}}a]  `us'

/magok/  [ma{{\textprimstress}gok}]  `woven band'

/makak/  [ma{{\textprimstress}kak}]  `brown quail'

/majona/  [ma{{\textprimstress}jona}]  `brown-collared bush turkey'

Both the distributional restrictions of some consonant phonemes and some regular sound correspondences between Mauwake and the related Kaukombaran languages point to earlier sound changes. My tentative suggestion is that the voiced plosives /b/ and /g/\footnote{There are too few words with /d/ and /t/ in the sample to make a meaningful comparison, and what data is available does not indicate that they participated in the change.}  in Mauwake became devoiced at some earlier stage, and the present-day voiced plosives are a later development. In the Kaukombaran languages voiced plosives are much more frequent than in Mauwake, and there is a clear sound correspondence between many cognates:\footnote{The Kaukombaran data is from \citet{LoewekeEtAlms} and \citet{ZGraggen1980}.} 

{\bfseries
Mauwake  Miani  Maia  Pila}

paapa\footnote{In the comparison the cognates are listed in phonetic form but without the brackets; the phonemic representation is basically the same.}   baba  bab  mbab  `elder sibling'

pok-  bug-  buge-  buge-  `sit'

perek-  bereg-  bered-  buroaind-  `tear (v.)'

kemena  kema  goama  {\ng}goama  `inside'

kukusa  gugun    gugut  `shadow, picture'

I suggest that /{\textphi}/ in Mauwake is a result of a sound change whereby /w/ in certain positions became devoiced and changed into a fricative. This can be seen in the sound correspondences in cognate words in related Kaukombaran languages.\footnote{Note that within the Kaukombaran group there has also been change from /w/ into /b/.  Another possibility is that /b/ has first changed into /w/ and further into /{\textphi}/ in Mauwake, but that seems less likely because there are numerous other words with /b/ which do not participate in this sound change.} 

{\bfseries
Mauwake  Miani  Maia  Pila}

a{\textphi}ila  abir  koawir  kuawir  `grease'

a{\textphi}ura  ab  kab  kap  `lime'

i{\textphi}era  ibor  ibor  iwor  `sea'

uru{\textphi}-  ruw-  uruw-    `see'

{\textphi}ar-  bar-  war-    `call'

u{\textphi}-    uw-  ube-  waguwa-  `dance'

\subsubsection{Vowels}
\hypertarget{RefHeading18741935131865}{}
There is variation in the Papuan languages from the 3-vowel systems in Ndu languages to an 8-vowel system in Vanimo. The basic and a very common one is a 5-vowel system \citep[49-54]{Foley1986}, also the most common worldwide \citep[126]{Maddieson1984}.  It is employed by Mauwake as well, and the vowels are the ones that \citet[125]{Maddieson1984} lists as the most common vowels universally. 





\begin{tabular}{llll}
\mytoprule
 & Front & Central & Back\\
High & i &  & u\\
Mid & e &  & o\\
Low &  & a & \\
\mybottomrule
\end{tabular}



\begin{table}
\caption{Vowel phonemes}
\label{tab:3}
\end{table}

The five vowel phonemes are voiced and oral. They contrast as to front, central and back points of articulation. Front and back vowels also have a high vs. mid contrast. There is only one set of mid vowels in Mauwake, which are phonetically between the IPA higher and lower mid vowels.  For the sake of simplicity, I have represented them with the IPA symbols for higher mid vowels, /e/ and /o/ .\footnote{To distinguish the true mid vowels from higher mid vowels \citet[123]{Maddieson1984} writes them with quote marks: ``e'' and ``o''.}  Both the front vowels are unrounded and the back vowels rounded.

The mid vowels could also be analysed as non-high vowels together with the low central vowel /a/, thus simplifying the chart, since there are no front or back low vowels. That grouping \textstyleEmphasizedWords{is} actually used in \sectref{sec:2.3.3}, where it simplifies the past tense suffix rule. But the distributional fact that there are no vowel glides beginning with either /e/ or /o/ justifies distinguishing them as a separate group of mid vowels.  

The high vowels /i/ and /u/ have an open allophone, [{\textsci}] and [{\textupsilon}] respectively, following a word-initial consonant and preceding a central vowel /a/. In other positions they have a more closed allophone [i] and [u]. The other vowels do not have allophonic variation.

V    $\rightarrow $  V  /  \_  V

+ high  + high      +central

+ close  + open  

/ikina/  [i{{\textprimstress}kina}]  `smell'

/lali/  [la{{\textprimstress}li}]  `small fish'

/mia/  [m{\textsci{{\textprimstress}}}a]  `body'

/uruwa/  [u{{\textprimstress}ruwa}]  `loincloth'

/lalu/  [la{{\textprimstress}lu}]  `parrotfish'

/mua/  [m{\textupsilon{{\textprimstress}}}a]  `man'

The vowels contrast word-initially, -medially and -finally:

/a{\textphi}a/  [a{{\textprimstress}}{\textphi}a]  `flying fox'

/e{\textphi}a/  [e{{\textprimstress}}{\textphi}a]  `me'

/i{\textphi}a/  [i{{\textprimstress}}{\textphi}a]  `snake'

/o{\textphi}a/  [o{{\textprimstress}}{\textphi}a]  `colour'

/u{\textphi}a/  [u{{\textprimstress}}{\textphi}a]  `swing (n.)'

/marari/  [ma{{\textprimstress}rari}]  `temporary (shelter)'

/maremuka/  [ma{{\textprimstress}remuka}]  `corn (med.)'

/marija/  [ma{{\textprimstress}riya}]  `he/she scrapes'

/maroka/  [ma{{\textprimstress}roka}]  `prawn'

/saruwa/  [sa{{\textprimstress}ruwa}]  `tree sp.'

/popoka/  [po{{\textprimstress}poka}]  `unripe fruit'

/ooke/  [{{\textprimstress}o:ke}]  `follow him!'

/loloki/  [lo{{\textprimstress}loki}]  `plant sp.'

/papako/  [pa{{\textprimstress}pako}]  `some'

/ooku/  [{{\textprimstress}o:ku}]  `let's (dual) follow him!'

Phonemic vowel length only occurs in word-initial syllables.  Long vowels are interpreted as two vowels of the same quality for the following reasons:


\begin{itemize}
\item Other vowel sequences are common in Mauwake;

\item The quality of the long and short vowel is the same;

\item Economy of description: there are five vowels instead of ten.


\end{itemize}
Long and short vowels contrast with each other:

/aasa/  [{{\textprimstress}a:sa}]  `canoe'

/asa/  [a{{\textprimstress}sa}]  `wild \textstyleForeignWords{galip} nut'

/peela/  [{{\textprimstress}pe:la}]  `rotten'

/pela/  [pe{{\textprimstress}la}]  `leaf'

/kiira/  [{{\textprimstress}ki:r}a]  `side, shin'

/kira/  [ki{{\textprimstress}ra}]  `wild sugarcane'

/{\textphi}uura/  [{{\textprimstress}}{\textphi}u:ra]  `steep'

/{\textphi}ura/  [{\textphi}u{{\textprimstress}ra}]  `knife'

\subsubsection{Suprasegmentals: stress and intonation}
\hypertarget{RefHeading18761935131865}{}
Since Mauwake is not a tonal language, the only suprasegmentals discussed here are stress and intonation.

\paragraph[Stress]{Stress}
\hypertarget{RefHeading18781935131865}{}
Stress is not phonemic in Mauwake, but three degrees of phonetic stress are discernible in a word.  Primary stress is marked by greater intensity, higher fundamental frequency and often, but not always, by non-phonemic lengthening of the vowel.  An ``unstressed'' syllable is considerably weaker, but the vowels still retain their essential quality.  A syllable with a secondary stress is weaker than one with primary stress, but stronger than an unstressed syllable. Since stress is a defining factor on the word level, it is discussed further in section 2.3.

Stress has a pragmatic function on clause and sentence level. The clausal stress manifests itself in slightly greater loudness and intensity than that of the ordinary word stress, and its default position is the verb or the non-verbal predicate. In a multi-clause sentence the final verb typically receives the strongest clausal stress; this may be called sentence stress if it needs to be distinguished from the clausal stress of the non-final clauses. The position of the clausal stress may be shifted to give added prominence to some element in the clause (9.2.3). When this is done, the loudness and intensity of the stressed syllable are increased, and non-phonemic lengthening of the vowel may take place.

\paragraph[Intonation]{Intonation}
\hypertarget{RefHeading18801935131865}{}
The three grammatical units important from the point of view of intonation are a phrase, a (non-final) clause and a sentence. All final clauses are here treated as sentences.

Pitch variations in Mauwake are not very prominent, and in general the register is quite low compared e.g. with English. There is more register variation in the inland than on the coast. 

The most common sentence intonation contour is falling. The first stressed syllable is the highest; after it the intonation falls very gradually until the word with the sentence stress, typically the final verb. There is a slight rise at the syllable with the sentence stress, and then a very sharp fall in the terminal contour.  This same basic pattern occurs both in statements (\stepcounter{nx}{\thenx}), commands (\stepcounter{nx}{\thenx}), in non-polar questions (\stepcounter{nx}{\thenx}) and certain polar questions (\stepcounter{nx}{\thenx}). (In the following examples, the word with the sentence stress is bolded.)


%\includegraphics[width=\textwidth]{a2baed703b528405dbd2e29ffc5720026-img8.jpg}
 

\ea%x898
\label{ex:x898}
\gll [jo  mo'ma  \textstyleEmphasizedVernacularWords{e'nim-i-jem}] \\
      \\
\glt
\z

I  taro  eat-Np-PR.1s

`I (am) eat(ing) taro.'

In commands the intonation contour is very much the same as in a statement, but the pronunciation is phonetically more tense.


%\includegraphics[width=\textwidth]{a2baed703b528405dbd2e29ffc5720026-img9.jpg}
 

\ea%x1769
\label{ex:x1769}
\gll [mo'ma  \textstyleEmphasizedVernacularWords{e'nim-eka}] \\
      \\
\glt
\z

taro  eat-IMP.2p

`Eat (pl.) taro!'

In non-polar questions the sentence-final intonation is also falling. The stressed syllable of the question word carries the sentence stress if it is emphasized (\stepcounter{nx}{\thenx}), but often there is only a slightly higher rise than there would be in other words in the same position, and the sentence stress is placed on the stressed syllable of the final verb (\stepcounter{nx}{\thenx}), (\stepcounter{nx}{\thenx}).


%\includegraphics[width=\textwidth]{a2baed703b528405dbd2e29ffc5720026-img10.jpg}
 

\ea%x901
\label{ex:x901}
\gll ['mu:ka  'nain  \textstyleEmphasizedVernacularWords{mo'ram}  o'mom-i-ja] \\
      \\
\glt
\z

boy  that1  why  cry-Np-PR.3s

`\textit{Why} is that boy crying?'


%\includegraphics[width=\textwidth]{a2baed703b528405dbd2e29ffc5720026-img11.jpg}
 

\ea%x899
\label{ex:x899}
\gll [ma:  'mauwa  \textstyleEmphasizedVernacularWords{e'nim-i-n}] \\
      \\
\glt
\z

thing  what  eat-Np-PR.2s

`What are you eating?'


%\includegraphics[width=\textwidth]{a2baed703b528405dbd2e29ffc5720026-img12.jpg}
 

\ea%x900
\label{ex:x900}
\gll [m{\textupsilon}'a  'na:rewe=ke  \textstyleEmphasizedVernacularWords{e'kap-o-k}] \\
      \\
\glt
\z

man  who=CF  come-PA-3s

`Who came?'

The only instance where there can be any rising intonation sentence-finally is a polar question.  It is only used when the speaker is uncertain whether the answer is going to be affirmative or negative (\stepcounter{nx}{\thenx}).  The rise is on the question clitic -\textstyleStyleVernacularWordsItalic{i}.  


%\includegraphics[width=\textwidth]{a2baed703b528405dbd2e29ffc5720026-img13.jpg}
 

\ea%x902
\label{ex:x902}
\gll ['auwa  \textstyleEmphasizedVernacularWords{e'kap-o-k=i}  ] \\
      \\
\glt
\z

father  come-PA-3s=QM

`Did father come?' 

If the speaker strongly expects the answer to agree with the polarity of the question, the intonation is falling (\stepcounter{nx}{\thenx}). Since polar questions are are also marked with a question marker \textstyleStyleVernacularWordsItalic{=}\textstyleStyleVernacularWordsItalic{i} sentence-finally, a separate intonation pattern is partly redundant.


%\includegraphics[width=\textwidth]{a2baed703b528405dbd2e29ffc5720026-img14.jpg}
 

\ea%x903
\label{ex:x903}
\gll ['auwa  \textstyleEmphasizedVernacularWords{e'kap-o-k=i}  ] \\
      \\
\glt
\z

father  come-PA-3s=QM

`Did father come?' (Expecting ``yes'' as an answer.)

The intonation pattern in medial clauses, instead of falling at the end, is either level or slightly rising.  The more expected the sequence, the more level the intonation is. In (\stepcounter{nx}{\thenx}) the two clauses are part of an ``expectancy chain'', because coconuts are scraped only for preparing food. (In the following three examples, the medial and subordinate clauses are bolded rather than the verb of the finite clause.)


%\includegraphics[width=\textwidth]{a2baed703b528405dbd2e29ffc5720026-img15.jpg}
   

\ea%x904
\label{ex:x904}
\gll [\textstyleEmphasizedVernacularWords{i'wera  mu-'ep}  maa  'uup-i-nen] \\
      \\
\glt
\z

coconut  scrape-SS.SEQ  food  cook-Np-FU.1s

`I will scrape a coconut and cook food.'

But (\stepcounter{nx}{\thenx}) tells about an unexpected event, a person finding a turtle when he had just gone fishing; instead of catching it he might have either chosen to leave it or failed to catch it.


%\includegraphics[width=\textwidth]{a2baed703b528405dbd2e29ffc5720026-img16.jpg}
 

\ea%x905
\label{ex:x905}
\gll [\textstyleEmphasizedVernacularWords{pon  u'ru{\textphi}-ap  'a:w-ep}  p-e'kap-e-m] \\
      \\
\glt
\z

turtle  see-SS.SEQ  take-SS.SEQ  BPx-come-PA-1s

`I saw a turtle, caught it and brought it (here).'

The rising intonation is more common in subordinate clauses; in conditional clauses (\stepcounter{nx}{\thenx}) it is particularly noticeable.  As a rule, the more important the speaker considers the clause as a presupposition for the main clause, the more clearly there is an intonational rise clause-finally.  


%\includegraphics[width=\textwidth]{a2baed703b528405dbd2e29ffc5720026-img17.jpg}
 

\ea%x906
\label{ex:x906}
\gll [\textstyleEmphasizedVernacularWords{i'{\textphi}a  u'ru{\textphi}-i-nen=na}  ke'ker  o'p-i-nen] \\
      \\
\glt
\z

snake  see-Np-FU.1s=TP  fear  hold-Np-FU.1s

`If I see a snake, I will be afraid.'

A phrase that is fronted as a left-dislocated theme (9.1) also has a rising intonation at the end of the phrase.  The phrase, bolded in the following example, occurs at the beginning of the clause.  The slash indicates a pause.


%\includegraphics[width=\textwidth]{a2baed703b528405dbd2e29ffc5720026-img18.jpg}
 

\ea%x907
\label{ex:x907}
\gll [\textstyleEmphasizedVernacularWords{'jos=na}/  o'wow  ma'neka  me  i'kiw-i-jem] \\
      \\
\glt
\z

1s.FC=TP  village  big  not  go-Np-PR.1s  

`As for me, I don't go to town.'

In listing, the intonation rises very slightly at the final syllable of each non-final phrase listed, or is retained at the same level as the previous syllable(s).


%\includegraphics[width=\textwidth]{a2baed703b528405dbd2e29ffc5720026-img19.jpg}
 

\ea%x908
\label{ex:x908}
\gll [ma:  u'nowa  se'senar-e-m/  \textstyleEmphasizedVernacularWords{o'wora}/  \textstyleEmphasizedVernacularWords{a'{\textphi}ura}/  \textstyleEmphasizedVernacularWords{e'pisowa}/  a'ria  \textstyleEmphasizedVernacularWords{mo'ma}] \\
      \\
\glt
\z

thing  many  buy-PA-1s  betelnut  lime  tobacco  alright  taro

`I bought many things: betelnut, lime, tobacco and taro.'

A polite way of calling a person, of getting someone's attention, is to call the name or relationship term in such a way that the stressed syllable has a slight rise and a sharp fall in pitch, and following unstressed syllables, if any, have a low pitch.


%\includegraphics[width=\textwidth]{a2baed703b528405dbd2e29ffc5720026-img20.jpg}
 

\ea%x909
\label{ex:x909}
\gll [e'remena] \\
      \\
\glt
\z

nephew

`Nephew!'

An impatient or exasperated call, or a call for someone distant, has a different pattern.  The voice is louder, the pitch is retained relatively high and level, and the last syllable gets lengthened and, if unstressed, receives a stress almost as strong as that of the stressed syllable.


%\includegraphics[width=\textwidth]{a2baed703b528405dbd2e29ffc5720026-img21.jpg}
 

\ea%x910
\label{ex:x910}
\gll ['aiteeee] \\
      \\
\glt
\z

mother

`Mother!'

Anger is typically expressed by shouting.  The intonation stays fairly level, and the sentence is short and produced in a staccato manner.

Disgust or impatience is expressed by sentence-final interjection \textstyleStyleVernacularWordsItalic{yaa} [ja:], which retains a fairly level pitch and can be lengthened considerably.  An impatient reaction to someone else's words or actions is expressed by sentence-initial interjection \textstyleStyleVernacularWordsItalic{se}, which has a very sharp falling intonation.


%\includegraphics[width=\textwidth]{a2baed703b528405dbd2e29ffc5720026-img22.jpg}
 

\ea%x911
\label{ex:x911}
\gll [i'kiw-eka  jaaaa] \\
      \\
\glt
\z

go-IMP.2p  INTJ

`Go, for heaven's sake!'


%\includegraphics[width=\textwidth]{a2baed703b528405dbd2e29ffc5720026-img23.jpg}
 

\ea%x912
\label{ex:x912}
\gll [se  na:p  \textstyleEmphasizedVernacularWords{me}  'ma-e] \\
      \\
\glt
\z

INTJ  thus  not  say-IMP.2s

`Goodness, don't say like that.'

\subsubsection{Orthographic symbols}
\hypertarget{RefHeading18821935131865}{}
\tableref{tab:1} shows the orthographic symbols for the phonemes. The semivowel /j/ is written as \textstyleEmphasizedWords{y} due to the influence of Tok Pisin and English.  Because the orthography represents the phoneme inventory so closely, it is the orthographic symbols that are used in the vernacular examples throughout this thesis after the phonology chapter.





\begin{tabular}{lllllllllllllll}
\mytoprule

Consonant phonemes & p & t & k & b & d & g & m & n & {\textphi} & s & l & r & w & j\\
Orthographic representation & p & t & k & b & d & g & m & n & f & s & l & r & w & y\\
Vowel phonemes & i & e & a & o & u & \multicolumn{9}{l}{}\\
Orthographic representation & i & e & a & o & u & \multicolumn{9}{l}{}\\
\mybottomrule
\end{tabular}



\begin{table}
\caption{Orthographic symbols for Mauwake phonemes}
\label{tab:4}
\end{table}

\subsection{Syllables and phonotactics}
\hypertarget{RefHeading18841935131865}{}
\subsubsection{Syllable patterns}
\hypertarget{RefHeading18861935131865}{}
The syllable in Mauwake consists of one or two vowels forming the nucleus, with optional onset and/or coda of one consonant, \textstyleAcronymallcaps{CV} being by far the most frequent syllable structure.\footnote{\citet[13]{Reesink1986} gives a short but good overview of syllable-final consonants in a number of TNG languages.} The syllable patterns are as follows:

V    VC

CV    CVC

VV    VVC

CVV  CVVC

Any vowel can fill the simple nucleus slot of the syllable. The complex nucleus slot is filled either by a geminate vowel or a diphthong. Diphthongs can occur in non-initial syllables too, but geminate vowels cannot. 

Any consonant can fill the onset slot, and all consonants except the voiced plosives, /{\textphi/} and /y/ can fill the coda slot of a syllable. The distribution of the voiced plosives and /{\textphi}/ is also restricted in that they very seldom occur later than in the second syllable of a word and, except for /y/, do not appear in inflectional morphology. \tableref{tab:1} shows the possible distribution of consonants in a syllable.

CV(V)C    CV(V)C

p  +  +  n  +  +

t  +  +  {\textphi}  +  --

k  +  +  s  +  +

b  +  --  l  +  +

d  +  --  r  +  +

g  +  --  w  +  +

m  +  +  j  +  --


\begin{table}
\caption{Consonant distribution in a syllable }
\label{tab:5}
\end{table}

\subsubsection{Vowel sequences}
\hypertarget{RefHeading18881935131865}{}
\tableref{tab:1} shows the possible two-vowel sequences in Mauwake. The only possible sequences beginning with the either of the two mid vowels are geminate vowels; no other vowel sequences begin with a mid vowel. The other three vowels may combine with any vowel.





\begin{tabular}{lllll}
\mytoprule

ii &  & ai &  & ui\\
ie & ee & ae &  & ue\\
ia &  & aa &  & ua\\
io &  & ao & oo & uo\\
iu &  & au &  & uu\\
\mybottomrule
\end{tabular}



\begin{table}
\caption{Vowel sequences}
\label{tab:6}
\end{table}

When the second vowel in a vowel sequence is articulatorily the same height or higher than the preceding vowel, the two form a diphthong , i.e. they are part of the same syllable.

/kae/  [{{\textprimstress}kae}]  `my/our grandfather'

/kuina/  [{{\textprimstress}kui.na}]  `woodborer'

/aowa/  [{{\textprimstress}ao.wa}]  `to tie around waist'

When the second vowel is lower than the first, the two vowels form the nuclei of two separate syllables. 

/sier/  [s{\textsci}.{{\textprimstress}er}]  `husking stick'

/luaka/  [l{\textupsilon}.{{\textprimstress}a.ka}]  `whitebait'

/kia/  [k{\textsci}.{{\textprimstress}a}]  `white'

The high back vowel /u/ is considered lower than the high front vowel /i/, as it behaves similarly to the non-high vowels when following /i/.

/niuk/  [n{\textsci}.{{\textprimstress}uk}]  `let them give you'

In an open syllable, all the diphthongs allowed by the language are possible. In a closed syllable, /ao/ is the only diphthong that has not been found; but it is very infrequent in an open syllable too. 

Sequences with three vowels are rare: /uau/ and /uai/ are the only ones I have found, and these only occur at morpheme breaks (marked with a hyphen in the examples), and there is a syllable break within the sequence as well.\footnote{A syllable break does not need to coincide with a morpheme break; in the examples above it does not.}  

/kua-i-jem/    [k{\textupsilon}.{{\textprimstress}}ai.jem]      `I build'

/kua-uk/      [k{\textupsilon}.{{\textprimstress}a}uk]      `let them build'

\subsubsection{Consonant sequences}
\hypertarget{RefHeading18901935131865}{}
No consonant sequences occur word-initially or -finally. In words with three or more syllables there are some word-medial clusters, which I believe to have resulted from vowel elision.  A vowel may be elided from a non-final syllable immediately following a stressed syllable, which is probably the least prominent syllable in the whole word.\footnote{According to \citet[11]{Sommerstein1977} this is a common process in languages.} It is mainly the high vowels that are dropped, since they are the least sonorant. In the following examples, vowel undergoing elision is underlined in the phonemic representation.

/ikemika/  [i{{\textprimstress}kemka}]  `wound (nn)'

/aakisa/  [{{\textprimstress}a:ksa}]  `now, today'

/pisikulaw/  [pi{{\textprimstress}siklaw}]  `grasshopper sp.'

A non-high vowel can also be elided if the adjacent stressed syllable has an identical vowel:

/kerekenam/  [ke{{\textprimstress}reknam}]  `dollar bird'

/toonowaw/  [{{\textprimstress}to:nwaw}]  `honey eater'

Occasionally vowel elision takes place in a later syllable than that immediately following the stressed syllable: 

/o{\textphi}a{\textphi}ilika/  [o{{\textprimstress}}{\textphi}a{\textphi}ilka]  `butterfly'

/aakuniwikin/  [a:kuniwkin]  `talk.2/3p.DS'

In some of these words the original vowel can still be perceived in slow pronunciation, but in others it has disappeared. Consequently, phonemic vowel clusters are currently developing in Mauwake,\footnote{The present orthography reflects this development in that consonant clusters are written especially 1) where the quality of the elided vowel cannot be established, and/or 2) when the elided form is in very frequent use.} and the distribution of \textstyleAcronymallcaps{CVC} syllables is being extended to include word-medial position as well, and that of \textstyleAcronymallcaps{VVC} and \textstyleAcronymallcaps{CVVC} to include initial position in two-syllable words that have earlier had three syllables (see \sectref{sec:2.3.2}).

No clear rules have been found for the site of the vowel elision, but some tendencies are as follows. Nouns have more elision than verb stems. A vowel is dropped much more often between non-homorganic than homorganic consonants. The voiceless velar plosive /k/ is the most frequent phoneme on either side of the elided vowel.

\subsection{Word} 
\hypertarget{RefHeading18921935131865}{}
\subsubsection{Defining a phonological word in Mauwake}
\hypertarget{RefHeading18941935131865}{}
A phonological word is defined on the basis of a primary stress. Words are composed of one or more syllables. The number of syllables seldom exceeds ten, but compound words can be longer. A majority of the words have two or three syllables.  Every word has one syllable with a primary stress, and usually one or more unstressed syllables. 

In words of two or more syllables, the syllable containing the second vowel is stressed. Thus the first syllable is stressed if it contains a geminate vowel or a diphthong. In all the other cases the second syllable is stressed. When the stressed syllable is long, the stress falls equally on the whole vowel sequence. 

/aasa/  [{{\textprimstress}a:.sa}]  `canoe'

/kuija/  [{{\textprimstress}kui.ja}]  `it bites'

/a{\textphi}ura/  [a.{{\textprimstress}}{\textphi}u.ra]  `lime'

/siowa/  [si.{{\textprimstress}o.wa}]  `dog'

/isaimija/  [i.{{\textprimstress}sai.mi.ja}]  `(s)he heats (food)'

Both derivational and inflectional affixes may receive primary stress provided they are in a position where stress is normally placed:

/aw-om-e/  [a.{{\textprimstress}wo.me}]  `weave it for me'

weave-BEN-BNFY1.IMP.2s

/um-o-k/  [u.{{\textprimstress}mok}]  `he/she died'

die-PA-3s

Clitics, on the other hand, never receive stress placement. Grammatically they are words, but phonologically they attach to the preceding word. If the preceding word is monosyllabic and has a short vowel, it still takes the primary stress when a clitic is added. The unmarked pronouns are a case in point: they retain their stress when clitics are added. Some non-phonemic lengthening takes place in the vowel of the pronoun stem.

/jo=ko/  [{{\textprimstress}}jo{\.{}.ko}]  `\textit{I} '

/jos=ke/  [{{\textprimstress}}jo{\.{}s.ke}]  `\textit{I} (and not someone else)' 

Compound words and some reduplicated words also have a secondary stress.  In the second (and third) compound of a compound word, that syllable has a secondary stress which in a single word would receive primary stress: 

/soomare-jiawem-ikemik/  [{{\textprimstress}so:mare-j}{\textsci}{{\textprimstress}}{{\textprimstress}}awem-i{{\textprimstress}}{{\textprimstress}}kemik]  `we were walking around'

/suuw-orom-ikua/  [{{\textprimstress}}su:w-o{{\textprimstress}}{{\textprimstress}}rom-i{{\textprimstress}}{{\textprimstress}}kua]  `he is pushing it down'

In those words where a long initial syllable is reduplicated as a whole, the second syllable is also long and receives a secondary stress:

/kui-kuisow/  [{{\textprimstress}}kui.{{\textprimstress}}{{\textprimstress}}kui.sow]  `a few'

/suu-suusia/  [{{\textprimstress}}su:.{{\textprimstress}}{{\textprimstress}}su:.sia]  'thorny'

\subsubsection{Distribution of syllables in a word}
\hypertarget{RefHeading18961935131865}{}
All syllable types except \textstyleAcronymallcaps{VC} can form a monosyllabic word. In polysyllabic words, the occurrence of a certain syllable type is determined by both its position in the word and the stress. 

\tableref{tab:1} shows what syllable types occur in which positions in a word.  A blank space indicates that the syllable type does not occur in that particular word position at all, and parentheses indicate a rare occurrence. Double parentheses indicate new positions for closed syllables formed as a result of vowel elision (see \sectref{sec:2.2.3}). 





\begin{tabular}{lllllllllll}
\mytoprule


Syllable type & \multicolumn{3}{l}{Stressed syllables}

 &  & \multicolumn{4}{l}{Unstressed syllables}

 &  & The only syllable\\
 & Initial & 2nd\footnotemark{} & Final &  & Initial & 2nd & 3rd- & Final &  & \\
\hhline{----~----~-}
V &  & +\par & +\par &  & +\par &  & +\par & +\par &  & +\par\\
\hhline{----~----~-}
CV &  & +\par & +\par &  & +\par & +\par & +\par & +\par &  & +\par\\
\hhline{----~----~-}
VV & +\par & (+)\par & (+)\par &  &  &  & (+)\par & +\par &  & +\par\\
\hhline{----~----~-}
CVV & +\par & +\par & (+)\par &  &  &  & +\par &  &  & +\par\\
\hhline{----~----~-}
VC &  &  & +\par &  &  &  &  & +\par &  & \\
\hhline{----~----~-}
CVC &  & ((+))\par & +\par &  &  & ((+))\par & ((+))\par & +\par &  & +\par\\
\hhline{----~----~-}
VVC & ((+))\par &  & +\par &  &  &  &  & +\par &  & +\par\\
\hhline{----~----~-}
CVVC & ((+))\par &  & +\par &  &  &  &  & +\par &  & +\par\\
\hhline{----~----~-}

\mybottomrule
\end{tabular}


\footnotetext{ `2nd' indicates the second non-final syllable, and `3rd-' stands for the third or later non-final syllable in a polysyllabic word.}

\begin{table}
\caption{Distribution of syllable types}
\label{tab:7}
\end{table}

Some distributional characteristics can be summarised as follows.  The most frequent syllable type, \textstyleAcronymallcaps{CV}, also has the widest distribution: a stressed initial syllable is the only position where it cannot occur, as an initial syllable with a single short vowel is always unstressed. The same reason accounts for the absence of V syllables in the same position. V syllables also never occur after a geminate vowel or diphthong, so they cannot occupy the second unstressed syllable position. A \textstyleAcronymallcaps{VV} syllable in medial or final position is possible but very rare. The two previous statements may be combined to make the claim that there is some resistance towards \textstyleAcronymallcaps{VVV} sequences in Mauwake. The syllables with a consonant coda only occur word finally, except where vowel elision has changed the syllable structure.  

\subsubsection{Morphophonology}
\hypertarget{RefHeading18981935131865}{}
There are not many morphophonological alternations in Mauwake.  The most important is the rule system governing the vowel of the past tense suffix and the medial verb same-subject sequential action and simultaneous action suffixes (called the medial verb suffixes\footnote{There are also other medial verb suffixes, which are not  affected by these morphophonological rules.} in the discussion below).  Others include the change in the verbaliser suffix and the form of the completive aspect marker.

\paragraph[Elision of word-final vowel]{Elision of word-final vowel}
\hypertarget{RefHeading19001935131865}{}
The phoneme /a/ has a very high frequency as the word-final phoneme, particularly in the nouns and adjectives.  It accounts for approximately 85\% of all the vowel-final words.  In normal and fast speech this /a/ is often dropped from an unstressed word-final CV syllable, especially when followed by a word with an initial vowel. 

V    $\rightarrow $    {\O}  /  C \_  \#  V

+central

-- stress

/koora unowa/  [{{\textprimstress}ko:r  u{{\textprimstress}}nowa}]  `many houses'

/takira {\textphi}aara/  [ta{{\textprimstress}kir  {{\textprimstress}}}{\textphi}a:ra]  `boys' house'

/siiwa eliwa/  [{{\textprimstress}si:w  e{{\textprimstress}}liva}]  `good/bright moon'

/ikoka uura/  [i{{\textprimstress}kok  {{\textprimstress}}u:ra}]  `later at night'

In some cases even a stressed /a/ is elided, and the stress moves to the following vowel in the utterance.  This mainly happens with the accusative pronouns, which tend towards cliticization (\sectref{sec:3.5.3}).

/me ne{\textphi}a uru{\textphi}am/  [{{\textprimstress}me ne}{\textphi} {{\textprimstress}uru}{\textphi}am]  `I didn't see you'

In compound words the final /a/ is dropped from the first constituent even when the second begins with a consonant, except when the final syllable of the first constituent is stressed.  

/aara muuka/  [{{\textprimstress}a:r {{\textprimstress}}mu:ka}]  `chick'

/emera tapaka/  [e{{\textprimstress}mer ta{{\textprimstress}}paka}]  `sago cake'

/mera soo/  [me{{\textprimstress}ra {{\textprimstress}}so:}]  `fish trap'

\paragraph[Reduplication]{Reduplication}
\hypertarget{RefHeading19021935131865}{}
There are various patterns of reduplication in Mauwake. With a few exceptions, reduplication takes place at the beginning of the word.  The meaning involves plurality in one way or another; with verbs it indicates repeated action and/or the object of the action ending up in several pieces. Occasionally with adjectives it also indicates enhanced quality (\sectref{sec:3.3}).

How a word is reduplicated can to some extent be predicted from the phonological shape of the word.  Type 1 below is the most common, 2 and 3 are the only possible ones for the words with a short and a long initial vowel respectively.  Reduplication process does not always respect syllable boundaries.

\subparagraph[Type 1]{Type 1}
\hypertarget{RefHeading19041935131865}{}
Everything up to and including the first vowel of the stressed syllable is reduplicated.  Even with the reduplication these words retain the normal stress pattern: the second syllable of the reduplicated form is stressed, because regardless of whether one or two syllables are reduplicated the first syllable in this type is always short. When two of syllables are reduplicated, the originally stressed syllable of the word root gets a secondary stress.

/pu-puukija/  [pu.{{\textprimstress}pu:.ki.ja}]  `cut into pieces'

/pu-puija/  [pu.{{\textprimstress}pui.ja}]  `break into pieces'

/pere-perekija/  [pe.{{\textprimstress}}re.pe.{{\textprimstress}}{{\textprimstress}}re.ki.ja]  `tear into pieces'

/kiri-kiripija/  [ki.{{\textprimstress}}ri.ki.{{\textprimstress}}{{\textprimstress}}ri.pi.ja]  `turn round \& round, mix'

/mane-maneka/  [ma.{{\textprimstress}}ne.ma.{{\textprimstress}}{{\textprimstress}}ne.ka]  `(many) big (things)'

\subparagraph[Type 2:  V1C1V1 - V1C1V2C2V(V3)X]{Type 2:  V\textsubscript{1}C\textsubscript{1}V\textsubscript{1} - V\textsubscript{1}C\textsubscript{1}V\textsubscript{2}C\textsubscript{2}V(V\textsubscript{3})X}
\hypertarget{RefHeading19061935131865}{}
In the words of the second type, the reduplication repeats the initial vowel and consonant of the word root, adding another vowel of the same quality after the consonant.  In these words the stress shifts from the second syllable of the root to the final vowel of the reduplicated element.  Phonetically this vowel usually merges into one with the following vowel, which always has the same quality.  Stresswise this creates an interesting pattern, where a syllable with a primary stress is followed by one with secondary stress.  Types 3 and 4 also have this kind of stress pattern.

/ele-eliwa/  [e.{{\textprimstress}}le.{{\textprimstress}}{{\textprimstress}}li.wa]  `(many) good (things)'

/ara-arow/  [a.{{\textprimstress}}ra.{{\textprimstress}}{{\textprimstress}}row]  `in threes'

/oko-okaiwi/  [o.{{\textprimstress}}ko.{{\textprimstress}}{{\textprimstress}}kai.wi]  `this side and that'

\subparagraph[Type 3:  V1V1C1 - V1V1C1V2X]{Type 3:  V\textsubscript{1}V\textsubscript{1}C\textsubscript{1} - V\textsubscript{1}V\textsubscript{1}C\textsubscript{1}V\textsubscript{2}X}
\hypertarget{RefHeading19081935131865}{}
A very small group of words has this type of reduplication, where the initial geminate vowel and the following consonant are reduplicated.  The result is a word where both the first and the second syllable have a complex nucleus, a word type not allowed in the simple non-reduplicated words.  In these reduplicated words the first syllable receives primary stress and the second syllable secondary stress.  The first syllable has a syllable pattern (VVC) which is not possible for the first syllable in a non-reduplicated polysyllabic word.

/iiw-iiwa/  [{{\textprimstress}}i: v.{{\textprimstress}}{{\textprimstress}} i:.va]  `(many) short (things)'

/iin-iinan/  [{{\textprimstress}}i: n.{{\textprimstress}{{\textprimstress}} i:.nan}]  `(the things) high up'

\subparagraph[Type 4:  C1V1V2 - C1V1V2X]{Type 4:  C\textsubscript{1}V\textsubscript{1}V\textsubscript{2} - C\textsubscript{1}V\textsubscript{1}V\textsubscript{2}X}
\hypertarget{RefHeading19101935131865}{}
In this type the long first syllable is repeated entirely, but nothing else.  The two vowels in the initial syllable may be identical or different in quality. This is not a very common pattern.

/kui-kuisow/  [{{\textprimstress}}kui.{{\textprimstress}}{{\textprimstress}}kui.sow]  `a few'

/soo-soomarija/  [{{\textprimstress}}so:.{{\textprimstress}}{{\textprimstress}}so:.ma.ri.ja]  `amble, stroll'

\subparagraph[Unusual reduplications]{Unusual reduplications}
\hypertarget{RefHeading19121935131865}{}
The word \textstyleStyleVernacularWordsItalic{gelemuta} `small' has two unusual reduplicated forms, where the end of the word is changed: \textstyleStyleVernacularWordsItalic{gelemutitik} and \textstyleStyleVernacularWordsItalic{gelemutumut} `(many) small (things)'.  Type 1 reduplication rule can also be applied to these already reduplicated forms, although not to the root.

/gele-gelemutitik/  [ge.{{\textprimstress}}le.ge.{{\textprimstress}}{{\textprimstress}}le.mu.ti.tik]  `very small (pl.)'

*/gele-gelemuta/

The verb \textstyleStyleVernacularWordsItalic{wafuriya} `throw' also has an irregular reduplicated form: only the second syllable is reduplicated.

/wa{\textphi}u{\textphi}urija/  [va.{{\textprimstress}}{\textphi}u.{\textphi}u.ri.ja]  `throw around'

The reduplication for the word \textstyleStyleVernacularWordsItalic{owowa} `village' occurs at the beginning of the word, but it does not follow any of the patterns above. So far it is the only one of its kind found.

/owow-owowa/  [o.{{\textprimstress}}wo.wo{{\textprimstress}}{{\textprimstress}}wo.wa]  `(many/all) villages'

Mauwake has a number of nouns of the following the pattern C\textsubscript{1}V\textsubscript{1}C\textsubscript{2}V\textsubscript{2} C\textsubscript{1}V\textsubscript{1}C\textsubscript{2} , which looks like reduplication, but with the word-final vowel deleted.  However, these words do not have any semantic relationship with a corresponding C\textsubscript{1}V\textsubscript{1}C\textsubscript{2}V\textsubscript{2} word in cases where the latter may exist.  Words of this type are not considered to have resulted from reduplication.

/mulamul/  [mu{{\textprimstress}lamul}]  `trevally'

/jawejaw/  [ja{{\textprimstress}vejav}]  `hunting magic'

Similarly, words of the pattern C\textsubscript{1}V\textsubscript{1} C\textsubscript{1}V\textsubscript{1} C\textsubscript{2}V\textsubscript{2 } are not considered reduplicated forms.  Firstly, there is no semantic relationship with a corresponding C\textsubscript{1}V\textsubscript{1} C\textsubscript{2}V\textsubscript{2 } word, even if the latter exists. Secondly, in Mauwake there is a very strong tendency to have the same vowel in the first two syllables of trisyllabic or longer words, whether the consonant is the same or not. 

/momora/  [mo{{\textprimstress}mo.ra}]  `fool(ish)'

/sisina/  [si{{\textprimstress}si.na}]  `edge'

\paragraph[Past tense and medial verb suffixes]{Past tense and medial verb suffixes}\footnotemark{}
\hypertarget{RefHeading19141935131865}{}
\footnotetext{ Most of these rules were originally worked out by Kwan Poh San.}
There are three past tense verb suffixes for second and third person singular forms, -\textstyleStyleVernacularWordsItalic{a}, -\textstyleStyleVernacularWordsItalic{e} and -\textstyleStyleVernacularWordsItalic{o}.  Which one is chosen for which verb is determined mainly by the phonemes in the stem final syllable. 

The two basic allomorphs for the past tense suffix are \{-a\} and \{-E\}. /\textstyleStyleVernacularWordsItalic{-}o/ is a subgroup of the allomorph \{-E\}. The subgrouping is based on the fact that the \textstyleStyleVernacularWordsItalic{-a/}\textstyleStyleVernacularWordsItalic{-e} distinction runs through the whole past tense paradigms and occurs in the medial verb suffixes as well, whereas the \textstyleStyleVernacularWordsItalic{-}\textstyleStyleVernacularWordsItalic{e/-o} distinction only occurs in the second and third person past tense forms of some verbs.  According to the rounding rule below, \{-E\} is realized as /-o/ when both following a [+ labial] phoneme (either a labial consonant or the high rounded vowel /u/ ) and preceding a non-labial consonant. 

\{e\} {\textgreater} /o/  /  X \textsubscript{LAB}  \_  C\textsubscript{NON-LAB}

aaw-o-k  `(s)he got (it)'  cf.  aaw-e-m  `I got (it)'

mu-o-n  `you swallowed'  cf.  mu-e-m  `I swallowed'

The discussion below only mentions the past tense suffixes. The vowels in the  the medial verb suffixes are the same but do not have the allophonic variation between /-e/ and /-o/.  

The morphophonological rules governing the choice of past tense suffixes are listed in their order of relative strength, with regard to the number of cases in the data\footnote{The count included 273 verbs with the past tense suffix  -a,  364 with the suffix -e.} as well as the number of exceptions.

\textstyleEmphasizedWords{\textsc{Rule 1}}.  With a stem-final high vowel /i/ or /u/, the past tense suffix is always \{\nobreakdash-E\}.

/waki-\textstyleEmphasizedVernacularWords{e}-k/  `(s)he fell down'

/nepi-\textstyleEmphasizedVernacularWords{e}-k/  `(s)he raised animals'

/mu-\textstyleEmphasizedVernacularWords{o}-k/  `(s)he swallowed'

/karu-\textstyleEmphasizedVernacularWords{o}-k/  `(s)he ran'

\textstyleEmphasizedWords{\textsc{Rule 2}}.  With a stem-final alveolar nasal /n/, the suffix is nearly always /-e/.

/kekan-\textstyleEmphasizedVernacularWords{e}-k/  `it hardened'

/peren-\textstyleEmphasizedVernacularWords{e}-k/  `it tore'

/riirin-\textstyleEmphasizedVernacularWords{e}-k/  `(s)he laughed'

/solon-\textstyleEmphasizedVernacularWords{e}-k/  `it glided'

/uuwun-\textstyleEmphasizedVernacularWords{e}-k/  `(s)he chatted'

In the data there are 128 verb stems ending in /n/, and only 15 take the suffix \{\nobreakdash-a\}. In some cases there is a conflict between rules 2 and 3 (below), and 13 of those exceptions follow Rule 3.

\textstyleEmphasizedWords{\textsc{Rule 3.} } When the stem final syllable has a low vowel, there is dissimilation between the vowels in the stem final syllable and the past tense suffix. For these morphophonological rules the mid vowels are also considered low, so that there is height distinction only between high and low vowels.

X  V  (C)      +  \textbf{V}  C

    +low    +low

    $\alpha $ central    -$\alpha $ central

The past tense suffix tends to be \{-a\}, when the last vowel in the stem is /e/ or /o/. 

/aner-\textstyleEmphasizedVernacularWords{a}-k/  `(s)he aimed at'

/sirek-\textstyleEmphasizedVernacularWords{a}-k/  `it scratched'

/imen-\textstyleEmphasizedVernacularWords{a}-k/  `(s)he found

/on-\textstyleEmphasizedVernacularWords{a}-k/  `(s)he did/made'

/soop-\textstyleEmphasizedVernacularWords{a}-k/  `(s)he buried'

In words with /a/ as the last vowel in the stem, the past tense suffix tends to be \{\nobreakdash-E\}.\footnote{In the data there are 187 verbs that follow this rule and 19 that do not.}


/serak-\textstyleEmphasizedVernacularWords{e}-k/  `(s)he wiped'

/war-\textstyleEmphasizedVernacularWords{e}-k/  `(s)he killed it'

/ma-\textstyleEmphasizedVernacularWords{e}-k/  `(s)he said'

/ekap-\textstyleEmphasizedVernacularWords{o}-k/  `(s)he came'

/aaw-\textstyleEmphasizedVernacularWords{o}-k/  `(s)he got/took'

\textstyleEmphasizedWords{\textsc{Rule 4}}. When a high vowel is followed by a stem-final consonant /k/, /t/, /s/, /r/ or /l/, the past tense suffix is \{-a\}.  This group of consonants includes nearly all of the non-labial consonant phonemes; /n/ is handled in Rule 2, the voiced stops never occur stem-finally and /j/ hardly ever does.  

/puuk-\textstyleEmphasizedVernacularWords{a}-k/  `(s)he cut'

/mik-\textstyleEmphasizedVernacularWords{a}-k/  `(s)he speared'

/itit-\textstyleEmphasizedVernacularWords{a}-k/  `(s)he smashed'

/anetir-\textstyleEmphasizedVernacularWords{a}-k/  `(s)he tied'

/{\textphi}uur-\textstyleEmphasizedVernacularWords{a}-k/  `(s)he blew'

/a{\textphi}ilil-\textstyleEmphasizedVernacularWords{a}-k/  `it was sweet'

With the rest of the verbs, i.e. total of about 25\% of all the basic verbs, it is very difficult to find any rules governing the choice of the past tense suffix.

/tiim-\textstyleEmphasizedVernacularWords{a}-k/  [{{\textprimstress}ti:mak}]  `(s)he touched'

/aru{\textphi}-\textstyleEmphasizedVernacularWords{a}-k/  [a{{\textprimstress}ru}{\textphi}ak]  `(s)he hit'

/oosip-\textstyleEmphasizedVernacularWords{o}-k/  [{{\textprimstress}o:sipok}]  `(s)he sweated'

/{\textphi}iririm-\textstyleEmphasizedVernacularWords{o}-k/  [{\textphi}i{{\textprimstress}ririmok}]  `(s)he squeezed'

/u{\textphi}-\textstyleEmphasizedVernacularWords{o}-k/  [u{\textphi}{{\textprimstress}ok}]  `(s)he danced'

/iw-\textstyleEmphasizedVernacularWords{o}-k/  [iw{{\textprimstress}ok}]   `(s)he gave him/her'

A few verbs apparently have dropped the past tense suffix altogether.  Most of these have the stem ending in the vowel sequence /ua/:

/kua-{\O-k}/  `he built'

/wua-{\O-k}/  `(s)he put'

/piipua-{\O-k}/  `(s)he left'

Another verb where the past tense suffix vowel seems to have disappeared is /oro-{\O}-k/ `(s)he went down'.  If the second vowel were to be taken as the suffix this verb would defy the basic rules, since the vowel /o/ is retained right through the past tense paradigm, and with the root vowel /o/ the suffix should be /-a/.  Positing /oro-/ as the root solves the question why the present tense form is /ora-/: since /oi/ is not a permitted vowel sequence on Mauwake, the low back vowel /o/ has changed into the low central vowel /a/ when preceding the high front vowel /i/ of the present tense suffix.

The verbs in the Mauwake dictionary are marked as belonging to Class 1 or Class 2, the former taking \{-a\} and the latter \{-E\} as the past tense suffix. This is because of the following reasons: 1) the rules are rather complicated, 2) there are a number of exceptions to the main rules, and 3) there are pairs of homophonous verb roots that take a different past tense suffix each.

/iw-\textstyleEmphasizedVernacularWords{a}-k/  `(s)he went'

/iw-\textstyleEmphasizedVernacularWords{o}-k/  `(s)he gave him/her'

/miim-\textstyleEmphasizedVernacularWords{a}-k/  `(s)he heard'

/miim-\textstyleEmphasizedVernacularWords{o}-k/  `(s)he preceded'

/op-\textstyleEmphasizedVernacularWords{a}-k/  `(s)he held'

/op-\textstyleEmphasizedVernacularWords{o}-k/  `it boiled'

/keen-\textstyleEmphasizedVernacularWords{a}-k/  `it touched'

/keen-\textstyleEmphasizedVernacularWords{e}-k/  `it was hot'

\paragraph[Inchoative suffix ]{Inchoative suffix} 
\hypertarget{RefHeading19161935131865}{}
The verbaliser for both adjectives and nouns is the inchoative suffix \{-aR\}, the root of the verb `to become' (\sectref{sec:3.8.2.2.2}).  In most environments it is realized as /\nobreakdash-ar/, but becomes /-al/ when the last syllable of the root contains the lateral consonant /l/.  An illustrative example is the word \textstyleStyleVernacularWordsItalic{samora/damola} `bad', which takes a different verbaliser depending on the root allomorph.

/supuk\textstyleEmphasizedVernacularWords{-ar}-e-k/  `it got wet'

/duduw-\textstyleEmphasizedVernacularWords{ar}-e-k/  `it became blunt'

/samor-\textstyleEmphasizedVernacularWords{ar}-e-k/  `it broke/spoiled'

/damol-\textstyleEmphasizedVernacularWords{al}-e-k/  `it broke/spoiled'

/memel-\textstyleEmphasizedVernacularWords{al}-e-k/  `it became tame'

In a few cases the inchoative suffix has the form /-al/ although there is no lateral consonant in the root.  This might be expected, since there is some fluctuation between the liquids /l/ and /r/ in Mauwake: /eliwa/ \~{} /eriwa/ `good', /samora/ \~{} /damola/ `bad'.\footnote{In Trans New Guinea, as well as other Papuan, languages it is also very common to have only one liquid, with /l/ and /r/ as allophones of the same phoneme (Wurm 1982:55, Foley 1986:55).} 

/masi-\textstyleEmphasizedVernacularWords{al}-e-k/  `it became bitter'

\paragraph[Completive aspect marker]{Completive aspect marker}
\hypertarget{RefHeading19181935131865}{}
The completive aspect marker (\sectref{sec:3.8.5.1.1.1}) has its origin in the verb for `put', \textstyleStyleVernacularWordsItalic{wua}\nobreakdash-,\footnote{The verb `put' is commonly used as a completive aspect marker in Papuan languages (\sectref{sec:3.8.5.1.1.1}).} but this connection has by now become opaque and the speakers consider it a morpheme on its own, \textstyleStyleVernacularWordsItalic{pu}-.  The initial /p/ results from assimilation with the final /p/ of the same-subject sequential action medial verb form obligatorily preceding the completive morpheme.

en-ep  wu-a-k  {\textgreater  enep-pu-a-k } `(s)he ate'

eat-SS.SEQ  put-PA-3s  {\textgreater}  eat-SS.SEQ-CMPL-PA-3s

\subsubsection{Loan words}
\hypertarget{RefHeading19201935131865}{}
When words are borrowed from other languages, they are usually made to conform to the Mauwake phonology, if they do not originally do so.  Thus Tok Pisin \textstyleForeignWords{kikim} `kick' becomes \textit{kiikim}- in Mauwake; the word retains the original Tok Pisin word-initial stress, and the vowel in the first syllable becomes a geminate.  The initial glottal fricative /h/ in the original becomes a lengthened vowel in Mauwake, e.g. Tok Pisin \textstyleForeignWords{handet} `hundred' changes into \textstyleStyleVernacularWordsItalic{aandet}  in Mauwake.

The only non-native phoneme regularly retained in the loan word is the velar nasal /{\ng}/, particularly prominent in the neighbouring language, Mala, and also used in personal names:

/nadi{\ng-ar-e-k/}  `(s)he decorated him/herself'

Since consonant sequences are quite rare in Mauwake, loan words with consonant clusters tend to have vowels inserted between the consonants.  With the ever-growing influence of Tok Pisin, vowel insertion is getting less common.  A combination of a nasal plus a homorganic stop is always retained in a loan word.

Tok Pisin  Mauwake  English

\textstyleForeignWords{glas}  galas  glass

\textstyleForeignWords{trinde}  tirinde  Wednesday

\textstyleForeignWords{namba}  naamba  number

\textstyleForeignWords{handet}  aandet  hundred

\section{Morphology}
\hypertarget{RefHeading19221935131865}{}
\subsection{Introduction}
\hypertarget{RefHeading19241935131865}{}
A grammatical word in Mauwake is defined on the basis of the following main criteria quoted from \citet[12-14]{Dixon2010b}:

A grammatical word


\begin{itemize}
\item has as its base one or more lexical roots to which morphological processes apply;

\item has a conventionalized coherence and meaning.


\end{itemize}
When a grammatical word involves compounding or affixation, its component grammatical elements 


\begin{itemize}
\item always occur together;

\item generally occur in a fixed order


\end{itemize}
The following supplementary criteria also apply. A word only allows one inflectional affix of any one type (ibid. 15). Also in derivation recursiveness is blocked except in the case of causatives (ibid. 16-17). Even here the recursion is more ostensible than real, as it does not add another argument into the clause (\sectref{sec:3.8.2.3.1}). Person/number suffixes act as word-final boundary markers in finite verbs (ibid. 17). Many words, especially those belonging to the major word classes, ``may constitute a complete utterance'' (ibid. 19) by themselves. 

The boundaries of the grammatical and phonological words coincide, except in the case of clitics. Grammatically a clitic is a word but phonologically it is bound to the preceding word.

The classes of nouns, adjectives, personal pronouns, quantifiers, verbs and adverbials can be reasonably clearly defined both morpho-syntactically and semantically. The classes of question words and deictics include words with  heterogeneous syntactic behaviour; question words have semantic and functional, and some morphological similarities as a group, whereas the category of deictics is based on strong morphological and semantic similarities. Connectives share the function of conjoining elements on the same level. As ``functor words'' postpositions and especially clitics are dependent on the preceding phrase. Interjections are different from all the other word classes in that they operate outside the normal syntax and often constitute a whole expression by themselves.

Nouns are naturally the largest category, but verbs are morphologically the most complex and interesting word class.

Although the great majority of the words in Mauwake can be assigned to one of the categories above, there is some indeterminacy with regard to some words that seem to belong to two or more word classes and the meanings which are clearly related.\footnote{In Austronesian languages it is common to have pre-categorial stems that may combine with affixation belonging to various word classes; only the whole word may be assigned to a particular word class.} They are not homonyms, since they are semantically related. Some transitive verbs have been derived by zero derivation from nouns and adjectives, and even from adverbs (\sectref{sec:3.8.2.2.1}, 3.8.4.4.3). Nominalized verbs (\sectref{sec:3.2.6.1}) function as nouns or adjectives. At the end of section 3.2.2 there is a list of words that are originally nouns but have become adjectives as well. Some non-numeral quantifiers (\sectref{sec:3.4.2}) also function as intensity adverbs (\sectref{sec:3.9.2}). Besides these, there are individual words that function in more than one word class; these are mentioned where they occur.

\subsection{Nouns}
\hypertarget{RefHeading19261935131865}{}
\subsubsection{General discussion}
\hypertarget{RefHeading19281935131865}{}
Although the traditional semantic definition of the noun as the ``name of a person, place or thing'' is not valid as a basis for assigning members to the class, it still gives a good general description of the prototypical members of the class in Mauwake. In Frawley's (1992:63) words, ``when the traditional definition is reversed, the definition turns out to be true. Nouns are not always persons, places or things, but persons, places and things always turn out to be nouns.''.\footnote{See also \textstyleBibliogBaseChar{Sapir 1921:117}, Jespersen 1924:60, Lyons 1977:449 and Schachter 1985:7.} Recognizing the semantic motivation of the class does not eliminate the need to define the class by its formal or functional  properties.

No good morphological definition of nouns is possible in Mauwake, as there is no inflection for number (\stepcounter{nx}{\thenx}), gender or class,\footnote{Gender or class systems are widespread in Papuan languages \citep[77]{Foley1986}. Especially in the TNG languages a covert system is common \citep[58]{Wurm1982}, where the noun class determines what existential verb is used with each noun.}  or case, in the noun itself. Especially the lack of plural marking is typical of the nouns in Trans-New Guinea languages \citep[36]{Wurm1982}. The glosses in the following example indicate a singular/plural alternative in the nouns, but the singular form in the glosses of other examples is to be understood as neutral regarding the number. 

\ea%x1
\label{ex:x1}
\gll siowa  wiawi \\
      \\
\glt
\z

dog(s)  father(s)

`The dog's/dogs' owner(s)'

Nouns are usually monomorphemic, with the exception of a small group of inalienably possessed nouns (\sectref{sec:3.2.4}), nouns derived from verbs (\sectref{sec:3.2.6.1}), reduplicated nouns (\sectref{sec:3.2.6.2}) and compound nouns (\sectref{sec:3.2.5}). The division into \textstyleEmphasizedWords{count} and \textstyleEmphasizedWords{mass} nouns is not very noticeable. It is mainly shown in the choice between the quantifiers \textstyleStyleVernacularWordsItalic{unowa} `many' and \textstyleStyleVernacularWordsItalic{maneka} `big, much', and to some extent in verb agreement morphology (\sectref{sec:3.4}).

The syntactic function provides the best criterion for defining a noun in Mauwake. Nouns function mainly as the head of a noun phrase, often the head being the only element in the \textstyleAcronymallcaps{NP}.\footnote{Sometimes an adjective, a quantifier or a genitive pronoun looks like a head of a NP, but those cases are elliptical, and the head noun is recoverable.} They can also function as a qualifier or, more rarely, as a modifier in a \textstyleAcronymallcaps{NP}. In (\stepcounter{nx}{\thenx}) \textstyleAcronymallcaps{NP}s, in this case manifested by just nouns, function as subject and object.

\ea%x2
\label{ex:x2}
\gll Emeria=ke  iwera  fiirim-i-mik. \\
      \\
\glt
\z

woman=CF  coconut  gather-Np-PR.1/3p

`(The) women gather coconuts.'

Hopper and \citet[710]{Thompson1984} also maintain that \textstyleBibliogCitationAAAstyleChar{``from the discourse point of view, nouns function to introduce participants and `props' and to deploy them''}\footnote{Actually this is the function of a NP rather than a noun.}\textstyleBibliogCitationAAAstyleChar{.} This is true in Mauwake as well, but it is not used as a criterion for defining the nouns.

\subsubsection{Nouns and adjectives: one or two word classes?}
\hypertarget{RefHeading19301935131865}{}
Since adjectives in Mauwake are phonologically, morphologically and syntactically very similar to nouns, the question must be asked whether the two form just one class of nominals or whether they belong to two separate word classes. In the following discussion they are treated on a semantic basis as if they were separate classes, i.e. certain words are called nouns and others adjectives, but a final conclusion as to their status is not drawn until the end of the section.

A \textstyleEmphasizedWords{\textsc{phonologically}} interesting feature common to nouns and adjectives is that the majority of both end in the vowel /a/.\footnote{In the other word classes words ending in /a/ do occur but they are very infrequent.} Inside noun phrases this vowel, when unstressed, is usually elided preceding a vowel and often also preceding a consonant. In cases like (\stepcounter{nx}{\thenx}), where there are two or more possible places for elision, the vowel most easily drops at the end of an adjective preceding an intensifier. Elision is also acceptable in two or more sites within one \textstyleAcronymallcaps{NP} (\stepcounter{nx}{\thenx}), (\stepcounter{nx}{\thenx}). 

\ea%x3
\label{ex:x3}
\gll koora  eliw(a)  akena,   also:  koor(a)  eliw(a)  akena  \\
      \\
\glt
\z

house  good  very

`a very good house'

\ea%x4
\label{ex:x4}
\gll koor(a)  kemena  manek(a)  akena  nain \\
      \\
\glt
\z

house  inside  big  very  that1

`the very big room'

\textstyleEmphasizedWords{\textsc{Morphologically}} nouns and adjectives resemble each other in that they lack inflection. There is no number, case, or gender marking in the adjectives, nor is there any inflection for comparison. (For comparison of adjectives, see \sectref{sec:6.5}). 

Both nouns (\stepcounter{nx}{\thenx}) and adjectives (\stepcounter{nx}{\thenx}) may be derived from verbs with the nominaliser suffix \nobreakdash-\textstyleStyleVernacularWordsItalic{owa}.

\ea%x7
\label{ex:x7}
\gll mua  \textstyleEmphasizedVernacularWords{soop-owa}  sira \\
      \\
\glt
\z

man  bury-NMZ  custom

`the burial custom (lit: the custom of burying men)'

\ea%x8
\label{ex:x8}
\gll Emi  \textstyleEmphasizedVernacularWords{kekan-owa}  nain  puuk-a-mik. \\
      \\
\glt
\z

taboo  be.strong-NMZ  that1  cut-PA-1/3p

`They broke the strong taboo rule.'

Verbs can be derived from both adjectives and nouns by zero verb formation (\stepcounter{nx}{\thenx}), (\stepcounter{nx}{\thenx}) or by the inchoative verbaliser \nobreakdash-\textstyleStyleVernacularWordsItalic{ar}  (\textstyleParagraphCharChar{\stepcounter{nx}{\thenx}}). (See \sectref{sec:3.8.2.2} for these processes and more examples.)

\ea%x482
\label{ex:x482}
\gll Miiw-aasa  samor-a-k. \\
      \\
\glt
\z

land-canoe  bad-PA-3s

`He broke/ruined the car.'

\ea%x484
\label{ex:x484}
\gll Iwer(a)  ififa  palis-i-ya. \\
      \\
\glt
\z

coconut  dry  pair.of.coconuts-Np-PR.3s

`He is tying dry coconuts into pairs.'

\ea%x483
\label{ex:x483}
\gll Miiw-aasa  samor-ar-e-k. \\
      \\
\glt
\z

land-canoe  bad-INCH-PA-3s

`The car broke.'

A clear morphological \textstyleEmphasizedWords{difference} between nouns and adjectives is that adverbs may be formed from some adjectives by deleting the word-final /a/, but they cannot be formed from nouns in the same way.

\ea%x19
\label{ex:x19}
\gll samora    {\textgreater  samor} \\
      \\
\glt
\z

`bad'      `badly'

\textstyleEmphasizedWords{\textsc{Syntactically}} there are a few similarities between nouns and adjectives. Both can function as a modifier following the head noun in a \textstyleAcronymallcaps{NP}, although adjectives (\stepcounter{nx}{\thenx}) are much more common in this position. In \textstyleBibliogBaseChar{Hopper and Thompson's (1985:161)} terms, it is nouns whose categorial status has been reduced, i.e. nouns that are not fully individuated in the discourse (\stepcounter{nx}{\thenx}), that can function in this modifier position.

\ea%x9
\label{ex:x9}
\gll aasa  \textstyleEmphasizedVernacularWords{awona}  fain \\
      \\
\glt
\z

canoe  old  this

`this old canoe'

\ea%x10
\label{ex:x10}
\gll mua  \textstyleEmphasizedVernacularWords{sira  eliwa} \\
      \\
\glt
\z

man  manner  good

`a well-mannered man (=a good man)'

The intensifier \textstyleStyleVernacularWordsItalic{akena} `real(ly), very' can also modify both adjectives (\stepcounter{nx}{\thenx}) and nouns (\stepcounter{nx}{\thenx}).

\ea%x11
\label{ex:x11}
\gll mua  \textstyleEmphasizedVernacularWords{akena} \\
      \\
\glt
\z

man  real/true

`a real man'

Complete or partial reduplication of adjectives is a common strategy for indicating plurality in Austronesian languages \citep[62]{Wurm1982}, and it also occurs to some extent in many Papuan languages, including Mauwake.  Reduplication is a more productive process in the adjectives (\stepcounter{nx}{\thenx}), (\stepcounter{nx}{\thenx}), but it is possible for a few  nouns too (\stepcounter{nx}{\thenx}), (\stepcounter{nx}{\thenx}) (\sectref{sec:3.2.6.2}).  

\ea%x12
\label{ex:x12}
\gll ifa  \textstyleEmphasizedVernacularWords{samo-samora} \\
      \\
\glt
\z

snake  RDP-bad

`bad snakes'

\ea%x481
\label{ex:x481}
\gll Maa  \textstyleEmphasizedVernacularWords{ele-eliwa}  sesek-a-mik. \\
      \\
\glt
\z

thing/food  RDP-good  sell-PA-1/3p

`They sold good foods (different kinds).'

\ea%x13
\label{ex:x13}
\gll \textstyleEmphasizedVernacularWords{Owow-owowa}  ikiw-e-mik. \\
      \\
\glt
\z

RDP-village  go-PA-1/3p

`They went to many villages.'

\ea%x1859
\label{ex:x1859}
\gll \textstyleEmphasizedVernacularWords{sira-sira} \\
      \\
\glt
\z

custom-custom

`many customs', `different kinds'

The syntactic \textstyleEmphasizedWords{\textsc{differences}} between nouns and adjectives are as follows. Adjectives do not function as the head of a noun phrase. The cases where they would seem to do so are in fact cases of ellipsis, and the head noun must be recoverable from the context, either linguistic or extra-linguistic. 

\ea%x14
\label{ex:x14}
\gll {\O  awona  nain  p-ekap-e!} \\
      \\
\glt
\z

{\O}  old  that1  BPx-come-IMP.2s

`Bring the old one!'

Only a noun may occur as a qualifier in a noun phrase, preceding the head noun (\stepcounter{nx}{\thenx}).  In some of these cases it is difficult to decide whether they are really \textstyleAcronymallcaps{NP}s with a qualifier and a head noun, or compound nouns. But if the latter is the case, then the restriction applies that an adjective cannot be the first element in a compound noun.

\ea%x15
\label{ex:x15}
\gll \textstyleEmphasizedVernacularWords{mera}  eka \\
      \\
\glt
\z

fish  water

`fish soup'

\ea%x16
\label{ex:x16}
\gll [[\textstyleEmphasizedVernacularWords{mera  eka}]  \textstyleEmphasizedVernacularWords{en-owa}]  sira \\
      \\
\glt
\z

fish  water  eat-NMZ  custom

`the custom of eating fish soup'

An adjective cannot be the only element following a genitive pronoun, but a noun can. Even in elliptical expressions an adjective following a genitive pronoun is not very acceptable (\stepcounter{nx}{\thenx}). 

\ea%x17
\label{ex:x17}
\gll ?Yiena  {\O } \textstyleEmphasizedVernacularWords{awona}  nain  p-ekap-e! \\
      \\
\glt
\z

1p.GEN  {\O}  old  that1  BPx-come-IMP.2s

`Bring our old one(s)!'

An exception to this rule is the adjective \textstyleStyleVernacularWordsItalic{maneka} `big'. The expression \textstyleStyleVernacularWordsItalic{yiena Maneka} `our Lord' (literally: our Big one), is probably formed following Tok Pisin \textstyleForeignWords{Bikpela bilong yumi.}\footnote{Non-prototypical adjectives are discussed later in this section; `big' is a prototypical adjective, so its use in a typically nominal position is an exception.}  

\ea%x105
\label{ex:x105}
\gll wi  Amerika  \textstyleEmphasizedVernacularWords{maneka},  unuma  Magerka \\
      \\
\glt
\z

3p.UNM  America  big  name  MacArthur

`the leader of the Americans, whose name was MacArthur'

Only an adjective functions as the head of an adjective phrase. In that position it may be modified by intensity adverbs (\sectref{sec:3.9.2}).  Of these, \textstyleStyleVernacularWordsItalic{lawisiw} `rather' does not modify nouns at all (\stepcounter{nx}{\thenx}); \textstyleStyleVernacularWordsItalic{akena} `very' and \textstyleStyleVernacularWordsItalic{pepek} `enough' may modify nouns as well; \textstyleStyleVernacularWordsItalic{wenup} `very'can do that too, but as a noun modifier it has a somewhat restricted use and a different meaning, `many'.

\ea%x18
\label{ex:x18}
\gll Mera  nain  \textstyleEmphasizedVernacularWords{lawisiw  maneka  akena}. \\
      \\
\glt
\z

fish  that1  rather  big  very

`That fish is rather huge.'

What further obscures the area of nouns and adjectives is the fact that there are a number of words that sometimes function like nouns (\stepcounter{nx}{\thenx}), sometimes like adjectives (\stepcounter{nx}{\thenx}), and also semantically could be like either.

\ea%x20
\label{ex:x20}
\gll \textstyleEmphasizedVernacularWords{Pina}  maneka  kamenap? \\
      \\
\glt
\z

weight  big  what.like

`What is the weight like?', `How big is the weight?'

\ea%x21
\label{ex:x21}
\gll Maa  nain  lawisiw  \textstyleEmphasizedVernacularWords{pina}. \\
      \\
\glt
\z

thing  that1  rather  heavy

`The thing is rather heavy.

The prototype view offers a plausible solution for the problem. Starting from the study of basic colour terms (Berlin and Kay 1969) it has been applied to other areas of semantics and also to linguistic categorization (e.g. Wierzbicka 1986, Taylor 1989 and Frawley 1992). The main idea that categories have more central, or focal, members as well as more marginal members was also recognized by \citet{Crystal1967} in his description of English word classes. The prototype approach allows for stability as well as flexibility \citep[53]{Taylor1989}, both of which are needed in an attempt to describe a human language.

If prototypical linguistic categories are focal, or optimal, instances on a continuum \citep[321]{Seiler1978} and maximally distinct from one another \citep[709]{HopperEtAl1984}%Thompson
, what are prototypical nouns like as opposed to prototypical adjectives? According to \citet{Wierzbicka1986}, noun indicates \textstyleEmphasizedWords{\textsc{categor}}\textstyleEmphasizedWords{\textsc{ization}}: most prototypical nouns identify a certain kind of person, thing or animal. Relative \textstyleEmphasizedWords{\textsc{temporal stability}} is for Giv\'on what characterizes nouns, and the most prototypical nouns denote concrete, physical, compact entities (1984:51). Instead of time stability, \citet[66]{Frawley1992} claims it is relative \textstyleEmphasizedWords{\textsc{atemporality}}\textstyleEmphasizedWords{} that makes an entity an entity.  Adjectives, or property concepts, indicate \textstyleEmphasizedWords{\textsc{description}}, and they denote single properties unlike nouns which denote a cluster of properties \citep{Wierzbicka1986}.

In Mauwake, a prototypical noun occurs as a head in a \textstyleAcronymallcaps{NP}, as a pre-modifier or, less frequently, as a post-modifier in a \textstyleAcronymallcaps{NP}, or as any element in a compound noun. It does not occur as the head in an \textstyleAcronymallcaps{AP}. It can be modified by adjectives or genitive pronouns but not by the intensity adverbs \textstyleStyleVernacularWordsItalic{lawisiw} `rather' and \textstyleStyleVernacularWordsItalic{wenup} `very'. Prototypical \textstyleEmphasizedWords{\textsc{adjectives}}\textstyleEmphasizedWords{\textsc{} }occur\textstyleEmphasizedWords{\textsc{} }as the head of an adjective phrase. They do not pre-modify nouns or function as the first element in a compound noun.

It turns out that in Mauwake the most prototypical nouns include names of concrete \textstyleEmphasizedWords{\textsc{non}}-human rather than human objects, when one would expect words referring to human beings to be nouns \textstyleEmphasizedWords{\textsc{par excellence}} (see Taylor 1989:192). Some human nouns may be used as post-modifiers in a \textstyleAcronymallcaps{NP}: from the cluster of properties denoted by the noun one has been picked out, and the noun is used like an adjective (\stepcounter{nx}{\thenx}), (\stepcounter{nx}{\thenx}). The adjectival use of \textstyleStyleVernacularWordsItalic{mua} `man' in (\stepcounter{nx}{\thenx}) is particularly interesting, because the adjectives \textstyleStyleVernacularWordsItalic{morena} `male' and \textstyleStyleVernacularWordsItalic{suwina} `female' are used for animals.

\ea%x23
\label{ex:x23}
\gll labuel(a)  mua \\
      \\
\glt
\z

pawpaw  man

`male pawpaw'

\ea%x24
\label{ex:x24}
\gll donki  takira \\
      \\
\glt
\z

donkey  young.person

`young donkey'

The less prototypical status of human nouns also shows in words like \textstyleStyleVernacularWordsItalic{apura} `widow' and \textstyleStyleVernacularWordsItalic{oosa} `widower' which may occur by themselves as heads of a \textstyleAcronymallcaps{NP}, but which are most typically used as post-modifiers of \textstyleStyleVernacularWordsItalic{emeria} `woman' and \textstyleStyleVernacularWordsItalic{mua} `man', respectively.\footnote{Other words in this group are \textstyleFootnoteBaseChar{\textit{muupera}}\textbf{\textit{} }`visitor, guest' and especially \textstyleFootnoteBaseChar{\textit{weria}}, which as a human noun only occurs in the combination \textstyleFootnoteBaseChar{\textit{mua weria}}, `uncle/ male cross cousin/ nephew'. The \textstyleFootnoteBaseChar{\textit{mua weria'}}s are responsible for burying a dead person and dispensing of his/her belongings (1.3.6).} As age in human beings tends to be to be treated as a crucial determinant of \textstyleEmphasizedWords{\textsc{kind}}, even languages with large adjective classes often have special nouns for referring to old persons \citep[368]{Wierzbicka1986}. In Mauwake, adjectives that indicate age in humans are non-prototypical, more noun-like than most adjectives: both \textstyleStyleVernacularWordsItalic{iperowa} `middle-aged' and \textstyleStyleVernacularWordsItalic{panewowa} `old' are used as the head of a \textstyleAcronymallcaps{NP} besides the typical adjectival use.

\ea%x25
\label{ex:x25}
\gll \textstyleEmphasizedVernacularWords{Iperowa}  opora  wiar  miim-i-yen. \\
      \\
\glt
\z

middle-aged  talk  3.DAT  hear-Np-FU.1p

`We will listen to the talk of the middle-aged (men).'

According to \citet[56]{Dixon1977}, if a language has adjectives at all, words expressing age, dimension, value and colour are likely to belong to the adjective class, however small the class. The most prototypical adjectives in Mauwake belong to these groups, with the exception of adjectives denoting human age, discussed above. In the group of adjectives denoting either physical property or human propensity, some are ambiguous as to their basic category: \textstyleStyleVernacularWordsItalic{anima} is both `blade' and `sharp', and \textstyleStyleVernacularWordsItalic{pina}  both `weight, burden' and `heavy'. Different groups of adjectives, as well as the use of adjectives, are discussed below in Section 3.3.

With the rules given above it is fairly straightforward to distinguish the nouns and adjectives in Mauwake. But a small group remains that seems to have a membership in both classes. Originally they are are nouns that have now been employed as adjectives as well. The claim is based on the fact that the noun category is the more basic and universally  recognized, whereas the existence of the adjective category is disputed in some languages; and in Mauwake the noun class is clearly established, large, and more easily definable.  Also, there are at least two nouns in Mauwake that currently seem to be in the process of becoming regular adjectives: the meaning of the phrase stays the same with the pre-modifying noun and the post-modifying adjective. 

  (\stepcounter{nx}{\thenx}x108)  \textstyleEmphasizedVernacularWords{napum(a)}  mua

sickness  man

`a sick man'

\ea%x107
\label{ex:x107}
\gll mua  \textstyleEmphasizedVernacularWords{napuma} \\
      \\
\glt
\z

man  sick

`a sick man', also: `human (lit: man's) sickness'

\ea%x1822
\label{ex:x1822}
\gll \textstyleEmphasizedVernacularWords{wadol(a)}  opora \\
      \\
\glt
\z

lie/false  talk

`a lie'

\ea%x1823
\label{ex:x1823}
\gll opor(a)  \textstyleEmphasizedVernacularWords{wadola} \\
      \\
\glt
\z

talk  lie/false

`a lie'

Below is a list of the most common of the words functioning both as nouns and as adjectives:

anima  `blade, point, edge'  `sharp'

afila  `grease'  `greasy, sweet'

foma  `ashes'  `grey'

ikina  `smell'  `smelly'

irauwa  `hole'  `deep'

makena  `true'  `truth, essential nature'

napuma  `sickness, corpse'  `sick'

pina  `weight, burden, guilt'  `heavy'

siisia  `design, pattern'  `spotted, patterned'

tumina  `dirt'  `dirty'

wadola  `lie'  `false, fake'

\subsubsection{Common  vs.  proper nouns}
\hypertarget{RefHeading19321935131865}{}
There is very little difference between common and proper nouns in Mauwake, and it can be questioned whether the two should be grouped separately as is traditionally often done in language descriptions. Proper nouns are sometimes classified separately because they are said to be unable to have modifiers \citep[152]{Roberts1987}, and in practice, they usually occur without any modifiers. This is related to the fact that they normally only have a referent, but no intension. In most of the cases where a proper noun is modified, ``it lacks a unique reference and is being used as a common noun'' (Van Valin and LaPolla 1997:59):

  (\stepcounter{nx}{\thenx}x26)  I mean the old and cranky Joe Smith, not the younger one. 

The most common type of a proper noun is a name of a \textstyleEmphasizedWords{\textsc{person}}. A proper noun may also become a true common noun, when one or more of the qualities of a person are used to characterise some other being \citep[66]{Jespersen1924}. For example, the name of a well-known expatriate, Jooren, was borrowed by Mauwake speakers to mean `a stingy shopkeeper' (that is, one who does not sell things on credit and does not give discount to relatives). 

In Mauwake proper names can be modified without difficulty, especially by the demonstrative \textstyleStyleVernacularWordsItalic{nain} `that', but also by adjectives. In a culture where there are several namesakes, and surnames are rarely used, modifiers are occasionally needed to distinguish between people (\stepcounter{nx}{\thenx}).

\ea%x27
\label{ex:x27}
\gll \textstyleEmphasizedVernacularWords{Adek  panewowa  nain}  ma-i-yem. \\
      \\
\glt
\z

Adek  old  that1  say-Np-PR.1s

`I am talking about the \textstyleEmphasizedWords{old} Adek.'

But even proper names that have a unique reference and do not need to be distinguished from any other referent can be modified:

\ea%x106
\label{ex:x106}
\gll \textstyleEmphasizedVernacularWords{Dabe  fain}  uuw-ow(a)  mua=ke. \\
      \\
\glt
\z

Dabe  this  work-NMZ  man=CF

`Dabe here is a hard worker.'

In this case the behaviour of proper names is similar to that of the personal pronouns, which also have unique reference, but can be modified nevertheless. \textstyleBibliogBaseChar{Van Valin and LaPolla} (ibid. 59-60) note that languages may vary in how freely they allow proper nouns and pronouns to take modifiers.

Name taboos influence the use of personal names in several ways. A person is given many different names: at least one from each parents' side (as in-laws may not mention each others' names), a baptismal name, and possibly others as well. These names are used by different people. Name taboos may be avoided by calling someone by a teknonym like `Sarak's father', or by calling a wife by the husband's name when she is with the in-laws and the husband is not around. Nicknames, often referring to physical properties, are also very common: \textstyleStyleVernacularWordsItalic{buburia} `bald', \textstyleStyleVernacularWordsItalic{mua kuuma} `lame' (literally `stick-man'). The term `namesake' is very common and even used of people who have been named after different names of the same person.  Two boys, Yoli and Wangali, were called namesakes of each other, as they were both named after the same ancestor. 

Perhaps the most characteristic feature of personal names is \textstyleEmphasizedWords{\textsc{discourse-pragmatic}}: in a text their token frequency is very low. Especially the main participant, once (s)he has been mentioned by name -- if (s)he ever is -- (s)he is then usually referred to by other means: a \textstyleAcronymallcaps{NP}, pronoun, or just person marking on the verb.  

Besides the names of people, \textstyleEmphasizedWords{\textsc{place names}} form another large group of proper names. In Mauwake, the proper name often modifies a generic noun:  \textstyleStyleVernacularWordsItalic{Moro} (\textstyleStyleVernacularWordsItalic{owowa}) `Moro (village), \textstyleStyleVernacularWordsItalic{Siburten} (\textstyleStyleVernacularWordsItalic{ema}) `Siburten (mountain/hill)', \textstyleStyleVernacularWordsItalic{Nemuru} (\textstyleStyleVernacularWordsItalic{eka}) `Nemuru (river)' (\sectref{sec:4.1}). 

The place name is also used when the inhabitants are referred to. When reference is made to an individual or a select group, the place name is used as a qualifier in the noun phrase: 

\ea%x421
\label{ex:x421}
\gll \textstyleEmphasizedVernacularWords{Amiten } mua  oko  ekap-o-k. \\
      \\
\glt
\z

Amiten  man  other  come-PA-3s

`A man from Amiten came.'

When the whole group is referred to, a plural pronoun is added to the place name:

\ea%x422
\label{ex:x422}
\gll \textstyleEmphasizedVernacularWords{I}  \textstyleEmphasizedVernacularWords{Moro=ke}  uf-e-mik. \\
      \\
\glt
\z

1p.UNM  Moro=CF  dance-PA-1/3p

`We Moro people danced.'

\ea%x423
\label{ex:x423}
\gll \textstyleEmphasizedVernacularWords{(Wi)  Lasen  wia}  nokar-e-k.\footnotemark{} \\
      \\
\glt
\z

3p.UNM  Lasen  3p.ACC  ask-PA-3s

`He asked the Lasen people'

\footnotetext{The optional initial pronoun \textit{wi} is part of the object here, not a subject pronoun.}

\subsubsection{Alienable and inalienable possession}
\hypertarget{RefHeading19341935131865}{}
The Austronesian languages in Melanesia tend to have very elaborate semantically based possessive systems that indicate the relationship between the ``possessor'' and the ``possession'': kin relation, body part, food etc.  Inalienable possession is indicated by affixation on the noun, alienable possession by a separate possessive pronoun. Because of this, the simpler inalienable possession marking also evident in many \textstyleAcronymallcaps{TNG} languages could easily be attributed to influence from Austronesian languages.  But \citet[28]{Ross1996} claims it is likely that even Proto \textstyleAcronymallcaps{TNG} had inalienable nouns before there was any contact with Austronesian languages.\footnote{On the time frames of TNG occupation and Austronesian migration, see e.g. \citet[39-41]{Ross2005}.} In Mauwake the division into alienably and inalienably possessed nouns is along the lines of kinship terms (see \sectref{sec:1.3.6} for a kinship chart). Most kin terms obligatorily indicate who the {\textquotedbl}possessor{\textquotedbl} is:

  1s/p  2s/p  3s/p  possessor  

a.  auwa  niawi  wiawi  `father'

b.  aite  niena  onak  `mother'

c.  paapa  neepe  weepe  `elder sibling'

d.  (y)aamun  niamun  wiamun  `younger sibling'

e.  yaaya  nie  wie  `uncle'

f.  paapan  noopan  woopan  `aunt'

g.  kae  neke  weke  `grandfather'

h.  kome  nokome  wokome  `grandmother'

i.  eremena  neremena  weremena  `nephew, niece'

j.  emar, yomar  nomar  womar  `(cross-)cousin'

k.  yomokowa  nomokowa  womokowa  `brother'\footnote{Among siblings, age is more important than sex: \textstyleFootnoteBaseChar{\textit{paapa}} and \textstyleFootnoteBaseChar{\textit{aamun}} are used very frequently and for siblings of either gender. When the gender is in focus, \textstyleFootnoteBaseChar{\textit{yomokowa}} is used for `my brother' and \textstyleFootnoteBaseChar{\textit{ekera}} for `my sister' especially by siblings of the opposite sex.}

l.  (y)ekera  nekera  wekera  `sister'

m.  (y)emi  nemi  wemi  `(man's) brother-in-law'

n.  epua  nepua  wepua  `(woman's) brother-in-law\footnote{A woman calls her elder sister's husband \textit{auwa} `father', but the other brothers-in-law are \textit{epua}.}

o.  yomora  nomora  womora  `sister-in-law'

p.  yopariw  nopariw  wopariw  `husband's brother's wife'

q.  yamekua  namekua  wamekua  `daughter-in-law'\footnote{Some in-law relations are non-symmetrical: even though there are special terms for sons- and daughters-in-law,  \textstyleFootnoteBaseChar{\textit{auwa}} `(my) father' and \textstyleFootnoteBaseChar{\textit{aite}} `(my) mother' are used for `(my) mother-in-law' and `(my) father-in-law'.}

r.  yar  nar  war  `son-in-law'

s.  yookati  nookati  wookati  `co-wife'\footnote{This term dates back to the time when polygamy was practiced; it was used for the wives of the same man.}

t.  yomawa  nomawa  womawa  `namesake'

The possessive prefixes  \textstyleStyleVernacularWordsItalic{y}-, \textstyleStyleVernacularWordsItalic{n}- and \textstyleStyleVernacularWordsItalic{w}- in the inalienably possessed nouns developed from the first, second, and third person pronouns. These prefixes are in the process of merging with the root. The terms in (a-j) above are somewhat more lexicalized than the ones in (k-s): the first person prefix is mostly lost, and in some cases there is suppletion in the stem. These are some of the socially most important and frequently used kinship terms.  The frequent use probably accounts for the omission of the possession prefix in the first person: these terms are used more as terms of address, whereas the other kinship nouns are only needed as terms of reference. Also, there is a tendency to drop the first person prefix before the front vowel /e/ regardless of the closeness of the kinship relation.

The ``possessors'' are differentiated as first, second or third person but not as single vs. plural. An unmarked (\stepcounter{nx}{\thenx}) or a genitive (\stepcounter{nx}{\thenx}), (\stepcounter{nx}{\thenx}) pronoun may be used to either make this number distinction or to emphasise the kin relationship, when the relationship  is used as a term of reference rather than as a term of address.

\ea%x1311
\label{ex:x1311}
\gll Kuuten  \textstyleEmphasizedVernacularWords{wiawi}  iperowa,  \textstyleEmphasizedVernacularWords{yo}  \textstyleEmphasizedVernacularWords{auwa}  kapa=ke. \\
      \\
\glt
\z

Kuuten  3s/p.father  firstborn  1s.UNM  1s/p.father  lastborn=CF

`Kuuten's father was the firstborn, my father the lastborn.'\footnote{Both of these fathers could be called \textit{auwa} `my/our father(s)' by the two men.}

\ea%x28
\label{ex:x28}
\gll Aakisa  \textstyleEmphasizedVernacularWords{yena} \textstyleEmphasizedVernacularWords{} \textstyleEmphasizedVernacularWords{auwa}  kapa  fain=ke  yia  uruf-i-ya. \\
      \\
\glt
\z

now  1s.GEN  1s/p.father  lastborn  this=CF  1p.ACC  see-Np-PR.3s

`Now this lastborn of my ``fathers'' watches over us.'

\ea%x1312
\label{ex:x1312}
\gll Sa,  a  \textstyleEmphasizedVernacularWords{nena  nie=ke},  \textstyleEmphasizedVernacularWords{nena  nepua=ke,} \\
      \\
\glt
\z

INTJ  INTJ  2s.GEN  2s/p.uncle=CF  2s.GEN  2s/p.brother-in-law

niawi=ke.

2s/p.father

`(Don't you understand,) those are \textit{your} uncle(-in-law), \textit{your} brother-in-law and father(-in-law).'

When a neutral, ``non-possessed'', kinship term is needed, the first person form is used. This is interesting, as the third person singular is typically considered the neutral, or unmarked, form. The terms `(my) mother' and `(my) father' are also used as respectful terms of address for almost any stranger regardless of age, or for anyone whose status in the kinship system is uncertain.\footnote{I have been addressed as {\textquotedbl}mother{\textquotedbl} by an old man who temporarily forgot what my status according to their kinship system was - I was actually his granddaughter!}

Four alienably possessed nouns, namely those for `man', `woman', `boy' and `girl', have been taken into the kinship system for terms of some nuclear family members: 

mua  `man, husband'

emeria  `woman, wife'

muuka  `boy, child, son'

wiipa  `girl, daughter'

Also the term \textstyleStyleVernacularWordsItalic{nembesir} `ancestor (beyond grandparents)' or `descendant (beyond grandchildren)' is an alienably possessed noun, possibly because relatives so far removed in time are considered less relevant. It is used both for males and females. But the term for `namesake', \textstyleStyleVernacularWordsItalic{yomawa}, is included in the inalienably possessed kinship terms, as a child is named after some relatives, and the namesake relation forms an additional bond between them.

\subsubsection{Noun compounding}
\hypertarget{RefHeading19361935131865}{}
The distinction between compound nouns and noun phrases is a problematic area in many languages, including Mauwake. Both are formed by combining independent elements into larger units, and their form and meaning are largely based on the form and meaning of those elements (\textstyleBibliogBaseChar{Anderson 1985a}:40). Phonological, morphological, syntactic as well as semantic criteria have been called upon to differentiate between compounds and phrases.

In many languages, ``word accent'' \citep[204]{Lyons1968}, i.e. stress and/or pitch, helps to distinguish compounds. In Mandarin Chinese, contrastive stress can only fall on the ``stress center'' of a word, including compounds (\textstyleBibliogBaseChar{Anderson 1985a}:41). In Finnish, the primary stress is on the first, and only on the first, syllable of even very long compound words like \textstyleForeignWords{kuluttajansuoja-asiamiesverkostokysymys} `the question of consumer ombudsman network', but even in Finnish there are unclear cases like \textstyleForeignWords{valveillaolo} vs. \textstyleForeignWords{valveilla olo}\textstyleEmphasizedWords{} `being awake'. In the latter, the varying writing convention reflects the ambiguity. 

Linguists differ in their views about the importance of stress placement in interpreting English compounds. \citet[228]{Bloomfield1935} and \citet[41]{Anderson1985a} consider it criterial, and so do \citet[1330]{QuirkEtAl1989}, although more cautiously. \citet[120]{Lees1968} takes it as one premise for his study of compounds while admitting that the case is not very well substantiated. Others, like \citet[31]{Jespersen1933}, \citet{Downing1977} and \citet{Bauer1983} do not consider a single primary stress essential for compounds. According to \citet[105]{Bauer1983}, Lyons' (1968:202) criteria for judging ``wordness'' in English, i.e. positional mobility and uninterruptability (or internal stability) do not distinguish between single- and double-stressed compounds.

Morphology may place constraints on compounding. In English, the genitive is common in phrases but rare in compounds: duck's egg vs. duck-egg (Anderson\textstyleBibliogBaseChar{ 1985a}:41).\footnote{But note also women's lib(eration), a compound.} In Finnish, the first part of a compound is often in the nominative or genitive case, whereas the other cases are infrequent in this position. In German, certain elements may serve as morphological ``glue'' between the parts of a compound (ibid. 42).

The two criteria for wordness by \citet[202]{Lyons1968} mentioned above are syntactic in nature: a word, hence also a compound, is moved as one unit, and cannot be interrupted by other words as a phrase often can. These criteria do not apply to all, and only, compound words, but they are useful in trying to establish the difference between compounds and phrases in a given language. \citet[232]{Bloomfield1933} adds another one: a member of a compound generally cannot serve as a constituent in a syntactic construction. One can say \textstyleEmphasizedWords{a very black bird} but not \textstyleEmphasizedWords{* a very blackbird}.

The semantic interpretation of phrases is generally quite compositional: the meaning of the whole can be deduced from the meanings of the words. Compounds are more heterogeneous in their interpretation: some are compositional, whereas others involve special interpretive principles not applicable to phrases. Also, compounds as words are subject to changes of meaning, so many compounds may have meanings that are only vaguely or metaphorically related to that which is predicted on the basis of the parts (Anderson\textstyleBibliogBaseChar{ 1985a}:42). Knowledge of the pragmatics of the situation may be needed for the interpretation of many compound words (Bauer\textstyleBibliogBaseChar{ 1983}:58). The more fully lexicalized the compounds are, the more the meaning of the whole may deviate from the meaning of the parts. The same compound word may also be fully lexicalized in a certain context, and still be open for other interpretations in other contexts (Andrew Pawley, p. c.). 

While there are languages where it is easy to distinguish between compound nouns and noun phrases, in others there is an intermediate area between the two. Thus \citet[810]{Downing1977} doubts that the dividing line is always well-defined, and Quirk et al\textstyleBibliogBaseChar{.} (1989:1569) suggest the concept of ``partial compounding'' to account for the formal and semantic gradience between compounds and phrases in English. Bringing a historical viewpoint to the question, citing developments in English both from phrase to compound and from compound to phrase, \citet[102]{Jespersen1924} offers a very liberal view: ``it is of no consequence whether we reckon [the] doubtful cases as one word or two words, for ... a word group (like a single word) may be either primary or an adjunct or a subjunct''. 

None of the criteria mentioned above can be easily applied in Mauwake. \textstyleEmphasizedWords{\textsc{Semantically}} there is a continuum between fully compositional noun phrases and fully lexicalized compounds. But \textstyleBibliogBaseChar{Bloomfield} (1933:227) warns that the greater specialization in meaning in the compound words as against phrases should not be used as a criterion, as ``we cannot gauge meanings accurately enough, and many a phrase is as specialized in meaning as any compound''. This warning is all the more relevant when one studies a language not one's own. 

The basic \textstyleEmphasizedWords{\textsc{stress pattern}} of noun phrases and compounds is similar, as one of the modifiers usually receives the phrase stress rather than the head noun (\stepcounter{nx}{\thenx}), (\stepcounter{nx}{\thenx}). Likewise, in compound nouns the modifying formative receives the main stress and the main formative is only weakly stressed (\stepcounter{nx}{\thenx}), (\stepcounter{nx}{\thenx}): the ``stress centre'' \citep[45]{Anderson1985a} is on another element than the head. 

\ea%x29
\label{ex:x29}
\gll yo  'auwa  aasa\footnotemark{} \\
      \\
\glt
\z

1s.UNM  1s/p.father  canoe

`my father's canoe'

\footnotetext{In the examples (\stepcounter{nx}{\thenx})-(\stepcounter{nx}{\thenx}) only the phrase stress is marked by ' preceding the stressed syllable.}

\ea%x30
\label{ex:x30}
\gll aas(a)  ge'lemuta \\
      \\
\glt
\z

canoe  small

`a small canoe'

\ea%x31
\label{ex:x31}
\gll 'miiw(a)-aasa\footnotemark{} \\
      \\
\glt
\z

\footnotetext{In Mauwake orthography, the parts of a compound word are usually written separately to help the new readers to identify the parts; \textit{miiw-aasa} `vehicle' is one of the exceptions.}

land-canoe

`vehicle, car'

\ea%x32
\label{ex:x32}
\gll enow(a)  ge'lemuta{\footnotemark} \\
      \\
\glt
\z

food/meal  small

`feast'
\footnotetext{\textit{Enow gelemuta} is not used with its literal meaning `small meal'.}
However, the head noun in a \textstyleAcronymallcaps{NP} may receive the phrase stress if it is emphasized for contrast, clarification or some other reason, whereas the stress centre in a compound stays the same. 

Since there is hardly any \textstyleEmphasizedWords{\textsc{morphology}} in nouns and noun phrases, one would not expect to find much help here in distinguishing between compounds and phrases. But there is a minor factor that is relevant in this respect: a phrase containing a noun and an adjective can be pluralized by adjectival reduplication when the adjective allows reduplication (\stepcounter{nx}{\thenx}), whereas a compound noun with a similar structure usually cannot (\stepcounter{nx}{\thenx}), even if it is possible in some rare cases (\stepcounter{nx}{\thenx}). 

\ea%x33
\label{ex:x33}
\gll maa  gelemuti-tik \\
      \\
\glt
\z

thing  small-RDP

`small things'

\ea%x34
\label{ex:x34}
\gll *enow(a)  gelemuti-tik \\
      \\
\glt
\z

food/meal  small

\ea%x35
\label{ex:x35}
\gll owow(a)  mane-maneka \\
      \\
\glt
\z

village  RDP-big

`towns', `big villages'

Uninterruptibility is more typical of compounds than phrases. The noun phrase \textstyleStyleVernacularWordsItalic{owow maneka} means `a big village', as a compound it means `a town/city'. As a phrase it is interruptible (\stepcounter{nx}{\thenx}), as a compound it is not.

\ea%x1768
\label{ex:x1768}
\gll owowa  lawisiw  maneka \\
      \\
\glt
\z

village  rather  big

`a rather big village'

Likewise, as a compound \textstyleStyleVernacularWordsItalic{kae sira} 'ancestral custom' (literally: `grandfather's custom') is uninterruptible. When  a genitive pronoun is inserted between the two parts, the meaning cannot be `ancestral custom':

\ea%x1860
\label{ex:x1860}
\gll kae  ona  sira \\
      \\
\glt
\z

grandfather  3s.GEN  custom

`grandfather's custom/habit'

In Mauwake word combinations are treated as compounds if they 1) have a specialized meaning, 2) have a stress centre not affected by contrastive stress, and 3) tend to be uninterruptible.  However, this distinction is very tentative in some cases. Some examples are provided where the same combination may be either a compound noun or a noun phrase.

Morphologically there are four compound noun types in Mauwake: \textstyleAcronymallcaps{N}+\textstyleAcronymallcaps{N,}  \textstyleAcronymallcaps{V}\textsubscript{NMZ}\textstyleAcronymallcaps{} +\textstyleAcronymallcaps{N,  N+V}\textsubscript{NMZ}\textstyleAcronymallcaps{ } and \textstyleAcronymallcaps{N}+\textstyleAcronymallcaps{ADJ}.  Syntactically these correspond to a head noun with a nominal pre- or post-modifier in a \textstyleAcronymallcaps{NP} or a head noun with an adjective post-modifier in the \textstyleAcronymallcaps{NP}. In most compound nouns the last noun is the head.  But in generic-specific compounds as well as the \textstyleAcronymallcaps{N}+\textstyleAcronymallcaps{ADJ} and\textstyleAcronymallcaps{} \textsc{N+V}\textsubscript{NMZ} compounds the first part is the main element and the scope of its meaning is restricted by the second part. In coordinate compounds the two parts are equally important.

On the basis of the semantic relations between the parts the \textstyleAcronymallcaps{N}+\textstyleAcronymallcaps{N} compounds can be divided into a few main groups. In the first one the relationship can be said to be characterized by \textstyleEmphasizedWords{\textsc{origin}} understood very widely, e.g. in the sense of place of origin (\stepcounter{nx}{\thenx}), source (\stepcounter{nx}{\thenx}), or ``possession'' (\stepcounter{nx}{\thenx}), (\stepcounter{nx}{\thenx}). 

\ea%x37
\label{ex:x37}
\gll piip(a)  mera \\
      \\
\glt
\z

seaweed  fish

`rainbow fish'

\ea%x40
\label{ex:x40}
\gll emeria  napuma \\
      \\
\glt
\z

woman  sick(ness)

`menstruation'

\ea%x41
\label{ex:x41}
\gll ibiamun  sama \\
      \\
\glt
\z

dove  ladder

`cross-beam (in a roof)'

The compound noun (\stepcounter{nx}{\thenx}) has the stress centre on the first part, but the noun phrase \textstyleStyleVernacularWordsItalic{emeria napuma}, with the phrase stress on \textstyleStyleVernacularWordsItalic{napuma}, may be used to mean either `a sick woman', or more commonly `a (dead) woman's body', a euphemistic expression.  

The second relationship is a \textstyleEmphasizedWords{\textsc{whole-part}} relationship: the first element states the whole, the second its part.

\ea%x42
\label{ex:x42}
\gll mokok(a)  oposia \\
      \\
\glt
\z

eye  meat

`pupil (of the eye)'

\ea%x43
\label{ex:x43}
\gll ekek(a)  muuna \\
      \\
\glt
\z

branch  joint/projection

`bud'

The third relationship is that of \textstyleEmphasizedWords{\textsc{container}}. As a compound \textstyleStyleVernacularWordsItalic{muuk(a) sia} (\stepcounter{nx}{\thenx}) has the stress centre on the first word, in a noun phrase (\stepcounter{nx}{\thenx}) the phrase stress may also be on the second item if it is emphasized; a third person singular genitive pronoun may be added between the parts as well. Example (\stepcounter{nx}{\thenx}) is an extended compound: \textstyleStyleVernacularWordsItalic{iinan aasa}  is a ``sky canoe'', or vehicle, for flying in the sky, and \textstyleStyleVernacularWordsItalic{iinan aasa epa} a place for those vehicles.

\ea%x45
\label{ex:x45}
\gll muuk(a)  sia \\
      \\
\glt
\z

son  netbag

`womb', `pouch (of a marsupial) 

\ea%x1770
\label{ex:x1770}
\gll muuk(a)  sia \\
      \\
\glt
\z

son  netbag

`a son's/child's netbag (used for carrying the baby)'

\ea%x46
\label{ex:x46}
\gll iinan  aasa  epa \\
      \\
\glt
\z

sky  canoe  place

`airstrip, airport'

As was mentioned above, the \textstyleEmphasizedWords{\textsc{generic-specific}} relationship is different in that the modifying part follows rather than precedes the main part. In this respect these compounds resemble phrases where the head noun has an adjective rather than a noun modifier.  A particularly common word for the first part in these compounds is the maximally generic word in Mauwake, \textstyleStyleVernacularWordsItalic{maa} `\textstyleFreeTranslationChar{thing'}(\stepcounter{nx}{\thenx}).\footnote{The scope of meaning for \textstyleFootnoteBaseChar{\textit{maa}} is like that of `thing\textit{'} in its widest sense in English.}

\ea%x47
\label{ex:x47}
\gll mera  nepa \\
      \\
\glt
\z

fish  bird

`eagle ray'

\ea%x48
\label{ex:x48}
\gll oon(a)  tiretira \\
      \\
\glt
\z

bone  horizontal.cane  (in  roof  structure)

`rib'

\ea%x49
\label{ex:x49}
\gll maa  pela \\
      \\
\glt
\z

thing  leaf

`(edible) greens'

There are two compound types with nominalized verbs. When the nominalized verb follows the other noun, it behaves like an adjective and receives the phrase stress.

\ea%x1521
\label{ex:x1521}
\gll maa  en-owa \\
      \\
\glt
\z

thing/food  eat-NMZ

`food'

\ea%x1522
\label{ex:x1522}
\gll emer(a)  ik-owa \\
      \\
\glt
\z

sago  roast-NMZ

`bread, roasted sago'

A compound type where the nominalized verb precedes the other noun is more common than the one above. When the second part is a human noun, it usually has to be the \textstyleEmphasizedWords{\textsc{agent}} of the verb (\stepcounter{nx}{\thenx}), but when the noun is non-human, it is harder to find a common denominator for the semantic relationships between the parts in different compounds. Quite often the meaning centers around function, purpose or ``typical'' action, place, time etc.

\ea%x52
\label{ex:x52}
\gll uuw-ow(a)  mua \\
      \\
\glt
\z

work-NMZ  man

`worker'

\ea%x53
\label{ex:x53}
\gll in-ow(a)  koora \\
      \\
\glt
\z

sleep-NMZ  house

`bedroom'

\ea%x54
\label{ex:x54}
\gll om-ow(a)  eka \\
      \\
\glt
\z

cry-NMZ  water

`tear'

This compound type particularly easily allows compounds with more than two roots: 

\ea%x55
\label{ex:x55}
\gll ikemik(a)  kaik-ow(a)  mua \\
      \\
\glt
\z

wound  tie-NMZ  man

`doctor'

\ea%x56
\label{ex:x56}
\gll emer(a)  en-ow(a)  mua \\
      \\
\glt
\z

sago  eat-NMZ  man

`a Sepik man (lit: a sago eater)'\footnote{Sepik province is known for its main staple, sago starch.}

\ea%x60
\label{ex:x60}
\gll ama  urup-ow(a)  (epa/kame) \\
      \\
\glt
\z

sun  rise-NMZ  place/side

`east'

In the example (\stepcounter{nx}{\thenx}) the main noun \textstyleStyleVernacularWordsItalic{epa}/\textstyleStyleVernacularWordsItalic{kame} can be dropped, and this happens in some other compounds as well:

\ea%x61
\label{ex:x61}
\gll epir(a)  suruk-ow(a)  (tetelka) \\
      \\
\glt
\z

plate  wipe-NMZ  finger

`forefinger' 

The \textstyleEmphasizedWords{\textsc{coordinate}} compounds are different from the other compounds in that neither of the parts modifies the other. The meaning of the whole is derived from the combined meaning of the two terms. Also, there is no stress centre: both parts of the compound are stressed equally. The number of these compounds is small.

\ea%x50
\label{ex:x50}
\gll emeria  mua \\
      \\
\glt
\z

woman  man

`people'

\ea%x51
\label{ex:x51}
\gll muuka  wiipa \\
      \\
\glt
\z

son  daughter

`children'

The \textstyleAcronymallcaps{N}+\textstyleAcronymallcaps{ADJ} compounds are as hard to distinguish from phrases as some of the other groups mentioned above. Again the uninterruptibility and lexicalized meaning are the main criteria. If the adjective \textstyleStyleVernacularWordsItalic{sepa} `black' is added between the two words in (\stepcounter{nx}{\thenx}), the meaning changes into `a small black man'.

\ea%x57
\label{ex:x57}
\gll mua  gelemuta \\
      \\
\glt
\z

man  small

`a little boy'

\ea%x58
\label{ex:x58}
\gll mia  yoowa \\
      \\
\glt
\z

body/skin  hot

`fever'

\ea%x59
\label{ex:x59}
\gll maa  samora \\
      \\
\glt
\z

thing  bad

`mosquito'

Compounding is a productive process in Mauwake, and it is the most common language-internal means used for adding new lexical items to the language. 

\subsubsection{Derived nouns}
\hypertarget{RefHeading19381935131865}{}
In this section I will discuss derivations where the \textstyleEmphasizedWords{\textsc{end result}} is a noun. There are only two of these: nouns made out of verbs, and noun reduplications. 

\paragraph[Action nominals]{Action nominals}
\hypertarget{RefHeading19401935131865}{}
The process of nominalizing verbs is a straightforward and fully productive process of adding the nominalizing suffix -\textstyleStyleVernacularWordsItalic{owa} to the verb stem. The nominalized verbs most commonly function as nouns, sometimes also as adjectives (\stepcounter{nx}{\thenx}).\footnote{In the Mauwake dictionary some of these nominalized forms have their own entry as if they were fully lexicalized as nouns, but this is to some extent a concession to other languages, where separate nouns may be required for the action nominals and more lexicalized deverbal nouns (for the distinction, see Ylikoski 2003: 193). In Mauwake it is often difficult to establish which of the nominalizations are lexicalized.}

\ea%x62
\label{ex:x62}
\gll uf-\textstyleEmphasizedVernacularWords{owa} \\
      \\
\glt
\z

dance-NMZ

`(the act of) dancing', `(traditional) dance'

\ea%x63
\label{ex:x63}
\gll irak-\textstyleEmphasizedVernacularWords{owa} \\
      \\
\glt
\z

fight-NMZ

`fighting', `fight/war' 

\ea%x1231
\label{ex:x1231}
\gll Fiirim-\textstyleEmphasizedVernacularWords{owa}=pa  opaimika  aakun-e-mik. \\
      \\
\glt
\z

gather-NMZ=LOC  talk  talk-PA-1/3p

`In the meeting we talked.'

\ea%x1247
\label{ex:x1247}
\gll Amina  puk\textstyleEmphasizedVernacularWords{-owa}  eliw(a)  marewa=ke. \\
      \\
\glt
\z

pot  break-NMZ  good  none=CF

`The pot is broken (and) not good' or: `The broken pot is not good.'

Action nominals function like any regular nouns in Mauwake. They can be, for example, a head (\stepcounter{nx}{\thenx}) or a qualifier (\stepcounter{nx}{\thenx}) in a \textstyleAcronymallcaps{NP}, and a first (\stepcounter{nx}{\thenx}) or last element (\stepcounter{nx}{\thenx}) in a compound noun.

\ea%x64
\label{ex:x64}
\gll Siowa  \textstyleEmphasizedVernacularWords{alu-owa}  miim-ap  ekap-o-k. \\
      \\
\glt
\z

dog  make.noise-NMZ  hear-SS.SEQ  come-PA-3s

`He heard the dog's noise and came' or: `The dog heard noise and came.'

\ea%x65
\label{ex:x65}
\gll \textstyleEmphasizedVernacularWords{Irak-owa}  \textstyleEmphasizedVernacularWords{kerer-owa}  epa  weeser-em-ik-eya  {\dots}{\footnotemark} \\
      \\
\glt
\z


fight-NMZ  appear-NMZ  time  finish-SS.SIM-be-2/3s.DS

`As the time of the war was getting close{\dots}' (Lit: `As the war-appearing time was coming to an end{\dots}')

\footnotetext{\textit{Kererowa} is both the head of \textit{irakowa kererowa} and part of the qualifier phrase in \textit{irakowa kererowa epa.} }

\ea%x66
\label{ex:x66}
\gll Oram  \textstyleEmphasizedVernacularWords{niir-ow(a)}  opora  ma-e-m. \\
      \\
\glt
\z

just  laugh-NMZ  talk  say-PA-1s

`I just said it as a joke.'

\ea%x67
\label{ex:x67}
\gll Kaul  \textstyleEmphasizedVernacularWords{wafur-owa } mera  \textstyleEmphasizedVernacularWords{aaw-owa}  eliw. \\
      \\
\glt
\z

hook  throw-NMZ  fish  get-NMZ  all.right

`As for throwing a hook, it is a good way of catching fish.' (Lit: `Hook-throwing is all right for fish-catching.') 

The following expressions form an interesting pair, as (\stepcounter{nx}{\thenx}) is a \textstyleAcronymallcaps{NP} with a nominalized verb as a head, and (\stepcounter{nx}{\thenx}) is a compound noun with a nominalized verb as the first part.

\ea%x424
\label{ex:x424}
\gll mua  aakun-\textstyleEmphasizedVernacularWords{owa} \\
      \\
\glt
\z

man  talk-NMZ

`talk(ing) of man/people', `people's talk'

\ea%x425
\label{ex:x425}
\gll aakun-\textstyleEmphasizedVernacularWords{ow}(\textstyleEmphasizedVernacularWords{a})  mua \\
      \\
\glt
\z

talk-NMZ  man

`a talker', `a spokesman'

Action nominals keep their verb-like property of  being able to take the same arguments and peripherals as the verb serving as the root of the noun. The result is a  nominalized clause, which functions like a noun phrase. This is discussed further in \sectref{sec:5.7} and \sectref{sec:8.3.2}.

Comrie and \citet[334-342]{Thompson2007} list various kinds of other nominalization possibilities,\footnote{Giv\'on calls all of these \textit{lexical nominalizations} (1990:500), and Ylikoski calls them \textit{deverbal nouns} (2003:193) to distinguish them from action nominals.}  but in Mauwake the corresponding expressions are compound nouns or noun phrases consisting of the nominalized verb (or clause) plus another noun, rather than simple nominalizations. 

\ea%x1232
\label{ex:x1232}
\gll ikemika  kaik-\textstyleEmphasizedVernacularWords{ow(a)}  mua \\
      \\
\glt
\z

wound  tie-NMZ  man

`doctor, nurse'

\ea%x1233
\label{ex:x1233}
\gll maa  eneka  teek-\textstyleEmphasizedVernacularWords{ow(a)}  (maa)\footnotemark{} \\
      \\
\glt
\z

thing  tooth  open-NMZ  (thing)

`can opener'

\footnotetext{\textit{Maa eneka} is a compound referring to edible animals; the very generic noun \textit{maa} `thing' may be omitted from the end.}

\paragraph[Noun reduplication]{Noun reduplication}
\hypertarget{RefHeading19421935131865}{}
Reduplication of nouns to denote plurality is a very marginal process in Mauwake, whereas reduplication of verbs (\sectref{sec:3.8.2.4.1}) is much more frequent, and that of adjectives (\sectref{sec:3.3}) also more common. Usually the whole noun is reduplicated; final /a/ is deleted in the reduplicated part of words that are longer than two syllables (\stepcounter{nx}{\thenx}). 

\ea%x68
\label{ex:x68}
\gll Dabuel  \textstyleEmphasizedVernacularWords{poka-poka}  nain=iw  biiris  on-am-ik-e-mik. \\
      \\
\glt
\z

pawpaw  RDP-trunk  that1=INST  bridge  make-SS.SIM-be-PA-1/3p

`They kept making the bridge with pawpaw trunks.'

\ea%x69
\label{ex:x69}
\gll Waaya  pa-ep  \textstyleEmphasizedVernacularWords{kio-kiowa}  naap  uup-e-mik. \\
      \\
\glt
\z

pig  butcher-SS.SEQ  RDP-piece  thus  cook-PA-1/3p

`We butchered the pig and cooked the pieces like that.'

\ea%x426
\label{ex:x426}
\gll \textstyleEmphasizedVernacularWords{Owow-owowa}  ikiw-e-mik. \\
      \\
\glt
\z

RDP-village  go-PA-1/3p

`We went to several villages.'

\subsection{Adjectives}
\hypertarget{RefHeading19441935131865}{}
The existence of noun and verb as universal categories is generally acknowledged, but the status of adjectives is less clear. There is considerable variation among languages as to what belongs to the adjective class, and sometimes a question is posed whether the class exists at all.  But when there is a class of adjectives, the following tendencies emerge: languages that have a small class of adjectives show a lot of similarity in what kinds of concepts they express through this class; and similarly, in languages where the adjective class is large the semantic content of the class is fairly constant \citep[20]{Dixon1977}. Semantically it is somewhat of an in-between category sharing similarities with both nouns and verbs \citep[447]{Lyons1977}. Nouns ``connote the possession of a complex of qualities, and [adjectives] the possession of one single quality'' (Jespersen 1924:81; see also Wierzbicka 1986:362). Nouns have reference, adjectives do not \citep[77]{HakulinenEtAl1979}%Karlsson
. Instead of categorizing like nouns do, adjectives describe \citep[357]{Wierzbicka1986}. They may also code transitory states, and in Giv\'on's (1984:52) time-stability scale they occupy the middle area between nouns and verbs.\footnote{But see \textstyleFootnoteBaseChar{Thompson}'s (1988) criticism on \textstyleFootnoteBaseChar{Giv\'on}'s placing of adjectives on the time-stability scale.} 

The morphological and syntactic coding of ``property concepts'' reflects their semantically ambivalent status: especially in languages which have either no adjectives or only a small adjective class, the concepts are usually expressed via verbs and/or nouns, sometimes by other means \citep[20]{Dixon1977}.

The adjective class in Mauwake is a relatively small open class when compared with nouns and verbs. But compared with some other Papuan languages \citep[50-51]{Dixon1977} it is a fairly large class: the number of non-derived adjectives currently in the dictionary is about 80.\footnote{Usan also has a relatively large adjective inventory (\textstyleFootnoteBaseChar{Reesink 1987}:63).}  The morphological and syntactic similarities and differences between nouns and adjectives were discussed above in \sectref{sec:3.2.2}.  Adjectives do not inflect at all.  

A prototypical adjective functions as the head of an adjective phrase\footnote{Often the head is the sole constituent of the adjective phrase.} (\sectref{sec:4.2}) and may be modified by different intensity adverbs (\sectref{sec:3.9.2}), including the pre-modifier \textstyleStyleVernacularWordsItalic{lawisiw} `rather' (\stepcounter{nx}{\thenx}) and various post-modifiers (\stepcounter{nx}{\thenx}). 

\ea%x70
\label{ex:x70}
\gll Nomokowa  \textstyleEmphasizedVernacularWords{maala}  war-e-k. \\
      \\
\glt
\z

tree  long  cut-PA-3s

`He cut a tall tree.'

\ea%x71
\label{ex:x71}
\gll Waaya  me  \textstyleEmphasizedVernacularWords{maneka},  muuka,  \textstyleEmphasizedVernacularWords{kia  gelemuta}. \\
      \\
\glt
\z

pig  not  big  son  white  small

`The pig was not big, it was a piglet, white (and) small.'

\ea%x72
\label{ex:x72}
\gll Malol  \textstyleEmphasizedVernacularWords{lawisiw  yoowa}. \\
      \\
\glt
\z

open.sea  rather  hard

`(Fishing in the) open sea is rather hard.'

\ea%x73
\label{ex:x73}
\gll Koora  nain  \textstyleEmphasizedVernacularWords{maneka  wenup}. \\
      \\
\glt
\z

house  that1  big  very

`That house is very big.'

Only the following adjectives have been found to be non-scalar:

morena  `male'

suwina  `female'

emi    `taboo(ed)'

enuma\footnote{\textit{Enuma} also means `new' and `green'.}  `alive'

The typical adjectives in Mauwake are all non-derived, and among them are all those listed by \citet[23]{Dixon1977} as the most common adjectives cross-linguistically: large, small, long, short, old, new, good, bad, black, white and red.

Of the various adjective groups mentioned by \citet{Dixon1977}, those of \textstyleEmphasizedWords{\textsc{age}} and value are quite small in Mauwake. Only two of the age adjectives are non-derived, the other two are derived:

awona  `old'  -  enuma  `new'

panewowa  `old'

iperowa  `middle-aged, elder'

 The adjective \textstyleStyleVernacularWordsItalic{awona} `old' refers to the age of things, not people; when used of people, the meaning is `previous' (\stepcounter{nx}{\thenx}). Correspondingly, its antonym \textstyleStyleVernacularWordsItalic{enuma} `new' refers to age of things or recency in humans (\stepcounter{nx}{\thenx}). The adjective referring to age in people,  \textstyleStyleVernacularWordsItalic{panewowa} `old'\footnote{\textstyleFootnoteBaseChar{\textit{Panewowa}} is derived from the verb \textstyleFootnoteBaseChar{\textit{pan}}- `grow old'.} does not have any adjective as an antonym; the noun \textstyleStyleVernacularWordsItalic{takira} `youth' is used instead. \textstyleStyleVernacularWordsItalic{Panewowa} `old' and \textstyleStyleVernacularWordsItalic{iperowa} `middle-aged' do  not indicate age only, but social status as well: it is the middle-aged men, rather than young or old, that have most power and make the important decisions in the community. \textstyleStyleVernacularWordsItalic{Iperowa} is also used for older siblings when the age of siblings is compared.

\ea%x74
\label{ex:x74}
\gll Emeria  \textstyleEmphasizedVernacularWords{panewowa}  nain  Kait  emeria  \textstyleEmphasizedVernacularWords{awona}=ke. \\
      \\
\glt
\z

woman  old  that1  Kait  woman  old=CF

`The old woman is Kait's old (=previous) wife.'

\ea%x75
\label{ex:x75}
\gll Ona  mua  \textstyleEmphasizedVernacularWords{enuma}  iiriw  pani-e-k. \\
      \\
\glt
\z

3s.GEN  man  new  already  grow.old-PA-3s

`Her new husband is (already) old.'

\textstyleEmphasizedWords{\textsc{Value}} adjectives are the following: 

eliwa  `good'  -  samora  `bad'

makena  `true'  -  wadola  `false'

emi    `taboo, forbidden'

\ea%x1760
\label{ex:x1760}
\gll Inasin  opaimika  \textstyleEmphasizedVernacularWords{eliwa}  me  yia  maak-e-mik. \\
      \\
\glt
\z

spirit  talk  good  not  1p.ACC  tell-PA-1/3p

`They did not speak good Tok Pisin (lit: spirit talk) to us.'

\ea%x1759
\label{ex:x1759}
\gll Iiriw  sira  nain  \textstyleEmphasizedVernacularWords{emi}  maneka  wiar  ik-ua. \\
      \\
\glt
\z

earlier  custom  that1  forbidden  big  3.DAT  be-PA.3s

`Earlier that custom was completely forbidden to them.'

The list of \textstyleEmphasizedWords{\textsc{colour}} terms is also very limited; only the first three terms in the list are purely colour terms, all the others have their origin elsewhere:

sepa  `black'

kia    `white'

oka    `red', `brown'

enuma  `green'  {\textless}  `new'

ligam  `yellow'  {\textless}  `turmeric'

ekapina  `blue'  {\textless}  `shrub sp. (used for blue dye)'

foma  `grey'  {\textless}  `ashes'\footnote{cf. Berlin and Kay 1969:4.}

\ea%x1753
\label{ex:x1753}
\gll Aalbok  mia  \textstyleEmphasizedVernacularWords{sepa} \textstyleEmphasizedVernacularWords{} \textstyleEmphasizedVernacularWords{akena}  kerer-e-k. \\
      \\
\glt
\z

black.cuckoo.shrike  body  black  very  become-PA-3s

`The body of the black cuckoo-shrike became very black.'

\ea%x109
\label{ex:x109}
\gll Konima  nain  \textstyleEmphasizedVernacularWords{sepa  kia}. \\
      \\
\glt
\z

cloth  that1  black  white

`The cloth is black-and-white.'

\ea%x1754
\label{ex:x1754}
\gll Mia  afif(a)  \textstyleEmphasizedVernacularWords{oka},  \textstyleEmphasizedVernacularWords{oka}  gelemuta. \\
      \\
\glt
\z

body  hair  red,  red  small

`The feathers were red, (it was) red and small.'

\ea%x1755
\label{ex:x1755}
\gll Komora  nain  \textstyleEmphasizedVernacularWords{kia  ne  maneka}  wenup. \\
      \\
\glt
\z

cuscus  that1  white  ADD  big  very

`That cuscus is/was white and very big.'

In (\stepcounter{nx}{\thenx}) the dimensional adjective for `small' may follow directly after the colour adjective, whereas the adjective \textstyleStyleVernacularWordsItalic{maneka} `big' needs a connective between the two adjectives in (\stepcounter{nx}{\thenx}), because \textstyleStyleVernacularWordsItalic{maneka} is used as an intensifier when immediately following a colour term, and \textstyleStyleVernacularWordsItalic{kia maneka} would mean `completely white'.

The darkness of a colour is expressed through the adjectives \textstyleStyleVernacularWordsItalic{sepa} `black' and \textstyleStyleVernacularWordsItalic{kia} `white' used as modifiers of the main colour adjective (\stepcounter{nx}{\thenx}).

\ea%x110
\label{ex:x110}
\gll ifa  \textstyleEmphasizedVernacularWords{enuma  lawisiw  sepa} \\
      \\
\glt
\z

leaf  new/green  rather  black

`a dark green leaf'

Among the adjectives denoting \textstyleEmphasizedWords{\textsc{dimension}} there are a number of terms describing various kinds of thinness and thickness, as well as shortness. 

maneka  `large'  -  gelemuta  `small'

maala  `long'  -  iiwa  `short'

kuruma  `thick'  -  gawela  `thin'

fula(kia)  `fat'  -  bebeta  `slim, skinny'

teena  `thin'

komosia  `small, short'

\ea%x1756
\label{ex:x1756}
\gll Epa  dabela=pa  mia  suuw-owa  \textstyleEmphasizedVernacularWords{gawela}  suuw-ap  \\
      \\
\glt
\z

place  cold=LOC  body  push-NMZ  thin  push-SS.SEQ  

mia  fulil-i-nan.

body  feel.cold-Np-FU.2s

`When you wear thin clothes (mia suuwowa) in a cold place you will feel cold.' 

\ea%x76
\label{ex:x76}
\gll Owor(a)  ara  \textstyleEmphasizedVernacularWords{teena } nain  ku-i-non. \\
      \\
\glt
\z

betelnut.palm  trunk  thin  that1  break-Np-FU.3s

`The thin betelnut palm trunk will break.'

\ea%x77
\label{ex:x77}
\gll Epa  dabel-al-eya  mia  suuw-owa  \textstyleEmphasizedVernacularWords{kuruma } wu-e. \\
      \\
\glt
\z

place  cold-INCH-2/3s.DS  body  push-NMZ  thick  put-IMP.2s

`When it gets cold, put thick clothes on.'

The group of adjectives denoting \textstyleEmphasizedWords{\textsc{physical property}} is larger than any of the other groups and includes several antonym pairs. The list below is just a sample:

yoowa  `hot, hard'  -  dabela  `cold'

supuka  `wet'  -  ififa  `dry'

pina  `heavy'  -  efefa  `light'

kaken  `straight'  -  meka  `crooked'

melina  `clear'  -  wiwisa  `murky'

anima  `sharp'  -  duduwa  `blunt'

dubila  `slippery, smooth'

itita  `soft'

masia  `bitter (taste)'

siina  `tight'

\ea%x78
\label{ex:x78}
\gll Iwera  \textstyleEmphasizedVernacularWords{ififa}  ora-eya  fiirim-i-mik. \\
      \\
\glt
\z

coconut  dry  descend-2/3s.DS  gather-Np-PR.1/3p

`When the dry coconuts drop we gather them'.

\ea%x1758
\label{ex:x1758}
\gll {\dots}epia  foma  lawisiw  \textstyleEmphasizedVernacularWords{yoowa}  ik-ua. \\
      \\
\glt
\z

fire(wood)  ashes  rather  hot  be-PA.3s

`{\dots} the ashes were rather hot.'

\textstyleEmphasizedWords{\textsc{Human propensity}}  adjectives is the second largest group. 

lebuma  `lazy'  -  topia  `diligent'

asia  `wild'  -  memela  `tame'

lebuma  `lazy'

momora  `foolish'

popora  `quiet'

yamunsia  `stingy'

\ea%x1757
\label{ex:x1757}
\gll Takira=ke  keker  op-ap  \textstyleEmphasizedVernacularWords{popor(a)}  maneka  ik-e-mik. \\
      \\
\glt
\z

boy=CF  fear  hold-SS.SEQ  quiet  big  be-PA-1/3p

`The boys were afraid and very quiet.'

\ea%x1418
\label{ex:x1418}
\gll Mua  \textstyleEmphasizedVernacularWords{lebuma}  nain  emeria  me  wi-i-mik. \\
      \\
\glt
\z

man  lazy  that1  woman  not  give.them-Np-PR.1/3p

`We do not give wives to lazy men.'

Although Mauwake has a considerable inventory of adjectives for a Papuan language, in actual use they are rather infrequent.\footnote{Their frequency in the text material is about 1.5\% of all the words.} Especially physical property and human propensity are frequently expressed through verbs which have been verbalized from adjectives. A true adjective is a more likely candidate to indicate a stable or essential quality of the head noun (\stepcounter{nx}{\thenx}), whereas the verbalized form is used for more temporary characteristics (\stepcounter{nx}{\thenx})-(\stepcounter{nx}{\thenx}).

\ea%x1419
\label{ex:x1419}
\gll Sama=pa  or-owa  nain  eliw,  nain  ikoka  or-op \\
      \\
\glt
\z

stairs=LOC  descend-NMZ  that1  well  that1  later  descend-SS.SEQ

or-op  or-op  \textstyleEmphasizedVernacularWords{lebum(a)-ar-i-nan},  epasia  akena.

descend-SS.SEQ  descend-SS.SEQ  lazy-INCH-Np-FU.2s  far  very

`Descending on the stairs is all right, but later when you have gone down and down and down you will be lazy/tired, (as) it is very far.'

\ea%x79
\label{ex:x79}
\gll Moma  \textstyleEmphasizedVernacularWords{kasu(a)-ar-eya } me  enim-i-mik. \\
      \\
\glt
\z

taro  hard-INCH-2/3s.DS  not  eat-Np-PR.1/3p

`We don't eat hard taro.' (Lit: `When taro is hardened, we don't eat it.')

\ea%x80
\label{ex:x80}
\gll \textstyleEmphasizedVernacularWords{Yamunsi(a)-ar-iwkin}  me  wia  nokar-e-m. \\
      \\
\glt
\z

stingy-INCH-2/3p.DS  not  3p.ACC  ask-PA-1s

`They were (being) stingy, (so) I didn't ask them.'

\textsc{Speed} is expressed through adverbs or verbs rather than adjectives.

\textstyleEmphasizedWords{\textsc{Comparison}} of adjectives is an area where there is very little differentiation in many Papuan languages, including Mauwake.\footnote{See \textstyleBibliogBaseChar{Roberts} (1987:134-5), \textstyleBibliogBaseChar{Reesink} (1987:68), \textstyleBibliogBaseChar{Hardin} (2002:63-4); \textstyleBibliogBaseChar{Haiman} reports only three or four true adjectives for Hua, and does not mention comparison (1980:268).} Intensifiers are used for this function, as well as the verb \textstyleStyleVernacularWordsItalic{nomak}\textstyleEmphasizedVernacularWords{-} `overcome, surpass'. 

\ea%x81
\label{ex:x81}
\gll Poka  fain  maala,  nain  \textstyleEmphasizedVernacularWords{nomak-e-k},  ne  oko  nain  \textstyleEmphasizedVernacularWords{maala \\
      \\
\glt
\z

stilt  this   long  that1  surpass-PA-3s  ADD  other  that1  long

\textstyleEmphasizedVernacularWords{akena}.

very

`This stilt is longer than that, and/but the other one is the longest (lit: very long).'

Two adjectives can also be compared by contrasting them: 

\ea%x441
\label{ex:x441}
\gll Nomokow(a)  kakawa  fain  \textstyleEmphasizedVernacularWords{iiwa},  oko  \textstyleEmphasizedVernacularWords{maala}  puuk-a-n. \\
      \\
\glt
\z

tree  part  this   short  other  long  cut-PA-2s

`You cut this plank shorter than the other one.' (Lit: `You cut this plank short, the other long.')

Adjectives denoting size form a scale of three: \textstyleStyleVernacularWordsItalic{gelemuta} `small', \textstyleStyleVernacularWordsItalic{manisiri} `biggish', \textstyleStyleVernacularWordsItalic{maneka} `big'. Usually, if three degrees of comparison are needed, it is possible to express them periphrastically, but that is seldom necessary.  Comparison as a functional domain is discussed in \sectref{sec:6.5}. 

Like nouns, adjectives can also be \textstyleEmphasizedWords{\textsc{reduplicated}} for plural (\sectref{sec:2.3.3.2}). Reduplication of adjectives is not very common, but it is more frequent than that of nouns. 

\ea%x85
\label{ex:x85}
\gll Maa  eneka  kes  \textstyleEmphasizedVernacularWords{mane-maneka}  oram  iw-e-mik. \\
      \\
\glt
\z

thing  tooth  case  RDP-big  just  give.him-PA-1/3p

`They just gave him big cases of meat tins.'

The adjective \textstyleStyleVernacularWordsItalic{gelemuta} `small' has several reduplicated forms: \textstyleStyleVernacularWordsItalic{gelemuti-tik}\textstyleEmphasizedVernacularWords{,} \textstyleStyleVernacularWordsItalic{gelemutu-mut}\textstyleEmphasizedVernacularWords{,} \textstyleStyleVernacularWordsItalic{gele-gelemuti-tik}\textstyleEmphasizedVernacularWords{.} 

\ea%x486
\label{ex:x486}
\gll Waaya  \textstyleEmphasizedVernacularWords{gelemutu-mut}  pu-puuk-e. \\
      \\
\glt
\z

pig  small-RDP  RDP-cut-IMP.2s

`Cut the pig into small pieces.'

Occasionally reduplication can be used for an intensifying function as well. The noun modified by the reduplicated adjective in (\stepcounter{nx}{\thenx}) is either singular or plural, in (\stepcounter{nx}{\thenx}) it is definitely singular.

\ea%x485
\label{ex:x485}
\gll Biiris  eliwa  me  on-a-mik,  \textstyleEmphasizedVernacularWords{damo-damola}=ko. \\
      \\
\glt
\z

bridge  good  not  make-PA-1/3p  RDP-bad=NF

`They didn't make a good bridge (but) very bad.' (or: `{\dots}good bridges but bad.')

\ea%x86
\label{ex:x86}
\gll {\dotsifa}=ke  keraw-a-k,  mamepaperuma  \textstyleEmphasizedVernacularWords{gele-gelemuti-tik}  nain=ke. \\
      \\
\glt
\z

{\dots}snake=CF  bite-PA-3s  death.adder  RDP-small-RDP  that1=CF

`{\dots} a snake bit him, a very small death adder.'

\textstyleEmphasizedWords{\textsc{New adjectives}} are derived from verbs with the nominalizing suffix \nobreakdash-\textstyleStyleVernacularWordsItalic{owa}. This is not a very productive process.

kekanowa  `strong'  {\textless}  kekan-  `be strong' 

panewowa  `old'  {\textless}  pan-  `become old'

kainowa  `high (voice)'  {\textless}  kain-  `be high (voice)'

bolonowa  `slack'  {\textless}  bolon-  `be slack'

\ea%x1766
\label{ex:x1766}
\gll No  mua  samora,  mua  emin(a)  \textstyleEmphasizedVernacularWords{kekan}\textstyleEmphasizedVernacularWords{-}\textstyleEmphasizedVernacularWords{owa} \\
      \\
\glt
\z

2s.UNM  man  bad  man  occiput  be.strong-NMZ  

nefa  na-i-kuan.

2s.ACC  say-Np-FU.3p

`They will call you a bad man, a pig-headed (lit: strong occiput) man.'

\ea%x1765
\label{ex:x1765}
\gll Someka  aw-i-ya  nain  iwakara  \textstyleEmphasizedVernacularWords{kain-owa}  maneka  \\
      \\
\glt
\z

song  weave-Np-PR.3s  that1  neck  be.high-NMZ  big  

aw-i-ya.

weave-Np-PR.3s

`When (s)he sings, (s)he sings with a very high voice.'

\ea%x1767
\label{ex:x1767}
\gll Makera  \textstyleEmphasizedVernacularWords{saawirin}\textstyleEmphasizedVernacularWords{-}\textstyleEmphasizedVernacularWords{owa}  kaik-a-m. \\
      \\
\glt
\z

cane  surround-NMZ  tie-PA-1s

`I tied the cane round.'

Adjectives can be made into verbs by zero verb formation (\sectref{sec:3.8.2.2.1}) or by the inchoative verbaliser \nobreakdash-\textit{ar}  (\sectref{sec:3.8.2.2.2}).

\subsection{Quantifiers}
\hypertarget{RefHeading19461935131865}{}
Quantifiers are a small closed class of words. The group can be divided into numeral and non-numeral quantifiers. The reasons for treating them as a group of their own, separate from adjectives, are the following. Their position is after the adjectives in a \textstyleAcronymallcaps{NP}.\footnote{Actually it is the Quantifier Phrase that comes after the Adjective Phrase, but usually the phrases consist of only one word, a quantifier in the former and an adjective in the latter.} Some of the numerals consist of a phrase or even a clause, but they still function as a single unit. And semantically quantifiers are quite different from adjectives.

A quantifier is the only obligatory element in a quantifier phrase (\textstyleAcronymallcaps{QP,} \sectref{sec:4.3}). These are used as post-modifiers in a \textstyleAcronymallcaps{NP}, where their position is between an adjective phrase (\textstyleAcronymallcaps{AP}) and demonstrative (\stepcounter{nx}{\thenx}), or by themselves as a non-verbal predicate (\stepcounter{nx}{\thenx}). 

\ea%x87
\label{ex:x87}
\gll I  koora  maneka  \textstyleEmphasizedVernacularWords{kuisow}  nain  yiar  aw-o-k. \\
      \\
\glt
\z

1p.UNM  house  big  one  that1  1p.DAT  burn-PA-3s

`That one big house of ours burned.'

\ea%x442
\label{ex:x442}
\gll Mua  iperowa  \textstyleEmphasizedVernacularWords{arow}  \textstyleEmphasizedVernacularWords{muutiw.} \\
      \\
\glt
\z

man  middle-aged  three  only.

`There are/were only three middle-aged men.' (Lit: `The middle-aged men (are) only three.') 

The numerals, especially \textit{erup} `two', may be added to a pronoun to quantify it: the numeral occurs following a reflexive (or occasionally unmarked) form of the pronoun, but the pronoun is used like an unmarked pronoun.

\ea%x89
\label{ex:x89}
\gll Ne  \textstyleEmphasizedVernacularWords{wiam  erup}  pun  epa  neeke  or-o-mik. \\
      \\
\glt
\z

ADD  3p.REFL  two  too  place  there.CF  descend-PA-1/3p

`And the two of them too went down there.'

\subsubsection{Numerals}
\hypertarget{RefHeading19481935131865}{}
The traditional counting system in Mauwake is quinary, i.e. based on five\footnote{In New Guinea languages, there are counting systems based on two, five ten and twenty, as well as systems that use different body parts as tallies. All of these systems are present in the Madang area languages as well (\textstyleBibliogBaseChar{Lean 1991}).}, and counting is gestured using the fingers.\footnote{To count, the fingers are bent down one by one, starting from the little finger of the right hand, and proceeding towards the thumb, then on to the little finger of the other hand etc.}  

kuisow  `one'

erup  `two'

arow  `three'

erepam  `four'

ikur / wapen inawiya  `five' / `a hand sleeps'

(ikur) okai(wi)=pa kuisow  `six' (lit: `(five) one on/from the other side')

(ikur) okai(wi)=pa erup  `seven'

(ikur) okai(wi)=pa arow  `eight'

(ikur) okai(wi)=pa erepam  `nine'

iimeka kuisow / okaipa okaipa inek  `ten' / `both sides sleep'

\ea%x90
\label{ex:x90}
\gll Uura  ama  \textstyleEmphasizedVernacularWords{ikur  okai}(\textstyleEmphasizedVernacularWords{wi})\textstyleEmphasizedVernacularWords{=pa  arow}  naap  in-e-mik. \\
      \\
\glt
\z

night  sun  five  other.side=LOC  three  thus  sleep-PA-1/3p

`In the evening we slept at around eight o'clock.'

Nowadays the borrowed Tok Pisin numerals have largely superseded the vernacular numerals, especially those indicating numbers ten and above (\stepcounter{nx}{\thenx}).  There are no terms for `hundred', `thousand' or bigger numbers in the vernacular system.

\ea%x91
\label{ex:x91}
\gll Mokoma  \textstyleEmphasizedVernacularWords{ten  arow}  aaw-o-k. \\
      \\
\glt
\z

year  ten  three  get-PA-3s

`He became 30 years old.'

Numerals can be modified with the intensity adverbs \textstyleStyleVernacularWordsItalic{kakeniw}  `\textstyleFreeTranslationChar{correctly, exactly}',  \textstyleStyleVernacularWordsItalic{akena}\textstyleEmphasizedVernacularWords{}  `really, truly' or \textstyleStyleVernacularWordsItalic{muutiw} `only'.

\ea%x443
\label{ex:x443}
\gll \textstyleEmphasizedVernacularWords{Erepam  kaken}\textstyleEmphasizedVernacularWords{=iw}  mik-a-mik. \\
      \\
\glt
\z

four  straight-ISOL  spear-PA-1/3p

`We speared exactly four.'

\ea%x661
\label{ex:x661}
\gll Mua  \textstyleEmphasizedVernacularWords{arow  akena}  epa  nain  iimar-e-mik. \\
      \\
\glt
\z

man  three  truly  place  that  stand.up-PA-1/3p

`Exactly three men stood at that place.'

When the number is somewhat uncertain and the disjunctive connective \textstyleStyleVernacularWordsItalic{e} `\textstyleFreeTranslationChar{or}' and/or the question marker -\textstyleStyleVernacularWordsItalic{i}  is used, either the smaller or the bigger number may be mentioned first.

\ea%x1416
\label{ex:x1416}
\gll Waaya  maneka  wiowa  \textstyleEmphasizedVernacularWords{erup=i  e  arow}  naap  mik-iwkin  \\
      \\
\glt
\z

pig  big  spear  two=QM  or  three  thus  hit-2/3p.DS  

um-i-ya.

die-Np-PR.3s

`When a big pig is hit with two or three spears it dies.'

\ea%x92
\label{ex:x92}
\gll Mua  wiam  \textstyleEmphasizedVernacularWords{ikur=i  erepam}  naap  wia  aaw-e-mik. \\
      \\
\glt
\z

man  3p.REFL  five=QM  four  thus  3p.ACC  get-PA-1/3p

`They took/got those four or five men.'

Repetition (\stepcounter{nx}{\thenx}) or reduplication (\stepcounter{nx}{\thenx}) of the numerals indicates manner: `so and so many \textstyleEmphasizedWords{\textsc{at a time}}'. The reduplicated form of \textstyleStyleVernacularWordsItalic{kuisow} `\textstyleFreeTranslationChar{one}', \textstyleStyleVernacularWordsItalic{kui-kuisow},  has two meanings: `one by one' and `a few'. 

\ea%x93
\label{ex:x93}
\gll Naap  \textstyleEmphasizedVernacularWords{kuisow  kuisow}  aaw-ikiw-e-mik. \\
      \\
\glt
\z

thus  one  one  get-go-PA-1/3p

`They kept getting them one at a time as they went.

\ea%x94
\label{ex:x94}
\gll Waaya  merena  \textstyleEmphasizedVernacularWords{ere-erup}  kaik-ap{\dots} \\
      \\
\glt
\z

pig  leg  RDP-two  tie-SS.SEQ

`I tied the pig's legs two and two together and {\dots}'

Money is counted using different nouns indicating certain amounts:

maamuma ({\textless maa mumua)}   '10 toea', also generic `money', lit: `seed'

fuluwa  `1 kina', lit: `hole' (the coin has a hole)

ifa    `2 kina', lit: `leaf'

ifa oka  `5 kina', lit: `red leaf'

kuuma  `10 kina', lit: `stick'\footnote{From a stick of tobacco, used for payment in the colonial days.} 

\ea%x97
\label{ex:x97}
\gll \textstyleEmphasizedVernacularWords{Kuuma  kuisow  ifa  erup}  naap  yia  sesenar-e-mik. \\
      \\
\glt
\z

stick  one  leaf  two  thus  1p.ACC  buy-PA-1/3p

`They paid to us (lit: bought us for) 14 kina.' 

Mauwake has no separate words for \textstyleEmphasizedWords{ordinal} numbers. To indicate numerical order, various structures are employed.  In many cases the cardinal numbers can be used:

\ea%x96
\label{ex:x96}
\gll Mua  arow  epa  nain  iimar-e-mik,  yos=ke  \textstyleEmphasizedVernacularWords{erepam}. \\
      \\
\glt
\z

man  three  place  that1  stand-PA-1/3p  1s.FC=CF  four

`Three men were standing there, and I was the fourth.'

\ea%x428
\label{ex:x428}
\gll Koora  tuun-e:  \textstyleEmphasizedVernacularWords{kuisow  iki}(\textstyleEmphasizedVernacularWords{w})\textstyleEmphasizedVernacularWords{-}(\textstyleEmphasizedVernacularWords{e})\textstyleEmphasizedVernacularWords{p  erepam},  ne  \textstyleEmphasizedVernacularWords{oko } nain \\
      \\
\glt
\z

house  count-IMP.2s  one  go-SS.SEQ  four  ADD  other  that1

ona  koora.

3s.GEN  house

`His house is the fifth one' (Lit: `Count the houses: one to four, and the other/next is his house.')

In the case of time units, cardinal numbers are combined with the verb \textstyleStyleVernacularWordsItalic{ikiw}- `go':

\ea%x427
\label{ex:x427}
\gll \textstyleEmphasizedVernacularWords{Fofa  okai}(\textstyleEmphasizedVernacularWords{wi})\textstyleEmphasizedVernacularWords{=pa  arow  ikiw-eya}  ekap-i-non. \\
      \\
\glt
\z

day  other.side=LOC  three  go-2/3s.DS  come-Np-FU.3s

`He will come on the ninth day.' (Lit: `When eight days have gone he will come').

Order can also be indicated through verbs like \textstyleStyleVernacularWordsItalic{murar}- and \textstyleStyleVernacularWordsItalic{ook}- `follow'. 

\ea%x98
\label{ex:x98}
\gll Wi  Ulingan=ke  nomak-e-mik.  Ne  i  Moro  \\
      \\
\glt
\z

3p.UNM  Ulingan=CF  win-PA-1/3p  ADD  1p.UNM  Moro

\textstyleEmphasizedVernacularWords{murar-e-mik}.

follow-PA-1/3p

`The Ulingan people/team won. And (we from) Moro came second.'

Numbers are \textstyleEmphasizedWords{not} used when listing one's children. The terms \textstyleStyleVernacularWordsItalic{iperowa} `firstborn', \textstyleStyleVernacularWordsItalic{ookap onarowa} `following' (used repeatedly, if necessary) and \textstyleStyleVernacularWordsItalic{kapa} `lastborn' are employed for that.

\subsubsection{Non-numeral quantifiers}
\hypertarget{RefHeading19501935131865}{}
Some non-numeral quantifiers can only be used with either count or mass nouns, others occur with both. Those that can be used with both are:

senam  `too much/too many'

unowiya  `all'  (from: \textstyleStyleVernacularWordsxiiptItalic{unowa} `\textstyleFreeTranslationChar{many}' plus comitative clitic =\textstyleStyleVernacularWordsxiiptItalic{iya})

iiwawun  `all/altogether'

\ea%x665
\label{ex:x665}
\gll Moma  \textstyleEmphasizedVernacularWords{senam}  en-e-mik. \\
      \\
\glt
\z

taro  too.much  eat-PA-1/3p

`We ate too much taro.'

\ea%x666
\label{ex:x666}
\gll Nomokowa  \textstyleEmphasizedVernacularWords{senam}  war-e-man. \\
      \\
\glt
\z

tree  too.many  cut.PA-2p

`You cut too many trees.'

\ea%x99
\label{ex:x99}
\gll Yagin  eka=pa  \textstyleEmphasizedVernacularWords{unow=iya}  nan  yaki-e-mik. \\
      \\
\glt
\z

Yagin  water=LOC  many=COM  there  bathe-PA-1/3p

`We all bathed there at Yagin together.'

The following are only used with \textstyleEmphasizedWords{\textsc{count}} nouns:

papako\footnote{\textstyleFootnoteBaseChar{\textit{Papako}} is actually a plural indefinite `other'(\sectref{sec:3.7.2}), but it has a secondary function as a quantifier.}  `other/\textstyleFreeTranslationChar{some/a few}'

unowa  `\textstyleFreeTranslationChar{many}'

unow onaiya  `all' (from \textstyleStyleVernacularWordsItalic{unowa} plus \textstyleStyleVernacularWordsItalic{onaiya} `together with')

wenup  `\textstyleFreeTranslationChar{lots of}'

\ea%x100
\label{ex:x100}
\gll Mua  \textstyleEmphasizedVernacularWords{unowa},  emeria  \textstyleEmphasizedVernacularWords{papako}  um-e-mik. \\
      \\
\glt
\z

man  many  woman  some  die-PA-1/3p

`Many men and some women died.'

\ea%x667
\label{ex:x667}
\gll Ipia  saana=pa  iina  \textstyleEmphasizedVernacularWords{wenup}. \\
      \\
\glt
\z

rain  season=LOC  mosquito  lots.of

`In the rainy season there are lots of mosquitoes.'

Both \textstyleStyleVernacularWordsItalic{wenup} and\textstyleEmphasizedVernacularWords{} \textstyleStyleVernacularWordsItalic{unowa}\textstyleEmphasizedVernacularWords{} can be intensified with \textstyleStyleVernacularWordsItalic{akena} `very'; \textstyleStyleVernacularWordsItalic{unowa}  may also be intensified with \textstyleStyleVernacularWordsItalic{wenup}\textstyleEmphasizedVernacularWords{} `\textstyleFreeTranslationChar{very}' (\stepcounter{nx}{\thenx}); or with\textstyleEmphasizedVernacularWords{} \textstyleStyleVernacularWordsItalic{maneka} (lit: `\textstyleFreeTranslationChar{big}') that gives it the meaning `\textstyleFreeTranslationChar{all}' (\stepcounter{nx}{\thenx}).

\ea%x809
\label{ex:x809}
\gll Siipepe  kokora  maroka  \textstyleEmphasizedVernacularWords{wenup  akena}  ika-i-ya. \\
      \\
\glt
\z

Siipepe  riverbed  prawn  lots.of  very  be-Np-PR.3s

`There are lots of prawns in the Siipepe riverbed.'

\ea%x101
\label{ex:x101}
\gll Iinan  aasa  nepa  saarik,  \textstyleEmphasizedVernacularWords{unow(a)  akena/wenup}. \\
      \\
\glt
\z

sky  canoe  bird  like  many  very

`The planes were like birds, very many.'

\ea%x102
\label{ex:x102}
\gll Emeria  \textstyleEmphasizedVernacularWords{unow}(\textstyleEmphasizedVernacularWords{a})  \textstyleEmphasizedVernacularWords{maneka}  sosora  bee-beela  a-e-mik. \\
      \\
\glt
\z

woman  many  big  grass.skirt  RDP-rotten  tie-PA-1/3p

`All the women put on rotten grass skirts.'

The negation of the universal quantifier is discussed below in \sectref{sec:6.2.2}.

The following quantifiers only occur with \textstyleEmphasizedWords{\textsc{mass}} nouns:

maneka   `\textstyleFreeTranslationChar{a lot/much}' (lit: `big')

gelemuta  `\textstyleFreeTranslationChar{little}'

lawiliw    `\textstyleFreeTranslationChar{somewhat/a little}'

\ea%x103
\label{ex:x103}
\gll Eka  yoowa=pa  aaya  \textstyleEmphasizedVernacularWords{maneka/gelemuta}  wu-e. \\
      \\
\glt
\z

water  hot=LOC  sugar  big/little  put-IMP.2s

`Put a lot of/a little sugar in the tea.'

The following non-numeral quantifiers also function as degree/intensity adverbs, modifying a verb: \textstyleStyleVernacularWordsItalic{iiwawun}, \textstyleStyleVernacularWordsItalic{lawiliw}, \textstyleStyleVernacularWordsItalic{senam} and \textstyleStyleVernacularWordsItalic{wenup} (\sectref{sec:3.9.2}).

  \stepcounter{nx}{\thenx}x511)  Yos=ke  \textstyleEmphasizedVernacularWords{lawiliw}  asip-i-yem.\\
1s=CF  somewhat  help-Np-PR.1s

`I am helping her somewhat/a little.'

\ea%x512
\label{ex:x512}
\gll Iperowa=ke  \textstyleEmphasizedVernacularWords{senam}  kekan-e-mik. \\
      \\
\glt
\z

middle.aged=CF  too.much  be.strong-PA-1/3p

`The middle-aged men were very strong (in their opinion).'

\ea%x513
\label{ex:x513}
\gll Waaya  mik-amkun  \textstyleEmphasizedVernacularWords{iiwawun}  um-o-k.  \\
      \\
\glt
\z

pig  spear-1s/p.DS  altogether  die-PA-3s

`When I speared the pig it died completely.'

\textstyleEmphasizedWords{\textsc{Fractions}} are hard to express in Mauwake.\footnote{I have not seen fractions treated in grammars of Papuan languages, but know from discussions with colleagues that translating fractions is a major problem not only in Mauwake but in other Papuan languages as well.} The noun \textstyleStyleVernacularWordsItalic{enakiwa} `half' is sometimes also used for unspecified `part', and \textstyleStyleVernacularWordsItalic{okaiwi} `one/other side' can be used for `half', when a clearly bounded entity is divided in half (\stepcounter{nx}{\thenx}). I have not found other terms indicating fractions. Longer expressions are needed for them e.g. `divide into ten parts and take one part'.

\ea%x104
\label{ex:x104}
\gll Yabuela  \textstyleEmphasizedVernacularWords{okaiwi}  enak-e. \\
      \\
\glt
\z

pawpaw  one.side  feed.me-IMP.2s

`Give me half of the pawpaw to eat.'

\subsection{Pronouns}\footnotemark{}
\hypertarget{RefHeading19521935131865}{}
\footnotetext{ Most of the material in this section has been published in my earlier paper (J\"arvinen 1991).}
\subsubsection{Introduction}
\hypertarget{RefHeading19541935131865}{}
Pronouns are a closed class of words. According to traditional grammar, pronouns can substitute for nouns, but actually they substitute for full noun phrases. 

Pronouns in Mauwake only include personal pronouns. Demonstratives, which are like pronouns in some respects, are discussed under deictics (\sectref{sec:3.6.2}). The indefinites, which are used as modifiers in a noun phrase, are closely related to question words and are treated in \sectref{sec:3.7.2}.

In principle all the pronouns in Mauwake are used for humans only. In legends also spirits can be referred to by these pronouns since they sometimes act like humans and can take human form. There is no third person singular pronoun for non-humans. 

\citet{Wurm1982} posited three typological sets of personal pronouns for Papuan languages, and mentioned Madang province as an area where set III is particularly widespread. The basic forms of Wurm's set III pronouns are:

  singular  plural

1  da\~{ta\~{}ya}  ki\~{ti}

2  na  nik

3  nu\footnote{The third person plural form is not included in Wurm's typology because of gaps in the material and greater variability than in the other person forms.}  \citep[40-42]{Wurm1982}

In all the three pronoun sets fronting of vowels often goes together with plurality (ibid. 78), the non-singular forms in Papuan languages being derived from the singular forms \citep[361]{Franklin1979}. 

With more data and after more rigorous and detailed work on the \textstyleAcronymallcaps{TNG} pronouns, \citet[5]{Ross1995} gives the following as reconstructions of Proto Madang and Proto Croisilles free pronouns:





\begin{tabular}{lllllll}
\mytoprule
 & 1s & 2s & 3s & 1p & 2p & 3p\\
Proto Madang & *ya & *na & *ua/*nu & *i- & *ni-/*ta- & {\dots}\\
Proto Croisilles & *ya & *na/*ni & *ua/*nu & *i[ge]/*i[na] & *ni[ge] & *ua[ge]/*ua[na]\\
\mybottomrule
\end{tabular}



\begin{table}
\caption{Proto Madang and Proto Croisilles free pronouns}
\label{tab:8}
\end{table}

For different functions in the clause Papuan languages often have one or two classes, or functional sets, of pronouns with or without prepositions or suffixes to mark the appropriate cases. Amele \citep{Roberts1987}, Maia \citep[71]{Hardin2002}, Hua \citep[215]{Haiman1980}, Waskia \citep[53]{RossEtAl1978}%Paol
 and Bargam \citep[29]{Hepner2002} have only one basic set each, to which postpositions or suffixes are added. Usan \citep{Reesink1987} and Siroi \citep{Wells1979} each have a nominative and a possessive set. Most Finisterre-Huon languages have different sets for regular and emphatic pronouns (McElhanon 1973).

Person is the more basic category than number in the pronoun systems of Papuan languages \citep[69]{Foley1986}. As for number, it is most common just to have a two-way distinction between singular and plural, but dual forms are also quite widespread in \textstyleAcronymallcaps{TNG} languages, and trial forms are found in some areas as well. An inclusive-exclusive distinction in the first person plural form is not common \citep[60]{Wurm1982} like it is in Austronesian languages, but according to \citet[56]{Ross2005} it has probably been an areal feature for a long time, even before the Austronesians arrived. 

Morphological resemblance between free pronouns and some verbal affixes, most commonly subject markers, is fairly widespread in Papuan languages \citep{Franklin1979}. It is not unusual to find that verbal affixes, e.g. object markers, make fewer person/number distinctions than free pronouns \citep[67]{Foley1986}.

In the following respects Mauwake manifests general typological features of \textstyleAcronymallcaps{TNG} Madang pronouns. There is no gender or noun class system that would be indicated through concord and marking with nouns and/or pronouns. Also, the morphology is suffixal rather than prefixal. There is no inclusive-exclusive distinction. Possession is marked through suffixation on the personal pronouns \citep[40-42]{Wurm1982}.  

The basic unmarked pronouns in Mauwake reflect the Proto Croisilles forms rather closely, apart from the third person plural form \textstyleStyleVernacularWordsItalic{wi}, which \citet[23]{Ross1996} mentions as an innovation *\textstyleStyleVernacularWordsItalic{u-i}- shared by the Kumil languages and the neighbouring Kaukombar languages. The ending -\textstyleStyleVernacularWordsItalic{fa} in the first and second person singular accusative pronouns is an innovation in the Kumil languages only. 

Some features in the Mauwake pronoun system not typical of Papuan languages are the existence of dative pronouns and also their use as possessives, and the distinction between the unmarked pronouns and the focal pronouns. 

The personal pronoun system in Mauwake is very regular, including the first, second and third persons both in singular and plural. Normally the plural form can also be used for dual; the dual number is only marked in one group, and there by adding a numeral rather than through affixation (\sectref{sec:3.5.8}). Since dual number does not occur in verb person marking either, apart from the first person imperative form, it is not very significant in the category of number. Spatial deixis is not marked in the personal  pronoun system in Mauwake. The case is marked to some extent. \tableref{tab:9} lists the personal pronouns in Mauwake:





\begin{tabular}{llllllllll}
\mytoprule
 & \multicolumn{2}{l}{{\bfseries FREE}

 & {\bfseries ACC} & {\bfseries GEN} & {\bfseries DAT} & {\bfseries ISOL} & {\bfseries RESTR} & {\bfseries REFL} & {\bfseries COM}\\
 & {\bfseries UNM} & {\bfseries Focal} &  &  &  &  &  &  & \\
1s & yo & yo-s & efa & y-ena & efa-r & ya-isow & yena-iw/yos-iw & y-ame & efa-m-iya\\
2s & no & no-s & nefa & n-ena & nefa-r & na-isow & nena-iw/nos-iw & n-ame & nefa-m-iya\\
3s & (w)o & (w)o-s & {\O} & o-na & wi-ar & wa-isow & ona-iw/os-iw & w-ame & wama-iya\\
1p & (y)i & (y)i-s & yia & yi-ena & yi-ar & (y)i-isow & yien-iw/is-iw & yi-am & yiam-iya\\
2p & ni & ni-s & nia & ni-ena & ni-ar & ni-isow & nien-iw/nis-iw & ni-am & niam-iya\\
3p & wi & wi-s & wia & wi-ena & wi-ar & wi-isow & wien-iw/wis-iw & wi-am & wiam-iya\\
\mybottomrule
\end{tabular}



\begin{table}
\caption{Personal pronouns}
\label{tab:9}
\end{table}

Mauwake is a so-called pro-drop language, and a complete sentence can consist of a verb alone. The person of the subject is marked fully in the final verbs and partially in the medial verbs, so that besides the pragmatic clues there are also grammatical means for tracing the participants. But the pronouns are not completely optional: their use is rather strictly dictated by textual factors.

It is a fairly common feature in languages that pronouns can either modify a noun in a \textstyleAcronymallcaps{NP} or replace a full \textstyleAcronymallcaps{NP}, but cannot be the head of a \textstyleAcronymallcaps{NP} taking modifiers (e.g. Hakulinen and Karlsson 1979, Saari 1985, Roberts 1987). In Mauwake the personal pronouns usually occur without modifiers, but they \textstyleEmphasizedWords{\textsc{can}} be modified by a demonstrative, provided there is no collocational clash between the demonstrative and the personal pronoun. 

\ea%x530
\label{ex:x530}
\gll \textstyleEmphasizedVernacularWords{Ni  fain=ke}  ekap-eka! \\
      \\
\glt
\z

2s.UNM  this=CF  come-IMP.2p

`You here (or: This group of you), come!'

\ea%x531
\label{ex:x531}
\gll \textstyleEmphasizedVernacularWords{O  nain}  fan  me  ik-ua. \\
      \\
\glt
\z

3s.UNM  that1  here  not  be-PA.3s

`He is not here.'

A pronoun copy after a full \textstyleAcronymallcaps{NP} is hardly ever used in Mauwake for the subject. The rare example (\stepcounter{nx}{\thenx}) is from a hortatory text and may show rhetoric style:

\ea%x683
\label{ex:x683}
\gll Maneka  fain  [wie  \textstyleEmphasizedVernacularWords{wi}]  eliw  wiar  op-i-kuan. \\
      \\
\glt
\z

big  this  uncle  3p.UNM  well  3.DAT  grab-NpFU.3p

`These big ones the uncles may take from her.'

The example (\stepcounter{nx}{\thenx}) is not a  case of a genuine pronoun copy, since the genitive pronoun \textstyleStyleVernacularWordsItalic{wiena} adds the emphasizing meaning `themselves':

\ea%x532
\label{ex:x532}
\gll \textstyleEmphasizedVernacularWords{Wi } iperowa  \textstyleEmphasizedVernacularWords{wi-ena}  ekap-e-mik. \\
      \\
\glt
\z

3p.UNM  middle.aged  3p-GEN  come-PA-1/3p

`The middle-aged (people) themselves came.' 

For a pronoun copy of the genitive in a possessive \textstyleAcronymallcaps{NP}, see \sectref{sec:4.1.1}.

Pronouns as deictic elements are discussed in \sectref{sec:6.3.1}.

\subsubsection{Free pronouns}
\hypertarget{RefHeading19561935131865}{}
There are two sets of free pronouns:  the unmarked pronouns, and the slightly longer focal pronouns.

\paragraph[Unmarked pronouns]{Unmarked pronouns}
\hypertarget{RefHeading19581935131865}{}
The unmarked pronouns are as follows:

  singular  plural

1  yo  (y)i

2  no  ni

3  (w)o  wi

The main use of the unmarked pronouns is as subjects. In narratives only the person marking on the verb, rather than a pronoun, is used for an established, continuing subject/topic (\sectref{sec:9.2.2}). Especially third person unmarked pronouns marking the subject are quite rare in narrative texts; first person pronouns are relatively much more common (J\"arvinen 1991:79-80). 

  \stepcounter{nx}{\thenx}x533)  Irak-owa=ke  kerer-eya  \textstyleEmphasizedVernacularWords{wi } puk-omak-e-mik.\\
fight-NMZ=CF  appear-2/3p.DS  3p.UNM  disperse-DISTR.PL-PA-1/3p

`When the fight started they (many) dispersed.'

\ea%x534
\label{ex:x534}
\gll \textstyleEmphasizedVernacularWords{O}  koora=pa  naap  ik-ok  um-o-k. \\
      \\
\glt
\z

3s.UNM  house=LOC  thus  be-SS  die-PA-3s

`She was like that in the house and died.'

\ea%x1803
\label{ex:x1803}
\gll Bogia=pa  nan  wu-ap  \textstyleEmphasizedVernacularWords{i}  kiiriw  ekap-e-mik. \\
      \\
\glt
\z

Bogia=LOC  there  put-SS.SEQ  1p.UNM  again  come-PA-1/3p

`We buried his body (lit: put it) there in Bogia and came (back) again.'

However, with imperative verbs the subject pronoun is common (\sectref{sec:3.5.11}, \sectref{sec:3.8.3.3.2}, \sectref{sec:7.3}). In this Mauwake provides an interesting exception to a very strong cross-linguistic tendency of dropping subject pronouns in imperative clauses (Giv\'on 1979:80, Sadock and Zwicky 1985:173-174).  In this position the pronoun is usually unstressed, unless it is contrasted with the subject of another  clause coordinated with the imperative clause (\stepcounter{nx}{\thenx}). 

\ea%x1771
\label{ex:x1771}
\gll ``\textstyleEmphasizedVernacularWords{No}  me  baurar-e,''  naap  maak-e-k.  \\
      \\
\glt
\z

2s.UNM  not  flee-IMP.2s  thus  tell-PA-3s

` ``Don't run away,'' he told her.'

\ea%x1772
\label{ex:x1772}
\gll \textstyleEmphasizedVernacularWords{I}  or-u. \\
      \\
\glt
\z

1p.UNM  descend-IMP.1d

`Let's go down.'

\ea%x1780
\label{ex:x1780}
\gll \textstyleEmphasizedVernacularWords{No}  feeke  ik-e,  yo  Amerika  wia  \\
      \\
\glt
\z

2s.UNM  here.CF  be-IMP.2s  1s.UNM  America  3p.ACC  

akup-ikiw-i-yem.

search-go-Np-PR.1s

`You stay here, I will go searching the Americans.'

In an imperative clause the subject pronoun may also be used appositionally with a \textstyleAcronymallcaps{NP} that has vocative function, to address a person (\stepcounter{nx}{\thenx}). 

\ea%x627
\label{ex:x627}
\gll Muuka,  \textstyleEmphasizedVernacularWords{no}  aakisa  emeria  aaw-e! \\
      \\
\glt
\z

son  2s.UNM  now  woman  take-IMP.2s

`Son, take a wife now!' (i.e. It is time for you to get married.)

There are some cases where the imperative clauses tend not to have a subject pronoun. When the clause has a theme (9.1) different from the subject, and especially when the theme is another pronoun (\stepcounter{nx}{\thenx}), the imperative subject is blocked:

\ea%x1773
\label{ex:x1773}
\gll A,  ifera\textsubscript{T}H  feeke  un-eka. \\
      \\
\glt
\z

ah  salt.water  here.CF  draw/fetch-IMP.2p

`Ah, fetch the salt water (right) here.'

\ea%x1774
\label{ex:x1774}
\gll Yo\textsubscript{TH, TP}  momor  me  yook-e. \\
      \\
\glt
\z

1s.UNM  foolishly  not  follow.me-IMP.2s

`Don't be foolish and follow me.' (Lit: `Don't follow me foolishly.')

When an imperative final clause is preceded by a different-subject medial clause, it does not have a subject pronoun either:

\ea%x1775
\label{ex:x1775}
\gll Nefa  war-iwkin  naap  ma-e. \\
      \\
\glt
\z

2s.ACC  shoot-2/3p.DS  thus  say-IMP.2s

`When they shoot you, (then) say like that.'

A sentence-initial subject pronoun is quite common, when one or more same-subject medial clauses precede the imperative final clause and the scope of the imperative extends backwards over all the verbs:

\ea%x628
\label{ex:x628}
\gll \textstyleEmphasizedVernacularWords{Ni}  ikiw-ep  moma  perek-eka! \\
      \\
\glt
\z

2p.UNM  go-SS.SEQ  taro  pull.out-IMP.2p

`Go and pull out (i.e. harvest) taro!'

The only example in the text data of a subject pronoun repeated in the final clause is a case where the medial clause is subordinated with the topic marker -\textstyleStyleVernacularWordsItalic{na}: 

\ea%x1776
\label{ex:x1776}
\gll \textstyleEmphasizedVernacularWords{Ni}  uf-ep-na  \textstyleEmphasizedVernacularWords{ni}  maadara  \\
      \\
\glt
\z

2p.UNM  dance-SS.SEQ=TP  2s.UNM  forehead.ornament  

me  iirar-eka  ...

not  remove-IMP.2p

`If/when you have danced, do not remove the forehead ornaments {\dots}' 

When the level of politeness is reduced, the subject pronoun is less common. Some acceptable reasons for this are urgency (\stepcounter{nx}{\thenx}), or speech by an official that is expected to be brusque (\stepcounter{nx}{\thenx}). Example (\stepcounter{nx}{\thenx}) is from a situation where the behaviour of some men has been offensive to their wives, and when the men return home and give a blunt command, their wives react to this additional insult by repeating the command and then stating their own grievance and their revenge. 

\ea%x1777
\label{ex:x1777}
\gll Karu-eka,  ikoka  Yaapan  ir-ami  {\dots} \\
      \\
\glt
\z

run-IMP.2p  later  Japan  come-SS.SIM

`Run, later the Japanese will come and {\dots}'

\ea%x1778
\label{ex:x1778}
\gll ...amia  mua=ke  ma-e-mik,  ``Nainiw  owowa  ikiw-eka.'' \\
      \\
\glt
\z

bow  man=CF  say-PA-1/3p  again  village  go-IMP.2p

`{\dots}the policemen said, ``Go back to the village.'' '

\ea%x1779
\label{ex:x1779}
\gll Ekap-emi  wia  maak-e-mik,  ``Maa  iiw-eka.''  \\
      \\
\glt
\z

come-SS.SIM  3p.ACC  tell-PA-1/3p  food  dish.out-IMP.2p

```Maa  iiw-eka.'  Nis=ke  sira  oko  on-ami...''

food  dish.out-IMP.2p  2p.FC=CF  custom  other  do-SS.SIM

`They came and told them, ``Dish out the food.''  `` `Dish out the food!' You acted offensively (lit: did another custom) and{\dots}'' '

There is some tendency to have a pronominal form to occupy the sentence-initial theme position (\sectref{sec:9.1}), especially when the pronoun refers to the main participant of the sentence. In some cases this results in the restructuring of the sentence so that a medial clause appears in the middle of the finite clause, instead of coming before it as would be more normal. In (\stepcounter{nx}{\thenx}) and (\stepcounter{nx}{\thenx}) the medial clauses are enclosed in square brackets. 

\ea%x539
\label{ex:x539}
\gll \textstyleEmphasizedVernacularWords{Yo}  [eka  yoowa  Magidar=ke  kirip-ap  yi-eya]  \\
      \\
\glt
\z

1s.UNM  water  hot  Magidar=CF  mix-SS.SEQ  give.me-2/3s.DS  

en-e-m.

eat-PA-1s

`Magidar made tea and gave it to me, and I drank it.'

\ea%x540
\label{ex:x540}
\gll \textstyleEmphasizedVernacularWords{No}  [um-eya]  or-o-n. \\
      \\
\glt
\z

2s.UNM  die-2/3s.DS  descend-PA-2s

`After he died you went down.'

Sentence-initial unmarked pronouns are also used when they are not subjects but rather mark a pronoun with other than subject function as the theme. The first person pronoun in particular is placed in the theme position very frequently, the second person less so and the third person least of all.

\ea%x535
\label{ex:x535}
\gll \textstyleEmphasizedVernacularWords{Yo  efa}  uruf-e! \\
      \\
\glt
\z

1s.UNM  1s.ACC  look-IMP.2s

`Look at me!'

\ea%x536
\label{ex:x536}
\gll \textstyleEmphasizedVernacularWords{I}  \textstyleEmphasizedVernacularWords{yiena}  mua  opora  \textstyleEmphasizedVernacularWords{yia}  asip-owa  ekap-e-mik  nain \\
      \\
\glt
\z

1p.UNM  1p.GEN  man  talk  1p.ACC  help-NMZ  come-PA-1/3p  that

`Our men who have come to help us with the language {\dots}'

Especially in spoken language the unmarked pronouns may also be used, instead of genitive pronouns, to indicate possession. This is most commonly done with kinship terms and body parts, sometimes with other nouns\footnote{The following list covers most of them: \textstyleFootnoteBaseChar{\textit{opora}} `talk, speech', \textit{opaimika} `mouth, speech', \textit{unuma} `name', \textstyleFootnoteBaseChar{\textit{koora}} `house, home', \textit{manina} `garden', \textit{siowa} `dog' and \textstyleFootnoteBaseChar{\textit{amina}} `saucepan'.} too, referring to things closely associated with a person. This usage can be seen as a kind of widening of the range of inalienably possessed nouns beyond the kinship terms (\sectref{sec:3.2.4}) to other nouns that would be inalienably possessed in related languages or some other languages in the area. 

\ea%x537
\label{ex:x537}
\gll \textstyleEmphasizedVernacularWords{Yo}  auwa  nan  ik-ua. \\
      \\
\glt
\z

1s.UNM  1s/p.father  there  be-PA.3s

`My father is there.'

\ea%x538
\label{ex:x538}
\gll Ikoka  Yaapan=ke  \textstyleEmphasizedVernacularWords{ni } umakuna  nia  puuk-i-kuan.  \\
      \\
\glt
\z

Later  Japan=CF  2p.UNM  neck  2p.ACC  cut-Np-FU.3s

`Later the Japanese will cut your necks.'

\ea%x1804
\label{ex:x1804}
\gll Aria,  \textstyleEmphasizedVernacularWords{yo}  opora  muut  nan-e-k. \\
      \\
\glt
\z

alright  1s.UNM  talk  only  there-PA-3s

`Alright, there is my talk.'

The third person plural unmarked pronoun is used to pluralise a noun phrase (\stepcounter{nx}{\thenx}). It is also often used with a place name to refer to the inhabitants of the place collectively (\stepcounter{nx}{\thenx}).

\ea%x625
\label{ex:x625}
\gll \textstyleEmphasizedVernacularWords{Wi}  sawur  nain=ke  kuura  puuk-a-mik. \\
      \\
\glt
\z

3p.UNM  spirit  that1=CF  fly  cut-PA-3s

`Those spirits changed into flies.'

\ea%x626
\label{ex:x626}
\gll \textstyleEmphasizedVernacularWords{Wi}  Lasen=ke  kuum-e-mik. \\
      \\
\glt
\z

3p.UNM  Lasen=CF  burn-PA-1/3p

`The Lasen people burned it.' (Or: `It was the Lasen people who burned it.')

The neutral focus marker -\textstyleStyleVernacularWordsItalic{ko} attaches itself to the unmarked pronoun rather than the focal pronoun. I do not know the reason for this.\footnote{Kwan Poh San suggests as a possible reason that as the irrealis focus does not give as strong an emphasis as the contrastive focus, it also attaches itself to a less emphasized form of the pronoun (p.c.).}

\ea%x547
\label{ex:x547}
\gll Waaya  en-e-man  nain  \textstyleEmphasizedVernacularWords{yo=ko}  me  uruf-a-m. \\
      \\
\glt
\z

pig  eat-PA-2p  that1  1s.UNM=NF  not  see-PA-1s  

`I didn't (get to even) see the pig that you ate.'

The unmarked pronouns are used as the basic form for focal, genitive, reflexive-reciprocal and isolative pronouns.

\paragraph[Focal pronouns]{Focal pronouns}
\hypertarget{RefHeading19601935131865}{}
The focal pronouns are similar to the unmarked pronouns but have final \textstyleStyleVernacularWordsItalic{-s: yos, nos, (w)os, (y)is, nis, wis.} These pronouns are never used for a neutral, non-focused subject. They are used in isolation and in lists (\stepcounter{nx}{\thenx}), as well as with the topic marker -\textstyleStyleVernacularWordsItalic{na} (\stepcounter{nx}{\thenx}), the contrastive focus marker -\textstyleStyleVernacularWordsItalic{ke} (\stepcounter{nx}{\thenx}), the question marker -\textstyleStyleVernacularWordsItalic{i} (\stepcounter{nx}{\thenx}) and the adverb \textstyleStyleVernacularWordsItalic{pun} `also' (\stepcounter{nx}{\thenx}). With the limiter -\textstyleStyleVernacularWordsItalic{iw} (\stepcounter{nx}{\thenx}) the focal pronoun forms one of the two kinds of restrictive pronoun.  (See \sectref{sec:3.5.7}.)

\ea%x541
\label{ex:x541}
\gll \textstyleEmphasizedVernacularWords{Yos},  yena  emeria,  ne  Yoli  gelemuta  {\dots} \\
      \\
\glt
\z

1s.FC  1s.GEN  woman  ADD  Yoli  little

`I, my wife and little Yoli {\dots}'

\ea%x542
\label{ex:x542}
\gll \textstyleEmphasizedVernacularWords{Nos}=na? \\
      \\
\glt
\z

2s.FC=TP

`What about you?'

\ea%x543
\label{ex:x543}
\gll \textstyleEmphasizedVernacularWords{Is}=ke  me  kuum-e-mik. \\
      \\
\glt
\z

1p.FC=CF  not  burn-PA-1/3p

`\textstyleEmphasizedWords{\textsc{We}}  didn't burn it.'

\ea%x544
\label{ex:x544}
\gll \textstyleEmphasizedVernacularWords{Yos}=i? \\
      \\
\glt
\z

1sg.FC=QM

`I?' 

\ea%x546
\label{ex:x546}
\gll \textstyleEmphasizedVernacularWords{Os}  pun  opora  kuisow  naap=iw  ma-e-k. \\
      \\
\glt
\z

3s.FC  also  talk  one  thus=LIM  say-PA-3s

`\textstyleEmphasizedWords{\textsc{He}}  also said the same thing.'

\ea%x545
\label{ex:x545}
\gll Anane  \textstyleEmphasizedVernacularWords{nos=iw}  nefa  maak-i-ya. \\
      \\
\glt
\z

always  2s.FC=LIM  2s.ACC  tell-Np-PR.3s

`He always talks to you only.'

When the subject of an imperative clause is contrasted with some other possible subject, the focal pronoun with contrastive focus clitic is employed: 

\ea%x629
\label{ex:x629}
\gll \textstyleEmphasizedVernacularWords{Nos=ke}  ikiw-e! \\
      \\
\glt
\z

2s.FC=CF  go-IMP.2s

`\textstyleEmphasizedWords{\textsc{You}} go (not someone else)!'

\subsubsection{Accusative pronouns}
\hypertarget{RefHeading19621935131865}{}
The accusative pronouns may have been derived from the unmarked pronouns, but because at present there is little similarity between the singular forms of the two sets, the accusative pronouns are treated as a set of their own. Their main use is to mark the syntactic object of a clause, which is typically the semantic patient but with a few verbs may be a recipient (\sectref{sec:5.2}). The plural forms are also used for the beneficiary, as the beneficiary suffix -\textstyleStyleVernacularWordsItalic{a} in the verb (\sectref{sec:3.8.3.1}) does not distinguish between singular and plural. The accusative pronouns serve as a basis for some other pronoun forms with different functions as well. The form of the accusative pronouns is reflected very closely in the plural stems of the object cross-referencing verbs but not in the singular stems (\sectref{sec:3.8.4.2.4}). 

The accusative pronouns are:

  singular  plural

1  efa  yia

2  nefa  nia

3  {\O} (zero)  wia

Only objects that are [+human] are marked with the pronoun. As there is no other case marking in \textstyleAcronymallcaps{NP}s, except for oblique case marking like locative and instrument for [\nobreakdash-human] \textstyleAcronymallcaps{NP}s, the accusative pronouns provide some of this case marking, when the object is a [+human] \textstyleAcronymallcaps{NP}. Much of the time there is no overt pronoun, as the third person singular form is zero.\footnote{Zero pronoun for the third person singular is not exceptional cross-lingustically (Lyons 1968:278, Foley 1986:66, Giv\'on 1976:166), and in Papuan languages it is common especially for the object pronoun. All the 25 Northern Adelbert Range languages compared by Z'\citet[9,160]{Graggen1980} have zero as object pronoun or object marking on the verb for the third person singular form.} 

The position of the accusative pronouns in Mauwake is immediately preceding the verb. This is probably the main reason why Z'\citet{Graggen1971} treats them as verbal prefixes. Likewise, \citet[108]{Reesink1987} states that Usan has object prefixes, even if they have a rather loose status and can be detached from the verb. But I consider the object pronouns in Mauwake independent words, as they all have two syllables and follow the normal stress pattern of the language. They are, however, very closely bound to the verb, and it seems that a cliticization process is going on.\footnote{In J\"arvinen (1991) I discussed this question whether Mauwake pronouns are full words, clitics or affixes, at some length.}  

The accusative pronouns are used for encoding semantic patient (\stepcounter{nx}{\thenx}), or recipient (\stepcounter{nx}{\thenx}), both of which are syntactic objects (\sectref{sec:5.2}, 5.3). 

\ea%x548
\label{ex:x548}
\gll Irakowa=pa  \textstyleEmphasizedVernacularWords{wia}  war-e-mik. \\
      \\
\glt
\z

fight=LOC  3p.ACC  kill-PA-1/3p

`In the fight they killed them.'

\ea%x550
\label{ex:x550}
\gll Opora  nain  \textstyleEmphasizedVernacularWords{efa}  maak-e-k. \\
      \\
\glt
\z

talk  that  1s.ACC  tell-PA-3s

`He told me the story.' 

The plural forms of the accusative pronouns are used together with the beneficiary form in the verb to disambiguate between the persons (\stepcounter{nx}{\thenx}) (\sectref{sec:3.8.3.1}).

\ea%x549
\label{ex:x549}
\gll Aite  maa  \textstyleEmphasizedVernacularWords{yia}  p-or-om-a-k. \\
      \\
\glt
\z

mother  food  1p.ACC  Bpx-descend-BEN-BNFY2.PA-3s

`Mother brought food down for us.'

The only grammatical difference between the semantic roles of patient and beneficiary is shown in the verb, which can incorporate the benefactive suffix; and between patient and recipient there is no syntactic or morphological difference. The following hierarchy is followed: if there is a recipient not incorporated in the verb root,\footnote{Verbs like `give' and `feed' incorporate the recipient object in the verb root itself (\sectref{sec:3.8.4.2.4}).} the accusative pronoun refers to it (\stepcounter{nx}{\thenx}), if there is no recipient but a plural beneficiary, the pronoun refers to the latter (\stepcounter{nx}{\thenx}). And if there is neither recipient nor beneficiary, the accusative pronoun refers to the patient (\stepcounter{nx}{\thenx}).\footnote{Cf. a rather similar hierarchy for the distributive suffix in verbs (\sectref{sec:3.8.2.3.2})} 

Transitive verbs in Mauwake usually require an overt object, and verbs like `teach', `tell', `ask', which can take two objects, require the presence of at least the human object, whether patient (\stepcounter{nx}{\thenx}), or recipient (\stepcounter{nx}{\thenx}). In (\stepcounter{nx}{\thenx}) the pronoun \textstyleStyleVernacularWordsItalic{wia} `3p.ACC' may be definite or indefinite, hence the alternative free translations.

\ea%x552
\label{ex:x552}
\gll \textstyleEmphasizedVernacularWords{Nefa}  nokar-i-yem. \\
      \\
\glt
\z

2s.ACC  ask-Np-PR.1s

`I'm asking you.'

\ea%x551
\label{ex:x551}
\gll Inglis  \textstyleEmphasizedVernacularWords{wia}  ofakow-i-ya. \\
      \\
\glt
\z

English  3p.ACC  teach-Np-PR.3s

`(S)he teaches them English.' (Or: `(S)he teaches English.')

In rare cases the human object may be left out:

\ea%x553
\label{ex:x553}
\gll Oram  nokar-i-yem. \\
      \\
\glt
\z

just  ask-Np-PR.1s

`I'm just asking.' (Asking nobody in particular, or for no particular reason.)

Transitive verbs with [+human] objects require pronouns even when the object is mentioned as a noun or a noun phrase.

\ea%x554
\label{ex:x554}
\gll Emeria  \textstyleEmphasizedVernacularWords{wia}  amukar-e-k. \\
      \\
\glt
\z

woman  3p.ACC  scold-PA-3s

`He/she scolded the women.'

\ea%x555
\label{ex:x555}
\gll Emeria  \textstyleEmphasizedVernacularWords{nia}  amukar-e-k. \\
      \\
\glt
\z

woman  2p.ACC  scold-PA-3s

`He/she scolded (you) women.'

Since the third person singular form is zero, all the cases with [+human] object noun without overt object pronoun by default indicate the third person singular (\stepcounter{nx}{\thenx}). Because there is no number or case distinction in the nouns for the arguments of the verb, without this indication by pronouns it would often be ambiguous whether the \textstyleAcronymallcaps{NP} was subject or object, or whether the object was singular or plural. 

\ea%x556
\label{ex:x556}
\gll Emeria  amukar-e-k. \\
      \\
\glt
\z

woman  scold-PA-3s

`He scolded his wife.'

In theory, the example (\stepcounter{nx}{\thenx}) could also mean `The woman scolded him/her' but in practice it does not. For when the subject is old/established information it is usually left out rather than marked by a \textstyleAcronymallcaps{NP}, and when it is new information, it is marked by the contrastive focus marker -\textstyleStyleVernacularWordsItalic{ke}.\footnote{To have the meaning `He/she scolded a/the \textstyleEmphasizedWords{woman'}, the noun would be followed by the non-numeral quantifier \textstyleFootnoteBaseChar{\textit{oko}} `a, a certain' or the demonstrative \textstyleFootnoteBaseChar{\textit{nain}} `that'.}

It must be clearly indicated whether the speaker or addressee is included in the object (\stepcounter{nx}{\thenx}), (\stepcounter{nx}{\thenx}), (\stepcounter{nx}{\thenx}).

\ea%x557
\label{ex:x557}
\gll Mua  \textstyleEmphasizedVernacularWords{yia}  aaw-i-kuan. \\
      \\
\glt
\z

man  1p.ACC  take-Np-FU.3p

`They will take (us) men.'

\citet[52-53]{Reesink1987} mentions that Usan, another Pihom Stock language, has object prefixes, but a free pronoun can also occupy the object position in the third person singular. This is not the case in Mauwake; in (\stepcounter{nx}{\thenx}) the free pronoun \textstyleStyleVernacularWordsItalic{o}  is a re-activated topic (\sectref{sec:9.2.3}). The negative clause (\stepcounter{nx}{\thenx}) shows that the position of the free pronoun is not directly preceding the verb. The clauses (\stepcounter{nx}{\thenx}), (\stepcounter{nx}{\thenx}) have a similar structure with pronouns in non-third person marking a theme. When the pronoun is fronted as a theme (9.1), it is this unmarked pronoun that is used in the theme position. 

\ea%x1354
\label{ex:x1354}
\gll Wi  teeria  papako  \textstyleEmphasizedVernacularWords{o}  {\O}  asip-a-mik... \\
      \\
\glt
\z

3p.UNM  group  other  3s.UNM  {\O}  help-PA-1/3p

`Another group helped him{\dots}' (Or: `He was helped by another group{\dots}')

\ea%x1353
\label{ex:x1353}
\gll \textstyleEmphasizedVernacularWords{O}  me  {\O  aaw-e-mik.} \\
      \\
\glt
\z

3s.UNM  not  {\O}  take-PA-1/3p

`They did not take/choose him.' (Or: `He was not taken by them.')

\ea%x684
\label{ex:x684}
\gll \textstyleEmphasizedVernacularWords{Yo}  me  \textstyleEmphasizedVernacularWords{efa}  aaw-e-mik. \\
      \\
\glt
\z

1s.UNM  not  1s.ACC  take-PA-1/3p

`They didn't take/choose me.' (Or: `I wasn't taken by them.')

\ea%x560
\label{ex:x560}
\gll \textstyleEmphasizedVernacularWords{Yo}  \textstyleEmphasizedVernacularWords{efa}  aaw-e-mik. \\
      \\
\glt
\z

1s.UNM  1s.ACC  take-PA-1/3p

`They took/chose me.' (Or: `I was taken by them.')

There is one instance where the free third person singular pronoun does occur after the negator and immediately preceding the verb, just like accusative pronouns. This is when there is constituent negation (\sectref{sec:6.2.2}.) on the object, which then also receives clausal stress (\stepcounter{nx}{\thenx}) (\sectref{sec:2.1.3.1}, 9.2.3). Here it is the negator that moves to precede the constituent it negates. The same process is also seen in (\stepcounter{nx}{\thenx}) where the negator has moved in front of the whole object \textstyleAcronymallcaps{NP}. 

\ea%x561
\label{ex:x561}
\gll Me  \textstyleEmphasizedVernacularWords{o}  uruf-a-m. \\
      \\
\glt
\z

not  3s.UNM  see-PA-1s

`It wasn't him/her that I saw.'

\ea%x562
\label{ex:x562}
\gll Me  \textstyleEmphasizedVernacularWords{wi}  owow  mua  \textstyleEmphasizedVernacularWords{wia}  arew-a-mik{\dots} \\
      \\
\glt
\z

not  3p.UNM  village  man  3p.ACC  wait-PA-1/3p

`It wasn't the village people that we waited for{\dots}'

There are situations where it is impossible to determine whether the unmarked  third person singular pronoun is marking a topic/subject or an object fronted as a theme (\sectref{sec:9.1}). The context would be needed to disambiguate between the slightly different meanings of (\stepcounter{nx}{\thenx}), which do not come out well in the English translation. The first meaning is likely if the context mentions some other people seeing something; the second meaning is more probable elsewhere. 

\ea%x563
\label{ex:x563}
\gll \textstyleEmphasizedVernacularWords{O}  me  uruf-a-k. \\
      \\
\glt
\z

3s.UNM  not  see-PA-3s

`\textit{(S)he}  didn't see him/her/it.' (Or: `(S)he didn't see \textit{him/her}.')

There are a few verbs in Mauwake that cross-reference the patient or recipient object in the verb root (\sectref{sec:3.8.4.2.4}). These verbs do not allow a separate accusative pronoun for the function that is already expressed by the verb root (\stepcounter{nx}{\thenx}), (\stepcounter{nx}{\thenx}), but it is possible to have a separate accusative pronoun for the patient when the verb cross-references the recipient rather than the patient (\stepcounter{nx}{\thenx}).

\ea%x564
\label{ex:x564}
\gll Ipia=ke  \textstyleEmphasizedVernacularWords{yiar-eya}  ekap-e-mik. \\
      \\
\glt
\z

rain=CF  hit.us-2/3s.DS  come-PA-1/3p

`The rain hit us and we/they came.'

\ea%x1526
\label{ex:x1526}
\gll Yomar,  no  uurika  \textstyleEmphasizedVernacularWords{yook}\textstyleEmphasizedVernacularWords{-}\textstyleEmphasizedVernacularWords{ap}  urup-e. \\
      \\
\glt
\z

friend  2s.UNM  tomorrow  follow.me-SS.SEQ  ascend-IMP.2s

`Friend, follow me up tomorrow.'

\ea%x565
\label{ex:x565}
\gll Iiriw  \textstyleEmphasizedVernacularWords{nefa}  \textstyleEmphasizedVernacularWords{wi-e-mik}. \\
      \\
\glt
\z

already  2s.ACC  give.them-PA-1/3p

`We have already given you to them.'

When other verbs require both a [+human] recipient and a [+human] patient, it is encoded as a clause chain. The first verb then takes one of the arguments and the second the other.

\ea%x566
\label{ex:x566}
\gll Uuriw  \textstyleEmphasizedVernacularWords{wia}  aaw-ep  \textstyleEmphasizedVernacularWords{nia}  p-ekap-om-i-yen.  \\
      \\
\glt
\z

morning  3p.ACC  take-SS.SEQ  2p.ACC  Bpx-come-BEN-Np-FU.1p

`In the morning we will bring them (people) to you.'

\subsubsection{Genitive pronouns}
\hypertarget{RefHeading19641935131865}{}
Since possession can be expressed by means of three different kinds of personal pronouns in Mauwake, I call the function \textstyleEmphasizedWords{\textsc{possessive}} and the different grammatical forms \textstyleEmphasizedWords{\textsc{genitive}}, \textstyleEmphasizedWords{\textsc{dative}}\textstyleDefinedWords{} and \textstyleEmphasizedWords{\textsc{unmarked pronoun}}. All these forms have other functions besides possessive, as has already been shown for the unmarked pronoun. 

The genitive pronouns are derived from the unmarked pronouns by the ending \nobreakdash-\textstyleStyleVernacularWordsItalic{ena}:\footnote{This ending is probably related to the specifier -\textit{ena.}}

  singular  plural

1  y-ena  yi-ena

2  n-ena  ni-ena

3  o-na  wi-ena

The main function of the genitive pronoun is to indicate the possessor in a \textstyleAcronymallcaps{NP,} and the main strategy for expressing the possessor in a \textstyleAcronymallcaps{NP} is to use either the genitive pronoun or a possessive noun phrase. Unlike most other modifiers of the noun, the genitive pronoun precedes the head noun. This is in accord with Giv\'on's (1984:202) implicational hierarchy of conformity to basic word order, as well as Dryer's (2007a:62) statement about word order correlations. In Mauwake only the nominal and genitive modifiers and noun complements, which are also at the top of Giv\'on's (1984) hierarchy, precede the head noun in the \textstyleAcronymallcaps{NP}s; all the other modifiers follow the head noun.

The genitive pronoun is used when the possessor is coreferential with the subject,\footnote{It does not have to be used when the possessive relationship is clear from the context; see (234)} and its meaning is often close to English `own'.

\ea%x1805
\label{ex:x1805}
\gll Sawur  emeria  nain=ke  \textstyleEmphasizedVernacularWords{ona}  soma  mua  nain  ifakim-o-k. \\
      \\
\glt
\z

spirit  woman  that1=CF  3s.GEN  lover  man  that1  kill-PA-3s

`The spirit woman killed her (own) lover.'

\ea%x1806
\label{ex:x1806}
\gll Mua  me  wia  imen-ap=na  \textstyleEmphasizedVernacularWords{niena}  maa=ke  ... \\
      \\
\glt
\z

man  not  3p.ACC  find-SS.SEQ=TP  2p.GEN  thing=CF

`If you don't find the men, it's your (own) business {\dots}'

\ea%x567
\label{ex:x567}
\gll \textstyleEmphasizedVernacularWords{Niena}  unuma  maifa  feeke  siisim-eka. \\
      \\
\glt
\z

2p.GEN  name  paper  here.CF  write-IMP.2p

`Write your names on the paper here/ on this paper.'

In descriptive or equative clauses genitive pronouns can modify both the subject \textstyleAcronymallcaps{NP} (\stepcounter{nx}{\thenx}) and the non-verbal predicate \textstyleAcronymallcaps{NP} (\stepcounter{nx}{\thenx}), whereas the dative pronouns can modify neither. 

\ea%x568
\label{ex:x568}
\gll \textstyleEmphasizedVernacularWords{Yena}  koora  maneka  wenup. \\
      \\
\glt
\z

1s.GEN  house  big  very

`My house is very big.'

\ea%x569
\label{ex:x569}
\gll Mua  fain  me  \textstyleEmphasizedVernacularWords{nena}  niawi  akena=ke. \\
      \\
\glt
\z

man  this   not  2s.GEN  2s/p.father  true=CF

`This man is not your real father.'

It is possible for a genitive pronoun to co-occur with a dative pronoun to modify the same noun which is not coreferential with the subject. (See \sectref{sec:3.5.5} for a further discussion on the differences between genitive and dative possessives.)

\ea%x570
\label{ex:x570}
\gll \textstyleEmphasizedVernacularWords{Yena}  koora  \textstyleEmphasizedVernacularWords{efar}  aw-o-k. \\
      \\
\glt
\z

1s.GEN  house  1s.DAT  burn-PA-3s

`My house burned.'

Even when the possessor is expressed by a noun or \textstyleAcronymallcaps{NP}, the genitive pronoun is sometimes explicit, \textstyleParagraphContinuationChar{occurring either between the possessor and the possessed} \textstyleAcronymallcaps{NP}\textstyleParagraphContinuationChar{ (}\textstyleParagraphContinuationChar{\stepcounter{nx}{\thenx}}\textstyleParagraphContinuationChar{) or, quite frequently, preceding both (}\textstyleParagraphContinuationChar{\stepcounter{nx}{\thenx}}\textstyleParagraphContinuationChar{).} 

\ea%x573
\label{ex:x573}
\gll Om-em-ik-eya  sawur  emeria  \textstyleEmphasizedVernacularWords{ona}  wiawi  \\
      \\
\glt
\z

cry-SS.SIM-be-2/3s.DS  spirit  woman  3s.GEN  3s/p.father

onak=ke  ekap-emi  maak-e-mik{\dots}

3s/p.mother=CF  come-SS.SIM  tell-PA-1/3p

`While she was crying, the spirit woman's father and mother came and told her, {\dots}'

\ea%x574
\label{ex:x574}
\gll \textstyleEmphasizedVernacularWords{Wiena}  mia  kia  maa=iw  on-a-mik. \\
      \\
\glt
\z

3p.GEN  skin  white  thing=INST  do-PA-1/3p

`They did it with the Europeans' things.'

The reason for this addition of a pronoun may be the lack of case marking in nouns, which makes the processing of possessed \textstyleAcronymallcaps{NP}s more difficult when there are modifying nouns in the \textstyleAcronymallcaps{NP}. But it is also quite common for a possessive \textstyleAcronymallcaps{NP} to occur without a genitive pronoun. 

\ea%x575
\label{ex:x575}
\gll Mua  oko  miira  inawera=pa  uruf-ap  ma-i-mik,  {\dots} \\
      \\
\glt
\z

man  other  face  dream=LOC  see-SS.SEQ  say-Np-PR.1/3p

`When we see another man's face in a dream we say, {\dots}'

The third person singular possessive pronoun provides an exception to the rule that the personal pronouns are only used for the humans. However, the cases where \textstyleStyleVernacularWordsItalic{ona} `3s. possessive' refers to a non-human possessor are few and seem to require the connotation `own':

\ea%x1808
\label{ex:x1808}
\gll {\dots}\textstyleEmphasizedVernacularWords{ona}  pia=pa  nan  karu-emi  {\dots} \\
      \\
\glt
\z

3s.GEN  bamboo=LOC  there  run-SS.SIM  

`{\dots}it (molten copper) runs there in its pipe (lit:bamboo) and {\dots}'

In those instances where the possessed \textstyleAcronymallcaps{NP} in the predicative position lacks an overt head noun, three different strategies may be used. I have not observed any  difference in meaning. The genitive pronoun may occur by itself, without a head noun, which can either be deleted completely (\stepcounter{nx}{\thenx}) or substituted by \textstyleStyleVernacularWordsItalic{nain} `that' (\stepcounter{nx}{\thenx}), or the \textstyleAcronymallcaps{NP} can be expressed by a possessive phrase (\stepcounter{nx}{\thenx}) (4.4). In all these instances the head noun occurs earlier in the same sentence, or occasionally in the preceding sentence. 

\ea%x576
\label{ex:x576}
\gll Ikiwosa  \textstyleEmphasizedVernacularWords{yena},  wapena  \textstyleEmphasizedVernacularWords{yena}{\dots} \\
      \\
\glt
\z

head  1s.GEN,  hand  1s.GEN

`The head is mine (to eat), the hands are mine{\dots}'

\ea%x577
\label{ex:x577}
\gll Fikera  pun  \textstyleEmphasizedVernacularWords{wiena}  nain=ke. \\
      \\
\glt
\z

kunai.grass  too  3p.GEN  that1=CF

`The kunai grass is theirs, too.'

\ea%x578
\label{ex:x578}
\gll Maa  nain  \textstyleEmphasizedVernacularWords{yo/yena}  \textstyleEmphasizedVernacularWords{efarik}. \\
      \\
\glt
\z

thing  that1  1s.UNM/1s.GEN  1s.DAT

`That thing is mine.'

Like possessives in many other languages, the genitive pronoun may function as the subject of a nominalized clause (\sectref{sec:5.7}). The unmarked pronoun is used in the same position too; I have not found any difference in their use.

\ea%x571
\label{ex:x571}
\gll \textstyleEmphasizedVernacularWords{Yiena}  owow  maneka  ikiw-owa  nain  ma-i-yem. \\
      \\
\glt
\z

1p.GEN  village  big  go-NMZ  that1  say-Np-PR.1s

`I'm telling about our going to town.'

As ordinary main clause subjects the genitive pronouns are more emphatic than the unmarked pronouns.\footnote{Usan \citep[55]{Reesink1984}, Siroi \citep[20]{Wells1979} and Maia \citep[73]{Hardin2002} also use the same pronoun forms for possessive and emphatic pronouns, whereas Waskia (Ross and Paol 1978) does not.} The pronunciation reflects the emphasis too: these pronouns receive a stronger stress than the unmarked pronouns when used as a subject.

\ea%x572
\label{ex:x572}
\gll Aasa  enuma  \textstyleEmphasizedVernacularWords{yena}  me  suuw-i-yem. \\
      \\
\glt
\z

canoe  new  1s.GEN  not  push-Np-PR.1s

`I don't take a new canoe down myself.'

The following example has two identical genitive pronouns, the first one functioning as an emphatic subject pronoun and the second one as a possessive pronoun:

\ea%x686
\label{ex:x686}
\gll \textstyleEmphasizedVernacularWords{Yiena}  iisow,  \textstyleEmphasizedVernacularWords{yiena}  garanga  muutiw  aaw-ep \\
      \\
\glt
\z

1p.GEN  1p.ISOL  1p.GEN  family  only  take-SS.SEQ  

uup-ep  en-e-mik.

cook-SS.SEQ  eat-PA-1/3p

`Only our family by ourselves (lit: we ourselves we only, our family only) took it, cooked and ate it.'

A genitive pronoun is also possible as the subject of a relative clause, when the subject is emphatic:

\ea%x1809
\label{ex:x1809}
\gll Wi  teeria  papako  o  asip-a-mik, \\
      \\
\glt
\z

3p.UNM  group  other  3s.UNM  help-PA-1/3p

[\textstyleEmphasizedVernacularWords{ona}  eka  sesenar-ep  wienak-e-k  nain]\textsubscript{RC}.

3s.GEN  water  buy-SS.SEQ  feed.them-PA-3s  that1

`Another group helped him, those for whom \textit{he} had bought and given beer.'

When the limiting clitic -\textstyleStyleVernacularWordsItalic{iw} `only' is added to the genitive pronoun, the result is a restrictive pronoun (\sectref{sec:3.5.7}):

\ea%x604
\label{ex:x604}
\gll Yo  me  nia  maak-i-nen,  \textstyleEmphasizedVernacularWords{nien=iw}  ma-eka. \\
      \\
\glt
\z

1s.UNM  not  2p.ACC  tell-Np-FU.1s  2p.GEN=LIM  say-IMP.2p

`I will not tell you (what to do); discuss it on your own (among yourselves/as a group).'

\subsubsection{Dative pronouns}
\hypertarget{RefHeading19661935131865}{}
The dative case is typically associated with the semantic function of goal. The pronouns called dative in Mauwake do sometimes function as goals, but mostly they  have a locative or source function. So the term here is to be understood more as a [\textstyleEmphasizedWords{\textsc{+}}human]\textstyleEmphasizedWords{\textsc{} }\textstyleEmphasizedWords{\textsc{locative}}, which includes not only locative but goal and source as well. The dative pronouns have also grammaticalized as possessives to form possessive predicate construction (\sectref{sec:5.5.2}) and as attributive possessives to indicate that the possessor is non-coreferential with the subject. 

The dative pronouns are formed by adding -r to the accusative pronouns, with the exception of third person singular, which is identical with the plural.\footnote{The third person singular form probably used to be \textstyleFootnoteBaseChar{\textit{wo-ar}},  which is still currently used by a few people.}

  singular  plural

1  efa-r  yia-r

2  nefa-r  nia-r

3  wia-r  wia-r

The syntactic function of a dative pronoun may be clausal (a locative adverbial phrase, see \sectref{sec:4.6.1}), or \textstyleAcronymallcaps{NP}-internal (a possessive modifier, see \sectref{sec:4.1.1}). Regardless of its function, the dative pronoun is always in immediately preverbal position. 

The semantic function of a dative pronoun is related to the verb of the clause. With motion verbs it has goal function: 

\ea%x1781
\label{ex:x1781}
\gll Pok-ap  ika-iwkin  mua  \textstyleEmphasizedVernacularWords{wiar}  ekap-e-mik. \\
      \\
\glt
\z

sit-SS.SEQ  be-2/3p.DS  man  3.DAT  come-PA-1/3p

`They were sitting and (their) husbands came to them.'

\ea%x580
\label{ex:x580}
\gll Mia  kokas-owa=ke  \textstyleEmphasizedVernacularWords{wiar}  kerer-e-k. \\
      \\
\glt
\z

skin  itch-NMZ=CF  3.DAT  appear/arrive-PA-3s

`Her skin started to itch.' (Lit: `Skin itch appeared to her.)'  

With stative verbs the pronouns indicate location. (Note that the free translation needs to use a comitative expression, since English does not have a [+human] locative expression equivalent to the Mauwake dative.)  

\ea%x1782
\label{ex:x1782}
\gll Feeke  \textstyleEmphasizedVernacularWords{wiar}  ik-ok  kiiriw  mua  wiar  urup-e. \\
      \\
\glt
\z

here.CF  3.DAT  be-SS  again  man  3.DAT  ascend-IMP.2s

`Having been here with him, go (back) to your husband again.'

\ea%x1783
\label{ex:x1783}
\gll Wi  sawur  nain  ir-ami  fan  \textstyleEmphasizedVernacularWords{yiar}  pok-a-mik.\footnote{Although the most natural free translation is `{\dots}with us', comitative connotation should not be read into the Mauwake text; this is a locative.} \\
      \\
\glt
\z

3p.UNM  spirit  that1  go.east-SS.SIM  here  1p.DAT  sit-PA-1/3p

`The spirits, going eastward, sat here with us.'

With verbs that indicate receiving something (take, get, buy, etc.) the dative has the semantic function of source:

\ea%x579
\label{ex:x579}
\gll Yo  emeria  Lasen=pa  \textstyleEmphasizedVernacularWords{wiar}  aaw-e-m. \\
      \\
\glt
\z

1s.UNM  woman  Lasen=LOC  3.DAT  get/take-PA-1s

`I got (my) wife from (the) Lasen (people).'

\ea%x1784
\label{ex:x1784}
\gll Kuisow  akena  ika-eya  yos=ke  \textstyleEmphasizedVernacularWords{wiar}  sesenar-ep  aaw-e-m. \\
      \\
\glt
\z

one  very  be-2/3s.DS  1s.FC=CF  3.DAT  buy-SS.SEQ  get-PA-1s

`There was only one and (it was) I (who) bought it from them.'

\ea%x1785
\label{ex:x1785}
\gll Mua  oko=ke  waaya  nain  mik-ap  \textstyleEmphasizedVernacularWords{nefar}  aaw-i-non. \\
      \\
\glt
\z

man  other=CF  pig  that1  spear-SS.SEQ  2s.DAT  get/take-Np-FU.3s

`Another man will spear the pig and take it from you.'

The ``source'' can also be more abstract. I have observed this use only with verbs indicating hearing or speaking.

\ea%x1786
\label{ex:x1786}
\gll Naap  \textstyleEmphasizedVernacularWords{wiar}  miim-a-m. \\
      \\
\glt
\z

thus  3.DAT  hear-PA-1s

`I heard thus about him/her/them.'

A locative phrase referring to a village or village area including its inhabitants is commonly used with a dative pronoun as well, otherwise it refers to just the location rather than the inhabitants. The pronoun may be used with towns or bigger areas as well, but the bigger the location, the less probable the pronoun is. In (\stepcounter{nx}{\thenx}) the people ran away to the people in the Bogia area, whereas in (\stepcounter{nx}{\thenx}) the people of Bogia town may not have been involved in the burial at all. In (\stepcounter{nx}{\thenx}) the speaker was going to the Highlands, not in order to meet the Highlanders but  to work in a location there. 

\ea%x586
\label{ex:x586}
\gll Lasen  \textstyleEmphasizedVernacularWords{wiar}  ek-a-mik. \\
      \\
\glt
\z

Lasen  3.DAT  go.east-PA-1/3p

`We went to Lasen (village).'

\ea%x587
\label{ex:x587}
\gll Baurar-ep  Bogia  kame  \textstyleEmphasizedVernacularWords{wiar}  ikiw-e-mik. \\
      \\
\glt
\z

run.away-SS.SEQ  Bogia  area  3.DAT  go-PA-1/3p

`They ran away to the Bogia area.'

\ea%x1802
\label{ex:x1802}
\gll P-ikiw-ep  Bogia=pa  nan  wu-a-mik. \\
      \\
\glt
\z

Bpx-go-SS.SEQ  Bogia=LOC  there  put-PA-1/3p

`We/They took it (a body) and buried it in Bogia.'

\ea%x1800
\label{ex:x1800}
\gll Uuriw  iinan  aasa  aaw-ep  Epa  Dabela  urup-e-mik. \\
      \\
\glt
\z

morning  sky  canoe  take-SS.SEQ  place  cold  ascend-PA-1/3p

`In the morning we took an airplane and went up to the Highlands.'

Cross-linguistically a \textstyleEmphasizedWords{\textsc{possessive predicate}} construction, a `have' construction,  has often been derived from a locative or a goal/dative construction, plus a verb of existence \citep[50-61]{Heine1997}. In the possessive predicates in Mauwake the dative pronoun precedes the verb \textstyleStyleVernacularWordsItalic{ik}\textstyleStyleVernacularWordsItalic{-} `be'. 

\ea%x1788
\label{ex:x1788}
\gll I  sira  naap  \textbf{yiar}  ik-ua. \\
      \\
\glt
\z

1p.UNM  custom  thus  1p.DAT  be-PA.3s

`We have a custom like that.' (Lit: `A custom like that is to us.)'

The possessive predicate construction is discussed in more detail in \sectref{sec:5.5.2}. 

The same dative pronoun  has also grammaticalized as a possessive attribute in a noun phrase , but here it is the semantic function of [+\textstyleEmphasizedWords{\textsc{human}}] \textstyleEmphasizedWords{\textsc{source}} that is behind the development. Conceptually the structures `\textstyleAcronymallcaps{X} took \textstyleAcronymallcaps{Y} from me' and `\textstyleAcronymallcaps{X} took my \textstyleAcronymallcaps{Y}' are very close. In (\stepcounter{nx}{\thenx}) \textstyleStyleVernacularWordsItalic{efar} can mean either `my' or `from me'. 

\ea%x581
\label{ex:x581}
\gll Nos=ke  anane  urema  \textstyleEmphasizedVernacularWords{efar}  ikum-ar-i-n. \\
      \\
\glt
\z

2s.FC=CF  always  bandicoot  1s.DAT  illicitly-INCH-Np-PR.2s

`You always steal bandicoots from me / my bandicoots.'

That it is difficult to distinguish between the roles of possessor and source is not unusual.\footnote{Sometimes it is hard to distinguish even between a possessor and a goal. In the following sentence \textit{efar} could also mean `to my place/house', with the head noun deleted: \textit{Yo me efar ekap-e}!  [1s.UNM not 1s.DAT come-IMP.2s] `Don't come to me!'           }  \citet[133]{Heine1997} mentions that early in the grammaticalization process ``these expressions can simultaneously be interpreted with reference to either their non-possessive source or to possession.'' In the following examples the source interpretation is not possible.  The example (\stepcounter{nx}{\thenx}) describes a situation in future when the speaker will already be dead and his son is made to lose his inheritance.

\ea%x1861
\label{ex:x1861}
\gll A,  yo  aamun  nan  \textstyleEmphasizedVernacularWords{efar}  ik-ua. \\
      \\
\glt
\z

ah  1s.UNM  younger.sibling  there  1s.DAT  be-PA.3s

`Ah, there is my younger brother.'

\ea%x1862
\label{ex:x1862}
\gll Ikoka  yena  yeepa  muuka=ke  yo  muuka  \textstyleEmphasizedVernacularWords{efar} \\
      \\
\glt
\z

later  1s.GEN  elder.sibling  son=CF  1s.UNM  son  1s.DAT

iirar-ep  maak-i-non  {\dots}

remove-SS.SEQ  tell-Np-FU.3s

`Later my elder brother's son will remove/displace/drive away my son and tell him, {\dots}'

This grammaticalization probably started with the verbs denoting taking and getting, but it is only a short step from there to interpreting the dative as a possessor with other verbs as well, especially as it is likely that the dative pronoun was already earlier established in the possessive predicate structure. 

\ea%x1789
\label{ex:x1789}
\gll Owowa  \textstyleEmphasizedVernacularWords{yiar}  kuuf-owa  ekap-e-mik. \\
      \\
\glt
\z

village  1p.DAT  see-NMZ  come-PA-1/3p

`They came to see our village.'

\ea%x1791
\label{ex:x1791}
\gll Auwa  afura  \textstyleEmphasizedVernacularWords{wiar}  akim-ap=ko  uruf-e. \\
      \\
\glt
\z

1s/p.father  lime  3.DAT  try-SS.SEQ=NF  see-IMP.2s

`Try father's lime and see (what it is like).'

\ea%x1790
\label{ex:x1790}
\gll Ikiwosa  \textstyleEmphasizedVernacularWords{wiar}  pepekim-ep  kaik-a-m. \\
      \\
\glt
\z

head  3.DAT  measure-SS.SEQ  tie-PA-1s

`I measured her head and tied it (a cane).'

\ea%x1795
\label{ex:x1795}
\gll No  me  emeria  \textstyleEmphasizedVernacularWords{nefar}  maak-i-mik. \\
      \\
\glt
\z

2s.UNM  not  woman  2s.DAT  tell-Np-PR.1/3p

`We are not telling/talking to your wife.'

Although the possessive is often associated with malefactive overtones as in (\stepcounter{nx}{\thenx}) and (\stepcounter{nx}{\thenx}), this is not part of its meaning (\stepcounter{nx}{\thenx}), (\stepcounter{nx}{\thenx}). 

\ea%x1787
\label{ex:x1787}
\gll Buburia  koora  \textstyleEmphasizedVernacularWords{wiar}  aw-o-k. \\
      \\
\glt
\z

bald  house  3.DAT  burn-PA-3s  

`The bald man's house burned (on him).'

\ea%x1792
\label{ex:x1792}
\gll Irak-emi  amina  \textstyleEmphasizedVernacularWords{wiar}  fo-fook-omak-e-mik.  \\
      \\
\glt
\z

fight-SS.SIM  pot  3.DAT  RDP-split-DISTR/PL-PA-1/3p

`They\textsubscript{i} fought and split their\textsubscript{j} pots.'

But since Mauwake already had genitive pronouns to indicate possession, why did another possessive strategy develop? The answer may lie in the original source function of the dative pronoun. The referent of the participant with the source function is normally another than the referent of the clausal subject, and it is this feature of non-coreferentiality with the subject that became the distinctive feature for the new possessive. 

The dative possessive construction is particularly useful for disambiguating between the subject and the possessor, if both of them are in third person. The following two pairs of examples show this clearly. The corresponding English sentences are ambiguous, whereas the Mauwake sentences are not: 

\ea%x1797
\label{ex:x1797}
\gll Yena  eremena=ke  \textstyleEmphasizedVernacularWords{ona }   siowa  aruf-eya  kepura  ku-o-k. \\
      \\
\glt
\z

1s.GEN  nephew=CF  3s.GEN  dog  hit-2/3s.DS  leg  break-PA-3s

`My nephew\textsubscript{i} hit his\textsubscript{i} dog and its leg broke.'

\ea%x1796
\label{ex:x1796}
\gll Yena  eremena=ke  siowa  \textstyleEmphasizedVernacularWords{wiar}  aruf-eya  kepura  ku-o-k. \\
      \\
\glt
\z

1s.GEN  nephew=CF  dog  3.DAT  hit-2/3s.DS  leg  break-PA-3s

`My nephew\textsubscript{i} hit his/her\textsubscript{j} dog and its leg broke.'

\ea%x1798
\label{ex:x1798}
\gll Wis=ke  wiawi  maak-e-mik. \\
      \\
\glt
\z

3p.FC=CF  3s/p.father  tell-PA-1/3p

`(It was) they\textsubscript{i} (who) told their\textsubscript{i} father.'

\ea%x1799
\label{ex:x1799}
\gll Wis=ke  wiawi  \textstyleEmphasizedVernacularWords{wiar}  maak-e-mik. \\
      \\
\glt
\z

3p.FC=CF  3s/p.father  3.DAT  tell-PA-1/3p

`(It was) they\textsubscript{i} (who) told their\textsubscript{j} father.'

Currently the dative possessive has to be used when the possessor is non-coreferential with the subject or recipient of the clause.

\ea%x588
\label{ex:x588}
\gll Marasin  nain=ke  kema  \textstyleEmphasizedVernacularWords{wiar}  iw-a-k. \\
      \\
\glt
\z

medicine  that1=CF  liver  3.DAT  go-PA-3s

`The medicine went into his liver.'

\ea%x853
\label{ex:x853}
\gll Wiowa  nain  o  wapena=pa  \textstyleEmphasizedVernacularWords{wiar } ku-o-k. \\
      \\
\glt
\z

spear  that1  3s.UNM  hand=LOC  3.DAT  break-PA-3s

`The spear broke in his hand.'

\ea%x1794
\label{ex:x1794}
\gll Pina  ...  \textstyleEmphasizedVernacularWords{nefar}  kaken-ami  welaw-i-kuan. \\
      \\
\glt
\z

guilt  {\dots}  2s.DAT  straighten-SS.SIM  finish-Np-FU.3p

`They will straighten your(sg) {\dots} guilt and finish it.'

It follows from the non-coreferentiality restriction that a possessed \textstyleAcronymallcaps{NP} with the possessive pronoun in the dative cannot be the subject of a clause.  

In the possessor function the dative pronoun does not co-occur with the accusative pronoun in the same clause (\stepcounter{nx}{\thenx}). In the rare occasion where there would be rivalry for the position immediately preceding the verb, the accusative is chosen (\stepcounter{nx}{\thenx}) rather than the dative (\stepcounter{nx}{\thenx}). 

\ea%x584
\label{ex:x584}
\gll *Yena  muuka  erup  \textstyleEmphasizedVernacularWords{efar}  \textstyleEmphasizedVernacularWords{wia}  aaw-o-k. \\
      \\
\glt
\z

1s.GEN  son  two  1s.DAT  3p.ACC  take-PA-3s

\ea%x583
\label{ex:x583}
\gll Yena  muuka  erup  \textstyleEmphasizedVernacularWords{wia}  aaw-o-k. \\
      \\
\glt
\z

1s.GEN  son  two  3p.ACC  take-PA-3s

`He took my two sons.'

\ea%x1928
\label{ex:x1928}
\gll ?Yena  muuka  erup  \textstyleEmphasizedVernacularWords{efar}  aaw-o-k. \\
      \\
\glt
\z

1s.GEN  son  two  3.DAT  take-PA-3s

But if the dative pronoun has the semantic role of goal, it may co-occur with an accusative pronoun; in this case it precedes the accusative pronoun.

\ea%x1576
\label{ex:x1576}
\gll O  \textstyleEmphasizedVernacularWords{wiar}  \textstyleEmphasizedVernacularWords{nefa}  sesek-i-yem. \\
      \\
\glt
\z

3s.UNM  3.DAT  2s.ACC  send-Np-PR.1s

`I am sending you to him.'

The use of the genitive possessive pronoun is much less restricted. Besides being employed where the possessor is coreferential with the subject (\stepcounter{nx}{\thenx}) or recipient (\stepcounter{nx}{\thenx}), it can also be used when a possessed \textstyleAcronymallcaps{NP} is the subject or non-verbal predicate of a descriptive or equative clause (\stepcounter{nx}{\thenx}). 

\ea%x589
\label{ex:x589}
\gll Eema=ke  \textstyleEmphasizedVernacularWords{ona}  kolos  Garamin  iw-o-k. \\
      \\
\glt
\z

Eema=CF  3s.GEN  dress  Garamin  give.him/her-PA-3s

`Eema\textsubscript{i} gave her\textsubscript{i} dress to Garamin.'

\ea%x590
\label{ex:x590}
\gll Eema=ke  Garamin  \textstyleEmphasizedVernacularWords{ona}  kolos  iw-o-k. \\
      \\
\glt
\z

Eema=CF  Garamin  3s.GEN  dress  give.him/her-PA-3s

`Eema\textsubscript{i} gave Garamin\textsubscript{j} her\textsubscript{j} dress.'

\ea%x591
\label{ex:x591}
\gll \textstyleEmphasizedVernacularWords{Yena}  koora  maneka  wenup. \\
      \\
\glt
\z

1s.GEN  house  big  very

`My house is very big.'

The genitive or unmarked pronoun may co-occur together with the dative pronoun referring to the same person, thus emphasizing the possessive function of the dative (\stepcounter{nx}{\thenx}), (\stepcounter{nx}{\thenx}).

\ea%x1863
\label{ex:x1863}
\gll \textstyleEmphasizedVernacularWords{Yo}  emeria  \textstyleEmphasizedVernacularWords{efar}  uruf-a-man=i  e  wia? \\
      \\
\glt
\z

1s.UNM  woman  1s.DAT  see-PA-2p=QM  or  no

`Have you seen my wife or not?'

\ea%x593
\label{ex:x593}
\gll \textstyleEmphasizedVernacularWords{Ona}  koora=pa  \textstyleEmphasizedVernacularWords{wiar}  wu-a-mik. \\
      \\
\glt
\z

3s.GEN  house=LOC  3.DAT  put-PA-1/3p

`They put it in his (own) house.'

Example (\stepcounter{nx}{\thenx}) shows how the genitive and dative possessives, in \textstyleEmphasizedWords{\textsc{different}} person forms, can modify the same noun. The dative pronoun can here be interpreted either as a possessive `your (wives)' or as a source `(wives) from you'.

\ea%x594
\label{ex:x594}
\gll Emeria  ikoka  Yaapan  \textstyleEmphasizedVernacularWords{wiena}  \textstyleEmphasizedVernacularWords{niar}  aaw-i-kuan. \\
      \\
\glt
\z

woman  later  Japanese  3p.GEN  2p.DAT  take-Np-FU.3p

`Later the Japanese will take your wives as their own.'

In the following example, where there are several possessive \textstyleAcronymallcaps{NP}s, the two genitive pronouns both refer to the man who is identified in the preceding text. In the second clause the possessor is a modifier in the subject \textstyleAcronymallcaps{NP}, so it has to be in the genitive. The subject in the third clause is the lover's spirit, and because only one dative possessive is possible in one clause, here it is naturally assigned to the man's wife whose things were thrown around, and the man is referred to by a genitive possessive. In this case the genitive possessive also underlines the fact that one of the women was the man's own wife. The clauses are separated by brackets.

\ea%x1318
\label{ex:x1318}
\gll [Ikiw-ep-ik-eya]  [\textstyleEmphasizedVernacularWords{ona}  soma  emeria  nain  kukusa  nain=ke \\
      \\
\glt
\z

go-SS.SEQ-be-2/3s.DS  3s.GEN  lover  woman  that1  spirit  that1=CF

ekap-ep]  [\textstyleEmphasizedVernacularWords{ona}  emeria  nain  maa  \textstyleEmphasizedVernacularWords{wiar}

come-SS.SEQ  3s.GEN  woman  that1  thing  3.DAT  

wafufur-eya]  [naap  maak-e-k,]  {\dots}

throw.around-2/3s.DS  thus  tell-PA-3s

`When he\textsubscript{i} was gone, his\textsubscript{i} lover-woman's\textsubscript{j} spirit came and threw around his\textsubscript{i} (own) wife's\textsubscript{k} things, and she\textsubscript{k} told her like this, {\dots}'

Dative pronouns also have a longer form, with the suffix -\textstyleStyleVernacularWordsItalic{ik}: \textstyleStyleVernacularWordsItalic{efarik}, \textstyleStyleVernacularWordsItalic{nefarik} etc. The pronoun is a contracted form of the `have' construction, with just the root left of the verb \textstyleStyleVernacularWordsItalic{ik}- `be', which has been suffixed to the pronoun. In natural text the frequency of these pronouns is extremely low. They have to be used when the dative pronoun is clause final (\stepcounter{nx}{\thenx})-(\stepcounter{nx}{\thenx}), as the regular dative pronoun only occurs pre-verbally. The longer form is often accompanied by either the genitive pronoun (\stepcounter{nx}{\thenx}) or the unmarked pronoun (\stepcounter{nx}{\thenx}), which suggests that it is more emphasized than the simple dative.

\ea%x597
\label{ex:x597}
\gll Miiw  ara  gelemuta  nain  \textstyleEmphasizedVernacularWords{yiena  yiarik}. \\
      \\
\glt
\z

land  piece  small  that1  1p.GEN  1p.DAT

`That small piece of ground is ours.'

\ea%x598
\label{ex:x598}
\gll Wiawi=ke  amap-or-o-k=i,  weke  \textstyleEmphasizedVernacularWords{wiarik}? \\
      \\
\glt
\z

3s/p.father=CF  Bpx-descend-PA-3s=QM  3s/p.grandfather  3.DAT

`Did her father take her down to her grandfather?'

The long dative with a ``receive'' type verb in the following  example can be traced back to \textstyleStyleVernacularWordsxiiptItalic{niar ikeya} `you had it, and{\dots}':

\ea%x596
\label{ex:x596}
\gll Yo  mesa  up-owa  fain  \textstyleEmphasizedVernacularWords{ni}  \textstyleEmphasizedVernacularWords{niarik} \\
      \\
\glt
\z

1s.UNM  winged.bean  plant-NMZ  this  2p.UNM  2p.DAT

aaw-ep  isak-e-m.

get-SS.SEQ  plant-PA-1s

`I got these winged bean seeds from you and planted them.'

\subsubsection{Isolative pronouns}
\hypertarget{RefHeading19681935131865}{}
The isolative pronoun forms are based on the unmarked pronouns. The ending \nobreakdash-\textstyleStyleVernacularWordsItalic{isow}\textstyleStyleVernacularWordsItalic{,} which the numeral \textstyleStyleVernacularWordsItalic{kuisow} `one' shares with these pronouns, may be an earlier morpheme possibly meaning `alone'. The meaning of the isolative pronouns is roughly `X alone' or `by -self'. In the singular forms the vowel /o/ is replaced by /a/, since  /oi/ is not a permissible vowel sequence in Mauwake. 

singular  plural

1  ya-isow  (y)i-isow

2  na-isow  ni-isow

3  wa-isow  wi-isow

When an isolative pronoun functions as a subject, which is \textstyleEmphasizedWords{\textsc{not}} theme (\sectref{sec:9.1}), it is alone (\stepcounter{nx}{\thenx}); but more commonly it is both theme and subject, and is preceded by the unmarked pronoun also showing the case marking overtly (\stepcounter{nx}{\thenx}). 

\ea%x599
\label{ex:x599}
\gll Manina  \textstyleEmphasizedVernacularWords{waisow}  mauw-ap  neeke  wu-a-k. \\
      \\
\glt
\z

garden  3s.ISOL  work-SS.SEQ  there.CF  put-PA-3s

`He made his garden alone/by himself and left it there.'

\ea%x600
\label{ex:x600}
\gll \textstyleEmphasizedVernacularWords{No}  \textstyleEmphasizedVernacularWords{naisow}  or-op  kaul  wafur-e. \\
      \\
\glt
\z

2s.UNM  2s.ISOL  descend-SS.SEQ  hook  throw-IMP.2s

`Go down alone/by yourself and do fishing (lit: throw the hook).' 

The example (\stepcounter{nx}{\thenx}) has an accusative pronoun to show the case and an initial unmarked pronoun \textstyleStyleVernacularWordsxiiptItalic{yo} `I' to mark the object as theme.

\ea%x601
\label{ex:x601}
\gll \textstyleEmphasizedVernacularWords{Yo}  \textstyleEmphasizedVernacularWords{yaisow}  me  \textstyleEmphasizedVernacularWords{efa}  keraw-a-k. \\
      \\
\glt
\z

1s.UNM  1s.ISOL  not  1s.ACC  bite-PA-3s

`It didn't bite only me.' (Or: `It wasn't only me that it bit.')

When the isolative pronoun is preceded by the genitive/emphatic pronoun it is intensified:

\ea%x1813
\label{ex:x1813}
\gll Aakisa  mua  iperowa  nain  \textstyleEmphasizedVernacularWords{ona}  \textstyleEmphasizedVernacularWords{waisow}  soor  owowa=pa \\
      \\
\glt
\z

now  man  middle.aged  that1  3s.GEN  3s.ISOL  jungle  village=LOC

ika-i-ya.

be-Np-PR.3s

`Now that middle-aged man is staying all by himself in a jungle hamlet.'

In the plural the meaning is `\textstyleEmphasizedWords{\textsc{only}} we/you/they (as a \textstyleEmphasizedWords{\textsc{group}})'. 

\ea%x602
\label{ex:x602}
\gll Wi  feeke  ika-uk,  \textstyleEmphasizedVernacularWords{i  iisow}  ikiw-i-yen.  \\
      \\
\glt
\z

3p.UNM  here.CF  be-IMP.3p  1p.UNM  1p.ISOL   go-Np-FU.1p

`Let them stay here, only we will go.'

When the first syllable of a plural isolative pronoun is reduplicated, the pronoun refers to \textstyleEmphasizedWords{\textsc{individuals}} in the group:

\ea%x603
\label{ex:x603}
\gll \textstyleEmphasizedVernacularWords{Ii-iisow}  pok-ap  opora  siisim-ep  weeser-eya  \\
      \\
\glt
\z

RDP-1p.ISOL  sit-SS.SEQ  talk  write-SS.SEQ  finish-2/3s.DS  

unow=iya  aakun-e-mik.

many=COM  talk-PA-1/3p

`We sat and wrote separately, and then talked together.'

Although the pronouns in (\stepcounter{nx}{\thenx}) and (\stepcounter{nx}{\thenx}) sound rather similar, there is a stress difference between them. In the former, \textstyleStyleVernacularWordsItalic{iisow} gets stronger stress than \textstyleStyleVernacularWordsItalic{i}, in the latter the first syllable of the reduplicated word is stressed.

\subsubsection{Restrictive pronouns}
\hypertarget{RefHeading19701935131865}{}
The restrictive pronouns are formed by adding the limiting clitic -\textstyleStyleVernacularWordsItalic{iw} `only' either to a genitive pronoun or to a focal pronoun (\sectref{sec:3.12.6}). When it is added to a genitive pronoun it means `on one's own':

\ea%x605
\label{ex:x605}
\gll No  \textstyleEmphasizedVernacularWords{nena=iw}  ma-i-n=i? \\
      \\
\glt
\z

2s.UNM  2s.GEN=LIM  say-Np-PR.2s=QM

`Do you say it on your own?' (i.e. `Did you think of it yourself?')

\ea%x606
\label{ex:x606}
\gll O=ko  me  efa  maak-e-k,  \textstyleEmphasizedVernacularWords{yena=iw} \\
      \\
\glt
\z

3s.UNM=NF  not  1s.ACC  tell-PA-3s  1s.GEN=LIM  

amis-ar-e-m.

knowledge-INCH-PA-1s

`He/she didn't tell me, I learned it on my own.'

\ea%x607
\label{ex:x607}
\gll \textstyleEmphasizedVernacularWords{Yien=iw}  ikiw-ik-ua. \\
      \\
\glt
\z

1p.GEN=LIM  go-be-PA.3s

`Let's go on our own (as a group, or one by one).'

When the limiting clitic is added to the focal form of the free pronoun it adds the meaning of exclusiveness to the pronoun:

\ea%x608
\label{ex:x608}
\gll Anane  \textstyleEmphasizedVernacularWords{nos=iw}  nefa  maak-i-ya. \\
      \\
\glt
\z

always  2s.FC=LIM  2s.ACC  tell-Np-PR.3s

`He always talks to you only.'

\ea%x609
\label{ex:x609}
\gll Wi  anane  \textstyleEmphasizedVernacularWords{is=iw}  yiam=iya  irak-i-mik. \\
      \\
\glt
\z

3p.UNM  always  1p.FC=LIM  1p.REFL=COM  fight-Np-PR.1/3p

`They always fight with us only.'

\subsubsection{Reflexive-reciprocal pronouns}
\hypertarget{RefHeading19721935131865}{}
The reflexive-reciprocal pronouns have the unmarked pronouns as their basis, but the derivative suffix is slightly different for singular and plural.  They are as follows:

  singular  plural

1  y-ame\footnote{In the coastal dialect the singular suffix is -\textit{ama}.}  yi-am

2  n-ame  ni-am

3  w-ame  wi-am

The singular forms are used as reflexives only (\stepcounter{nx}{\thenx}), (\stepcounter{nx}{\thenx}),  the plural forms both as reflexives (\stepcounter{nx}{\thenx}), (\stepcounter{nx}{\thenx})  and as reciprocals (\stepcounter{nx}{\thenx}), (\stepcounter{nx}{\thenx}).

\ea%x610
\label{ex:x610}
\gll Naap  on-ap  \textstyleEmphasizedVernacularWords{yame}  amukar-e-m. \\
      \\
\glt
\z

thus  do-SS.SEQ  1s.REFL  scold-PA-1s

`Having done so I scolded myself (i.e. was angry at myself).'

\ea%x1864
\label{ex:x1864}
\gll Iinan  akena  ikiw-ep  \textstyleEmphasizedVernacularWords{wame}  pipilim-ep  \\
      \\
\glt
\z

on.top  very  go-SS.SEQ  3s.REFL  hide-SS.SEQ  

aakun-em-ika-i-non.

speak-SS.SIM-be-Np-FU.3s

`It (= a bird) will go very high up and hide itself and keep making its calls.'

\ea%x611
\label{ex:x611}
\gll \textstyleEmphasizedVernacularWords{Niam}  tuun-ap  teeria  erup  wu-eka. \\
      \\
\glt
\z

2p.REFL  count-SS.SEQ  group  two  put-IMP.2p

`Count yourselves and form two groups.'

\ea%x1865
\label{ex:x1865}
\gll Nainiw  sande  uura  \textstyleEmphasizedVernacularWords{yiam}  fiirim-e-mik. \\
      \\
\glt
\z

again  Sunday  night  1p.REFL  gather-PA-1/3p

`Again on Sunday night we gathered.'

\ea%x612
\label{ex:x612}
\gll \textstyleEmphasizedVernacularWords{Wiam}  fook-ap  irak-e-mik. \\
      \\
\glt
\z

3p.REFL  split-SS.SEQ  fight-PA-1/3p

`They split from each other and fought.'

\ea%x1866
\label{ex:x1866}
\gll Sarir-ap  {\dots } \textstyleEmphasizedVernacularWords{yiam}  far-i-mik. \\
      \\
\glt
\z

surround-SS.SEQ  {\dots}  1p.REFL  call-Np-PR.1/3p

`We surround (the fish) {\dots} and call each other.'

In many contexts only the reflexive or the reciprocal interpretation is natural. But a potential ambiguity in some contexts is resolved by adding a genitive pronoun to mark the reflexive (\stepcounter{nx}{\thenx}) and an unmarked or restrictive pronoun to mark the reciprocal pronoun (\stepcounter{nx}{\thenx}).

\ea%x614
\label{ex:x614}
\gll \textstyleEmphasizedVernacularWords{Niena  niam}  kookal-eka. \\
      \\
\glt
\z

2p.GEN  2p.REFL  like-IMP.2p

`Like/love yourselves.'

\ea%x613
\label{ex:x613}
\gll \textstyleEmphasizedVernacularWords{Ni/nieniw  niam}  kookal-eka. \\
      \\
\glt
\z

2p.UNM/2p.LIM  2p.REFL  like-IMP.2p

`Like/love each other.'

The reflexives are not very frequent in Mauwake, because they seem to be fairly strongly connected with [+Control]. If one hurts oneself unintentionally, the cause(r) or instrument occupies the subject position instead of the person hurt. Thus, (\stepcounter{nx}{\thenx}) is a semantically appropriate equivalent for the English clause `I cut myself with a knife':

\ea%x617
\label{ex:x617}
\gll Fura=ke  efa  puuk-a-k. \\
      \\
\glt
\z

knife=CF  1s.ACC  cut-PA-3s

`A knife cut me.'

But a reflexive pronoun is used especially in expressions involving \textstyleParagraphChari{body parts}  when one does something to oneself, and the instrument is not known or mentioned (\stepcounter{nx}{\thenx}). In corresponding expressions English often uses possessive rather than reflexive pronouns.

\ea%x618
\label{ex:x618}
\gll Merena  \textstyleEmphasizedVernacularWords{yame}  puuk-a-m. \\
      \\
\glt
\z

leg  1s.REFL  cut-PA-1s

`I cut my leg.' (Or: `I cut myself in the leg.')

The plural forms of the reflexive pronouns have another, quite different use: when they are followed by numerals, especially by `two' or `three', they function as dual/trial etc. forms for the personal pronouns. They are considered to be in the nominative case when not followed by other pronoun forms (\stepcounter{nx}{\thenx}). Other cases need to be shown by appropriate additional pronouns (\stepcounter{nx}{\thenx}).

\ea%x615
\label{ex:x615}
\gll \textstyleEmphasizedVernacularWords{Yiam  arow}  nain  miim-ap  soran-e-mik. \\
      \\
\glt
\z

1p.REFL  three  that1  hear-SS.SEQ  be.startled-PA-1/3p

`The three of us heard that and were startled.'

\ea%x616
\label{ex:x616}
\gll Amia  mua=ke  \textstyleEmphasizedVernacularWords{wiam  erup}  nain  \textstyleEmphasizedVernacularWords{wia}  nokar-e-k,  {\dots} \\
      \\
\glt
\z

bow  man=CF  3p.REFL  two  that1  3p.ACC  ask-PA-3s

`The policeman asked those two {\dots}'

\subsubsection{Comitative pronouns}
\hypertarget{RefHeading19741935131865}{}
The comitative set is a mixture as far as the basic forms are concerned. The first and second person singular forms have accusative pronouns, all the others have the reflexive pronouns as their roots. The ending is the comitative clitic -\textstyleStyleVernacularWordsItalic{iya} (\sectref{sec:3.12.1}), which can also be added to nouns and is one of several ways of expressing accompaniment in Mauwake. The first and second person singular forms have a transition consonant -\textstyleStyleVernacularWordsItalic{m}- preceding the comitative clitic.

  singular  plural

1  efa-m-iya  yiam-iya

2  nefa-m-iya  niam-iya

3  wama-iya  wiam-iya

\ea%x619
\label{ex:x619}
\gll Lasen  mua  emeria  \textstyleEmphasizedVernacularWords{wiam=iya}  me  aakun-e-mik. \\
      \\
\glt
\z

Lasen  man  woman  3p.REFL=COM  not  talk-PA-1/3p

`We didn't talk with the Lasen people.'

\ea%x620
\label{ex:x620}
\gll Liisa  Poh  San  ikos  \textstyleEmphasizedVernacularWords{yiam=iya}  soomar-emi  {\dots} \\
      \\
\glt
\z

Liisa  Poh  San  with  1p.REFL=COM  walk-SS.SIM

`Liisa and Poh San walked with us and {\dots}'

\subsubsection{Primary and secondary reference of personal pronouns}
\hypertarget{RefHeading19761935131865}{}
Typically pronouns refer to the persons the form indicates: first person singular to the speaker, second person singular to the addressee etc.  Besides this primary, or default, reference some pronouns may also have a secondary reference, if the person and/or number of the referent(s) is different from that indicated by the pronoun.

In Mauwake both the first and second person singular forms as well as the third person plural marking on verbs can be used for non-specific, or generic, reference. They occur particularly in explanations of customs or general principles, and in examples. The sentences are usually in the future tense and therefore hypothetical. In these texts the second person singular pronoun and the third person verb marking can alternate quite freely. Example (\stepcounter{nx}{\thenx}) is from a text describing the adoption process in general, and example (\stepcounter{nx}{\thenx}) was said to a person who does not even have a spirit name to call upon, nor does know how to spear pigs. Here the pronouns have acquired a non-deictic role: their correct interpretation does not depend on the non-linguistic context \citep[260]{AndersonEtAl1985}%Keenan
.

\ea%x621
\label{ex:x621}
\gll \textstyleEmphasizedVernacularWords{Yo}  muuka  kookal-ep  \textstyleEmphasizedVernacularWords{yena}  samapora  wia  \\
      \\
\glt
\z

1s.UNM  son  like-SS.SEQ  1s.GEN  clan  3p.ACC

maak-i-nen.

tell-Np-FU.1s

`When I like to have a son/child I will tell my clan.' (Or: `When \textit{one} wants a child he will tell his own clan.')

\ea%x622
\label{ex:x622}
\gll \textstyleEmphasizedVernacularWords{No}  waaya  mik-ap  inasina  unuma  me  unuf-i-nan=na \\
      \\
\glt
\z

2s.UNM  pig  spear-SS.SEQ  spirit  name  not  call-Np-FU.2s=TP

mua  oko=ke  nainiw  mik-ap  \textstyleEmphasizedVernacularWords{nefar}  aaw-i-non.

man  other=CF  again  spear-SS.SEQ  2s.DAT  take-Np-FU.3s

`If you spear a pig and don't call your spirit name, another man will spear it again and take it from you.' (Or: `If \textit{one} spears a pig{\dots}')

When a maximally generic object is needed for a transitive verb, or when there is no overt object available, the first person plural accusative form is used. 

\ea%x623
\label{ex:x623}
\gll Ifa  nain=ke  \textstyleEmphasizedVernacularWords{yia}  keraw-i-ya. \\
      \\
\glt
\z

snake  that1=CF  1p.ACC  bite-Np-PR.3s

`That snake bites.'

\ea%x624
\label{ex:x624}
\gll Marasin  fain  \textstyleEmphasizedVernacularWords{yia}  girin-i-ya. \\
      \\
\glt
\z

medicine  this   1p.ACC  smart-Np-PR.3s

`This medicine smarts.'

\subsubsection{Use of personal pronouns in text}
\hypertarget{RefHeading19781935131865}{}
In Mauwake it is possible to leave the subject pronoun out, as the person and number of the subject are marked on the verb suffix. And this is not only possible but very common: approximately only 6\% of all the clauses in narrative and descriptive texts have a pronominal subject, compared to about 30\% of the clauses having a subject \textstyleAcronymallcaps{NP} of any kind. As the other arguments are not marked on the verb, except for a two-way distinction for beneficiary (\sectref{sec:3.7.3.1}), other than subject pronouns need to be used for them if there is no full \textstyleAcronymallcaps{NP}, and they are often employed even when there is a \textstyleAcronymallcaps{NP}.

The frequency of subject pronouns depends on whether the person referred to is first, second or third, and on the type of text as well. The first person, both in singular and plural, is commonly referred to with a pronoun, instead of just a verb suffix.  Second person pronouns are very frequent in hortatory texts and are used somewhat in conversations. Most narratives in the data have their main participants in third person, but pronouns are used to refer to them quite rarely. 

A pronoun may be used for the second mention of a newly established topic (\sectref{sec:9.2.2}). In particular when an important participant has been introduced by a proper name, in the next sentence (s)he can be referred to by a personal pronoun. 

\ea%x1867
\label{ex:x1867}
\gll Eema=ke  waisow  amis-ar-e-k.  \textstyleEmphasizedVernacularWords{Os=ke}  uuriw  \\
      \\
\glt
\z

Eema=CF  3s.ISOL  knowledge-INCH-PA-3s  3s.FC=CF  morning

urup-emi{\dots}

rise-SS.SIM

`Only Eema knew. She got up in the morning and {\dots}'

When a participant has been established as the topic, (s)he is referred to with a verb suffix only, or with a \textstyleAcronymallcaps{NP} if a better identification is needed. A pronoun is used mainly when the topic is re-activated after being inactive for a while (\sectref{sec:9.2.3}). The example (\stepcounter{nx}{\thenx}) is from a text where a couple goes down to the husband's village and then returns to the wife's village. The wife's relatives, inactive as a topic for the span of five clauses, are re-assigned the topic status with the pronoun \textstyleStyleVernacularWordsItalic{wi}  `they'. 

\ea%x1922
\label{ex:x1922}
\gll Or-op  ik-ok  nainiw  urup-e-mik.  Aria  \textstyleEmphasizedVernacularWords{wi} \\
      \\
\glt
\z

descend-SS.SEQ  be-SS  again  ascend-PA-1/3p  alright  3p.UNM

samapora  maneka  fook-ap  {\dots}

floor  big  split-SS.SEQ  

`They (=the couple) went down and after a while they came up again. Alright they (=the wife's relatives) split (wood for) a big floor and {\dots}'

In commands (\sectref{sec:7.3}) the subject pronouns are more frequent than in statements.\footnote{As many as 39\% of commands in the text material have a pronoun subject, as against 6\% in statements.} The pronoun here is not a vocative; that would be separated from the rest of the clause by a pause, whereas a subject is not. The following is a fairly typical command:

\ea%x685
\label{ex:x685}
\gll Ni  ikiw-eka! \\
      \\
\glt
\z

2p  go-IMP.2p

`Go (2p)!'

This is an unusual feature cross-linguistically, as languages tend to drop the subject pronoun in imperative clauses (Giv\'on 1979:80).\footnote{The relatively high frequency of subject pronouns in imperative clauses may not be a peculiarity of Mauwake only. The grammatical descriptions of Papuan languages often state that the subject pronoun is optional in these clauses, but give no information as to their actual frequency. Personal communication with other field linguists working on Papuan languages gives reason to suggest that an overt personal pronoun with the imperative may be more common than is generally assumed.}  

\subsection{Spatial deictics}
\hypertarget{RefHeading19801935131865}{}
This section brings together what are often called demonstrative pronouns and deictic locative adverbs. What is common to them is the spatial orientation based on the location of the speaker, as well as morphological similarity. The whole deictic system, which also includes personal and temporal deixis, is discussed briefly in 6.3. 

Deictics operate on the scale of proximity, making reference to something else on the basis of location (Halliday and Hasan 1976:57-58). The relative proximity may be measured either from the speaker or from the speaker and addressee. Papuan languages manifest both these types as well as a combination of the two. Elevation and visibility may be additional parameters, so the demonstrative systems range from a simple and rather common two-term system to quite complicated ones \citep[75-77]{Foley1986}. Two-way distinctions are found in Siroi \citep[20]{Wells1979} and Golin \citep{Bunn1974}, three-way distinctions in Waskia \citep[59]{RossEtAl1978}%Paol
, Bine \citep{Saari1985} and Korafe \citep[65]{FarrEtAl1981}%Whitehead
. Usan has four basic deictics, but derivations extend the system into an elaborate one \citep[76-81]{Reesink1987}. \citet[38-39]{Murane1974} reports 19 locatives in Daga that are also used as demonstrative pronouns. 

\subsubsection{The basic spatial deixis in Mauwake}
\hypertarget{RefHeading19821935131865}{}
The main factors dividing the deictic space in Mauwake are the relative proximity to the speaker, and visibility. There are four deictic roots, one of them proximal and three distal. They are as follows: 

fa-  `here' (close to speaker, visible)  proximate

na-  `there' (away from the speaker; generic)  distal-1

eef-  `here/there'  (rather close, usually visible)  distal-2

een-  `there' (far away, usually not visible)  distal-3

The proximal deictic \textstyleStyleVernacularWordsItalic{fa}- indicates close proximity to the speaker: prototypically the referent marked with \textstyleStyleVernacularWordsItalic{fain} `this' can be touched by the speaker, and \textstyleStyleVernacularWordsItalic{fan} `here' indicates the speaker's location or close proximity to it. The distal-1 deictic \textstyleStyleVernacularWordsItalic{na}\textit{-} indicates a distance that is out of touching distance to the speaker; the distance to the addressee is irrelevant. \textstyleStyleVernacularWordsItalic{Na}- is the most neutral and the least restricted of the three distal deictics, and its frequency is extremely high because of the various functions that the demonstrative \textstyleStyleVernacularWordsItalic{nain} has. On the other hand, the words formed with both the distal\nobreakdash-2 root \textstyleStyleVernacularWordsItalic{eef}\textit{\nobreakdash-} and the distal\nobreakdash-3 root \textstyleStyleVernacularWordsItalic{een}\nobreakdash-, although available, are rarely used. They may be employed when the pragmatic situation meets the semantic specification for their occurrence, and they are needed when more than one far deictic is called for. Often the distance is a relative matter, and the speaker has a subjective choice between the different deictics.

The deictic roots suffixed with -\textstyleStyleVernacularWordsItalic{in}, marking given information, are used as demonstratives. When the roots are suffixed with -\textstyleStyleVernacularWordsItalic{an} `locative', the words function as locative adverbs. The distribution of  both these suffixes is very restricted: they are only attached to deictic or question word (\sectref{sec:3.7.1}) roots.

The deictic manner adverbs (\sectref{sec:3.6.4}) are also based on the same roots.

\subsubsection{Demonstratives}
\hypertarget{RefHeading19841935131865}{}
The four demonstratives in Mauwake are formed by one of the deictic roots plus the suffix -\textstyleStyleVernacularWordsItalic{in} indicating given information. 

In Mauwake the demonstratives are like the personal pronouns in that they can function as the sole head of a \textstyleAcronymallcaps{NP}. But they differ from the personal pronouns in that they do not have the case forms typical of the latter. In this respect the demonstratives are more like adjectives. Another feature that they share with adjectives is that they mainly function as modifiers in a \textstyleAcronymallcaps{NP}. But unlike the adjectives, which only occur alone in complement position (unless the \textstyleAcronymallcaps{NP} is elliptical), the demonstratives occur by themselves in several clause positions. 

The numeral modifiers are positioned between an adjective and a demonstrative in a \textstyleAcronymallcaps{NP} (\stepcounter{nx}{\thenx}), but never between two adjectives (\stepcounter{nx}{\thenx}).

\ea%x631
\label{ex:x631}
\gll koora  maneka  arow  \textstyleEmphasizedVernacularWords{nain} \\
      \\
\glt
\z

house  big  three  that1

`those three big houses'

\ea%x632
\label{ex:x632}
\gll siowa  sepa  gelemuta  erup \\
      \\
\glt
\z

dog  black  small  two

`two small black dogs'

There is a clear distinction in Mauwake between human and non-human reference, which shows in the choice of a pronoun vs. a demonstrative. A third person pronoun is not used for non-humans, whereas demonstratives in isolation\footnote{Demonstratives are common as \textit{modifiers} of NPs referring to humans.} are normally only used for non-humans. The only exception in my data is example (\stepcounter{nx}{\thenx}); \textstyleStyleVernacularWordsItalic{nain} `that' would not be acceptable even here.

\ea%x633
\label{ex:x633}
\gll No\footnote{\textit{No} `you' is an extra-clausal theme, not part of the subject.}  \textstyleEmphasizedVernacularWords{fain}  me  nena  niawi  akena=ke. \\
      \\
\glt
\z

2s.UNM  this  not  2s.GEN  2s/p.father  true=CF

`This is not your true father.'

Apart from the proximal demonstrative \textstyleStyleVernacularWordsItalic{fain} `this', the other demonstratives are not mutually exclusive. The distal-1 demonstrative \textstyleStyleVernacularWordsItalic{nain} `that' is the least restricted of the three, and it is extremely frequent, whereas both \textstyleStyleVernacularWordsItalic{eefin} `this/that' and \textstyleStyleVernacularWordsItalic{eenin} `that' are very rarely used. In (\stepcounter{nx}{\thenx}) the distances of the two mountains fit the specifications for \textstyleStyleVernacularWordsItalic{eefin} and \textstyleStyleVernacularWordsItalic{eenin} , and more than one distal demonstrative is needed for contrastive purposes:

\ea%x1749
\label{ex:x1749}
\gll Ema  \textstyleEmphasizedVernacularWords{eenin}  fikera=ke  aw-o-k,  aria  \textstyleEmphasizedVernacularWords{eefin} \\
      \\
\glt
\z

mountain  that3  kunai.grass=CF  burn-PA-3s,  alright  that2  

fikera=ke  me  aw-o-k.

kunai.grass=CF  not  burn-PA-3s

`The kunai grass on that mountain (far away, invisible) burned, but the grass on this/that one (somewhat closer) did not burn.'

There is no number distinction in demonstratives. When they modify a [+human] noun, plurality is shown in the person/number marking of the verb and optionally by an additional personal pronoun.

\ea%x635
\label{ex:x635}
\gll (\textstyleEmphasizedVernacularWords{Wi})  takira  \textstyleEmphasizedVernacularWords{fain=ke}  niir-e-mik. \\
      \\
\glt
\z

3p.UNM  boy  this=CF  play-PA-1/3p

`It was these boys that played.'

With [-human] nouns, a quantifier in the \textstyleAcronymallcaps{NP} may be used (\stepcounter{nx}{\thenx}), or distributive suffix on the verb (\stepcounter{nx}{\thenx}) to indicate plurality, or the number may be left unspecified (\stepcounter{nx}{\thenx}).

\ea%x636
\label{ex:x636}
\gll Mera  \textstyleEmphasizedVernacularWords{arow  nain}  aaw-e-m. \\
      \\
\glt
\z

fish  three  that1  get-PA-1s

`I caught those three fish.'

\ea%x637
\label{ex:x637}
\gll Mera  \textstyleEmphasizedVernacularWords{nain}  aaw-\textstyleEmphasizedVernacularWords{omak}-e-m. \\
      \\
\glt
\z

fish  that1  get-DISTR.PL-PA-1s

`I caught those (many) fish.'

\ea%x638
\label{ex:x638}
\gll Amina  \textstyleEmphasizedVernacularWords{fain}  p-ekap-e-mik. \\
      \\
\glt
\z

pot  this  Bpx-come-PA-1/3p

`We brought this pot / these pots.'

Besides the exophoric (text-external) deictic use described above, another common function for demonstratives cross-linguistically is endophoric, or text-internal anaphoric and cataphoric reference. The proximity in the case of demonstratives relates to the participants in the text, rather than the speech situation \citep[278]{Lyons1968}. 

Mauwake follows the typical pattern: the neutral distal demonstrative \textstyleStyleVernacularWordsItalic{nain} `that' is anaphoric: it only refers to the text preceding it, as in (\stepcounter{nx}{\thenx}), where the example sentence comes after the description of fishing with a fish trap. The proximal \textstyleStyleVernacularWordsItalic{fain} `this'is cataphoric, referring to the text following it (\stepcounter{nx}{\thenx}). The other two demonstratives, \textstyleStyleVernacularWordsItalic{eefin} and \textstyleStyleVernacularWordsItalic{eenin}, are not used for text-internal reference at all.

\ea%x639
\label{ex:x639}
\gll \textstyleEmphasizedVernacularWords{Nain}  soo  era=ke. \\
      \\
\glt
\z

that1  fish.trap  way=CF

`That is the way (to catch fish) with a fish trap.'

\ea%x640
\label{ex:x640}
\gll Mua  arow  \textstyleEmphasizedVernacularWords{fain}:  Kuuten,  Dogimaw,  aria  Olas  {\dots} \\
      \\
\glt
\z

man  three  this:  Kuten,  Dogimaw,  alright  Olas

`These three men: Kuuten, Dogimaw and Olas {\dots}'

The demonstrative \textstyleStyleVernacularWordsItalic{nain} `that' marks given/established information, and often has a similar function to a definite article (cf. Dryer 2007c:154). It has an important pragmatic function of marking topic continuity in Mauwake. A continuing [+human] topic, especially the main participant, is usually marked only by person/number inflection on the verb, whereas a minor participant or a [-human] established topic uses \textstyleAcronymallcaps{NP}s modified by \textstyleStyleVernacularWordsItalic{nain}.

Still another function for the demonstrative \textstyleStyleVernacularWordsItalic{nain} `that' is that of a nominaliser of otherwise finite verbal clauses (\sectref{sec:5.7.2}). A nominalized clause of this type may be a relative clause (\stepcounter{nx}{\thenx}) (\sectref{sec:8.3.1}), a complement clause (\stepcounter{nx}{\thenx}) (\sectref{sec:8.3.2}) or a temporal subordinate clause (\stepcounter{nx}{\thenx}) (\sectref{sec:8.3.3.1}).\footnote{All these clauses have a function that is consistent with the core meaning of `givenness' \citep{Haiman1978} or presupposition \citep{Reesink1987}.}

\ea%x687
\label{ex:x687}
\gll [Merena  ifa  keraw-a-k  \textstyleEmphasizedVernacularWords{nain}]\textsubscript{RC}  puuk-a-mik. \\
      \\
\glt
\z

leg  snake  bite-PA-3s  that1  cut-PA-1/3p

`They cut the leg that the snake had bitten.'

\ea%x689
\label{ex:x689}
\gll [Mukuna  kerer-e-k  \textstyleEmphasizedVernacularWords{nain}]\textsubscript{CC}  i  me  paayar-e-mik. \\
      \\
\glt
\z

fire  start-PA-3s  that1  1p.UNM  not  understand-PA-1/3p

`We didn't know that a fire had started.'

\ea%x688
\label{ex:x688}
\gll [Goron-ep  ora-i-ya  \textstyleEmphasizedVernacularWords{nain},]  maa  muutitik  \\
      \\
\glt
\z

fall-SS.SEQ  descend-Np-PR.3s  that1  thing  all.kinds  

iiwawun  lalat-i-ya.

altogether  sweep-Np-3s

`When it goes down, it sweeps everything with it.'

The same demonstrative is also used as a strong adversative `but' (\stepcounter{nx}{\thenx}) (\sectref{sec:8.1.3}). In that function it is placed clause-initially rather than clause-finally.

\ea%x690
\label{ex:x690}
\gll Wiawi  eliw  naak-e-k,  \textstyleEmphasizedVernacularWords{nain}  me  ikiw-o-k. \\
      \\
\glt
\z

3s/p.father  all.right  say-PA-3s  that1  not  go-PA-3s

`He said yes (lit: all right) to his father, but didn't go.'

\subsubsection{Deictic locative adverbs} 
\hypertarget{RefHeading19861935131865}{}
The undebatable locative adverbs in Mauwake are all deictic (\sectref{sec:3.9.1.1}). For each of the four deictic roots there are two corresponding locative adverbs. The first set contains the deictic root and the locative suffix -\textstyleStyleVernacularWordsItalic{an}. The homorganic vowels in the root and affix have merged into one. The second set is suffixed with the contrastive focus clitic -(\textstyleStyleVernacularWordsItalic{e})\textstyleStyleVernacularWordsItalic{ke}. When the clitic is added, the deictic adverb is in focus, but not necessarily contrastive. The morphophonological change that has taken place in the root is unusual: the vowel /a/ has assimilated with the initial /e/ of the contrastive focus clitic.  

Adv  Adv + CF

fa-an{\textgreaterfan  fa-eke{\textgreater}feeke } `here' (close to speaker, visible)  

na-an{\textgreater}nan  na-eke{\textgreater}neeke  `there' (away from the speaker; generic) 

eef-an  eef-eke  `here'  (rather close, usually visible) 

een-an  een-eke  `there' (far away, usually not visible) 

The difference in the usage between the neutral and focused member of each pair is that the first is \textstyleEmphasizedWords{\textsc{only}} used with realis-type verb forms, i.e. past (\stepcounter{nx}{\thenx}), (\stepcounter{nx}{\thenx}) and present tense (\stepcounter{nx}{\thenx}), whereas the second one is \textstyleEmphasizedWords{\textsc{mainly}} used with future (\stepcounter{nx}{\thenx}), imperative (\stepcounter{nx}{\thenx}), and counterfactual (\stepcounter{nx}{\thenx}), i.e. irrealis-type forms. Yet Mauwake does not differentiate between realis and irrealis in verbs, and a possible explanation here is that only locative adverbs that are in focus can make it into a future, imperative or counterfactual clause, whereas past or present clauses are less restrictive and use either focal or non-focal form. 

\ea%x463
\label{ex:x463}
\gll Owowa=pa  \textstyleEmphasizedVernacularWords{fan}  ik-emkun  aasa  maneka  ekap-o-k. \\
      \\
\glt
\z

village=LOC  here  be-1s/p.DS  canoe  big  come-PA-3s

`As I was here in the village the big ship came.'

\ea%x464
\label{ex:x464}
\gll Eliw  \textstyleEmphasizedVernacularWords{feeke}  soop-i-yen. \\
      \\
\glt
\z

well  here.CF  bury-Np-FU.1p

`We can bury him \textstyleEmphasizedWords{\textsc{here}}.'

\ea%x1213
\label{ex:x1213}
\gll Yo  fura  belemuta  \textstyleEmphasizedVernacularWords{eefan}  piipu-a-m. \\
      \\
\glt
\z

1s.UNM  knife  small  there2  leave-PA-1s

`I left the small knife (somewhere) here.'

\ea%x465
\label{ex:x465}
\gll Ni  koora  epa  \textstyleEmphasizedVernacularWords{eefeke}  ku-eka. \\
      \\
\glt
\z

2p.UNM  house  place  there2.CF  build-IMP.2p

`Build a/the house \textstyleEmphasizedWords{\textsc{over here}} in this place.'

\ea%x1214
\label{ex:x1214}
\gll Wi  aakisa  fain  manina  \textstyleEmphasizedVernacularWords{eenan}  on-i-mik. \\
      \\
\glt
\z

3p.UNM  now  this  garden  there3  make-Np-PR.1/3p

`Nowadays they make the garden(s) there (far away).'

\ea%x1573
\label{ex:x1573}
\gll Ni  \textstyleEmphasizedVernacularWords{eeneke}  ikiw-ep  momor  naap  niir-eka. \\
      \\
\glt
\z

2p  there3.CF  go-SS.SEQ  foolish  thus  play-IMP.2p

`Go \textstyleEmphasizedWords{\textsc{there}} (out of my sight) and play your foolish game.'

\ea%x466
\label{ex:x466}
\gll \textstyleEmphasizedVernacularWords{Neeke}  ik-ek-a-k=na  iwer(a)  ififa=ke  ifakim-ek-a-k. \\
      \\
\glt
\z

there1.CF  be-CNTF-PA-3s=TP  coconut  dry=CF  kill-CNTF-PA-3s

`If he had been \textstyleEmphasizedWords{\textsc{there}} a (falling) dry coconut would have killed him.'

\ea%x1197
\label{ex:x1197}
\gll Soo  nainiw  muf-owa  pun  naap,  aana=pa  \textstyleEmphasizedVernacularWords{neeke}  muf-i-mik. \\
      \\
\glt
\z

trap  again  pull-NMZ  too  thus  rattan=LOC  there1.CF  pull-Np-PR.1/3p

`Pulling the trap again is also like that, we/they pull it \textstyleEmphasizedWords{\textsc{there}} by the rattan.'

\ea%x1198
\label{ex:x1198}
\gll Malol=pa  \textstyleEmphasizedVernacularWords{neeke}  nainiw  suuw-urup-i-ya. \\
      \\
\glt
\z

open.sea  there1.CF  again  push-ascend-Np-PR.3s

`\textstyleEmphasizedWords{\textsc{There}} from the open sea it (= tsunami wave) again pushes up (to the coast).'

In the following examples \textstyleStyleVernacularWordsItalic{neeke} and\textstyleStyleVernacularWordsItalic{ feeke} are used with past or present tense verbs and indicate a temporary rather than permanent location, but this is probably secondary, or related, to the adverbs being focal: there is less need to focus on a permanent location than on a temporary one. Note that in these clauses it is possible to have two constituents with contrastive focus marking.

\ea%x1146
\label{ex:x1146}
\gll Miiw(a)  aasa  fa-ow(a)  mua=ke  \textstyleEmphasizedVernacularWords{neeke}  wia  aaw-o-k. \\
      \\
\glt
\z

land  canoe  drive-NMZ  man=CF  there1.CF  3p.ACC  take-PA-3s

`\textstyleEmphasizedWords{\textsc{There}} the truck driver picked them up.'

\ea%x1199
\label{ex:x1199}
\gll Or-op  \textstyleEmphasizedVernacularWords{neeke}  ika-iwkin  kokom-ar-e-k. \\
      \\
\glt
\z

descend-SS.SEQ  there1.CF  be-2/3p.DS  dark-INCH-PA-3s

`When they had gone down and were \textstyleEmphasizedWords{\textsc{there}} it became dark.'

\ea%x1200
\label{ex:x1200}
\gll Nainiw  mukuna  mamaiya  \textstyleEmphasizedVernacularWords{neeke}  ikiw-o-k. \\
      \\
\glt
\z

again  fire  close  there1.CF  go-PA-3s

`Again he went \textstyleEmphasizedWords{\textsc{there}} close to the fire.'

The following example is a comment from a man after he sees Japanese bombers in the sky:

\ea%x1572
\label{ex:x1572}
\gll Fa,  Yaapan=ke  \textstyleEmphasizedVernacularWords{feeke}  ik-e-mik! \\
      \\
\glt
\z

INTJ  Japan=CF  here.CF  be-PA-1/3p

`Damn, the Japanese are \textstyleEmphasizedWords{\textsc{here}}!'

\subsubsection{Deictic manner adverbs}
\hypertarget{RefHeading19881935131865}{}
The four deictic manner adverbs are based on the deictic roots, but their derivation is less regular than that of either the demonstratives or the deictic locatives, due to the restriction that a geminate vowel is only possible in an initial syllable. Again, the proximate and especially the distal-1 adverbs are common but the others are very infrequent.

feenap  `like this'  proximate  

naap  `like that, thus'  distal-1

eefenap  `like that (further away)'  distal-2

eenap  `like that (far away)'  distal-3

\ea%x701
\label{ex:x701}
\gll Ikiw-e-mik=na  \textstyleEmphasizedVernacularWords{feenap}  ma-em-ik-e-mik  {\dots} \\
      \\
\glt
\z

go-PA-1/3p=TP  like.this  say-SS.SIM-be-PA-1/3p

`They went and (unexpectedly) kept saying like this {\dots}'

\ea%x702
\label{ex:x702}
\gll \textstyleEmphasizedVernacularWords{Naap}  maak-iwkin  \textstyleEmphasizedVernacularWords{naap}  ik-ua. \\
      \\
\glt
\z

thus  tell-2/3.DS  thus  be-PA.3s

`They told him like that, and he was like that.'

In (\stepcounter{nx}{\thenx}) there is a long temporal distance between the hearing and the recounting of the story, which is apparently reflected in the choice of the adverbial.

\ea%x1857
\label{ex:x1857}
\gll Iiriw  auwa-ke  ma-iwkin  \textstyleEmphasizedVernacularWords{eefenap}  miim-a-m. \\
      \\
\glt
\z

earlier  1s/p.father=CF  say-2/3p.DS  thus2  hear-PA-1s

`The fathers spoke (about this) long ago and I heard it like that.'

In (\stepcounter{nx}{\thenx}) there is both some temporal and a considerable locative distance between the original time and place of the quote and that of the rest of the example: 

\ea%x1858
\label{ex:x1858}
\gll ``Mua  nain  opora=pa  wu-ami  ifakim-e,''  \textstyleEmphasizedVernacularWords{eenap} \\
      \\
\glt
\z

man  that1  talk=LOC  put-SS.SIM  kill-IMP.2s  thus3

efa  maak-e-mik.

1s.ACC  tell-PA-1/3p

` ``Accuse (lit: put to talk) that man and kill him,'' they told me like that.'

Location verbs (\sectref{sec:3.8.4.4.3}) are also based on the deictic roots, but directional verbs (\sectref{sec:3.8.4.4.5}), which also participate in the spatial deictic system in Mauwake, have different roots. 

\subsection{Question words and indefinites}
\hypertarget{RefHeading19901935131865}{}
Most of the indefinites in Mauwake are also question words, hence the treatment of both in the same subsection.

\subsubsection{Question words}
\hypertarget{RefHeading19921935131865}{}
The question words are here grouped together because of their shared semantic features and their function and position in content questions, although on the basis of their syntactic function on clause level some are pronouns, others adjectives or adverbs. 

The majority of the question words have an initial morpheme \textstyleStyleVernacularWordsItalic{ka}-, which indicates a question and is below in the derivations given the gloss `what', although it is unrelated to the question word \textstyleStyleVernacularWordsItalic{mauwa} `what'. The morphemes that make up the question words in the list below are given in parentheses when they can be reasonably clearly established.  

The question words are:

iikamin  `when?'\footnote{\textstyleFootnoteBaseChar{\textit{Ama kamin}} `sun how much' is used when time measured by clock is inquired; \textstyleFootnoteBaseChar{\textit{iikamin}} is less specific.}   ({\textless}iir-kamin  `time-how.much')

kaakew(e)  `of what place?'

kaan  `where'  ({\textless}ka-an  `what=LOC')

kaaneke  `where?'   ({\textless}ka-an-eke   `what=LOC=CF\textstyleAcronymallcaps{'})

kaanin  `which (of two)?'   ({\textless}ka-an-in   `what=LOC-\textstyleAcronymallcaps{GIVEN'})

kain  `which?'   ({\textless}ka-in   `what-\textstyleAcronymallcaps{GIVEN'})

kamin  `how many?', `how much?'

kamenap  `how?', `what {\dots} like?' ({\textlesskamin-naap `how.much-thus')}

mauwa  `what?'

moram  `why?'

naarew(e)  `who?'

kamenion  `(or) what/how?'  ({\textless}kamin-yon `how.much-perhaps')

naap-i  `like that?'

Both the words translated with\textit{} `which', \textstyleStyleVernacularWordsItalic{kain} and \textstyleStyleVernacularWordsItalic{kaanin}, have the suffix -\textstyleStyleVernacularWordsItalic{in} marking givenness. They are both morphologically and semantically related to the demonstratives \textstyleStyleVernacularWordsxiiptItalic{fain} `this' and \textstyleStyleVernacularWordsxiiptItalic{nain} `that' (\sectref{sec:3.6.2}). 

\textstyleStyleVernacularWordsItalic{Kaan} `where' is formed by the question root \textstyleStyleVernacularWordsItalic{ka}- and the same locative affix -\textstyleStyleVernacularWordsItalic{an} that is used in the deictic locative adverbs \textstyleStyleVernacularWordsItalic{fan} `here' and \textstyleStyleVernacularWordsItalic{nan} `there' (\sectref{sec:3.6.3}). The derivation with the contrastive focus marker \textstyleStyleVernacularWordsItalic{-(e)}\textstyleStyleVernacularWordsItalic{ke} is more frequently used than the non-focused form, possibly because the the other two most frequent question words, \textstyleStyleVernacularWordsItalic{mauwa} `what' and \textstyleStyleVernacularWordsItalic{naarewe} `who', so often take the contrastive focus clitic. 

\ea%x1852
\label{ex:x1852}
\gll Mua  nain  unuf-ami  ma-i-kuan,  {\textquotedbl}Mua  nain  \textstyleEmphasizedVernacularWords{kaan}  ik-ua?{\textquotedbl} \\
      \\
\glt
\z

man  that1  call-SS.SIM  say-Np-FU.3p  man  that1  where  be-PA.3s

`They call the man's name and say, ``Where is that man?'' '

\ea%x1854
\label{ex:x1854}
\gll Oo  Sarak,  no  \textstyleEmphasizedVernacularWords{kaan=eke}  ik-ok  kerer-e-n  a? \\
      \\
\glt
\z

INTJ  Sarak  2s.UNM  where=CF  be-SS  arrive-PA-2s  INTJ

`Oh Sarak, where have you been (lit: where were you and arrived)?'

\textstyleStyleVernacularWordsItalic{Kaanin} `which of two' also shares the locative morpheme \textstyleStyleVernacularWordsItalic{an}- with \textstyleStyleVernacularWordsItalic{kaan}- `where' as well as \textstyleStyleVernacularWordsItalic{fan} `here' and \textstyleStyleVernacularWordsItalic{nan} `there', although in its present meaning it is not a locative question.

There is also a morphological relationship between \textstyleStyleVernacularWordsItalic{kamenap} `how/ what{\dots}like?' and \textstyleStyleVernacularWordsItalic{kamin} `how many/much?' and the deictic adverb \textstyleStyleVernacularWordsItalic{naap} `thus', but synchronically their semantic relationship is opaque. \textstyleStyleVernacularWordsItalic{Kamenion} `or what? / how is it?' has obviously developed from \textstyleStyleVernacularWordsxiiptItalic{kamin} `how many/much?' and the modal clitic \textstyleStyleVernacularWordsItalic{\nobreakdash-yon} `perhaps' (\sectref{sec:3.9.3}), but again, the relationship is not transparent any more.

The question words, except for \textstyleStyleVernacularWordsItalic{kamenion} and \textstyleStyleVernacularWordsItalic{naap-i}, occupy the same syntactic position and clausal function as the corresponding non-interrogative element would have:

\ea%x520
\label{ex:x520}
\gll Mua  nain  \textstyleEmphasizedVernacularWords{iikamin}  ekap-o-k? \\
      \\
\glt
\z

man  that  when  come-PA-3s

`When did that/the man come?'

\ea%x647
\label{ex:x647}
\gll Mua  nain  \textstyleEmphasizedVernacularWords{unan}  ekap-o-k. \\
      \\
\glt
\z

man  that  yesterday  come-PA-3s

`That/the man came yesterday.'

\ea%x521
\label{ex:x521}
\gll Maa  \textstyleEmphasizedVernacularWords{mauwa}  en-e-n? \\
      \\
\glt
\z

thing/food  what  eat-PA-2s

`What did you eat?'

\ea%x648
\label{ex:x648}
\gll Maa  \textstyleEmphasizedVernacularWords{oposia}  en-e-m. \\
      \\
\glt
\z

thing/food  meat  eat-PA-1s

`I ate meat.'

Neither number nor case is marked on the interrogative words themselves. If either marking is required, it is done through personal pronouns, but for [+human] \textstyleAcronymallcaps{NP}s only.

\ea%x522
\label{ex:x522}
\gll Mua  \textstyleEmphasizedVernacularWords{naarew  wia}  uruf-a-n? \\
      \\
\glt
\z

man  who  3p.ACC  see-PA-2s

`Whom (pl) did you see?'

\ea%x523
\label{ex:x523}
\gll \textstyleEmphasizedVernacularWords{Naarew  wiar}  aaw-o-k? \\
      \\
\glt
\z

who  3.DAT  get-PA-3s

`Who did he get it from?'

When an interrogative word is used as a subject, the contrastive focus marker \nobreakdash-\textstyleStyleVernacularWordsItalic{ke} is added. This is natural since it is the question word that is the focal element in questions. 

\ea%x524
\label{ex:x524}
\gll \textstyleEmphasizedVernacularWords{Mauwa}\textstyleEmphasizedVernacularWords{=ke}  nefa  aruf-a-k? \\
      \\
\glt
\z

what=CF  2s.ACC  hit-PA-3s

`What hit you?'

\ea%x525
\label{ex:x525}
\gll Mua  \textstyleEmphasizedVernacularWords{kain=ke}  nomak-e-k? \\
      \\
\glt
\z

man  which=CF  win-PA-3s

`Which man won?'

\ea%x526
\label{ex:x526}
\gll Masin  \textstyleEmphasizedVernacularWords{kaanin=ke}  samor-ar-e-k? \\
      \\
\glt
\z

engine  which.of.2=CF  bad-INCH-PA-3s

`Which engine (of the two) broke?'

\textstyleStyleVernacularWordsItalic{Naarew}\textstyleStyleVernacularWordsItalic{(e)} `who?' is only used for [+human] referents. When the contrastive focus maker -\textstyleStyleVernacularWordsItalic{ke} is suffixed to the question word, the last syllable is normally deleted. \textstyleStyleVernacularWordsItalic{Mauwa} `what', on the other hand, is used almost solely for [-human] nouns. The only natural expression with \textstyleStyleVernacularWordsItalic{mauwa} referring to humans that I have encountered is of the type (\stepcounter{nx}{\thenx}). When a person's name is inquired, either \textstyleStyleVernacularWordsItalic{naarewe} (\stepcounter{nx}{\thenx}) or \textstyleStyleVernacularWordsItalic{kamenap} (\stepcounter{nx}{\thenx}) is used rather than \textstyleStyleVernacularWordsItalic{ma}\textstyleStyleVernacularWordsItalic{uwa}.

\ea%x649
\label{ex:x649}
\gll Emeria  nain  no/nena  \textstyleEmphasizedVernacularWords{mauwa=ke}? \\
      \\
\glt
\z

woman  that  1s.UNM/1s.GEN  what=CF

`What (relation) of yours is that woman?'

\ea%x1855
\label{ex:x1855}
\gll O  unuma  \textstyleEmphasizedVernacularWords{naare=ke}? \\
      \\
\glt
\z

3s.UNM  name  who=CF

`What is his/her name?'

\textstyleStyleVernacularWordsItalic{Kaanin} `which of two?' is specified for number (\stepcounter{nx}{\thenx}), but \textstyleStyleVernacularWordsItalic{kain} `which?' is not.

\ea%x691
\label{ex:x691}
\gll No  \textstyleEmphasizedVernacularWords{kain}  kookal-i-n? \\
      \\
\glt
\z

2s.UNM  which  like-Np-PR.2s

`Which one (of two or many) do you like?'

The locative question word \textstyleStyleVernacularWordsItalic{kaan}(\textstyleStyleVernacularWordsItalic{eke}) `where' is often used as a phrase by itself (\stepcounter{nx}{\thenx}), (\stepcounter{nx}{\thenx}). but it is also employed as a modifier of a locative noun phrase rather than \textstyleStyleVernacularWordsItalic{kain} or \textstyleStyleVernacularWordsItalic{kaanin}: 

\ea%x1853
\label{ex:x1853}
\gll [Epa  ara  \textstyleEmphasizedVernacularWords{kaan=eke}]\textsubscript{NP}  ikiw-e-mik? \\
      \\
\glt
\z

place  section  where=CF  go-PA-1/3p

`What/which area did they go to?'

\textstyleStyleVernacularWordsItalic{Kamenap} is a question word both for manner `how?' (\stepcounter{nx}{\thenx}) and for adjectives `what {\dots} like?'. In the latter sense it usually modifies the noun \textstyleStyleVernacularWordsItalic{sira} `custom, kind' (\stepcounter{nx}{\thenx}).

\ea%x527
\label{ex:x527}
\gll No  \textstyleEmphasizedVernacularWords{kamenap}  ik-o-n? \\
      \\
\glt
\z

2s.UNM  how  be-PA-2s

`How are/were you?'

\ea%x528
\label{ex:x528}
\gll O  koora  \textstyleEmphasizedVernacularWords{sira}  \textstyleEmphasizedVernacularWords{kamenap}  ku-a-k? \\
      \\
\glt
\z

3s.UNM  house  custom/kind  what.like  build-PA-3s

`What kind of house did he build?'

It is also used with the noun \textstyleStyleVernacularWordsItalic{unuma} `name' when the name of someone or something is inquired:

\ea%x650
\label{ex:x650}
\gll O  unuma  \textstyleEmphasizedVernacularWords{kamenap}? \\
      \\
\glt
\z

3s.UNM  name  what.like?

`What is his/her name?'

\ea%x651
\label{ex:x651}
\gll Nomokowa  fain  unuma  \textstyleEmphasizedVernacularWords{kamenap}? \\
      \\
\glt
\z

tree  this   name  what.like

`What is the name of this tree?'

In example (\stepcounter{nx}{\thenx}) \textstyleStyleVernacularWordsItalic{kamenap} is interchangeable with \textstyleStyleVernacularWordsItalic{naare}(\textstyleStyleVernacularWordsItalic{we})-\textstyleStyleVernacularWordsItalic{ke} `who', but in (\stepcounter{nx}{\thenx}) it is not interchangeable with \textstyleStyleVernacularWordsItalic{mauwa}\textstyleStyleVernacularWordsItalic{-ke} `what'.

The interrogative \textstyleStyleVernacularWordsItalic{kamenion} forms a clause by itself and only occurs after the question clitic -\textstyleStyleVernacularWordsItalic{i} and/or the connective \textstyleStyleVernacularWordsItalic{e} `or'. 

\ea%x529
\label{ex:x529}
\gll Maa  en-owa=ko  p-ekap-e-mik=i  \textstyleEmphasizedVernacularWords{kamenion}? \\
      \\
\glt
\z

thing  eat-NMZ=NF  Bpx-come-PA-1/3p=QM  or.what

`Did they bring food, or what (happened)?'

The question word \textstyleStyleVernacularWordsItalic{naap-i} `like that?' is different from the other question words. It is formed by adding the question marker -\textstyleStyleVernacularWordsItalic{i}  to the demonstrative \textstyleStyleVernacularWordsItalic{naap} `thus, like that', and it occurs by itself or sentence-finally after a statement, which often follows another question. It is mainly used in argumentation. 

\ea%x1194
\label{ex:x1194}
\gll Siiwa  arow  ikiw-eya  maa  en-owa  perek-i-mik.  \textstyleEmphasizedVernacularWords{Naap}\textstyleEmphasizedVernacularWords{=i}? \\
      \\
\glt
\z

moon  three  go-2/3s.DS  thing  eat-NMZ  harvest-Np-1/3p  thus=QM

`After three months we'll harvest the food, right?'

\ea%x1195
\label{ex:x1195}
\gll Feenap  eliw  ma-i-yen=i?  Sira  nain  eliw  marew,  \\
      \\
\glt
\z

like.this  well  say-Np-FU.1p=QM  custom  that1  good  none

\textstyleEmphasizedVernacularWords{naap}\textstyleEmphasizedVernacularWords{=i}?

thus=QM

`Should we say that that custom is not good -- is that what you are saying?'

Questions are discussed in \sectref{sec:7.2}, which has more examples as well.

\subsubsection{Indefinites}
\hypertarget{RefHeading19941935131865}{}
Indefinites are sometimes classified as pronouns, although they often are not very pronoun-like; sometimes they are grouped together with quantifiers \citep[81]{HakulinenEtAl1979}%Karlsson
. By definition they lack definiteness which is typical of other pronouns (Quirk et al. 1985:376). Also their status as \textstyleAcronymallcaps{NP} substitutes is questionable.

In Mauwake, the indefinites behave syntactically very much like quantifiers. The position of the indefinites in the \textstyleAcronymallcaps{NP} is after the adjective phrase and immediately preceding the demonstrative. They rarely co-occur with a quantifier phrase, but if they do, they follow the \textstyleAcronymallcaps{QP}.

The number of indefinites in Mauwake is very small. The last four in the list are actually question words (\sectref{sec:3.7.1}) that also function as indefinites:

oko    `a certain, (an)other'

papako  `some, other'

naarew(e)  `whoever, someone, one'

mauwa  `whatever, something'

kain  `whichever'

kaanin  `whichever (of two)'

\ea%x641
\label{ex:x641}
\gll Iiriw  muuka  \textstyleEmphasizedVernacularWords{oko}  wiawi  onak  urera  \\
      \\
\glt
\z

long.ago  boy  other  3s/p.father  3s/p.mother  evening  

maa  uup-e-mik.

food  cook-PA-1/3p

`Long ago, a certain boy's father and mother cooked food.'

\ea%x642
\label{ex:x642}
\gll Ne  wia,  \textstyleEmphasizedVernacularWords{papako=ke}  ma-e-mik,  {\dots} \\
      \\
\glt
\z

ADD  no,  some/other=CF  say-PA-1/3p

`But no, some/others said, {\dots}'

The indefinite \textstyleStyleVernacularWordsItalic{oko} `a certain, (an)other' also has the meaning `otherwise' when it introduces an apprehensive clause (8.1.6).

\ea%x741
\label{ex:x741}
\gll Gurun-owa  epasia=pa  miim-am-ika-i-kuan,  \textstyleEmphasizedVernacularWords{oko}  mua  \\
      \\
\glt
\z

rumble-NMZ  far=LOC  hear-SS.SIM-be-Np-FU.3p  other  man

papako  maa  ik-em-ik-owa  nain  kawus  wiar

some  thing/food  roast-SS.SIM-be-NMZ  that1  smoke  3.DAT

uruf-i-kuan.

see-Np-FU.3p

`They (villagers) keep listening to the rumble from far away, otherwise/lest they (pilots) see the smoke from some men's/people's food-roasting fire.'

Those question words (\sectref{sec:3.7.1}) that may function as indefinites behave similarly to question words as \textstyleAcronymallcaps{NP} constituents, but on the sentence level there are differences between them. The interrogatives occur either in a simple interrogative sentence or occasionally in a medial clause (\stepcounter{nx}{\thenx}). The indefinites can occur in a medial clause (\stepcounter{nx}{\thenx}), but they are more common in subordinate clauses, especially relative clauses (\stepcounter{nx}{\thenx}). 

\ea%x643
\label{ex:x643}
\gll \textstyleEmphasizedVernacularWords{Naarew}  wia  far-ep  ekap-o-n? \\
      \\
\glt
\z

who  3p.ACC  call-SS.SEQ  come-PA-2s

`Who did you call, and then came?'\footnote{A more natural translation into English would be `Who did you call before you came?', but it would hide the fact that medial clauses are coordinate.}

\ea%x644
\label{ex:x644}
\gll Masin  \textstyleEmphasizedVernacularWords{kaanin=ke}  samor-ar-eya  oko  fain=ke  \\
      \\
\glt
\z

engine  which.of.2=CF  bad-INCH-2/3s.DS  other  this=CF  

asip-i-non.

help-Np-FU.3s

`Whichever engine breaks down, this other one will help/substitute.'\footnote{With question intonation it would mean: `Which engine\textsubscript{i} will this other one\textsubscript{j} help, if it\textsubscript{i} breaks down?'}

\ea%x645
\label{ex:x645}
\gll Prais  aaw-ep  [\textstyleEmphasizedVernacularWords{uf-owa}  \textstyleEmphasizedVernacularWords{kain=ke}  nomak-e-k  nain]\textsubscript{RC}   \\
      \\
\glt
\z

prize  take-SS.SEQ  dance-NMZ  which=CF  win-PA-3s  that1  

wi-e-mik.

give.them-PA-1/3p

`They took the prize and, whichever dance won, they gave it (the prize) to them (the dancers).'

The indefinite \textstyleStyleVernacularWordsItalic{mauwa} `what' is also used as a generic substitute for any [\nobreakdash-human] \textstyleAcronymallcaps{NP} that is left unmentioned because the name of the particular thing is not known or is temporarily forgotten, like \textstyleForeignWords{whatchamacallit} in English.

\ea%x646
\label{ex:x646}
\gll Mua  nain  \textstyleEmphasizedVernacularWords{mauwa}  nain  akim-a-k=na  weetak,  \textstyleEmphasizedVernacularWords{mauwa } nain  \\
      \\
\glt
\z

man  that1  what  that1  try-PA-3s=TP  no,  what  that1  

me  or-o-k.

not  descend-PA-3s

`The man tried the thing (press button), but the thing (lift) didn't go down.'

The locative question word \textstyleStyleVernacularWordsxiiptItalic{kaaneke} is also used as an indefinite locative adverb:

\ea%x1869
\label{ex:x1869}
\gll No  \textstyleEmphasizedVernacularWords{kaaneke}  ikiw-i-nan=na,  yos  pun  nook-i-nen. \\
      \\
\glt
\z

2s.UNM  where.CF  go-Np-FU.2s=TP  1s.FC  too  follow.you-Np-FU.1s

`Wherever you go, I will follow you.'

\subsection{Verbs}
\hypertarget{RefHeading19961935131865}{}
\subsubsection{General discussion}
\hypertarget{RefHeading19981935131865}{}
\paragraph[Definition]{Definition}
\hypertarget{RefHeading20001935131865}{}
The verb category can be defined morphologically, syntactically, semantically and pragmatically. Of these, the first criterion is the most critical in Mauwake and covers the whole class; the others are less definitive, but help define a \textstyleEmphasizedWords{\textsc{prototypical}} verb.

According to the \textstyleEmphasizedWords{\textsc{morphological}}, or structural, criterion, a verb is a word that can be inflected for tense as well as the person and number of the subject. The derivational suffix categories of verbaliser, distributive and benefactive are not as useful in defining the class of verbs, as these can be used in the nominalized forms of verbs as well. \citet[190]{Anderson1985b} also adds aspect and mood into inherent verbal inflections, but in Mauwake aspect is coded syntactically (see verbal groups in \sectref{sec:3.8.5.1}), and modal categories either morphologically, syntactically or lexically.

\textstyleEmphasizedWords{\textsc{Syntactically}} a verb functions as the nucleus of a predication independently or as part of a verbal cluster (\sectref{sec:3.8.5}). Since single verbs and verbal clusters have such similar functions, the latter are described in the morphology chapter immediately after the verbs, and not in the chapter on phrase. Also, the term \textstyleEmphasizedWords{\textsc{verb}} is often used below as a generic term to cover both a single verb and a verbal cluster, unless specifically the verbal cluster is meant. The verb is the last element in a pragmatically neutral clause.

The verbal predicate is the only obligatory element in an intransitive clause. A transitive clause does require an object, but even it can often consist of a verb only, as the third person singular accusative  pronoun, used for object, is zero (\stepcounter{nx}{\thenx}). The directional verbs (\sectref{sec:3.8.4.4.5}) often co-occur with a goal, but when it is left implied the verb can be the only element (\stepcounter{nx}{\thenx}). In a verbless clause the predicate is a noun, adjective, possessive pronoun or adverb. 

\ea%x177
\label{ex:x177}
\gll \textstyleEmphasizedVernacularWords{Aaw-e-m.} \\
      \\
\glt
\z

get-PA-1s

`I got it.'

\ea%x178
\label{ex:x178}
\gll \textstyleEmphasizedVernacularWords{Urup-e-mik}.  \\
      \\
\glt
\z

go/come.up-PA-1/3p 

`We went/came up.' 

The predicate verb selects the arguments in a predication. This argument selection can be used as an important basis for the division into different verb classes (\sectref{sec:3.8.4}).

\textstyleEmphasizedWords{\textsc{Semantically}}, according to \textstyleBibliogBaseChar{Giv\'on} (1984:64), a prototypical verb encodes ``\textstyleBibliogCitationAAAstyleChar{less time-stable experiences, primarily transitory states, events and actions}''. In Mauwake this lack of time-stability feature shows in the strong tendency to use inchoative verbs (\stepcounter{nx}{\thenx}) (\sectref{sec:3.8.2.2.2}) instead of adjectives to describe non-permanent states. 

\ea%x179
\label{ex:x179}
\gll \textstyleEmphasizedVernacularWords{supuk-ar-e}\textstyleEmphasizedVernacularWords{-k}   vs.  \textstyleEmphasizedVernacularWords{supuka}  `wet' \\
      \\
\glt
\z

wet-INCH-PA-3s  

`(it) is wet' (lit: `has become wet') 

But it is also possible to express less prototypical, time-stable states and events with verbs. In \textstyleBibliogBaseChar{Frawley}'s (1992:66) words\textstyleBibliogCitationAAAstyleChar{,} ``\textstyleBibliogCitationAAAstyleChar{verbs {\dots} require temporal fixing''}, when compared with the ``\textstyleBibliogCitationAAAstyleChar{relative atemporality}'' of an entity. So the \textstyleEmphasizedWords{\textsc{relative temporality}} is the main defining factor for verbs, regardless of the time-stability.

\textstyleBibliogBaseChar{Hopper and Thompson} (1984:726) add a \textstyleEmphasizedWords{\textsc{discourse}} perspective to the definition of verbs by suggesting that ``\textstyleBibliogCitationAAAstyleChar{verbs which do not report discourse events fail to show the range of oppositions characteristic of those which do}'', and are therefore less prototypical. According to them, categoriality is only weakly associated with the root forms, and the discourse use determines how clearly the verbhood manifests itself (ibid. 747). Theirs is an important viewpoint for the study of language in general and of those languages in particular that have plenty of root forms that can be used for different word classes. But for Mauwake I assume the existence of rather discrete categories of noun and verb, which the root forms belong to, rather than just having \textstyleBibliogCitationAAAstyleChar{``a propensity or predisposition to become} \textstyleAcronymallcaps{\textup{N}}\textstyleBibliogCitationAAAstyleChar{'s or} \textstyleAcronymallcaps{\textup{V}}\textstyleBibliogCitationAAAstyleChar{'s''} (ibid. 747). The number of roots that can be used across categories without special derivational suffixes is small.

\paragraph[General characteristics of verbs in Mauwake]{General characteristics of verbs in Mauwake}
\hypertarget{RefHeading20021935131865}{}
Mauwake is a strongly verb-oriented language, and often a verb is the only element in the clause. In running text, there are roughly three words per clause, so approximately one word in three is a verb, as most of the clauses are verbal clauses.

The verb morphology is agglutinative; this shows mainly in the structure of the verbs. Suffixing is the basic strategy, but a few prefixes are used as well. Reduplication is of the prefixing type, with few exceptions. 

Although the verb morphology in Mauwake is quite extensive, for a Papuan language it is not very complex, and the patterns are quite transparent.  The verb morphology marks features of the event itself: tense, mood, sequentiality vs. simultaneity of actions, but also features related to the participants in the clause: subject, beneficiary, and distributive indicating the number of \textstyleAcronymallcaps{S}, \textstyleAcronymallcaps{O} or \textstyleAcronymallcaps{REC}. Aspect is expressed through verbal groups (\sectref{sec:3.8.5.1.1}).

To enlarge its verb inventory, Mauwake uses serial verbs (\sectref{sec:3.8.5.1.2}) or adjunct\footnote{Adjunct is here used in the sense of ``a secondary element in a construction [,~which] may be removed without the structural identity of the rest of the construction being affected'' \citep[9]{Crystal1997}.} plus verb constructions (\sectref{sec:3.8.5.2}). The serial verbs are mostly formed by a productive process, whereas the adjunct plus verb constructions tend to be lexicalized forms. 

Some verbs have roots that are very similar to nouns. Especially in Austronesian languages the question arises whether these roots are originally nouns, verbs, or unspecified as to the grammatical category (\textstyleBibliogBaseChar{Bugenhagen 1995}:162-5). This question for Mauwake is discussed in the section on verb derivation (\sectref{sec:3.8.2}).

Mauwake has no passive voice.  The subject demotion strategy is described in \sectref{sec:3.8.4.3.3}. 

There is a distinction in Mauwake between medial (\sectref{sec:3.8.3.5}) and final verbs (\sectref{sec:3.8.3.4}).\footnote{Sometimes they are also called dependent and independent verbs (e.g. \textstyleBibliogBaseChar{Foley 1986}:11).} This distinction is very important on both sentence and discourse levels. 

The verbs can be divided into two conjugation classes based on the past tense suffix vowel. Semantically these classes are arbitrary; the division is made on the basis of morphophonology and is discussed in \sectref{sec:2.3.3.3}.  But the classification done according to transitivity (\sectref{sec:3.8.4.2}) and that based on semantic characteristics (\sectref{sec:3.8.4.4}) are more interesting grammatically and reveal more of the nature of the language. 

\paragraph[Verb structure]{Verb structure}
\hypertarget{RefHeading20041935131865}{}
A verb consists of a root optionally preceded by a derivational prefix and followed by various derivational and inflectional suffixes, as shown in the diagram below (\figref{fig:1}). Only tense and person/number suffixes are obligatory in a finite verb in the \textstyleEmphasizedWords{\textsc{indicative}} mood. The obligatory elements are bolded in the diagrams.

{\bfseries
Deriv.        Derivation              Inflection}

[Warning: Draw object ignored][Warning: Draw object ignored][Warning: Draw object ignored]

Prefix -- \textbf{ROOT} -- INCH -- CAUS -- DISTR -- BEN -- BNFY -- CNTF -- \textbf{TNS -- PRS/NUM}

[Warning: Draw object ignored]

{\bfseries
S  T  E  M}


\begin{figure}
\caption{Verb derivation and finite inflection (indicative)}
\label{fig:1}
\end{figure}

\ea%x180
\label{ex:x180}
\gll Soomia  wia  \textstyleEmphasizedVernacularWords{amap-ep-om-i-ya.} \\
      \\
\glt
\z

spoon  3p.ACC  Bpx-go-BEN-Np-PR.3s

`He takes spoons to them.'

\ea%x181
\label{ex:x181}
\gll Iwera  pun  wiar  \textstyleEmphasizedVernacularWords{aw-omak-e-k}. \\
      \\
\glt
\z

coconut  too  3.DAT  burn-DISTR/PL-PA-3s

`Many of his coconut palms burned too.'

\ea%x182
\label{ex:x182}
\gll Lawiliw  akena  \textstyleEmphasizedVernacularWords{um-ek-a-m.} \\
      \\
\glt
\z

nearly  very  die-CNTF-PA-1s  

`I very nearly died.'

The \textstyleEmphasizedWords{\textsc{imperative}} verb structure is understandably different in that it cannot take counterfactual, tense or indicative person/number suffixes. Instead, an imperative person/number suffix needs to be attached as the final suffix of the verb.

Prefix -- \textbf{ROOT} -- \textstyleAcronymallcaps{INCH} -- \textstyleAcronymallcaps{CAUS} -- \textstyleAcronymallcaps{DISTR} -- \textstyleAcronymallcaps{BEN} -- \textstyleAcronymallcaps{BNFY} -- \textbf{IMP.PRS/NUM}

\ea%x183
\label{ex:x183}
\gll Ni  \textstyleEmphasizedVernacularWords{ekap-omak-eka.} \\
      \\
\glt
\z

2p.UNM  come-DISTR/PL-IMP.2p

`Come!'  (said to several people together) 

\ea%x184
\label{ex:x184}
\gll Muuka  \textstyleEmphasizedVernacularWords{arim-ow-e.} \\
      \\
\glt
\z

son  grow-CAUS-IMP.2s  

`Bring up the boy.' 

\textstyleEmphasizedWords{\textsc{Medial}} verbs likewise can have only the medial suffix after the derivational suffixes, if there are any. The medial suffix distinguishes between sequentiality (\stepcounter{nx}{\thenx}) and simultaneity (\stepcounter{nx}{\thenx}) of the actions when the subject stays the same; with a different subject (\stepcounter{nx}{\thenx}) the actions are understood to be sequential, and simultaneity needs to be marked through continuous aspect form (\sectref{sec:3.8.5.1.1.2}).

{\bfseries
[Warning: Draw object ignored]                                    SEQ}

{\bfseries
                              SS}

{\bfseries
[Warning: Draw object ignored][Warning: Draw object ignored]                                   } 

[Warning: Draw object ignored]Prefix - \textbf{Root} - INCH - CAUS - DISTR - BEN - BNFY -             \textbf{SIM}

                                \textbf{DS}


\begin{figure}
\caption{Medial verb inflection}
\label{fig:2}
\end{figure}

\ea%x185
\label{ex:x185}
\gll Oposia  \textstyleEmphasizedVernacularWords{pu-puuk-ap}  uup-e-mik.  \\
      \\
\glt
\z

meat  RDP-cut-SS.SEQ  cook-PA-1/3p  

`They cut the meat in many pieces and cooked it.'

\ea%x186
\label{ex:x186}
\gll Ewar=ke  \textstyleEmphasizedVernacularWords{wuun-ow-ami}  epia  faker-a-k,  mukuna.  \\
      \\
\glt
\z

wind=CF  blow-CAUS-SS.SIM  firewood  raise-PA-3s  fire  

`The wind blew and raised the fire(wood),  the fire.' 

\ea%x187
\label{ex:x187}
\gll \textstyleEmphasizedVernacularWords{Kees-om-a-ya}  en-ek. \\
      \\
\glt
\z

spit-BEN-BNFY2-2/3s.DS  eat-PA-3s  

`He spat/regurgitated it for her and she ate.'

\subsubsection{Verb derivatives}
\hypertarget{RefHeading20061935131865}{}
This section deals with derivational processes in which the end result is always a verb. Verbs can be derived from other word classes through two category-\textstyleEmphasizedWords{\textsc{changing}} strategies. In category-\textstyleEmphasizedWords{\textsc{maintaining} }derivations affixes are added to the verb root to change the semantics of the root. Among the latter, the semantic changes can be considerable especially in cases where the valence changes, whereas in category-changing derivations the semantic difference is not always so great (\textstyleBibliogBaseChar{Bybee 1985}:83).  

\paragraph[Derivation vs. inflection]{Derivation vs. inflection}
\hypertarget{RefHeading20081935131865}{}
According to \textstyleBibliogBaseChar{Bybee} (1985:81), ``\textstyleBibliogCitationAAAstyleChar{an inflectional morpheme {\dots} is a bound non-root morpheme whose appearance in a particular position is compulsory}.'' It is ``\textstyleBibliogCitationAAAstyleChar{required by syntax}''. In contrast, derivational affixes are non-obligatory (\textstyleBibliogBaseChar{Greenberg 1954}:191). In Mauwake, all the derivations are non-obligatory. Of the inflections, the beneficiary suffix and the counterfactual suffix as such are not required by syntax like the tense and person marking, but they have an interdependence relationship with other suffixes: the beneficiary suffix has to occur in a past tense or imperative form of a verb that also has the benefactive suffix, and the counterfactual suffix restricts the tense marking to past tense.

In Mauwake verb structure the derivational suffixes always precede the inflectional ones. This agrees with one of \textstyleBibliogBaseChar{Greenberg}'s universals: ``\textstyleBibliogCitationAAAstyleChar{If the derivation and inflection follow the root {\dots} the derivation is always between the root and the inflection}'' (1966:93).

Inflectional suffixes in Mauwake form paradigms, even if in some cases the paradigms only have two members.

The greater syntagmatic freedom of derivational affixes (\textstyleBibliogBaseChar{Malkiel 1978}:128-9) is shown in Mauwake by the fact that a verb with any of the derivations can be nominalized with the nominalizing suffix \nobreakdash-\textstyleStyleVernacularWordsItalic{owa}, whereas one with inflectional suffixes cannot.\footnote{It is possible to nominalize whole \textit{clauses} where the main verb has inflectional suffixes, by adding the demonstrative \textstyleFootnoteBaseChar{\textit{nain}} `that' after the clause, but this strategy is not available for individual verbs (5.7.2).} This ability of verb stems with derivational suffixes to be nominalized is the main distinction between derivation and inflection in Mauwake. 

A special feature in Mauwake is the dividing point between the derivational and inflectional suffixes: the benefactive suffix is derivational, whereas the beneficiary suffix is inflectional. The latter can only be present when there are other verbal suffixes following, whereas the former can also be followed by a nominaliser suffix. Other differences between the two suffixes are described below (\sectref{sec:3.8.2.3.3}), (\sectref{sec:3.8.3.1}). In the following section, the derivational suffixes are introduced in the order that they occur following the verb root; the prefixes are discussed last.

There is clear iconicity in the linear ordering of the derivational suffixes: the closer the suffix is to the root, the more profound the change it effects on it. The verbalizing suffixes change the word class; the causative adds an argument; the distributive pluralizes an argument, and the benefactive adds a peripheral.





\begin{tabular}{llllll}
\mytoprule


PREFIX & ROOT & VERBALISER & CAUSATIVE & DISTRIBUTIVE & BENEFACTIVE\\
p\textbf{-} &  & \textbf{-}{\O} & \textbf{-}ow & \textbf{-}omak & \textbf{-}om\\
amap\textbf{-} &  & \textbf{-}ar &  & \textbf{-}urum & \\
aap\textbf{-} &  &  &  &  & \\
\textstyleAcronymallcaps{RDP} &  &  &  &  & \\
\mybottomrule
\end{tabular}



\begin{table}
\caption{Verbal derivation}
\label{tab:10}
\end{table}

\paragraph[Category-changing derivation: verb formation]{Category-changing derivation: verb formation}
\hypertarget{RefHeading20101935131865}{}
There are two strategies in Mauwake whereby words from other word classes can be changed into verbs.  Zero verb formation is less productive than the inchoative. Also the meanings of the verbs resulting from zero verb formation are in some cases more lexicalized, or less transparent, than the meanings of the verbs formed with the inchoative suffix. Often roots can be used for both the strategies, but not always: words like \textstyleStyleVernacularWordsItalic{amisa} `\textstyleFreeTranslationChar{knowledge'} and \textstyleStyleVernacularWordsItalic{ewur} `\textstyleFreeTranslationChar{quickly, fast'} only allow the inchoative suffix.

\subparagraph[Zero verb formation]{Zero verb formation}
\hypertarget{RefHeading20121935131865}{}
Mauwake has a number of verbs where the root is originally a noun, an adjective or an adverb, and the verb is formed without any overt morpheme to mark the category change. Hopper and \citet[745]{Thompson1984} remark that ``\textstyleBibliogCitationAAAstyleChar{languages often possess rather elaborate morphology whose sole function is to convert verbal roots into} \textstyleAcronymallcaps{N}\textstyleBibliogCitationAAAstyleChar{'s, but no morphology whose sole function is to convert nominal roots into} \textstyleAcronymallcaps{V}\textstyleBibliogCitationAAAstyleChar{'s}''. Zero verb formation is here understood, not as adding a zero morpheme, but as a lexical process (following Payne 1997:224). A noun, adjective or adverb is used as a root for the verb, and in this process it becomes a true verb, unlike nominalizations which are nouns but retain a lot of their verbal nature as well \citep[747]{HopperEtAl1984}%Thompson
.\footnote{\textstyleBibliogBaseChar{Hopper and Thompson} (1984) propose that a word root is unspecified as to the grammatical category, and that discourse function assigns categoriality. This fits many Austronesian languages in which there are plenty of words where only the non-root morphology, or else syntactic behaviour, shows what class the word belongs to. Mauwake has relatively few forms like this and it is reasonable to assign words to specific word classes even without reference to discourse function.}

The resulting verb is usually transitive, with a few exceptions. The final vowel of a noun or an adjective, usually  /a/, is deleted before the verbal inflection. 

From nouns:

akuwa  `knot'  akuw-  `knot/bind/tie with a knot'

anima  `blade'  anim-  `sharpen'

eneka  `tooth, flame'  enek-  `light (a fire)'

ilen  `sign'  ilen-  `recognise sign'

nanar  `story'  nanar-  `tell a story'

From adjectives:

dubila  `smooth'  dubil-  `smoothen'

enuma  `new'  enum-  `renew'

iiwa  `short'  iiw-  `shrink' 

itita  `soft'  itit-  `smash'

kaken  `straight'  kaken-  `straighten'

maneka  `big'  manek-  `enlarge'

momora  `fool'  momor-  `confuse'

samora  `bad'  samor-  `destroy'

siina  `tight'  siin-  \textstyleTableEntryChar{`diminish' (intr.)}\footnote{Another intransitive verb can be derived from \textstyleFootnoteBaseChar{\textit{siina}} with the inchoative suffix: \textstyleFootnoteBaseChar{\textit{siin-ar}}- `become tight/narrow'.}

From adverbs:

bilik  `mixed'  bilik-  `mix'

ikum  `illicitly'  ikum-  `speculate'

kerew  `strongly'  kerew-  `be angry at'

fan    `here'  fan-  `be/come here'

nan  `there'  nan-  `be/come here'

  \stepcounter{nx}{\thenx}x188)  Yo  aakisa  inasina  Rubaruba  \textstyleEmphasizedVernacularWords{nanar-i-yem}.\\
1s.UNM  now  spirit  Rubaruba  story-Np-PR.1s  

`Now I tell about spirit Rubaruba.'

  \stepcounter{nx}{\thenx}x189)  Aruf-ami  me  \textstyleEmphasizedVernacularWords{samor-eka}!\\
hit-SS.SIM  not  bad-IMP.2p  

`Don't hit/beat and destroy it.' 

Semantically the resulting verb is usually very close to the word that serves as the root, but in a few instances like (\stepcounter{nx}{\thenx}) the semantic link is not very strong.

\ea%x190
\label{ex:x190}
\gll Nefa  \textstyleEmphasizedVernacularWords{ikum-am-ika-iwkin}  nan  kerer-e-n. \\
      \\
\glt
\z

2s.ACC  illicitly-SS.SIM-be-2/3p.DS  there  appear-PA-2s  

`They were just speculating about you when you arrived.' 

\subparagraph[Inchoative suffix ]{Inchoative suffix} 
\hypertarget{RefHeading20141935131865}{}
The second verb formation process in Mauwake takes a noun, adjective or adverb root and adds an inchoative suffix -\textstyleStyleVernacularWordsItalic{ar}  (\sectref{sec:2.3.3.4}) to form a new verb usually meaning `become n'.\footnote{The term `inchoative' is used for  derivation; `inceptive' for aspect, following  \textstyleBibliogBaseChar{Payne} (1997:95).} Although in the majority of the cases a word from one of the other word classes is made into a verb, the basic meaning is inchoative rather than verbalizing, as the same suffix can also be added to a few verbs. The suffix has been grammaticalized from the verb \textstyleStyleVernacularWordsItalic{ar}\textstyleEmphasizedVernacularWords{-} `become', `enter into a state', and there are a few cases where it is difficult to decide with certainty which one it is. The differences between the full verb and the suffix are listed below. 

The full resultative verb \textstyleStyleVernacularWordsItalic{ar}\textstyleEmphasizedVernacularWords{-} `become' is more common with nouns, and the meaning of the verb is transparent (\stepcounter{nx}{\thenx}). It is also used with numerals (\stepcounter{nx}{\thenx}). Both words retain their word stress.

\ea%x191
\label{ex:x191}
\gll Arim-emi  mu'a  \textstyleEmphasizedVernacularWords{ar-'e-k}. \\
      \\
\glt
\z

grow-SS.SIM  man  become-PA-3s  

`He grew up and became man/adult.'

\ea%x192
\label{ex:x192}
\gll Aruf-owa  e\textstyleEmphasizedVernacularWords{'}repam  \textstyleEmphasizedVernacularWords{ar-'e-m}. \\
      \\
\glt
\z

hit-NMZ  four  become-PA-1s  

`I hit it four times.' (Lit: `Hitting it I became four.')

The inchoative suffix \textstyleEmphasizedVernacularWords{\nobreakdash-}\textstyleStyleVernacularWordsItalic{ar}  can occur with nouns (\stepcounter{nx}{\thenx}), but is more common with adjectives (\stepcounter{nx}{\thenx}) and adverbs (\stepcounter{nx}{\thenx}), and can attach to a few verb roots too (\stepcounter{nx}{\thenx}). Since the result is one word it only has one word stress. 

\ea%x193
\label{ex:x193}
\gll Yiena  opaimika  me  baliwep  \textstyleEmphasizedVernacularWords{a'mis-ar-e-mik}. \\
      \\
\glt
\z

1p.GEN  talk  not  well  knowledge-INCH-PA-1/3p  

`They don't know our language well.' 

\ea%x194
\label{ex:x194}
\gll Miiw-aasa  \textstyleEmphasizedVernacularWords{sa'mor-ar-ek}.\footnote{\textstyleFootnoteBaseChar{\textit{Miiw-aasa samor-a-k}}  `He broke the car' would be a corresponding sentence with zero verbalization. } \\
      \\
\glt
\z

land-canoe  bad-INCH-PA-3s  

`The car broke.'

\ea%x195
\label{ex:x195}
\gll Kau  pun  weeser-owa  \textstyleEmphasizedVernacularWords{e'wur-ar-ek.} \\
      \\
\glt
\z

cow  too  finish-NMZ  quickly-INCH-PA-3s  

`The beef finished quickly too.'

\ea%x196
\label{ex:x196}
\gll Mua  \textstyleEmphasizedVernacularWords{i'men-ar-ep}  opora  pun  \textstyleEmphasizedVernacularWords{i'men-ar-ek}. \\
      \\
\glt
\z

man  find-INCH-SS.SEQ  talk  too  find-INCH-PA-3s

`When man appeared, talk/language appeared too.'

Verbs derived from adjectives are often used rather than adjectives in the predicative position (\stepcounter{nx}{\thenx}), (\stepcounter{nx}{\thenx}).  And instead of a modifying adjective, a whole relative clause with a verb derived from an adjective may be used (\stepcounter{nx}{\thenx}). This happens especially when the property denoted by the adjective is not static. 

\ea%x82
\label{ex:x82}
\gll Sia  nain  senam  \textstyleEmphasizedVernacularWords{pin(a)-ar-e-k}.  \\
      \\
\glt
\z

netbag  that1  too.much  heavy-INCH-PA-3s

`The netbag is/was (lit: became) very heavy.' 

\ea%x1764
\label{ex:x1764}
\gll Muuka  nain  op-iya  \textstyleEmphasizedVernacularWords{dubil}\textstyleEmphasizedVernacularWords{(a)-}\textstyleEmphasizedVernacularWords{al}\textstyleEmphasizedVernacularWords{-}\textstyleEmphasizedVernacularWords{e}\textstyleEmphasizedVernacularWords{-}\textstyleEmphasizedVernacularWords{k}. \\
      \\
\glt
\z

boy  that1  hold-2/3s.DS  slippery-INCH-PA-3s

`When he\textsubscript{1} held the boy\textsubscript{2}, he\textsubscript{2} was slippery.'

\ea%x83
\label{ex:x83}
\gll [Konima  \textstyleEmphasizedVernacularWords{supuk(a)-ar-e-k  nain}]  yasuw-e. \\
      \\
\glt
\z

cloth  wet-INCH-PA-3s  that1  wash-IMP.2s

`Wash the wet cloth.' (Lit: `Wash the cloth that has become wet.')

The consonant /r/ in the suffix is lateralized into /l/ when the root has /l/ in the immediately preceding syllable.  Lateralization takes place arbitrarily in a few other cases as well (\stepcounter{nx}{\thenx}).

\ea%x197
\label{ex:x197}
\gll Yo  damol(a)-\textstyleEmphasizedVernacularWords{al}-e-m  oo.  \\
      \\
\glt
\z

1s.UNM  bad-INCH-PA-1s  oh  

`I feel terrible.' (Lit: `I'm destroyed/ruined.') 

\ea%x198
\label{ex:x198}
\gll Epa  dabel(a)-\textstyleEmphasizedVernacularWords{al}-ek. \\
      \\
\glt
\z

place  cold-INCH-PA-3s  

`It is cold.'

\ea%x199
\label{ex:x199}
\gll Opaimika  efa  masi(a)-\textstyleEmphasizedVernacularWords{al}-i-ya. \\
      \\
\glt
\z

mouth  1s.ACC  bitter-INCH-Np-PR.3s  

`It tastes bitter to me / in my mouth.'

If two preceding syllables contain /l/, the consonant in the verbaliser is not lateralized.\footnote{This rule is very tentative, as \textstyleFootnoteBaseChar{\textit{ilelar}}- is the only example found so far.}

\ea%x200
\label{ex:x200}
\gll Aasa  puuk-ap  ilel(a)-\textstyleEmphasizedVernacularWords{ar}-i-ya. \\
      \\
\glt
\z

canoe  cut-SS.SEQ  gouge-INCH-Np-PR.3s  

`He has cut the canoe (length from a tree) and is gouging/carving it' 

The suffix \textstyleStyleVernacularWordsItalic{ar}- often retains its original verbal meaning `become' when adjectives are made into verbs (\stepcounter{nx}{\thenx}), but when the other word classes are used as the root the original meaning tends to become more opaque or get lost (\stepcounter{nx}{\thenx}). 

\ea%x201
\label{ex:x201}
\gll \textstyleEmphasizedVernacularWords{Dubil}\textstyleEmphasizedVernacularWords{(a)-al-e-k.} \\
      \\
\glt
\z

slippery/smooth-INCH-PA-3s  

`It became slippery/smooth.' 

\ea%x202
\label{ex:x202}
\gll No  \textstyleEmphasizedVernacularWords{wadol}\textstyleEmphasizedVernacularWords{(a)-al-i-n.} \\
      \\
\glt
\z

2s.UNM  lie-INCH-Np-PR.2s  

`You are lying.' 

Most of the verbs formed with the inchoative suffix are intransitive, but some are active, transitive verbs:

\ea%x203
\label{ex:x203}
\gll Muuka  kuisow  \textstyleEmphasizedVernacularWords{muuk}\textstyleEmphasizedVernacularWords{(a)-ar-e-k}. \\
      \\
\glt
\z

son  one  son-INCH-PA-3s  

`She gave birth to one son.'

\ea%x204
\label{ex:x204}
\gll Epa  \textstyleEmphasizedVernacularWords{mores-ar-ep}  ikiw-o-k. \\
      \\
\glt
\z

place  like(ADV)-INCH-SS.SEQ  go-PA-3s  

`He made the place ready and went.'

The inchoative verb formation is also used with verb loans from other languages, especially Tok Pisin.\footnote{The Tok Pisin loans often originally come from English.} Both of the loan words below also have a vernacular synonym.

\ea%x487
\label{ex:x487}
\gll Muuka  wia  \textstyleEmphasizedVernacularWords{was-ar-e-mik}.\textstyleParagraphChari{   (from Tok Pisin} \textstyleForeignWords{was} \textstyleParagraphChari{`look after')} \\
      \\
\glt
\z

son  3p.ACC  look.after-INCH-PA-1/3p

`They were looking after the boys/children'

\ea%x488
\label{ex:x488}
\gll \textstyleEmphasizedVernacularWords{Nading-ar-ep}  uf-e-mik.  (from Mala \textstyleForeignWords{nading} `decoration') \\
      \\
\glt
\z

decoration-INCH-SS.SEQ  dance-PA-1/3p

`We decorated ourselves and danced.'

\paragraph[Category-maintaining derivation: suffixes]{Category-maintaining derivation: suffixes}
\hypertarget{RefHeading20161935131865}{}
\subparagraph[Causative suffix]{Causative suffix}
\hypertarget{RefHeading20181935131865}{}
The causative suffix \nobreakdash-\textstyleStyleVernacularWordsItalic{ow} (or \nobreakdash-\textstyleStyleVernacularWordsItalic{aw}) transitivizes an intransitive verb \citep[2]{Peterson2007}: the clause gets a new subject, and the subject of the intransitive verb becomes the direct object. Usually it adds a causative meaning `cause someone to do something', or `cause something to happen'. The object of a causative construction has no control, or only minimal control, over the action or event indicated by the verb.

In many verbs there is free variation between \nobreakdash-\textstyleStyleVernacularWordsItalic{ow} and \nobreakdash-\textstyleStyleVernacularWordsItalic{aw}. Some verbs seem to prefer one or the other, but there is no clear pattern. There is also some dialectal and possibly age-based variation depending on the speaker. \nobreakdash-\textstyleStyleVernacularWordsItalic{ow}  is taken here as the basic form, since it is the more common of the two, and because in ``double causatives'' it is always used at least as the first one.  

CAUSATIVE    FROM:  

arim-ow-  `bring up / raise'  arim-  `grow'

in-aw-  `put to bed'  in-  `lie down'

bagiwir-ow  `cause to be angry'  bagiwir-  `be angry'

iimar-ow-  `make sg. stand up'  iimar-  `stand up'

imenar-ow-  `create/cause to appear'  imenar-  `appear'

waki-ow-aw-  `cause to stumble'  waki-  `stumble'

ook-ow-  `place alongside'  ook-  `follow'

Sometimes the causative suffix occurs reduplicated as a ``double causative'', but these still add only one argument. Many of the short directional verbs (\sectref{sec:3.8.4.4.5}) take a double causative instead of a single one. 

\ea%x205
\label{ex:x205}
\gll Eewua  ir\textstyleEmphasizedVernacularWords{-ow-aw-}ap  osaiwa  ar-e-k. \\
      \\
\glt
\z

wing  climb-CAUS-CAUS-SS.SEQ  bird.of.paradise  become-PA-3s  

`She put the wing up (on herself) and became a bird of paradise.'

A single or double causative can be added to the intransitive verb \textstyleStyleVernacularWordsItalic{reen}- `(become) dry' with the result of two different meanings, but both of these still only add one more argument: \textstyleStyleVernacularWordsItalic{reenow}- `dry (something)', \textstyleStyleVernacularWordsItalic{reenowaw}- `smoke (something)'.

The only two transitive verbs that have been found to take the causative are \textstyleStyleVernacularWordsItalic{mik}- `spear/hit' and \textstyleStyleVernacularWordsItalic{op}- `hold/grab', with the causative forms \textstyleStyleVernacularWordsItalic{mik-ow-aw}- `join (the ends of two long items)' and \textstyleStyleVernacularWordsItalic{op-aw}- `accuse falsely'. 

Verbs that do \textstyleEmphasizedWords{\textsc{not}} have any causative meaning include the following:

aakun-ow-  `grumble (at)'  from:  aakun-  `speak'

baun-ow-  `bark (at)'  baun-  `bark'

kirir-ow-  `shout (about)'  kirir-  `shout'

op-aw-  `accuse falsely'  op-  `hold'

\ea%x991
\label{ex:x991}
\gll Mukuna  kuuf-ap  kirir-e-k. \\
      \\
\glt
\z

fire  see-SS.SEQ  shout-PA-3s

`She saw the fire and shouted.'

\ea%x489
\label{ex:x489}
\gll Yiok-ami  naap  \textstyleEmphasizedVernacularWords{yia}  \textstyleEmphasizedVernacularWords{kirir-ow-am-ik-ua.} \\
      \\
\glt
\z

follow.us-SS.SIM  thus  1p.ACC  shout-CAUS-SS.SIM-be-PA.3s

`She was following us and shouting about us like that.'

The causative as a valence-increasing device is discussed in \sectref{sec:3.8.4.3.1}. 

\subparagraph[Distributive suffix]{Distributive suffix}
\hypertarget{RefHeading20201935131865}{}
A distributive suffix pluralizes one of the verbal arguments. There are two distributive suffixes: \nobreakdash-\textstyleStyleVernacularWordsItalic{urum}\textstyleEmphasizedVernacularWords{} `all' and \nobreakdash-\textstyleStyleVernacularWordsItalic{omak} `many'. They are fully productive in the whole verb class, as long as the semantics of the verb allows multiple arguments. 

The hierarchy of which argument the distributive applies to is as follows: if there is a recipient (\stepcounter{nx}{\thenx}) or beneficiary (\stepcounter{nx}{\thenx}), the distributive applies to that; if there is no recipient or beneficiary but an object, the distributive applies to the object (\stepcounter{nx}{\thenx}); and in case the clause has neither a recipient or beneficiary nor an object, the distributive applies to the subject (\stepcounter{nx}{\thenx}). Since transitive verbs need an object, the subject can be pluralized with the distributive only when the verb is intransitive.

{\bfseries
REC/BEN  {\textgreater}  O  {\textgreater}  S


\ea%x209
\label{ex:x209}
\gll Mua  teeria  opaimika  wia  sesek-\textstyleEmphasizedVernacularWords{omak}-e-mik. \\
      \\
\glt
\z

man  family  talk  3p.ACC  send-DISTR/PL-PA-1/3p  

`They sent word to (many members of) the man's family.'

\ea%x429
\label{ex:x429}
\gll Wiena  wiawi=ke  amia  wia  keraw-om-\textstyleEmphasizedVernacularWords{omak}-e-mik. \\
      \\
\glt
\z

3p.GEN  3s/p.father=CF  spear  3p.ACC  carve-BEN-DISTR/PL-PA-1/3p

`Their fathers carved spears for them (\textit{many} beneficiaries).'

\ea%x492
\label{ex:x492}
\gll Emeria  unowa  fain  nia  aaw-\textstyleEmphasizedVernacularWords{urum}-i-kuan. \\
      \\
\glt
\z

woman  many  this   2s.ACC  take-DISTR/A-Np-FU.3p

`They will take all of you women.'

\ea%x491
\label{ex:x491}
\gll Emeria  teeria  koka  ikiw-\textstyleEmphasizedVernacularWords{urum}-e-mik. \\
      \\
\glt
\z

woman  group  jungle  go-DISTR/A-PA-1/3p

`The whole group of women / all the women went to the jungle.'

In verbal groups (\sectref{sec:3.8.5.1}) the distributive suffix usually attaches to the last verb root, but it can occasionally also attach to the first root, i.e. the main verb  in  a verb+\textstyleAcronymallcaps{AUX} combination (\stepcounter{nx}{\thenx}).

\ea%x207
\label{ex:x207}
\gll Iinan  aasa  ikiw-emi  paran-em-\textstyleEmphasizedVernacularWords{mi-omak-e-k}. \\
      \\
\glt
\z

sky  canoe  go-SS.SIM  rumble-SS.SIM-go.around-DISTR/PL-PA-3s  

 `Many planes went rumbling around.'

\ea%x490
\label{ex:x490}
\gll Iinan  aasa  fan  or-om\textstyleEmphasizedVernacularWords{-ik-omak-eya}  {\dots} \\
      \\
\glt
\z

sky  canoe  here1  descend-SS.SIM-be-DISTR/PL-2/3s.DS

`When many planes were coming down here {\dots}'

\ea%x208
\label{ex:x208}
\gll Wi  ifa  saarik  \textstyleEmphasizedVernacularWords{in-urum-ep}-ik-e-mik.  \\
      \\
\glt
\z

3p.UNM  snake  like  sleep-DISTR/A-SS.SEQ-be-PA-1/3p  

`They all slept/lay like snakes.'

Both suffixes can be attached to the same verb but it is rare. In that case \textstyleEmphasizedVernacularWords{\nobreakdash-}\textstyleStyleVernacularWordsItalic{urum} precedes \textstyleEmphasizedVernacularWords{\nobreakdash-}\textstyleStyleVernacularWordsItalic{omak}.

\ea%x206
\label{ex:x206}
\gll Wia  ifakim\textstyleEmphasizedVernacularWords{-urum}-\textstyleEmphasizedVernacularWords{omak}-e-mik. \\
      \\
\glt
\z

3p.ACC  kill-DISTR/A-DISTR/PL-PA-1/3p  

`They killed each and every one of them.' (There were many of those killed.) 

\subparagraph[Benefactive suffix ]{Benefactive suffix} 
\hypertarget{RefHeading20221935131865}{}
The benefactive suffix, indicating the fact that the action of the verb is done \textstyleEmphasizedWords{\textsc{for someone}}, for their benefit or detriment, is a borderline case among the derivations. It is the last one of the derivational suffixes, and the \textstyleEmphasizedWords{\textsc{beneficiary suffix}} (\sectref{sec:3.8.3.1}) following it and marking the person that the action is done for, is inflectional even if the two suffixes go together semantically. The position of the benefactive is not as stable as that of the other suffixes: it comes after the distributive when the beneficiary is first person singular (\stepcounter{nx}{\thenx}), (\stepcounter{nx}{\thenx}) but occurs preceding it with the other persons (\stepcounter{nx}{\thenx}), (\stepcounter{nx}{\thenx}). 

\ea%x210
\label{ex:x210}
\gll Mua  Maneka=ke  maa  maneka  on-omak-\textstyleEmphasizedVernacularWords{om}-e-k. \\
      \\
\glt
\z

Man  Big=CF  thing  big  do-DISTR/PL-BEN-BNFY1.PA\footnote{The vowel of the beneficiary suffix deletes the vowel of the past tense suffix. The relationship between the beneficiary suffix and the suffix following it is discussed in detail in \sectref{sec:3.8.3.1}, and the medial suffix forms are discussed in \sectref{sec:3.8.3.5}.} -3s  

`God did great things to/for me.'

\ea%x1925
\label{ex:x1925}
\gll Buk  aaw-omak-\textstyleEmphasizedVernacularWords{om}-e! \\
      \\
\glt
\z

book  get-DISTR/PL-BEN-BNFY1.IMP.2s

`Get the books for me!'

\ea%x211
\label{ex:x211}
\gll Buk  aaw-\textstyleEmphasizedVernacularWords{om}-omak-e! \\
      \\
\glt
\z

book  get-BEN-DISTR/PL-IMP.2s  

`Get the books for him!'

\ea%x1926
\label{ex:x1926}
\gll Wiena  wiawi=ke  amia  wia  keraw-\textstyleEmphasizedVernacularWords{om}-omak-e-mik. \\
      \\
\glt
\z

3p.GEN  3s/p.father=CF  bow  3p.ACC  carve-BEN-DISTR/PL-PA-1/3p  

`Their fathers carved bows for them.'

In verbal groups the benefactive suffix is usually attached to the finite verb or auxiliary (\stepcounter{nx}{\thenx}) but can occasionally occur on the non-finite root (\stepcounter{nx}{\thenx}) or even on both of the two (\stepcounter{nx}{\thenx}).

\ea%x212
\label{ex:x212}
\gll Iwera  wia  uruk-am-ik-\textstyleEmphasizedVernacularWords{om}-a-mik.  \\
      \\
\glt
\z

coconut  3p.ACC  drop-SS.SIM-be-BEN-BNFY2.PA-1/3p

`We kept dropping coconuts for them.' 

\ea%x213
\label{ex:x213}
\gll Maamuma  wia  p-ikiw-\textstyleEmphasizedVernacularWords{om}-ap-pu-ap  {\dots} \\
      \\
\glt
\z

money  3p.ACC  BPf-go-BEN-BNFY2.SS.SEQ-CMPL-SS.SEQ  

`Having taken money to them, {\dots} 

\ea%x214
\label{ex:x214}
\gll Moro  mua  wia  wu-\textstyleEmphasizedVernacularWords{om}-am-ik-\textstyleEmphasizedVernacularWords{om}-a-mik. \\
      \\
\glt
\z

Moro  man  3p.ACC  put-BEN-BNFY2.SS.SIM-be-BEN-BNFY2.PA-1/3p

`They put them (=carts) for the Moro men.' 

The benefactive form does not always mean that something happens for someone's\textstyleEmphasizedWords{} \textstyleEmphasizedWords{\textsc{benefit}}. The benefactive may be strengthened with the adverb \textstyleStyleVernacularWordsItalic{orawin} `for the benefit' (\stepcounter{nx}{\thenx}), which makes it unambiguous.

\ea%x215
\label{ex:x215}
\gll Iwera  \textstyleEmphasizedVernacularWords{orawin}  kais-\textstyleEmphasizedVernacularWords{om}-e-mik. \\
      \\
\glt
\z

Coconut  for.the.benefit  husk-BEN-BNFY1.PA-1/3p  

`They husked coconuts for me (for free).'

By using a suffix completely unrelated to the verb `give', Mauwake shows itself different from all of those reasonably closely related languages that have grammatical descriptions available. A serial verb construction involving the verb `give' is a very common way of expressing benefactive in Papuan languages (\textstyleBibliogBaseChar{Foley 1986}:141).  Waskia (\textstyleBibliogBaseChar{Ross and Paol 1978}:45) and Maia (\textstyleBibliogBaseChar{Hardin 2002}:125) employ this strategy, and in Usan it is behind one of the two strategies: the benefactive verb form has been grammaticalized from a serial verb with the verb `give' (\textstyleBibliogBaseChar{Reesink 1987}:110-1). The other strategy for Usan is to use a postposition with the appropriate noun phrase (ibid. 154). Bargam is similar to it (\textstyleBibliogBaseChar{Hepner 2002}:65-6, 99), but Amele utilizes an indirect object clitic attached to the verb to express the beneficiary as well as other semantic relations (\textstyleBibliogBaseChar{Roberts 1987}:167). 

\paragraph[Derivational prefixes]{Derivational prefixes}
\hypertarget{RefHeading20241935131865}{}
Although Mauwake is very strongly a suffixing language, it makes use of some derivational prefixes as well. Reduplication is the most common among these. 

\subparagraph[Reduplication]{Reduplication}
\hypertarget{RefHeading20261935131865}{}
The morphophonological aspect of reduplication was already described in \sectref{sec:2.3.3.2}. In \sectref{sec:6.4.2} reduplication is discussed as one of the many quantification strategies in Mauwake.

Reduplication in verbs is used in Mauwake to indicate continuity or iterativity of action and/or plurality of the resulting object. Mostly the reduplication is done only once, but especially motion verbs can have several identical reduplicative prefixes. 

In verbs of motion reduplication means continuity (\stepcounter{nx}{\thenx}), and the passing of time may be shown by the number of reduplications (\stepcounter{nx}{\thenx}).

\ea%x218
\label{ex:x218}
\gll \textstyleEmphasizedVernacularWords{Biri-birin-emi}  wia  akim-omak-e-mik.  \\
      \\
\glt
\z

RDP-fly-SS.SIM  3p.ACC  try-DISTR/PL-PA-1/3p  

`They were flying and teasing them.' 

\ea%x216
\label{ex:x216}
\gll Ne  \textstyleEmphasizedVernacularWords{oro-oro-oro}-oro-mi  \textstyleEmphasizedVernacularWords{oro-oro}-or-o-k,  \\
      \\
\glt
\z

and  RDP-RDP-RDP-descend-SS.SIM  RDP-RDP-descend-PA-3s

onoma.

horizon.

`And it went down and down and down all the way to the horizon.'

In other intransitive verbs reduplication indicates either iterative action (\stepcounter{nx}{\thenx}) or occasionally continuity (\stepcounter{nx}{\thenx}).

\ea%x217
\label{ex:x217}
\gll Nomokowa  \textstyleEmphasizedVernacularWords{ku-ku-ep}  or-om-ik-ua. \\
      \\
\glt
\z

tree  RDP-break-SS.SEQ  descend-SS.SIM-be-PA.3s

`The timber (in a bridge) kept breaking and falling down.' 

\ea%x692
\label{ex:x692}
\gll Epa  \textstyleEmphasizedVernacularWords{wii-wiim-ik-ua},  {\dots} \\
      \\
\glt
\z

place  RDP-dawn-be-PA.3s

`It was dawning, {\dots}'

Both of these meanings fit in well with \textstyleBibliogBaseChar{Moravcsik}'s description of the various meanings that reduplication in verbs can have (1978:319). In transitive verbs reduplication indicates iterative action as well as the plurality of an inanimate object (\stepcounter{nx}{\thenx}). The form is used especially when the action \textstyleEmphasizedWords{results} in a plural object (\stepcounter{nx}{\thenx}).

\ea%x219
\label{ex:x219}
\gll Iinan  aasa=ke  maifa  \textstyleEmphasizedVernacularWords{fu-fuurk-ikiw-o-k}. \\
      \\
\glt
\z

sky  canoe=CF  paper  RDP-throw-go-PA-3s  

`The plane went  throwing paper slips down' 

\ea%x220
\label{ex:x220}
\gll Oposia  nain  \textstyleEmphasizedVernacularWords{pu-puuk-ap}  uup-e-mik. \\
      \\
\glt
\z

meat  that1  RDP-cut-SS.SEQ  cook-PA-1/3p  

`We cut up the meat (into many pieces) and cooked it.'

Usan differs from Mauwake in that it does not use reduplication very much in verbs, and never in main clause final verbs (\textstyleBibliogBaseChar{Reesink 1987}:116). Also, when reduplication is used to indicate duration or repetition the whole verb word is reduplicated (ibid. 117). In Bargam reduplication occurs but is not very productive. In transitive verbs reduplication indicates plurality of objects, in intransitive verbs plurality of subjects (\textstyleBibliogBaseChar{Hepner 2002}:19). In Maia ``\textstyleBibliogCitationAAAstyleChar{verb roots may be partially or completely reduplicated. Verb reduplication broadly indicates an augmented action which may include a greater, more massive, more intensified or very often repetitive form of the action}'' (\textstyleBibliogBaseChar{Hardin 2002}:50).

\subparagraph[Bring-prefixes]{Bring-prefixes}
\hypertarget{RefHeading20281935131865}{}
The prefixes in this group change the directional verbs (see \sectref{sec:3.8.4.4.5}) into transitive verbs with the meaning `bring' or `take'.  \textstyleStyleVernacularWordsItalic{p}\nobreakdash- is a neutral prefix and by far the most common one (\stepcounter{nx}{\thenx}), \textstyleStyleVernacularWordsItalic{amap}\nobreakdash- is used when something is brought out in the open, often with the meaning `bring forth'. Usually there is a clear goal, a person or a place, which may not be mentioned in the clause itself but occurs in an earlier one (\stepcounter{nx}{\thenx}), or is understood from the context (\stepcounter{nx}{\thenx}). If the goal is explicitly mentioned in the clause, the neutral prefix is used (\stepcounter{nx}{\thenx}), (\stepcounter{nx}{\thenx}).  The prefix \textstyleStyleVernacularWordsItalic{aap}\nobreakdash- (\stepcounter{nx}{\thenx}) is very rare and I have been unable to establish whether it really differs from \textstyleStyleVernacularWordsItalic{amap}\nobreakdash- or whether it is just a matter of idiolectal use.\footnote{The bring\nobreakdash-prefixes may have been grammaticalized from a medial verb construction involving the verb \textit{aaw}\nobreakdash- `take'. It is easy to see how \textit{aawep ekap}\nobreakdash- `take (and) come' could have developed into \textit{aapekap}\nobreakdash- `bring' and possibly also into \textit{pekap}-. Another possibility is that it is a result of a related process to that in Usan where the verb \textstyleFootnoteBaseChar{\textit{ba}} `take' has contracted into \textstyleFootnoteBaseChar{\textit{b}}\nobreakdash-, which has combined with verbs of motion and been lexicalized with the meaning of bringing or taking (\textstyleBibliogBaseChar{Reesink 1987}:144-5). The \textstyleFootnoteBaseChar{\textit{amap}}\nobreakdash-prefix may have its origin in the expression \textstyleFootnoteBaseChar{\textit{ama-pa}} `in the sun', which implies `in the open'. There is also a very slight possibility that the \textit{p}\nobreakdash-prefix might be an Austronesian loan, as p(V\textit{)\nobreakdash-} is a common causative or transitivizer prefix in Austronesian languages \citep[61]{Bugenhagen1995}. \textstyleFootnoteBaseChar{But all of this is just conjecture at this point.}}

\ea%x221
\label{ex:x221}
\gll Amina  aaw-ep  Liisa  ame  wia \\
      \\
\glt
\z

pot  take/get-SS.SEQ  Liisa  others  3p.ACC  

\textstyleEmphasizedVernacularWords{p-er}-om-a.

Bpx-go-BEN-BNFY2.IMP.2s

`Get the pot and take it to Liisa and the others.' 

\ea%x430
\label{ex:x430}
\gll Pita  pensil  wiar  or-op  ik-ua  nain  aaw-ep \\
      \\
\glt
\z

Pita  pencil  3.DAT  fall-SS.SEQ  be-PA.3s  that1  take-SS.SEQ

\textstyleEmphasizedVernacularWords{amap-ikiw}-om-aka.

Bpx-go-BEN-BNFY2.IMP.2p

`Take to Pita his pencil that has dropped.'

\ea%x222
\label{ex:x222}
\gll Wiipa  oko  \textstyleEmphasizedVernacularWords{amap-ora}-iwkin  ma-e-k  {\dots} \\
      \\
\glt
\z

daughter  other  Bpx-descend-2/3p.DS  say-PA-3s  

`When they took another daughter down (from the house out in the open), he said{\dots}' 

\ea%x223
\label{ex:x223}
\gll Ni  auwa  maa  \textstyleEmphasizedVernacularWords{p-urup}-om-aka.  \\
      \\
\glt
\z

2p.UNM  father  food  Bpx-ascend-BEN-BNFY2.IMP.2p  

`Take food (up) to father.'

\ea%x224
\label{ex:x224}
\gll Iwera  ir-ap  erup  op-ap  \textstyleEmphasizedVernacularWords{aap-or}-e. \\
      \\
\glt
\z

coconut  go.up-SS.SEQ  two  grab-SS.SEQ  Bpx-descend-IMP.2s  

`Climb the coconut palm, grab two coconuts and bring them down.' 

\subsubsection{Verb inflection} 
\hypertarget{RefHeading20301935131865}{}
The following table 11 shows those inflectional suffixes for the Mauwake verbs that change with the person and/or number of the subject. All of these are discussed in more detail below.





\begin{tabular}{llllllllll}
\mytoprule


\multicolumn{1}{l}{} & {\bfseries BNFY} & \multicolumn{1}{l}{{\bfseries CNTF}

 & {\bfseries TENSE} & \multicolumn{2}{l}{{\bfseries PERS./ NUMBER}

 & \multicolumn{1}{l}{{\bfseries IMPERAT.}

 & \multicolumn{3}{l}{{\bfseries MEDIAL}

\\
\multicolumn{1}{l}{} &  & \multicolumn{1}{l}{} & NON-PAST:

\textbf{-}i

PAST:

\textbf{-}E / -a & \multicolumn{1}{l}{PRES}

 & PAST & \multicolumn{1}{l}{} & \multicolumn{3}{l}{SAME SUBJECT}

\\
\multicolumn{1}{l}{1s

 & -e & -ek &  & \multicolumn{1}{l}{\textbf{-}yem}

 & \textbf{-}m & \multicolumn{1}{l}{-u \textstyleTableEntryChar{(1d)}

 & \multicolumn{3}{l}{SEQUENTIAL:  -ep/ap

\\
\hhline{---~------}
2s &  &  &  & \multicolumn{2}{l}{\textbf{-}n}

 & \multicolumn{1}{l}{\textbf{-}e /-a}

 & \multicolumn{3}{l}{SIMULTANEOUS:  -emi/ami

\\
\hhline{---~------}
\multicolumn{1}{l}{3s

 & -a &  &  & \multicolumn{1}{l}{\textbf{-}ya}

 & \textbf{-}k & \multicolumn{1}{l}{\textbf{-}inok}

 & \multicolumn{3}{l}{DIFFERENT  SUBJECT

\\
\hhline{---~------}
1p &  &  &  & \multicolumn{2}{l}{\textbf{-}mik}

 & \multicolumn{1}{l}{\textbf{-}ikua\textbf{} }

 & \multicolumn{2}{l}{\textbf{-}Vmkun  (s \& p)}

 & 1\\
\hhline{-~-~------}
2p &  &  &  & \multicolumn{2}{l}{\textbf{-}man}

 & \multicolumn{1}{l}{\textbf{-}eka\textbf{ /-}aka}

 & \multicolumn{1}{l}{\textbf{-}eya (s)}

 & \multicolumn{1}{l}{\textbf{-}iwkin (p)}

 & 2\\
\hhline{-~-~------}
3p &  &  &  & \multicolumn{2}{l}{\textbf{-}mik}

 & \textbf{-}uk &  &  & 3\\
\hhline{---~------}
\multicolumn{1}{l}{} &  &  &  & \multicolumn{2}{l}{FUTURE}

 & \multicolumn{4}{l}{}\\
\hhline{---~--~~~~}
1s &  &  &  & \multicolumn{2}{l}{\textbf{-}nen}

 &  &  &  & \\
\hhline{-~-~--~~~~}
2s &  &  &  & \multicolumn{2}{l}{\textbf{-}nan}

 &  &  &  & \\
\hhline{-~-~--~~~~}
3s &  &  &  & \multicolumn{2}{l}{\textbf{-}non}

 &  &  &  & \\
\hhline{-~-~--~~~~}
1p &  &  &  & \multicolumn{2}{l}{\textbf{-}yen}

 &  &  &  & \\
\hhline{-~----~~~~}
2p &  & \multicolumn{1}{l}{} & \textbf{-}o (Np) & \multicolumn{2}{l}{\textbf{-}wen}

 &  &  &  & \\
\hhline{-~----~~~~}
3p &  & \multicolumn{1}{l}{} &  & \multicolumn{2}{l}{\textbf{-}kuan}

 &  &  &  & \\
\hhline{-~----~~~~}

\mybottomrule
\end{tabular}



\begin{table}
\caption{Inflectional suffixes of Mauwake verbs}
\label{tab:11}
\end{table}

\paragraph[Beneficiary]{Beneficiary}
\hypertarget{RefHeading20321935131865}{}
The beneficiary suffix indicates the person the action is done for. Its position is directly after the benefactive suffix, or after the distributive suffix in those few cases where the benefactive comes before the distributive (\sectref{sec:3.8.2.3.3}). It is inflectional rather than derivational because 1) when it is used, nominalization is blocked and 2) it has a paradigm for different persons, even if the paradigm only consists of two members. 

The only two forms for the beneficiary are  \textstyleEmphasizedVernacularWords{\nobreakdash-}\textstyleStyleVernacularWordsItalic{e}  for first or second person singular (\stepcounter{nx}{\thenx}) and \textstyleEmphasizedVernacularWords{\nobreakdash-}\textstyleStyleVernacularWordsItalic{a}  for all the other persons (\stepcounter{nx}{\thenx}). The context often provides more person distinctions, as the plural requires accusative pronouns to precede the verb to indicate the beneficiary, like third person plural in (\stepcounter{nx}{\thenx}). 

\ea%x225
\label{ex:x225}
\gll Wafur-om-\textstyleEmphasizedVernacularWords{e}!   \\
      \\
\glt
\z

throw-BEN-BNFY1.IMP.2s  

`Throw it to me!' 

\ea%x226
\label{ex:x226}
\gll Marasin  wu-om-\textstyleEmphasizedVernacularWords{a}-mik=na  weetak. \\
      \\
\glt
\z

medicine  put-BEN-BNFY2.PA-1/3p=TP  no  

`They put medicine on him but no (it didn't help).'

\ea%x227
\label{ex:x227}
\gll Na-iwkin  \textstyleEmphasizedVernacularWords{wia}  uf-om-\textstyleEmphasizedVernacularWords{a}-mik.  \\
      \\
\glt
\z

say-2/3p.DS  3p.ACC  dance-BEN-BNFY2.PA-1/3p  

`They said so and we danced for them.'

When the beneficiary suffix is followed by a vowel, a mid vowel is deleted adjacent to a low vowel (\stepcounter{nx}{\thenx}) and both a mid and a low vowel are deleted preceding a high vowel (\stepcounter{nx}{\thenx}). In the latter case the person distinction gets neutralized in the singular (\stepcounter{nx}{\thenx}), but not in the plural where the obligatory accusative pronouns maintain the distinction (\stepcounter{nx}{\thenx}). The examples (\stepcounter{nx}{\thenx})-(\stepcounter{nx}{\thenx}) below show how the beneficiary suffix affects the past tense suffix. In (\stepcounter{nx}{\thenx}) a sequence of two identical vowels is reduced to one vowel.

\ea%x228
\label{ex:x228}
\gll aaw-om-\textstyleEmphasizedVernacularWords{ak}-a-m    {\textless  aaw-om}-\textstyleEmphasizedVernacularWords{a-ek}-a-m  \\
      \\
\glt
\z

get-BEN-BNFY2.CNTF-PA-1s  

`I would have gotten it for him' 

\ea%x444
\label{ex:x444}
\gll aaw-om-\textstyleEmphasizedVernacularWords{i-non}      {\textless  aaw-om-}\textstyleEmphasizedVernacularWords{e-i-non}, aaw-om-\textstyleEmphasizedVernacularWords{a-i-non} \\
      \\
\glt
\z

get-BEN-BNFY.Np-FU.3s

`he will get it for me/you/him/her'

\ea%x1750
\label{ex:x1750}
\gll aaw-om-\textstyleEmphasizedVernacularWords{uk}        {\textless aaw-om-}\textstyleEmphasizedVernacularWords{e-uk}, aaw-om-\textstyleEmphasizedVernacularWords{a-uk} \\
      \\
\glt
\z

get-BEN-BNFY.IMP.3p

`let them get it for me/you/him/her'

\ea%x445
\label{ex:x445}
\gll Panewowa  maa  \textstyleEmphasizedVernacularWords{wia}  p-ikiw-om-\textstyleEmphasizedVernacularWords{uk}. \\
      \\
\glt
\z

old  food  3p.ACC  BPx-go-BEN-BNFY.IMP.3p

`Let them take food for the old people.'

\ea%x494
\label{ex:x494}
\gll Uf-\textstyleEmphasizedVernacularWords{o}-k. \\
      \\
\glt
\z

dance-PA-3s

`He danced.'

\ea%x495
\label{ex:x495}
\gll Uf-om-\textstyleEmphasizedVernacularWords{e}-k. \\
      \\
\glt
\z

dance-BEN-BNFY1.PA-3s

`He danced for me/you.'

\ea%x496
\label{ex:x496}
\gll Uf-om-\textstyleEmphasizedVernacularWords{a}-k. \\
      \\
\glt
\z

dance-BEN-BNFY2.PA-3s

`He danced for him.'

\paragraph[Counterfactual]{Counterfactual}
\hypertarget{RefHeading20341935131865}{}
The counterfactual modality is the only modal distinction made in the verb morphology. It is an expression of the truth value of the statement: something could or would have happened, but did not, or something might be the case but for some reason is not. The counterfactual is marked by the suffix \nobreakdash-\textstyleStyleVernacularWordsItalic{ek} (\stepcounter{nx}{\thenx}) and is only used with the past tense suffix even if the verb refers to the present (\stepcounter{nx}{\thenx}) or future (\stepcounter{nx}{\thenx}) time\textstyleEmphasizedVernacularWords{\textmd{\textit{.}}} The counterfactual is used in both hypothetical and counterfactual conditional clauses (\sectref{sec:8.3.5}). 

\ea%x234
\label{ex:x234}
\gll Lawiliw  akena  waki-\textstyleEmphasizedVernacularWords{ek}-a-m. \\
      \\
\glt
\z

nearly  very  fall-CNTF-PA-1s  

`I very nearly fell.' 

\ea%x433
\label{ex:x433}
\gll Yena  aamun  aakisa  uruf-\textstyleEmphasizedVernacularWords{ek-a}-m=na  kemel-\textstyleEmphasizedVernacularWords{ek-a}-m. \\
      \\
\glt
\z

1s.GEN  yonger.brother  now  see-CNTF-PA-1s=TP  rejoice-CNTF-PA-1s

`If I saw my younger brother now, I would be happy.'

\ea%x434
\label{ex:x434}
\gll Morauta  fan  ik-\textstyleEmphasizedVernacularWords{ek-a}-k=na  uurika  ikiw-ep  \\
      \\
\glt
\z

Morauta  here  be-CNTF-PA-3s=TP  tomorrow  go-SS.SEQ  

maak-\textstyleEmphasizedVernacularWords{ek-a}-mik.

tell-CNTF-PA-1/3p

`If Morauta were here, tomorrow we would go and tell him.'

If there is a beneficiary suffix \textstyleStyleVernacularWordsItalic{-a} preceding the counterfactual, the vowel /e/ of the counterfactual suffix is deleted (\stepcounter{nx}{\thenx}):

\ea%x235
\label{ex:x235}
\gll Maifa  yia  aaw-om-\textstyleEmphasizedVernacularWords{ak}-a-k=na{\dots \\
      \\
\glt
\z

paper  1p.ACC  get-BEN-BNFY2.CNTF-PA-3s=TP  

`If he had gotten tickets for us{\dots}'

\paragraph[Mood]{Mood}
\hypertarget{RefHeading20361935131865}{}
Mood in Mauwake is defined as a morphological category of the verb, relating to the pragmatic function of the sentence (cf. Palmer 1986:21). Mauwake has a mixed tense-mood system, where the indicative present, past and future, and the imperative are in contrast. 

The mood distinctions only show in the finite verb. Same-subject medial verbs take the interpretation of their mood from the following finite verb, but different-subject medial verbs may be independent of the final verb as to their mood.

\subparagraph[Indicative]{Indicative}
\hypertarget{RefHeading20381935131865}{}
The indicative is the neutral, morphologically unmarked mood.  It is characterized by the tense distinctions between present, past and future, and the person/number distinctions of first, second, and third person in singular and plural. 

\ea%x693
\label{ex:x693}
\gll I  me  yia  damol-a-mik. \\
      \\
\glt
\z

1s.UNM  not  1s.ACC  harm-PA-1/3p

`They didn't harm us.'

\ea%x694
\label{ex:x694}
\gll Aria,  iperowa  opora  wiar  ook-i-yen. \\
      \\
\glt
\z

alright,  middle-aged  talk  3.DAT  follow-Np-FU.1p

`Alright, we'll follow the advice of the middle-aged men.'

\subparagraph[Imperative]{Imperative}
\hypertarget{RefHeading20401935131865}{}
The term imperative is used for ``mands'' (\textstyleBibliogBaseChar{Lyons 1977}:745)\footnote{Lyons borrows the term from B.F. Skinner as a useful cover term, without subscribing to Skinner's behaviouristic position.} showing in the verbal morphology, regardless of person. In Mauwake the imperatives form a full paradigm (with the first person singular being replaced with the first person dual), and their syntactic behaviour is similar.  So there is no valid reason to divide them into different categories such as imperatives, jussives and hortatives, just because semantically giving orders to oneself differs from giving orders to an addressee or to a third person.\footnote{For a discussion on this question, see Palmer 1986:109-111.}

There are no tense distinctions in the imperative forms.  The initial (or only) vowel in the second person imperatives is usually /e/, but in very few cases it is /a/.\footnote{The only verbs found with -\textstyleFootnoteBaseChar{\textit{a}}  in the imperative are \textstyleFootnoteBaseChar{\textit{iw}}- `go', \textstyleFootnoteBaseChar{\textit{mik}}- `spear, hit', \textstyleFootnoteBaseChar{\textit{op}}- `hold' and \textstyleFootnoteBaseChar{\textit{pok}}- `sit'.}





\begin{tabular}{ll}
\mytoprule


PERSON/NUMBER & \\
-u & 1d\\
-e  (-a) & 2s\\
-inok & 3s\\
-ikua & 1p\\
-eka  (-aka) & 2p\\
-uk & 3p\\
\mybottomrule
\end{tabular}



\begin{table}
\caption{Imperative suffixes}
\label{tab:12}
\end{table}

\ea%x229
\label{ex:x229}
\gll Or-op  mua  nain  uruf-\textstyleEmphasizedVernacularWords{e}. \\
      \\
\glt
\z

descend-SS.SEQ  man  that1  see-IMP.2s  

`Go down and see that man.'

\ea%x1847
\label{ex:x1847}
\gll Ikoka  amap-urup-eya  op-\textstyleEmphasizedVernacularWords{ikua}. \\
      \\
\glt
\z

later  Bpx-ascend-2/3s.DS  hold-IMP.1p

`Later when he comes up, let's hold/grab him.'

\ea%x230
\label{ex:x230}
\gll Wi  urup-ep  mukuna  nain  umuk-\textstyleEmphasizedVernacularWords{uk}. \\
      \\
\glt
\z

3p.UNM  ascend-SS.SEQ  fire  that1  extinguish-IMP.3p

`Let them go up and extinguish the fire.' 

The imperative differs from the other moods in that it has a dual form in the first person but no singular: 

\ea%x446
\label{ex:x446}
\gll Aria,  i  owowa=ko  or-\textstyleEmphasizedVernacularWords{u}. \\
      \\
\glt
\z

alright,  1p.UNM  village=NF  descend-IMP.1d

`Alright, let's (d.) go down to the village.'

\ea%x1196
\label{ex:x1196}
\gll Yiena  ikos  akena  iw-\textstyleEmphasizedVernacularWords{u}. \\
      \\
\glt
\z

1p.GEN  two.together  truly  go-IMP.1d

`Let's just the two of us go together.'

The initial (or only) vowel in the second person imperative forms is deleted after the beneficiary suffix (\stepcounter{nx}{\thenx})-(\stepcounter{nx}{\thenx}).  

\ea%x432
\label{ex:x432}
\gll Iwera  ir-\textstyleEmphasizedVernacularWords{e.} \\
      \\
\glt
\z

coconut  ascend-IMP.2s

`Climb up the coconut palm (to get coconuts).'

\ea%x431
\label{ex:x431}
\gll Iwera  ir-om-\textstyleEmphasizedVernacularWords{e}. \\
      \\
\glt
\z

coconut  ascend-BEN-BNFY1.IMP.2s

`Climb up the coconut palm for me.'

\ea%x232
\label{ex:x232}
\gll Iwera  ir-om-\textstyleEmphasizedVernacularWords{a.} \\
      \\
\glt
\z

coconut  ascend-BEN-BNFY2.IMP.2s  

`Climb the coconut for him.'

\ea%x233
\label{ex:x233}
\gll Iwera  \textstyleEmphasizedVernacularWords{yia}  ir-om-\textstyleEmphasizedVernacularWords{a}ka. \\
      \\
\glt
\z

coconut  1p.ACC  ascend-BEN-BNFY2.IMP.2p  

`Climb (plural) the coconut for us.'

The semantics of the imperative and the functional aspects of commands are discussed in \sectref{sec:7.3}. On the use of subject pronouns with imperatives, see \sectref{sec:3.5.11}. The imperative forms of the verbs are also used in desiderative (\sectref{sec:8.3.2.1.3}) and conative (\sectref{sec:8.3.2.1.5}) constructions. 

\paragraph[Tense and person/number in final verbs]{Tense and person/number in final verbs}
\hypertarget{RefHeading20421935131865}{}
Tense is a ``\textstyleBibliogCitationAAAstyleChar{grammaticalized expression of  location in time}'' \citep[9]{Comrie1985}. Mauwake has a straightforward three-tense system in the finite verbs marking past, present and future time reference. The tense system is simple compared with most other Papuan languages, many of which have more than three genuine tense distinctions and/or interaction between tense and status\footnote{`Status' here refers to the distinction between realis and irrealis.} resulting in several ``tenses'' (\textstyleBibliogBaseChar{Foley 1986}:158-63). Of the most closely related languages well studied so far, Usan has five tenses (\textstyleBibliogBaseChar{Reesink 1987}:98) out of which one, uncertain future/subjunctive, is semantically related to irrealis. Maia has a complete status system instead of a tense system, and temporal relations are inferred from the realis or irrealis status and the aspects (\textstyleBibliogBaseChar{Hardin 2002}:55). According to \textstyleBibliogBaseChar{Foley}, \textstyleBibliogCitationAAAstyleChar{``most Papuan languages are tense-dominated [rather than status-dominated]''} (1986:162). In Mauwake the status hardly plays any role at all.

Portmanteau morphemes of the tense and person/number markers are very common in Papuan languages, but having the two distinct from each other is not uncommon either (\textstyleBibliogBaseChar{Foley 1986}:137). The tense and person suffixes are separate morphemes in Mauwake, but have an interesting interplay with each other.

The speech event is taken as the reference point. The tense suffixes in themselves only distinguish between two tenses, past and non-past, and the further distinction between present and future is made by the person/number suffixes.  The person/number suffixes, on the other hand, are the same in past and present tense except for the first and third person singular forms. 

\ea%x1029
\label{ex:x1029}
\gll Unan  \textstyleEmphasizedVernacularWords{aakun-e-mik},  aakisa  \textstyleEmphasizedVernacularWords{aakun-i-mik}  ne    \\
      \\
\glt
\z

yesterday  talk-PA-1/3p  now/today  talk-Np-PR.1/3p  ADD  

uurika  nainiw  \textstyleEmphasizedVernacularWords{aakun-i-yen}.

tomorrow  again  talk-Np-FU.1p

`Yesterday we talked, now/today we talk and tomorrow we'll talk again.'

The non-past marker in the second person plural future form is \textstyleEmphasizedVernacularWords{\nobreakdash-}\textstyleStyleVernacularWordsItalic{o}  instead of \textstyleEmphasizedVernacularWords{\nobreakdash-}\textstyleStyleVernacularWordsItalic{i}  possibly because of  assimilation to the labial consonant /w/ in the person/number suffix. 

The verb conjugation classes determining the past tense suffix vowels are discussed in the section on morphophonology (\sectref{sec:2.3.3.3}). The beneficiary and the counterfactual suffixes influence the past tense suffix in the following way. After the counterfactual the past tense suffix is always \textstyleEmphasizedVernacularWords{\nobreakdash-}\textstyleStyleVernacularWordsItalic{a}. When the beneficiary suffix is present, the vowel of the past tense suffix is assimilated to it.

The following table presents the full paradigms for the tense and person/number suffixes.





\begin{tabular}{llll}
\mytoprule
 & \raggedleft Non-past  -  Present\& {\quad}{\quad}person  \par & \raggedleft Non-past  -  Future\& person\par & Past    -    person\\
1s & -i    -yem & -i    -nen & -a/E    -m\\
2s & -i    -n & -i    -nan & -a/E    -n\\
3s & -i    -ya & -i    -non & -a/E    -k\\
1p & -i    -mik & -i    -yen & -a/E    -mik\\
2p & -i    -man & -o    -wen & -a/E    -man\\
3p & -i    -mik & -i    -kuan & -a/E    -mik\\

\mybottomrule 
\end{tabular}



\begin{table}
\caption{Tense and person/number suffixes}
\label{tab:13}
\end{table}

The person/number marking in the verb distinguishes three persons in both singular and plural. There is no dual number, nor is there inclusive-exclusive distinction in the first person plural form. The plural is marked only for humans, spirits and important animals. The singular form is used for less important and small animals as well as all inanimates:

\ea%x236
\label{ex:x236}
\gll Waa  muuka  arow  ekap-o-\textstyleEmphasizedVernacularWords{k}. \\
      \\
\glt
\z

pig  boy  three  come-PA-3s  

`Three piglets came.'

Besides their primary meaning, the present and future tenses also have secondary meanings. The present tense form of the first or third person plural is used for generic or time-neutral statements (\stepcounter{nx}{\thenx}). For the habitual aspect in the present, the simple present tense (\stepcounter{nx}{\thenx}) is an alternative to the full habitual aspect form.\footnote{Continuous aspect form is required for the past habitual (\sectref{sec:3.8.5.1.1.2}).} 

\ea%x1034
\label{ex:x1034}
\gll Ifa  yia  keraw-i-ya  nain  miira  \textstyleEmphasizedVernacularWords{saawirin-i-mik}. \\
      \\
\glt
\z

snake  1p.ACC  bite-Np-PR.3s  that1  face  become.round-Np-PR.1/3p

 `When a snake bites us, we become dizzy.'

\ea%x1035
\label{ex:x1035}
\gll Nos=ke  anane  urema  efar  \textstyleEmphasizedVernacularWords{ikum-ar-i-n}. \\
      \\
\glt
\z

2s.FC=CF  always  bandicoot  1s.DAT  illicitly-INCH-Np-PR.2s

`You always steal bandicoots from me.'

The future tense in any person form is used for habitual or generic conditionals (8.3.5). The example (\stepcounter{nx}{\thenx}) refers to a traditional custom and is generic, even if the first person form of the verb is used, and the first person pronoun as well.

\ea%x1640
\label{ex:x1640}
\gll Waaya  \textstyleEmphasizedVernacularWords{ika-i-non},  waaya  \textstyleEmphasizedVernacularWords{uup-i-nan},  naap. \\
      \\
\glt
\z

pig  be-Np-FU.3s  pig  cook-Np-FU.2s  thus

`If there is a pig, you will cook it - it is like that.'

\ea%x1641
\label{ex:x1641}
\gll Ikoka  yo  \textstyleEmphasizedVernacularWords{um-i-nen},  muuka  nain  nainiw  wiena  \\
      \\
\glt
\z

later  1s.UNM  die-Np-FU.1s  son  that1  again  3p.GEN

\textstyleEmphasizedVernacularWords{aaw-i-kuan}.

take-Np-FU.3p

`Later if I die (without paying the bride price) they will take the son back.'

The second person singular form of the future tense has two other usages as well. It can be used when referring to generic or habitual  situations,  especially in process descriptions which can also be understood as instructions. For these, the first person plural form of the present tense is much more common, but often the two alternate. The following example describes work involved in harvesting taro roots (and the addressee that the story was told to, had no garden, so the speaker did not refer to her personally). 

\ea%x1038
\label{ex:x1038}
\gll Perek-ami  en-ow(a)  gelemuta  \textstyleEmphasizedVernacularWords{on-i-nan}. \\
      \\
\glt
\z

harvest-SS.SIM  eat-NMZ  little  make-Np-FU.2s

`When you harvest it you make a little feast.'

It is also used for a command, or a statement of obligation:

\ea%x1039
\label{ex:x1039}
\gll Ikoka  kuisow  kuuma  kuisow  \textstyleEmphasizedVernacularWords{yi-i-nan}. \\
      \\
\glt
\z

later  one  stick  one  give.me-Np-FU.2s

`Very soon you have to give me 10 kina.' Or: `Give me 10 kina very soon.'

\paragraph[Medial verb marking]{Medial verb marking}
\hypertarget{RefHeading20441935131865}{}
The distinction between medial and final verbs is common in Papuan languages (\textstyleBibliogBaseChar{Foley 1986}:11). Especially in the \textstyleAcronymallcaps{TNG} languages the medial verbs are ``\textstyleBibliogCitationAAAstyleChar{very common, universal over a wide area [and the] systems often highly complex}'' (\textstyleBibliogBaseChar{Wurm 1982}:63, also Roberts 1997). The medial verbs typically lack the full tense and person/number marking of the finite verbs. Instead, they usually indicate whether the subject is the same as the subject of the following verb, and/or whether the action of the verb is simultaneous or sequential with the action of the following verb.\footnote{This is called the ``switch-reference system''. \textstyleBibliogBaseChar{The question whether} the system really tracks the topic (pragmatic subject) or the syntactic subject  is discussed further in 8.2.3.} As for person reference, the verbs in the simplest systems only show whether the two subjects are the same or different, but in the most elaborate systems the subjects of both the medial verb and that of the following verb are shown in the medial verb, which is thus even more specific than the finite verb.\footnote{Usan makes a distinction between neutral and future medial verbs, and in both of these there is a division between same-subject and different-subject forms, but not between sequentiality and simultaneity \textstyleBibliogBaseChar{(Reesink 1987}:87-92). Maia only uses medial verbs when a clause has the same subject as the following clause; a distinction is made between simultaneous and sequential actions. When the following clause has a different subject, finite forms plus the contrast clitic \textstyleFootnoteBaseChar{\nobreakdash-(}\textstyleFootnoteBaseChar{\textit{d)i}} is used (\textstyleBibliogBaseChar{Hardin 2002}:87). Amele makes the basic distinction between the same-subject and different-subject medial verbs, and has simultaneous and sequential forms in both. But it also has different-subject simultaneous irrealis forms (\textstyleBibliogBaseChar{Roberts 1987}:275). Particularly the East New Guinea Highlands languages are known for marking the anticipatory subject in their medial verbs. See \citet[40-41]{Franklin1983} for a succinct list of switch-reference characteristics in Papuan languages, and \citet{Roberts1997} for a more comprehensive overview} 

In Mauwake the medial verb system is relatively simple. The suffixes indicate whether the subject of the medial verb stays the same in the following verb as well, and in the ``same subject following'' (\textstyleAcronymallcaps{SS}) verbs there is a further distinction between simultaneous and sequential action. The ``different subject following'' (\textstyleAcronymallcaps{DS}) verbs indicate sequential action; for simultaneous action one needs to use the continuous (\sectref{sec:3.8.5.1.1.2}) or stative aspect (\sectref{sec:3.8.5.1.1.3}). The \textstyleAcronymallcaps{DS} verbs also have some person marking but not as detailed as the finite verbs have.

The two sections below give a general outline of the person reference in medial clauses, but it is discussed in more detail in \sectref{sec:8.2.3}.

Typically, medial verbs have much fewer inflectional possibilities than finite verbs \citep[11]{Foley1986}. This is the case in Mauwake too: mood or tense and full person/number marking cannot be suffixed to the medial verbs. Derivational suffixes, on the other hand, can freely occur on the medial verbs. In Tail-Head linkage a new sentence begins with a medial verb copy of the finite verb that ended the previous sentence (8.2.3.5). Often the derivational morphology of the two verbs is the same, and sometimes the medial verb has less derivation than the final verb; very rarely it has even \textstyleEmphasizedWords{\textsc{more}} (\stepcounter{nx}{\thenx}):

\ea%x237
\label{ex:x237}
\gll Ikiwosa  wiar  pepekim-ep  kaik-a-m.  Kaik-\textstyleEmphasizedVernacularWords{om}-\textbf{a}p{\dots} \\
      \\
\glt
\z

head  3.DAT  measure-SS.SEQ  tie-PA-1s  tie-BEN-BNFY2.SS.SEQ

`I measured her head and tied it (=headdress). I tied it for her and {\dots}'

\subparagraph[Same-subject marking]{Same-subject marking}
\hypertarget{RefHeading20461935131865}{}
When the subject of the medial clause is the same as that of the following clause, the verb itself does not give any indication of the person and number of the subject, only that the same subject continues in the next clause.\footnote{For exceptions to this, see \sectref{sec:8.2.3} where the functional aspects of switch reference are discussed.} If the actions are sequential, i.e. the action indicated by the verb in the medial clause precedes that of the following clause, the suffix is \nobreakdash-\textstyleStyleVernacularWordsItalic{ap}  or \nobreakdash-\textstyleStyleVernacularWordsItalic{ep} (\stepcounter{nx}{\thenx}) depending on the conjugation class (\sectref{sec:2.3.3.3}).\footnote{For the second conjugation class verb \textstyleFootnoteBaseChar{\textit{or}}\textit{-} `descend' the suffix is -\textstyleFootnoteBaseChar{\textit{op}}\textstyleEmphasizedVernacularWords{.}}  

\ea%x238
\label{ex:x238}
\gll Owowa  ek-\textstyleEmphasizedVernacularWords{ap},  wailal-\textstyleEmphasizedVernacularWords{ep}  akia  ik-e-k. \\
      \\
\glt
\z

village  go-SS.SEQ  be.hungry-SS.SEQ  banana  roast-PA-3s  

`He went to the village, was hungry and roasted bananas.' 

If the verb has the beneficiary suffix \textstyleStyleVernacularWordsItalic{-a} or \textstyleStyleVernacularWordsItalic{-e} (\sectref{sec:3.8.3.1}), the vowel of the medial verb suffix gets assimilated to it (\stepcounter{nx}{\thenx}), (\stepcounter{nx}{\thenx}).

\ea%x1930
\label{ex:x1930}
\gll {\dots}eka=pa  merena  yasuw-om-\textstyleEmphasizedVernacularWords{e}p...  (cf.  yasuw-\textstyleEmphasizedVernacularWords{a}p) \\
      \\
\glt
\z

water=LOC  foot  wash-BEN-BNFY.1-SS.SEQ

`{\dots} she washed my feet in water (and) {\dots}'

\ea%x1929
\label{ex:x1929}
\gll ...waaya  nain  uup-om-\textstyleEmphasizedVernacularWords{a}p  samapora=pa \\
      \\
\glt
\z

pig  that1  cook-BEN-BNFY2.SS.SEQ  floor=LOC  

wu-ap  maak-e-mik...  (cf.  uup-\textstyleEmphasizedVernacularWords{e}p)

put-SS.SEQ  tell-PA-1/3p

`{\dots} they cooked the pig for him, put it on the floor and told him, {\dots}'

When the medial clause subject is the same as the subject in the following clause but the two actions are simultaneous, or at least overlapping, the suffix is \nobreakdash-\textstyleStyleVernacularWordsItalic{ami} or \nobreakdash-\textstyleStyleVernacularWordsItalic{emi}  (or \nobreakdash-\textstyleStyleVernacularWordsItalic{omi}) according to the conjugation class of the verb. Even if the action of the medial verb may often be \textstyleEmphasizedWords{\textsc{interpreted}} as continuous (\stepcounter{nx}{\thenx}), the suffix in itself only indicates simultaneity (\stepcounter{nx}{\thenx}). Continuous aspect marking may be needed for clarity when continuity is in focus (\stepcounter{nx}{\thenx}).

\ea%x239
\label{ex:x239}
\gll Wi  sawur  ir-\textstyleEmphasizedVernacularWords{ami}  fan  yiar  pok-a-mik. \\
      \\
\glt
\z

3p.UNM  spirit  go-SS.SIM  here  1p.DAT  sit.down-PA-1/3p  

`As the spirits were going they sat down here with us.' 

\ea%x240
\label{ex:x240}
\gll {\dots}ekap-\textstyleEmphasizedVernacularWords{emi}  koora=pa  yia  wua-i-mik.  \\
      \\
\glt
\z

come-SS.SIM  house=LOC  1p.ACC  put-Np-PR.1/3p  

`{\dots}coming (=upon arrival) they put us in the house.' 

\ea%x241
\label{ex:x241}
\gll \textstyleEmphasizedVernacularWords{Soomar-em-ik}\textstyleEmphasizedVernacularWords{-ok}  ifara  oko  uruf-a-k.  \\
      \\
\glt
\z

walk-SS.SIM-be-SS  vine  other  see-PA-3s  

`He was walking and saw another vine. '

The verb \textstyleStyleVernacularWordsItalic{ik}- `be' is different from other verbs in that there is no differentiation between the simultaneous and sequential forms in the same-subject medial verb: in example (\stepcounter{nx}{\thenx}) the actions are simultaneous, in (\stepcounter{nx}{\thenx}) sequential. Also, the verb does not take either one of the normal same-subject suffixes.

\ea%x242
\label{ex:x242}
\gll Owowa=pa  neeke  \textstyleEmphasizedVernacularWords{ik-ok}  mua  maak-ek{\dots} \\
      \\
\glt
\z

village=LOC  there.CF  be-SS  man  tell-PA-3s  

`While they were there in the village she told her husband, {\dots}'

\ea%x243
\label{ex:x243}
\gll No  kaaneke  \textstyleEmphasizedVernacularWords{ik-ok}  kerer-e-n? \\
      \\
\glt
\z

2s.UNM  where.CF  be-SS  appear-PA-2s  

`Where have you been and now come?' 

\subparagraph[Different-subject marking]{Different-subject marking}
\hypertarget{RefHeading20481935131865}{}
If the subject of the medial clause is different from that of the following clause, the suffix of the \textstyleAcronymallcaps{DS} verb reflects this. There are some person/number distinctions in these suffixes, even though not as many as in the finite verbs. The first person singular and plural are distinguished from all the other forms; in the other persons the distinction is based on the number: second and third person singular share the same suffix, and second and third person plural likewise.\footnote{There is great variation in this area in Papuan languages. Some only have one form to indicate that the subject changes, others have partial or full differentiation according to the person, some even show the subject of the following clause in the medial verb.}  





\begin{tabular}{lll}
\mytoprule
 & \multicolumn{1}{l}{Singular}

 & Plural\\
1 & \multicolumn{2}{l}{-Vmkun}

\\
2 & \multicolumn{1}{l}{-eya}

 & -iwkin\\
\multicolumn{1}{l}{3}

 &  & \\
\hhline{-~~}

\mybottomrule
\end{tabular}



\begin{table}
\caption{Suffixes marking a different subject in the following clause}
\label{tab:14}
\end{table}

\ea%x244
\label{ex:x244}
\gll Imen-ap  maak-\textstyleEmphasizedVernacularWords{iwkin}  o  miim-o-k. \\
      \\
\glt
\z

find-SS.SEQ  tell-2/3p.DS  3s.UNM  precede-PA-3s  

`They found him and told him, and he went ahead of them.' 

\ea%x245
\label{ex:x245}
\gll Mik-\textstyleEmphasizedVernacularWords{amkun}  me  um-o-k,  wiowa  onaiya  ikiw-em-ik-\textstyleEmphasizedVernacularWords{eya} \\
      \\
\glt
\z

spear-1s/p.DS  not  die-PA-3s  spear  with  go-SS.SIM-be-2/3s.DS  

Olas=ke  war-ek.

Olas=CF  shoot-PA-3s

`When I speared it, it didn't die, (but) as it was going with the spear Olas shot it.'

The suffix  is -\textstyleStyleVernacularWordsItalic{aya} instead of \nobreakdash-\textstyleStyleVernacularWordsItalic{eya}  in a few short conjugation class 1 verbs (\sectref{sec:3.8.4.1}) (\stepcounter{nx}{\thenx})\footnote{The vowel \textstyleFootnoteBaseChar{\textit{-a}}  is somewhat more common in the Muaka dialect group where the 2/3s.DS marker is -\textit{era} instead of \nobreakdash-\textit{eya.}} and in those benefactive verbs where the first vowel of the suffix is assimilated to the preceding vowel of the beneficiary suffix (\stepcounter{nx}{\thenx}). 

\ea%x493
\label{ex:x493}
\gll Iw-\textstyleEmphasizedVernacularWords{aya}  nan  miira  saawirin-e-k. \\
      \\
\glt
\z

enter-2/3s.DS  there  face  become.round-PA-3s

`As [the poison] entered [his liver], he became dizzy.'

\ea%x695
\label{ex:x695}
\gll Aaya=ko  yia  aaw-om-\textstyleEmphasizedVernacularWords{aya}  enim-i-yan.  \\
      \\
\glt
\z

sugarcane=NF  1p.ACC  get-BEN-BNFY2.2/3s.DS  eat-Np-FU.1p

`Get us sugarcane and we'll eat it.'

The different-subject marking \nobreakdash-\textstyleStyleVernacularWordsItalic{eya} is also used with some non-verbs. This seems to be uncommon in PNG languages: in Roberts' (1997:137) survey the very few examples where the switch-reference marking was on non-verbs these were pro-clausal substitutes like a demonstrative or vocative. In Mauwake the \textstyleAcronymallcaps{DS} suffix can be added to nouns (\stepcounter{nx}{\thenx}) or adjectives (\stepcounter{nx}{\thenx}), or the negative adverbs \textstyleStyleVernacularWordsItalic{weetak} and \textstyleStyleVernacularWordsItalic{marew} (\stepcounter{nx}{\thenx}) functioning as predicates in verbless clauses. When it is added to words ending in -\textstyleStyleVernacularWordsItalic{a}, the first vowel in the suffix gets assimilated to this vowel (\stepcounter{nx}{\thenx}).

\ea%x250
\label{ex:x250}
\gll Enakiwa-\textstyleEmphasizedVernacularWords{ya}  me  aaw-e-m. \\
      \\
\glt
\z

half-2/3s.DS  not  take-PA-1s  

`There was (only) half (left), so I didn't take any/it.'

\ea%x251
\label{ex:x251}
\gll Mauwow  maneka-\textstyleEmphasizedVernacularWords{ya}=na  yia  maak-i-non. \\
      \\
\glt
\z

work  big-2/3s.DS=TP  1p.ACC  tell-Np-FU.3s  

`If the work is big, she will tell us.'

\ea%x252
\label{ex:x252}
\gll Soomia  marew-\textstyleEmphasizedVernacularWords{eya}  amap-ep-om-a-m. \\
      \\
\glt
\z

spoon  none-2/3s.DS  BPx-come-BEN-BNFY2.PA-1s  

`She has/had no spoons (lit: there are/were no spoons) so I brought them to her.'

When the different subject marking \nobreakdash-\textstyleStyleVernacularWordsItalic{eya} is added to the adverb \textstyleStyleVernacularWordsItalic{naap} `thus', the outcome is a consecutive connective `therefore, so' (\sectref{sec:3.11.2}).

\subparagraph[Tense and medial verbs]{Tense and medial verbs}
\hypertarget{RefHeading20501935131865}{}
The medial verbs have no tense marking, so the tense in a medial clause is interpreted in relation to that of the next finite clause. When the finite clause is in the past tense, both a simultaneous (\stepcounter{nx}{\thenx}) and a sequential medial clause (\stepcounter{nx}{\thenx}) are also understood to be in the past tense.

\ea%x1025
\label{ex:x1025}
\gll Iwera  uruk-am-ika-iwkin  wi  \textstyleEmphasizedVernacularWords{ikiw-emi} \\
      \\
\glt
\z

coconut  drop-SS.SIM-be-2/3p.DS  3p.UNM  go-SS.SIM

\textstyleEmphasizedVernacularWords{aaw-em-ik-e-mik}.

take-SS.SIM-be-PA-1/3p

`They\textsubscript{i} kept dropping coconuts, and they\textsubscript{j} went and got them.'

\ea%x1023
\label{ex:x1023}
\gll Owowa  \textstyleEmphasizedVernacularWords{or-op,  wailal-ep},  akia  \textstyleEmphasizedVernacularWords{ik-e-k}. \\
      \\
\glt
\z

village  descend-SS.SEQ  be.hungry-SS.SEQ  banana  roast-PA-3s

`He came down to the village, was hungry and roasted bananas.'

Since a sequential verb indicates that the action takes place before another action, a sequential medial clause preceding a present tense final clause has to be interpreted to be in the past tense, whereas a simultaneous clause is interpreted to be in the present tense like the final verb.

\ea%x1024
\label{ex:x1024}
\gll Iperuma  nain=ke  mua  \textstyleEmphasizedVernacularWords{puuk-ap}  owora  \textstyleEmphasizedVernacularWords{en-emi} \\
      \\
\glt
\z

eel  that1=CF  man  become-SS.SEQ  betelnut  eat-SS.SIM

afura  \textstyleEmphasizedVernacularWords{buan-em-ika-i-ya}.

lime.container  knock-SS.SIM-be-Np-PR.3s

`The eel has become man, and is eating betelnut and knocking the lime container.'

Both a sequential and a simultaneous medial clause preceding a future final clause are also understood as future clauses. The action in a sequential medial clause takes place before that in the final clause, but it is still in the future (\stepcounter{nx}{\thenx}). The action in a simultaneous clause is partly or fully overlapping with that in the final clause (\stepcounter{nx}{\thenx}).

\ea%x1026
\label{ex:x1026}
\gll Is=ke  maa  uup-emkun  wi  \textstyleEmphasizedVernacularWords{ekap-ep}  \textstyleEmphasizedVernacularWords{enim-i-kuan}. \\
      \\
\glt
\z

1p.FC=CF  food  cook-1s/p.DS  3p.UNM  come-SS.SEQ  eat-Np-FU.3p

`We'll cook the food and they'll come and eat it.' Or: `When we have cooked the food they will come and eat it.'

\ea%x1027
\label{ex:x1027}
\gll Wi  \textstyleEmphasizedVernacularWords{ir-ami}  nia  \textstyleEmphasizedVernacularWords{aaw-emi}  efa  \textstyleEmphasizedVernacularWords{ifakim-i-kuan}. \\
      \\
\glt
\z

3p.UNM  come-SS.SIM  2p.ACC  take-SS.SIM  1s.ACC  kill-Np-FU.3p

`They will come and take you and kill me.'

The medial verb form cannot be used in the following example, because the first verb refers to time preceding the speech event and the second verb to time following it. Final verbs with different tenses have to be used, and in this case it is most natural to place the past tense verb in a relative clause: 

\ea%x1030
\label{ex:x1030}
\gll Mukuna  kerer-e-k  nain  kamenap  umuk-i-yan? \\
      \\
\glt
\z

fire  appear-PA-3s  that1  how  extinguish-Np-FU.1p

`How shall we extinguish the fire that started?'

The medial verbs acquire more absolute-relative tense character of ``past in the past'' \citep[65]{Comrie1985} in those cases where sequential medial clauses are either right-dislocated and placed after the final clause (\stepcounter{nx}{\thenx}) or placed inside another medial clause (\stepcounter{nx}{\thenx}), or when there is a separate time expression referring to earlier time than that indicated by the final verb (\stepcounter{nx}{\thenx}). 

\ea%x1031
\label{ex:x1031}
\gll Rubaruba  nain=ke  ona  emeria  nain  aaw-ep  p-ikiw-o-k, \\
      \\
\glt
\z

Rubaruba  that1=CF  3s.GEN  woman  that1  take-SS.SEQ  BPx-go-PA-3s

\textstyleEmphasizedVernacularWords{iw-iwkin}.

give.him-2/3p.DS

`That Rubaruba took his wife and took her (away), when they had given her to him.'

\ea%x1032
\label{ex:x1032}
\gll Um-eya  merena  ere-erup  [\textstyleEmphasizedVernacularWords{ifara  aaw-ep}]  kaik-ap  \\
      \\
\glt
\z

die-2/3s.DS  leg  RDP-two  vine  get-SS.SEQ  tie-SS.SEQ

nabena  suuw-ap  akua  aaw-ep  or-o-m.

carrying.pole  push-SS.SEQ  shoulder  take-SS.SEQ  descend-PA-1s

 `It died, and I tied its legs in pairs with a vine that I had gotten, and pushed it to the carrying pole and carried it down on my shoulder.

\ea%x1033
\label{ex:x1033}
\gll \textstyleEmphasizedVernacularWords{Iiriw}  inasin  mua  nain=ke  naap  wia  \textstyleEmphasizedVernacularWords{maak-eya} \\
      \\
\glt
\z

earlier  spirit  man  that=CF  thus  3p.ACC  tell-2/3p.DS

wi  naap  on-a-mik.

3p.UNM  thus  do-PA-1/3p

 `The spirit man had earlier told them like that and they did so.'

\subsubsection{Verb classes} 
\hypertarget{RefHeading20521935131865}{}
Verbs can be divided into classes on the basis of various criteria.  Conjugation classes based on morphological/inflectional criteria are usually arbitrary and unrelated to other parts of the grammar \citep[191]{Anderson1985b}. They are only touched on briefly in the next subsection. Transitivity as a basis of verb classes is discussed in \sectref{sec:3.8.4.2}, and valence-changing operations in \sectref{sec:3.8.4.3}.  Verb classes based on semantic features are described in \sectref{sec:3.8.4.4}.

\paragraph[Conjugation classes]{Conjugation classes}
\hypertarget{RefHeading20541935131865}{}
In the Mauwake lexicon the verbs are divided into classes 1 and 2 depending on whether they have /a/ or /e\~{o}/ as the past tense suffix. There are morphophonological rules for deriving the past tense marking for most verbs (see \sectref{sec:2.3.3.3}), but since some of the rules are rather complicated, and because they do not cover a number of cases like (\stepcounter{nx}{\thenx}) and (\stepcounter{nx}{\thenx}) below, the division into two separate classes is maintained.

\ea%x253
\label{ex:x253}
\gll miim-\textstyleEmphasizedVernacularWords{a}-k \\
      \\
\glt
\z

hear-PA-3s 

`he heard'

\ea%x254
\label{ex:x254}
\gll miim-\textstyleEmphasizedVernacularWords{o}-k \\
      \\
\glt
\z

precede-PA-3s 

`he went ahead'

In Class 1, transitive verbs outnumber intransitive verbs over four times, but Class 2 is divided almost equally between transitive and intransitive verbs.\footnote{For this count, the verbs formed with the verbalizer \textstyleFootnoteBaseChar{\textit{--ar}} and the causative \textstyleFootnoteBaseChar{\textit{--ow}} were deleted from the total of 857 verbs, since both these suffixes influence the choice of the past tense vowel.}

\ea%x255
\label{ex:x255}
\gll puuk-\textstyleEmphasizedVernacularWords{a}-k    vs.      puk-\textstyleEmphasizedVernacularWords{o}-k  \\
      \\
\glt
\z

cut-PA-3s          burst-PA-3s  

`he cut (it)'          `it burst' 

\ea%x256
\label{ex:x256}
\gll teek-\textstyleEmphasizedVernacularWords{a}-k    vs.      ten-\textstyleEmphasizedVernacularWords{e}-k  \\
      \\
\glt
\z

pluck-PA-3s        collapse-PA-3s

`he plucked (it)'      `it collapsed'  (also: `it broke away')

\paragraph[Verb classes based on transitivity]{Verb classes based on transitivity}
\hypertarget{RefHeading20561935131865}{}
With the term \textstyleEmphasizedWords{\textsc{transitivity}} of a verb I refer to its \textstyleEmphasizedWords{\textsc{syntactic}} transitivity, i.e. ``\textstyleBibliogCitationAAAstyleChar{the number of overt morpho-syntactically coded arguments it takes}'' (Van Valin and LaPolla 1997:147). 

Intransitive verbs in Mauwake only require a subject, whereas transitive verbs require a direct object as well. This definition differs slightly from that of \citet[397]{Crystal1997}, who defines as transitive verbs those that \textstyleEmphasizedWords{\textsc{can}} take a direct object, and as intransitive those that \textstyleEmphasizedWords{\textsc{cannot}} (emphasis mine). Crystal's definition works for Mauwake when considering prototypical patient/undergoer objects, but it fails in the cases where the syntactic object manifests other roles not required by the semantic structure of the verb.\footnote{The syntactic transitivity of a verb can differ from both its semantic and macrorole transitivity (Van Valin and La Polla 1997).}  

In some languages verb roots can be neutral as to transitivity (Kittil\"a 2002:53), but in Mauwake each verb has a basic transitivity value. Most verbs are either intransitive (\sectref{sec:3.8.4.2.1}) or transitive (\sectref{sec:3.8.4.2.2}). There are only a few ambitransitives (\sectref{sec:3.8.4.2.3}). Mauwake does not have a regular class of ditransitive verbs that would require two objects. Instead, some verbs that are transitive easily allow a second object. And in the small class of the object cross-referencing verbs (\sectref{sec:3.8.4.2.4}), in which the pronominal object is in the verb root, two of the verbs require a second object as well. 

The basic transitivity of a verb can be changed with valence-changing strategies (\sectref{sec:3.8.4.3}). Causative (\sectref{sec:3.8.2.3.1}) and benefactive morphology (\sectref{sec:3.8.2.3.3}) as well as  possessor raising (\sectref{sec:5.3.2.3}) are processes that increase the number of syntactic objects in a clause.

\subparagraph[Intransitive verbs]{Intransitive verbs}
\hypertarget{RefHeading20581935131865}{}
In Mauwake the class of basic, or ``ordinary'', intransitive verbs consists of a semantically very diverse group including involuntary processes (\stepcounter{nx}{\thenx}), many motion verbs (\stepcounter{nx}{\thenx}), and some bodily function verbs (\stepcounter{nx}{\thenx}).

\ea%x266
\label{ex:x266}
\gll Fikera  \textstyleEmphasizedVernacularWords{aw-o-k}. \\
      \\
\glt
\z

kunai.grass  burn-PA-3s  

`The kunai grass burned.'

\ea%x267
\label{ex:x267}
\gll Kuuten  ikos  \textstyleEmphasizedVernacularWords{karu-e-mik}. \\
      \\
\glt
\z

Kuuten  with  run-PA-1/3p  

`I ran with Kuuten.'

\ea%x269
\label{ex:x269}
\gll Niir-emi  \textstyleEmphasizedVernacularWords{pisi-e}\textstyleEmphasizedVernacularWords{-k}. \\
      \\
\glt
\z

laugh-SS.SIM  fart-PA-3s  

`He laughed and farted.'

Some experience verbs expressing physiological states are also regular intransitive verbs:

\ea%x1485
\label{ex:x1485}
\gll Maa  enowa  nopa-yiaw-ep  \textstyleEmphasizedVernacularWords{wailal-ep} \\
      \\
\glt
\z

food  eat-NMZ  search-move.around-SS.SEQ  get.hungry-SS.SEQ

ma-e-mik, ``...''

say-PA-1/3p

`They searched  around for food and got hungry and said, ``{\dots}'' '

The verbs derived with the inchoative suffix \nobreakdash-\textstyleStyleVernacularWordsItalic{ar}  (\sectref{sec:3.8.2.2.2}) are mostly intransitive, but a few of them are transitive (\stepcounter{nx}{\thenx}). 

\ea%x271
\label{ex:x271}
\gll Uuw-ap  uuw-ap  \textstyleEmphasizedVernacularWords{lebum-ar-e-m}. \\
      \\
\glt
\z

work-SS.SEQ  work-SS.SEQ  lazy-INCH-PA-1s  

`I worked and worked and got tired.'

\ea%x1486
\label{ex:x1486}
\gll Nan  teeria  \textstyleEmphasizedVernacularWords{manek-ar-e-k},  owowa  pun  manek-ar-e-k. \\
      \\
\glt
\z

there  family  big-INCH-PA-3s  village  also  big-INCH-PA-3s

`The family grew big there, and the village grew big too.'

\ea%x1836
\label{ex:x1836}
\gll Maa  unowa  oram  me  \textstyleEmphasizedVernacularWords{amis}\textstyleEmphasizedVernacularWords{-}\textstyleEmphasizedVernacularWords{ar}\textstyleEmphasizedVernacularWords{-}\textstyleEmphasizedVernacularWords{i}\textstyleEmphasizedVernacularWords{-}\textstyleEmphasizedVernacularWords{mik},  weetak. \\
      \\
\glt
\z

thing  many  just  not  knowledge-INCH-Np-PR.1/3p  no

`We don't just gain knowledge of many things (without learning them), no.'

Climate expressions often use intransitive verbs. There is no separate class of verbs for climate expressions.\footnote{Climate expressions also use directional verbs (\textstyleFootnoteBaseChar{\textit{ipia oraiya} }`the rain descends'), inchoative verbs (\textstyleFootnoteBaseChar{\textit{kokomarek} }`it got dark') and transitive verbs (\textstyleFootnoteBaseChar{\textit{ama fookak } }`the sun split (tr.)').}

\ea%x270
\label{ex:x270}
\gll Aapereka  \textbf{paran-em-ika-i-ya}. \\
      \\
\glt
\z

cloud  rumble-SS.SIM-be-Np-PR.3s

`It is thundering.'

Intransitive clauses are discussed in 5.4.

\subparagraph[Transitive verbs]{Transitive verbs}
\hypertarget{RefHeading20601935131865}{}
Transitive verbs require a subject and an object. A [+human] object needs to be marked with an accusative pronoun (\sectref{sec:3.5.3}) regardless of the presence or absence of a separate object \textstyleAcronymallcaps{NP}. 

\ea%x294
\label{ex:x294}
\gll Yaapan  wia  ifakim-e-mik. \\
      \\
\glt
\z

Japan  3p.ACC  kill-PA-1/3p  

`They killed the Japanese.'

Besides the prototypical transitive verbs with an agent subject and a patient-of-change object (Giv\'on 1984:96) like (\stepcounter{nx}{\thenx}) and (\stepcounter{nx}{\thenx}), also many verbs of perception (\stepcounter{nx}{\thenx}), cognition (\stepcounter{nx}{\thenx}) and emotion (\stepcounter{nx}{\thenx}) are transitive.

\ea%x295
\label{ex:x295}
\gll Wiipa  erup  wia  \textstyleEmphasizedVernacularWords{sesek-a-mik}. \\
      \\
\glt
\z

girl  two  3p.ACC  send-PA-1/3p  

`They sent the two girls.'

\ea%x296
\label{ex:x296}
\gll Yo  me  efa  \textstyleEmphasizedVernacularWords{enim-uk}. \\
      \\
\glt
\z

1s.UNM  not  1.ACC  eat-IMP.3p  

`Let them not eat me.'\footnote{This was said in a traditional story by a spirit that was able to change into a man or into an eel, which the people in the story were preparing to eat.}

\ea%x297
\label{ex:x297}
\gll Nomokowa  unowa  aakisa  wia  \textstyleEmphasizedVernacularWords{uruf-i-n.} \\
      \\
\glt
\z

2s/p.brother  many  now  3p.ACC  see-Np-PR.2s  

`Now you see many brothers of yours.

\ea%x298
\label{ex:x298}
\gll Nefa  \textstyleEmphasizedVernacularWords{amis-ar-ep}  ma-i-yem. \\
      \\
\glt
\z

2s.ACC  knowledge-INCH-SS.SEQ  say-Np-PR.1s  

`I am saying (this) because I know you.'

\ea%x299
\label{ex:x299}
\gll Yena  mua=ke  efa  \textstyleEmphasizedVernacularWords{kookal-ep}  manin(a)  uuw-owa  \\
      \\
\glt
\z

1s.GEN  man  1s.ACC  like-SS.SEQ  garden  work-NMZ  

efa  asip-i-ya.

1s.ACC  help-Np-PR.3s

`My husband likes me and helps me in the garden.'

If there is no other overt object available for a transitive verb, the maximally generic noun \textstyleStyleVernacularWordsItalic{maa} `thing'\footnote{The semantic area of \textstyleFootnoteBaseChar{\textit{maa}} is at least as wide that of its English equivalent `thing'. Because it is used so often with verbs denoting eating and preparing food, it has acquired a secondary meaning `food'.} is used as a dummy object. Compare the next two examples: in (\stepcounter{nx}{\thenx}) \textstyleStyleVernacularWordsItalic{maa} is added because of syntactic requirements, whereas in (\stepcounter{nx}{\thenx}) the lack of an overt object indicates a third person singular object.

\ea%x300
\label{ex:x300}
\gll (Yo)  \textstyleEmphasizedVernacularWords{maa}  uruf-i-yem. \\
      \\
\glt
\z

I  thing  see-Np-PR.1s  

`I see.'  (=I see something, or: I can see.) 

\ea%x301
\label{ex:x301}
\gll (Yo)  uruf-i-yem. \\
      \\
\glt
\z

I  see-Np-PR.1s  

`I see him/her/it.'

\ea%x1825
\label{ex:x1825}
\gll Iir  oko  \textstyleEmphasizedVernacularWords{maa}  enim-i-yem,  iir  oko  \textstyleEmphasizedVernacularWords{maa}  me  enim-i-yem. \\
      \\
\glt
\z

time  other  thing  eat-Np-PR.1s  time  other  thing  not  eat-Np-PR.1s

`Sometimes I eat, sometimes I don't eat.'

The language-specific characteristic of syntactic transitivity (Kittil\"a 2002:49-51) is illustrated by a number of verbs that are transitive in Mauwake but intransitive in English:

aner-  `aim (at), refer (to)'

ikum-  `wonder (about)'

kerew-  `be angry (at)'

\ea%x302
\label{ex:x302}
\gll Wi  wia  amukar-emi  me  nefa  \textstyleEmphasizedVernacularWords{aner-a-m}. \\
      \\
\glt
\z

3p.UNM  3p.ACC  scold-SS.SIM  not  2s.ACC  refer.to-PA-1s  

`When I scolded them I didn't refer to you.'

\ea%x303
\label{ex:x303}
\gll Nefa  \textstyleEmphasizedVernacularWords{ikum-am-ika-iwkin}  nan  kerer-e-n. \\
      \\
\glt
\z

2s.ACC  wonder.about-SS.SIM-be-2/3p.DS  there  arrive-PA-2s

`As they were wondering about you, you arrived there.'

There are a a few verbs that require an undergoer object, but usually have recipient object as well. The verbs \textstyleStyleVernacularWordsItalic{ofakow}- `show, teach' and \textstyleStyleVernacularWordsItalic{maak}- `tell' are the most common of these.\footnote{The verb `send' is cross-linguistically typically ditransitive, but in Mauwake it requires the benefactive suffix in order to be able to take a second object.} 

\ea%x1838
\label{ex:x1838}
\gll Tunde  urera  Liisa  ame=ke  [\textstyleEmphasizedVernacularWords{epa}]\textsubscript{O}  [\textstyleEmphasizedVernacularWords{yia}]\textsubscript{O} \\
      \\
\glt
\z

Tuesday  afternoon  Liisa  ASSOC=CF  place  1p.ACC  

\textstyleEmphasizedVernacularWords{ofakowa-y}\textstyleEmphasizedVernacularWords{i}\textstyleEmphasizedVernacularWords{aw}\textstyleEmphasizedVernacularWords{-}\textstyleEmphasizedVernacularWords{e}\textstyleEmphasizedVernacularWords{-}\textstyleEmphasizedVernacularWords{mik}.

show-move.around-PA-1/3p

`On Tuesday afternoon Liisa and the othes showed us around the place.'

\ea%x943
\label{ex:x943}
\gll Nena  panewowa  pun  [\textstyleEmphasizedVernacularWords{wadol  opora}]\textsubscript{O} \textstyleEmphasizedVernacularWords{} [\textstyleEmphasizedVernacularWords{yia}]\textsubscript{O}  \textstyleEmphasizedVernacularWords{maak-i-n.} \\
      \\
\glt
\z

2s.GEN  old  also  lie  talk  1p.ACC  tell-Np-PR.2s

`You yourself -- an old person too! -- tell us lies.'

The verb \textstyleStyleVernacularWordsItalic{wu}- `put' requires both an undergoer object and a locative adverbial: 

\ea%x1837
\label{ex:x1837}
\gll [Sosora  nain]\textsubscript{O}  [pona-pa]\textsubscript{AdvP}  wu-a-mik. \\
      \\
\glt
\z

grass.skirt  that1  riverbank  put-PA-1/3p

`They put those grass skirts on the riverbank.'

The directional verbs (\sectref{sec:3.8.4.4.5}) could be treated as weakly transitive, in which case the goal \textstyleAcronymallcaps{NP}, which is never marked with the locative clitic -\textstyleStyleVernacularWordsxiiptItalic{pa}, could be a locative object. There are two main reasons against this analysis. When the goal of a directional verb is a personal pronoun, the dative case is used rather than the accusative:

\ea%x1870
\label{ex:x1870}
\gll {\dots}ona  wiawi  \textstyleEmphasizedVernacularWords{wiar}  ikiw-o-k. \\
      \\
\glt
\z

3s.GEN  3s/p.father  3.DAT  go-PA-3s

`{\dots}she went to her father.'

Also, if the directional verbs were considered weakly transitive and the the goal a locative object, the locative adverb \textstyleStyleVernacularWordsxiiptItalic{nan} `there' in the following clauses would be treated as a locative adverb in (\stepcounter{nx}{\thenx}) but as a locative object in (\stepcounter{nx}{\thenx}):

\ea%x1871
\label{ex:x1871}
\gll Kerer-ep  \textstyleEmphasizedVernacularWords{nan}  soomare-miaw-e-mik. \\
      \\
\glt
\z

arrive-SS.SEQ  there  walk-move.around-PA-1/3p

`They arrived and walked around there.'

\ea%x1872
\label{ex:x1872}
\gll Or-op  \textstyleEmphasizedVernacularWords{nan}  ikiw-ep  wia  uruf-a-k. \\
      \\
\glt
\z

descend-SS.SEQ  there  go-SS.SEQ  3p.ACC  see-PA-3s

`He went down and went there and saw them.'

Transitive clauses are discussed in \sectref{sec:5.3}.

\subparagraph[Ambitransitive verbs]{Ambitransitive verbs}
\hypertarget{RefHeading20621935131865}{}
Although most verb roots in Mauwake are clearly transitive or intransitive, there are a few that are ambitransitive. Many of their English equivalents would be intransitive. Only the following roots have been found to be neutral with regard to transitivity. Of them \textstyleStyleVernacularWordsItalic{ofof}- and \textstyleStyleVernacularWordsItalic{taan}- are of the S=O type, where the intransitive subject is an undergoer; the others are of the S=A type, where the intransitive subject is an actor \citep[124]{Dixon2010b}.

ofof-  `shake'

taan-  `become full'; `fill (a place)'

karu-  `run'; `visit'

om(om)-  `cry'; `mourn (for)'

pepek er-  `be enough'; `suffice'

aakun-  `speak, talk'; `discuss'

\ea%x1827
\label{ex:x1827}
\gll Ifar(a)  makena  wulewul  \textstyleEmphasizedVernacularWords{ofof-i-ya}. \\
      \\
\glt
\z

vine  fruit  wulewul  shake-Np-PR.3s

`The vine fruit (called) \textstyleForeignWords{wulewul} shakes.'

\ea%x1826
\label{ex:x1826}
\gll Maa-ofofona  saarik  \textstyleEmphasizedVernacularWords{wia  ofof-a-k}. \\
      \\
\glt
\z

earthquake  like  3p.ACC  shake-PA-3s

`It shook them like an earthquake.'

\ea%x304
\label{ex:x304}
\gll Ifa  uruf-ap  baurar-ep  \textstyleEmphasizedVernacularWords{karu-or-o-mik}. \\
      \\
\glt
\z

snake  see-SS.SEQ  flee-SS.SEQ  run-descend-PA-1/3p  

`We saw a snake and fled and ran down (to the village).'

\ea%x305
\label{ex:x305}
\gll Epasia=pa  ik-omkun  me  \textstyleEmphasizedVernacularWords{efa}  \textstyleEmphasizedVernacularWords{karu-e-mik.} \\
      \\
\glt
\z

far=LOC  be-1s/p.DS  NEG  1s.ACC  run/visit-PA-1/3p

`When I lived far away, they didn't visit me.'

\ea%x1059
\label{ex:x1059}
\gll En-em-ika-eya  ona  wiamun=ke  uruf-ap  \\
      \\
\glt
\z

eat-SS.SIM-be-2/3s.DS  3s.GEN  younger.sibling=CF  see-SS.SEQ

\textstyleEmphasizedVernacularWords{om-o-k}.

cry-PA-3s

`When he was eating it his younger sibling saw it/him and cried.'

\ea%x1060
\label{ex:x1060}
\gll \textstyleEmphasizedVernacularWords{Efa  om-em-ik-eya}  epa  wiim-o-k. \\
      \\
\glt
\z

1s.ACC  cry-SS.SIM-be-2/3s.DS  place  dawn-PA-3s

`While she was mourning for me it dawned.'

The subject of the adjunct plus verb \textstyleStyleVernacularWordsItalic{pepek er}- `be enough, suffice' is typically inanimate, whereas the object, when there is one, is usually human.  

\ea%x1058
\label{ex:x1058}
\gll Kemuka  nain  \textstyleEmphasizedVernacularWords{pepek  er-eya}  onak  ona  \\
      \\
\glt
\z

string  that1  enough  go-2/3s.DS  3s/p.mother  3s.GEN  

wiar  puuk-a-k.

3.DAT  cut-PA-3s

`When the string was (long) enough, their mother herself cut it.'

\ea%x306
\label{ex:x306}
\gll \textstyleEmphasizedVernacularWords{Wia}  \textstyleEmphasizedVernacularWords{pepek  er-a-k}. \\
      \\
\glt
\z

3p.ACC  enough  come/go-PA-3s  

`It was enough for them.'

\subparagraph[Object cross-referencing verbs]{Object cross-referencing verbs}
\hypertarget{RefHeading20641935131865}{}
One feature very common to a small group of verbs in the Trans-New Guinea languages is that the ``\textstyleBibliogCitationAAAstyleChar{verb stem undergoes changes according to the person of the object or beneficiary}'' \citep[62]{Wurm1982}.\footnote{Wurm actually seems to be referring to \textit{recipient} rather than beneficiary, as `give' is the most common of these verbs, and the verb stem changes according to the recipient.} In Mauwake this group consists of only five members. 

I call these verbs object cross-referencing because, besides marking the subject with a suffix like all other verbs do, they also \textstyleEmphasizedWords{\textsc{obligatorily}} mark the object in the verb root. What has clearly been a prefix\footnote{Phonetically this prefix is closer to the unmarked pronouns than the accusative pronouns.} earlier has been grammaticalized as part of the verb itself: there is no neutral root that would not be linked to any particular person.\footnote{When a ``neutral'' form is required, the third person singular is used.} In this respect these verbs differ from all the other verbs in Mauwake. In the case of `give' the verb root \textstyleStyleVernacularWordsItalic{i}- has assimilated into the prefix, so currently the person marking of the recipient object is the only root that there is. Four of the object cross-referencing verbs are listed below.

`give'  `feed'  `follow'  `shoot'

yi-  `give me'  enak  `feed me'  yook-  follow me'  enar-  `shoot me'

ni-  give you  nenak-  feed  you  nook-  follow you  nenar-  shoot you

iw-  give him  onak-  feed him  ook-  follow him  war-  shoot him

yi-  give us  yienak-  feed us  yiok-  follow us  yiar-  shoot us

ni-  give you    nienak-  feed you  niok-  follow you  niar-  shoot you

wi-  give them  wienak-  feed them  wiok-  follow them  wiar-  shoot them

\ea%x334
\label{ex:x334}
\gll Maa  fain  me  \textstyleEmphasizedVernacularWords{iw}-o-k. \\
      \\
\glt
\z

thing  this   not  give.him/her-PA-3s  

`He did not give this thing to him/her.'

\ea%x335
\label{ex:x335}
\gll Waaya  pun  \textstyleEmphasizedVernacularWords{enak}-e-mik. \\
      \\
\glt
\z

pig  too  feed.me-PA-1/3p  

`They also gave me pork to eat.'

\ea%x336
\label{ex:x336}
\gll Amia=iya  \textstyleEmphasizedVernacularWords{nenar}-e-mik=i? \\
      \\
\glt
\z

bow=COM  shoot.you-PA-1/3p=QM

`Did they shoot you with a gun?'

The cross-referenced objects are semantically quite different.  In the verbs \textstyleStyleVernacularWordsItalic{iw}- `give (him)' and \textstyleStyleVernacularWordsItalic{onak}- `feed (him)' it is the recipient,\footnote{\textstyleFootnoteBaseChar{\textit{onak-}} `feed (him)' requires a food term as the undergoer object, so a better translation, but longer, would be `give him (something) to eat'.} in \textstyleStyleVernacularWordsItalic{war}- `shoot (him)' and \textstyleStyleVernacularWordsItalic{ook}- `follow (him)' the undergoer. The verb \textstyleStyleVernacularWordsItalic{wionar}-\footnote{A possible origin for this is \textit{PRON+onaiya+ar}- `become together-with PRON' (Kwan, p.c.)} `hide among (them)' is a special case in two ways: the cross-referenced argument `among a group' is quite untypical as a verbal argument; and only plural forms of this verb can be used because of semantic restrictions. 

yionar-  `hide among us'

nionar-  `hide among you (pl)'

wionar-  `hide among them'

\ea%x337
\label{ex:x337}
\gll Wi  \textstyleEmphasizedVernacularWords{wionar}-ep  pok-ap  ik-ua. \\
      \\
\glt
\z

3p.UNM  hide.among.them-SS.SEQ  sit.down-SS.SEQ  be-PA.3s  

`He sat hiding among them.'

Maia does not have any verbs behaving like this \citep{Hardin2002}, and Hepner only reports one for Bargam: \textstyleForeignWords{\nobreakdash-g}  `give' (2002:87). Usan has three verbs involving a stem change of this kind: \textstyleForeignWords{ut\^ab} `give (him)', \textstyleForeignWords{w\^ab} `shoot' and \textstyleForeignWords{w\^aramb} `hit'\citep[44]{Reesink1987}. \textstyleForeignWords{Ut\^ab}, which coreferences the recipient, has quite strict co-occurrence restrictions with other arguments or even with peripheral elements in the same clause (ibid. 129-30). 

Unlike Usan, in Mauwake the clauses with object cross-referencing verbs can easily have a locative or instrument phrase, and the verb itself can take a benefactive suffix. A sentence like (\stepcounter{nx}{\thenx}) would be possible for instance when sending money to people travelling in the same vehicle as the addressee.  

\ea%x338
\label{ex:x338}
\gll Miiw-aasa=pa  \textstyleEmphasizedVernacularWords{wi-om}-\textstyleEmphasizedVernacularWords{e}. \\
      \\
\glt
\z

land-canoe=LOC  give.them-BEN-BNFY1.IMP.2s  

`Give it to them for me in the car.'

\paragraph[Valence changes]{Valence changes}
\hypertarget{RefHeading20661935131865}{}
The term \textstyleEmphasizedWords{\textsc{valence}} refers to the number of arguments that have a grammatical relation with the verb. As was mentioned above, almost all of the verb roots in Mauwake have a basic valence of one or two: they take either a subject only (intransitive verbs \sectref{sec:3.8.4.2.1}) or a subject and an object (transitive verbs \sectref{sec:3.8.4.2.2}) as their arguments. There are some ways to change the valence of verbs, even if strategies like passivization and dative shift are not possible in Mauwake.  The valence is increased, when an intransitive verb is made into transitive or a transitive verb into causative with the addition of an causative suffix, or when a benefactive suffix is added to a verb. There are no processes to decrease the syntactic valence of a verb. The \textstyleEmphasizedWords{\textsc{semantic}} valence is decreased when the object of a transitive verb is a reflexive or reciprocal pronoun, since the subject and object have the same referent(s). Subject demotion is another way to decrease the semantic valence. 

\subparagraph[Causatives]{Causatives}
\hypertarget{RefHeading20681935131865}{}
The causative always increases the number of arguments a verb can take: the subject of an intransitive verb becomes the object of a transitive verb, and a new subject is added. The causative suffix -\textstyleStyleVernacularWordsItalic{ow} was described above in \sectref{sec:3.8.2.3.1}.  In most cases the meaning of a causative is `to cause someone or something to do something'. The caused `doing' is usually \textstyleEmphasizedWords{\textsc{not} }agentive. 

\ea%x997
\label{ex:x997}
\gll Iwera  nainiw  kaken  iimar-e-k. (Intransitive) \\
      \\
\glt
\z

coconut  again  straight  stand-PA-3s

`The coconut palm stood straight again.'

\ea%x998
\label{ex:x998}
\gll [Eka  napia]\textsubscript{O}  koor  miira=pa  iimar-\textstyleEmphasizedVernacularWords{ow}-a-mik. \\
      \\
\glt
\z

water  bamboo  house  face=LOC  stand-CAUS-PA-1/3p

`We made the bamboo water containers stand in front of the house.'

\ea%x992
\label{ex:x992}
\gll [Wiowa  erup]\textsubscript{O}  ar-\textstyleEmphasizedVernacularWords{ow}-amkun  um-o-k. \\
      \\
\glt
\z

spear  two  become-CAUS-1s/p.DS  die-PA-3s

`I speared it a second time and it (=the pig) died.'(Lit: `I caused a spear to become two and it died.')

The mental state of being angry is expressed via a verb in Mauwake (\stepcounter{nx}{\thenx}), and it can take the causative suffix (\stepcounter{nx}{\thenx}).

\ea%x993
\label{ex:x993}
\gll Kema  bagiwir-a-m. \\
      \\
\glt
\z

liver  be.angry-PA-1s

`I was angry.'

\ea%x994
\label{ex:x994}
\gll Yo  kema  [efa]\textsubscript{O}  bagiwir-\textstyleEmphasizedVernacularWords{ow}-a-n,  yaa! \\
      \\
\glt
\z

1s.UNM  liver  1s.ACC  be.angry-CAUS-PA-2s  INTJ

`Boy, have you made me angry!'

In some cases the causative suffix acts simply as a transitiviser. The subject in (\stepcounter{nx}{\thenx}) does not actually cause the children to grow. Also in this case the suffix  increases the valency of the verb: \textstyleStyleVernacularWordsItalic{arim}- `grow' in (\stepcounter{nx}{\thenx}) is intransitive, but \textstyleStyleVernacularWordsItalic{arimow}- in (\stepcounter{nx}{\thenx}) is transitive and takes an object.

\ea%x995
\label{ex:x995}
\gll Aakisa  arim-o-n,  aakisa  muew-o-n. \\
      \\
\glt
\z

now  grow-PA-2s  now  marry-PA-2s

`Now you have grown, now you have married.'

\ea%x996
\label{ex:x996}
\gll No  nena  maa  fariar-ep  [muuka  nain]\textsubscript{O} \\
      \\
\glt
\z

2s.UNM  2s.GEN  food  abstain-SS.SEQ  son  that1  

arim-\textstyleEmphasizedVernacularWords{ow}-e.

grow-CAUS-IMP.2s

`You yourself have to abstain from (certain) food(s) and bring the son up.'

When the causative suffix is added to the intransitive verb \textstyleStyleVernacularWordsItalic{sail}- `(tell a) lie', its meaning changes into `lie to someone', `cheat'. 

\ea%x448
\label{ex:x448}
\gll Opor(a)  makena  ma-i-yem,  me  [nia]\textsubscript{O}  sail-\textstyleEmphasizedVernacularWords{ow}-iyem. \\
      \\
\glt
\z

talk  true  say-Np-PR.1s  not  2p.ACC  lie-CAUS-PR.1s

`I am telling the truth, I am not cheating you.'

Bring-prefixes (\sectref{sec:3.8.2.4.2}) are another causative strategy, used only with the directional verbs (\sectref{sec:3.8.4.4.5}) and a couple of other motion verbs. The subject of the verb causes the object to move by undertaking the transfer himself/herself.

\ea%x999
\label{ex:x999}
\gll Maa  unowa  ifer  aasa=ke  \textstyleEmphasizedVernacularWords{p}-urup-o-k. \\
      \\
\glt
\z

thing  many  sea  canoe=CF  BPx-ascend-PA-3s

`A lot of things were brought/taken up by ships.'

\ea%x1001
\label{ex:x1001}
\gll O  mua  imen-ap=na  feeke  wia  \textstyleEmphasizedVernacularWords{p}-ekap-eka. \\
      \\
\glt
\z

3s.UNM  man  find-SS.SEQ=TP  here.CF  3p.ACC  BPx-come-IMP.2p

`If/when you find a/any man, bring them/him here.'

\ea%x1000
\label{ex:x1000}
\gll Gomi  kawus  \textstyleEmphasizedVernacularWords{p}-irapar-i-ya. \\
      \\
\glt
\z

east.wind  smoke  BPx-move.to.and.fro-Np-PR.3s

`The east wind moves the smoke around.'

Forming a causative from an agentive verb (\textstyleEmphasizedWords{\textsc{inducive causative}}, Talmy 2007:112) is not done morphologically with an affix but syntactically with a verbal construction involving the nominalized form of the main verb and \textstyleStyleVernacularWordsItalic{suuw}- `push' as the causative auxiliary (5.7.1). 

\ea%x1003
\label{ex:x1003}
\gll O  uruf-ap  op-ap  Yeesus  nomokowa  moke \\
      \\
\glt
\z

3s.UNM  see-SS.SEQ  hold-SS.SEQ  Jesus  tree  slanting  

\textstyleEmphasizedVernacularWords{akua-aaw-om-owa  suuw-a-mik}.

shoulder-take-BEN-NMZ  push-PA-1/3p

`They saw him and took hold of him, and made him carry Jesus' cross on his shoulder.'

\ea%x1002
\label{ex:x1002}
\gll Sira  enuma  \textstyleEmphasizedVernacularWords{ook-owa  nia  suuw-i-mik}. \\
      \\
\glt
\z

custom  new  follow-NMZ  2p.ACC  push-Np-PR.1/3p

`They make you follow new customs/ways.'

In the following examples the three different causative strategies have been applied to the same verb \textstyleStyleVernacularWordsItalic{ikiw}- `go', and in all of them the patient is [+human]. In (\stepcounter{nx}{\thenx}) and (\stepcounter{nx}{\thenx}) the object of the causative verb has no influence on what happens to him/her, but in (\stepcounter{nx}{\thenx}) the object of the inducive causative is active and becomes the actor of the verb resulting from the causation. 

\ea%x1016
\label{ex:x1016}
\gll Ipamsika  mua=ke  \textstyleEmphasizedVernacularWords{ikiw-ow}\textstyleEmphasizedVernacularWords{-a-k}. \\
      \\
\glt
\z

nail  man=CF  go-CAUS-PA-3s  

`A sorcerer (lit: nail man) killed him (lit: caused him to go).'

\ea%x1829
\label{ex:x1829}
\gll Kes  tepak=pa  wu-ap  \textstyleEmphasizedVernacularWords{p-ikiw-e-mik}. \\
      \\
\glt
\z

coffin  inside=LOC  put-SS.SEQ  Bpx-go-PA-1/3p

`They put him inside the coffin and took him (away).'

\ea%x1873
\label{ex:x1873}
\gll Yo  mua  oko  \textstyleEmphasizedVernacularWords{ikiw-owa  suuw-amkun}  ikiw-i-non. \\
      \\
\glt
\z

1s.UNM  man  other  go-NMZ  push-1s/p.DS  go-Np-FU.3s

`I make a man go and he goes.'

\subparagraph[  Benefactive]{  Benefactive}
\hypertarget{RefHeading20701935131865}{}
The benefactive form of a verb (\sectref{sec:3.8.2.3.3}) is used when an action is done \textstyleEmphasizedWords{\textsc{for} }someone, for their benefit (\stepcounter{nx}{\thenx}), or in some cases for their detriment (\stepcounter{nx}{\thenx}). With the addition of the benefactive suffix to the verb, the beneficiary  becomes an obligatory argument. The beneficiary is always animate, and usually human. 

\ea%x1004
\label{ex:x1004}
\gll Wi  owow  mua=ke  wilkar  wia \\
      \\
\glt
\z

3p.UNM  village  man=CF  cart  3p.ACC

muf-em-ik-\textstyleEmphasizedVernacularWords{om}-a-mik.

pull-SS.SIM-be-BEN-BNFY2.PA-1/3p

`The village men kept pulling carts for them.'

\ea%x1005
\label{ex:x1005}
\gll Epia  wilin-owa  uruf-ap  bom  yia \\
      \\
\glt
\z

fire(wood)  shine-NMZ  see-SS.SEQ  bomb  1p.ACC  

fuurk-\textstyleEmphasizedVernacularWords{om}-i-kuan.

throw-BEN-Np-FU.3p

`When they see the light from the fire(s) they will throw bombs at us.'

More than one valency-increasing strategy can be applied to a verb simultaneously. In both (\stepcounter{nx}{\thenx}) and (\stepcounter{nx}{\thenx}) the valency of the verb increases from one to three: besides the subject of the original verb, the derived verbs also have both an object and a beneficiary.

\ea%x1007
\label{ex:x1007}
\gll Koor  poka  iimar-\textstyleEmphasizedVernacularWords{ow}-\textstyleEmphasizedVernacularWords{om}-e. \\
      \\
\glt
\z

house  post  stand.up-CAUS-BEN-BNFY1.IMP.2s

`Stand up the house posts for me.'

\ea%x1008
\label{ex:x1008}
\gll Ona  soomia  marew-eya  \textstyleEmphasizedVernacularWords{amap}-ep-\textstyleEmphasizedVernacularWords{om}-a-m. \\
      \\
\glt
\z

3s.GEN  spoon  no(ne)-2/3s.DS  BPx-come-BEN-BNFY2.PA-1s

`She has/had no spoons of her own, so I brought them for her.'

\subparagraph[Decreasing semantic valence]{Decreasing semantic valence}
\hypertarget{RefHeading20721935131865}{}
There are no morphological means in Mauwake for decreasing syntactic valence. A verb that is inherently reflexive, like \textstyleStyleVernacularWordsItalic{yaki}- `wash oneself', is intransitive. But the semantic valence of transitive verbs is decreased when they are made either reflexive or reciprocal. Syntactically the reflexive/reciprocal pronoun is an object, but the pronoun refers to the same referent(s) as the subject.  

\ea%x1834
\label{ex:x1834}
\gll Birin-ep  nomokowa  iinan  akena  ikiw-ep  wame \\
      \\
\glt
\z

fly-SS.SEQ  tree  top  very  go-SS.SEQ  3s.REFL  

pipilim-ep  aakun-em-ika-i-non.

hide-SS.SEQ  speak-SS.SIM-be-Np-FU.3s

`It will fly and hide (itself) in the very top of a tree and keep making noise.'

\ea%x1835
\label{ex:x1835}
\gll Osaiwa  aalbok  ikos  uf-owa  na-ep \\
      \\
\glt
\z

bird.of.paradise  black.cuckoo-shrike  together  dance-NMZ  say-SS.SEQ  

ofa  wiam  if-e-mik.

colour  3p.REFL  paint-PA-1/3p

`A bird of paradise and a black cuckoo-shrike wanted to dance together and painted each other with colour.'

A common valence-decreasing device in many languages is the passive voice, which demotes or deletes the subject. In Mauwake verbs there is no passive voice. The standard way of demoting the subject is to have the verb in third person plural form and leave the subject \textstyleAcronymallcaps{NP} unexpressed.\footnote{Cf. the English impersonal ``they'': \textit{They say it is going to be cold tomorrow}.} None of the arguments change their syntactic function. The example (\stepcounter{nx}{\thenx}) comes from a story where the main point was that the people responsible for the fire were never found, and it was not known if only one person was involved or many. 

\ea%x1009
\label{ex:x1009}
\gll Fikera  ikum  \textstyleEmphasizedVernacularWords{kuum-e-mik}  nain  ma-i-yem. \\
      \\
\glt
\z

kunai.grass  illicitly  burn-PA-1/3p  that1  say-Np-PR.1s

`I tell about when the kunai grass was burned (by arson).'

\ea%x1010
\label{ex:x1010}
\gll Nefa  \textstyleEmphasizedVernacularWords{war-iwkin}  naap  ma-e. \\
      \\
\glt
\z

2s.ACC  shoot-2/3p.DS  thus  say-IMP.2s

`If/when you are shot, then say like that.' (Or: `If they shoot you, then say like that.')

Another strategy to demote the subject is to use the same-subject sequential form of the main verb and the auxiliary \textstyleStyleVernacularWordsItalic{ik}- `be' agreeing with the object of the verb. This can only be used when the end result is a state. 

\ea%x1011
\label{ex:x1011}
\gll Nomokowa  puuk-ap  ik-ua. \\
      \\
\glt
\z

tree  cut-SS.SEQ  be-PA.3s

`The tree is cut.'

\paragraph[Semantically based verb classes ]{Semantically based verb classes} 
\hypertarget{RefHeading20741935131865}{}
Even though the following classification is based on semantic characteristics of the verbs, the verbs within the resulting groups tend to have similarities in their syntactic behaviour as well.

\subparagraph[Stative/existential verb ik- ]{Stative/existential verb \textit{ik}-} 
\hypertarget{RefHeading20761935131865}{}
The basic meaning of the stative verb \textstyleStyleVernacularWordsItalic{ik}(\textstyleStyleVernacularWordsItalic{a})- is `be'. The vowel /a/ gets deleted elsewhere except in the present tense and the medial different-subject non-first plural form; in the corresponding singular form the vowel may be optionally deleted (\stepcounter{nx}{\thenx}). 

\ea%x257
\label{ex:x257}
\gll Nan  mukuna=pa  \textstyleEmphasizedVernacularWords{ik(a)-eya}  o  nan  samor  aaw-o-k. \\
      \\
\glt
\z

there  fire=LOC  be-2/3s.DS  3s.UNM  there  badly  get-PA-3s

`They (=bananas) were there on the fire and he really got bad there.'

Like intransitive verbs, it may form a complete clause by itself. Example (\stepcounter{nx}{\thenx}) is from a situation where the speaker was in a plane for the first time, refused to eat and declined any help offered to him.

\ea%x1455
\label{ex:x1455}
\gll \textstyleEmphasizedVernacularWords{Ika-i-nen}. \\
      \\
\glt
\z

be-Np-FU.1s

`I will just be (like this).'

Often it is used for `be/live (somewhere)', and in this use it naturally co-occurs with a locative adverbial:

\ea%x497
\label{ex:x497}
\gll I  naap  koora=pa  \textstyleEmphasizedVernacularWords{ik-e-mik}. \\
      \\
\glt
\z

1p.UNM  thus  house=LOC  be-PA-1/3p

`We were in the house like that.'

Together with the dative pronouns \textstyleStyleVernacularWordsItalic{ik}- is used to form possessive constructions (\stepcounter{nx}{\thenx}) (\sectref{sec:3.5.5}, 5.5.2). 

\ea%x258
\label{ex:x258}
\gll Yo  waaya  arow  \textstyleEmphasizedVernacularWords{efar  ik-ua.} \\
      \\
\glt
\z

1s.UNM  pig  three  1s.DAT  be-PA.3s  

`I have three pigs.'

The function of \textstyleStyleVernacularWordsItalic{ik}- as a copular verb is very restricted. In equative or descriptive clauses it is normally not used in the present tense finite form, but in the past (\stepcounter{nx}{\thenx}) and future (\stepcounter{nx}{\thenx}) tenses it is employed. It could be said, following Giv\'on, that in these clauses its primary function is to be the carrier of the tense (1984:92). 

\ea%x259
\label{ex:x259}
\gll Yo  um-ep  ik-owa  saarik  \textstyleEmphasizedVernacularWords{ik-e-m}. \\
      \\
\glt
\z

1s.UNM  die-SS.SEQ  be-NMZ  like  be-PA-1s

`I was like dead.'

\ea%x1070
\label{ex:x1070}
\gll Ikoka  maa  marew,  eliw  manek=iw  \textstyleEmphasizedVernacularWords{ika-i-nan}. \\
      \\
\glt
\z

later  thing  none  well  big=LIM  be-Np-FU.2s

`Later there will be no problem, you will just be very well.'

In Mauwake it can also be used when the non-verbal predicate is understood to be transitory (\stepcounter{nx}{\thenx}) rather than stable over time:

\ea%x499
\label{ex:x499}
\gll No  kamenap  \textstyleEmphasizedVernacularWords{ika-i-n}? \\
      \\
\glt
\z

2s.UNM  how  be-Np-PR.2s

`How are you?' 

The verb \textstyleStyleVernacularWordsItalic{ik}- `be' is in a class of its own for several reasons. Its morphology is irregular, and so are the semantics of some of its morphology. In (\stepcounter{nx}{\thenx}) the past tense and the person/number marker in the third person singular form are merged into one portmanteau morpheme. An alternative form for the different-subject first person form \textstyleStyleVernacularWordsItalic{ikemkun} is \textstyleStyleVernacularWordsItalic{ikomkun} (\stepcounter{nx}{\thenx}). The same-subject medial form is \textstyleStyleVernacularWordsItalic{ikok} (\stepcounter{nx}{\thenx}), not *\textstyleStyleVernacularWordsItalic{ikep}\textstyleEmphasizedVernacularWords{} and\textstyleEmphasizedVernacularWords{} *\textstyleStyleVernacularWordsItalic{ikemi}\textstyleEmphasizedVernacularWords{\textmd{\textit{.}}}\footnote{\textstyleFootnoteBaseChar{\textit{ikep}} and \textstyleFootnoteBaseChar{\textit{ikemi}} are the same subject medial forms of the homophonous verb \textstyleFootnoteBaseChar{\textit{ik-}} `roast'.} There is no formal differentiation between a simultaneous (\stepcounter{nx}{\thenx}) and a sequential (\stepcounter{nx}{\thenx}) form in the same-subject medial verb. 

\ea%x1931
\label{ex:x1931}
\gll Siowa  nain  kakalt-am-\textstyleEmphasizedVernacularWords{ik}\textstyleEmphasizedVernacularWords{-}\textstyleEmphasizedVernacularWords{emkun}  arim-o-k. \\
      \\
\glt
\z

dog  that1  look.after-SS.SIM-be-1s/p.DS  grow-PA-3s

`I was looking after the dog and it grew.'

\ea%x260
\label{ex:x260}
\gll Naap  \textstyleEmphasizedVernacularWords{ik-ok}  uruf-am-ika-iwkin  wia. \\
      \\
\glt
\z

thus  be-SS  see-SS.SIM-be-2/3p.DS  no

`As he was/stayed like that they were watching him (but) no (=he didn't get better).' 

\ea%x262
\label{ex:x262}
\gll Owowa  ekap-o-k,  amia  mua=pa  \textstyleEmphasizedVernacularWords{ik-ok}. \\
      \\
\glt
\z

village  come-PA-3s  bow  man=LOC  be-SS  

`He came to the village, having been in the police (force).'

It also differs from ordinary intransitive verbs in that in a verb+auxiliary construction it cannot be the main verb, but can be used as the aspectual auxiliary (\stepcounter{nx}{\thenx}) (see also \sectref{sec:3.8.4.5}). But it is like other intransitive verbs in that it can take an causative suffix (\stepcounter{nx}{\thenx}).\footnote{Reesink notes that in Usan the corresponding verb \textstyleFootnoteBaseChar{\textit{igo}} `be' cannot occur with the causative suffix (1987:142). In Mauwake there is no similar restriction.} 

\ea%x261
\label{ex:x261}
\gll Nomokowa  war-ep  miiwa=pa  \textstyleEmphasizedVernacularWords{ik-ow-a-mik.} \\
      \\
\glt
\z

tree  cut-SS.SEQ  ground=LOC  be-CAUS-PA-1/3p

`We cut trees and laid them on the ground' 

The tense distinction is partly neutralized: the past tense form is used for past (\textstyleParagraphChari{\stepcounter{nx}{\thenx}}) and present (\textstyleParagraphChari{\stepcounter{nx}{\thenx}}). The present tense form is not very common and is mainly used for less time-stable situations (\stepcounter{nx}{\thenx}), (\stepcounter{nx}{\thenx}), or to replace the missing  continuous aspect form (\textstyleParagraphChari{\stepcounter{nx}{\thenx}}). The verb \textstyleStyleVernacularWordsItalic{ik}- is used as the regular continuous aspect auxiliary (\sectref{sec:4.4.1}), and as a main verb \textstyleStyleVernacularWordsItalic{ik}- `be' cannot take this auxiliary. 

\ea%x263
\label{ex:x263}
\gll Yo  unan  koka=pa  \textstyleEmphasizedVernacularWords{ik-e-m}. \\
      \\
\glt
\z

1s.UNM  yesterday  jungle=LOC  be-PA-1s  

`Yesterday I was in the jungle.' 

\ea%x264
\label{ex:x264}
\gll Ni  kaaneke  \textstyleEmphasizedVernacularWords{ik-e-man}  oo,  ni  ekap-omak-eka  oo! \\
      \\
\glt
\z

2p.UNM  where  be-PA-2p  oh  2p.UNM  come-DISTR-IMP.2p  oh  

`Wherever you are, come!'

\ea%x1028
\label{ex:x1028}
\gll Mesa  asia  fiker  gone=pa  \textstyleEmphasizedVernacularWords{ika-i-ya}  nain \\
      \\
\glt
\z

winged.bean  wild  kunai.grass  middle=LOC  be-Np-PR.3s  that1  

aaw-em-ik-e-m.

take-SS.SIM-be-PA-1s

`I was picking wild winged bean that was (lit: is) in the middle of the kunai grass.'

\ea%x265
\label{ex:x265}
\gll Yo  nan  \textstyleEmphasizedVernacularWords{ika-i-yem}  nain  yo  nia  asip-i-yem,  {\dots} \\
      \\
\glt
\z

1s.UNM  there  be-Np-PR.1s  that1  1s.UNM  2p.ACC  help-Np-PR.1s

`Now that I am living there I help you, {\dots}'

The verb \textstyleStyleVernacularWordsItalic{ik}-  mainly functions in intransitive clauses, but it is also needed as a copula for those cases where a non-verbal predicate is in a non-present tense. 

\ea%x969
\label{ex:x969}
\gll O  ikoka  somek  mua  maneka  \textstyleEmphasizedVernacularWords{ika-i-non}. \\
      \\
\glt
\z

3p.UNM  later  song  man  big  be-Np-FU.3s

`He will later be the headmaster.'

An equative or descriptive medial clause requires \textstyleStyleVernacularWordsItalic{ik}- as a copula regardless of the tense of the final verb (\stepcounter{nx}{\thenx}).

\ea%x498
\label{ex:x498}
\gll Koora  naap  \textstyleEmphasizedVernacularWords{ik-eya}  uruf-i-mik. \\
      \\
\glt
\z

house  thus  be-2/3s.DS  see-Np-PR.1p

`We see the house as it is like that.'

\subparagraph[Position-taking verbs]{Position-taking verbs}
\hypertarget{RefHeading20781935131865}{}
The three position-taking verbs are among the most frequently used verbs in Mauwake: \textstyleStyleVernacularWordsItalic{pok}- `sit down', \textstyleStyleVernacularWordsItalic{iimar}- `stand up' and \textstyleStyleVernacularWordsItalic{in}- `lie down/ fall asleep'. They are essentially punctiliar verbs with an inceptive meaning (\stepcounter{nx}{\thenx}), but they are most typically used in the same-subject sequential form together with the auxiliary \textstyleStyleVernacularWordsItalic{ik}- (\sectref{sec:3.8.4.5}) to convey stative meaning: `sit' (\stepcounter{nx}{\thenx}), `stand', and `lie/sleep'.

\ea%x273
\label{ex:x273}
\gll Kokom-ar-eya  \textstyleEmphasizedVernacularWords{in-e-mik}. \\
      \\
\glt
\z

darkness-INCH-2/3s.DS  lie.down-PA-1/3p  

`When it got dark we went to bed.'

\ea%x274
\label{ex:x274}
\gll Ona  koora=pa  arew-ap  \textstyleEmphasizedVernacularWords{pok-ap  ik-e-mik}. \\
      \\
\glt
\z

3s.GEN  house=LOC  wait-SS.SEQ  sit.down-SS.SEQ  be-PA-1/3p

`We sat and waited (lit: waited and sat) in his house.'

The verb \textstyleStyleVernacularWordsItalic{pok}- is occasionally used without the auxiliary to mean `sit': 

\ea%x1824
\label{ex:x1824}
\gll Neek(e)  \textstyleEmphasizedVernacularWords{pok-aka}. \\
      \\
\glt
\z

there  sit-IMP.2p

`Sit there/Keep sitting there.' (Commonly used as a conversational ``filler'' for people that are already sitting, when there is a lull in the conversation.)

The continuous aspect form of the position-taking verbs is not used with progressive meaning, only with the meaning `habitual' (\sectref{sec:3.8.5.1.1.2}). 

\ea%x275
\label{ex:x275}
\gll Irak-ow  epa=pa  koka=pa  \textstyleEmphasizedVernacularWords{in-em-ik-e-mik}. \\
      \\
\glt
\z

fight-NMZ  time=LOC  jungle=LOC  lie.down-SS.SIM-be-PA-1/3p  

`During the war we used to sleep in the jungle.'

\subparagraph[Location verbs]{Location verbs}
\hypertarget{RefHeading20801935131865}{}
The two verbs that have been verbalized from the demonstrative adverbs \textstyleStyleVernacularWordsItalic{fan} `here' and \textstyleStyleVernacularWordsItalic{nan} `there' (\sectref{sec:3.8.2.2.1}), are very restricted in their use. The original meaning of the verbs must refer to arrival at some place, but since they are only used in the past tense, they currently tend to indicate presence at a place rather than movement.\footnote{This may indicate that the past tense used to encode perfectivity (Malcolm Ross, p.c.)} They can even be used with immobile objects (\stepcounter{nx}{\thenx}). 

\ea%x1270
\label{ex:x1270}
\gll No  ikiw-e,  irak-owa  maneka  \textstyleEmphasizedVernacularWords{fan}\textstyleEmphasizedVernacularWords{-}\textstyleEmphasizedVernacularWords{e}\textstyleEmphasizedVernacularWords{-}\textstyleEmphasizedVernacularWords{k}  a. \\
      \\
\glt
\z

2s.UNM  go-IMP.2s  fight-NMZ  big  here-PA-3s  INTJ

`Go (home), the big war is here.'

\ea%x1271
\label{ex:x1271}
\gll Aakisa  i  \textstyleEmphasizedVernacularWords{fan-e-mik}. \\
      \\
\glt
\z

Now  1p.UNM  here-PA-1/3p  

`Now we are / have come here.'

\ea%x1272
\label{ex:x1272}
\gll No  niawi  akena  \textstyleEmphasizedVernacularWords{nan-e-k}. \\
      \\
\glt
\z

2s.UNM  2s/p.father  true  there-PA-3s

`Your real father is there.'

\ea%x1276
\label{ex:x1276}
\gll Aa,  o  koora  \textstyleEmphasizedVernacularWords{fan-e-k}  a. \\
      \\
\glt
\z

INTJ  3s.UNM  house  here-PA-3s  INTJ

`Ah, his house is here.'

\subparagraph[Resultative verbs]{Resultative verbs}
\hypertarget{RefHeading20821935131865}{}
The resultative verbs with the meaning `become' are another small group of intransitive verbs. Besides the semantic similarity they also share the syntactic characteristic that, in addition to the subject, they require another argument expressing the result of change. This other obligatory argument is a noun with the verbs \textstyleStyleVernacularWordsItalic{ar}- `become' and \textstyleStyleVernacularWordsItalic{puuk}- `change into',\footnote{This verb is homonymous with the transitive verb \textstyleFootnoteBaseChar{\textit{puuk-}} `cut'. They may be historically related, but synchronically the meanings are quite different.} and a colour adjective with the verb \textstyleStyleVernacularWordsItalic{kir}- `turn'. 

\ea%x276
\label{ex:x276}
\gll Takira  arim-ep  mua  \textstyleEmphasizedVernacularWords{ar-e-k}. \\
      \\
\glt
\z

boy  grow-SS.SEQ  man  become-PA-3s  

`The boy grew and became a man.'

\ea%x277
\label{ex:x277}
\gll Emeria  nain  afa  \textstyleEmphasizedVernacularWords{ar-e-mik}. \\
      \\
\glt
\z

woman  that1  flyng.fox  become-PA-1/3p  

`Those women became flying foxes.'

\ea%x278
\label{ex:x278}
\gll Inasin  mua  ifa  \textstyleEmphasizedVernacularWords{puuk-ap}  solon-ep  {\dots} \\
      \\
\glt
\z

spirit  man  snake  change.into-SS.SEQ  glide-SS.SEQ  

`The spirit man changed into a snake, glided and {\dots}'

\ea%x279
\label{ex:x279}
\gll Oona  kia  \textstyleEmphasizedVernacularWords{kir-em-ik-eya}  uruf-ap  ma-e-k  {\dots} \\
      \\
\glt
\z

bone  white  turn-SS.SIM-be-2/3s.DS  see-SS.SEQ  say-PA-3s  

`She saw that the bones were turning white and said, {\dots}'

The verb \textstyleStyleVernacularWordsItalic{ar}- is mostly used when the subject stays essentially the same but undergoes some change (\stepcounter{nx}{\thenx}). However, it can also be used when the subject  changes into something else (\stepcounter{nx}{\thenx}). The verb \textstyleStyleVernacularWordsItalic{puuk}- is only used in the latter context (\stepcounter{nx}{\thenx}), and it is always an intentional action. It is most common in traditional stories where spirits change into various inanimate things or animate beings. The verb \textstyleStyleVernacularWordsItalic{kir}- is used with most colour terms (\stepcounter{nx}{\thenx}), but for `black' there is a separate verb formed with the inchoative suffix \nobreakdash-\textstyleStyleVernacularWordsItalic{ar}\textstyleEmphasizedVernacularWords{} : \textstyleStyleVernacularWordsItalic{sepenar}-\footnote{This is related to the adjective \textstyleFootnoteBaseChar{\textit{sepa}} `black'.} `become black'. The inchoative suffix (\sectref{sec:3.8.2.2.2}) is the standard device used for verbalizing adjectives. 

\subparagraph[Directional verbs]{Directional verbs}
\hypertarget{RefHeading20841935131865}{}
The verbs indicating coming and going are among the most frequent verbs in Mauwake. These verbs have the direction inherent in the verb root. Verbs of this kind are quite common among Papuan languages: in some languages the directional is an affix, in others it is part of the meaning of the root itself \citep[149]{Foley1986};  Mauwake is of the latter type. The directional verb group contains verbs that in many languages would be prototypically intransitive.

Most of these verbs can be translated into English as either `go' or `come', depending on the context. Since the elevation of the goal, the direction of the compass and the distance all influence the choice of the verb, and may conflict with each other, the speaker has some freedom of choice. Also, with regard to proximity, it is a very relative notion how close or far away something is.

ikiw-  `go', `leave' (away from the deictic centre; generic)

iw-    `go' (away from the deictic centre)\footnote{In the Moro area \textstyleFootnoteBaseChar{\textit{iw-}} also has the meaning `enter': \textstyleFootnoteBaseChar{\textit{Marasin kema wiar iwak}}  `The poison entered his liver.'}

ekap-  `come' (towards the deictic centre; generic)

urup-  `go/come up', `ascend' (uphill/away from sea)

or(a)-  `go/come down', `descend' (downhill/towards sea)

ek-    `go (close/east)'

ep-    `come (close/west)'

er-    `go (not close/west/downriver)'

ir-    `come/go (not close/east/upriver)', `climb'

\ea%x280
\label{ex:x280}
\gll Manina  \textstyleEmphasizedVernacularWords{urup-ep}  nan  uuw-ap  owowa  \textstyleEmphasizedVernacularWords{or-o-k}. \\
      \\
\glt
\z

garden  ascend-SS.SEQ  there  work-SS.SEQ  village  descend-PA-3s

`She went up to the garden, worked there and came down to the village.'

\ea%x281
\label{ex:x281}
\gll Fofa  \textstyleEmphasizedVernacularWords{er-ap}  \textstyleEmphasizedVernacularWords{ir-i-mik}. \\
      \\
\glt
\z

market  go.SS.SEQ  come-Np-PR.1/3p  

`We are coming back from the market.' (Lit: `We went west to the market and are coming east.')

The deictic orientation of \textstyleStyleVernacularWordsItalic{ikiw}- `away from speaker/deictic centre' and \textstyleStyleVernacularWordsItalic{ekap}- `towards the speaker/deictic centre'is stricter in Mauwake than in many European languages where the deictic centre especially for `come' can vary considerably. The sentence (\stepcounter{nx}{\thenx}) is all right in Finnish regardless of the location of the speaker, but the corresponding sentence in Mauwake would be acceptable only if the speaker were in Tampere at the time of speaking.

\ea%x282
\label{ex:x282}
\gll Isois\"ani \textstyleForeignWords{tuli}  Tampereelle vuonna 1912.        (Finnish)  \\
      \\
\glt
\z

`My grandfather \textstyleEmphasizedWords{came} to Tampere in 1912.' 

The equivalent of the English `come' in (\stepcounter{nx}{\thenx}) has to be `go' in Mauwake (\stepcounter{nx}{\thenx}). This is discussed further in 6.3. 

\ea%x283
\label{ex:x283}
\gll I'll \textstyleEmphasizedWords{\textsc{come t}}o your place tomorrow.  \\
      \\
\glt
\z



\ea%x284
\label{ex:x284}
\gll Uurika  nefa  uruf-owa  \textstyleEmphasizedVernacularWords{ikiw-i-nen}. \\
      \\
\glt
\z

tomorrow  2s.ACC  see-NMZ  go-Np-FU.1s  

`Tomorrow I'll go to see you.'

When these verbs occur with a locative phrase containing the locative marker (\sectref{sec:3.12.4}), the phrase almost always refers to either source (\stepcounter{nx}{\thenx}), or location/path (\stepcounter{nx}{\thenx}).  The goal is very seldom marked with the locative marker -\textstyleStyleVernacularWordsItalic{pa}; this happens when the goal is important mainly as the location of the next event (\stepcounter{nx}{\thenx}). Also, in (\stepcounter{nx}{\thenx}) \textstyleStyleVernacularWordsItalic{mukuna} `fire' is an untypical goal for a directional verb.

\ea%x285
\label{ex:x285}
\gll \textstyleEmphasizedVernacularWords{Manina}\textstyleEmphasizedVernacularWords{=pa  ekap-ep}  maa  uup-e-mik. \\
      \\
\glt
\z

garden=LOC  come-SS.SEQ  food  cook-PA-1/3p  

`We came from the garden and cooked food.'

\ea%x447
\label{ex:x447}
\gll Iinan  aasa  \textstyleEmphasizedVernacularWords{iinan=pa}  fan  \textstyleEmphasizedVernacularWords{ekap-emi}  {\dots} \\
      \\
\glt
\z

sky  canoe  sky=LOC  here  come-SS.SIM

`The airplane came here in the sky and{\dots}'

\ea%x1878
\label{ex:x1878}
\gll Ne  soran-emi  \textstyleEmphasizedVernacularWords{epia  mukuna}\textstyleEmphasizedVernacularWords{=}\textstyleEmphasizedVernacularWords{pa} \\
      \\
\glt
\z

ADD  get.startled-SS.SIM  firewood  fire=LOC  

\textstyleEmphasizedVernacularWords{or}\textstyleEmphasizedVernacularWords{-}\textstyleEmphasizedVernacularWords{omi}  aw-o-k.

descend-SS.SIM  burn-PA-3s

`And he got startled and fell on the fire and burned himself.'

The directional verbs differ from other verbs in Mauwake in that they can be transitivized with the `bring' prefixes \textstyleStyleVernacularWordsItalic{p}-, \textstyleStyleVernacularWordsItalic{amap}- and \textstyleStyleVernacularWordsItalic{aap}- (\sectref{sec:3.8.2.4.2}) to indicate either bringing or taking something somewhere.

\ea%x286
\label{ex:x286}
\gll Ona  owowa  \textstyleEmphasizedVernacularWords{p-ikiw-ep}  soop-i-yan. \\
      \\
\glt
\z

3s.GEN  village  BPx-go-SS.SEQ  bury-Np-FU.1p  

`We'll take him (=his body) in his village and bury him (there).'

The causative suffix \nobreakdash-\textstyleStyleVernacularWordsItalic{ow}\textstyleEmphasizedWords{} (\sectref{sec:3.8.4.3.1})\textstyleEmphasizedWords{} can be added to the roots; when following a one-syllable root the suffix is often reduplicated, but the meaning is still the same as with a single causative suffix.

\ea%x435
\label{ex:x435}
\gll Purowa  ir-\textstyleEmphasizedVernacularWords{ow}-(\textstyleEmphasizedVernacularWords{ow})-eya  siin-ar-e-k. \\
      \\
\glt
\z

armband  go.up-CAUS-CAUS-2/3s.DS  tight-INCH-PA-3s

`She pushed the armband up and it got tight.'

The directional verbs are very frequent as the second root in serial verbs (\stepcounter{nx}{\thenx}) (\sectref{sec:3.8.5.1.2}) and as the main verb in verb plus auxiliary constructions (\stepcounter{nx}{\thenx}) (\sectref{sec:3.8.5.1.1}). Some of them also enter into adjunct plus verb constructions (\stepcounter{nx}{\thenx}) (\sectref{sec:3.8.5.2}). 

\ea%x287
\label{ex:x287}
\gll Wi  Amerika  ``epa  eliwa''  nae-\textstyleEmphasizedVernacularWords{ekap}-e-mik. \\
      \\
\glt
\z

3p.UNM  America  time  good  say-come-PA-1/3p  

`The Americans came saying, ``peace''.'

\ea%x288
\label{ex:x288}
\gll Wi  Yaapan  saa=iw  \textstyleEmphasizedVernacularWords{ir}-am-ika-i-mik. \\
      \\
\glt
\z

3p.UNM  Japan  sand=INST  go-SS.SIM-be-Np-PR.1/3p  

`The Japanese are going along the beach.'

\ea%x289
\label{ex:x289}
\gll Kemuka  \textstyleEmphasizedVernacularWords{pepek  er-}eya  puuk-a-k. \\
      \\
\glt
\z

string  enough  go-2/3s.DS  cut-PA-3s  

`When the string was (long) enough she cut it.'

The meaning of the verbs \textstyleStyleVernacularWordsItalic{ekap}- `come' and \textstyleStyleVernacularWordsItalic{ikiw}- `go' can be metaphorically extended to time, to signal time spans. The former is used when the time span is extended from the past to the present (\stepcounter{nx}{\thenx}), the latter is more common when the time extends from the present to the future (\stepcounter{nx}{\thenx}), but it can also refer to the past (\stepcounter{nx}{\thenx}).

\ea%x290
\label{ex:x290}
\gll Naap  on-am-ik-e-mik,  \textstyleEmphasizedVernacularWords{ekap-ep } aakisa. \\
      \\
\glt
\z

thus  do-SS.SIM-be-PA-1/3p  come-SS.SEQ  now

`We have been doing like that (all the time) up until now.'

\ea%x437
\label{ex:x437}
\gll No  naap  ik-ok  \textstyleEmphasizedVernacularWords{iki(w-e)p}  mokoma  enuma  iiwawun \\
      \\
\glt
\z

2s.UNM  thus  be-SS  go-SS.SEQ  year  new  altogether  

aakun-i-nan.

talk-Np-FU.2s

`You will be like that (long time) but next year you will talk.'

\ea%x291
\label{ex:x291}
\gll Buren  \textstyleEmphasizedVernacularWords{ife-iki}\textstyleEmphasizedVernacularWords{(}\textstyleEmphasizedVernacularWords{w-e}\textstyleEmphasizedVernacularWords{)}\textstyleEmphasizedVernacularWords{p}  aakisa  arim-o-n. \\
      \\
\glt
\z

ceremonial.liquid  rub-go-SS.SEQ  now  grow-PA-2s  

`You have kept rubbing the \textstyleEmphasizedWords{buren}\textit{}  liquid on (for years), and now you have grown up.'

On the fringe of directional verbs are \textstyleStyleVernacularWordsItalic{kerer}- `arrive', \textstyleStyleVernacularWordsItalic{yiaw}-/\textstyleStyleVernacularWordsItalic{miaw}- `walk/move around, wander' and \textstyleStyleVernacularWordsItalic{irapar}- `move back and forth (aimlessly)', which share some of their grammatical characteristics but not all of them. Of these three verbs, \textstyleStyleVernacularWordsItalic{kerer}- cannot be prefixed with the bring-prefixes, but it mainly occurs with an unmarked goal instead of a locative phrase (\stepcounter{nx}{\thenx}).

\ea%x292
\label{ex:x292}
\gll Emeria  mua  manina  \textstyleEmphasizedVernacularWords{kerer-e-mik}. \\
      \\
\glt
\z

woman  man  garden  arrive-PA-1/3p  

`The people arrived in the garden.'

With the other two a bring-prefix is acceptable (\stepcounter{nx}{\thenx}), but they do not take a goal/path argument. If a locative phrase occurs with them it requires a locative clitic (\stepcounter{nx}{\thenx}).

\ea%x293
\label{ex:x293}
\gll Gomi  kawus  \textstyleEmphasizedVernacularWords{p-irapar-i-ya}. \\
      \\
\glt
\z

east.wind  smoke  BPx-move.back.and.forth-Np-PR.3s  

`The east wind moves/blows the smoke around.'

\ea%x436
\label{ex:x436}
\gll Soora=pa  nan  \textstyleEmphasizedVernacularWords{yiaw-e-mik}. \\
      \\
\glt
\z

jungle=LOC  there  walk.around-PA-1/3p  

`They walked around in the jungle.'

\subparagraph[Utterance verbs]{Utterance verbs}
\hypertarget{RefHeading20861935131865}{}
Utterance verbs may be either intransitive (\stepcounter{nx}{\thenx}), ambitransitive (\stepcounter{nx}{\thenx}), (\stepcounter{nx}{\thenx}), or transitive (\stepcounter{nx}{\thenx}).  They may be used to introduce a quote complement, but not to close it. They often occur with one of the `saying' verbs described below (\stepcounter{nx}{\thenx}).

\ea%x309
\label{ex:x309}
\gll Takira  niir-emi  \textstyleEmphasizedVernacularWords{kirir-i-mik}. \\
      \\
\glt
\z

boy  play-SS.SIM  shout-Np-PR.1/3p  

`The boys are playing and shouting.'

\ea%x310
\label{ex:x310}
\gll Wi  iperowa=ke  \textstyleEmphasizedVernacularWords{aakun-ep}  ma-e-mik,  ``{\dots}'' \\
      \\
\glt
\z

3p.UNM  middle.aged=CF  discuss-SS.SEQ  say-PA-1/3p

`The middle-aged men discussed (it) / talked and said, ``{\dots}'' '

\ea%x1927
\label{ex:x1927}
\gll Maapora  kamenap  \textstyleEmphasizedVernacularWords{aakun-i-yan}? \\
      \\
\glt
\z

feast  how  discuss-Np-FU.1p

`How shall we discuss the feast?'

\ea%x311
\label{ex:x311}
\gll Yena  mua  \textstyleEmphasizedVernacularWords{far-e-m},  ``Sarak  oo,  {\dots}'' \\
      \\
\glt
\z

1s.GEN  man  call-PA-1s  Sarak  oh  

`I called to my husband, ``Oh Sarak,{\dots}'' '

The `\textstyleEmphasizedWords{\textsc{saying verbs}}' described in this section below include three, or four, verbs that between them divide the semantic area of `tell/say/speak/think'.  They are frequently used as frame verbs in quote formulas, but they have other functions as well.

maak-/naak-  `tell'

ma-  `say/speak'

na-   `say/speak/think'

The verb \textstyleStyleVernacularWordsItalic{maak}- `tell' is used in the same two main senses as its English equivalent: telling someone \textstyleEmphasizedWords{\textsc{about}} something (\stepcounter{nx}{\thenx}) and telling someone \textstyleEmphasizedWords{\textsc{to do}} something (\stepcounter{nx}{\thenx}). In direct quote formulas it is used mainly preceding a quote (\stepcounter{nx}{\thenx}), not directly following it as a short closing formula. It is not used in indirect quotes at all.

\ea%x312
\label{ex:x312}
\gll Ne  \textstyleEmphasizedVernacularWords{maak-e-mik},  ``Ifa  yia  keraw-i-ya  nain,  {\dots''} \\
      \\
\glt
\z

and  tell-PA-1/3p  snake  1p.ACC  bite-Np-PR.3s  that1

`And they told him, ``When a snake bites us, {\dots}'' '

\ea%x313
\label{ex:x313}
\gll Moma  yia  \textstyleEmphasizedVernacularWords{maak-i-mik}. \\
      \\
\glt
\z

taro  1p.ACC  tell-Np-PR.1/3p  

`They are telling us (to get them) taro roots.'

\ea%x314
\label{ex:x314}
\gll Efa\textstyleFreeTranslationChar{ } \textstyleEmphasizedVernacularWords{maak-ek}\textstyleFreeTranslationChar{, } ``Opora  tep=pa  wu-e.'' \\
      \\
\glt
\z

\textstyleFreeTranslationChar{1s.ACC  tell-PA-3s  talk  tape.recorder=LOC  put-IMP.2s}

`She told me, ``Put the talk on a tape recorder.'' '

When \textstyleStyleVernacularWordsItalic{maak}- closes a direct quote, it requires the manner adverb \textstyleStyleVernacularWordsItalic{naap} `thus' to precede it:

\ea%x315
\label{ex:x315}
\gll ``Aaw-ep  p-ekap-eka,''  \textstyleEmphasizedVernacularWords{naap}  yia \\
      \\
\glt
\z

get-SS.SEQ  BPx-come-IMP.2p  thus  1p.ACC  

\textstyleEmphasizedVernacularWords{maak-em-ik-e-mik}.

tell-SS.SIM-be-PA-1/3p

` ``Bring it'', they were telling us like that.'

The default object for \textstyleStyleVernacularWordsItalic{maak}- is the addressee (\stepcounter{nx}{\thenx}) and a possible second object is the speech itself (\stepcounter{nx}{\thenx}).

\ea%x316
\label{ex:x316}
\gll [Wadol  opora]\textsubscript{O}  [yia]\textsubscript{O}  \textstyleEmphasizedVernacularWords{maak-i-n}. \\
      \\
\glt
\z

lie  talk  1p.ACC  tell-Np-PR.2s  

`You are telling us lies.'

The status of the verb \textstyleStyleVernacularWordsItalic{naak}- is unclear. It is infrequent, and in natural texts only occurs in closing formulas (\stepcounter{nx}{\thenx}). It may have developed as an analogy to the verb pair \textstyleStyleVernacularWordsItalic{ma}-/\textstyleStyleVernacularWordsItalic{na}-.

\ea%x317
\label{ex:x317}
\gll ``No  bom  fain=iw  mera  kuum-e,''  \textstyleEmphasizedVernacularWords{naak-e-mik}. \\
      \\
\glt
\z

2s.UNM  bomb  this=INST  fish  burn-IMP.2s  tell-PA-1/3p  

` ``Blast fish with this bomb,'' they told him.'

With the verb \textstyleStyleVernacularWordsItalic{ma}- `say/speak/tell' the addressee is not in focus, and is hardly ever even mentioned. Instead, the verb requires either an object referring to the speech content (\stepcounter{nx}{\thenx}) or an adverb \textstyleStyleVernacularWordsItalic{naap} `thus' (\stepcounter{nx}{\thenx}) preceding the verb, or a quote complement following it (\stepcounter{nx}{\thenx}). 

\ea%x318
\label{ex:x318}
\gll Yo  yena  yaaya  ifa  ku-o-k  nain  opora \\
      \\
\glt
\z

1s.UNM  1s.GEN  1s/p.uncle  snake  bite-PA-3s  that1  talk  

\textstyleEmphasizedVernacularWords{ma-i-yem.}

say-Np-PR.1s

`I am telling a story about my uncle that was bitten by a snake.'

\ea%x319
\label{ex:x319}
\gll Momora,  no  naap  me  \textstyleEmphasizedVernacularWords{ma-e.} \\
      \\
\glt
\z

Fool  2s.UNM  thus  not  say-IMP.2s  

`Fool, don't say like that.'

\ea%x320
\label{ex:x320}
\gll En-e-mik  na\footnote{Tok Pisin \textit{na} `and' is increasingly used instead of the vernacular additive connective \textit{ne}.}   \textstyleEmphasizedVernacularWords{ma-e-mik},  ``Eliwa,  aara  oposia  saarik.'' \\
      \\
\glt
\z

eat-PA-1/3p  ADD  say-PA-1/3p  good  hen  meat  like

`They ate it and said, ``It is good, like chicken meat.'' '

Occasionally the verb can occur without any of the above objects:

\ea%x321
\label{ex:x321}
\gll Yena  oram  \textstyleEmphasizedVernacularWords{ma-i-yem}. \\
      \\
\glt
\z

1s.GEN  just  say-Np-PR.1s  

`I'm just speaking (without any reason ).'

The difference between the verbs \textstyleStyleVernacularWordsItalic{maak}- and \textstyleStyleVernacularWordsItalic{ma}- in regard to the semantic role of a person object is shown clearly in the next example:

\ea%x322
\label{ex:x322}
\gll Naap  \textstyleEmphasizedVernacularWords{yi}\textstyleEmphasizedVernacularWords{a}  \textstyleEmphasizedVernacularWords{ma-i-}\textstyleEmphasizedVernacularWords{kuan}  na-ep  yo \\
      \\
\glt
\z

thus  1p.ACC  say-Np-FU.3p  think-SS.SEQ  1s.UNM  

ariman \textstyleEmphasizedVernacularWords{} \textstyleEmphasizedVernacularWords{nefa}  \textstyleEmphasizedVernacularWords{maak-i-yem}.

openly  2s.ACC  tell-Np-PR.1s

`Thinking that they will \textstyleEmphasizedWords{\textsc{say}} like that \textstyleEmphasizedWords{\textsc{about us}} I'm openly \textstyleEmphasizedWords{\textsc{telling you}} (this).'

The verb \textstyleStyleVernacularWordsItalic{na}- `say/speak/call/think' is the most interesting of the speech verbs. In quote formulas it is only used for closing the quote (\stepcounter{nx}{\thenx}), with or without another utterance verb in an opening formula.

\ea%x323
\label{ex:x323}
\gll {\dots}\textstyleEmphasizedVernacularWords{ma-em-ik-e-mik},  ``Oo,  {\dots}''  \textstyleEmphasizedVernacularWords{na-em-ik-e-mik}. \\
      \\
\glt
\z

{\dots}say-SS.SIM-be-PA-1/3p  oh  ...  say-SS.SIM-be-PA-1/3p  

`...they kept saying, ``Oh...'',  they kept saying (like that).'

\ea%x942
\label{ex:x942}
\gll Amerika  fan  ``Epa  eliwa''  \textstyleEmphasizedVernacularWords{nae-ekap-e-mik}. \\
      \\
\glt
\z

America  here  time  good  say-come-PA-1/3p

`The Americans came saying ``peace''.'

In a Tail-Head type construction (\sectref{sec:8.2.3.5}) it is often used as a generic verb to replace another utterance verb, when normally the first verb would be repeated.\footnote{Other types of verbs, when not repeated in a Tail-Head construction, are replaced with the generic verb \textstyleFootnoteBaseChar{\textit{on-}} `do'.} 

\ea%x324
\label{ex:x324}
\gll Wia  \textstyleEmphasizedVernacularWords{maak-e-mik},  ``Yia  uf-om-aka.'' \\
      \\
\glt
\z

3p.ACC  tell-PA-1/3p  1p.ACC  dance-BEN-BNFY2.IMP.2p  

\textstyleEmphasizedVernacularWords{Na-iwkin}{\dots}

say-2/3p.DS

`They told them, ``Dance for us.'' When they said (that){\dots}'

When \textstyleStyleVernacularWordsItalic{na}- replaces another utterance verb in that way, the replaced verb may influence what semantic argument becomes the object. In (\stepcounter{nx}{\thenx}) \textstyleStyleVernacularWordsItalic{maak}- requires the addressee of the verb as the default object, and in the following sentence with \textstyleStyleVernacularWordsItalic{na}- the same accusative pronoun \textstyleStyleVernacularWordsItalic{wia}\textit{ }still refers to the addressees, even if with \textstyleStyleVernacularWordsItalic{na}-  it would normally refer to the people spoken about.

\ea%x325
\label{ex:x325}
\gll Ekap-emi  \textstyleEmphasizedVernacularWords{wia}  \textstyleEmphasizedVernacularWords{maak-e-mik},  ``Maa  iiw-eka.'' \\
      \\
\glt
\z

come-SS.SIM  3p.ACC  tell-PA-1/3p  food  dish.out-IMP.2p  

\textstyleEmphasizedVernacularWords{Wia  na-iwkin}  ma-e-mik,  ...

3p.ACC  say-2/3p.DS  say-PA-1/3p

`They\textsubscript{i} came and told them\textsubscript{j}, ``Dish out food.'' They\textsubscript{i} said to them\textsubscript{j} like that and they\textsubscript{j} said, {\dots}'

The verb \textstyleStyleVernacularWordsItalic{na}- is also used in a somewhat different sense `call (by some name)'. In (\stepcounter{nx}{\thenx}) the speaker tells that the word used by the Japanese soldiers for `coconut' was \textstyleStyleVernacularWordsItalic{yasi}, a foreign word for her.\footnote{The verb \textit{unuf}-  is used when the calling by name or giving a name is emphasized.}

\ea%x326
\label{ex:x326}
\gll Iwera  ``yasi''  yia  \textstyleEmphasizedVernacularWords{na-em-ik-e-mik}. \\
      \\
\glt
\z

coconut  yasi  1p.ACC  say-SS.SIM-be-PA-1/3p

`They kept calling coconut (by the name) ``yasi'' to us.' 

The ``speaking'' expressed by \textstyleStyleVernacularWordsItalic{na}- can also be internal speech, i.e. thinking (\stepcounter{nx}{\thenx}). This characteristic is quite common to speech verbs in Papuan languages. When the thinking \textstyleEmphasizedWords{\textsc{process}} itself is more in focus, an adjunct plus verb construction \textstyleStyleVernacularWordsItalic{kema} \textstyleStyleVernacularWordsItalic{suuw}- `think' (literally: `push the liver') is used.

\ea%x327
\label{ex:x327}
\gll Maa  eliwa=ke  \textstyleEmphasizedVernacularWords{na-ep}  aaw-e-m. \\
      \\
\glt
\z

thing  good=CF  say-SS.SEQ  get-PA-1s  

`I thought it was a good thing and got it.'

Related to the inner speech is another usage typical of verbs for `saying' in Papuan languages: to convey desire, intention or plan. For this function only the same subject sequential form\textit{} \textstyleStyleVernacularWordsxiiptItalic{naep} is used, and the verb that indicates the desired or intended  action is in a preceding speech complement clause. This is discussed more fully in the section on complements of utterance verbs (8.3.2.1). 

\ea%x328
\label{ex:x328}
\gll [Yo  manina  urup-i-nen]  \textstyleEmphasizedVernacularWords{na-ep}.  \\
      \\
\glt
\z

1s.UNM  garden  ascend-Np-FU.1s  say-SS.SEQ  

`I want to go to the garden.'

\ea%x329
\label{ex:x329}
\gll [Irak-u]  \textstyleEmphasizedVernacularWords{na-ep}  ikiw-e-mik. \\
      \\
\glt
\z

fight-IMP.1d  say-SS.SEQ  go-PA-1/3p

`They went to fight.' (Lit: ` ``Let's fight'' they said/thought and went.')

\ea%x1608
\label{ex:x1608}
\gll [Ununa  owowa  p-or-owa]  \textstyleEmphasizedVernacularWords{na-ep}  maa  eno-wa \\
      \\
\glt
\z

slit.gong  village  Bpx-descend-NMZ  say-SS.SEQ  food  eat-NMZ  

maneka  on-i-kuan.

big  make-Np-FU.3p

`When they want to take the slit gong down to the village they make a big feast.'

In this function \textstyleStyleVernacularWordsxiiptItalic{naep} is becoming less like a regular medial verb. It can occur in sentence-final position, without being right-dislocated (\stepcounter{nx}{\thenx}).  It usually does retain its word stress, but there is a tendency to un-stress and shorten it by dropping the vowel /a/ in speech (\stepcounter{nx}{\thenx}). When the verb in the speech complement clause is in the counterfactual form, all that is sometimes left of \textstyleStyleVernacularWordsItalic{na-ep} is only the suffix, which is then added as a suffix to the other verb (\stepcounter{nx}{\thenx}).

\ea%x1830
\label{ex:x1830}
\gll Ifana  wu-am-ika-i-kuan,  [unuma  wia  miim-u]   \\
      \\
\glt
\z

ear  put-SS.SIM-be-Np-FU.3p  name  3p.ACC  hear-1d.IMP  

\textstyleEmphasizedVernacularWords{n}\textstyleEmphasizedVernacularWords{-}\textstyleEmphasizedVernacularWords{ep}.

say-SS.SEQ

`They\textsubscript{i} are listening carefully (lit: putting their ear), wanting to hear their\textsubscript{j} names.'

\ea%x1609
\label{ex:x1609}
\gll Yo  aakisa  nanar  nain  \textstyleEmphasizedVernacularWords{ma-ek-a-m-{\O}-ep}. \\
      \\
\glt
\z

1s.UNM  now  story  that1  say-CNTF-PA-1s-{\O}-SS.SEQ

`Now I would like to tell that story.'

The verb \textstyleStyleVernacularWordsItalic{na}- quite freely combines with sound words, and a number of these combinations have been lexicalized (\stepcounter{nx}{\thenx}), (\stepcounter{nx}{\thenx}). The onomatopoeic word has become part of the verb, and the vowel /a/ has been deleted from the verb in the process.

\ea%x330
\label{ex:x330}
\gll Oro-mi  \textstyleEmphasizedVernacularWords{bulak  na-i-ya}\textstyleEmphasizedVernacularWords{\textmd{\textit{.}}} \\
      \\
\glt
\z

drop-SS.SIM  plop  say-Np-PR.3s  

`When it drops it says ``plop''.'

\ea%x331
\label{ex:x331}
\gll Siowa  \textstyleEmphasizedVernacularWords{baun-i-ya}.  ({\textless}  bau  na-i-ya) \\
      \\
\glt
\z

dog  bark-Np-PR.3s  (  bau  say-Np-PR.3s)

`The dog barks.'

\ea%x332
\label{ex:x332}
\gll Ema  \textstyleEmphasizedVernacularWords{buun-eya}  mua  erup  um-e-mik.  ({\textless}  buu  na-eya) \\
      \\
\glt
\z

mountain  erupt-2/3s.DS  man  two  die-PA-1/3p  (  buu  say-2/3s.DS)

`The mountain (=volcano) erupted and two men died.'

In fast speech \textstyleStyleVernacularWordsItalic{na}- is often reduced to \textstyleStyleVernacularWordsItalic{a}- when the verb follows a consonant-final word.

\ea%x333
\label{ex:x333}
\gll ``Uruf-a-mik''  \textstyleEmphasizedVernacularWords{a-e-k}. \\
      \\
\glt
\z

see-PA-1/3p  say-PA-3s  

` ``They saw it,'' he said.'

The medial form \textstyleStyleVernacularWordsItalic{na-eya} is also used as resultative connective `so, therefore' (\sectref{sec:3.11.2}).

\ea%x500
\label{ex:x500}
\gll Iwera  yia  na-em-ik-e-mik.  \textstyleEmphasizedVernacularWords{Naeya}  iwera  wia \\
      \\
\glt
\z

coconut  1p.ACC  say-SS.SIM-be-PA-1/3p  So  coconut  3p.ACC

uruk-am-ik-om-a-mik. 

drop-SS.SIM-be-BEN-BNFY2.PA-1/3p

`They kept speaking to us about coconuts /asking us for coconuts. So we kept dropping coconuts for them.'

\subparagraph[Impersonal experience verbs]{Impersonal experience verbs}
\hypertarget{RefHeading20881935131865}{}
This very small group mainly consists of verbs indicating some kind of pain. They look like transitive verbs, but the syntactic subject is inanimate, usually a body part, and the human experiencer is the object. 

gilin-  `smart (v.)'

kokas-  `itch'

liilin-  `sting'

tiitin-  `hurt, ache (generic)'

tukun-  `throb'

sirir-  `ache'

\ea%x1013
\label{ex:x1013}
\gll Maara  efa  \textstyleEmphasizedVernacularWords{tiitin-i-ya}. \\
      \\
\glt
\z

forehead  1s.ACC  hurt-Np-PR.3s

`My head hurts.'/ `I have a headache.' (Lit: `It hurts my forehead.')

\ea%x1014
\label{ex:x1014}
\gll Uuw-ap  uuw-ap  oona=ke  efa  \textstyleEmphasizedVernacularWords{sirir-i-ya}. \\
      \\
\glt
\z

work-SS.SEQ  work-SS.SEQ  bone=CF  1s.ACC  ache-Np-PR.3s

`I have worked and worked, and my bones ache.'

Most of the experience verbs in Mauwake are adjunct plus verb constructions (\sectref{sec:3.8.5.2.1}); a few are ordinary intransitive verbs (\sectref{sec:3.8.4.2.1}). 

\paragraph[Auxiliary verbs]{Auxiliary verbs}
\hypertarget{RefHeading20901935131865}{}
The small group of auxiliary verbs in Mauwake consists of two ordinary verbs that have also grammaticalized as auxiliaries indicating aspect. In this function the lexical meaning of the verbs is somewhat bleached. The auxiliary is the last verb in a verbal group (\sectref{sec:3.8.5.1}). 

The paradigms of the auxiliaries are similar to those of main verbs. The auxiliary verbs are:

\textstyleAcronymallcaps{AUX:}  \textstyleAcronymallcaps{MEANING:}  \textstyleAcronymallcaps{MAIN VERB FORM:}

ik-    `continuous'  \textstyleAcronymallcaps{SS.SIM}

    `stative'  \textstyleAcronymallcaps{SS.SEQ}

pu- ({\textless}wu-)  `completive'   \textstyleAcronymallcaps{SS.SEQ}

The auxiliary \textstyleStyleVernacularWordsItalic{ik}- is very frequent and has several functions.  When it is used with a main verb in the same-subject simultaneous form (\textstyleAcronymallcaps{SS.SIM}), it indicates continuous aspect, which can have either progressive (\stepcounter{nx}{\thenx}) or habitual (\stepcounter{nx}{\thenx}) meaning. For position-taking verbs (\sectref{sec:3.8.4.4.2}) and other semantically punctiliar verbs the habitual interpretation is the only possible one, but for other verbs the context is needed to determine the correct interpretation. 

\ea%x339
\label{ex:x339}
\gll Fikera  aw-em-\textstyleEmphasizedVernacularWords{ik}-eya  uruf-a-k.  (progressive) \\
      \\
\glt
\z

kunai.grass  burn-SS.SIM-be-2/3s.DS  see-PA-3s  

`When the kunai grass was burning she saw it.' (Or: `She saw the kunai grass burning.')

\ea%x340
\label{ex:x340}
\gll I  yabuela  aaw-ep  {\dots}  wi-em-\textstyleEmphasizedVernacularWords{ik}-e-mik.  (habitual) \\
      \\
\glt
\z

1p.UNM  papaya  get-SS.SEQ  {\dots}  give.them-SS.SIM-be-PA-1/3p  

`We kept getting papayas and {\dots} giving them to them.'

When the main verb is in the same-subject sequential form (\textstyleAcronymallcaps{SS.SEQ}), the auxiliary \textstyleStyleVernacularWordsItalic{ik}- indicates stativity (\stepcounter{nx}{\thenx}). With non-punctiliar verbs this form can often be translated into English with a past perfect (\stepcounter{nx}{\thenx}).

\ea%x341
\label{ex:x341}
\gll Pok-ap-\textstyleEmphasizedVernacularWords{ik}-emkun  epa  wiim-o-k.  (stative) \\
      \\
\glt
\z

sit.down-SS.SEQ-be-1s/p.DS  place  dawn-PA-3s  

`As I was sitting it became light.'

\ea%x342
\label{ex:x342}
\gll Ikiw-ep-\textstyleEmphasizedVernacularWords{ik}-eya  ona  emeria=ke  ekap-o-k.  (perfect) \\
      \\
\glt
\z

go-SS.SEQ-be-2/3s.DS  3s.GEN  woman=CF  come-PA-3s  

``After he was/had gone his wife came.'

The auxiliary \textstyleStyleVernacularWordsItalic{pu}- `completive', is obviously derived from \textstyleStyleVernacularWordsItalic{wu}- `put'\footnote{`Put' is one of the verbs commonly used in Papuan languages to indicate completion \citep[145]{Foley1986}.} through assimilation with the final /p/ of the same-subject sequential form in the main verb preceding it. Synchronically, the Mauwake speakers do not recognise the origin of the auxiliary.

\ea%x343
\label{ex:x343}
\gll Maa  en-ep-\textstyleEmphasizedVernacularWords{pu}-ap  soomar-eka. \\
      \\
\glt
\z

food  eat-SS.SEQ-CMPL-SS.SEQ  walk-IMP.2p  

`Having finished eating you may go.' (Lit: `Eat the food and go'.)

\ea%x501
\label{ex:x501}
\gll Nan  efa  wu-ap-\textstyleEmphasizedVernacularWords{pu}-ami  o  Ulingan  ikiw-o-k. \\
      \\
\glt
\z

there1  1s.ACC  put-SS.SEQ-CMPL-SS.SIM  3s.UNM  Ulingan  go-PA-3s

`He left (lit: put) me there and went to Ulingan.'

\subsubsection{Verbal clusters}
\hypertarget{RefHeading20921935131865}{}
The verbal clusters are described here under verb morphology, because they function as a unit very much like single verbs. There are two kinds of verbal clusters: verbal groups and adjunct plus verb constructions. The definition of a verbal group is from \citet[175]{Halliday1985}:  ``a sequence of words in the primary class of verb''. 

\ea%x347
\label{ex:x347}
\gll Ifara  \textstyleEmphasizedVernacularWords{mokak-ikiw-em-ik-ok}  ifara  oko  uruf-a-k. \\
      \\
\glt
\z

vine  stare-go-SS.SIM-be-SS  vine  other  see-PA-3s

`He kept looking for a vine and saw one vine.'

Adjunct plus verb combinations\footnote{\citet[184]{Halliday1985} calls these ``phrasal verbs''.}  contain a verb (or a verbal group) plus an element from another word class that is obligatory and contributes to the meaning of the verb. 

\ea%x348
\label{ex:x348}
\gll Owora  efar  \textstyleEmphasizedVernacularWords{ikum  aaw-iwkin}  wia  maak-e-m. \\
      \\
\glt
\z

betelnut  1s.DAT  illicitly  get-2/3p.DS  3p.ACC  tell-PA-1s

`They stole my betelnut and I talked to them.'

The status of a verb phrase in Mauwake is somewhat questionable. It is discussed in \sectref{sec:4.5}. 

\paragraph[Verbal groups]{Verbal groups}
\hypertarget{RefHeading20941935131865}{}
A verbal group consists of two or more verbs that function grammatically and semantically as one unit. The semantic unity within the group varies between different types of verbal groups. 

The verbal groups containing a main verb plus auxiliary have developed by merging clauses as can still be seen from the verbs involved. But since they synchronically function as a unit very much like an individual verb they are treated on the word level. Features that identify them as one close-knit unit are as follows:


\begin{itemize}
\item Shared subject (and object, if relevant)

\item No non-verbal elements intervening between the parts

\item Scope of negation spans over the whole group

\item No coordinators are allowed between the parts

\item Phrasal intonation and pause structure, i.e. no pauses between the words.


\end{itemize}
Mauwake has two kinds of verbal groups. The verbs in the first group consist of a main verb and an aspectual auxiliary. The second group consists of serial verbs, where all the verb stems contribute to the semantic, rather than grammatical, meaning of the verb.

\subparagraph[Main verb plus auxiliary: aspect]{Main verb plus auxiliary: aspect}
\hypertarget{RefHeading20961935131865}{}
The importance of tense as a verbal category in Mauwake  shows in its obligatory morphological marking, but aspect is a relatively important category as well. Aspects are `\textstyleBibliogCitationAAAstyleChar{different ways of viewing the internal temporal constituency of a situation}' \citep[3]{Comrie1976}. 

Aspect in Mauwake is expressed periphrastically, through verbal groups that have a main verb and an auxiliary. The main verb, which is in the medial form, largely gives the semantic content to the whole, and the auxiliary adds the grammatical meaning of aspect. In the continuous and stative aspects also the medial form of the \textstyleEmphasizedWords{\textsc{main}} verb contributes to the aspectual meaning. What distinguishes these constructions from medial clauses (8.2) is that the two verbs function as a unit rather than individual verbs, and their phonological stress, intonation and pause pattern is that of a word or phrase rather than a medial clause. 

 As is typical of \textstyleAcronymallcaps{\textup{SOV}} languages, the auxiliary follows the main verb (Greenberg 1966:85, Dryer 2007a:90). The more common of the aspectual auxiliaries is \textstyleStyleVernacularWordsxiiptItalic{ik}- `be', which can combine with two different medial forms. The other aspectual auxiliary is  \textstyleStyleVernacularWordsxiiptItalic{pu}- `completive' (\sectref{sec:3.8.4.5}). 

The neutral, aspectually unmarked verb form is used in Mauwake whenever the speaker chooses not to pay special attention to the internal structure of the situation. It could be claimed that this is a neutral perfective, since the situation is viewed as a whole, but that term would be confusing, as the neutral forms can also be used in clauses that are aspectually habitual (cf. Payne 1997:239). The majority of the verb forms used in all kinds of texts in Mauwake are aspectually neutral.

The marked completive aspect is only used when completion of an action is stressed. The continuous aspect is used for both progressive and habitual actions, and the stative aspect for a state continuing over some time.

{\bfseries
\hypertarget{RefHeading20981935131865}{}
Completive aspect}

When the \textstyleEmphasizedWords{\textsc{completion}} of an action is in focus, the completive aspect is used. It is formed by a main verb in the same-subject sequential form, followed by the auxiliary \textstyleStyleVernacularWordsItalic{pu}- `completive' (\sectref{sec:3.8.4.5}).

\ea%x361
\label{ex:x361}
\gll Ifakim-ep  nomokow  ekeka=pa  \textstyleEmphasizedVernacularWords{sererim-ep-pu-a-k}. \\
      \\
\glt
\z

kill-SS.SEQ  tree  branch=LOC  hang-SS.SEQ-CMPL-PA-3s  

`He killed it and hung it on a tree branch.'

The completive aspect verb is often used in a medial same-subject sequential form, which in itself only indicates sequentiality but often implies completion of the first action as well. 

\ea%x362
\label{ex:x362}
\gll \textstyleEmphasizedVernacularWords{Sererim-ep-pu-ap}  owowa  or-o-k. \\
      \\
\glt
\z

hang-SS.SEQ-CMPL-SS.SEQ  village  descend-PA-3s

`He hung it up and went/came down to the village.'

\ea%x1041
\label{ex:x1041}
\gll Manina  \textstyleEmphasizedVernacularWords{n}\textstyleEmphasizedVernacularWords{op-ap-pu-ap}  nomokowa  war-i-mik. \\
      \\
\glt
\z

garden  burn-SS.SEQ-CMPL-SS.SEQ  tree  cut-Np-PR.1/3p

`We burn (the undergrowth for new) garden and (when it is done we) cut the trees.'

\ea%x1042
\label{ex:x1042}
\gll Nomokowa  \textstyleEmphasizedVernacularWords{war-ep-pu-ap}  arew-i-mik. \\
      \\
\glt
\z

tree  cut-SS.SEQ-CMPL-SS.SEQ  wait-Np-PR.1/3p

`We cut the trees and wait.'

But it is not uncommon either to have the completive aspect with a simultaneous action medial form, when the second action coincides with the completion of the first one:

\ea%x363
\label{ex:x363}
\gll Wia  \textstyleEmphasizedVernacularWords{maak-ep-pu-ami}  i  ikiw-e-mik. \\
      \\
\glt
\z

3p.ACC  tell-SS.SEQ-CMPL-SS.SIM  1p.UNM  go-PA-1/3p

`We told them and went.' 

\ea%x1040
\label{ex:x1040}
\gll Aria  yo  nan  efa  \textstyleEmphasizedVernacularWords{wu-ap-pu-ami}  o \\
      \\
\glt
\z

alright  1s.UNM  there  1s.ACC  put-SS.SEQ-CMPL-SS.SIM  3s.UNM

Ulingan  ikiw-o-k.

Ulingan  go-PA-3s

`Alright he put me there and he went to Ulingan.'

\ea%x1043
\label{ex:x1043}
\gll Maa  en-owa  \textstyleEmphasizedVernacularWords{wakesim-ep-pu-ami}  ikiw-o-k. \\
      \\
\glt
\z

thing  eat-NMZ  cover-SS.SEQ-CMPL-SS.SIM  go-PA-3s

`Covering the food she left.'

The completive aspect form is also used when \textstyleEmphasizedWords{\textsc{momentaneity}} of the action is emphasized:

\ea%x364
\label{ex:x364}
\gll \textstyleEmphasizedVernacularWords{En-ep-pu-ap}  ikiw-e! \\
      \\
\glt
\z

eat-SS.SEQ-CMPL-SS.SEQ  go-IMP.2s

`Get done with your eating and go!'

The origin of the auxiliary, the verb `put', shows in the fact that it cannot be used with non-controlled actions.\footnote{In general, control vs. non-control is not a prominent feature in the verb system in Mauwake, unlike many other Papuan languages (Foley 1986:127, Reesink 1987:128).}

\ea%x365
\label{ex:x365}
\gll *Waki-ep-pu-a-k \\
      \\
\glt
\z

fall-SS.SEQ-CMPL-PA-3s

In process descriptions a medial verb, followed by the verb \textstyleStyleVernacularWordsItalic{weeser}- `finish', which stresses the endpoint of the action, is used more than the completive aspect. This, however, is a case of clause chaining (8.2), not a verbal group.

\ea%x366
\label{ex:x366}
\gll Uup-ep  \textstyleEmphasizedVernacularWords{weeser-eya}  wienak-e-m. \\
      \\
\glt
\z

cook-SS.SEQ  finish-2/3s.DS  feed.them-PA-1s

`I finished cooking it and fed it to them.' [Lit: `I cooked it and when it (=the cooking) was finished I fed it to them.']

{\bfseries
\hypertarget{RefHeading21001935131865}{}
Continuous aspect: progressive and habitual}

Continuity, or duration, is the semantic component shared by the aspects called progressive and habitual in many languages: continuation of the same action or of repeated actions of the same kind \citep[26]{Comrie1986}. The continuous aspect form in Mauwake can have either progressive (\stepcounter{nx}{\thenx}), (\stepcounter{nx}{\thenx}) or habitual (\stepcounter{nx}{\thenx}), (\stepcounter{nx}{\thenx}) interpretation. The main verb is in the same-subject simultaneous medial form, but with the final /i/ deleted, and the auxiliary \textstyleStyleVernacularWordsItalic{ik}- `be' is inflected for tense and person/number (\stepcounter{nx}{\thenx}). 

\ea%x349
\label{ex:x349}
\gll Maa  \textstyleEmphasizedVernacularWords{en-em-ik-omkun}  ama  or-o-k. \\
      \\
\glt
\z

food  eat-SS.SIM-be-1s/p.DS  sun  descend-PA-3s  

`As I was eating the sun went down.'

\ea%x1044
\label{ex:x1044}
\gll Fikera  \textstyleEmphasizedVernacularWords{aw-em-ik-eya}  nain  umuk-i-nen \\
      \\
\glt
\z

kunai.grass  burn-SS.SIM-be-2/3s.DS  that1  extinguish-Np-FU.1s

na-ep  urup-o-k.

say-SS.SEQ  ascend-PA-3s

`The kunai grass was burning, and she went up in order to extinguish it.'

\ea%x350
\label{ex:x350}
\gll Iwera=ke  wia  aruf-eya  \textstyleEmphasizedVernacularWords{ma-em-ik-e-mik},  ``{\dots''} \\
      \\
\glt
\z

coconut=CF  3p.ACC  hit-2/3s.DS  say-SS.SIM-be-PA-1/3p  

`When coconuts hit them, they kept saying, `` ...'' '

\ea%x1045
\label{ex:x1045}
\gll Wi  Yaapan  naap  kuisow=iw  \textstyleEmphasizedVernacularWords{ekap-em-ik-e-mik}. \\
      \\
\glt
\z

3p.UNM  Japan  thus  one=INST  come-SS.SIM-be-PA-1/3p

`The Japanese kept coming like that, one by one.'

For punctiliar verbs the habitual interpretation (\stepcounter{nx}{\thenx}) is the only one possible, whereas for non-punctiliar verbs both habitual and progressive interpretations  are possible.

\ea%x351
\label{ex:x351}
\gll Koka=pa  nan  \textstyleEmphasizedVernacularWords{in-em-ik-e-mik.} \\
      \\
\glt
\z

jungle=LOC  there  lie.down-SS.SIM-be-PA-1/3p

`We kept sleeping in the jungle'

\ea%x1932
\label{ex:x1932}
\gll Owowa  oko  wiam=iya  \textbf{irak-em-ik-e-mik}. \\
      \\
\glt
\z

village  other  3p.ACC=COM  fight-SS.SIM-be-PA-1/3p

`We were fighting (or: kept fighting repeatedly) with the other village.'

  When the verbal group is in the medial form, the progressive interpretation (\stepcounter{nx}{\thenx}) is the more common:

\ea%x353
\label{ex:x353}
\gll Waaya  \textstyleEmphasizedVernacularWords{urup-em-ik-eya } mik-a-m. \\
      \\
\glt
\z

pig  ascend-SS.SIM-be-2/3s.DS  spear-PA-1s

`As the pig was going/coming up I speared it.'

Often the context provides the only clue as to whether the continuous aspect form should be interpreted as progressive or habitual. The example (\stepcounter{nx}{\thenx}) describes a situation where the villagers kept feeding the Japanese soldiers who asked them for food; the sentence (\stepcounter{nx}{\thenx}) is from a text describing a coconut plantation fire and its consequences.

\ea%x354
\label{ex:x354}
\gll Waaya  yia  na-iwkin  waaya  \textstyleEmphasizedVernacularWords{wienak-em-ik-e-mik}. \\
      \\
\glt
\z

pig  1p.ACC  say-2/3p.DS  pig  feed.them-SS.SIM-be-PA-1/3p

`They asked us for pigs and we kept giving them pigs to eat.'

\ea%x355
\label{ex:x355}
\gll Kawus  \textstyleEmphasizedVernacularWords{ir-am-ik-eya}  kuuf-a-k. \\
      \\
\glt
\z

smoke  rise-SS.SIM-be-2/3s.DS  see-PA-3s

`The smoke was rising and she saw it.'

Cross-linguistically the habitual aspect more commonly receives overt marking in the past tense than in the present \citep[154]{Cristofaro2006}. In Mauwake the continuous aspect can be used for habitual in any of the three tenses. The example (\stepcounter{nx}{\thenx}) was said about particular work that the speaker was not involved in continuously; he used to do it time to time because of his position as need arose. The example (\stepcounter{nx}{\thenx}) refers to a couple needing to keep visiting an ailing father. 

\ea%x1063
\label{ex:x1063}
\gll Yo  anane  maneka  naap  \textstyleEmphasizedVernacularWords{mauw-am-ika-i-yem}. \\
      \\
\glt
\z

1s.UNM  always  very  thus  work-SS.SIM-be-Np-PR.1s

`I always/forever keep working like that.'

\ea%x1064
\label{ex:x1064}
\gll O  me  sariar-i-non-(na)  neeke  \textstyleEmphasizedVernacularWords{in-em-ika-i-kuan}. \\
      \\
\glt
\z

3s.UNM  not  get.well-Np-FU.3s-(TP)  there.CF  sleep-SS.SIM-be-Np-FU.3p

`If he doesn't get well, they will keep sleeping/staying \textit{there}.' 

For a clause to have habitual interpretation it is not obligatory to use the continuous aspect form in the verb. For instance in process descriptions, which tell how something is habitually done, the unmarked, aspectually neutral present tense form is more common than the continuous aspect. Three of the four verbs in (\stepcounter{nx}{\thenx}) are aspectually unmarked, although all the clauses have habitual interpretation, describing seclusion customs.

\ea%x1049
\label{ex:x1049}
\gll Moma  ik-owa  \textstyleEmphasizedVernacularWords{enim-i-mik}.  Eka  me  \textstyleEmphasizedVernacularWords{enim-i-mik},  iwer \\
      \\
\glt
\z

taro  roast-NMZ  eat-Np-PR.1/3p  water  not  eat-Np-PR.1/3p  coconut

eka  me  \textstyleEmphasizedVernacularWords{enim-i-mik}.  Aaya  muutiw  \textstyleEmphasizedVernacularWords{en-em-ika-i-mik}.

water  not  eat-Np-PR.1/3p  sugarcane  only  eat-SS.SIM-be-Np-PR.1/3p

`We do not eat roasted taro. We do not drink water or coconut water. We only eat / keep eating sugarcane.'

{\bfseries
\hypertarget{RefHeading21021935131865}{}
Stative aspect}

The same semantic component of continuity is also shared by the other aspect using the auxiliary \textstyleStyleVernacularWordsxiiptItalic{ik}- `be': this time it is a \textstyleEmphasizedWords{\textsc{state}} rather than activity that continues the same over time. In the stative aspect the auxiliary is combined with a main verb that is in the same-subject sequential form. This usage is most common with the position-taking verbs like \textstyleStyleVernacularWordsItalic{pok}- `sit down', \textstyleStyleVernacularWordsItalic{iimar}- `stand up' and \textstyleStyleVernacularWordsItalic{in}- `lie down/ fall asleep'.

\ea%x356
\label{ex:x356}
\gll \textstyleEmphasizedVernacularWords{Pok-ap-ik-omkun } epa  wiim-o-k. \\
      \\
\glt
\z

sit.down-SS.SEQ-be-1s/p.DS  place  dawn-PA-3s  

`As we were sitting it dawned.'

\ea%x1046
\label{ex:x1046}
\gll Yena  koor  miira=pa  \textstyleEmphasizedVernacularWords{iimar-ep-ik-e-m},  {\dots} \\
      \\
\glt
\z

1s.GEN  house  face=LOC  stand.up-SS.SEQ-be-PA-1s

`I was standing in front of my house, {\dots}'

Other punctiliar verbs (\stepcounter{nx}{\thenx}), as well as non-punctiliar verbs can be used in this aspect to indicate the state resulting from an action (\stepcounter{nx}{\thenx}), or process (\stepcounter{nx}{\thenx}), but they are less frequent.

\ea%x357
\label{ex:x357}
\gll Ifakim-eya  \textstyleEmphasizedVernacularWords{pu-ep-ik-eya } om-em-ik-ua. \\
      \\
\glt
\z

kill-2/3s.DS  die-SS.SEQ-be-2/3s.DS  cry-SS.SIM-be-PA.3s

`When she killed him and he was dead, she was crying.'

\ea%x358
\label{ex:x358}
\gll \textstyleEmphasizedVernacularWords{Ikiw-ep-ik-eya}  ona  emeria=ke  ekap-o-k. \\
      \\
\glt
\z

go-SS.SEQ-be-2/3s.DS  3s.GEN  woman=CF  come-PA-3s

`While he was gone his wife came.'

\ea%x1047
\label{ex:x1047}
\gll Ewar  pun  wuun-e-k  ne  epa  \textstyleEmphasizedVernacularWords{reen-ep-ik-ua}. \\
      \\
\glt
\z

west.wind  too  blow-PA-3s  and  place  dry-SS.SEQ-be-PA.3s

`The west wind blew, too, and the ground was dry.'

In the example (\stepcounter{nx}{\thenx}) the continuous form indicates more active waiting process than is the case in (\stepcounter{nx}{\thenx}) with the stative aspect.  In (\stepcounter{nx}{\thenx}) the people were getting impatient with the vehicle that should already have come to get them. The example (\stepcounter{nx}{\thenx}) is from a description of garden work, and part of the work process is the state of patiently waiting for the felled trees and undergrowth to dry. 

\ea%x359
\label{ex:x359}
\gll Arew\textstyleEmphasizedVernacularWords{-am-}ik-omkun  ama  ikur  miiw-aasa  kerer-ek. \\
      \\
\glt
\z

wait-SS.SIM-be-1s/p.DS  sun  five  land-canoe  arrive-PA-3s

`As we were waiting the car arrived at five.'

\ea%x360
\label{ex:x360}
\gll Nomokowa  war-ep-pu-ap  arew\textstyleEmphasizedVernacularWords{-ap-}ika-iwkin \\
      \\
\glt
\z

tree  cut-SS.SEQ-CMPL-SS.SEQ  wait-SS.SEQ-be-2/3p.DS

reen-eya  saama  kuum-i-mik.

dry-2/3s.DS  cleared.bush  burn-Np-PR.1/3p

`They cut the trees and while they are waiting it dries and then they burn the cleared bush.' 

\subparagraph[Serial verbs ]{Serial verbs} 
\hypertarget{RefHeading21041935131865}{}
Verbal groups called serial verbs are very common in Papuan languages \citep[116]{Foley1986}. Finding a cross-linguistic definition for serial verbs has proved to be an extremely hard task (Sebba 1987:5, Lord 1993:1). Instead of one definition covering all the possible serial verbs, \citet[19]{Crowley2002} suggests defining these verbs within ``\textstyleBibliogCitationAAAstyleChar{specific typological and linguogenetic groupings}'' for comparative purposes. 

For a working definition I borrow one given by \citet[28]{James1983} describing the serial verbs in Siane, another Papuan language: 

``A serial verb construction consists of two or more verbs which occur in series with neither normal coordinating nor subordinating markers, which share at least some core argument (normally subject and/or object/goal), and which in some sense function together semantically like a single predication''. 

Typically, even if not obligatorily, one of the verbs in the series is finite and the other(s) more or less ``stripped-down''.  In a verb-final language the finite verb is the last one in the series. After describing the serial verb construction in Mauwake I will discuss the question whether serial verbs are actually compound verbs, and the relationship of the serial verbs to main verb + auxiliary verbal groups and medial clauses.

In Mauwake a non-final verb in a serial construction consists of a bare root without any inflection at all. This restriction is tighter than those given for serial verbs in many other languages (Crowley 2002:19, Sebba 1987:86-7, James 1983:28). Each of the verbs in a serial construction contribute to the overall semantic meaning of the predicate. Even if the meaning is not exactly the same as the combination of the same verbs would have in a tight medial verb chain (cf. Payne 1997:310), it does not get bleached either, like that of the auxiliaries.\footnote{Since a serial verb construction has only one main stress it is written as one word in the orthography, but the verb stems are separated by hyphens to make reading easier.}

\ea%x377
\label{ex:x377}
\gll Sama=pa  \textstyleEmphasizedVernacularWords{oro-boon-ek}. \\
      \\
\glt
\z

ladder=LOC  descend-get.loose-PA-3s

`He fell from the ladder.'

The last verb in a series is either a finite verb with tense and person/number inflection, or a medial verb. The arguments are shared by the whole verbal complex, even if they would originally have been associated with only one of the verbs (\stepcounter{nx}{\thenx}). Also negation and obliques (\stepcounter{nx}{\thenx}) are shared. All this points to serial verbs being a nuclear-level phenomenon in Mauwake, rather than a core-level one (Foley and Van Valin 1984:189-193). 

\ea%x378
\label{ex:x378}
\gll Yo  Amerika  wia  \textstyleEmphasizedVernacularWords{akup-ikiw-i-yem}. \\
      \\
\glt
\z

1s.UNM  America  3p.ACC  search-go-Np-PR.1s  

`I am going to look for the Americans. / I go searching the Americans.' 

\ea%x379
\label{ex:x379}
\gll Neeke  \textstyleEmphasizedVernacularWords{aw(e)-or-om-ik-eya}  {\dots} \\
      \\
\glt
\z

there.CF  burn-descend-SS.SIM-be-2/3s.DS

`As it was burning (towards) down \textit{there}{\dots}'

Semantically the verb combinations are of two types. In the more common one a directional or another motion verb follows another verb stem, giving the meaning of \textstyleEmphasizedWords{\textsc{movement}} to the whole (\stepcounter{nx}{\thenx})-(\stepcounter{nx}{\thenx}), and often the meaning of \textstyleEmphasizedWords{\textsc{directionality}} as well (\stepcounter{nx}{\thenx})-(\stepcounter{nx}{\thenx}).\footnote{Cross-linguistically motion and location verbs are very common in serial verbs \citep[9]{Lord1993}.} This is a productive process, as long as the verbs are semantically compatible.

\ea%x438
\label{ex:x438}
\gll Wia  \textstyleEmphasizedVernacularWords{mokak-urup-o-k},  wia  \textstyleEmphasizedVernacularWords{mokak-or-o-k}. \\
      \\
\glt
\z

3p.ACC  stare-ascend-PA-3s  3p.ACC  stare-descend-PA-3s

`He stared them up and down.'

\ea%x381
\label{ex:x381}
\gll Aasa  \textstyleEmphasizedVernacularWords{suuw-or-o-mik}. \\
      \\
\glt
\z

canoe  push-descend-PA-1/3p  

`We pushed the canoe down (towards the sea).'\footnote{Compare this with a medial construction:  \textstyleFootnoteBaseChar{\textit{Aasa suuw-ap or-o-mik}}  `We pushed the canoe and went down (to sea)'}

If the first stem is also a motion verb, it indicates the \textstyleEmphasizedWords{\textsc{manner}} of movement:

\ea%x380
\label{ex:x380}
\gll Merena  kir-ep  \textstyleEmphasizedVernacularWords{segen-ikiw-o-k}. \\
      \\
\glt
\z

foot  turn-SS.SEQ  limp-go-PA-3s  

`He twisted his foot and limped.'

A motion verb in a serial construction can also indicate \textstyleEmphasizedWords{\textsc{temporal continuity}} over a long period of time. In (\stepcounter{nx}{\thenx}) the length of time is emphasized even more by the repetition of the motion verb.

\ea%x439
\label{ex:x439}
\gll \textstyleEmphasizedVernacularWords{Ife-iki}(w-e)\textstyleEmphasizedVernacularWords{p  iki}(w-e)\textstyleEmphasizedVernacularWords{p}  aakisa  arim-o-n. \\
      \\
\glt
\z

rub-go-SS.SEQ  go-SS.SEQ  now  grow-PA-3s

`You kept rubbing it (over the years) and now you have grown up.'

In the second type, any two verbs can, in principle, combine into a serial verb. But this process is less productive, and both the type and token frequency of this type is low when compared with the frequency of the first type. Usually, like in (\stepcounter{nx}{\thenx}) the meaning of the whole is transparent and can be inferred from the meanings of the component roots, but sometimes the semantics are more opaque (\stepcounter{nx}{\thenx}).

\ea%x382
\label{ex:x382}
\gll Emera  \textstyleEmphasizedVernacularWords{kue-puuk-ap}  okaiwi  siowa  onak-e-k. \\
      \\
\glt
\z

sago  bite-cut-SS.SEQ  other.side  dog  feed.him-PA-3s  

`He bit off half of the sago cake and fed it to the dog.'

\ea%x383
\label{ex:x383}
\gll Aakun-emi  \textstyleEmphasizedVernacularWords{mika-kof-a-m}. \\
      \\
\glt
\z

speak-SS.SIM  spear-knock-PA-1s  

`I stumbled in my speech.'

This type of serialization in Mauwake is very close to what \citet[1-5]{James1983} calls \textstyleEmphasizedWords{\textsc{lexical}} serialization. 

A special case among the roots forming serial verbs is \textstyleStyleVernacularWordsItalic{afur}- `do well'/`augmentative', which is not used as an independent verb, only as a second element in a serial verb structure.\footnote{See James 1983:32 for the use of a similar verb, \textstyleFootnoteBaseChar{\textit{ito,}} in Siane.}

\ea%x384
\label{ex:x384}
\gll Koora  ku-owa  \textstyleEmphasizedVernacularWords{amis-ar-afur-a-k}. \\
      \\
\glt
\z

house  build-NMZ  knowledge-INCH-do.well-PA-3s  

`He really knew how to build a house.'

It is quite possible even if not very common to form a three-root serial verb by combining the two types:

\ea%x385
\label{ex:x385}
\gll \textstyleEmphasizedVernacularWords{Mika-fien-ikiw-o-k}. \\
      \\
\glt
\z

hit-push.aside-go-PA-3s  

`He went on countering (an attack).'

It is far more common to have three verbs in a combination where an auxiliary is attached to a serial verb:

\ea%x386
\label{ex:x386}
\gll Naap  \textstyleEmphasizedVernacularWords{amis-ar-ikiw-em-ik-o-wen}. \\
      \\
\glt
\z

thus  knowledge-INCH-go-SS.SIM-be-Np-FU.2p  

`That way you will gain more and more knowledge.'

Combining four or more roots into one verbal group is more of a theoretical possibility than a practical reality.  Examples are easy enough to obtain through elicitation, but very rare in non-elicited texts.

Mauwake does \textstyleEmphasizedWords{\textsc{not}} use serial verbs for a benefactive like many languages do \citep[174-80]{Sebba1987}; it utilizes benefactive morphology for that purpose (\sectref{sec:3.8.2.3.3}, 3.8.3.1). Neither is the serial verb structure used for aspect, as a verb plus auxiliary construction takes care of that. Another function often associated with serial verbs is that of instrument marking, but for that Mauwake uses either an ordinary switch-reference construction or an adverbial phrase (\sectref{sec:4.6.3}).

Distinguishing serial verbs from compound verbs on the one hand and medial clauses on the other is not a problem for Mauwake only, as  serial verbs can behave very much like either \citep[17]{Crowley2002}. Crowley suggests the following continuum of gradually loosening syntactic juncture: verbal compounds {\textgreater} nuclear serial verbs {\textgreater} core serial verbs {\textgreater} clause chains {\textgreater} subordinate clauses {\textgreater} coordinate clauses (ibid. 18). In the following I will briefly discuss the relationship of serial verbs to adjunct plus verb constructions, to verbal groups consisting of a main verb plus auxiliary, and to medial clauses in Mauwake. 

The serial verbs in Mauwake show the following characteristics of compounding (cf. James 1983:69 regarding Papuan languages). The first verb appears as a mere root (or as a stem, if it has undergone derivation); secondly, the verbs obligatorily share the same arguments; thirdly, the meaning of the whole may differ from the combined meanings of the parts. Furthermore, the stress and intonation contour of a serial verb is that of a single word rather than that of a phrase or a clause. There are two main reasons for calling them serial verbs. The first one is that especially the first type is productive. I also want to link them to a typologically widespread phenomenon instead of looking at them from a strictly language-specific point of view. In this I follow \citet[101]{Margetts1999}, who maintains that ``\textstyleBibliogCitationAAAstyleChar{the term `compound' does not by definition contradict an analysis as serialization}''. A similar position is also strongly defended by \citet[16]{Crowley2002} and by Giv\'on (1991:17).

Because of the tight restriction of ``root only'' for the first element in a serial verb in Mauwake, the main verb plus auxiliary combinations are left outside the group by definition. Another reason for this differential treatment is the fact that different processes seem to be going on in the two groups: grammaticalization in the main verb + \textstyleAcronymallcaps{AUX} group, lexicalization in the true serial verbs.\footnote{In some other languages main verb + AUX constructions are included among serial verbs (e.g. James 1983: 29, Crowley 2002:178).  Farr notes the ``staging'' aspects of the two constructions: in medial verbs the temporal relationship of the two verbs may be specified, but as ``the verbal constituents of SVCs [serial verb constructions] do not specify temporal borders or overlapping relationships, the events they represent can blend into a unit {\dots} and present the SVC is a complex but integrated event'' (1999:174).} 

In Mauwake the clause chaining is structurally midway between serialization and main clause coordination, and may consequently be used instead of either in some cases. The instrumental may in Mauwake be expressed by a `take-instrument-do' structure (\stepcounter{nx}{\thenx}) which in many serializing languages is a serial verb construction \citep[162-74]{Sebba1987}; but in Mauwake there is no good reason to call the structure anything other than a combination of a medial and final clause. This shows more clearly in example (\stepcounter{nx}{\thenx}), which does not pass the rule for verbal groups: ``no non-verbal elements between the parts''. 

\ea%x387
\label{ex:x387}
\gll Fura  \textstyleEmphasizedVernacularWords{aaw-ep}  puuk-a-m. \\
      \\
\glt
\z

knife  take-SS.SEQ  cut-PA-1s  

`I took a knife and cut it.' Or: `I cut it with a knife.'

\ea%x388
\label{ex:x388}
\gll Burir  aaw-ep  nomokowa  unowa  war-e-mik. \\
      \\
\glt
\z

axe  take-SS.SEQ  tree  many  fell-PA-1/3p  

`We took an axe and felled many trees.' Or: `We felled many trees with an axe.'

For Mauwake, I propose the following continuum where the syntactic juncture gradually loosens: serial verb {\textgreater} verb + \textstyleAcronymallcaps{AUX} group {\textgreater} subordinate+main clause {\textgreater} clause chain {\textgreater} coordinate main clauses.

The borderline between serial verbs and medial verbs on the one hand, and between verb + \textstyleAcronymallcaps{AUX} groups and medial verbs on the other is not absolutely clear-cut. In (\stepcounter{nx}{\thenx}) the medial verb structure is used instead of a serial verb, even though the two actions are simultaneous, not sequential as indicated by the form of the medial verb.\footnote{Mauwake does not allow same subject simultaneous forms following each other except in a strictly coordinate structure where the verbs do not so much indicate simultaneity with each other as with the final verb.}  

\ea%x389
\label{ex:x389}
\gll Wi  Malala=ke  \textstyleEmphasizedVernacularWords{muf-ep  ekap-emi}{\dots} \\
      \\
\glt
\z

3p.UNM  Malala=TP  pull-SS.SEQ  come-SS.SIM  

`The Malala people came pulling it and{\dots}'

Likewise, the four verbs in (\stepcounter{nx}{\thenx}) describe \textstyleEmphasizedWords{\textsc{one}} protracted action in spite of the sequential form in the medial verbs:

\ea%x390
\label{ex:x390}
\gll Ifa  nain  \textstyleEmphasizedVernacularWords{murar-ep  wiok-ap  ekap-ep} \\
      \\
\glt
\z

snake  that1  follow-SS.SEQ  follow.them-SS.SEQ  come-SS.SEQ

\textstyleEmphasizedVernacularWords{ekap-ep}  owowa  kerer-ek.

come-SS.SEQ  village  arrive-PA-3s

`The snake kept following them and arrived in the village.'

The main verb in verb plus \textstyleAcronymallcaps{AUX} combinations has to be in medial form. The only exception found is the continuous aspect form of the verb \textstyleStyleVernacularWordsItalic{wiaw}- `move around'. The mere root of this verb is used when it is the second verb in a serial structure which then takes an aspectual auxiliary:

\ea%x391
\label{ex:x391}
\gll Ifara  mufe-\textstyleEmphasizedVernacularWords{wiaw}-ik-ok{\dots} \\
      \\
\glt
\z

vine  pull-move.around-be-SS  

`As he was pulling the vine around{\dots}'

\paragraph[Adjunct plus verb constructions ]{Adjunct plus verb constructions} 
\hypertarget{RefHeading21061935131865}{}
Papuan languages typically enlarge their verb inventories through adjunct plus verb combinations \citep[127]{Foley1986}. Foley only discusses nominal adjuncts, but adverbial adjuncts are commonly used in these structures as well. 

Mauwake is not nearly as productive in the use of the adjunct plus verb  construction as many other Papuan languages. Some of them use almost exclusively generic verbs (Foley 1986:117, Roberts 1987:309, Whitehead 2004:145), whereas others employ a larger set of verbs \citep[62-66]{Farr1999} in these constructions.\footnote{Farr divides the nominals in these constructions into `complements' an `adjuncts'. Korafe does not seem to use adverbial adjuncts in these structures. } 

\subparagraph[Nominal adjunct plus verb]{Nominal adjunct plus verb}
\hypertarget{RefHeading21081935131865}{}
The nominal adjuncts look like object \textstyleAcronymallcaps{NP}s, and the origin of at least some of them probably is in object \textstyleAcronymallcaps{NP}s, but currently there are syntactic and semantic differences between the two. An object \textstyleAcronymallcaps{NP} may be separated from the verb by the negator adverb \textstyleStyleVernacularWordsItalic{me}  or by an accusative or a dative pronoun, but a nominal adjunct must immediately precede the verb. The meaning of the nominal adjunct plus verb construction often cannot be derived from the meanings of its constituent parts.  

\ea%x450
\label{ex:x450}
\gll Meta  yia  miim-ap  yia  \textstyleEmphasizedVernacularWords{miira  puuk-ekap-e-mik}. \\
      \\
\glt
\z

fame  1p.ACC  hear-SS.SEQ  1p.ACC  face  cut-come-PA-1/3p

`They heard about us and came to greet us.'

An object \textstyleAcronymallcaps{NP} only occurs with a transitive verb, but a nominal adjunct can also occur with an intransitive verb:

\ea%x451
\label{ex:x451}
\gll Uura  or-op  \textstyleEmphasizedVernacularWords{arua  karu-e-mik}. \\
      \\
\glt
\z

night  descend-SS.SEQ  torch  run-PA-1/3p

`At night we went down to sea and fished with a torch.'

Those nominal adjunct plus verb structures where the verb is transitive look like two-object clauses, and in a few cases behave like them syntactically. In (\stepcounter{nx}{\thenx}) the nominal adjunct \textstyleStyleVernacularWordsItalic{kema} `liver' is in its normal adjunct position, but in (\stepcounter{nx}{\thenx}) it is in object \textstyleAcronymallcaps{NP} position. The basic meanings of the two sentences are the same, but with a different prominence: (\stepcounter{nx}{\thenx}) encodes marked negative focus and (\stepcounter{nx}{\thenx}) verb focus. The clause (\stepcounter{nx}{\thenx}) with an initial theme pronoun \textstyleStyleVernacularWordsItalic{yo} `I' is pragmatically more neutral than the others except in cases where  the initial pronoun receives extra stress. Note the intervening negator also in (\stepcounter{nx}{\thenx}). 

\ea%x452
\label{ex:x452}
\gll Me  efa  \textstyleEmphasizedVernacularWords{kema  suuw-a-k}. \\
      \\
\glt
\z

not  1s.ACC  liver  push-PA-3s

`He did \textstyleEmphasizedWords{\textsc{not}} think of me.'

\ea%x453
\label{ex:x453}
\gll \textstyleEmphasizedVernacularWords{Kema}  me  efa  \textstyleEmphasizedVernacularWords{suuw-a-k}. \\
      \\
\glt
\z

liver  not  1s.ACC  push-PA-3s

`He didn't \textstyleEmphasizedWords{\textsc{think}} of me.'

\ea%x1874
\label{ex:x1874}
\gll Yo  me  efa  \textstyleEmphasizedVernacularWords{kema  suuw-a-k}. \\
      \\
\glt
\z

1s.UNM  not  1s.ACC  liver  push-PA-3s

`He didn't think of me.'

In cases where the adjunct only occurs with a certain verb it is difficult to give it a specified meaning apart from the verb. The same is true for verbs that do not occur independently, only with an adjunct.

\ea%x454
\label{ex:x454}
\gll \textstyleEmphasizedVernacularWords{Naruw  ir-a-mik}. \\
      \\
\glt
\z

?  ascend-PA-1/3p

`They acted silly.'

\ea%x455
\label{ex:x455}
\gll Naap  \textstyleEmphasizedVernacularWords{kema  tuup-am-ika-i-ya}. \\
      \\
\glt
\z

thus  liver  ?-SS.SIM-be-Np-PR.3s

`He is hoping so.'

Most of the verbs in Mauwake indicating physiological or  psychological states and cognition are nominal adjunct plus verb constructions. The verb takes the person marking from the experiencer. The following list gives only a small sample of these constructions, where the most common nominal is \textstyleStyleVernacularWordsItalic{kema} `liver'.\footnote{A good list of these is in \citet[47-63]{Kwan1989}, where she has described a large number of body image concepts formed with \textit{kema} from semantic point of view. For that study the syntactic characteristics of the structures were not relevant.} The second column provides a literal translation. A few more examples of these constructions are in the sentences (\stepcounter{nx}{\thenx})-(\stepcounter{nx}{\thenx}).

kema enekar-  liver catch.fire  `be thirsty'

kema kaalal-  liver float  `be enthusiastic'

kema korin-  liver get.stuck  'be confused'

kema peelal-  liver rot  `be grieved'

kema ten-  liver collapse  `be relieved'

eneka maayar-  tooth become.long  `be hungry for meat'

miira ikiw-  face go  `feel dizzy'

\ea%x1490
\label{ex:x1490}
\gll Uura  \textstyleEmphasizedVernacularWords{uroma  ikiw-e-m}. \\
      \\
\glt
\z

night  stomach  go-PA-1s

`Last night I had diarrhea.'

\ea%x1487
\label{ex:x1487}
\gll \textstyleEmphasizedVernacularWords{Kema  samor-ar-ep}  maa  me  enim-i-yem. \\
      \\
\glt
\z

liver  spoil-INCH-SS.SEQ  food  not  eat-Np-PR.1s

`I am sad and don't eat.'

\ea%x1488
\label{ex:x1488}
\gll ...oko  \textstyleEmphasizedVernacularWords{emina} \textstyleEmphasizedVernacularWords{} \textstyleEmphasizedVernacularWords{urur}\textstyleEmphasizedVernacularWords{-}\textstyleEmphasizedVernacularWords{ep}  soomar-ikiw-i-kuan. \\
      \\
\glt
\z

...other  occiput  drop-SS.SEQ  walk-go-Np-FU.3p

`{\dots}lest they feel ashamed and walk away.'

\ea%x1489
\label{ex:x1489}
\gll Muuka  gelemuta  akena  \textstyleEmphasizedVernacularWords{kema}  me  \textstyleEmphasizedVernacularWords{puk}\textstyleEmphasizedVernacularWords{-}\textstyleEmphasizedVernacularWords{e}\textstyleEmphasizedVernacularWords{-}\textstyleEmphasizedVernacularWords{mik}. \\
      \\
\glt
\z

son  small  very  liver  not  burst-PA-1/3p

`Little boys/children do not think well (yet).'

\subparagraph[Adverbial adjunct plus verb]{Adverbial adjunct plus verb}
\hypertarget{RefHeading21101935131865}{}
Adverbial adjuncts also have to precede the verb without any intervening words.  

\ea%x456
\label{ex:x456}
\gll Maamuma  efar  \textstyleEmphasizedVernacularWords{ikum  aaw-e-mik}. \\
      \\
\glt
\z

money  1s.DAT  illicitly  get-PA-1/3p

`They stole money from me.'

\ea%x457
\label{ex:x457}
\gll Maa  me  efa  \textstyleEmphasizedVernacularWords{pepek  er-a-k}. \\
      \\
\glt
\z

food  not  1s.ACC  enough  go-PA-3s

`The food wasn't enough for me.'

Some of the adverbial adjuncts, like \textstyleStyleVernacularWordsItalic{ikum} `illicitly' (\stepcounter{nx}{\thenx}) and \textstyleStyleVernacularWordsItalic{pepek} `enough' (\stepcounter{nx}{\thenx}), also function as independent adverbs, shown by an intervening pronoun (\stepcounter{nx}{\thenx}) and/or negator (\stepcounter{nx}{\thenx}).

\ea%x458
\label{ex:x458}
\gll Yo  oram  \textstyleEmphasizedVernacularWords{ikum}  efa  wu-a-n. \\
      \\
\glt
\z

1s.UNM  for.nothing  illicitly  1s.ACC  put-PA-2s

`You accused me for theft without grounds.'

\ea%x459
\label{ex:x459}
\gll No  \textstyleEmphasizedVernacularWords{pepek}  me  ma-e-n. \\
      \\
\glt
\z

2s.UNM  enough  not  say-PA-2s

`You didn't say right.'

Other adjuncts like \textstyleStyleVernacularWordsItalic{ane} `together' and \textstyleStyleVernacularWordsItalic{anu} `apart', only combine with verbs to form verbal groups, and it is hard to give them an exact meaning; the glosses below are just approximations.

\ea%x460
\label{ex:x460}
\gll Apura  \textstyleEmphasizedVernacularWords{ane  suuw-am-ika-iwkin}  pok-ap  ik-ok \\
      \\
\glt
\z

widow  together  push-SS.SIM-be-2/3p.DS  sit.down-SS.SEQ  be-SS  

om-o-k.

cry-PA-3s

`They were supporting the widow (sitting against her back) and she sat and wailed.'

\ea%x461
\label{ex:x461}
\gll Opora  \textstyleEmphasizedVernacularWords{anu  fien-owa}  me  pepek. \\
      \\
\glt
\z

talk  apart/aside  brush.off-NMZ  not  enough  

`He wasn't able to disregard the talk.'

It was mentioned above that the meanings of the adjunct plus verb combinations are often idiomatic rather than analytically derivable from the meanings of the parts. But this is a somewhat dangerous statement for one to make who comes from outside the speech community. For example, how literally \textstyleStyleVernacularWordsItalic{kema} `liver', which figures very strongly in the adjunct plus verb constructions, is understood to be really involved in the emotional and cognitive processes would need to be established in a separate study.

\subsection{Adverbs}
\hypertarget{RefHeading21121935131865}{}
Adverbs in Mauwake are a heterogeneous class morphologically, syntactically and semantically. Schachter's (1985:20) definition of adverbs as words functioning ``\textstyleBibliogCitationAAAstyleChar{as modifiers of constituents other than nouns}'' is quite usable for Mauwake. Functionally the adverbs can be divided into four groups. The \textstyleEmphasizedWords{\textsc{material}} adverbs \citep{Ahlman1933}\footnote{Ahlman used the term in classifying adverbs in Finnish, and I find it useful in describing the adverbs in Mauwake as well, since the temporal, locative and manner adverbs share some characteristics which differentiate them from the other adverbs.} form the largest group, which contains the subgroups of locative, temporal and manner adverbs. The second group, that of \textstyleEmphasizedWords{\textsc{intensity}} adverbs,\footnote{In some grammars these form a class of their own, called ``intensifiers''. But that name is somewhat misleading as it may contain words like \textstyleFootnoteBaseChar{\textit{somewhat}} or \textstyleFootnoteBaseChar{\textit{hardly}} which do not intensify the meaning of the adjacent adjective or adverb.} consists of a small group of adverbs that function on phrase level and modify an adjective or adverb. \textstyleEmphasizedWords{\textsc{Sentential}} (or \textstyleEmphasizedWords{\textsc{modal}}) adverbs modify a whole sentence. The last group consists of the two \textstyleEmphasizedWords{\textsc{free}} adverbs \textstyleStyleVernacularWordsItalic{pun} `also' and \textstyleStyleVernacularWordsItalic{muutiw} `only'.

A material adverb may function as the head of an adverbial phrase. In this respect, however, adverbs differ from most other word classes: whereas the head of a \textstyleAcronymallcaps{NP} is usually a noun, that of a \textstyleAcronymallcaps{VP} a verb and an \textstyleAcronymallcaps{AP} an adjective, an adverbial phrase typically either consists of an adverb only, or does not contain an adverb word at all (\sectref{sec:4.6}.). The material and sentential adverbs may be modified by an intensity adverb, in particular by \textstyleStyleVernacularWordsItalic{akena} `very, truly' (\stepcounter{nx}{\thenx}).

\ea%x462
\label{ex:x462}
\gll \textstyleEmphasizedVernacularWords{baliwep}  \textstyleEmphasizedVernacularWords{akena} \\
      \\
\glt
\z

well  very

`very well'

The position of adverbs within a clause is also discussed under adverbial phrase (\sectref{sec:4.6}).

\subsubsection{Material adverbs}
\hypertarget{RefHeading21141935131865}{}
The material adverbs function as peripherals in a clause. They are divided into locative, temporal, and manner adverbs.  The temporal and manner adverbs may be subdivided into deictic and non-deictic adverbs, and the locative adverbs are practically all deictic; in this they differ from the intensity and modal adverbs, which cannot be deictic.

\paragraph[Locative adverbs]{Locative adverbs}
\hypertarget{RefHeading21161935131865}{}
All the non-controversial locative adverbs are deictic, and they were discussed above in section on spatial deictics (\sectref{sec:3.6.3}). 

\ea%x1933
\label{ex:x1933}
\gll {\dots}mokoma  kuisow  naap  \textstyleEmphasizedVernacularWords{fan}  yiam=iya  ik-e-mik. \\
      \\
\glt
\z

year  one  thus  here  1p.REFL=COM  be-PA-1/3p

`{\dots}for about a year they were here with us.'

\ea%x1934
\label{ex:x1934}
\gll {\dots}mua  owawiya  \textstyleEmphasizedVernacularWords{neeke}  ik-ok  uruf-ap{\dots}  kiiriw  ep-i-kuan. \\
      \\
\glt
\z

man  with  there.CF  be-SS  see-SS.SEQ  again  come-Np-FU.3p

`{\dots}having been with her husband there and seeing [her father] they will come (back) again.'

The words that are formed with a noun plus the locative clitic \nobreakdash-\textstyleStyleVernacularWordsxiiptItalic{pa} are treated as (adverbial) locative phrases, since they are expandable.

The words \textstyleStyleVernacularWordsxiiptItalic{mamaiya} `near, close' and \textstyleStyleVernacularWordsItalic{epasia} \footnote{\textit{Epasia} has probably developed from \textit{epa asia} `wild place'.} `far (away)' are actually locative nouns, but may be in the process of becoming adverbs. They optionally take the locative clitic \nobreakdash-\textstyleStyleVernacularWordsItalic{pa}, but its presence or absence causes no semantic difference. \textstyleStyleVernacularWordsItalic{Tiil}  `edgewise, close' cannot take the locative clitic. Its use is quite restricted, and it might be more accurately classified as a manner adverb. 

\ea%x467
\label{ex:x467}
\gll \textstyleEmphasizedVernacularWords{Epasia}  ikiw-em-ik-omkun  yia  far-e-k. \\
      \\
\glt
\z

far  go-SS.SIM-be-1s/p.DS  1p.ACC  call-PA-3s

`As we were (still) walking at a distance, he called us.'

\ea%x469
\label{ex:x469}
\gll Fikera  \textstyleEmphasizedVernacularWords{mamaiya=pa}  nan  pok-ap  {\dots} \\
      \\
\glt
\z

kunai.grass  near=LOC  there  sit-SS.SEQ  

`Having sat there near the kunai grass {\dots}'

\ea%x1856
\label{ex:x1856}
\gll Mua  oko=ke  \textstyleEmphasizedVernacularWords{mamaiya}  pok-a-k. \\
      \\
\glt
\z

man  other=CF  near  sit-PA-3s

`Another man slept with her (lit: sat near).'

\ea%x470
\label{ex:x470}
\gll Saapipia  baliwep  me  wu-a-m,  \textstyleEmphasizedVernacularWords{tiil}  wu-a-m. \\
      \\
\glt
\z

trap  well  not  put-PA-1s  on.edge  put-PA-1s

`I didn't put the trap well, I put it right on the edge (of the reef).'

Locative expressions that in some other languages would be expressed through pre- or postpositions or adverbs are formed with locative phrases containing locative relational nouns in Mauwake. 

\ea%x468
\label{ex:x468}
\gll koor  \textstyleEmphasizedVernacularWords{kuenuma}\textstyleEmphasizedVernacularWords{=pa} \\
      \\
\glt
\z

house  underside=LOC

`underneath (lit: in/on the underside of) the house'

\paragraph[Temporal adverbs]{Temporal adverbs}
\hypertarget{RefHeading21181935131865}{}
The temporal adverbs can be classified semantically as deictic or non-deictic. The meaning of the former is tied to the time of the utterance, whereas the meaning of the latter is independent of it.  Both the deictic and non-deictic temporal adverbs are either specific or non-specific. This grouping is relevant on the syntactic level, as it influences the ordering of multiple temporal adverbials within a clause (\sectref{sec:4.6.2}).

\textstyleEmphasizedWords{\textsc{Deictic}}\textsc{} \textstyleEmphasizedWords{\textsc{specific}} temporal adverbs refer to a certain day in relation to the time of the utterance.\footnote{The only exception to this in the data is \textstyleFootnoteBaseChar{\textit{uurika}}, which in the forms \textstyleFootnoteBaseChar{\textit{uurik ona}}\textstyleEmphasizedVernacularWords{} (lit: `tomorrow place') and \textstyleFootnoteBaseChar{\textit{uurika naap nain}} (lit: `tomorrow thus that') means `the following day' and takes the time of the event as the deictic centre.} They are the following:

aakisa\footnote{\textstyleFootnoteBaseChar{\textit{Aakisa}} `today, now' may be either specific or non-specific.}  `today' 

unan  `yesterday'

erekema  `the day before yesterday'

uurika  `tomorrow'

ere    `the day after tomorrow'

arowona  `third day from today'

\ea%x471
\label{ex:x471}
\gll \textstyleEmphasizedVernacularWords{Unan}  nainiw  yiam  fiirim-e-mik. \\
      \\
\glt
\z

Yesterday  again  1p.REFL  gather-PA-1/3p

`Yesterday we met again.'

\ea%x472
\label{ex:x472}
\gll \textstyleEmphasizedVernacularWords{Uurika}  emeria  manina  ikiw-ep  en-owa  nop-ap \\
      \\
\glt
\z

tomorrow  woman  garden  go-SS.SEQ  eat-NMZ  fetch-SS.SEQ  

or-eka.

descend-IMP.2p

`You women, go to the garden tomorrow and fetch food (and come) down.'

The \textstyleEmphasizedWords{\textsc{deictic non-specific temporals}} refer to a time that is related to the time of the utterance (or in some cases to the time of the event), but is not restricted to a certain day. 

aakisa  `now'

aakisa fain  `nowadays,  now', literally: `now this'

aakisa fan  `just a while ago,  just now (past)', literally: `now here'

aakisa kuisow  `right now,  in a minute' (future), literally: `now one'

eewuar  `not yet'

iirakuma  `a few days ago'

iiriw  `already,  earlier,  long ago'

iiriwiw  `long time ago'

ikoka  `later'

ikoka kuisow  `right now' (future), literally: `later one'

uurik ona  `the following day', literally: `tomorrow place'

wiimar  `later, some other time' 

\ea%x473
\label{ex:x473}
\gll Aria,  no  \textstyleEmphasizedVernacularWords{aakisa}  maa  enim-e. \\
      \\
\glt
\z

alright  2s.UNM  now  thing/food  eat-IMP.2s

`Alright, eat now.'

\ea%x1215
\label{ex:x1215}
\gll \textstyleEmphasizedVernacularWords{Eewuar, } eka  me  saanar-owa  ik-ua. \\
      \\
\glt
\z

not.yet  water  not  dry-NMZ  be-PA.3s

`Not yet, the water hadn't dried.'

\ea%x474
\label{ex:x474}
\gll No  emeria  \textstyleEmphasizedVernacularWords{iiriw}  sesek-a-mik. \\
      \\
\glt
\z

2s.UNM  woman  already  send-PA-1/3p

`We already sent your wife (away).'

Both \textstyleStyleVernacularWordsItalic{ikoka} and \textstyleStyleVernacularWordsItalic{wiimar } mean `later', and they can occasionally be used interchangeably. \textstyleStyleVernacularWordsItalic{Ikoka} is the more common of the two, and has to be used when referring to a later time the same day.  \textstyleStyleVernacularWordsItalic{Wiimar}  always refers to a less specific time somewhere in the future, but the use of \textstyleStyleVernacularWordsItalic{ikoka} is spreading to cover that too. The sentence (\stepcounter{nx}{\thenx}) is from a wedding speech, and it was unlikely that the young couple would be fighting later the very same day.

\ea%x476
\label{ex:x476}
\gll \textstyleEmphasizedVernacularWords{Wiimar}  ikiw-i-yan,  \textstyleEmphasizedVernacularWords{ikoka}  weetak. \\
      \\
\glt
\z

later  go-Np-FU.1p  later  no

`We'll go some other time, not later today.'

\ea%x477
\label{ex:x477}
\gll No  \textstyleEmphasizedVernacularWords{ikoka}  mua  ikos  irak-ep  me  efar  kerer-e. \\
      \\
\glt
\z

2s.UNM  later  man  with  fight-SS.SEQ  not  1s.DAT  arrive-IMP.2s

`Later when you fight with your husband, don't come to me.'

\textstyleStyleVernacularWordsItalic{Aakisa} `now' can be modified to further specify the meaning, as the exact present moment is so short that a word referring to it only is practically useless. \textstyleStyleVernacularWordsItalic{Aakisa kuisow} (lit: `now one') refers to something that \textstyleEmphasizedWords{\textsc{will take place}} `just now', in a moment (\stepcounter{nx}{\thenx}), \textstyleStyleVernacularWordsItalic{aakisa fan} (lit: `now here') refers to something that \textstyleEmphasizedWords{\textsc{has happened}} just now (\stepcounter{nx}{\thenx}) and \textstyleStyleVernacularWordsItalic{aakisa fain} (lit: `now this') compares the present situation with earlier times (\stepcounter{nx}{\thenx}).

\ea%x478
\label{ex:x478}
\gll \textstyleEmphasizedVernacularWords{Aakisa  kuisow}  on-e,  ikoka  weetak. \\
      \\
\glt
\z

now  one  do-IMP.2s  later  no

`Do it right now, not later.'

\ea%x479
\label{ex:x479}
\gll Muuna  kirip-owa  ma-e-mik  nain  \textstyleEmphasizedVernacularWords{aakisa  fan}  kirip-a-mik. \\
      \\
\glt
\z

debt  return-NMZ  say-PA-1/3p  that1  now  here  return-PA-1/3p

`They (only) just now returned the debt they have talked about returning.'

\ea%x480
\label{ex:x480}
\gll Iiriw  miiw-aasa  marew,  \textstyleEmphasizedVernacularWords{aakisa  fain}  miiw-aasa  nepik  akena. \\
      \\
\glt
\z

earlier  land-canoe  none  now  this  land-canoe  crowd  real

`Earlier there were no cars, now(adays) there are lots of cars.'

The interpretation of the \textstyleEmphasizedWords{\textsc{non-deictic}} temporals is not tied to the time of the utterance or to the time of the event. The following ones are \textstyleEmphasizedWords{\textsc{specific}}:

uuriw  `morning'

amirika  `day(time), noon'

urera  `(late) afternoon'

uura  `evening/night'

uur gonegon\footnote{The word \textit{gonegon}, which I have not come across elsewhere, is a partial reduplication of the locative noun \textit{gone} `middle'. As a reduplication it is unusual in that the partial reduplication follows rather than precedes the root.}  `midnight'

epa wiiwim\footnote{This is a back-formation of the expression \textit{epa wii-wiim-ik-ua} [place RDP-dawn-be-PA.3s] `It is/was beginning to dawn'.}  `close to dawn' 

\ea%x698
\label{ex:x698}
\gll \textstyleEmphasizedVernacularWords{Amirika}  ama  kekan-eya  uurar-i-mik. \\
      \\
\glt
\z

day  sun  strong-2/3.DS  rest-Np-PR.1/3p

`During the day (or: at noon) when the sun is strong, we take a rest.'

\ea%x699
\label{ex:x699}
\gll Yaapan=ke  \textstyleEmphasizedVernacularWords{uura}  ifera=pa  nan  pok-om-ow-a-mik. \\
      \\
\glt
\z

Japan=CF  evening/night  sea=LOC  there  sit-BEN-CAUS-PA-1/3p

`In the evening the Japanese made him sit in the sea.'

The following temporal adverbs are both \textstyleEmphasizedWords{\textsc{non-deictic}} and \textstyleEmphasizedWords{\textsc{non-specific}}:

aawurun  `forever'

anane  `always', `every day'

ewur  `soon, quickly, fast'

ewursow  `soon, at once'

iir oko  `once upon a time, at some point' (lit: `(an)other time')

kiikir  `first'

kiiriw  `again'

mokomokoka  `first'

nainiw  `again'    ({\textless}nain=iw)

muri\footnote{This is an Austronesian borrowing (M. Ross, p.c.). It also occurs in the verb \textit{murar-} `follow', which has grammaticalized from the adjunct plus verb compound \textit{muri ar-} `behind become'.}  `later, behind'

\ea%x502
\label{ex:x502}
\gll Yo  \textstyleEmphasizedVernacularWords{anane}  naap  mauw-am-ika-i-yem. \\
      \\
\glt
\z

I  always  thus  work-SS.SIM-be-Np-PR.1p

`I always work like that.'

\ea%x504
\label{ex:x504}
\gll Irak-owa  maneka  \textstyleEmphasizedVernacularWords{ewur}  me  imen-ar-e-k. \\
      \\
\glt
\z

fight-NMZ  big  quickly  not  find-INCH-PA-3s

`The big fight/war didn't start quickly.'

\textstyleStyleVernacularWordsItalic{Kiiriw} and \textstyleStyleVernacularWordsItalic{nainiw}  both mean `again', and they can be used interchangeably when referring to repeated action. 

\ea%x697
\label{ex:x697}
\gll Ne  \textstyleEmphasizedVernacularWords{nainiw}  sande  uura  yiam  fiirim-e-mik. \\
      \\
\glt
\z

ADD  again  Sunday  evening  1p.REFL  gather-PA-1/3p

`And again on Sunday evening we gathered together.'

\ea%x1762
\label{ex:x1762}
\gll Ne  \textstyleEmphasizedVernacularWords{kiiriw}  enuma  on-am-ik-e-mik. \\
      \\
\glt
\z

ADD  again  new  make-SS.SIM-be-PA-1/3p

`And again they kept making a new one.'

When some action or event results in a state that is the same or similar as before, even if the action itself is not repeated, only\textstyleStyleVernacularWordsItalic{} \textstyleStyleVernacularWordsItalic{kiiriw} can be used. Thus only \textstyleStyleVernacularWordsItalic{kiiriw} is possible in (\stepcounter{nx}{\thenx}). \textstyleStyleVernacularWordsItalic{Kiiriw} indicates that Jesus is alive again, as he had been before, whereas \textstyleStyleVernacularWordsItalic{nainiw}  would indicate that he had risen from the dead earlier too. 

\ea%x503
\label{ex:x503}
\gll Yeesus  \textstyleEmphasizedVernacularWords{kiiriw}  iikir-a-k. \\
      \\
\glt
\z

Jesus  again  rise-PA-3s

`Jesus rose again (= rose from the dead).'

Also, if the action is the same but the situation changes, \textstyleStyleVernacularWordsItalic{kiiriw} is used. The example (\stepcounter{nx}{\thenx}) describes a situation where a grandmother first sent her younger grandchild, a girl, to listen to a sound. Later she sent the grandson for the same errand; the act of sending was repeated but the person who was sent changed:

\ea%x1761
\label{ex:x1761}
\gll \textstyleEmphasizedVernacularWords{Kiiriw}  morena  iperowa  nain  sesek-a-k. \\
      \\
\glt
\z

again  male  older  that1  send-PA-3s

`Again she sent the elder male (grandchild).'

Occasionally \textstyleStyleVernacularWordsItalic{kiiriw} and \textstyleStyleVernacularWordsItalic{nainiw}  can be used together:

\ea%x700
\label{ex:x700}
\gll Ar-ep  ik-eya  aria  \textstyleEmphasizedVernacularWords{kiiriw}  mua  nain \\
      \\
\glt
\z

become-SS.SEQ  be-2/3s.DS  alright  again  man  that1

\textstyleEmphasizedVernacularWords{nainiw}  urup-o-k.

again  ascend-PA-3s

`When she had become like that, alright the man came up again.'

\paragraph[Manner adverbs]{Manner adverbs}
\hypertarget{RefHeading21201935131865}{}
The manner adverbial phrase is often manifested by just an adverb word rather than a longer phrase. The same distinction between deictic and non-deictic adverbs that was made with the other material adverbs can be made with the manner adverbs as well. The description of the deictic manner adverbs is in 3.6.4.

\ea%x1935
\label{ex:x1935}
\gll ...maa  oposia  pun  \textstyleEmphasizedVernacularWords{naap}  sesek-a-mik. \\
      \\
\glt
\z

thing  meat  also  thus  sell-PA-1/3p

`{\dots}like that they also sold meat.'

\ea%x1936
\label{ex:x1936}
\gll Soo  nain  \textstyleEmphasizedVernacularWords{feenap}:  era  erup  ik-ua. \\
      \\
\glt
\z

fishtrap  that1  like.this  way  two  be-PA.3s

`The fishtrap (custom) is like this: there are two ways.'

A few of the non-deictic manner adverbs have been derived from adjectives by the deletion of word-final /a/, but this process is not productive. Below is a list of some of the more common non-deictic manner adverbs.

ariman  `openly, publicly'

baliwep/balisow  `well'

damol/samor  `badly, poorly' (from \textstyleStyleVernacularWordsItalic{damola/samora} `bad')

ewur/ewuriw  `quickly'

ikum  `illicitly'

kapi  `askew'

kaken/kakeniw  `straight, correctly'

kekelka  `quietly, gently'

kerew  `strongly'

kokot  `secretly'

momasia  `slowly'  (cf.  adjective \textstyleStyleVernacularWordsItalic{momasia} `slow')

momor  `indiscriminately', `foolishly' (from \textstyleStyleVernacularWordsItalic{momora} `foolish')

pepek  `correctly'

oram/moram  `without reason', `without doing anything'\footnote{This word is difficult to gloss in English; its meaning is close to that of Tok Pisin \textit{nating}.}

orawin  `for the benefit'

\ea%x704
\label{ex:x704}
\gll Naap  yia  ma-i-kuan  na-ep  yo  \textstyleEmphasizedVernacularWords{ariman} \\
      \\
\glt
\z

thus  1p.ACC  say-Np-FU.3p  say/think-SS.SEQ  1s.UNM  openly

nefa  maak-i-yem.

2s.ACC  tell-Np-PR.1s

`I am telling you this openly, thinking that they will say like that about us.'

\ea%x505
\label{ex:x505}
\gll Opaimika  \textstyleEmphasizedVernacularWords{baliwep}  me  wiar  amis-ar-e-m. \\
      \\
\glt
\z

talk  well  not  3.DAT  knowledge-INCH-PA-1s

`I don't/didn't know their language well.'

\ea%x507
\label{ex:x507}
\gll Fikera  \textstyleEmphasizedVernacularWords{ikum}  kuum-e-mik  nain  ma-i-yem. \\
      \\
\glt
\z

kunai.grass  illicitly  burn-PA-1/3p  that1  say-Np-PR.1s

`I tell about that when the kunai grass was burned by arson.'

\ea%x506
\label{ex:x506}
\gll \textstyleEmphasizedVernacularWords{Samor}  akena  aruf-a-mik. \\
      \\
\glt
\z

badly  very  hit-PA-1/3p

`They beat him very badly.'

\subsubsection{Intensity adverbs}
\hypertarget{RefHeading21221935131865}{}
Intensity adverbs are a small and heterogeneous group of adverbs that modify a verb, an adjective, a quantifier or another adverb. Some of them (\textstyleStyleVernacularWordsItalic{akena, maneka}) are also adjectives, some others (\textstyleStyleVernacularWordsItalic{lawisiw, iiwawun, wenup}) are non-numeral quantifiers (\sectref{sec:3.4.2}) with a second function as intensity adverbs. The distribution is different for each of the intensity adverbs.

akena  `very, really, truly'

iiwawun  `altogether'

kakeniw  `exactly'

lawisiw/lawiliw  `somewhat'

maneka  `very'

oram  `very, just'

pepek  `enough'

wenup  `very'

\ea%x508
\label{ex:x508}
\gll Moma  fain  eliw(a)  \textstyleEmphasizedVernacularWords{oram}. \\
      \\
\glt
\z

taro  this  good  just/very  

`This taro is very good.'

\ea%x510
\label{ex:x510}
\gll Koora  nain  maala  \textstyleEmphasizedVernacularWords{pepek}. \\
      \\
\glt
\z

house  that  long  enough

`That house is long enough.'

\textstyleStyleVernacularWordsItalic{Akena} `really, truly' is more flexible than the other intensity adverbs in that it can modify a word belonging to almost any word class.

\ea%x706
\label{ex:x706}
\gll Eka  mamaiya  \textstyleEmphasizedVernacularWords{akena}  i  yoowa  me  aaw-i-yen \\
      \\
\glt
\z

river  near  very  1p.UNM  hot  not  get-Np-FU.1p

`Very near the river we'll not get hot.'

\ea%x708
\label{ex:x708}
\gll Iikamin  \textstyleEmphasizedVernacularWords{akena=ko}  imen-ar-i-non? \\
      \\
\glt
\z

when  really=NF  find-INCH-Np-FU.3s

`Exactly when is it going to appear?'

\ea%x709
\label{ex:x709}
\gll Sira  samora  piipu-eka  \textstyleEmphasizedVernacularWords{akena}. \\
      \\
\glt
\z

habit  bad  leave-IMP.2p  really

`You (pl) must really leave (your) bad habits.'

\ea%x710
\label{ex:x710}
\gll Yiena  ikos  \textstyleEmphasizedVernacularWords{akena}  iw-u. \\
      \\
\glt
\z

1p.GEN  two.together  really  go-IMP.1d

`Lets's go \textstyleEmphasizedWords{\textsc{just}} the two of us together.'

\ea%x1875
\label{ex:x1875}
\gll Weetak  \textstyleEmphasizedVernacularWords{akena},  i=ko  me  kuum-e-mik. \\
      \\
\glt
\z

no  really,  1p.UNM=NF  not  burn-PA-1/3p

`\textstyleEmphasizedWords{\textsc{Really no}}, we did not burn it.'

\textstyleStyleVernacularWordsItalic{Lawisiw} `somewhat' is different from the rest in that it precedes the expression it modifies, rather than following it.

\ea%x703
\label{ex:x703}
\gll Uuw-owa  nain  \textstyleEmphasizedVernacularWords{lawisiw } yoowa. \\
      \\
\glt
\z

work-NMZ  that1  somewhat  hot/hard

`That work is somewhat hard.'

As an adjective \textstyleStyleVernacularWordsItalic{maneka} `big' is very common, but as an intensity adverb `very' it is very restricted in its distribution. \textstyleStyleVernacularWordsItalic{Maneka} cannot modify a verb, but it can intensify some non-numeral quantifiers like \textstyleStyleVernacularWordsItalic{unowa} `many' and \textstyleStyleVernacularWordsItalic{iiwawun} `altogether', as well as the temporal adverb \textstyleStyleVernacularWordsItalic{anane} `always'. 

\ea%x509
\label{ex:x509}
\gll Yo  anane  \textstyleEmphasizedVernacularWords{maneka}  naap  mauw-am-ika-i-yem. \\
      \\
\glt
\z

I  always  very  thus  work-SS.SIM-be-Np-PR.1s

`I \textstyleEmphasizedWords{\textsc{always}} keep working like that.'

\subsubsection{Modal adverbs}
\hypertarget{RefHeading21241935131865}{}
The two modal adverbs in Mauwake differ from each other not only semantically, but morphologically and syntactically as well. Modality of a predication is discussed in \sectref{sec:6.1}.

\textstyleStyleVernacularWordsItalic{Eliw}  `all right, well'\footnote{The manner adverb `well' is \textstyleFootnoteBaseChar{\textit{baliwep} }\textstyleFootnoteBaseChar{(\sectref{sec:3.8.1.3})}.} is a deontic adverb and expresses permission or desirability: `it is all right/good that{\dots}'. It can often be translated with the auxiliary `may' in English. It follows the subject, if there is any, but precedes the other clause constituents (\stepcounter{nx}{\thenx}). It may also be in the tail position after the clause, either following a clause that already has \textstyleStyleVernacularWordsItalic{eliw} in it (\stepcounter{nx}{\thenx}), or by itself (\stepcounter{nx}{\thenx}). 

\ea%x514
\label{ex:x514}
\gll Wie  wi  \textstyleEmphasizedVernacularWords{eliw}  wiar  op-i-kuan. \\
      \\
\glt
\z

3s/p.uncle  3p.UNM  well  3.DAT  hold-Np-FU.3p

`Her uncles may get (lit: hold) them (=clay pots) from her.' 

\ea%x515
\label{ex:x515}
\gll \textstyleEmphasizedVernacularWords{Eliw}  Kululu  ma-e-man,  \textstyleEmphasizedVernacularWords{eliw}. \\
      \\
\glt
\z

well  Kululu  say-PA-2p  well

`It is all right that you mentioned Kululu, that is OK.'

\ea%x516
\label{ex:x516}
\gll Nomokowa,  nie  owowa=pa  fan  pok-a-n,  \textstyleEmphasizedVernacularWords{eliw}. \\
      \\
\glt
\z

2s/p.brother  2s/p.uncle  village=LOC  here  sit-PA-2s  well

`It is good/OK that you settled here in your brother's and uncle's village.'

An epistemic modal adverb is the clitic -\textstyleStyleVernacularWordsItalic{yon} (with an alternative form -\textstyleStyleVernacularWordsItalic{nion}), expressing hesitation or non-committal assumption: `perhaps', `maybe', `I suppose'.  It is attached to the predicate, which usually is a verb but can also be non-verbal (\stepcounter{nx}{\thenx}).

\ea%x517
\label{ex:x517}
\gll Maa  me  wu-om-a-mik=\textstyleEmphasizedVernacularWords{yon}. \\
      \\
\glt
\z

thing/food  not  put-BEN-BNFY2.PA-1/3p-perhaps

`Perhaps they didn't put food (aside) for him.'

\ea%x518
\label{ex:x518}
\gll Yo  me  efa  ma-e-n=\textstyleEmphasizedVernacularWords{yon}  aa? \\
      \\
\glt
\z

1s.UNM  not  1s.ACC  say-PA-2s-perhaps  aa

`I suppose you weren't saying it about me?'

\ea%x519
\label{ex:x519}
\gll Ni  kema  puk-owa  marewa=ke=\textstyleEmphasizedVernacularWords{yon}! \\
      \\
\glt
\z

2p.UNM  liver  burst-NMZ  none=CF-perhaps

 `You must be crazy!' (Lit: `I suppose your liver hasn't burst (yet).')

The question word \textstyleStyleVernacularWordsItalic{kamenion} `or what' is related to the modal adverb -\textstyleStyleVernacularWordsItalic{yon} (\sectref{sec:3.9.3}).

\subsubsection{Free adverbs}
\hypertarget{RefHeading21261935131865}{}
The adverbs \textstyleStyleVernacularWordsItalic{muut(a}\textstyleStyleVernacularWordsItalic{)}/\textstyleStyleVernacularWordsItalic{muutiw} `just/only' and \textstyleStyleVernacularWordsItalic{pun} `also, too' are called free adverbs, as they can move around quite freely and attach themselves to various elements in a clause. \textstyleStyleVernacularWordsItalic{Muutiw}  is a combination of \textstyleStyleVernacularWordsItalic{muut(a)} and the limiter clitic \nobreakdash-\textstyleStyleVernacularWordsItalic{iw},  and it restricts restricts the scope of a preceding noun phrase or adverbial phrase. \textstyleStyleVernacularWordsItalic{Muut(a)} is used almost exclusively with noun phrases.

\ea%x747
\label{ex:x747}
\gll Aaya  \textstyleEmphasizedVernacularWords{muutiw}  en-em-ika-i-mik. \\
      \\
\glt
\z

sugarcane  only  eat-SS.SIM-be-Np-PR.1/3p

`They are only eating sugarcane.' 

\ea%x748
\label{ex:x748}
\gll Ofa  sepa  \textstyleEmphasizedVernacularWords{muutiw } (if-o-k). \\
      \\
\glt
\z

paint  black  only  paint-PA-3s

`He painted with only black paint.'

\ea%x757
\label{ex:x757}
\gll Ewar  wuun-i-ya  nain  \textstyleEmphasizedVernacularWords{muutiw}  miim-i-nan. \\
      \\
\glt
\z

wind  blow-Np-PR.3s  that  only  hear-Np-FU.2s

`You will hear only the wind blowing.'

\ea%x758
\label{ex:x758}
\gll Lotu  koora  Ulingan=pa  \textstyleEmphasizedVernacularWords{muutiw}  ik-ua=i? \\
      \\
\glt
\z

worship  house  Ulingan=LOC  only  be-PA.3s=QM

`Is there a church only at Ulingan?'

\ea%x806
\label{ex:x806}
\gll Aakisa  \textstyleEmphasizedVernacularWords{muutiw}  niir-i-mik. \\
      \\
\glt
\z

today  only  play-Np-PR.1/3p

`They play only today.'

\ea%x807
\label{ex:x807}
\gll Eliw  \textstyleEmphasizedVernacularWords{muutiw}. \\
      \\
\glt
\z

well  only

`It's just all right.'

\ea%x1820
\label{ex:x1820}
\gll Yo  opora  \textstyleEmphasizedVernacularWords{muut}  naap. \\
      \\
\glt
\z

1s.UNM  talk  only  thus

`That's my talk.'

\ea%x1821
\label{ex:x1821}
\gll Uf-owa  erup  \textstyleEmphasizedVernacularWords{muuta}  naap   uf-e-mik. \\
      \\
\glt
\z

dance-NMZ  two  only  thus  dance-PA-1/3p

`We only danced two dances like that.'

\textstyleStyleVernacularWordsItalic{Pun} `also' has even wider distribution than \textstyleStyleVernacularWordsItalic{muutiw}: it can occur following almost any element in a clause.\footnote{\textit{Pun} may be in the process of developing into a clitic.  As a one-syllable word it it often has a weak stress, and some speakers also write it attached to the preceding word with a hyphen, the way clitics are written in the Mauwake orthography.} 

\ea%x749
\label{ex:x749}
\gll Ne  waaya  nain  \textstyleEmphasizedVernacularWords{pun}  afila  marew,  waaya  asia  \textstyleEmphasizedVernacularWords{pun.} \\
      \\
\glt
\z

and  pig  that1  also  grease  no(ne)  pig  wild  also

`And that pig also didn't have fat, (as) it was a wild pig too.'

\ea%x1937
\label{ex:x1937}
\gll Yos  \textstyleEmphasizedVernacularWords{pun}  wie  opora  nainiw  ma-i-yem. \\
      \\
\glt
\z

1s.FC  too  3s/p.uncle  talk  again  say-Np-PR.1s

`I, too, will again give ``uncle-talk'' (=cultural instruction).'

\ea%x750
\label{ex:x750}
\gll Ne  \textstyleEmphasizedVernacularWords{pun}  aakisa  iperowa  korokor  or-owa  sira \\
      \\
\glt
\z

and  also  now  middle.aged  initiation  descend-NMZ  custom

iiriw  wafur-a-mik.

earlier  throw-PA-1/3p

`Also, now the middle-aged people have already rejected the initiation custom.'

\ea%x751
\label{ex:x751}
\gll Iiriw  \textstyleEmphasizedVernacularWords{pun}  miiwa  muuta  nain  irak-owa  marew. \\
      \\
\glt
\z

earlier  also  ground  because.of  that1  fight-NMZ  no(ne)

`Earlier there were also no fights over ground' (or: `Earlier, too, there were no fights over ground.')

\ea%x808
\label{ex:x808}
\gll Teeria  maneka  wadol  opora  mik-a-mik  \textstyleEmphasizedVernacularWords{pun}  naap,  {\dots} \\
      \\
\glt
\z

group  big  lie  talk  hit-PA-1/3p  also  thus

`(When) the big group lied it was also like that, {\dots}'

\subsection{Negators} 
\hypertarget{RefHeading21281935131865}{}
Mauwake has four negators: \textstyleStyleVernacularWordsItalic{weetak}, \textstyleStyleVernacularWordsItalic{wia}, \textstyleStyleVernacularWordsItalic{me} and \textstyleStyleVernacularWordsItalic{marew}. They are morphologically free and syntactically heterogeneous, each one having its specific position. Of the four negators \textstyleStyleVernacularWordsItalic{me} is positioned before the negated element, while \textstyleStyleVernacularWordsItalic{marew} follows the negated element. \textstyleStyleVernacularWordsItalic{Weetak} and \textstyleStyleVernacularWordsItalic{wia} either form a complete utterance by themselves, or they are sentence-initial when used as negative interjections (\stepcounter{nx}{\thenx}) but clause-final when functioning as non-verbal predicates (\stepcounter{nx}{\thenx}), and when replacing full clauses they take the position of the clause they replace (\stepcounter{nx}{\thenx}), (\stepcounter{nx}{\thenx}). 

\ea%x654
\label{ex:x654}
\gll Maamuma  \textstyleEmphasizedVernacularWords{me}  tuun-owa  ik-e-mik. \\
      \\
\glt
\z

money  not  count-NMZ  be-PA-1/3p

`They haven't counted the money (yet).'

\ea%x1112
\label{ex:x1112}
\gll Mukuna  \textstyleEmphasizedVernacularWords{me}  op-a,  nefa  kuum-i-non! \\
      \\
\glt
\z

fire  not  touch-IMP.2s  2s.ACC  burn-Np-FU.3s

`Don't touch the fire, it will burn you!'

\ea%x655
\label{ex:x655}
\gll I  muuka  \textstyleEmphasizedVernacularWords{marew}. \\
      \\
\glt
\z

1p.UNM  son  no(ne).

`We have no son.'

\ea%x707
\label{ex:x707}
\gll \textstyleEmphasizedVernacularWords{Wia},  me  kookal-i-yem. \\
      \\
\glt
\z

No  not  like-Np-PR.1s  

`No, I don't like it.'

\ea%x1212
\label{ex:x1212}
\gll Yo  uuw-owa  oko  \textstyleEmphasizedVernacularWords{weetak}. \\
      \\
\glt
\z

1s.UNM  work-NMZ  other  no

`I have no other work.'

\ea%x705
\label{ex:x705}
\gll Wafur-a-k  na  \textstyleEmphasizedVernacularWords{weetak},  ufer-a-k. \\
      \\
\glt
\z

throw-PA-3s  but  no,  miss-PA-3s

`He threw it (a spear), but no (=he didn't succeed), he missed (the pig).'

\ea%x1111
\label{ex:x1111}
\gll Akup-a-mik,  akup-a-mik,  \textstyleEmphasizedVernacularWords{wia}. \\
      \\
\glt
\z

search-PA-1/3p  search-PA-1/3p  no

`We searched and searched, but no (=we did not find it).'

According to a rough generalization the most frequent negator \textstyleStyleVernacularWordsItalic{me} is basically a clause and constituent negator. It is also used to negate imperatives. \textstyleStyleVernacularWordsItalic{Weetak} and \textstyleStyleVernacularWordsItalic{wia} are negative interjections or predicates in verbless clauses, and \textstyleStyleVernacularWordsItalic{marew} can negate non-verbal predicates and occasionally noun phrase constituents. \textstyleStyleVernacularWordsItalic{Marew} often has the meaning `none at all'.

\textstyleStyleVernacularWordsItalic{Weetak}\textstyleStyleVernacularWordsItalic{} (\stepcounter{nx}{\thenx}), \textstyleStyleVernacularWordsItalic{wia} and occasionally \textstyleStyleVernacularWordsItalic{marew}, may be intensified by a postposed intensity adverb \textstyleStyleVernacularWordsItalic{akena} `truly, very'. \textstyleStyleVernacularWordsItalic{Me}  can only be intensified as a verbal negator, in which case \textstyleStyleVernacularWordsItalic{akena} comes after the verb rather than after the negator.

\ea%x652
\label{ex:x652}
\gll Ni  niam  erup  kema\textbf{  marew  akena}! \\
      \\
\glt
\z

2p.UNM  2p.REFL  two  liver  no(ne)  really

`The two of you have \textstyleEmphasizedWords{\textsc{really no}} sense at all!'

\ea%x653
\label{ex:x653}
\gll \textstyleEmphasizedVernacularWords{Me}  on-a-m  \textstyleEmphasizedVernacularWords{akena}. \\
      \\
\glt
\z

not  do-PA-1s  really

`I \textstyleEmphasizedWords{\textsc{really didn't}} do it.'

A fuller treatment of the negators is in \sectref{sec:6.2}, where negation as a functional category is discussed.\footnote{Bergh\"all (2006) gives a somewhat more comprehensive treatment of negation in Mauwake, but some of the analysis has changed since the writing of the article.}

\subsection{Connectives}
\hypertarget{RefHeading21301935131865}{}
The inventory of connectives in Mauwake is small. They are called connectives rather than conjunctions, because conjunctions are normally understood as a class of words, but in Mauwake a connective may be a word or a phrase. The term \textstyleEmphasizedWords{\textsc{conjunction}} is reserved for the conjunctive coordination construction (\sectref{sec:8.1.1}). Many of the connectives also have another primary function. 

The main division is into pragmatic and semantic connectives; all of them are coordinate. Subordination is discussed in \sectref{sec:8.3}. The connectives mostly operate on sentence level, joining clauses (\sectref{sec:8.1}). Almost all of the coordinators also conjoin sentences. Only the pragmatic connectives and the disjunctive connective \textstyleStyleVernacularWordsItalic{e} `or' are able to conjoin elements on the word and phrase levels as well. 

The most typical way of combining clauses is clause chaining through medial verbs, with no connective words at all (\sectref{sec:8.2}). When there are connectives, they are always placed between the two clauses. 

\subsubsection{Pragmatic connectives}
\hypertarget{RefHeading21321935131865}{}
Instead of clearly specifying the semantic relationship between the units they connect, like semantic connectives do, the pragmatic connectives signal a pragmatic relationship between them.\footnote{For this distinction on pragmatic and semantic connectives I am indebted to Stephen Levinsohn.} In Haspelmath's (2007:8) terms they are `medial [and] prepositive', meaning that they occur between the items they conjoin, and are linked more closely to the following constituent rather than the preceding one.

The connective \textstyleStyleVernacularWordsItalic{ne} `additive' only indicates that something is added to what has just been said.  It can connect word and phrase level units (\sectref{sec:4.1.2}), but is mostly used between clauses (\sectref{sec:8.1}) and even sentences. It is semantically neutral. When it conjoins words (\stepcounter{nx}{\thenx}) or phrases (\stepcounter{nx}{\thenx}), and often when it coordinates clauses (\stepcounter{nx}{\thenx}) or sentences (\stepcounter{nx}{\thenx}), it can be translated into English with `and'. 

\ea%x711
\label{ex:x711}
\gll kumin,  wutkekela  \textstyleEmphasizedVernacularWords{ne}  mera  ... \\
      \\
\glt
\z

hermit.crab  calamari  ADD  fish

`hermit crabs, calamari and fish {\dots}'

\ea%x713
\label{ex:x713}
\gll Inawera  sira  unowa,  \textstyleEmphasizedVernacularWords{ne}  kemena  unowa. \\
      \\
\glt
\z

dream  custom  many  ADD  inside  many

`There are many kinds of dreams, and (they have) many meanings.'

\ea%x714
\label{ex:x714}
\gll \textstyleEmphasizedVernacularWords{Ne}  yo  aakisa  tep=pa  ma-i-yem. \\
      \\
\glt
\z

ADD  1s.UNM  now  tape.recorder=LOC  say-Np-PR.1s

`And now I say it to a tape recorder.'

Words or phrases in lists are most commonly joined by juxtaposition only. If a connective is used, \textstyleStyleVernacularWordsItalic{ne} usually joins the last two (\stepcounter{nx}{\thenx}) coordinands. It is also possible to place the connective(s) closer to the beginning of the list.

\ea%x1359
\label{ex:x1359}
\gll Sesa  nain  waaya  erup  arow  \textstyleEmphasizedVernacularWords{ne}  maamuma  kuuma  erepam  ikur \\
      \\
\glt
\z

price  that1  pig  two  three  ADD  money  stick  four  five

\textstyleEmphasizedVernacularWords{ne}  manar  kuisow,  waa  eneka,  naap  muuka

ADD  forehead.ornament  one  pig  tooth  thus  son  

sesenar-i-nen.

buy-Np-FU.1s

 `(As for) the price, I will buy my son with two-three pigs and forty-fifty kina and a forehead ornament (and) pig's tusk(s), like that.'

There is no emphatic coordinate connective of the type `both {\dots} and' in Mauwake. 

If the propositions connected by \textstyleStyleVernacularWordsItalic{ne} contrast with each other in some way, it may be interpreted as adversative and translated into English with `but'.\footnote{Many Papuan languages have a connective that is glossed `and/but'. I suspect it is an additive connective like \textit{ne}, which is only interpreted as either `and' or `but' according to the content of the clauses conjoined.} In these cases it is always a ``weak'' adversative in contrast to the demonstrative \textstyleStyleVernacularWordsItalic{nain}  used in ``strong'' adversative clauses  (\sectref{sec:8.1.3}).

\ea%x715
\label{ex:x715}
\gll Maa  en-owa  iw-e-mik,  \textstyleEmphasizedVernacularWords{ne}  rais  weetak. \\
      \\
\glt
\z

thing  eat-NMZ  give.him-PA-1/3p  ADD  rice  no

`They gave him food, but not rice.'

\ea%x716
\label{ex:x716}
\gll Wi  me  kuum-e-mik,  \textstyleEmphasizedVernacularWords{ne}  wi  murar-owa=pa \\
      \\
\glt
\z

3p.UNM  not  burn-PA-1/3p  ADD  3p.UNM  follow-NMZ=LOC  

mukuna  nain  kerer-e-k.

fire  that  appear-PA-3s

`They didn't burn it, but the fire started after them.'

In a number of cases either neutral additive or contrastive interpretation is possible:

\ea%x1361
\label{ex:x1361}
\gll Wiam  erup  irak-ep  puk-e-mik,  aalbok=ke \\
      \\
\glt
\z

3p.REFL  two  fight-SS.SEQ  disperse-PA-1/3p  black.cuckoo-shrike=CF

ifera  or-o-k  ne  osaiwa=ke  soor(a)  asia  ikiw-o-k.

sea  descend-PA-3s  ADD  bird.of.paradise=CF  forest  wild  go-PA-3s

`The two of them fought and went their separate ways, the black cuckoo-shrike went down to the coast and/but the bird of paradise went to the wild (rain)forest.'

There are two discourse-marking pragmatic connectives, \textstyleStyleVernacularWordsItalic{aria} and \textstyleStyleVernacularWordsItalic{ne aria}.  They both mark discontinuity in the text.

\textstyleStyleVernacularWordsItalic{Aria} `alright'\footnote{The translation reflects the Tok Pisin word \textit{orait}, which sometimes has a similar discourse function. \textit{Aria} occurs in many Madang languages, and the speakers of those languages tend to use \textit{aria} in Tok Pisin too.}  usually comes sentence-initially, but can also or occur sentence-medially. Its main function is to indicate a break in the topic chain. In (\stepcounter{nx}{\thenx}) the topic changes from the snake to the man, and in (\stepcounter{nx}{\thenx}) from a health extension officer to a group of men:

\ea%x717
\label{ex:x717}
\gll Keraw-eya  \textstyleEmphasizedVernacularWords{aria}  nomokowa  gelemuta  puuk-ap  ifa  nain \\
      \\
\glt
\z

bite-2/3s.DS  alright  tree  small  cut-SS.SEQ  snake  that

ifakim-o-k.

kill-PA-3s

`It (=the snake) bit him, and he cut a small tree and killed the snake.'

\ea%x718
\label{ex:x718}
\gll {\dotso  miim-o-k. } \textstyleEmphasizedVernacularWords{Aria}  wi  kiiriw  neeke  {\dots} \\
      \\
\glt
\z

{\dots}3s.UNM  precede-PA-3s.  Alright  3p.UNM  again  there.CF  ...

`{\dots} he went ahead. (When) they were \textstyleEmphasizedWords{\textsc{there}} again {\dots}'

It often signals the beginning of a turn in a conversation (\stepcounter{nx}{\thenx}), or beginning of a speech (\stepcounter{nx}{\thenx}), again indicating a break with the preceding text. 

\ea%x721
\label{ex:x721}
\gll \textstyleEmphasizedVernacularWords{Aria}  wiipa,  i  yia  uruf-e. \\
      \\
\glt
\z

alright  daughter,  1p.UNM  1p.ACC  see-IMP.2s

`Daughter, look at us.'

\ea%x720
\label{ex:x720}
\gll \textstyleEmphasizedVernacularWords{Aria},  i  owowa=ko  urup-u. \\
      \\
\glt
\z

alright,  1p.UNM  village=NF  ascend-IMP.1d

`Alright, let's go back to the village.'

Even if the topic stays the same, \textstyleStyleVernacularWordsItalic{aria} can be used, especially when there is a contrast between alternatives (\stepcounter{nx}{\thenx}), or sometimes when an expected sequence of events is broken (\stepcounter{nx}{\thenx}).

\ea%x719
\label{ex:x719}
\gll Mua  maneka  maamuma  erup,  \textstyleEmphasizedVernacularWords{aria}  wi  suule  takira \\
      \\
\glt
\z

man  big  money  two  alright  3p.UNM  school  child  

maamuma  kuisow,  naap  omopora  sesenar-e-mik.

money  one  thus  door  buy-PA-1/3p

`The grown men paid two coins (=20 toea) for entrance, the schoolchildren one coin.'

\ea%x722
\label{ex:x722}
\gll Wiawi  onak  urera  maa  uup-e-mik,  \textstyleEmphasizedVernacularWords{aria} \\
      \\
\glt
\z

3s/p.father  3s/p.mother  evening  food  cook-PA-1/3p  alright

maa  me  wu-om-a-mik=yon.

food  not  put-BEN-BNFY2.PA-1/3p-perhaps

`In the evening his parents cooked food, (but) perhaps they didn't put any food for him.'

\textstyleStyleVernacularWordsItalic{Ne aria} `and alright' occurs less often than \textstyleStyleVernacularWordsItalic{aria}, and only sentence-initially. It marks major points of development in the plot of a story. 

\ea%x723
\label{ex:x723}
\gll Naap  wia  maak-e-mik.  \textstyleEmphasizedVernacularWords{Ne  aria},  ifa  nain  murar-ep{\dots} \\
      \\
\glt
\z

thus  3p.ACC  tell-PA-1/3p  ADD  alright  snake  that  follow-SS.SEQ

`They told them like that.  Now, the snake followed them and {\dots}'

Sometimes it also signals return to foreground text (i.e. main story line) after some backgrounded material. 

\subsubsection{Semantic connectives}
\hypertarget{RefHeading21341935131865}{}
The semantic connectives specify the relationship between two propositions. 

The disjunctive connective \textstyleStyleVernacularWordsItalic{e} `or' can connect not only propositions but words or phrases as well. It is used both for standard (\stepcounter{nx}{\thenx}) and interrogative (\stepcounter{nx}{\thenx}) disjunction\footnote{This terminology is from \citet{Haspelmath2007}.} (\sectref{sec:8.1.2}, 7.2.2). When there are two alternatives, the connective occurs between them. It is also common to have the question marker -\textstyleStyleVernacularWordsItalic{i}  cliticized to the end of the first alternative, especially in questions, but also elsewhere.

\ea%x724
\label{ex:x724}
\gll ama  arow  naap,  \textstyleEmphasizedVernacularWords{e}  erepam  naap,  {\dots} \\
      \\
\glt
\z

sun  three  thus  or  four  thus  

`at about three o'clock, or at about four {\dots}'

\ea%x725
\label{ex:x725}
\gll Emeria=ko  efar  uruf-a-man=\textstyleEmphasizedVernacularWords{i  e}  weetak? \\
      \\
\glt
\z

woman=NF  1s.DAT  see-PA-2p=QM  or  no

`Did you see my wife or not?'

When there are more alternatives than one and the question clitic is present, the connective may be left out altogether (\stepcounter{nx}{\thenx}), or it may occur between the first two alternatives (\stepcounter{nx}{\thenx}). 

\ea%x726
\label{ex:x726}
\gll maa  oposia=i  moma,  emera,  naap \\
      \\
\glt
\z

thing  meat=QM  taro,  sago,  thus

`meat, or taro, or sago, (things) like that'

\ea%x727
\label{ex:x727}
\gll iwer  eka=ki  \textstyleEmphasizedVernacularWords{e}  mauwa=ki,  a  episowa=ki, \\
      \\
\glt
\z

coconut  water=CF.QM  or  what=CF.QM  ah  tobacco=CF.QM,  

ufia=ki {\dots}

betel.pepper=CF.QM

`coconut juice or - ummm - tobacco, or betel pepper {\dots}'

The following consecutive connectives marking effect or result\footnote{It is typical for Papuan languages to mark the effect/result clause rather than the cause/reason clause. For a Papuan language which has several connectives both for result and for reason, see \citet[267-273]{Farr1999}.} are used in sentences where the clauses have a consecutive, i.e. a cause-effect or reason-result relationship: \textstyleStyleVernacularWordsItalic{naapeya/naeya}, \textstyleStyleVernacularWordsItalic{neemi}, and \textstyleStyleVernacularWordsItalic{naap nain}. They can all be glossed with `therefore, (and) so'.

\textstyleStyleVernacularWordsItalic{Naapeya/naeya} is the most generic and frequently used of the four.  \textstyleStyleVernacularWordsItalic{Naapeya} has developed from the manner adverb \textstyleStyleVernacularWordsItalic{naap} `thus' followed by the different-subject marker -\textstyleStyleVernacularWordsItalic{eya}\textstyleParagraphChari{ (\sectref{sec:3.8.3.5.2})};\footnote{Actually this connective in the coastal villages is \textit{naapera}, but because of the language committee's decision to use -\textit{eya} for the 2/3s.DS marker, this form is used here too.} the resulting meaning is `it being thus'. The origin of \textstyleStyleVernacularWordsItalic{naeya} is in the medial different-subject form of the verb \textstyleStyleVernacularWordsItalic{na}- `say, think'. The difference between the two is mainly dialectal, or areal: \textstyleStyleVernacularWordsItalic{naapeya} is used more on the coast, \textstyleStyleVernacularWordsItalic{naeya} in the inland. They are used for marking the effect or result clause in a consecutive sentence. 

\ea%x731
\label{ex:x731}
\gll I  maamuma  marew,  \textstyleEmphasizedVernacularWords{naapeya}  ifera=ko  me  sesenar-e-mik. \\
      \\
\glt
\z

1p.UNM  money  no(ne),  so  salt=NF  not  buy-PA-1/3p

`We didn't have money, so we didn't buy salt.'

\ea%x732
\label{ex:x732}
\gll Ben  uuw-owa  piipu-a-k.  \textstyleEmphasizedVernacularWords{Naapeya}  emina  urur-ep \\
      \\
\glt
\z

Ben  work-NMZ  left-PA-3s  therefore  occiput  fall-SS.SEQ

me  ekap-o-k.

not  come-PA-3s

`Ben has left the work. Therefore he was ashamed to come.'

\ea%x735
\label{ex:x735}
\gll Pika  oona  me  kekan-ow-a-k.  \textstyleEmphasizedVernacularWords{Naeya}  uura  ewar=ke \\
      \\
\glt
\z

wall  bone  not  strong-CAUS-PA-3s  therefore  night  wind=CF  

teek-a-k.

tear-PA-3s

`He didn't strengthen the wall studs. So at night the wind tore it (the house) down.'

\ea%x1413
\label{ex:x1413}
\gll I  miiw-aasa=pa  ekap-e-mik,  \textstyleEmphasizedVernacularWords{naeya}  o  me \\
      \\
\glt
\z

1p.UNM  land-canoe=LOC  come-PA-1/3p  therefore  3s.UNM  not  

yook-a-k.

follow.us-PA-3s

`We came in a car, so he didn't follow/come with us.'

The origin of \textstyleStyleVernacularWordsItalic{naeya} is so transparent that there are many cases where two different  interpretations for \textstyleStyleVernacularWordsItalic{naeya} are acceptable (\stepcounter{nx}{\thenx}). 

\ea%x734
\label{ex:x734}
\gll ``Yo  koka=pa  ik-e-m.''  \textstyleEmphasizedVernacularWords{Na-eya}  Magerka=ke  (ma-e-k){\dots} \\
      \\
\glt
\z

1s.UNM  jungle=LOC  be-PA-1s  say-2/3s.DS  MacArthur  (say-PA-3s)

` ``I was in the jungle.'' He said that, and (or: So) MacArthur said, {\dots}'

But in (\stepcounter{nx}{\thenx}) \textstyleStyleVernacularWordsItalic{naeya} clearly means `therefore' and cannot be interpreted as a medial verb, as the correct verb form in this case would be plural \textstyleStyleVernacularWordsItalic{naiwkin} `they said and{\dots}', not singular \textstyleStyleVernacularWordsItalic{naeya} `you/(s)he said and{\dots}'.

\ea%x733
\label{ex:x733}
\gll Iwera  yia  na-em-ik-e-mik. \\
      \\
\glt
\z

coconut  1s.ACC  say-SS.SIM-be-PA-1/3p

\textstyleEmphasizedVernacularWords{Naeya}  iwera  wia  uruk-am-ik-om-a-mik.

So  coconut  3p.ACC  drop-SS.SIM-be-BEN-BNFY2.PA-1/3p

`They kept asking us for coconuts. So we kept dropping coconuts for them.'

The originally dialectal difference may be developing into a semantic one. In the original text data from three decades ago there is no clear semantic distinction between the use of \textstyleStyleVernacularWordsItalic{naapeya} and \textstyleStyleVernacularWordsItalic{naeya}, but fairly recently when a group with members from different dialects, discussing language matters, produced consecutive clauses, nearly all of the sentences with \textstyleStyleVernacularWordsItalic{naapeya} were cases of cause-effect (\stepcounter{nx}{\thenx}), and all of the sentences with \textstyleStyleVernacularWordsItalic{naeya} were cases of reason and result (\stepcounter{nx}{\thenx}). 

\ea%x1414
\label{ex:x1414}
\gll I  fiirim-owa=pa  ik-emkun  ama  or-o-k, \\
      \\
\glt
\z

1p.UNM  gather-NMZ=LOC  be-1s/p.DS  sun  descend-PA-3s

\textstyleEmphasizedVernacularWords{naapeya}  epa  kokom-ar-e-k.

therefore  place  dark-INCH-PA-3s

`When we were in the meeting the sun went down, so it became dark.'

\ea%x1415
\label{ex:x1415}
\gll I  fiirim-owa=pa  ik-emkun  ama  or-o-k, \\
      \\
\glt
\z

1p.UNM  gather-NMZ=LOC  be-1s/p.DS  sun  descend-PA-3s

\textstyleEmphasizedVernacularWords{naeya}  maa  me  wiar  en-owa  ikiw-o-k.

therefore  food  not  3.DAT  eat-NMZ  go-PA-3s

`When we were in the meeting the sun went down, so he went without eating the food.'

\textstyleStyleVernacularWordsItalic{Naapeya} can also co-occur with the conjunctive coordinators \textstyleStyleVernacularWordsItalic{ne} or \textstyleStyleVernacularWordsItalic{aria}. In argumentation, \textstyleStyleVernacularWordsItalic{ne naapeya} or \textstyleStyleVernacularWordsItalic{aria naapeya} has to be used, when the reason is not confined to one clause but extends to a longer stretch of the discourse.

\ea%x1406
\label{ex:x1406}
\gll \textstyleEmphasizedVernacularWords{Aria}  \textstyleEmphasizedVernacularWords{naapeya}  niena  soomar-owa  ne  aakun-owa  pun \\
      \\
\glt
\z

alright  therefore  2p.GEN  walk-NMZ  ADD  talk-NMZ  also

sira  yi-e-k  nain  kaken=iw  ook-ap  soomar-eka.

custom  give.us-PA-3s  that1  straight=LIM  follow-SS.SEQ  walk-IMP.2p

`So therefore, as concerns your walk and talk too, follow straight the behaviour that he gave us and walk that way.'

\textstyleStyleVernacularWordsItalic{Neemi}  is used only in reasoning. It requires some point of similarity between the antecedent and the result clause.

\ea%x736
\label{ex:x736}
\gll Teeria  fain  K10  wu-a-mik.  \textstyleEmphasizedVernacularWords{Neemi}  wi  teeria  nain  pun \\
      \\
\glt
\z

group  this  K10  put-PA-1/3p  therefore  3p.UNM  group  that1  too

K10  wu-a-mik.

K10  put-PA-1/3p

`This group put down K10. Thefore that group put down K10, too.'

\textstyleStyleVernacularWordsItalic{Naap nain} can be translated into English with `therefore', `in that case', `if so, then'. It is made up of the manner adverb \textstyleStyleVernacularWordsItalic{naap} `thus' and the distal demonstrative \textstyleStyleVernacularWordsItalic{nain} `that'.  It is a strong connective, stressing the fact that the proposition following the connective is a logical conclusion from the preceding proposition.

\ea%x737
\label{ex:x737}
\gll Ni  moma  uup-i-man=i?  \textstyleEmphasizedVernacularWords{Naap  nain}  yo  saa \\
      \\
\glt
\z

2p.UNM  taro  cook-Np-2p=QM  thus  that  1s.UNM  rice  

uup-i-nen.

cook-Np-FU.1s

`Are you cooking taro? In that case I'll cook rice.'

It is much less common in Mauwake to mark the reason clause than the result clause with a connective.  When the reason clause is emphasized, it is marked with the connective \textstyleStyleVernacularWordsItalic{moram} (\textstyleStyleVernacularWordsItalic{wia}) `because' and always follows the result clause rather than preceding it. The origin of the reason connective is in the question word \textstyleStyleVernacularWordsItalic{moram} `why?' and the negator \textstyleStyleVernacularWordsItalic{wia} `no(t)'.\footnote{\textit{Moram} as a reason connective is probably a calque on Tok Pisin \textit{bilong wanem} `why, because'. The negator, which does not influence the meaning of the connective, may have been added in Mauwake to help distinguish the connective from the question word.} The difference between \textstyleStyleVernacularWordsItalic{moram wia} and \textstyleStyleVernacularWordsItalic{moram} is that the former is mainly used across sentence boundary (\stepcounter{nx}{\thenx}), and the latter within a sentence (\stepcounter{nx}{\thenx}).

\ea%x738
\label{ex:x738}
\gll Maamuma  senam  aaw-e-mik.  \textstyleEmphasizedVernacularWords{Moram}  \textstyleEmphasizedVernacularWords{wia},  maa  ele-eliwa \\
      \\
\glt
\z

money  a.lot  get-PA-1/3p  why  not  thing  RDP-good  

sesek-a-mik. 

sell-PA-1/3

`They got a lot of money. (That's) because they sold good things/foods.'

\ea%x739
\label{ex:x739}
\gll Miiw-aasa  muf-owa  me  ikiw-e-mik,  \textstyleEmphasizedVernacularWords{moram}  os=ke  naap \\
      \\
\glt
\z

land-canoe  pull-NMZ  not  go-PA-1/3p  why  3s.FC=CF  thus

ar-eya.

become-2/3s.DS

`We didn't go to fetch a truck, because she had become like that (=died).'

\subsection{Postpositions and clitics}
\hypertarget{RefHeading21361935131865}{}
Since Mauwake is an \textstyleAcronymallcaps{SOV} language, it is natural that it has postpositions rather than prepositions.  But their number is small: besides the comitative postpositions there are only two others, one for comparison and one indicating reason.

Unlike the postpositions, which are both phonological and grammatical words, clitics are grammatical words that together with the preceding word form one  phonological word. The stress assignment rule does not affect them: they are always unstressed.  If there are any derivational and inflectional suffixes in the host word, the clitics  are added after all of them (Dixon 2010a:221\nobreakdash-2).

The nominal clitics associate with noun phrases and attach themselves phonologically to the last element of the noun phrase. They mark either the case role or the pragmatic function of the \textstyleAcronymallcaps{NP}. The only sentential clitic is the question marker \nobreakdash-\textstyleStyleVernacularWordsItalic{i}.  The modal clitic -\textstyleStyleVernacularWordsItalic{yon} `perhaps' was discussed above in 3.9.3.

The postpositions and clitics are discussed together because of their shared origin in some cases, causing similarity in form, and because some of them have  similarities in function.

\subsubsection{Comitative clitic and postpositions}
\hypertarget{RefHeading21381935131865}{}
Accompaniment, or a\textstyleEmphasizedWords{\textsc{ comitative}} relationship may be expressed by one clitic or by five different postpositions, three of which are formed with the clitic. 

The comitative clitic is -\textstyleStyleVernacularWordsItalic{iya} `with, and, both {\dots} and'. The clitic may be attached to either of the two related \textstyleAcronymallcaps{NP}s, or to both.

\ea%x775
\label{ex:x775}
\gll Nan  pok-ap-ik-e-mik,  \textstyleEmphasizedVernacularWords{mua=iya  emeria}. \\
      \\
\glt
\z

there  sit-SS.SEQ-be-PA-1/3p  man=COM  woman.

`They were sitting there, (both) husband and wife.'

\ea%x776
\label{ex:x776}
\gll \textstyleEmphasizedVernacularWords{Muuka  wiip=iya}  kerer-e-mik. \\
      \\
\glt
\z

son  daughter=COM  appear-PA-1/3p

`(Both) a son and a daughter appeared.'

\ea%x777
\label{ex:x777}
\gll \textstyleEmphasizedVernacularWords{Bom=iya  kateres=iya,  bom=iya  kateres=iya}  (fuurk-a-mik). \\
      \\
\glt
\z

bomb=COM  cartridge=COM  bomb=COM  cartridge=COM  drop-PA-1/3p

`They dropped (both) bombs and cartridges, (both) bombs and cartridges.'

It combines with pronouns to form comitative pronouns (\sectref{sec:3.5.9}), and the word for `all', \textstyleStyleVernacularWordsItalic{unowiya}, is made up of \textstyleStyleVernacularWordsItalic{unowa} `many' plus the comitative clitic.

Occasionally the clitic can also be used to indicate instrument.  

\ea%x778
\label{ex:x778}
\gll Mauwa  ar-e-n,  \textstyleEmphasizedVernacularWords{amia=iya}  nenar-e-mik=i? \\
      \\
\glt
\z

what  become-PA-2s  bow=COM  shoot.you-PA-1/3p=QM

`What happened to you, did they shoot you with a gun?'

\textstyleStyleVernacularWordsItalic{Owawiya}\textstyleStyleVernacularWordsItalic{/owawik} `with, together with' is used only for humans; it can refer to two or more people. It can also occur by itself (\stepcounter{nx}{\thenx}). The origin of the first part \textstyleStyleVernacularWordsItalic{owaw}- is unknown. The second part is either the comitative clitic -\textstyleStyleVernacularWordsItalic{iya} or the root of the existential verb \textstyleStyleVernacularWordsItalic{ik}\nobreakdash- `be', a reflection of an earlier construction \textstyleStyleVernacularWordsItalic{owawiya ik}\nobreakdash- `be together'. 

\ea%x821
\label{ex:x821}
\gll Yoli  onak  \textstyleEmphasizedVernacularWords{owawiya}  efa  amukar-e-mik. \\
      \\
\glt
\z

Yoli  3s/p.mother  with  1s.ACC  scold-PA-1/3p

`Yoli and his mother scolded me.'

\ea%x822
\label{ex:x822}
\gll \textstyleEmphasizedVernacularWords{Owawiya}  feeke  pok-ap  ik-ok  soomar-ek-eka. \\
      \\
\glt
\z

with  here.CF  sit-SS.SEQ  be-SS  walk-go-IMP.2p

`(First) sit here with us and (then) go.'

\ea%x1817
\label{ex:x1817}
\gll Iikir-ami  onak  \textstyleEmphasizedVernacularWords{owawik}  soomar-e-mik. \\
      \\
\glt
\z

get.up-SS.SIM  3s/p.mother  with  walk-PA-1/3p

`He got up and walked with his mother.'

The postposition \textstyleStyleVernacularWordsItalic{onaiya/onaria}\textstyleStyleVernacularWordsItalic{/onaiyik} may be based on the third person singular genitive pronoun ona and the clitic -\textstyleStyleVernacularWordsItalic{iya}; \textstyleStyleVernacularWordsItalic{onaiyik} also includes the root of the verb \textstyleStyleVernacularWordsItalic{ik}\nobreakdash- `be'.  This postposition is more generic and can be used with noun phrases referring to people (\stepcounter{nx}{\thenx}) but is also common when referring to things (\stepcounter{nx}{\thenx}). When the relationship between the two noun phrases is unequal, \textstyleStyleVernacularWordsItalic{onaiya} may used, like in (\stepcounter{nx}{\thenx}), where the other people carried the sick man. The subject marking on the verb is influenced by how active part the referents of the comitative \textstyleAcronymallcaps{NP} take in the action. When all the participants are active, the subject marking on the verb is plural. The speech in (\stepcounter{nx}{\thenx}) was directed towards the villagers who were instructed to stay away from the Japanese troops. 

\ea%x754
\label{ex:x754}
\gll Mua  unowa  \textstyleEmphasizedVernacularWords{onaiya}  ikiw-e-mik. \\
      \\
\glt
\z

man  many  with  go-PA-1/3p

`We went with many people.'

\ea%x755
\label{ex:x755}
\gll Urom(a)  \textstyleEmphasizedVernacularWords{onaiya}  ik-ua. \\
      \\
\glt
\z

stomach  with  be-PA.3s

`She is/was pregnant.'

\ea%x823
\label{ex:x823}
\gll Mua  napuma  \textstyleEmphasizedVernacularWords{onaiya}  Medebur  ek-a-mik. \\
      \\
\glt
\z

man  sick  with  Medebur  go-PA-1/3p

`They went to Medebur with the sick man '

\ea%x1819
\label{ex:x1819}
\gll No  ara  sepa  ara  kia  \textstyleEmphasizedVernacularWords{onaiyik}  bilik \\
      \\
\glt
\z

2s.UNM  trunk  black  trunk  white  together  mixed

ar-i-nan=na  {\dots}

become-Np-FU.2s=TP

`If you, a black person, are together with the white people mixed with them {\dots}'

Another comitative postposition that mainly refers to things is \textstyleStyleVernacularWordsItalic{feekiya} `with'. It originates from the combination of \textstyleStyleVernacularWordsItalic{feeke} `here' and -\textstyleStyleVernacularWordsItalic{iya} `comitative', but the meaning does not reflect the deictic origin of the initial part. In those rare cases when it is attached to a [+human] \textstyleAcronymallcaps{NP}, the referent of this \textstyleAcronymallcaps{NP} is subordinate to the referent of the other \textstyleAcronymallcaps{NP} and not in control, but still influencing the subject marking of the verb (\stepcounter{nx}{\thenx}). 

\ea%x824
\label{ex:x824}
\gll Mokok  urupa  kaik-i-man  nain  \textstyleEmphasizedVernacularWords{feekiya}  baurar-eka. \\
      \\
\glt
\z

eye  cup  tie-Np-PR.2p  that1  with  flee-IMP.2p

`Flee with your ``eye cups'' (a singsing decoration) still on.'

\ea%x825
\label{ex:x825}
\gll Wiamun  gelemuta  pun  aaw-ep  \textstyleEmphasizedVernacularWords{feekiya}  ikiw-e-mik. \\
      \\
\glt
\z

3s/p.brother  small  also  take-SS.SEQ  with  go-PA-1/3p

`He took his little brother too, and went with him.'

\ea%x1818
\label{ex:x1818}
\gll Maa  eliw  akena  nain  aaw-ep  \textstyleEmphasizedVernacularWords{feekiya}  ikiw-o-k. \\
      \\
\glt
\z

thing  good  very  that1  take-SS.SEQ  with  go-PA-3s

`He took the very good thing and went with it.'

The dual comitative \textstyleStyleVernacularWordsItalic{ikos} `with, together (with)' can only be used when two human participants are referred to (\stepcounter{nx}{\thenx}). It can also occur alone, without a preceding noun phrase, when the participants are known from the person/number suffix in the verb, or from the context (\stepcounter{nx}{\thenx}). The parties are considered equally active, so the verb is always in the plural.

\ea%x752
\label{ex:x752}
\gll Wekera  \textstyleEmphasizedVernacularWords{ikos}  irak-e-mik. \\
      \\
\glt
\z

3s/p.sister  with  fight-PA-1/3p

`He fought with his sister.'

\ea%x753
\label{ex:x753}
\gll \textstyleEmphasizedVernacularWords{Ikos}  ikiw-i-yen. \\
      \\
\glt
\z

with  go-Np-FU.1p

`Let's go together (just the two of us).'

Associative \textstyleStyleVernacularWordsItalic{ame} `with others' is different from the comitative postpositions above in that only one of the parties is specified. The identity of `the others' is left unspecified.

\ea%x826
\label{ex:x826}
\gll Auwa  \textstyleEmphasizedVernacularWords{ame}  wia  maak-eya  res  aaw-ep \\
      \\
\glt
\z

1s/p.father  ASSOC  3p.ACC  tell-2/3s.DS  razor  take-SS.SEQ  

merena  puuk-a-mik.

leg  cut-PA-1/3p

`He told my father and the others, and they took a razor and made a cut on his leg.'

\ea%x827
\label{ex:x827}
\gll Kuuten  \textstyleEmphasizedVernacularWords{ame}=ke  miim-e-mik. \\
      \\
\glt
\z

Kuuten  ASSOC=CF  precede-PA-1/3p

`Kuuten with (some) others went ahead of them.'

\subsubsection{Reason postposition \textit{muuta (nain)}}
\hypertarget{RefHeading21401935131865}{}
\textstyleStyleVernacularWordsItalic{Muuta}\textstyleStyleVernacularWordsItalic{ (nain)} `because of, for' gives a reason for an action, when the reason is expressed in a noun phrase rather than a full clause. It has developed from the adverb \textstyleStyleVernacularWordsItalic{muuta} `a little, only', and in some cases the meaning `only' is retained with the new function as well (\stepcounter{nx}{\thenx}), (\stepcounter{nx}{\thenx}). The distal-1 demonstrative \textstyleStyleVernacularWordsItalic{nain} `that' is optional, and is left out especially when there is another demonstrative \textstyleStyleVernacularWordsItalic{nain} preceding \textstyleStyleVernacularWordsItalic{muuta} (\stepcounter{nx}{\thenx}).

\ea%x756
\label{ex:x756}
\gll Iiriw  miiwa  \textstyleEmphasizedVernacularWords{muuta  nain}  irak-owa  marew,  oram \\
      \\
\glt
\z

earlier  land  for  that1  fight-NMZ  no(ne)  just  

momor  mauw-am-ik-e-mik.

indiscriminately  work-SS.SIM-be-PA-1/3p

`Earlier there was no fighting for land, they just worked indiscriminately (on any land).'

\ea%x1876
\label{ex:x1876}
\gll Yia  amukar-owa  \textstyleEmphasizedVernacularWords{muuta  nain}  nan  iiriw  ifakim-e-mik. \\
      \\
\glt
\z

1p.ACC  scold-NMZ  for  that1  there  earlier  kill-PA-1/3p

`We killed her earlier (only) because she scolded us (lit: {\dots}for her scolding of us).'

\ea%x759
\label{ex:x759}
\gll Opora  ara  nain  \textstyleEmphasizedVernacularWords{muuta}  ifakim-u  na-ep  on-a-mik. \\
      \\
\glt
\z

Talk  section  that1  for  kill-IMP.1d  say-SS.SEQ  do-PA-1/3p

`(Only) because of that talk they tried to kill him.'

\subsubsection{Comparison postposition \textit{saarik}}
\hypertarget{RefHeading21421935131865}{}
\textstyleStyleVernacularWordsItalic{Saarik} `like' occurs with noun phrases (\stepcounter{nx}{\thenx}) and with nominalized clauses (\stepcounter{nx}{\thenx}).  It indicates a point of similarity between two essentially \textstyleEmphasizedWords{different} things.

\ea%x760
\label{ex:x760}
\gll Mua  eliwa  \textstyleEmphasizedVernacularWords{saarik}  aakun-e-k. \\
      \\
\glt
\z

man  good  like  speak-PA-3s

`He spoke like a good man.'

\ea%x761
\label{ex:x761}
\gll No  sia  on-owa  \textstyleEmphasizedVernacularWords{saarik}  magimal  puuk-a-n. \\
      \\
\glt
\z

2s.UNM  netbag  make-NMZ  like  vine.sp.  cut-PA-2s

`You cut \textstyleForeignWords{magimal} vine as if you were going to make a netbag.'

For the functional category of comparison, see \sectref{sec:6.5}.

\subsubsection{Locative clitic -\textit{pa}}
\hypertarget{RefHeading21441935131865}{}
The locative clitic -\textstyleStyleVernacularWordsItalic{pa}  mainly marks locative in noun phrases (\stepcounter{nx}{\thenx}). The most common verb that it collocates with is \textstyleStyleVernacularWordsItalic{ik}- `be' (\stepcounter{nx}{\thenx}). When it occurs with the directional verbs (\sectref{sec:3.8.4.4.5}), it often indicates source (\stepcounter{nx}{\thenx}), but it can also be used for path (\stepcounter{nx}{\thenx}), or for instrument in cases where it has a strong locative meaning as well (\stepcounter{nx}{\thenx}). It is rarely used for goal (\stepcounter{nx}{\thenx}); this is possible in cases where the goal is the location for an event taking place immediately.

\ea%x762
\label{ex:x762}
\gll Pon  piipa  unowa=\textbf{pa}  soomar-em-ik-eya  mik-a-m. \\
      \\
\glt
\z

turtle  seaweed  many=LOC  walk-SS.SIM-be-2/3s.DS  spear-PA-1s

`The turtle was walking among the seaweeds and I speared it.'

\ea%x1880
\label{ex:x1880}
\gll Ona  owowa=\textstyleEmphasizedVernacularWords{pa}  ik-eya  epa  wiim-o-k. \\
      \\
\glt
\z

3s.GEN  village=LOC  be-2/3s.DS  place  dawn-PA-3s

`When he was in his village it dawned.'

\ea%x1877
\label{ex:x1877}
\gll Ifa  maneka=ke  iinan=\textstyleEmphasizedVernacularWords{pa}  or-o-k. \\
      \\
\glt
\z

snake  big=CF  on.top=LOC  descend-PA-3s

`A big snake dropped from above.'

\ea%x770
\label{ex:x770}
\gll Saa=\textstyleEmphasizedVernacularWords{pa}  ir-am-ika-i-mik. \\
      \\
\glt
\z

sand=LOC  come/go-SS.SIM-be-Np-PR.1/3p

`They are coming on/along the beach.'

\ea%x771
\label{ex:x771}
\gll Miiw  aasa=\textstyleEmphasizedVernacularWords{pa}  ikiw-e-mik. \\
      \\
\glt
\z

land  canoe=LOC  go-PA-1/3p

`We went by car/in a car.'

\ea%x1879
\label{ex:x1879}
\gll Mua  nain  ...  eka  kapa\textstyleEmphasizedVernacularWords{=pa}  ir-ap  eka  nain \\
      \\
\glt
\z

man  that1  ...  river  top=LOC  come/go-SS.SEQ  river  that1  

up-o-k.

block-PA-3s

`The man went to the top/source of the river and blocked the river.'

As temporal phrases locate an event in time, they also use the same locative clitic (\stepcounter{nx}{\thenx}).

\ea%x763
\label{ex:x763}
\gll Fraide=\textstyleEmphasizedVernacularWords{pa}  maapora  puk-o-k,  urera. \\
      \\
\glt
\z

Friday=LOC  celebration  burst-PA-3s,  afternoon.

`On Friday the celebration started, in the afternoon.'

It can also be used with an essive meaning, when referring to people's jobs:

\ea%x765
\label{ex:x765}
\gll Yena  mua  owowa  ekap-o-k,  amia  mua=\textstyleEmphasizedVernacularWords{pa}  ik-ok. \\
      \\
\glt
\z

1s.GEN  man  village  come-PA-3s  bow  man=LOC  be-SS 

`My husband came back to the village, having been a policeman.'

The locative clitic has its origin in the word \textstyleStyleVernacularWordsItalic{epa} `place'; the transition vowel [e] can sometimes be heard between the clitic and its host, when the host word ends in a consonant (\stepcounter{nx}{\thenx}).

\ea%x764
\label{ex:x764}
\gll Ne  Sarak  ikos  Gawar=(\textstyleEmphasizedVernacularWords{e})\textstyleEmphasizedVernacularWords{pa}  ik-emkun  yia  maak-e-mik  {\dots} \\
      \\
\glt
\z

ADD  Sarak  with  Gawar=LOC  be-1s/p.DS  1p.ACC  tell-PA-1/3p

`And as Sarak and I were in Gawar, they told us, {\dots} '

\subsubsection{Instrumental clitic -\textit{iw}}
\hypertarget{RefHeading21461935131865}{}
The instrumental clitic -\textstyleStyleVernacularWordsItalic{iw}\textstyleStyleVernacularWordsItalic{} is used both for concrete (\stepcounter{nx}{\thenx}) and abstract (\stepcounter{nx}{\thenx}) instruments.\footnote{A less emphasized way to add an instrument is to use the chaining structure: `take instrument do something' (\stepcounter{nx}{\thenx}), (\stepcounter{nx}{\thenx}).} 

\ea%x766
\label{ex:x766}
\gll Nomokowa  galua-galua  nain=\textstyleEmphasizedVernacularWords{iw}  biiris  on-am-ik-e-mik \\
      \\
\glt
\z

tree  soft-soft  that1=INST  bridge  make-SS.SIM-be-PA-1/3p

`They kept making bridges with soft timber.'

\ea%x768
\label{ex:x768}
\gll ...wiena  opaimik=\textstyleEmphasizedVernacularWords{iw}  yia  maak-em-ik-e-mik. \\
      \\
\glt
\z

3p.GEN  mouth=INST  1p.ACC  tell-SS.SIM-be-PA-1/3p

`They talked to us in their language.'

It is also used for path and has the meaning `along'; the verb indicates action that continues for some time.

\ea%x767
\label{ex:x767}
\gll Saa=\textstyleEmphasizedVernacularWords{iw}  ir-am-ika-i-mik, ... \\
      \\
\glt
\z

sand=INST  ascend-SS.SIM-be-Np-PR.3p

`They are coming along the beach{\dots}'

The difference between (\stepcounter{nx}{\thenx}) and (\stepcounter{nx}{\thenx}) above is that in (\stepcounter{nx}{\thenx}) the people were coming along the beach at least some of the way, and more specifically at the time of the speaking; whereas (\stepcounter{nx}{\thenx}) indicates that they travelled along the beach more or less the whole way.

The instrumental may also be utilized to indicate manner: 

\ea%x1881
\label{ex:x1881}
\gll Uurik  ona  naap=\textstyleEmphasizedVernacularWords{iw}  iw-ap  poka  aaw-e-mik. \\
      \\
\glt
\z

tomorrow  place  thus=INST  go-SS.SEQ  housepost  get-PA-1/3p

`The following day they went in the same way and got houseposts.'

\ea%x773
\label{ex:x773}
\gll Karu-(o)w=\textstyleEmphasizedVernacularWords{iw}  ekap-o-k. \\
      \\
\glt
\z

run-NMZ=INST  come-PA-3s

`He came running.'

\ea%x1814
\label{ex:x1814}
\gll Ne  ikoka  maa  marew  eliw \textstyleEmphasizedVernacularWords{} manek=\textstyleEmphasizedVernacularWords{iw}  ika-i-nan. \\
      \\
\glt
\z

ADD  later  thing  none  well  big=INST  be-Np-FU.2s

`And later you will have no problems, you will just be very well.'

\ea%x774
\label{ex:x774}
\gll Waaya=ke  anane  wiar  en-ow=\textstyleEmphasizedVernacularWords{iw}  ika-i-ya. \\
      \\
\glt
\z

pig=CF  always  3.DAT  eat-NMZ=INST  be-Np-PR.3s

`A pig stays eating their (taro) all the time.'

Another usage is in those temporal phrases that refer to something taking place repeatedly at the same time: 

\ea%x1882
\label{ex:x1882}
\gll I  amirik=\textstyleEmphasizedVernacularWords{iw}  ...  Gawar  wiar  ikiw-e-mik. \\
      \\
\glt
\z

1p.UNM  daytime=INST  {\dots}  Gawar  3.DAT  go-PA-1/3p

`In the daytime we always went to Gawar {\dots}'

\subsubsection{Limiter -\textit{iw}}
\hypertarget{RefHeading21481935131865}{}
The limiter clitic -\textstyleStyleVernacularWordsItalic{iw}  `only, just' is homophonous with the instrumental. The two probably are of common origin, but synchronically their meanings and positions in the word are distinct (\sectref{sec:3.12.9}). The limiter does not mark a case but restricts the applicability of the predication to the element that it is attached to.  

\ea%x769
\label{ex:x769}
\gll [Maa  eka]\textsubscript{NP}=\textstyleEmphasizedVernacularWords{iw}  en-ep  en-ep  lebum-ar-i-nan. \\
      \\
\glt
\z

food  water=LIM  eat-SS.SEQ  eat-SS.SEQ  lazy-INCH-Np-FU.2s

`When you keep eating only food cooked with water you become tired of it.'

\ea%x772
\label{ex:x772}
\gll Mua=\textstyleEmphasizedVernacularWords{iw}  pok-aka. \\
      \\
\glt
\z

man=LIM  sit-IMP.2p

`Sit just among the men.'

The limiter clitic may attach itself to genitive and focal pronouns (\sectref{sec:3.5.7}). 

The free adverb \textstyleStyleVernacularWordsItalic{muutiw} `only' is a combination of \textstyleStyleVernacularWordsItalic{muut(a)} `only' and the limiter clitic.

\subsubsection{Topic and focus markers}
\hypertarget{RefHeading21501935131865}{}
The topic and focus markers indicate the discourse function of the noun phrases that they are attached to.

\paragraph[Topic markers]{Topic markers}
\hypertarget{RefHeading21521935131865}{}
Of the two topic markers \textstyleStyleVernacularWordsItalic{ena} is fairly low in frequency, and the description given here is only tentative. It seems that \textstyleStyleVernacularWordsItalic{ena} as an independent word originally had a topic marking function, but later the topic clitic \textstyleStyleVernacularWordsxiiptItalic{-(e)na}  developed from it and is now used for highlighted topic (\sectref{sec:9.1.2.4}) in main clauses. \textstyleStyleVernacularWordsItalic{Ena} still marks a topic, but only in relative clauses. It often has a specifying function as well: `the/that (particular one)'.  

\ea%x1681
\label{ex:x1681}
\gll [\textstyleEmphasizedVernacularWords{Mua}  \textstyleEmphasizedVernacularWords{ena}  ma-e-k  nain]  makena  yos. \\
      \\
\glt
\z

man  SPEC  say-PA-3s  that1  true  1s.FC

`The man that he talked about is I.'

In long relative clauses, where it is attached to the \textstyleAcronymallcaps{RelNP}, it helps to distinguish it from all the other \textstyleAcronymallcaps{NP}s in the clause. 

\ea%x1815
\label{ex:x1815}
\gll [\textstyleEmphasizedVernacularWords{Mua}  \textstyleEmphasizedVernacularWords{papako}  \textstyleEmphasizedVernacularWords{ena}  Australia=ke  wia  aaw-ep  wiena \\
      \\
\glt
\z

man  other  SPEC  Australia=CF  3p.ACC  take-SS.SEQ  3p.GEN

feekiya  yiaw-e-mik  nain]  me  epa  fan  irak-owa

with  walk.around-PA-1/3p  that1  not  place  here  fight-NMZ  

uruf-a-mik {\dots}

see-PA-1/3p

`Those other (particular) men whom the Australians took and with whom they walked around did not see the war here in this place {\dots}'

\ea%x1683
\label{ex:x1683}
\gll [\textstyleEmphasizedVernacularWords{I}  mua  owowa=pa  ik-ok  \textstyleEmphasizedVernacularWords{ena}  irakowa  uruf-a-mik \\
      \\
\glt
\z

1p.UNM  man  village=LOC  be-SS  SPEC  fight-NMZ  see-PA-1/3p

nain]  nanar  nain  yo  fan  ma-i-yem.

that1  story  that1  1s.UNM  here  say-Np-PR.1s

`I am telling the story of us (particular) people who stayed in the village and saw the fighting.'

If the head noun of the \textstyleAcronymallcaps{NP} is is recoverable from the context, it may be deleted, leaving behind only \textstyleStyleVernacularWordsItalic{ena}. In (\stepcounter{nx}{\thenx}) the head noun \textstyleStyleVernacularWordsItalic{epira} `bowl(s)' has been omitted.

\ea%x1682
\label{ex:x1682}
\gll [Aakisa  fan  \textstyleEmphasizedVernacularWords{ena}  maneka  wu-a-mik  nain]  eliw, \\
      \\
\glt
\z

now  here  SPEC  big  put-PA-1/3p  that1  well  

wie  wi  eliw  wiar  op-i-kuan.

3s/p.uncle  3p.UNM  well  3.DAT  grab-Np-FU.3p

`Those big (bowls) that we put just now, all right, the uncles may take those from them.'

The following example, taken from Bible translation, has a highlighted topic marker -\textstyleStyleVernacularWordsItalic{na} on the sentence-initial topic \textstyleAcronymallcaps{NP,} which is part of the main clause\textstyleAcronymallcaps{,} and \textstyleStyleVernacularWordsItalic{ena} inside the relative clause:

\ea%x1816
\label{ex:x1816}
\gll Ni  Samaria=na  [\textstyleEmphasizedVernacularWords{o  ena}  me  baliwep  amis-ar-e-man \\
      \\
\glt
\z

2p.UNM  Samaria=TP  3s.UNM  SPEC  not  well  knowledge-INCH

nain]\textsubscript{RC}  lotu  on-i-man.

that1  worship  do-Np-PR.2p

`You Samaritans worship [the one that you do not know well].'

Without \textstyleStyleVernacularWordsItalic{ena} in the relative clause the sentence would mean `You Samaritans do not know him well but (still) worship him.'

The more common topic clitic \textstyleStyleVernacularWordsItalic{-(e)na}\footnote{The clitic has mostly lost the phoneme /e/, but it can sometimes be heard when the host word ends in a voiceless consonant. }  is used to highlight a changed topic, to which attention is drawn. The topic may have been introduced in the immediately preceding clause. The use of this device is infrequent in texts.  It can often be glossed with `as for X'. Highlighted topics are discussed in \sectref{sec:9.1.2.4}.

Example (\stepcounter{nx}{\thenx}) is from a traditional story, where a man has gone hunting and the spirit of his lover comes to his home. When the wife sees her, she knows what the spirit woman has come to look for and comments:

\ea%x779
\label{ex:x779}
\gll Nena  mua=\textstyleEmphasizedVernacularWords{na}  urema  osarena  ikiw-o-k. \\
      \\
\glt
\z

2s.GEN  man=TP  bandicoot  path  go-PA-3s

`(As for) your husband, he went to catch bandicoots.'

In (\stepcounter{nx}{\thenx}) the answer to the question reveals the identity of the person asked about; the topic marker may be used even in a short exchange like this but especially if the text continues to tell more about the topic. 

\ea%x1751
\label{ex:x1751}
\gll Mua  nain  naareke?  Mua  nain=\textbf{na}  owow  saria  maneka=ke. \\
      \\
\glt
\z

Man  that1  who.CF  man  that1=TP  village  headman  big=CF

`Who is that man? -That man is the big village headman.'

In (\stepcounter{nx}{\thenx}) the speaker changes the topic to the addressee after a discussion on something else:

\ea%x780
\label{ex:x780}
\gll Nos=\textstyleEmphasizedVernacularWords{na}? \\
      \\
\glt
\z

2s.FC=TP

`(So,) what about you?'

An important function for the topic marker -\textstyleStyleVernacularWordsItalic{na}  is to mark conditional clauses (\sectref{sec:8.3.5}). This is a common function for topic markers in Papuan languages (Haiman 1978, Reesink 1983b and 1987:242, Foley 1986:203).

\ea%x744
\label{ex:x744}
\gll Opora  wiar  ika-i-ya=\textstyleEmphasizedVernacularWords{na}  eliw  urup-ep  wia \\
      \\
\glt
\z

talk  3.DAT  be-Np-PR.3s=TP  well  ascend-SS.SEQ  3p.ACC  

maak-uk.

tell-IMP.3p

`If they have something to say, they can get up and tell them.'

\ea%x745
\label{ex:x745}
\gll O  emeria  aaw-owa  kookal-ek-a-k=\textstyleEmphasizedVernacularWords{na} \\
      \\
\glt
\z

3s.UNM  woman  get-NMZ  like-CNTF-PA-3s=TP  

iw-ek-a-mik.

give.him-CNTF-PA-1/3p

`If he had liked to get a wife, they would have given him one.'

It is also used in adversative subordinate clauses (\sectref{sec:8.3.4}) when the main clause expresses a frustrated effort or a cancelled expectation (\stepcounter{nx}{\thenx}), or surprise.

\ea%x1399
\label{ex:x1399}
\gll Ikiw-ep  mukuna  nain  umuk-a-mik=\textstyleEmphasizedVernacularWords{na}  me  pepek. \\
      \\
\glt
\z

go-SS.SEQ  fire  that1  extinguish-PA-1/3p=TP  not  enough/able

`We went and (tried to) extinguish the fire, but couldn't.'

\paragraph[Focus clitics]{Focus clitics}
\hypertarget{RefHeading21541935131865}{}
There are two focus clitics, the contrastive focus marker -(\textstyleStyleVernacularWordsItalic{e})\textstyleStyleVernacularWordsItalic{ke} and the neutral focus marker -\textstyleStyleVernacularWordsItalic{ko}, which has developed from the indefinite \textstyleStyleVernacularWordsItalic{oko} `a certain, other'.\footnote{Most of this section is based on J\"arvinen (1988b:81-96).} The main candidate for the contrastive focus marker is the subject of a noun phrase (\stepcounter{nx}{\thenx}). When the object is fronted as a theme (\sectref{sec:9.1}), the subject usually gets the contrastive focus marking to distinguish it from the object (\stepcounter{nx}{\thenx}).  

\ea%x781
\label{ex:x781}
\gll Iiriw  ifa  marasin=\textstyleEmphasizedVernacularWords{ke}  kekan-e-k. \\
      \\
\glt
\z

earlier  snake  poison=CF  be.strong-PA-3s

`The snake poison had already taken effect.'

\ea%x782
\label{ex:x782}
\gll Episowa  ifa  nain  atua=\textstyleEmphasizedVernacularWords{ke}  en-e-k. \\
      \\
\glt
\z

tobacco  leaf  that1  worm=CF  eat-PA-3s

`The tobacco leaves were eaten by worms.'

Another possible host is the non-verbal predicate of a verbless clause (\sectref{sec:5.6}).

\ea%x783
\label{ex:x783}
\gll Iperuma  nain  me  enim-eka,  inasin  mua=\textstyleEmphasizedVernacularWords{ke}. \\
      \\
\glt
\z

eel  that1  not  eat-IMP.2p  spirit  man=CF

`Do not eat that eel, it is a spirit man.'

There are a few isolated cases where it occurs on some other contrasted element of a clause.

\ea%x784
\label{ex:x784}
\gll Amirika=\textstyleEmphasizedVernacularWords{ke}  eliw  ika-i-yem,  uura=\textstyleEmphasizedVernacularWords{ke}  napum-ar-i-yem. \\
      \\
\glt
\z

noon=CF  well  be-Np-PR.1s  night=CF  sickness-INCH-Np-PR.1s

`At noon I'm well, at night I am sick.'

Contrastive focus as a pragmatic device in a text is discussed in \sectref{sec:9.3.1}.

The neutral focus clitic \textstyleStyleVernacularWordsxiiptItalic{-ko} commonly occurs in irrealis-type clauses:\footnote{Because of this, it was called \textit{Irrealis focus clitic} in J\"arvinen (1988b).} questions, commands, negative clauses, or those with future tense. Unlike the contrastive focus marker, the neutral focus marker can be attached to almost any element of a clause except the final verb. 

\ea%x785
\label{ex:x785}
\gll Mukuna=\textstyleEmphasizedVernacularWords{ko}  op-a-man=i? \\
      \\
\glt
\z

fire=NF  hold-PA-2p=QM

`Did you hold any fire?'

\ea%x786
\label{ex:x786}
\gll Mua  nain=\textstyleEmphasizedVernacularWords{ko}  onak-e! \\
      \\
\glt
\z

man  that1=NF  give.to.eat-IMP.2s

`Give it to that man to eat!'

\ea%x787
\label{ex:x787}
\gll Oposia  en-e-man  nain  yo=\textstyleEmphasizedVernacularWords{ko}  me  uruf-a-m. \\
      \\
\glt
\z

meat  eat-PA-2p  that1  1s.UNM=NF  not  see-PA-1s

`I didn't (even) see the meat that you ate.'

\ea%x788
\label{ex:x788}
\gll Akim-ap=\textstyleEmphasizedVernacularWords{ko}  uruf-i-yen. \\
      \\
\glt
\z

try-SS.SEQ=NF  see-Np-FU.1p

`We'll try and see.'

Neutral focus as a textual device is discussed further in \sectref{sec:9.3.2}.

\subsubsection{Question marker}
\hypertarget{RefHeading21561935131865}{}
The question marker -\textstyleStyleVernacularWordsItalic{i}  is a sentential clitic, used to form polar questions, and it attaches itself to the clause-final verb or another clause-final element. Its relationship to the alternative connective \textstyleStyleVernacularWordsItalic{e} `or' (\sectref{sec:3.11.2}) is unclear; it is possible that \textstyleStyleVernacularWordsItalic{i} was originally an alternative connective but was also employed as a question marker and became so established in this function that a new  alternative connective \textstyleStyleVernacularWordsItalic{e}  developed. It is not uncommon in \textstyleAcronymallcaps{TNG} Madang languages that the question marker and the alternative connective are either the same or closely related.\footnote{In Usan and Amele the alternative connector and the question marker are the same (Reesink 1987:293, Roberts 1987:99). Maia has \textit{-i} `QM' and \textit{e} `or' like Mauwake \citep[83,159]{Hardin2002}. Bargam has borrowed the Tok Pisin \textit{o} as an alternative connector but has retained the clitic \textit{-e} as the question marker \citep[53,122]{Hepner2002}. Kobon may use the alternative interrogative connector \textit{aka} `or' sentence-finally in leading polar questions, where the speaker expects the addressee to agree with the proposition. }   

\ea%x789
\label{ex:x789}
\gll Sira  nain  piipua-i-nan=\textstyleEmphasizedVernacularWords{i}? \\
      \\
\glt
\z

habit  that1  leave-Np-FU.2s=QM

`Will you give up that habit?'

\ea%x790
\label{ex:x790}
\gll Nobonob  ikiw-e-man  nain,  owowa  eliwa=\textstyleEmphasizedVernacularWords{i}? \\
      \\
\glt
\z

Nobonob  go-PA-2p  that1  village  good=QM

`You went to Nobonob, is it a good village?'

The question marker can also be used in statements, when two or more alternatives are given:

\ea%x791
\label{ex:x791}
\gll Mua  kuisow  manina  erup=\textstyleEmphasizedVernacularWords{i}  (e)  arow=\textstyleEmphasizedVernacularWords{i}  (e)  naap. \\
      \\
\glt
\z

man  one  garden  two=QM  (or)  three=QM  (or)  thus

`A man can have two gardens, or three, like that.'

\subsubsection{Co-occurrence of the clitics}
\hypertarget{RefHeading21581935131865}{}
It is possible to have two or more clitics attached to the same word; only the topic marker does not allow other clitics with it.  A case marking clitic, forming a constituent with the preceding noun phrase, is placed first. Next comes the focus clitic (\stepcounter{nx}{\thenx}), with the scope over a phrase but not forming a constituent. The limiter \nobreakdash-\textstyleStyleVernacularWordsItalic{iw}  may have a phrase with the focus marking in its scope, so it follows the focus marker.  When the limiter follows another clitic, there is a transition consonant /r/  between the clitics.The modal clitic \nobreakdash-\textstyleStyleVernacularWordsItalic{yon} `perhaps' (\sectref{sec:3.9.3}) may have a scope over a whole predication, so its position is after the limiter. The sentential clitic, with a scope over the whole sentence, comes last. When the contrastive focus marker -\textstyleStyleVernacularWordsItalic{ke} and the question marker -\textstyleStyleVernacularWordsItalic{i} are adjacent they become a portmanteau clitic -\textstyleStyleVernacularWordsItalic{ki}  (\stepcounter{nx}{\thenx}).

\ea%x792
\label{ex:x792}
\gll Fura=\textstyleEmphasizedVernacularWords{iw}=\textstyleEmphasizedVernacularWords{ko}  me  puuk-a-mik. \\
      \\
\glt
\z

knife=INST=NF  not  cut-PA-1/3p

`They didn't cut it with a knife.'

\ea%x795
\label{ex:x795}
\gll Fikera=\textstyleEmphasizedVernacularWords{pa}-r=\textstyleEmphasizedVernacularWords{iw}  fiirim-eka. \\
      \\
\glt
\z

kunai.grass=LOC-{\O}=LIM  gather-IMP.2p

`Gather them right at the \textstyleForeignWords{kunai} grass.'

\ea%x796
\label{ex:x796}
\gll Os=\textstyleEmphasizedVernacularWords{ke}-r=\textstyleEmphasizedVernacularWords{iw}  maa  en-emi  ewur-ar-e-k. \\
      \\
\glt
\z

3s.FC=CF-{\O}=LIM  food  eat-SS.SIM  haste-INCH-PA-3s

`Only he rushed with his food.'

\ea%x793
\label{ex:x793}
\gll Wi  anim  onoma  pun  o  makena  Krais=ke \\
      \\
\glt
\z

3p.UNM  blade  basis  also  3s.UNM  truly  Christ=CF

na-i-mik=\textstyleEmphasizedVernacularWords{yon}=\textstyleEmphasizedVernacularWords{i}?

say-Np-PR.1/3p=perhaps=QM

`Are also the authorities perhaps saying that he truly is Christ?'

\ea%x794
\label{ex:x794}
\gll Mua  fain  Saror  muuka=\textstyleEmphasizedVernacularWords{ki}? \\
      \\
\glt
\z

man  this  Saror  son=CF.QM

`Is this man Saror's son?'

\subsection{Interjections}
\hypertarget{RefHeading21601935131865}{}
There are a lot of interjections in Mauwake; the following list is not even nearly exhaustive. The pronunciation of interjections may differ from that of other words: intonational variations are greater, and lengthening, even extreme lengthening, of the final vowel is common. Interjections are not part of the normal clause structure, and they usually occur either sentence-initially or finally. Some can also be placed between clauses in a coordinate sentence (\stepcounter{nx}{\thenx}). The glosses given below for the interjections are just rough approximations.  

a      impatience, also used as a filler

aa    `oh'  emphasizes what has been said

ae    `yes'  agreement\footnote{The negators \textit{weetak} and \textit{wia} `no' also function as interjections but are not listed here, because they have other functions as well (see \sectref{sec:3.10} and \sectref{sec:6.2}).} 

aiyoo    distress, disapproval

arika\footnote{This is obviously an imperative derived from the discourse-marker \textit{aria} (\sectref{sec:3.9.1}), as it is only used with second person plural, whereas \textit{aria} is used with all other persons. Elsewhere \textit{aria} has no verb-like qualities.}  `OK, let's go'  exhorting others to get going

awue  `wow  any strong emotion, surprise

ee      delight

ei    `hey!'  being surprised or startled

emawa  `sorry'  expression of empathy, especially grief or pity

emawik  `excuse me'  speech opening in a controversial situation

faa      disgust or astonishment

maa senam  `watch out!'  grave warning (lit: `thing too.much')

oo    `o'  calling someone

oo [oo[241?]]  `yes'  agreement

na      strengthens an imperative

nii?  `really? \textstyleDefinitionE{oh?'  r}\textstyleEncyclopedicinfoE{esponse to hearing something surprising} 

nom\textstylefstandard{  `please'  when repeating a request or command}

noma\textstylefstandard{  `oh dear' } 

sa, se    impatience, disapproval

se-ek  `wow  great happiness

wiisak  `sorry'  mild regret, for minor losses

yaa      impatience

yee    `oh'  recognition, emphasis  

yii    `eek', `oh'  fear, sorrow

\ea%x797
\label{ex:x797}
\gll \textstyleEmphasizedVernacularWords{Aa},  kema=ko  kir-ek-a-n  \textstyleEmphasizedVernacularWords{aa}! \\
      \\
\glt
\z

oh  liver=NF  turn-CNTF-PA-2s  oh

`Oh, if only you had changed your ways (lit: turned your liver)!'

\ea%x798
\label{ex:x798}
\gll \textstyleEmphasizedVernacularWords{Arika},  takira,  yo  yook-eka. \\
      \\
\glt
\z

let's.go  boy  1s.UNM  follow.me-IMP.2p

`Alright boys, follow me (and will get going).'

\ea%x1225
\label{ex:x1225}
\gll Laman  tapala  wu-a-k,  \textstyleEmphasizedVernacularWords{aiyoo}! \\
      \\
\glt
\z

Laman  hat  put-PA-3s  INTJ

`My goodness, Laman put a hat on (and exposed himself and us to the fighter pilots)!'

\ea%x799
\label{ex:x799}
\gll \textstyleEmphasizedVernacularWords{Emawa},  nena  niawi  um-o-k. \\
      \\
\glt
\z

sorry  2s.GEN  2s/p.father  die-PA-3s

`Sorry, your father is dead.'

\ea%x800
\label{ex:x800}
\gll Naap  ma-emi  om-em-ika-i-nan,  \textstyleEmphasizedVernacularWords{na}. \\
      \\
\glt
\z

thus  say-SS.SIM  cry-SS.SIM-be-Np-FU.2s  INTJ

`You must cry saying like that.'

\ea%x801
\label{ex:x801}
\gll Yo  damol-al-e-m  \textstyleEmphasizedVernacularWords{oo},  fiker  fufa  iw-a-m  \textstyleEmphasizedVernacularWords{oo}. \\
      \\
\glt
\z

1s.UNM  bad-INCH-PA-1s  oh  kunai.grass  old.grass  enter-PA-1s  oh

` Oh, I'm in a bad way, I am hiding among the grass.'

\ea%x802
\label{ex:x802}
\gll Ni  kaaneke  ik-e-man  \textstyleEmphasizedVernacularWords{oo},  ni  ekap-omak-eka \\
      \\
\glt
\z

2p.UNM  where  be-PA-2p  oh  2p.UNM  come-DISTR/PL-IMP.2p

\textstyleEmphasizedVernacularWords{oo}!

oh

`O where are you? -- come!'

\ea%x803
\label{ex:x803}
\gll \textstyleEmphasizedVernacularWords{Yii},  ifa=ke  \textstyleEmphasizedVernacularWords{yee}! \\
      \\
\glt
\z

Eek  snake=CF  INTJ

`Eek, that's a snake oh!'

\section{Phrase level syntax}
\hypertarget{RefHeading21621935131865}{}
\subsection{Noun phrase}
\hypertarget{RefHeading21641935131865}{}
The noun phrase in Mauwake functions in a clause as subject, object or non-verbal predicate. It can also function in an adverbial phrase, or as a possessor, qualifier or post-modifier in another noun phrase.

\subsubsection{Basic noun phrase}
\hypertarget{RefHeading21661935131865}{}
The the most common noun phrase structure consists of only the head noun. That is slightly more frequent than a head noun plus one or more attributive elements. The head noun may have either pre- or postmodifiers, or both. The relative order of the \textstyleAcronymallcaps{NP} constituents is as follows:\footnote{The superscript \textsuperscript{n} indicates that it is possible to have more than one of these constituents within a single NP.}

{\bfseries
Unmarked/Genitive pronoun - Temporal phrase - Possessive NP - Genitive pronoun - Qualifying NP - HEAD NOUN - Modifying NP - Adjective phrase\textsuperscript{n} - Quantifier phrase\textsuperscript{n} / Indefinite - Demonstrative - Dative pronoun}

The relative clause, where the head noun is modified by a whole clause, is discussed in \sectref{sec:8.3.1}.

The order of \textstyleAcronymallcaps{NP} constituents following the head noun agrees with a cross-linguistic generalization of \textstyleAcronymallcaps{SOV} languages:  N-A-Num-Dem \citep[112]{Dryer2007a}.

Theoretically it is quite possible, and grammatically correct, to have a \textstyleAcronymallcaps{NP} like the one in (\stepcounter{nx}{\thenx}), but natural language data seldom has any \textstyleAcronymallcaps{NP}s with more than two modifiers; example (\stepcounter{nx}{\thenx}) is one of those.

PossNP  GenPr  QualNP  HN  AP  QP  Dem

\ea%x392
\label{ex:x392}
\gll auwa  ona  mera  sia  maala  erup  nain \\
      \\
\glt
\z

1s/p.father  3s.GEN  fish  net  long  two  that1

`my father's two long fish nets' / `the/those two long fish nets of my father'

TmpP  PossNP  HN  AP  Dem

\ea%x393
\label{ex:x393}
\gll iiriw  Naawura  miiw-aasa  awona  nain \\
      \\
\glt
\z

earlier  Naawura  land-canoe  old  that1

`the/that earlier truck of Naawura's'

\ea%x394
\label{ex:x394}
\gll yiena  iiriw  kae  sira  nain \\
      \\
\glt
\z

1p.GEN  earlier  1s/p.grandfather  custom  that1

`that traditional custom of ours'

The only modifier in a noun phrase most typically is either a possessor (\stepcounter{nx}{\thenx}), a deictic (\stepcounter{nx}{\thenx}) or a qualifying noun or noun phrase \textstyleAcronymallcaps{NP}. In (\stepcounter{nx}{\thenx}) the qualifying noun is a compound noun.

\ea%x395
\label{ex:x395}
\gll ona  siowa \\
      \\
\glt
\z

3s.GEN  dog

`his/her dog'

\ea%x396
\label{ex:x396}
\gll ifa  nain \\
      \\
\glt
\z

snake  that1

`the/that snake'

\ea%x397
\label{ex:x397}
\gll owow(a)  maneka  mua \\
      \\
\glt
\z

village  big  man

`townsman'

In the following, each \textstyleAcronymallcaps{NP} position is discussed in turn, starting with the leftmost one.  

An \textstyleEmphasizedWords{\textsc{unmarked third person plural pronoun}} is used as an optional plural marking for humans and other human-like beings (\stepcounter{nx}{\thenx}), (\stepcounter{nx}{\thenx}). 

\ea%x398
\label{ex:x398}
\gll \textstyleEmphasizedVernacularWords{Wi}  sawur=ke  kuura  puuk-a-mik. \\
      \\
\glt
\z

3p.UNM  spirit=CF  fly  cut-PA-1/3p

`The spirits changed into flies.'

\ea%x1831
\label{ex:x1831}
\gll Ne  nan  \textstyleEmphasizedVernacularWords{wi}  owow  mua  wia  maak-e-mik,  {\dots} \\
      \\
\glt
\z

ADD  there  3p.ACC  village  man  3p.ACC  tell-PA-1/3p

`And there they told the village men, {\dots}'

The plural-marking pronoun differs from the appositive use (\stepcounter{nx}{\thenx}) of the unmarked pronoun in that the former is unstressed, whereas the latter is stressed and, furthermore, may be any person and either singular or plural. (For appositional NPs, see \sectref{sec:4.1.4}.)

\ea%x399
\label{ex:x399}
\gll \textbf{'}\textstyleEmphasizedVernacularWords{Yo}  nena  niawi=ke  nefa  maak-i-yem. \\
      \\
\glt
\z

1s.UNM  2s.GEN  2s/p.father=CF  2s.ACC  tell-Np-PR.1s

`I, your father, tell you...'

A special case of the plural-marking unmarked pronoun is where it occurs with a place name to refer to the people of that place (\stepcounter{nx}{\thenx}).  The head noun \textstyleStyleVernacularWordsItalic{mua} `men, people' or \textstyleStyleVernacularWordsItalic{emeria} \textstyleStyleVernacularWordsItalic{mua} `people' is not needed; it may be used, but is usually left out.

\ea%x400
\label{ex:x400}
\gll \textstyleEmphasizedVernacularWords{Wi}  Lasen=ke  ekap-e--mik. \\
      \\
\glt
\z

1p.UNM  Lasen=CF  come-PA-1/3p

`The Lasen (village) people came.'

A \textstyleEmphasizedWords{\textsc{temporal phrase}} is rare as a \textstyleAcronymallcaps{NP} constituent. Mainly the temporal words \textstyleStyleVernacularWordsItalic{aakis}  'present-day' from \textstyleStyleVernacularWordsItalic{aakisa} 'now, today' (\stepcounter{nx}{\thenx}) and \textstyleStyleVernacularWordsItalic{iiriw} 'earlier' (\stepcounter{nx}{\thenx}) may be used, but a temporal phrase is also allowed : 

\ea%x401
\label{ex:x401}
\gll ni  \textstyleEmphasizedVernacularWords{aakis}  takira \\
      \\
\glt
\z

2p.UNM  present-day  young.person

`you young people of today'

\ea%x1883
\label{ex:x1883}
\gll wi  [\textstyleEmphasizedVernacularWords{iiriw  akena}]  mua \\
      \\
\glt
\z

3p.UNM  earlier  truly  man

`the people of long ago'

The structure of the two pre-modifying \textstyleAcronymallcaps{NP}s, possessive \textstyleAcronymallcaps{NP} and qualifying \textstyleAcronymallcaps{NP}, is similar to that of the basic \textstyleAcronymallcaps{NP}. It is because of their position and function inside another \textstyleAcronymallcaps{NP} that they are here called by different names.

The head noun of a \textstyleEmphasizedWords{\textsc{possessive} }\textstyleAcronymallcaps{\textup{NP}} can only be [+human], with `human' including spirits (\stepcounter{nx}{\thenx}) and sometimes some domestic animals like dogs or pigs (\stepcounter{nx}{\thenx}). The humanness of the \textstyleAcronymallcaps{PossNP} is stressed by the fact that it may be followed by a pronoun copy in the genitive (\stepcounter{nx}{\thenx}).

\ea%x402
\label{ex:x402}
\gll \textstyleEmphasizedVernacularWords{sawur  emeria } ona  onak  wiawi \\
      \\
\glt
\z

spirit  woman  3s.GEN  3s/p.mother  3s/p.father

`the spirit woman's parents'

\ea%x403
\label{ex:x403}
\gll \textstyleEmphasizedVernacularWords{siowa}  wiawi \\
      \\
\glt
\z

dog  3s/p.father

`the dog's owner'

The head noun of the \textstyleAcronymallcaps{PossNP} may itself be possessed:

\ea%x404
\label{ex:x404}
\gll \textstyleEmphasizedVernacularWords{yiena  kae}  sira \\
      \\
\glt
\z

1p.GEN  1s/p.grandfather  custom

`our ancestors' (lit: grandfathers') custom'

[PossNP          [Poss NP            [HN ]]]   Dem 

\ea%x405
\label{ex:x405}
\gll \textstyleEmphasizedVernacularWords{i}  \textstyleEmphasizedVernacularWords{emeria  apura  yiena  mua  weria}  emeria  nain=ke \\
      \\
\glt
\z

1p.UNM  woman  widow  1p.GEN  man  planting.stick  woman  that1=CF

`the wives of the ``weria-men''\footnote{The \textit{weria}-men are relatives responsible for a person's burial. For more information, see \sectref{sec:1.3.6}.} of us widows'

The semantic relation of the ``possessor'' to the ``possessed'' may be that of real ownership, paraphrasable with `have' (\stepcounter{nx}{\thenx}), a human relationship (\stepcounter{nx}{\thenx}), origin (\stepcounter{nx}{\thenx}) or subjecthood (\stepcounter{nx}{\thenx}). 

\ea%x406
\label{ex:x406}
\gll ona  koora \\
      \\
\glt
\z

3s.GEN  house

`his house'

\ea%x407
\label{ex:x407}
\gll takira  niir-owa \\
      \\
\glt
\z

youth  play-NMZ

`young  people's play(ing)'

Either an unmarked pronoun or a genitive pronoun may be used as a \textstyleEmphasizedWords{\textsc{possessive pronoun}}. Often the two can be used interchangeably, but the following rules and tendencies have been observed. When the pronoun is a pronoun copy of a preceding possessive \textstyleAcronymallcaps{NP} it must be in the genitive (\stepcounter{nx}{\thenx}). 

\ea%x409
\label{ex:x409}
\gll \textstyleEmphasizedVernacularWords{Ona}  apura  maa  oposia  me  enim-i-non. \\
      \\
\glt
\z

3s.GEN  widow  thing  meat  not  eat-Np-FU.3s

`His widow will not eat meat.'

\ea%x410
\label{ex:x410}
\gll sawur  emeria  \textstyleEmphasizedVernacularWords{ona}  onak  wiawi \\
      \\
\glt
\z

spirit  woman  3s.GEN  3s/p.mother  3s/p.father

`the spirit woman's parents'

An unmarked pronoun is used especially with things that are closely related to a person, and the genitive pronoun tends to be used more when the ownership is emphasized. 

\ea%x1315
\label{ex:x1315}
\gll Oo,  \textstyleEmphasizedVernacularWords{no  emeria}  iiriw  sesek-a-mik. \\
      \\
\glt
\z

Oh  2s.UNM  woman  already  send-PA-1/3p

`Oh, we already sent your wife away.'

\ea%x1314
\label{ex:x1314}
\gll Nep(a)  opaimika  me  amis(a)-ar-ep  \textstyleEmphasizedVernacularWords{wiena} \\
      \\
\glt
\z

bird  talk  not  knowledge-INCH-SS.SEQ  3p.GEN

\textstyleEmphasizedVernacularWords{opaimik(a)}\textstyleEmphasizedVernacularWords{=iw}  yia  maak-em-ik-e-mik.

talk=INST  1p.ACC  tell-SS.SIM-be-PA-1/3p

`They did not know Tok Pisin and talked to us in their (own) language.'

An unmarked pronoun used possessively is often stressed in speech (\stepcounter{nx}{\thenx}).

\ea%x408
\label{ex:x408}
\gll Nain  \textstyleEmphasizedVernacularWords{'i } sira=ke. \\
      \\
\glt
\z

that1  1p.UNM  custom=CF

`That is our custom.'

In recursive genitive structures like (\stepcounter{nx}{\thenx}) more than one possessive pronoun may occur as a pronoun copy, so (\stepcounter{nx}{\thenx}) is a possible alternative for (\stepcounter{nx}{\thenx}):

\ea%x411
\label{ex:x411}
\gll i  emeria  apura  \textstyleEmphasizedVernacularWords{yiena}  mua  weria  \textstyleEmphasizedVernacularWords{wiena} \\
      \\
\glt
\z

1p.UNM  woman  widow  1p.GEN  man  planting.stick  3p.GEN

emeria  nain=ke

woman  that1=CF

`the wives of the ``weria-relatives'' of us widows'

A \textstyleEmphasizedWords{\textsc{qualifying noun phrase}} usually consists of the head noun only. If it has other elements, the structure is the same as that of the basic \textstyleAcronymallcaps{NP}. The distinction between a qualifying \textstyleAcronymallcaps{NP} and a possessive \textstyleAcronymallcaps{NP} on the one hand, and between a qualifying \textstyleAcronymallcaps{NP} and a \textstyleAcronymallcaps{N+N} compound on the other, is often hard to make. (See \sectref{sec:3.2.5}  for a discussion on the distinction between compound nouns and \textstyleAcronymallcaps{NP}s.) Unlike a possessive \textstyleAcronymallcaps{NP}, a qualifying \textstyleAcronymallcaps{NP} may not take a genitive pronoun copy. 

\ea%x413
\label{ex:x413}
\gll \textstyleEmphasizedVernacularWords{Fiker(a)  epia}  nain  aw-i-non. \\
      \\
\glt
\z

kunai.grass  fire  that1  burn-Np-FU.3s

`The grass fire will burn.'

\ea%x412
\label{ex:x412}
\gll \textstyleEmphasizedVernacularWords{Mua}  \textstyleEmphasizedVernacularWords{takira}  unowa  ne  \textstyleEmphasizedVernacularWords{emeria  wii}\textstyleEmphasizedVernacularWords{p-}\textstyleEmphasizedVernacularWords{takira}\textstyleEmphasizedVernacularWords{=}\textstyleEmphasizedVernacularWords{ke} \\
      \\
\glt
\z

man  youth  many  ADD  woman  daughter-youth=CF  

me  unowa  akena.

not  many  very

`There are many young boys but not very many young girls.'

\ea%x1832
\label{ex:x1832}
\gll Epa  kokom-ar-eya  urera  \textstyleEmphasizedVernacularWords{siowa  mua}  ookinon. \\
      \\
\glt
\z

place  dark-INCH-2/3s.DS  afternoon  dog  man  follow-Np-FU.3s

`When it gets dark in the afternoon he will follow the ``dog man'' (a certain nominated person in the \textit{singsing} traditions).'

A place name may be a qualifier for a locative noun functioning as head noun. 

\ea%x834
\label{ex:x834}
\gll \textstyleEmphasizedVernacularWords{Bogia}  era \\
      \\
\glt
\z

Bogia  road

`the Bogia road'

\ea%x833
\label{ex:x833}
\gll \textstyleEmphasizedVernacularWords{Malala}  owowa \\
      \\
\glt
\z

Malala  village

`Malala village'

The qualifying \textstyleAcronymallcaps{NP} can also be a nominalized clause; this is most common when the head noun is an abstract noun like \textstyleStyleVernacularWordsItalic{sira} `custom' or \textstyleStyleVernacularWordsItalic{opora} `talk, story'.

\ea%x414
\label{ex:x414}
\gll [\textstyleEmphasizedVernacularWords{garanga}  \textstyleEmphasizedVernacularWords{oko  muuka  wiar  aaw-owa}]\textsubscript{NP}  sira \\
      \\
\glt
\z

family  other  son  3.DAT  get-NMZ  custom

`adoption custom (lit: the custom of getting a son from another family)'

The \textstyleEmphasizedWords{\textsc{head noun}} is either a single or a compound noun.  If the head noun is replaced by a pronoun, it can only take post-modifiers (\stepcounter{nx}{\thenx}):

\ea%x415
\label{ex:x415}
\gll \textstyleEmphasizedVernacularWords{wi(am)}  arow  nain  \\
      \\
\glt
\z

3p.UNM(REFL)  three  that1

`the three of them / those three'

A \textstyleEmphasizedWords{\textsc{post-modifying noun phrase}} often expresses qualities that in many European languages would be expressed by true adjectives (\stepcounter{nx}{\thenx}), or via adjectivalized (\stepcounter{nx}{\thenx}) or comitative expressions (\stepcounter{nx}{\thenx}).

\ea%x416
\label{ex:x416}
\gll labuel(a)  \textstyleEmphasizedVernacularWords{mua} \\
      \\
\glt
\z

pawpaw  man

`male pawpaw'

\ea%x417
\label{ex:x417}
\gll takira  \textstyleEmphasizedVernacularWords{emin(a)  kekanowa} \\
      \\
\glt
\z

boy  occiput  strong

`pig-headed boy'

\ea%x418
\label{ex:x418}
\gll mua  \textstyleEmphasizedVernacularWords{bug  maala}  nain  \\
      \\
\glt
\z

man  wind  long  that1

`the man with good lungs'

A noun phrase can have one or more \textstyleEmphasizedWords{\textsc{adjective phrases}} as modifiers. The adjective phrase typically consists of an adjective only. If there are more \textstyleAcronymallcaps{APs} than one, the order is as follows: colour - physical property or human propensity - size/age - value.

\ea%x419
\label{ex:x419}
\gll Waa(ya)  muuka  \textstyleEmphasizedVernacularWords{kia  gelemuta}  op-a-m. \\
      \\
\glt
\z

pig  son  white  small  catch-PA-1s

`I caught  a small white piglet.'

In recorded texts the maximum number of adjective phrases per a \textstyleAcronymallcaps{NP} is two, but the speakers have no difficulty producing \textstyleAcronymallcaps{NP}s with more \textstyleAcronymallcaps{AP}s (\stepcounter{nx}{\thenx}):

\ea%x420
\label{ex:x420}
\gll Emer(a)  \textstyleEmphasizedVernacularWords{itita  enum(a)  eliwa}  nain  enak-e. \\
      \\
\glt
\z

sago  soft  new  good  that1  feed.me-IMP.2s

`Give me the good new soft sago/bread to eat.'

The position of either a \textstyleEmphasizedWords{\textsc{quantifier phrase}} or an \textstyleEmphasizedWords{\textsc{indefinite}} is after the adjective phrase.

\ea%x804
\label{ex:x804}
\gll Siowa  morena  \textstyleEmphasizedVernacularWords{oko}  aruf-a-k. \\
      \\
\glt
\z

dog  male  another  hit-PA-3s

`He hit another male dog.'

The last regular post-modifier in a noun phrase is a \textstyleEmphasizedWords{\textsc{demonstrative}}. Especially the distal demonstrative \textstyleStyleVernacularWordsItalic{nain} `that' is very common, and in many cases it is no more than a marker for given information.

\ea%x805
\label{ex:x805}
\gll koora  erepam  \textstyleEmphasizedVernacularWords{nain} \\
      \\
\glt
\z

house  four  that1

`the/those four houses' or `the fourth house'

The \textstyleEmphasizedWords{\textsc{dative pronoun}} (\sectref{sec:3.5.5}) is unusual as a modifier. Semantically it belongs to the noun phrase, marking a possessive relationship, but syntactically it still reflects its origin as a [+human] locative adverbial (\sectref{sec:4.6.1}) of the verb. It is often non-contiguous with the rest of the \textstyleAcronymallcaps{NP}, which can be fronted for as a theme  while the dative pronoun needs to stay in its pre-verbal position (\stepcounter{nx}{\thenx}). Other elements that can separate the dative pronoun from the rest of the \textstyleAcronymallcaps{NP} are \textstyleStyleVernacularWordsItalic{me} `not' (\stepcounter{nx}{\thenx}), and the free adverbs \textstyleStyleVernacularWordsItalic{muutiw} `only' and \textstyleStyleVernacularWordsItalic{pun} `also'(\stepcounter{nx}{\thenx}). 

\ea%x1793
\label{ex:x1793}
\gll Owow  emeria  mua  unowa  \textstyleEmphasizedVernacularWords{sira  eliwa}  \textstyleEmphasizedVernacularWords{wiar}  uruf-ap  {\dots} \\
      \\
\glt
\z

village  woman  man  many  custom  good  3.DAT  see-SS.SEQ  ...

`The many villagers saw his good manners and {\dots}'

\ea%x1811
\label{ex:x1811}
\gll \textstyleEmphasizedVernacularWords{Pina}  \textstyleEmphasizedVernacularWords{gelemuta}  eliw  owowa=pa  \textstyleEmphasizedVernacularWords{nefar}  kaken-ami  \\
      \\
\glt
\z

guilt  small  well  village=LOC  2s.DAT  straighten-SS.SIM  

welaw-i-kuan.

finish-Np-FU.3p

`Your small guilt they can well straighten and finish in the village.'

\ea%x1812
\label{ex:x1812}
\gll \textstyleEmphasizedVernacularWords{Amina  fain}  me  \textstyleEmphasizedVernacularWords{wiar}  op-aka. \\
      \\
\glt
\z

pot  this  not  3.DAT  hold-IMP.2p

`Don't hold/touch these pots of hers.'

\ea%x1938
\label{ex:x1938}
\gll Yo  miira  me  uruf-a-m,  \textbf{afifa}  muutiw  \textbf{wiar}  uruf-a-m. \\
      \\
\glt
\z

1s.UNM  face  not  see-PA-1s  hair  only  3.DAT  see-PA-1s

`I didn't see the face, I only saw his hair.'

\subsubsection{Coordinate noun phrase}
\hypertarget{RefHeading21681935131865}{}
Joining noun phrases into a coordinate noun phrase can be done either by simple juxtaposition or with connectives. Juxtaposition is the default strategy.  In spoken texts the juxtaposed \textstyleAcronymallcaps{NP}s are separated by a longer pause, in written texts by a comma.

\ea%x810
\label{ex:x810}
\gll \textstyleEmphasizedVernacularWords{Amina},  \textstyleEmphasizedVernacularWords{wiowa},  \textstyleEmphasizedVernacularWords{eka  napia}  koor  miira=pa  iimar-ow-a-mik. \\
      \\
\glt
\z

pot  spear,  water  bamboo  house  face=LOC  stand-CAUS-PA-1/3p

`We placed the pots, spears and water bamboos in front of the house.'

\ea%x811
\label{ex:x811}
\gll I  \textstyleEmphasizedVernacularWords{mua  unowa},  \textstyleEmphasizedVernacularWords{emeria  papako}  ikiw-e-mik. \\
      \\
\glt
\z

1s.UNM  man  many  woman  some  go-PA-1/3p

`Many men (including the narrator) and some women went.'

Coordinate compound nouns (\sectref{sec:3.2.5}) are the result of conjoining by juxtaposition two nouns that very commonly go together. 

The pragmatic connective \textstyleStyleVernacularWordsItalic{ne} `additive' (\sectref{sec:3.11.1}) is used rather infrequently to connect the parts of a coordinate noun phrase. When it is used and there are more than two noun phrases to connect, it is usually placed between the last two noun phrases, but other positions are possible too, see (\stepcounter{nx}{\thenx}). 

\ea%x812
\label{ex:x812}
\gll Nie  \textstyleEmphasizedVernacularWords{ne}  neke  nomokow  fiira=ke. \\
      \\
\glt
\z

2s/p.maternal.uncle  ADD  2s/p.grandfather  tree  root=CF

`Your maternal uncle and your grandfather are the most important relatives.'

\ea%x814
\label{ex:x814}
\gll Mera  kas,  mulamul  \textstyleEmphasizedVernacularWords{ne}  popotimaw  aaw-i-mik. \\
      \\
\glt
\z

fish  mackerel  trevally.sp  ADD  trevally.sp  get-Np-PR.1/3p

`We catch mackerel, \textstyleForeignWords{mulamul}  trevally and \textstyleForeignWords{popotimaw}  trevally.'

A focus or case marking clitic is only added to the last noun phrase in a coordinate noun phrase:

\ea%x893
\label{ex:x893}
\gll Manin  koora  nain  \textstyleEmphasizedVernacularWords{koka  ne  ifara=ke}  wakesim-o-k. \\
      \\
\glt
\z

garden  house  that1  jungle  ADD  vine=CF  cover-PA-3s

`The garden house was covered by jungle and vines.'

\ea%x894
\label{ex:x894}
\gll \textstyleEmphasizedVernacularWords{Wiena  merena  ne  wapen=iw}  era  akup-a-mik. \\
      \\
\glt
\z

3p.GEN  foot  ADD  hand=INST  road  search-PA-1/3p

`They felt (lit: searched) for the road with their feet and hands.'

Also the pragmatic connective \textstyleStyleVernacularWordsItalic{aria} `alright' can occasionally join the elements of a coordinate noun phrase. As a sentential or clausal connective it indicates a break in the discourse, but when it joins two noun phrases there does not seem to be a significant difference between that and \textstyleStyleVernacularWordsItalic{ne} `additive'. It may be that \textstyleStyleVernacularWordsItalic{aria} draws more attention to the separate noun phrases being joined than either juxtaposition or \textstyleStyleVernacularWordsItalic{ne} does.

\ea%x815
\label{ex:x815}
\gll Yos,  yena  auwa,  \textstyleEmphasizedVernacularWords{aria}  wi  emer  en-ow(a)  mua  \\
      \\
\glt
\z

1s.FC  1s.GEN  1s/p.father  alright  3p.UNM  sago  eat-NMZ  man

kuisow  ikiw-e-mik.

one  go-PA-1/3p

`I, my father, and one Sepik man went.'

\ea%x816
\label{ex:x816}
\gll Moma,  \textstyleEmphasizedVernacularWords{aria}  emera  naap  lawisiw  eeyar-e-k. \\
      \\
\glt
\z

taro,  alright  sago  thus  rather  last-PA-3s

`Taro, and sago, lasted a little (longer).'

The disjunctive connective \textstyleStyleVernacularWordsItalic{e} `or' (\sectref{sec:3.11.2}), and/or the question marker -\textstyleStyleVernacularWordsItalic{i}  is used in a coordinate noun phrase, if the noun phrases are presented as alternatives. 

\ea%x817
\label{ex:x817}
\gll Mera  aaw-owa  sira  \textstyleEmphasizedVernacularWords{e}  era  ikur  okaiwi=pa  kuisow  \\
      \\
\glt
\z

fish  get-NMZ  custom  or  way  five  other.side=LOC  one  

ik-ua.

be-PA.3s

`There are six means, or ways, of catching fish.'

\ea%x818
\label{ex:x818}
\gll Maa  oposia\textstyleEmphasizedVernacularWords{=i } moma,  emera,  naap  sesek-a-mik. \\
      \\
\glt
\z

thing  meat=QM  taro  sago  thus  sell-PA-1/3p

`They sold meat, or taro, (or) sago, (things) like that.'

\subsubsection{Comitative noun phrase}
\hypertarget{RefHeading21701935131865}{}
A comitative noun phrase is made up of one or two basic noun phrases plus a comitative postposition or clitic (\sectref{sec:3.12.1}). A comitative pronoun (\sectref{sec:3.5.9}) either by itself or attached to a \textstyleAcronymallcaps{NP} can also form a comitative phrase (\stepcounter{nx}{\thenx}). When there is only one overt noun phrase and it  is unmarked for number, the plurality is shown both by the comitative marking and in the verb person marking (\stepcounter{nx}{\thenx}). The choice of the comitative marker and the number marking in the verb, when relevant,  reflect whether the noun phrases in the comitative relationship are co-subjects/co-objects of the same verb, or whether one is a dominant member. 

\ea%x828
\label{ex:x828}
\gll Ikoka  \textstyleEmphasizedVernacularWords{mua  owawiya}  irak-ep  me  efar  kerer-e. \\
      \\
\glt
\z

later  man  with  fight-SS.SEQ  not  1s.DAT  appear-IMP.2s

`Later when you fight with your husband, do not come to me.'

\ea%x829
\label{ex:x829}
\gll \textstyleEmphasizedVernacularWords{Wi  Yaapan  oos  onaiya}  Madang  ikiw-e-mik. \\
      \\
\glt
\z

3p.UNM  Japan  horse  with  Madang  go-PA-1/3p

`The Japanese went with horses to Madang.'

\ea%x830
\label{ex:x830}
\gll Parosifa  siisim-ep  \textstyleEmphasizedVernacularWords{muuka  feekiya}  sesek-i-nen. \\
      \\
\glt
\z

letter  write-SS.SEQ  son  with  send-Np-FU.1s

`I will write a letter and send it with my son.'

\ea%x819
\label{ex:x819}
\gll \textbf{Ona  siowa  ikos } manina  ikiw-e-mik, ... \\
      \\
\glt
\z

3s.GEN  dog  with  garden  go-PA-1/3p

`He went to the garden with his dog, {\dots}' or: `He and his dog went to the garden, {\dots}'

\ea%x832
\label{ex:x832}
\gll Rabaul  kemena=pa  naap  pok-ap  ik-e-mik,  \textstyleEmphasizedVernacularWords{mua=iya  emeria}. \\
      \\
\glt
\z

Rabaul  bay=LOC  thus  sit-SS.SEQ  be-PA-1/3p  man=COM  woman

`They are now sitting in the Rabaul bay, the husband and/with the wife.'

\ea%x831
\label{ex:x831}
\gll \textstyleEmphasizedVernacularWords{Wiamiya}  irak-owa  na-ep  ikiw-e-mik. \\
      \\
\glt
\z

3p.COM  fight-NMZ  say-SS.SEQ  go-PA-1/3p

`We went to fight with them.'

With the dual comitative postposition \textstyleStyleVernacularWordsItalic{ikos}  there may be an additive connective \textstyleStyleVernacularWordsItalic{ne} between the two noun phrases (\stepcounter{nx}{\thenx}). It seems to be more common with younger speakers. 

\ea%x820
\label{ex:x820}
\gll \textstyleEmphasizedVernacularWords{Osaiwa  ne  aalbok  ikos}  womar  \\
      \\
\glt
\z

bird.of.paradise  ADD  black.cuckoo.shrike  with  3s/p.friend  

wiam  op-a-mik.

3p.REFL  hold-PA-1/3p

`The bird of paradise and/with the black cuckoo-shrike were friends.'

\subsubsection{Appositional noun phrase}
\hypertarget{RefHeading21721935131865}{}
An appositional noun phrase consists of two noun phrases which have identical or similar reference \citep[24]{Crystal1997}.  Very commonly the first noun phrase is either a personal pronoun or a kinship term, the second one a proper name; but there are other possibilities as well.  

\ea%x835
\label{ex:x835}
\gll \textstyleEmphasizedVernacularWords{Yo  nena  nie=ke}  nefa  maak-i-yem. \\
      \\
\glt
\z

1s.UNM  2s.GEN  2s/p.uncle  2s.ACC  tell-Np-PR.1s

`I, your uncle, am telling you this.'

\ea%x836
\label{ex:x836}
\gll \textstyleEmphasizedVernacularWords{Yena  yaiya  Tup}  ifa=ke  keraw-a-k. \\
      \\
\glt
\z

1s.GEN  1s/p.uncle  Tup  snake=CF  bite-PA-3s

`My Uncle Tup was bitten by a snake.'

\ea%x837
\label{ex:x837}
\gll \textstyleEmphasizedVernacularWords{Inasina}  \textstyleEmphasizedVernacularWords{Rubaruba  nain=ke}  ona  emeria  aaw-ep  \\
      \\
\glt
\z

spirit  Rubaruba  that1=CF  3s.GEN  woman  take-SS.SEQ  

p-ikiw-o-k.

Bpx-go-PA-3s

`The spirit Rubaruba took his wife and went.'

\ea%x838
\label{ex:x838}
\gll \textstyleEmphasizedVernacularWords{Manina}  \textstyleEmphasizedVernacularWords{gelemuta,  esewa,}  nena  kookal-owa=pa  \\
      \\
\glt
\z

garden  small  esewa  2s.GEN  like-NMZ=LOC  

perek-i-nan.

pull.out-Np-FU.2s

`You may harvest the little garden, ``esewa'', at your desire.'

\ea%x839
\label{ex:x839}
\gll Wokome=ke  \textstyleEmphasizedVernacularWords{wiimasip  oko,  suwina  gelemuta \\
      \\
\glt
\z

3s/p.grandmother=CF  3s/p.grandchild  other  female  small  

\textstyleEmphasizedVernacularWords{nain}  maak-e-k  {\dots}

that  tell-PA-3s

`The grandmother told her other grandchild, the little girl {\dots}'

\subsection{Adjective phrase}
\hypertarget{RefHeading21741935131865}{}
The head of an adjective phrase (\textstyleAcronymallcaps{AP}) is an adjective. Most commonly it occurs alone, but it can be intensified by an intensity adverb either preceding or following it, or both (\stepcounter{nx}{\thenx}).  The negator \textstyleStyleVernacularWordsItalic{marew} `none, no' when following the adjective, negates its quality, thus creating its opposite (\stepcounter{nx}{\thenx}).  

\ea%x841
\label{ex:x841}
\gll Owowa  nain  \textstyleEmphasizedVernacularWords{lawiliw  manek(a)  akena}. \\
      \\
\glt
\z

village  that1  rather  big  very

`The village is rather big'

\ea%x842
\label{ex:x842}
\gll Koora  \textstyleEmphasizedVernacularWords{eliw(a)  marew}  nan  ik-e-mik. \\
      \\
\glt
\z

house  good  none  there  be-PA-1/3p

`They live in the bad (lit: no-good) house.'

When the adjective \textstyleStyleVernacularWordsItalic{masia}  `bitter' takes a nominalized verb as its modifier, the meaning of the adjective changes to indicate that one is doing a lot of some action (\stepcounter{nx}{\thenx}).

\ea%x840
\label{ex:x840}
\gll Mua  \textstyleEmphasizedVernacularWords{manin(a)  mauw-ow(a)  masia}  nain  emeria  \\
      \\
\glt
\z

man  garden  work-NMZ  compulsive  that1  woman  

wi-i-mik.

give.him-Np-PR.1/3p

`We give a wife to a hard-working man.'

The adjective phrase functions as a post-modifier in a noun phrase (\stepcounter{nx}{\thenx}), or as a non-verbal predicate (\stepcounter{nx}{\thenx}). 

A coordinate adjective phrase is also possible:

\ea%x891
\label{ex:x891}
\gll Oka  keraw-a-k  nain  \textstyleEmphasizedVernacularWords{efefa  ne  eliwa  akena}. \\
      \\
\glt
\z

hand.drum  carve-PA-3s  that1  light  ADD  good  very  

`The hand drum that he carved is light and very good.'

The pragmatic function of adjectives in discourse\footnote{For \textstyleFootnoteBaseChar{the function of adjectives in English and Mandarin Chinese spoken text see \citet{Thompson1988} and \citet{Croft1991}. The former claims the main function is to predicate the property of an established discourse referent; attributive function, or modification, is secondary and used almost exclusively for new participants. But Croft considers modification the main discourse function of the adjectives.  As for Papuan languages, Roberts reports that in Amele the adjective normally functions as a modifying (lit: attributive) element in a NP (1987:319).}} seems to vary according to the language. In Mauwake the modification of a new participant is the main function of adjective phrases.\footnote{In the text data nearly half of the occurrences (48\%) of adjectives were in attributive positions where the adjective modified a \textit{new} participant.} Also a known participant is modified by an adjective especially in cases where the adjective is needed for contrast: \textstyleStyleVernacularWordsItalic{manin}(\textstyleStyleVernacularWordsItalic{a}) \textstyleStyleVernacularWordsItalic{maneka} `big garden' and \textstyleStyleVernacularWordsItalic{manin}(\textstyleStyleVernacularWordsItalic{a}) \textstyleStyleVernacularWordsItalic{gelemuta}\textstyleEmphasizedWords{} `small garden', referring to two different \textstyleEmphasizedWords{\textsc{types}} of garden (also called \textstyleStyleVernacularWordsItalic{ekina} and \textstyleStyleVernacularWordsItalic{esewa} respectively), were repeated several times in a text describing garden work. 

\subsection{Quantifier phrase}
\hypertarget{RefHeading21761935131865}{}
A quantifier phrase usually only consists of a quantifier head (\stepcounter{nx}{\thenx}) (\sectref{sec:3.4}), but it can be modified by a few intensity adverbs (\stepcounter{nx}{\thenx}) (\sectref{sec:3.9.2}). 

\ea%x845
\label{ex:x845}
\gll I  koora  \textstyleEmphasizedVernacularWords{kuisow}  yiar  aw-o-k. \\
      \\
\glt
\z

1p.UNM  house  one  1p.DAT  burn-PA-3s

`One of our houses burned.'

\ea%x844
\label{ex:x844}
\gll Koora  \textstyleEmphasizedVernacularWords{arow  akena}  ku-a-mik. \\
      \\
\glt
\z

house  three  truly  build-PA-1/3p

`We built exactly three houses.'

A quantifier phrase most commonly functions as a post-modifier in a noun phrase (\stepcounter{nx}{\thenx}), but it can also be used as a non-verbal predicate (\stepcounter{nx}{\thenx}).

\ea%x846
\label{ex:x846}
\gll Maamuma  \textstyleEmphasizedVernacularWords{unowa  akena}  aaw-e-mik. \\
      \\
\glt
\z

money  much  truly/very  get-PA-1/3p

`They got very much money.'

\ea%x843
\label{ex:x843}
\gll Yo  muuka  \textstyleEmphasizedVernacularWords{arow}. \\
      \\
\glt
\z

1s.UNM  son  three.

`I have three sons.'  (lit: `My sons are three.)

Quantifier phrases may also be coordinated. Semantically the most plausible coordination is disjunction: 

\ea%x1360
\label{ex:x1360}
\gll Waaya  maneka  wiowa  \textstyleEmphasizedVernacularWords{erup-i  e  arow}  naap  mik-iwkin \\
      \\
\glt
\z

pig  big  spear  two=QM  or  three  thus  spear-2/3p.DS  

um-i-ya.

die-Np-PR.3s

`They spear a big pig with two or three spears and it dies.'

\subsection{Possessive phrase}
\hypertarget{RefHeading21781935131865}{}
The possessive phrase\footnote{Not to be confused with the Possessive NP.} is a very specific and rarely occurring structure. It consists of an unmarked or genitive pronoun, followed by the long form of the dative pronoun (\sectref{sec:3.5.5}), which has developed from the dative pronoun and the the verb \textstyleStyleVernacularWordsItalic{ik}- `be'. The verb has lost all inflection and only retains the root, which has merged to the dative pronoun. The possessive phrase only functions as a non-verbal predicate. It is always without a head noun; a co-referential noun or pronoun is in an earlier \textstyleAcronymallcaps{NP} in the same clause.

\ea%x847
\label{ex:x847}
\gll Auwa  maa  unowa  nain  pun  \textstyleEmphasizedVernacularWords{yo/yena  efarik}. \\
      \\
\glt
\z

father  thing  many  that1  also  1s.UNM/1s.GEN  1s.DAT

`My father's possessions, too, are mine.'

\subsection{Verb phrase}
\hypertarget{RefHeading21801935131865}{}
There is no justification in Mauwake for a verb phrase as it is understood in the generative sense, as a constituent including almost everything else than the subject of the sentence.\footnote{The verb phrases in the traditional sense of the word, a group of verbs functioning as one unit, are treated under verbal clusters (\sectref{sec:3.8.5}).}  But there is one structure that can be called a verb phrase: an accusative pronoun plus a verb.  In this structure nothing can come between the two elements, not even a verbal negation, which is usually placed immediately before the verb.

Every transitive verb requires an accusative pronoun for a [+human] object, regardless of whether there is an object noun phrase or not. The accusative pronoun is also required with a plural beneficiary. 

\ea%x848
\label{ex:x848}
\gll Nan  wi  owow  mua  \textstyleEmphasizedVernacularWords{wia  maak-e-mik},  {\dots} \\
      \\
\glt
\z

there  3p.UNM  village  man  3p.ACC  tell-PA-1/3p

`There they told the village men, ... '

\ea%x852
\label{ex:x852}
\gll \textstyleEmphasizedVernacularWords{Nefa  war-iwkin}  naap  ma-e. \\
      \\
\glt
\z

2s.ACC  shoot-2/3p.DS  thus  say-IMP.2s

`When they shoot you, say like that.'

\ea%x849
\label{ex:x849}
\gll Mua  me  \textstyleEmphasizedVernacularWords{wia  imen-a-mik}. \\
      \\
\glt
\z

man  not  3p.ACC  find-PA-1/3p

`We didn't find the men.

\ea%x850
\label{ex:x850}
\gll Yaapan=ke  i  emeria  \textstyleEmphasizedVernacularWords{yia  aaw-urum-i-kuan}. \\
      \\
\glt
\z

Japan=CF  1p.UNM  woman  1p.ACC  take-DISTR/A-Np-FU.3p

`Japan will take all of us women.'

\ea%x851
\label{ex:x851}
\gll Takira  enow  gelemuta  \textstyleEmphasizedVernacularWords{wia  on-om-a-mik}. \\
      \\
\glt
\z

child  meal  small  3p.ACC  make-BEN-BNFY2-PA-1/3p

`We made a feast for the children.

\subsection{Adverbial phrases}
\hypertarget{RefHeading21821935131865}{}
An adverbial phrase may consist of an adverb word alone or modified by an intensity adverb, a noun phrase plus a clitic or a postposition, or a dative pronoun functioning as a [+human] locative phrase.

The main function of an adverbial phrase is to modify the verb. An \textstyleAcronymallcaps{AdvP} is an optional constituent in a clause, not an obligatory argument. 

The default position of the adverbial phrase depends on the semantic type of the \textstyleAcronymallcaps{AdvP}. Recursion is possible, and is more common in the case of locative and temporal phrases than the others.

\subsubsection{Locative phrases}
\hypertarget{RefHeading21841935131865}{}
The number of locative adverbs is small (\sectref{sec:3.6.3}). Most locative phrases are made up of a noun phrase plus a clitic if they indicate a location, source or path, and of a noun phrase only if they indicate a goal. Giv\'on (1984:78, 110-112) distinguishes between the locative adverbials and the locative objects of certain verbs. The former have the whole clause in their scope, the latter only the verb. While there is this scope difference between the two, in Mauwake they are syntactically similar. 

The locative adverbs (\sectref{sec:3.6.3}), all of which are deictic, occur by themselves, without modifiers. The same form can be used for location, source, or goal, depending on the verb.

\ea%x870
\label{ex:x870}
\gll Miiw-aasa  \textstyleEmphasizedVernacularWords{nan}  ik-eya  mua  nain  nabena  suuw-a-mik. \\
      \\
\glt
\z

land-canoe  there  be-2/3s.DS  man  that1  carrying.pole  push-PA-1/3p

`The car stayed there, and they carried the man on their shoulders.'

\ea%x871
\label{ex:x871}
\gll Fura  op-ap  ik-o-n  nain  \textstyleEmphasizedVernacularWords{feeke}  wu-e. \\
      \\
\glt
\z

knife  hold-SS.SEQ  be-PA-2s  that1  here.CF  put-IMP.2s

`Put here the knife that you are holding.'

\ea%x1833
\label{ex:x1833}
\gll Manin(a)  onoma  maa  en-owa  \textstyleEmphasizedVernacularWords{nan}  aaw-i-ya. \\
      \\
\glt
\z

garden  basis  thing  eat-NMZ  there  get-Np-PR.3s

`An owner of a garden (lit: the garden basis) gets his food from there.'

When the locative phrase is based on a noun phrase, one form is used both for  a location where something takes place and a source, but a goal is marked differently.  The phrases indicating a location are formed by adding the locative clitic \nobreakdash-\textstyleStyleVernacularWordsItalic{pa}  to a noun phrase:

\ea%x856
\label{ex:x856}
\gll Pon  \textstyleEmphasizedVernacularWords{sisina=pa } ik-eya  mik-a-m. \\
      \\
\glt
\z

turtle  shallow.water=LOC  be-2/3s.DS  spear-PA-1s

`The turtle was in shallow water and I speared it.'

\ea%x857
\label{ex:x857}
\gll \textstyleEmphasizedVernacularWords{Sapara=pa  nan}  suusa  iw-e-mik. \\
      \\
\glt
\z

Sapara=LOC  there  needle  give.him-PA-1/3p

`There in Sapara he was given an injection.'

\ea%x865
\label{ex:x865}
\gll Nomokowa  unowa  serer-iw-ap  \textstyleEmphasizedVernacularWords{Takora=pa  nan} \\
      \\
\glt
\z

tree  many  hang-go-SS.SEQ  Takora=LOC  there  

or-o-mik.

descend-PA-1/3p

`They went hanging to many trees, and at Takora they got down.'

Source is also marked as a location, with the clitic -\textstyleStyleVernacularWordsItalic{pa}. In some cases there is possible ambiguity as to the interpretation, but the context usually provides a clue.

\ea%x858
\label{ex:x858}
\gll Parosifa  siisim-ep  \textstyleEmphasizedVernacularWords{iinan  aasa=pa}  wafur-a-mik. \\
      \\
\glt
\z

paper  write-SS.SEQ  sky  canoe=LOC  throw-PA-1/3p

`They wrote papers and threw them from airplanes.'

\ea%x859
\label{ex:x859}
\gll Aite=ke  \textstyleEmphasizedVernacularWords{manina=pa}  yia  aaw-om-iwkin  enim-i-mik. \\
      \\
\glt
\z

1s/p.mother=CF  garden=LOC  1p.ACC  get-BEN-1s/p.DS  eat-Np-PR.1/3p

`Our mothers get (it) from the garden for us and we eat (it).'

\ea%x864
\label{ex:x864}
\gll Me  fan  \textstyleEmphasizedVernacularWords{Madang  kame=pa}  ekap-e-mik. \\
      \\
\glt
\z

not  here  Madang  side=LOC  come-PA-1/3p

`They didn't come here from the Madang side.'

The noun phrase indicating a goal normally does not take the locative clitic or any other marking.  The directional verbs are the most common ones used with goal, but other verbs of motion can be used as well (\stepcounter{nx}{\thenx}), (\stepcounter{nx}{\thenx}). 

\ea%x860
\label{ex:x860}
\gll Ae,  o  \textstyleEmphasizedVernacularWords{fiker  gone}  urup-o-k. \\
      \\
\glt
\z

yes  3s.UNM  kunai.grass  middle  ascend-PA-3s

`Yes, he went up to the middle of the \textstyleForeignWords{kunai} grass area.'

\ea%x861
\label{ex:x861}
\gll [Manina=pa  nan]\textsubscript{Source}  [\textstyleEmphasizedVernacularWords{koka}]\textsubscript{Goal}  iw-a-mik. \\
      \\
\glt
\z

garden=LOC  there  jungle  go-PA-1/3p

`From the garden there the day they went into the jungle.'

\ea%x862
\label{ex:x862}
\gll \textstyleEmphasizedVernacularWords{Medebur}  karu-eka,  baurar-eka. \\
      \\
\glt
\z

Medebur  run-IMP.2p  flee-IMP.2p

`Run to Medebur, flee!'

\ea%x863
\label{ex:x863}
\gll \textstyleEmphasizedVernacularWords{Ulingan  nan}  bom  fu-fuurk-ikiw-e-mik. \\
      \\
\glt
\z

Ulingan  there  bomb  RDP-throw-go-PA-1/3p

`They went throwing bombs to Ulingan.'

It is possible to mark the goal with the locative clitic if the goal is mainly important as the location of the following verb. The frequency of this usage for the clitic is low. Example (\stepcounter{nx}{\thenx}) is repeated below as (\stepcounter{nx}{\thenx}): 

\ea%x1884
\label{ex:x1884}
\gll Ne  soran-emi  \textstyleEmphasizedVernacularWords{epia  mukuna=pa} \\
      \\
\glt
\z

ADD  get.startled  firewood  fire=LOC

or-omi  aw-o-k.

descend-SS.SIM  burn-PA-3s

`And he got startled and fell on the fire and burned himself.'

When the locative phrase is [+human], the dative pronoun (\sectref{sec:3.5.5}) must be used:

\ea%x1061
\label{ex:x1061}
\gll Mua  oko=ke  waaya  nain  mik-ap  \textstyleEmphasizedVernacularWords{nefar } aaw-i-non. \\
      \\
\glt
\z

man  other=CF  pig  that1  spear-SS.SEQ  2s.DAT  take-Np-FU.3s

`Another man will spear the pig and take it from you.'

\ea%x1939
\label{ex:x1939}
\gll Feeke  \textstyleEmphasizedVernacularWords{wiar}  ik-ok  kiiriw  mua  \textstyleEmphasizedVernacularWords{wiar}  urup-e. \\
      \\
\glt
\z

here.CF  3.DAT  be-SS  again  man  3.DAT  ascend-IMP.2s

`Stay here with him and (then) go (back) to your husband again.'

The dative pronoun is also commonly added when the location is a village or a larger area, which is seen mainly as a setting for the people. In both (\stepcounter{nx}{\thenx}) and (\stepcounter{nx}{\thenx}) above it is the \textstyleEmphasizedWords{\textsc{location}} which is in focus, in the former as the closest village to flee to, and in the latter as an object of bombing, so the dative pronoun is not used. In (\stepcounter{nx}{\thenx}) a certain culturally important place referred to is in the area of the Koran people and considered their property:

\ea%x1801
\label{ex:x1801}
\gll Koran  epa=pa  \textstyleEmphasizedVernacularWords{wiar}  ik-ua. \\
      \\
\glt
\z

Koran  place=LOC  3.DAT  be-PA.3s

`It is in Koran area.'

The noun phrase indicating a path is marked with the instrumental clitic -\textstyleStyleVernacularWordsItalic{iw}, or occasionally with the locative clitic -\textstyleStyleVernacularWordsItalic{pa} (\stepcounter{nx}{\thenx}).

\ea%x866
\label{ex:x866}
\gll \textstyleEmphasizedVernacularWords{Iinan}\textstyleEmphasizedVernacularWords{=iw  iinan=iw}  wu-ami  feenap  {\dots  ikiw-o-k.} \\
      \\
\glt
\z

on.top=INST  on.top=INST  putSS.SIM  like.this  {\dots}  go-PA-3s

`They (airplanes) flew (lit: put) high up, high up, and went like this{\dots}'

\ea%x867
\label{ex:x867}
\gll \textstyleEmphasizedVernacularWords{Saa}\textstyleEmphasizedVernacularWords{=iw}  ir-am-ika-i-mik,  oos  ono-onaiya. \\
      \\
\glt
\z

sand=INST  ascend-SS.SIM-be-Np-PR.1/3p,  horse  RDP-with

`They are coming along the beach, with horses.'

When a clause has more locative phrases than one, the following rules apply. If  the phrases have a different function, source is placed before goal (\stepcounter{nx}{\thenx}). When they have the same function and a deictic locative adverb strengthens another locative phrase, the adverb follows the other locative (\stepcounter{nx}{\thenx}), (\stepcounter{nx}{\thenx}), (\stepcounter{nx}{\thenx}). But  the dative pronoun, when used locatively, has to be placed even after the locative adverb (\stepcounter{nx}{\thenx}). When both of the phrases have an independent meaning, the phrase indicating the more general location comes first, and the one marking the more specific location follows.

\ea%x868
\label{ex:x868}
\gll Mia  aka  nain  aaw-ep  p-ikiw-ep  \textstyleEmphasizedVernacularWords{manina=pa \\
      \\
\glt
\z

body  blood  that1  take-SS.SEQ  Bpx-go-SS.SEQ  garden=LOC  

\textstyleEmphasizedVernacularWords{upuna}\textstyleEmphasizedVernacularWords{=pa}  wu-a-k.

row=LOC  put-PA-3s

`She took the menstrual blood with her and put it in a (plant) row in a garden.'

\ea%x869
\label{ex:x869}
\gll Ikiw-ep  \textstyleEmphasizedVernacularWords{eeneke  wiena  owowa=pa}  uruf-a-mik,  {\dots \\
      \\
\glt
\z

go-SS.SEQ  there  3p.GEN  village=LOC  see-PA-1/3p

`They went and there, in their village, they saw, {\dots}'

\subsubsection{Temporal phrase}
\hypertarget{RefHeading21861935131865}{}
Temporals mark location in time, so it is natural that temporal phrases behave very similarly to locative phrases.  They can consist of a temporal adverb (\sectref{sec:3.9.1.2}), possibly modified by an intensity adverb (\sectref{sec:3.9.2}); or of a noun phrase (\sectref{sec:4.1}) with a head noun indicating time, plus a locative clitic (\sectref{sec:3.12.4}).

\ea%x872
\label{ex:x872}
\gll \textstyleEmphasizedVernacularWords{Uuriw  akena}  mukuna  nain  kerer-e-k. \\
      \\
\glt
\z

morning  truly/very  fire  that1  appear-PA-3s

`The fire started early in the morning.'

\ea%x873
\label{ex:x873}
\gll Ne  \textstyleEmphasizedVernacularWords{fraide=pa}  maapora  puk-o-k,  \textstyleEmphasizedVernacularWords{urera}. \\
      \\
\glt
\z

ADD  Friday=LOC  feast  burst-PA-3s  afternoon

`And on Friday the feast started, in the afternoon.'

Recursion of temporal phrases is possible and quite common.  When there are two or more temporal phrases in the same clause, the order is determined by whether the temporals are deictic and/or specific (\sectref{sec:3.9.1.2}).  Their default order relative to each other is as follows:

(non-deictic non-specific) {\textgreater} deictic non-specific {\textgreater} deictic specific {\textgreater} TempNP (day) {\textgreater} non-deictic specific {\textgreater} TempNP (time of day) {\textgreater} (non-deictic non-specific)

The position of the non-deictic non-specific temporal is either as the first or the last element of the group of temporals.

\ea%x874
\label{ex:x874}
\gll Ne  \textstyleEmphasizedVernacularWords{nainiw  sande  uura}  yiam  fiirim-e-mik. \\
      \\
\glt
\z

ADD  again  Sunday  night  1p.REFL  gather-PA-1/3p

`And we gathered again on Sunday night.'

\ea%x875
\label{ex:x875}
\gll \textstyleEmphasizedVernacularWords{Uurika}  \textstyleEmphasizedVernacularWords{mande  uuriw}  amia  mua  feeke  kerer-i-non. \\
      \\
\glt
\z

tomorrow  Monday  morning  bow  man  here.CF  appear-Np-FU.3s

`Tomorrow Monday a policeman will come here in the morning.'

\ea%x877
\label{ex:x877}
\gll \textstyleEmphasizedVernacularWords{Unan  urera  ama  ikur  naap}  on-a-mik. \\
      \\
\glt
\z

yesterday  afternoon  sun  five  thus  do-PA-1/3p

`We did it yesterday afternoon around five o'clock.'

\ea%x876
\label{ex:x876}
\gll \textstyleEmphasizedVernacularWords{Ikoka  trinde}\textstyleEmphasizedVernacularWords{=pa  nainiw } aakun-i-yen. \\
      \\
\glt
\z

later  Wednesday=LOC  again  talk-Np-FU.1p

`We'll talk again later on Wednesday.'

When a noun phrase acts as a temporal phrase, the locative clitic -\textstyleStyleVernacularWordsItalic{pa} is attached to it (\stepcounter{nx}{\thenx}), unless it is followed by another temporal phrase specifying it further or it includes a demonstrative (\stepcounter{nx}{\thenx}). If there are several of these temporal noun phrases, their relative order is from the larger time unit to the smaller one. 

\ea%x878
\label{ex:x878}
\gll \textstyleEmphasizedVernacularWords{Mokoma  fain  siiwa  Mas}\textstyleEmphasizedVernacularWords{=pa}  weeser-i-non. \\
      \\
\glt
\z

year  this  month  March=LOC  finish-Np-FU.3s

`It will finish in March this year.' 

A temporal phrase may be formed with the instrumental clitic -\textstyleStyleVernacularWordsItalic{iw} (\sectref{sec:3.12.5}) when something takes place repeatedly at the same time of the day. The example (\stepcounter{nx}{\thenx}) is here repeated as (\stepcounter{nx}{\thenx}):

\ea%x1907
\label{ex:x1907}
\gll I  amirik=\textstyleEmphasizedVernacularWords{iw}  ...  Gawar  wiar  ikiw-e-mik. \\
      \\
\glt
\z

1p.UNM  daytime=INST  {\dots}  Gawar  3.DAT  go-PA-1/3p

`In the daytime we (always) went to Gawar {\dots}'

\subsubsection{Manner phrase}
\hypertarget{RefHeading21881935131865}{}
An adverbial phrase indicating manner most often consists of just a manner adverb (\sectref{sec:3.9.1.3}).  That is occasionally intensified by an intensity adverb (\sectref{sec:3.9.2}). 

\ea%x880
\label{ex:x880}
\gll Iwera  nainiw  \textstyleEmphasizedVernacularWords{kaken}  iimar-e-k. \\
      \\
\glt
\z

coconut  again  straight  stand.up-PA-3s

`The coconut palm stood up straight again.'

\ea%x879
\label{ex:x879}
\gll Koran  wiena  \textstyleEmphasizedVernacularWords{balisow  akena}  epa  nain  \\
      \\
\glt
\z

Koran  3p.GEN  well  truly/very  place  that1  

amis-ar-e-mik.

knowledge-INCH-PA-1/3p

`The Koran people themselves know that place very well.'

\ea%x881
\label{ex:x881}
\gll O  \textstyleEmphasizedVernacularWords{iiwawun  samor}  aaw-o-k.  \\
      \\
\glt
\z

3p.UNM  altogether  badly  get-PA-3s

`He got it really bad (= he got into a very bad condition).'

A manner phrase can also be formed by a noun phrase plus a clitic, instrumental \nobreakdash-\textstyleStyleVernacularWordsItalic{iw} or, less frequently, with locative -\textstyleStyleVernacularWordsItalic{pa}.  

\ea%x882
\label{ex:x882}
\gll Siowa  wiawi=ke  siowa  aluowa  miim-ap  \textstyleEmphasizedVernacularWords{karu-(o)w(a)=iw} \\
      \\
\glt
\z

dog  3s/p.father  dog  noise  hear-SS.SEQ  run-NMZ=INST  

ekap-o-k.

come-PA-3s

`The dog's master heard its noise and came running.'

\ea%x884
\label{ex:x884}
\gll \textstyleEmphasizedVernacularWords{Yiena  kae  sira=pa}  mauw-owa  ik-ua. \\
      \\
\glt
\z

1p.GEN  grandfather  custom=LOC  work-NMZ  be-PA.3s

`We have to work according to the custom of our grandfathers.'

If there are more manner phrases than one, one of them is usually deictic \textstyleStyleVernacularWordsItalic{naap} `thus, like that' or \textstyleStyleVernacularWordsItalic{feenap} `like this' either preceding or following the other manner phrase(s).

\ea%x883
\label{ex:x883}
\gll Wi  Yaapan  \textstyleEmphasizedVernacularWords{naap  kuisow=iw}  ekap-em-ik-e-mik. \\
      \\
\glt
\z

3p.UNM  Japan  thus  one=INST  come-SS.SIM-be-PA-1/3p

`The Japanese came like that one by one.'

When comparison is indicated in the manner phrase, the postposition \textstyleStyleVernacularWordsItalic{saarik} `like, as' follows the noun phrase.

\ea%x885
\label{ex:x885}
\gll Wie,  wiawi  nain  \textstyleEmphasizedVernacularWords{ifa  saarik} \\
      \\
\glt
\z

3s/p.uncle  3s/p.father  that1  snake  like  

in-urum-ep-ik-e-mik.

sleep-DISTR/A-SS.SEQ-be-PA-1/3s

`His uncles and fathers were all sleeping like snakes.'

One type of a manner phrase is one that indicates instrument.  It is always formed with a noun phrase plus one of three clitics: instrumental -\textstyleStyleVernacularWordsItalic{iw}\textstyleStyleVernacularWordsItalic{} (\sectref{sec:3.12.5}), locative -\textstyleStyleVernacularWordsItalic{pa}  (\sectref{sec:3.12.4})\textstyleStyleVernacularWordsItalic{} or comitative -\textstyleStyleVernacularWordsItalic{iya}\textstyleStyleVernacularWordsItalic{} (\sectref{sec:3.12.1}). The instrumental clitic is the most common. 

\ea%x886
\label{ex:x886}
\gll Ifa  mia  nain  \textstyleEmphasizedVernacularWords{fura=iw}  lalat-em-ik-om-a-mik. \\
      \\
\glt
\z

snake  skin  that1  knife=INST  sweep-SS.SIM-be-BEN-BNFY2.PA-1/3p

`They kept scraping the snake skin off her with a knife.'

\ea%x889
\label{ex:x889}
\gll ...\textstyleEmphasizedVernacularWords{wiena  opaimik=iw}  yia  maak-em-ik-e-mik. \\
      \\
\glt
\z

{\dots}3p.GEN  mouth/language=INST  1p.ACC  tell-SS.SIM-be-PA-1/3p

`{\dots}they kept telling us in their language.'

When a coordinate noun phrase is made into an instrumental manner phrase, the instrumental clitic only follows the last noun phrase.

\ea%x892
\label{ex:x892}
\gll \textstyleEmphasizedVernacularWords{Wiena merena ne wapen=iw} era akup-amik. \\
      \\
\glt
\z

3p.GEN foot ADD hand=INST way search-PA-1/3p

`With their feet and hands they felt (lit: searched) for the road.'

The use of locative clitic is restricted almost exclusively to those cases where the instrument is a vehicle (\stepcounter{nx}{\thenx}), so they could also be understood as locatives. In other cases it is used rarely (\stepcounter{nx}{\thenx}).

\ea%x887
\label{ex:x887}
\gll Yo  iiriw  \textstyleEmphasizedVernacularWords{iinan  aasa=pa}  karu-owa  erup  ar-ep  \\
      \\
\glt
\z

1s.UNM  earlier  sky  canoe=LOC  run-NMZ  two  become-SS.SEQ

me  keker  op-a-m.

not  fear  hold-PA-1s

`I had already travelled by plane twice, and was not afraid.'

\ea%x888
\label{ex:x888}
\gll \textstyleEmphasizedVernacularWords{Sureka}\textstyleEmphasizedVernacularWords{=pa}  owora  nain  teek-ap  aaw-e-mik. \\
      \\
\glt
\z

harvesting.stick=LOC  betelnut  that1  pluck-SS.SEQ  get-PA-1/3p

`They picked the betelnuts with a harvesting stick.'

The comitative clitic is also possible but infrequent in instrumental manner phrases. Its use in this function may be influenced by Tok Pisin, where \textstyleForeignWords{wantaim} `together (with)' is used not only for accompaniment, but for instruments as well.

\ea%x890
\label{ex:x890}
\gll Mauwa  ar-e-n,  \textstyleEmphasizedVernacularWords{amia=iya}  nenar-e-mik=i? \\
      \\
\glt
\z

what  become-PA-2s  bow/gun=COM  shoot.you-PA-1/3p=QM

`What happened to you, did they shoot you with a gun?'

In Mauwake texts manner phrases are much less frequent than either locative or temporal phrases.

\section{Clause}
\hypertarget{RefHeading21901935131865}{}
A clause,\footnote{I use the separate terms \textit{clause} and \textit{sentence} to avoid confusion.  A simple sentence consists of just one clause, but most of the sentences in Mauwake have more than one clause in either coordinate, chaining or subordinate relationship.} or simple sentence, typically expresses one predication and is a minimal utterance that can stand alone.

In Mauwake the predicate is the only obligatory element in those clauses that have a verbal predicate. Verbless clauses need to have both an overt subject and a predicate. The different clause types are discussed in {\S}\sectref{sec:5.3}-5.6.

Instead of the common two-way distinction between main and subordinate clauses, in Trans New Guinea languages it is practical to talk about main, medial and subordinate clauses. Main clauses have a finite verb, and most commonly it is the last element in a sentence. Medial clauses (\sectref{sec:8.2.1}) are coordinate with the main clauses but dependent on them, and the verbs are in medial form (\sectref{sec:3.8.3.4}). The default position for a medial clause is non-final, but for pragmatic purposes it may be postposed to follow the main clause.  Also a subordinate clause (\sectref{sec:8.3}) usually precedes the main clause. 

\subsection{Order of constituents}
\hypertarget{RefHeading21921935131865}{}
Two seemingly conflicting statements about the clausal constituent order in Papuan languages have been given by \citet{Wurm1982} and \citet{Foley1986}. \citet[64]{Wurm1982} maintains that they have a rigid \textstyleAcronymallcaps{SOV} order; \citet[167]{Foley1986} claims that the order in most Papuan languages is relatively free, and therefore he prefers to call them just verb-final (ibid. 10). But it seems that the two linguists are talking about somewhat different things, and both of them are correct in what they say.  The \textstyleEmphasizedWords{\textsc{default}} constituent order in neutral sentences is \textstyleAcronymallcaps{SOV}, as Wurm claims, but Foley is right in that the \textstyleEmphasizedWords{\textsc{interpretation}}\textstyleEmphasizedWords{} of the arguments of a verb as subject or object does not rely heavily on the constituent order. Especially in languages with extensive verb morphology marking the syntactic roles on the verb itself the order of the nominals can be relatively free, and is mainly constrained by pragmatic factors. 

The basic constituent order in Mauwake clauses is quite rigid \textstyleAcronymallcaps{SOV}, even if the verb morphology cross-references the syntactic roles to some extent. Although only a fraction of the clauses in the text corpus -- less than 10\% -- have an overt subject and object \textstyleAcronymallcaps{NP,} it is possible to establish the dominant order. About nine out of ten of those clauses that do have an overt subject and object \textstyleAcronymallcaps{NP} manifest \textstyleAcronymallcaps{SOV} order.\footnote{SOV: 210 clauses, OSV: 22 clauses} They are also pragmatically neutral (\stepcounter{nx}{\thenx}), whereas the other possible order, \textstyleAcronymallcaps{OSV}, only occurs when the object is fronted as a theme (\stepcounter{nx}{\thenx}). 

\ea%x896
\label{ex:x896}
\gll [Owow  mua]\textsubscript{S}  [kau  kuisow]\textsubscript{O}  aaw-e-mik. \\
      \\
\glt
\z

village  man  cow  one  get-PA-1/3p

`The village men got one cow.'

\ea%x897
\label{ex:x897}
\gll [Yena  aamun]\textsubscript{O}  [ariwa=ke]\textsubscript{S}  aaw-o-k. \\
      \\
\glt
\z

1s.POSS  1s/p.younger.sibling  arrow=CF  get-PA-3s

`My younger brother was killed by an arrow.'

As was described in \sectref{sec:1.4.2.2}, Mauwake exhibits many typological characteristics associated with \textstyleAcronymallcaps{SOV} languages.

The basic constituent order is always based on the structure of a transitive clause. Intransitive clauses (\sectref{sec:5.3}) do not have objects, but otherwise the structure is the same as that in the transitive clauses. The structure of other types of clauses is described in the relevant sections.

The constituent order in an extended predication is harder to establish, because a clause typically has very few constituents, the average being only 1.2 non-verb constituents per clause; because any non-verbal element can be fronted as topic; and because the subject is often shown only by a verbal suffix and the object by an accusative pronoun in the \textstyleAcronymallcaps{VP}. A clause formula for a maximally extended predication is hypothetical, and mainly shows the order of the constituents on the basis of their attested orders in relation to each other: 

S    X\textsubscript{1}  O\textsubscript{1}  X\textsubscript{2}  O\textsubscript{2}  X\textsubscript{3}  V

There are two object positions\footnote{The two objects are discussed further in the next sections 5.2 and 5.3.}  and three X-positions for adverbial phrases.\footnote{Depending on the grammatical model, these may be called peripherals, obliques, satellites or adjuncts. I call them \textit{peripherals} and reserve the term \textit{adjunct} for the non-verb part of an adjunct plus verb construction.}  If a clause has only one object, it occupies the O\textsubscript{2} position immediately preceding the verb regardless of whether the semantic function is that of a patient, a recipient or a beneficiary. When there are two objects, their position is dictated mainly by their relative topicality. A [+human] argument tends to be more topical than a [\nobreakdash-human] one, so an object that is semantically a recipient (\stepcounter{nx}{\thenx}), (\stepcounter{nx}{\thenx}), or a beneficiary (\stepcounter{nx}{\thenx}), typically occupies the first object position, and the other object, typically a [\nobreakdash-human] patient, fills the second object position. 

\ea%x928
\label{ex:x928}
\gll [Muuka]\textsubscript{O1}  [sira]\textsubscript{O2}  iw-i-mik. \\
      \\
\glt
\z

son  custom  give.him-Np-PR.1/3p

`They teach the right behaviour to the son.'

\ea%x933
\label{ex:x933}
\gll Sarak=ke  [wi  takira]\textsubscript{O1}  [inglis]\textsubscript{O2}  [wia]\textsubscript{O1}  ofakow-i-ya.\footnote{Compare this with: \textit{Sarak=ke inglis wia ofakowiya} `Sarak teaches (them) English' and \textit{Sarak=ke wi takira wia ofakowiya} `Sarak teaches the children.' Both the recipient and the patient are coded in the same way as an object.} \\
      \\
\glt
\z

Sarak=CF  3p.UNM  child  English  3p.ACC  teach-Np-PR.3s

`Sarak teaches the children English.'

\ea%x916
\label{ex:x916}
\gll Ni  [auwa]\textsubscript{O1}  [maa]\textsubscript{O2}  p-ikiw-om-aka.  \\
      \\
\glt
\z

2p.UNM  1s/p.father  food  BPx-go-BEN-BNFY2.IMP.2p

`Take food to/for father.'

If a [-human] patient object is more topical than a [+human] object, it can occupy the first object position. A more topical [+human] object in (\stepcounter{nx}{\thenx}) would have an unmarked third person plural pronoun before the [-human] object.\footnote{More examples of can be found in \sectref{sec:5.3.2}-5.3.4.}

\ea%x917
\label{ex:x917}
\gll Onak=ke  [aaya]\textsubscript{O1}  [wia]\textsubscript{O2}  aaw-om-aya  \\
      \\
\glt
\z

3s/p.mother  sugarcane  3p.ACC  get-BEN-BNFY2.2/3s  

enim-or-om-ik-e-mik.

eat-descend-SS.SIM-be-PA-1/3p

`Their mother got sugarcane for them and they went down eating it.'

If both the objects are [-human], the one that is more clearly the patient, i.e. more profoundly affected by the action, occupies the O\textsubscript{2} position. Usually the object in O\textsubscript{1} position has a more locative-type meaning.

\ea%x927
\label{ex:x927}
\gll [Epira]\textsubscript{O1}  [lolom]\textsubscript{O2}  if-e-mik. \\
      \\
\glt
\z

plate  mud  smear-PA-1/3p

`They smeared the plate with mud' or: `They smeared mud on the plate.'

\ea%x934
\label{ex:x934}
\gll [Wut  makena  nain]\textsubscript{O1}  [ona]\textsubscript{O2}  puuk-a-m. \\
      \\
\glt
\z

Derris.root.tree  seed  that1  hole  cut-PA-1s

`I cut a hole in the seed of a derris root tree.'

The normal position of the peripherals is between the subject and the object \textstyleAcronymallcaps{NP,} if any (\stepcounter{nx}{\thenx}), or between the first and second object (\stepcounter{nx}{\thenx}). 

\ea%x895
\label{ex:x895}
\gll Yo  \textstyleEmphasizedVernacularWords{uura}  arua  isim-ap  ... \\
      \\
\glt
\z

1s.UNM  night  torch  light-SS.SEQ

`I lighted a torch in the night and ...'

\ea%x913
\label{ex:x913}
\gll [Wiipa  nain]\textsubscript{O}1  [\textstyleEmphasizedVernacularWords{samapora  iinan=pa}]\textsubscript{AdvP}  [epia]\textsubscript{O}2  \\
      \\
\glt
\z

daughter  that  floor  top=LOC  fire  

ururum-om-ap{\dots}

light-BEN-BNFY2.SS.SEQ

`They lighted a fire for the daughter on top of the floor, and ...'

A locative adverbial can also come between an object \textstyleAcronymallcaps{NP} and a verb. A deictic locative phrase or another short locative phrase is common in this position:

\ea%x914
\label{ex:x914}
\gll Emer  en-ow(a)  mua=ko  [emeria]\textsubscript{O}  [\textstyleEmphasizedVernacularWords{fan}]\textsubscript{AdvP}  aaw-o-k. \\
      \\
\glt
\z

sago  eat-NMZ  man=NF  woman  here  get-PA-3s

`A Sepik man got a wife here.'

\ea%x931
\label{ex:x931}
\gll Yo  [maa  unowa]\textsubscript{O}  [\textstyleEmphasizedVernacularWords{koora=pa}]\textsubscript{AdvP}  wu-a-m. \\
      \\
\glt
\z

1s.UNM  thing  many  house=LOC  put-PA-1s

`I put (the) many things in the house.'

The position immediately before the verb is also the only possible place for a [+human] locative adverbial, manifested by a dative pronoun (\sectref{sec:3.5.5}). In both (\stepcounter{nx}{\thenx}) and (\stepcounter{nx}{\thenx}) there are two locative adverbials, a [-human] and a [+human] one. The [+human] locative adverbial refers to the people of the location. If it is left out, the other locative refers to the location but not the people. 

\ea%x854
\label{ex:x854}
\gll [Ni  [koka-pa]\textsubscript{AdvP}  [\textstyleEmphasizedVernacularWords{wiar}]\textsubscript{AdvP} \textstyleEmphasizedVernacularWords{} in-em-ik-e-man  \\
      \\
\glt
\z

2p.UNM  jungle=LOC  3.DAT  sleep-SS.SIM-be-PA-2p  

nain]\textsubscript{RC} kerer-omak-eka.

that1 appear-DISTR/PL-IMP.2p

`You(pl.) who have slept in the jungle (villages), come!' 

\ea%x855
\label{ex:x855}
\gll I  amirk=iw  [Gawar]\textsubscript{AdvP}  [\textstyleEmphasizedVernacularWords{wiar}]\textsubscript{AdvP} \textstyleEmphasizedVernacularWords{} urup-e-mik. \\
      \\
\glt
\z

1p.UNM  day=INST  Gawar  3.DAT  ascend-PA-1/3p

`During the day we went to Gawar.'

If there are more adverbial phrases than one, a temporal phrase normally precedes any others (\stepcounter{nx}{\thenx}).  The relative order of the other adverbial phrases is syntactically quite free and depends on their relative topicality.

\ea%x915
\label{ex:x915}
\gll I  \textstyleEmphasizedVernacularWords{amirika  owowa  ewur}  me  ekap-em-ik-e-mik. \\
      \\
\glt
\z

1p.UNM  day  village  quickly  not  come-SS.SIM-be-PA-1/3p

`In the daytime we didn't come quickly to the village.'

\ea%x918
\label{ex:x918}
\gll Niena  \textstyleEmphasizedVernacularWords{ikoka  oram  neeke}  ika-i-non. \\
      \\
\glt
\z

2s/p.mother  later  for.nothing  there.CF  be-Np-FU.3s

`Your mother will later just be there (without you).'

\ea%x919
\label{ex:x919}
\gll \textstyleEmphasizedVernacularWords{Mokoma  kuisow}  \textstyleEmphasizedVernacularWords{naap  fan  yiam=iya}  ik-e-mik. \\
      \\
\glt
\z

year  one  thus  here  1p.REFL=COM  be-PA-1/3p

`They were here with us for about a year.' 

Both transitive and intransitive clauses are negated with the verbal negator \textstyleStyleVernacularWordsItalic{me} `not' placed immediately before the verb phrase.\footnote{For the placement of \textit{me} as a constituent negator, see \sectref{sec:6.2.2}.} 

\ea%x981
\label{ex:x981}
\gll I  \textstyleEmphasizedVernacularWords{me}  wia  amukar-e-mik. \\
      \\
\glt
\z

1p.UNM  not  3.ACC  scold-PA-1/3p

`We didn't scold them.'

\ea%x1057
\label{ex:x1057}
\gll Nain  yo  \textstyleEmphasizedVernacularWords{me}  baurar-em-ik-e-m. \\
      \\
\glt
\z

but  1s.UNM  not  run.away-SS.SIM-be-PA-1s

`But I didn't keep running away.'

As was mentioned above, pragmatic factors influence the constituent order.  A constituent that is fronted as a theme to the beginning of the clause is still part of the constituent structure of the clause (for theme, see \sectref{sec:9.1}).\footnote{In Amele the pre-verbal position is a focus position \citep[142]{Roberts1987}, but in Mauwake this does not seem to be the case. Focus is indicated by a heavier stress and sometimes by focus markers.} 

\ea%x930
\label{ex:x930}
\gll [\textstyleEmphasizedVernacularWords{Oposia}  \textstyleEmphasizedVernacularWords{gelemuta}]\textsubscript{O}1  [wiam  erup  fain  wia]\textsubscript{O}2  \\
      \\
\glt
\z

meat  little  3p.REFL  two  this  3p.ACC  

wu-om-a-m.

put-BEN-BNFY2.PA-1s

`A bit of the meat I put (aside) for these two.'

A left-dislocated theme (\sectref{sec:9.1}) and an afterthought are outside the clause proper.  A left-dislocated theme (\stepcounter{nx}{\thenx}) is separated from the clause by a short pause and a comma intonation, slightly rising pitch at the end of the utterance.  An afterthought (\stepcounter{nx}{\thenx}), right-dislocated, is also separated from the rest of the clause by a short pause.

\ea%x935
\label{ex:x935}
\gll \textstyleEmphasizedVernacularWords{Irak-owa  fa},  opora  unowa  akena. \\
      \\
\glt
\z

fight-NMZ  EXC  talk  much  very

`The war, now - there is much to talk about.'

\ea%x929
\label{ex:x929}
\gll Inasin  opaimika  eliwa  me  yia  maak-e-mik,  \textstyleEmphasizedVernacularWords{wi \\
      \\
\glt
\z

spirit  talk  good  not  1p.ACC  tell-PA-1/3p  3p.UNM  

\textstyleEmphasizedVernacularWords{Yaapan=ke}.

Japan=CF

`They didn't speak good Pidgin to us, the Japanese (didn't).'

\subsection{Syntactic arguments}
\hypertarget{RefHeading21941935131865}{}
Syntactic arguments together with the verb form the core of a clause. They differ from the peripherals in that they have a grammatical relation to the verb (Foley and Van Valin 1984:77-80), and therefore have to do with the valence of the clause. The basic syntactic structure is influenced by the arguments but not by the peripherals.  In Mauwake the only syntactic arguments are subject and object. 

Since Mauwake is very clearly a nominative-accusative type language, the grammatical role of \textstyleEmphasizedWords{\textsc{subject}}\textstyleEmphasizedWords{\textsc{} }and the semantic role of \textstyleEmphasizedWords{\textsc{agent}}\textstyleEmphasizedWords{\textsc{} }or\textstyleEmphasizedWords{\textsc{} }\textstyleEmphasizedWords{\textsc{actor}} normally converge on the same constituent, which usually, but not always, also has the pragmatic role of \textstyleEmphasizedWords{\textsc{topic}}. Another semantic role the subject may have is that of \textstyleEmphasizedWords{\textsc{experiencer}}, and in verbless clauses that of ``\textstyleEmphasizedWords{\textsc{theme}}''\footnote{This semantic role ``theme'' is different from the pragmatic function and refers to the participant which is said to be in some state, or located in some place \citep[140]{Andrews2007a}. Because of a possible confusion with the pragmatic role of theme, the term for the semantic role is written inside double quotes.}.

The syntactic coding of the subject includes both the clausal constituent order and cross-referencing on the verb.  In pragmatically neutral clauses the subject is the first of two argument noun phrases.  It is also obligatorily marked on the person/number suffix of the verb. The same distinctions are made in the subject marking of the verb as in the personal pronouns: first, second or third person and singular or plural number.

\ea%x936
\label{ex:x936}
\gll Komori  emeria  wu-a-k. \\
      \\
\glt
\z

Komori  woman  put-PA-3s

`Komori buried his wife.'

The subject governs reflexivization.  A noun phrase itself is marked as subject only when the subject \textstyleAcronymallcaps{NP} is a pronoun: then it has to be unmarked (\sectref{sec:3.5.2.1}) or in the genitive case (\sectref{sec:3.5.4}).

\ea%x920
\label{ex:x920}
\gll Tirinde  uura  \textstyleEmphasizedVernacularWords{i}  nainiw  \textstyleEmphasizedVernacularWords{yiam}  fiirim-e-\textstyleEmphasizedVernacularWords{mik}. \\
      \\
\glt
\z

Wednesday  evening  1p.UNM  again  1p.REFL  gather-PA-1/3p

`On Wednesday evening we gathered again.'

The switch-reference system (\sectref{sec:8.2.3}) basically tracks the subject.\footnote{Roberts maintains that in Amele and most other Papuan languages the switch reference system tracks the thematic notion of topic across clauses (1988b:105, 1997). But his definition of topic (1988b:96) is such that in Mauwake it practically excludes all other clause constituents except the subject.} 

\ea%x921
\label{ex:x921}
\gll {\dots}imen-\textstyleEmphasizedVernacularWords{ap}  maak-\textstyleEmphasizedVernacularWords{iwkin}  o  miim-o-k. \\
      \\
\glt
\z

find-SS.SEQ  tell-2/3p.DS  3s.UNM  precede-PA-3s

`{\dots}when they found him and told him, he went ahead.'

However, in the following example, the initial unmarked pronoun \textstyleStyleVernacularWordsItalic{wi}  pluralizes the object/theme, the Australians; in the second clause the Australians are the subject.

\ea%x922
\label{ex:x922}
\gll Wi  Australia  \textstyleEmphasizedVernacularWords{Amerika=ke}  wia  asip-\textstyleEmphasizedVernacularWords{iwkin} \\
      \\
\glt
\z

3p.UNM  Australia  America=CF  3p.ACC  help-2/3p.DS

irak-owa  nomak-e-mik.

fight-NMZ  win-PA-1/3p

`Australians were helped by Americans and won the war.'

Syntactic operations like passivization and dative shift do not apply to Mauwake and consequently cannot be used to define either the subject or other syntactic arguments. 

Although the prototypical subject is a [+human] agent, also an instrument that is unable to initiate an action can become a metaphorical agent, thus the subject (Giv\'on 1984:106). This is very common in expressions describing cases where one involuntarily hurts oneself with some instrument. (In the following example there is also possessor raising (\sectref{sec:5.3.2.3}) resulting in two objects.)

\ea%x958
\label{ex:x958}
\gll [\textstyleEmphasizedVernacularWords{Fura=ke}]\textsubscript{S}  [merena]\textsubscript{O1}  [efa]\textsubscript{O2}  puuk-a-k. \\
      \\
\glt
\z

knife=CF  leg  1s.ACC  cut-PA-3s

`I cut my leg / myself in the leg.' (Lit: `A knife cut my leg.')

Because the subject is so often marked by only a verbal suffix, it would be possible to treat the subject marking on the verb as the real subject, as Van Valin and LaPolla suggest for those languages that mark the core arguments on the verb (1997:33-34). Although this approach would have some advantages,\footnote{The main advantage would be having an overt subject in every clause, regardless of the presence of a separate subject NP.} I choose the more traditional way of treating the \textstyleAcronymallcaps{NP} as the subject, both because 1) an object is not marked in the verb inflection but requires either a separate \textstyleAcronymallcaps{NP} or an accusative pronoun outside the verb proper,\footnote{The small group of object cross-referencing verbs (\sectref{sec:3.8.4.2.4}) are an exception.} and because 2) the constituent order, based on the position of the subject and object \textstyleAcronymallcaps{NP}s, has many interconnections with various parts of the syntax. 

The \textstyleEmphasizedWords{\textsc{object}} as a syntactic role is not coded on the verb word except in the few object cross-referencing verbs (\sectref{sec:3.8.4.2.4}). A [+human] object must be referenced by an accusative pronoun (\sectref{sec:3.5.3}) preceding the verb in the verb phrase (\sectref{sec:4.5}), even if the object is also expressed by a full noun phrase earlier in the clause (\stepcounter{nx}{\thenx}). The position of the object \textstyleAcronymallcaps{NP} in the argument structure of the clause is between the subject and the verb (\stepcounter{nx}{\thenx}), but this syntactic definition is not very useful, as subject and object do not co-occur very often, and sometimes when they do, the object is fronted to the clause-initial theme position (\stepcounter{nx}{\thenx}). 

\ea%x923
\label{ex:x923}
\gll \textstyleEmphasizedVernacularWords{Emeria}  naap  \textstyleEmphasizedVernacularWords{wia}  aruf-i-nen  na-ep  on-a-k. \\
      \\
\glt
\z

woman  thus  3p.ACC  hit-Np-FU.1s  say-SS.SEQ  do-PA-3s

`He tried to hit women that way.'

\ea%x924
\label{ex:x924}
\gll Amia  mua=ke  \textstyleEmphasizedVernacularWords{wiam  erup  nain  wia}  nokar-e-k. \\
      \\
\glt
\z

bow  man=CF  3p.REFL  two  that1  3p.ACC  ask-PA-3s

`The policeman asked those two.'

\ea%x925
\label{ex:x925}
\gll \textstyleEmphasizedVernacularWords{Mua  emeria  muuka  wiipa}  eka=ke  \textstyleEmphasizedVernacularWords{wia}  mu-o-k. \\
      \\
\glt
\z

man  woman  son  daughter  river=CF  3p.ACC  swallow-PA-3s

`A man and his wife and children were drowned by the river.'

There is not enough basis in Mauwake for positing a separate syntactic category \textstyleEmphasizedWords{\textsc{indirect object}}. In many languages the most typical verb requiring an indirect object for the semantic role of recipient is `give'.  But in Mauwake the verbs `give' and `feed/give to eat' are among the few object cross-referencing verbs (\sectref{sec:3.8.4.2.4}), which change their stem according to the patient or recipient object. The verb `send', another cross-linguistically typical verb taking an indirect object, in Mauwake requires the benefactive suffix on the verb (\sectref{sec:3.8.4.3.2}), rather than a marking on the \textstyleAcronymallcaps{NP}.  In (\stepcounter{nx}{\thenx}) the verb \textstyleStyleVernacularWordsItalic{maak}- `tell' has two objects, the patient object \textstyleStyleVernacularWordsItalic{moma} `taro' and the recipient object \textstyleStyleVernacularWordsItalic{yia} `us', which is marked by an accusative pronoun in the \textstyleAcronymallcaps{VP} in the same way as a [+human] undergoer/patient. 

\ea%x932
\label{ex:x932}
\gll Wi  [\textstyleEmphasizedVernacularWords{moma}]\textsubscript{O}  [\textstyleEmphasizedVernacularWords{yia}]\textsubscript{O}  maak-i-mik. \\
      \\
\glt
\z

3p.UNM  taro  1p.ACC  tell-Np-PR.1/3p

`They are telling us (to get them) taro.'

This is consistent with Whitehead's (1981:51) survey results showing that ``a large number of [Papuan] languages ... do not differentiate between Patient and Recipient''. Rather, there are verbs that are capable of taking two objects (ibid. 52).\footnote{Usan behaves in the same way as Mauwake \citep[160]{Reesink1987}.} Amele is one of those Papuan languages that clearly have indirect object as a syntactic category \citep[69]{Roberts1987}.

It could be argued that a locative adverbial is an argument rather than a peripheral with the directional verbs\footnote{For Usan, where motion verbs have either the goal or locative as a nuclear argument, see \citet[130]{Reesink1987}.} (\sectref{sec:3.8.4.4.5}) and with the verb \textstyleStyleVernacularWordsItalic{ik}- `be/live (somewhere)' (\sectref{sec:3.8.4.4.1}), as these verbs so often co-occur with a locative. But these verbs also occur without a locative so often that it would be both unnatural to interpret all of those instances as elliptical constructions and sometimes difficult to posit the ``deleted'' locative. 

\ea%x1456
\label{ex:x1456}
\gll Maak-e-mik,  {\textquotedbl}No  ikiw-e,  irak-owa  maneka \\
      \\
\glt
\z

tell-PA-1/3p  2s.UNM  go-IMP.2s  fight-NMZ  big  

fan-e-k  a.''

here-PA-3s  INTJ

`They told him, ``Go, the war is here.'' '

\ea%x1457
\label{ex:x1457}
\gll Iwera  uruk-am-ika-iwkin  wi  ikiw-emi \\
      \\
\glt
\z

coconut  drop-SS.SIM-be-2/3p.DS  3p.UNM  go-SS.SIM  

aaw-em-ik-e-mik ...

get-SS.SIM-be-PA-1/3p

`When they\textsubscript{i} kept dropping coconuts, they\textsubscript{j} kept going and getting them {\dots}'

The verb \textstyleStyleVernacularWordsItalic{ik}- `be' seldom occurs alone (\stepcounter{nx}{\thenx}), repeated below as (\stepcounter{nx}{\thenx}). This is probably due to its very neutral semantic character. When it denotes being or living somewhere, it is accompanied by a locative adverbial phrase (\stepcounter{nx}{\thenx}). Another very common adverbial phrase accompanying the verb is a manner phrase, especially the adverb \textstyleStyleVernacularWordsItalic{naap} `thus' (\stepcounter{nx}{\thenx}). Rather than positing separate clause types with adverbials as arguments it seems reasonable to subsume clauses like these under intransitive clauses.

\ea%x1458
\label{ex:x1458}
\gll Ika-i-nen. \\
      \\
\glt
\z

be-Np-FU.1s

`I will just be like this.'

\ea%x1459
\label{ex:x1459}
\gll Siiwa  erepam  naap  \textstyleEmphasizedVernacularWords{nan  ik-ok}  napuma  sariar-e-k. \\
      \\
\glt
\z

moon  four  thus  there  be-SS  sickness  heal-PA-3s

`He was there about four months and his sickness was healed.'

\ea%x1460
\label{ex:x1460}
\gll Komor(a)  muuka  nain  memel-am-ik-emkun \\
      \\
\glt
\z

cuscus  son  that1  tame-SS.SIM-be-1s/p.DS

\textstyleEmphasizedVernacularWords{naap  ik}\textstyleEmphasizedVernacularWords{-}\textstyleEmphasizedVernacularWords{ok}  iir  oko  uura  baurar-e-k.

thus  be-SS  time  other  night  escape-PA-3s

`I was taming the cuscus and it was like that and then one night it escaped.'

\subsection{Transitive clauses}
\hypertarget{RefHeading21961935131865}{}
Transitivity is an important characteristic of both a verb and a clause; which of these is primary has been an object of a great deal of discussion.\footnote{In transformational grammar (EST) verbs had to be subcategorized in the lexicon according to whether they allowed a NP-object or not \citep[120]{Radford1981}. Also Van Valin and La\citet[147-157]{Polla1997} consider transitivity essentially a characteristic of a verb, distinguishing between the semantic, syntactic and macrorole transitivity of each verb. Giv\'on (1995:76), Kittil\"a (2002:25) and \citet[115]{Dixon2010}, among others, maintain that transitivity is primarily a characteristic of a whole clause. Taking still another angle, Hopper and \citet[294]{Thompson1980} claim that transitivity is very closely bound with discourse features, namely background and foreground.} This may be language-specific. In languages like English where an intransitive verb like \textstyleForeignWords{sneeze} can be made transitive in a construction \textstyleForeignWords{He sneezed the napkin off the table} \citep[9]{Goldberg1995}, it makes sense to say that the verb combines with a transitive argument structure construction \citep[6]{Goldberg2006}. But in Mauwake it can be claimed that the transitivity of the verb is primary. The claim is supported by the clear distinction between transitive and intransitive verbs and the fact that transitive verbs often require a dummy object when there is no real object available.

Clauses are here looked at from the point of view of \textstyleEmphasizedWords{\textsc{syntactic transitivity}}: clauses that have an overt object are treated as transitive clauses, regardless of the semantic role of the object. 

Linguistically the most interesting transitive clauses are those that have two or possibly even three objects. These can be divided into three different groups: clauses where a transitive verb can take more than one object without requiring any morphological or syntactic operation (\sectref{sec:5.3.2.1}) and those where an object has been added by a valence-increasing operation (\sectref{sec:5.3.2.2}) or by possessor raising (\sectref{sec:5.3.2.3}).

\subsubsection{Monotransitive clauses}
\hypertarget{RefHeading21981935131865}{}
Monotransitive clauses have a transitive verb and one object, which is prototypically a patient. 

\ea%x937
\label{ex:x937}
\gll [Sawur  emeria  nain=ke]\textsubscript{S}  [\textstyleEmphasizedVernacularWords{ona  soma  mua  nain}]\textsubscript{O}  ifakim-o-k. \\
      \\
\glt
\z

spirit  woman  that1=CF  3s.GEN  lover  man  that1  kill-PA-3s

`The spirit woman killed her lover.'

\ea%x939
\label{ex:x939}
\gll Amirika  [i]\textsubscript{S}  [\textstyleEmphasizedVernacularWords{maa  eneka  fain}]\textsubscript{O}  uup-ep\textstyleFreeTranslationChar{ \\
      \\
\glt
\z

noon  1p.UNM  thing  tooth  this  cook-SS.SEQ } 

enim-i-yen.

eat-Np-FU.1p

`At noon we'll cook and eat this (edible) animal.'

If there is only one \textstyleAcronymallcaps{NP} argument in a transitive clause, it is usually the object rather than subject (\stepcounter{nx}{\thenx}), unless marked with the contrastive focus marker -\textstyleStyleVernacularWordsxiiptItalic{ke} (\stepcounter{nx}{\thenx}).

\ea%x941
\label{ex:x941}
\gll [\textstyleEmphasizedVernacularWords{Wiawi}]\textsubscript{O}  kuum-eya  aw-ep  eka  iw-a-k=na  wia. \\
      \\
\glt
\z

3s/p.father  burn-2/3.DS  burn-SS.SEQ  river  go-PA-3s=TP  no

`(It) burned their father and when he burned he went into the river but it didn't help.'

\ea%x938
\label{ex:x938}
\gll Ufer-iwkin  urup-em-ik-eya  [yos=ke]\textsubscript{S}  mik-a-m. \\
      \\
\glt
\z

miss-2/3p.DS  ascend-SS.SIM-be-2/3s.DS  1s.FC=CF  shoot-PA-1s

`When they missed and it was going up, I shot it.'

But in the following clause the contrastive focus marker is not needed to disambiguate the subject from the object: \textstyleStyleVernacularWordsItalic{wiam arow} `the three of them' has to be the subject; if it were the object, it would require the third person plural accusative pronoun \textstyleStyleVernacularWordsItalic{wia} in the \textstyleAcronymallcaps{VP}.

\ea%x940
\label{ex:x940}
\gll [Ne  wiam  arow]\textsubscript{S}  miim-ap  {\dots} \\
      \\
\glt
\z

and  3p.REFL  three  hear-SS.SEQ

`And the three of them heard it, and {\dots}'

Clauses with an impersonal experience verb (\sectref{sec:3.8.4.4.7}) as the predicate are also transitive. The subject is inanimate, usually a body part where the pain is felt, and the human experiencer is the object. The possibility of adding the contrastive focus clitic to the noun indicating body part shows that it is the subject rather than a second object (\stepcounter{nx}{\thenx}).

\ea%x1012
\label{ex:x1012}
\gll Uuw-ap  uuw-ap  [oona=ke]\textsubscript{S}  [efa]\textsubscript{O}  sirir-i-ya. \\
      \\
\glt
\z

work-SS.SEQ  work-SS.SEQ  bone=CF  1s.ACC  hurt-Np-3s

`I have worked and worked, and my bones hurt (me).'

\ea%x1015
\label{ex:x1015}
\gll Yo  [uroma]\textsubscript{S}  [efa]\textsubscript{O}  op-am-ik-eya  yo  haussik  \\
      \\
\glt
\z

1s.UNM  stomach  1s.ACC  hold-SS.SIM-be-2/3s.DS  1s.UNM  aidpost

me  ikiw-e-m.

not  go-PA-1s

`I was having birth pains (lit: My stomach was holding/grabbing me) but I did not go to the aidpost.'

\subsubsection{Ditransitive clauses}
\hypertarget{RefHeading22001935131865}{}
A number of ditransitive clauses (\stepcounter{nx}{\thenx})-(\stepcounter{nx}{\thenx}) were already listed under \sectref{sec:5.1}. They belong to the three different groups below.

\paragraph[Inherent ditransitivity ]{Inherent ditransitivity} 
\hypertarget{RefHeading22021935131865}{}
Some  ditransitive clauses are called inherently ditransitive, because they do not require a morphological or syntactic process to make them ditransitive. The most common verbs in ditransitive clauses of this type are the object cross-referencing verbs and the utterance verb \textstyleStyleVernacularWordsItalic{maak}- `tell', and the verb \textstyleStyleVernacularWordsItalic{ofakow}- `show, teach'.

The (recipient) object is marked in the verb root of the object cross-referencing verbs denoting giving and feeding (\sectref{sec:3.8.4.2.4}), but it may appear as a separate \textstyleAcronymallcaps{NP} as well (\stepcounter{nx}{\thenx}):

\ea%x950
\label{ex:x950}
\gll [Mua  yiar  ekap-e-mik  nain]\textsubscript{O}  [pura  kui-kuisow\textbf{]}\textsubscript{O} \\
      \\
\glt
\z

man  1p.DAT  come-PA-1/3p  that  bunch  RDP-one

wi-e-mik.

give.them-PA-1/3p

`We gave a bunch each to the men who came to us.'

The section on utterance verbs (\sectref{sec:3.8.4.4.6}) describes in some detail how these verbs behave in clauses. \textstyleStyleVernacularWordsItalic{Maak}- `tell' requires the addressee/recipient to be a [+human] obligatory object, and as a second object it often has a \textstyleAcronymallcaps{NP} denoting the speech itself or the contents of that speech. 

\ea%x1839
\label{ex:x1839}
\gll I  \textbf{[}\textbf{opora  muut  nain}\textbf{]}\textsubscript{O}\textbf{ } \textbf{[}\textbf{nefa}\textbf{]}\textsubscript{O}\textbf{  maak-u  na-ep} \\
      \\
\glt
\z

1p.UNM  talk  only  thus  2s.ACC  tell-IMP.2d  say-SS.SEQ

ep-a-mik.

come-PA-1/3p

`We came wanting to tell you just that (talk).'

\ea%x955
\label{ex:x955}
\gll Wi  \textbf{[}\textstyleEmphasizedVernacularWords{moma}\textbf{]}\textsubscript{O} \textstyleEmphasizedVernacularWords{} \textbf{[}\textstyleEmphasizedVernacularWords{yia}\textbf{]}\textsubscript{O}  \textstyleEmphasizedVernacularWords{maak-i-mik},  moma=ko  \\
      \\
\glt
\z

3p.UNM  taro  1p.ACC  tell-Np-PR.1/3p  taro=NF

wi-i-yan.

give.them-Np-FU.1p

`They tell us (to get them) taro, (so) we'll give them taro.'

\textstyleStyleVernacularWordsItalic{Na}- `say, speak, call, think' most commonly has the quotation as a speech complement, but it may also have up to two nominal objects instead.

\ea%x956
\label{ex:x956}
\gll [\textstyleEmphasizedVernacularWords{Waaya}\textbf{]}\textsubscript{O} \textstyleEmphasizedVernacularWords{} \textbf{[}\textstyleEmphasizedVernacularWords{yia}\textbf{]}\textsubscript{O}  \textstyleEmphasizedVernacularWords{na-iwkin}  waaya  wienak-em-ik-e-mik. \\
      \\
\glt
\z

pig  1p.ACC  say-2/3p.DS  pig  feed.them-SS.SIM-be-PA-1/3p

`They spoke about pigs to us and we kept giving them pigs to eat.'

There are a few verbs that are ordinary transitive verbs but which can take semantically different objects. It is also possible to have one of each kind in the same clause. The verb \textstyleStyleVernacularWordsItalic{if}- `paint, spread' can have a patient or goal object; the sentence in example (\stepcounter{nx}{\thenx}) includes both. Another such verb is \textstyleStyleVernacularWordsItalic{mik}\nobreakdash- `spear, hit' see (\stepcounter{nx}{\thenx}).  

\ea%x944
\label{ex:x944}
\gll [Yena  aasa\textbf{]}\textsubscript{O}  \textbf{[}ofa\textbf{]}\textsubscript{O}  if-e-m. \\
      \\
\glt
\z

1s.GEN  canoe  colour  paint-PA-1s

`I painted my canoe with paint.' Or: `I spread paint on my canoe.'

\paragraph[Derived ditransitivity ]{Derived ditransitivity} 
\hypertarget{RefHeading22041935131865}{}
When the transitivity is increased by one of the valence-increasing strategies (\sectref{sec:3.8.4.3}), a recipient or beneficiary (\stepcounter{nx}{\thenx}) becomes a second object. The linear order of the two objects depends on their relative topicality.

\ea%x947
\label{ex:x947}
\gll [Moma  pura  oko]\textsubscript{O}  [Kuuten]\textsubscript{O}  amap-urup-om-a-mik. \\
      \\
\glt
\z

taro  bunch  other  Kuuten  BPx-ascend-BEN-BNFY2.PA-1/3p

`They took another bunch of taro up for Kuuten.'

\ea%x1840
\label{ex:x1840}
\gll Ne  [mua  nain]\textsubscript{O}  [waaya]\textsubscript{O}  mik-om-a-mik. \\
      \\
\glt
\z

ADD  man  that1  pig  spear-BEN-BNFY2.PA-1/3p

`And they speared that man a pig.'

\paragraph[Possessor raising]{Possessor raising}
\hypertarget{RefHeading22061935131865}{}
There are also cases with two patient-type objects, either one of which could be the single patient of the same verb. One of these objects has resulted from possessor raising: the possessor of the initial object \textstyleAcronymallcaps{NP,} which has to be a semantic patient, has been ``raised'' to become a second object (Van Valin and LaPolla 1997:258, Payne 1997:194-6). Especially when something is done to a body part or name, or something closely identified with a person, both the person and the other noun occur as objects.

\ea%x951
\label{ex:x951}
\gll [\textstyleEmphasizedVernacularWords{Merena}]\textsubscript{O} \textstyleEmphasizedVernacularWords{} [\textstyleEmphasizedVernacularWords{efa}\textbf{]}\textsubscript{O} \textstyleEmphasizedVernacularWords{} keraw-a-k. \\
      \\
\glt
\z

leg  1s.ACC  bite-PA-3s

`It bit me in the leg.' Or: `It bit my leg.'

\ea%x952
\label{ex:x952}
\gll [\textstyleEmphasizedVernacularWords{No  unuma}]\textsubscript{O}\textbf{ } [\textstyleEmphasizedVernacularWords{nefa}\textbf{]}\textsubscript{O} \textstyleEmphasizedVernacularWords{} faker-i-kuan. \\
      \\
\glt
\z

2s.UNM  name  2s.ACC  raise-Np-FU.3p

`They will praise (lit: lift up) your name.'

\ea%x957
\label{ex:x957}
\gll [\textstyleEmphasizedVernacularWords{Opaimika}\textbf{]}\textsubscript{O}\textbf{ } [\textstyleEmphasizedVernacularWords{efa}\textbf{]}\textsubscript{O} \textstyleEmphasizedVernacularWords{} fien-a-man. \\
      \\
\glt
\z

talk  1s.ACC  push.aside-PA-2p

`You disregarded/disobeyed my talk.'

\ea%x948
\label{ex:x948}
\gll Era=pa  [\textstyleEmphasizedVernacularWords{ekera  wiam  erup}]\textsubscript{O1} \textstyleEmphasizedVernacularWords{} [\textstyleEmphasizedVernacularWords{kukusa}]\textsubscript{O2} \textstyleEmphasizedVernacularWords{} [\textstyleEmphasizedVernacularWords{wia}]\textsubscript{O1} \\
      \\
\glt
\z

way=LOC  1s/p.sister  3p.REFL  erup  picture  3p.ACC

aaw-o-k.

take-PA-3s

`On the way he took a picture of my two sisters.'

\ea%x949
\label{ex:x949}
\gll Mua  papako=ke  [\textstyleEmphasizedVernacularWords{irak-owa}]\textsubscript{O} \textstyleEmphasizedVernacularWords{} [\textstyleEmphasizedVernacularWords{wia}]\textsubscript{O} \textstyleEmphasizedVernacularWords{} puuk-a-mik. \\
      \\
\glt
\z

man  some=CF  fight-NMZ  3p.ACC  cut-PA-1/3p

`Some men\textsubscript{i} stopped their\textsubscript{j} fight.'

Even three objects are allowed, but this is rare (\stepcounter{nx}{\thenx}): the verb \textstyleStyleVernacularWordsItalic{mik}- `spear, hit' itself allows two different objects, and the third one is added via possessor raising. The objects have to be in this order. Note that in the English translation, only one direct object is allowed, and the other two phrases have to be either possessive or oblique.

\ea%x953
\label{ex:x953}
\gll [\textstyleEmphasizedVernacularWords{Keema-muuna,  umakuna}\textbf{]}\textsubscript{O} \textstyleEmphasizedVernacularWords{} [\textstyleEmphasizedVernacularWords{meta}]\textsubscript{O} \textstyleEmphasizedVernacularWords{} [\textstyleEmphasizedVernacularWords{yia}]\textsubscript{O}  \textstyleEmphasizedVernacularWords{mik-i-mik}. \\
      \\
\glt
\z

knee-joint  neck  ritual.paste  1p.ACC  hit-Np-PR.1/3p

`They stick the \textstyleForeignWords{meta} paste on our knees and necks' or: `They mark our knees and necks with the \textstyleForeignWords{meta} paste.'

As the preferred clause structure in Mauwake is short and because it is harder to process a verb with many arguments, a common strategy is to divide the arguments between more than one clause, so that each clause has only one or two arguments:

\ea%x967
\label{ex:x967}
\gll I  dabuela  aaw-ep  Yaapan  wi-em-ik-e-mik. \\
      \\
\glt
\z

1p.UNM  pawpaw  take-SS.SEQ  Japan  give.them-SS.SIM-be-PA-1/3p

`We took pawpaws and gave them to the Japanese' or: `We gave pawpaws to the Japanese.'

Even if having more than one \textstyleAcronymallcaps{NP} in non-subject argument or peripheral positions in the same clause is not preferred, it is still reasonably common. But having more than one pronoun as arguments or peripherals is unusual. In the rare case that that does happen, the accusative pronoun occupies the position closest to the verb, next the dative pronoun, then the others. The first two of the following three examples have been elicited.

\ea%x1574
\label{ex:x1574}
\gll Mua  nain  teeria  muutiw  \textstyleEmphasizedVernacularWords{wame  wia}  ofakow-a-k. \\
      \\
\glt
\z

man  that  group  only  3s.REFL  3p.ACC  show-PA-3s

`He only showed himself to that man's group.'

\ea%x1577
\label{ex:x1577}
\gll O  \textstyleEmphasizedVernacularWords{wiar  nefa}  sesek-i-yem. \\
      \\
\glt
\z

3s.UNM  3.DAT  2s.ACC  send-Np-PR.1s

`I am sending you to him.'

\ea%x1575
\label{ex:x1575}
\gll Emeria  ikoka  Yaapan  \textbf{wiena}  \textbf{niar}  aaw-i-kuan \\
      \\
\glt
\z

woman  later  Japan  3p.GEN  2p.DAT  take-Np-FU.3p

`Later the Japanese will take your wives as their own.'

\subsection{Intransitive clauses}
\hypertarget{RefHeading22081935131865}{}
An intransitive clause in Mauwake is a verbal clause that does not have an object. It normally indicates an event of some kind (action or process), or a state. This differs from the definition used for typological studies of an intransitive predication consisting of ``a one-place predicate and its argument'' \citep[9]{Stassen1997} in that in Mauwake those predications that indicate some property or quality, or designate a class, are not treated as intransitive but as verbless clauses (\sectref{sec:5.6}). Any of the intransitive verbs (\sectref{sec:3.8.4.2.1}) can be the predicate in an intransitive clause, whereas a verbless clause characteristically has no verb. The only negation strategy for clauses with a verbal predicate is the negator \textstyleStyleVernacularWordsItalic{me} `not', whereas verbless clauses have more negator options. 

The following clauses are typical intransitive clauses:

\ea%x961
\label{ex:x961}
\gll Epa  wiim-eya  mua  karer-omak-e-mik. \\
      \\
\glt
\z

place  dawn-2/3s.DS  man  gather-DISTR/PL-PA-1/3p

`When it got light a lot of people gathered.'

\ea%x964
\label{ex:x964}
\gll O  koora=pa  naap  ik-ok  um-o-k. \\
      \\
\glt
\z

3s.UNM  house=LOC  thus  be-SS  die-PA-3s

`She was in the house like that and died.'

\ea%x959
\label{ex:x959}
\gll Uuriw  akena  mukuna  nain  kerer-e-k. \\
      \\
\glt
\z

morning  truly  fire  that1  appear-PA-3s

`The fire started early in the morning.'

\ea%x960
\label{ex:x960}
\gll Iiwawun  iwera  pun  wiar  aw-omak-e-k. \\
      \\
\glt
\z

altogether  coconut  also  3.DAT  burn-DISTR/PL-PA-3s

`His many coconut trees too burned altogether.'

\ea%x965
\label{ex:x965}
\gll I  Sarak  ikos  owowa  ekap-em-ik-e-mik. \\
      \\
\glt
\z

1p.UNM  Sarak  with  village  come-SS.SIM-be-PA-1/3p

`Sarak and I kept coming back to the village.'

\ea%x962
\label{ex:x962}
\gll Fikera  mamaiya=pa  nan  pok-ap  ik-e-mik. \\
      \\
\glt
\z

kunai.grass  close=LOC  there  sit-SS.SEQ  be-PA-1/3p

`We were sitting near the \textstyleForeignWords{kunai} grass.'

\ea%x968
\label{ex:x968}
\gll Ne  kiiriw  miiw-aasa  nan  ik-eya  {\dots} \\
      \\
\glt
\z

and  again  land-canoe  there  be-2/3s.DS

`And again the car was/stayed there, and {\dots}'

Many climate expressions are normal intransitive clauses.

\ea%x1020
\label{ex:x1020}
\gll Moram  \textstyleEmphasizedVernacularWords{ewar } pun  \textstyleEmphasizedVernacularWords{wuun-e-k}  ne  {\dots} \\
      \\
\glt
\z

why  west.wind  too  blow-PA-3s  and

`Because wind blew too, and {\dots}' Or: `Because it was windy too, and {\dots}'

\ea%x1022
\label{ex:x1022}
\gll \textstyleEmphasizedVernacularWords{Ipia  or-om-ik-eya}  owora  aaw-ep  up-o-k. \\
      \\
\glt
\z

rain  descend-SS.SIM-be-2/3s.DS  betelnut  take-SS.SEQ  plant-PA-3s

`When it was raining he took betelnuts and planted them.'

\ea%x1021
\label{ex:x1021}
\gll \textstyleEmphasizedVernacularWords{Epa  kokom-ar-eya}  in-e-mik. \\
      \\
\glt
\z

place  dark-INCH-2/3s.DS  sleep-PA-1/3p

`It became dark and we slept.'

The resultative verbs (\sectref{sec:3.8.4.4.4}) require a nominal argument expressing the result of change: 

\ea%x963
\label{ex:x963}
\gll Mua  eneka,  woosa  \textstyleEmphasizedVernacularWords{kia  kir-em-ik-ua}. \\
      \\
\glt
\z

man  tooth  head  white  turn-SS.SIM-be-PA.3s

`The people's teeth and skulls were turning white.'

\ea%x966
\label{ex:x966}
\gll Arim-emi  \textstyleEmphasizedVernacularWords{emeria  ar-e-k}. \\
      \\
\glt
\z

grow-SS.SIM  woman  become-PA-3s

`She grew and became a woman.'

A few intransitive verbs can occur with a syntactic object or object-like element whose semantic role is not a patient. These differ from true patient objects in that the range of possible ``objects'' for those verbs is very restricted, they cannot be substituted with an accusative pronoun, and the verb cannot occur with the dummy object \textstyleStyleVernacularWordsItalic{maa} `thing'. The first type can be called a ``content object'' \citep[179]{HakulinenEtAl1979}%Karlsson
: 

\ea%x308
\label{ex:x308}
\gll Wis  pun  wiisa  uf-e-mik. \\
      \\
\glt
\z

3p.FC  too  wiisa  dance-PA-1/3p

`They, too, danced ``wiisa''.'

The second type is an object-like adverbial, as it functions in the same way as an adverbial phrase.

\ea%x307
\label{ex:x307}
\gll Era  maala  soomar-e-mik\textstyleEmphasizedVernacularWords{\textmd{\textit{.}}} \\
      \\
\glt
\z

way  long  walk-PA-1/3p

`We walked a long way.'

\subsection{Existential and possessive clauses}
\hypertarget{RefHeading22101935131865}{}
Existential clauses and possessive clauses are distinguished from the intransitive clauses. Only the verb \textstyleStyleVernacularWordsItalic{ik}- `be' is used as the predicate in both of them. 

\subsubsection{Existential clauses}
\hypertarget{RefHeading22121935131865}{}
Existential clauses are not very common. Giv\'on (1990:741) names these clauses as one of the main devices for introducing a new topic into a discourse, but in Mauwake they are not used very much in that function (\sectref{sec:9.1.2.1}). Existential clauses use the verb \textstyleStyleVernacularWordsItalic{ik}- `be' as their predicate, and they often contain a locative phrase (\stepcounter{nx}{\thenx}), but it is not necessary (\stepcounter{nx}{\thenx}), (\stepcounter{nx}{\thenx}). 

\ea%x970
\label{ex:x970}
\gll \textstyleEmphasizedVernacularWords{Aaya=ko}  \textstyleEmphasizedVernacularWords{feeke  ik-eya}  nefa  aaw-ep  enim-i-yen. \\
      \\
\glt
\z

sugarcane=NF  here.CF  be-2/3s.DS  2s.ACC  take-SS.SEQ  eat-Np-FU.1p

`If there is (any) sugarcane here, we'll take and eat you (the sugarcane).'

\ea%x971
\label{ex:x971}
\gll Aakisa  Malala  suule  ik-ua,  {\dots} \\
      \\
\glt
\z

now  Malala  school  be-PA.3s

`Now there is the Malala school, {\dots}'

Both the past and future tense forms can be used; the past tense may be used for both present and past meaning. 

\ea%x1068
\label{ex:x1068}
\gll Kuisow  owowa=pa=ko  me  ik-ua. \\
      \\
\glt
\z

one  village=LOC=NF  not  be-PA.3s

`There was/is not even one in the village.'

\ea%x1067
\label{ex:x1067}
\gll Waaya  ika-i-non-(na)  waaya  uup-i-nan. \\
      \\
\glt
\z

pig  be-Np-FU.3s-(TP)  pig  cook-Np-FU.2s

`If there is a pig, you will cook a pig.'

When an existential clause of this type is negated with a negator other than \textstyleStyleVernacularWordsItalic{me}, it becomes a verbless clause (\sectref{sec:5.6.3}).

A special type of existential clause has one of the two location verbs (\sectref{sec:3.8.4.4.3}) as the predicate. These verbs are only used in the past tense, even with the present tense meaning.

\ea%x1154
\label{ex:x1154}
\gll Nomokowa  unowa  \textstyleEmphasizedVernacularWords{fan-e-mik},  aakisa  wia  uruf-i-n.  \\
      \\
\glt
\z

2s/p.brother  many  here-PA-1/3p  now  3p.ACC  see-Np-PR.2s

 `Many of your brothers are here, now you see them.'

\ea%x1155
\label{ex:x1155}
\gll No  niawi  akena  \textstyleEmphasizedVernacularWords{nan-e-k},  no  fain  \\
      \\
\glt
\z

2s.UNM  2s/p.father  real  there-PA-3s  2s.UNM  this

me  nena  niawi  akena=ke.

not  2s.GEN  2s/p.father  real=CF

`Your real father is there, this isn't your real father.'

\subsubsection{Possessive clauses}
\hypertarget{RefHeading22141935131865}{}
Possessive clauses, or so-called `have' clauses, are formed with a dative pronoun and the verb \textstyleStyleVernacularWordsItalic{ik}- `be'. This is a grammaticalization from [+human] locative constructions with the semantic function of goal or locative \citep[50-61]{Heine1997}, as was briefly mentioned in \sectref{sec:3.5.5}.  

The possessee is the patient-of-state subject, which is shown by the fact that it may take the contrastive focus clitic -\textstyleStyleVernacularWordsItalic{ke} (\stepcounter{nx}{\thenx}) and it determines the person inflection on the verb as well (\stepcounter{nx}{\thenx}). The possessor is a \textstyleEmphasizedWords{\textsc{habitive adverbial}}, like  a corresponding construction in Finnish is called \citep[209]{HakulinenEtAl1979}%Karlsson
. Giv\'on calls it a dative object (1984:104), but I prefer to keep the term ``object'' for those arguments in a transitive clause that can take an accusative form when they are [+human].\footnote{\citet[302]{Dixon2010b} calls the initial argument position copula subject (CS) and the second one copula complement (CC), regardless of whether the position is filled by the possessor or the possessee.}

\ea%x595
\label{ex:x595}
\gll Aaya  \textstyleEmphasizedVernacularWords{efar}  \textstyleEmphasizedVernacularWords{ikua},  ifera  wia. \\
      \\
\glt
\z

sugar  1s.DAT  be-PA.3s  salt  no

`I have sugar, but no(t) salt.'

\ea%x1065
\label{ex:x1065}
\gll Apu  maa  epira  marok  maneka=ke  \textstyleEmphasizedVernacularWords{wiar  ik-ua}. \\
      \\
\glt
\z

Apu  food  plate  prawn  big=CF  3.DAT  be-PA.3s

`Apu has/had big prawns on his food plate.' (Lit: `Apu's food plate\textsubscript{Theme} he has/had big prawns.')

\ea%x1323
\label{ex:x1323}
\gll Woos(a)  mua  \textstyleEmphasizedVernacularWords{yiar  ik-e-mik},  wis=ke  eliw  nia \\
      \\
\glt
\z

head  man  1p.DAT  be-PA-1/3p  3p.FC=CF  well  2p.ACC

kaken-i-kuan.

straight-Np-FU.3p

`We have leaders, they can straighten you out.'

Because the possessee is typically inanimate and often indefinite whereas the possessor is human and definite, this causes a violation to the universal discourse-pragmatic principle, according to which animate/human and definite participants tend to precede inanimate and indefinite participants \citep[135]{Heine1997}. In order to follow the principle, Mauwake often makes the possessor a theme by moving the possessor \textstyleAcronymallcaps{NP} to sentence-initial position; only the dative pronoun keeps its position immediately preceding the verb (\stepcounter{nx}{\thenx}). If there is no other possessor \textstyleAcronymallcaps{NP}, an unmarked pronoun is used as a theme (\stepcounter{nx}{\thenx}). In these two sentences, moving part of the \textstyleAcronymallcaps{NP} to the theme position causes the \textstyleAcronymallcaps{NP} to be non-contiguous. In the example (\stepcounter{nx}{\thenx}) the possessee subject \textstyleStyleVernacularWordsItalic{aaya} `sugar' is also the theme, and in (\stepcounter{nx}{\thenx}) the possessee is animate/human, so in those clauses there is less pressure to make the possessor into the theme.

\ea%x973
\label{ex:x973}
\gll [\textstyleEmphasizedVernacularWords{I}]\textsubscript{T}heme  sira  naap  me  \textstyleEmphasizedVernacularWords{yiar  ik-ua}. \\
      \\
\glt
\z

1p.UNM  custom  thus  not  1p.DAT  be-PA.3s

`We do not have a custom like that.'

\ea%x972
\label{ex:x972}
\gll [\textstyleEmphasizedVernacularWords{Mua}  \textstyleEmphasizedVernacularWords{oko}]\textsubscript{T}heme  ona  koor  miira=pa]  [nan]  [waaya  \\
      \\
\glt
\z

man  other  3s.GEN  house  face=LOC  there  pig  

unowa]  \textstyleEmphasizedVernacularWords{wiar  ik-ua}.

many  3.DAT  be-PA.3s

`Another man has many pigs there in front of his house.'

Clauses like the example (\stepcounter{nx}{\thenx}), where the possessed noun is [+human] and [+plural], triggering the plural form of the verb, are quite rare, and it seems that the singular verb form is also becoming possible in these cases:

\ea%x1321
\label{ex:x1321}
\gll Mua  nain  pun  muuka  wiipa  \textstyleEmphasizedVernacularWords{wiar  ik-ua}. \\
      \\
\glt
\z

man  that1  also  son  daughter  3.DAT  ik-PA.3s

`That man also has children/son(s) and daughter(s).'

All the tenses are possible. The past tense form normally covers both present and past meaning.

\ea%x1066
\label{ex:x1066}
\gll Naap  on-i-non=na  pina  \textstyleEmphasizedVernacularWords{wiar  ika-i-non}. \\
      \\
\glt
\z

thus  do-Np-FU.3s=TP  guilt  3.DAT  be-Np-FU.3s

`If he does like that he will have guilt.'

When the present tense form is used, it indicates a more transitory possession:

\ea%x1201
\label{ex:x1201}
\gll Wis  pun  maa  eliwa=ko  wiar  \textstyleEmphasizedVernacularWords{ika-i-ya=na} \\
      \\
\glt
\z

3p.UNM  too  thing  good=NF  3.DAT  be-Np-PR.3s=TP

iw-i-mik.

give.him-Np-PR.1/3p

`They too, if they (happen to) have good things, give to him.'

A possessive clause may be elliptical, with the verb deleted, in cases where the possessed \textstyleAcronymallcaps{NP} has at least one post-modifier, which most commonly is a quantifier.

\ea%x1322
\label{ex:x1322}
\gll Yo  muuka  arow,  wiipa  kuisow  muuta  {\O}. \\
      \\
\glt
\z

1s.UNM  son  three,  daughter  one  only

`I have three sons, (and/but) only one daughter.'

When the possessor is not human, the possessive clause is made with the existential verb \textstyleStyleVernacularWordsItalic{ik}- `be' plus a comitative construction rather than the dative pronoun; and the possessor always precedes the possessee. These are cross-linguistically typical features for the grammaticalization strategy that uses a comitative phrase for a possessive predication \citep[53-57]{Heine1997}. As was noted in \sectref{sec:3.5.4}, in this case the third person singular genitive pronoun \textstyleStyleVernacularWordsItalic{ona} is used for a non-human possessor.

\ea%x1807
\label{ex:x1807}
\gll Parina  ona  wakesim-owa  \textstyleEmphasizedVernacularWords{onaiya}  ika-i-ya. \\
      \\
\glt
\z

lamp  3s.GEN  cover-NMZ  with  be-Np-PR.3s

`The lamp has a cover.'

\ea%x1810
\label{ex:x1810}
\gll Miiwa  ona  mua  \textstyleEmphasizedVernacularWords{onaiya}  ik-ua. \\
      \\
\glt
\z

land  3s.GEN  man  with  be-PA.3s

`The land has its men.' (Each piece of ground ``has'' men whose responsibility it is to see how the land is allocated for gardens.)

Possessive clauses are similar to existential clauses in that when a possessive clause is negated with a negator other than \textstyleStyleVernacularWordsItalic{me}, it becomes a verbless clause (\sectref{sec:5.6.3} and \sectref{sec:6.2.1}).

\subsection{Verbless clauses}
\hypertarget{RefHeading22161935131865}{}
The predicate of a verbless clause belongs to some other phrase class than the verbs. The two subtypes below, equative and descriptive clauses, are very similar syntactically; their differences are mainly in the semantics of the predicates. Their negation strategies are also slightly different from each other. 

\ea%x1036
\label{ex:x1036}
\gll Mua  nain  yena  kae  panewowa=ke. \\
      \\
\glt
\z

man  that1  1s.GEN  1s/p.grandfather  old=CF

`That man is my old grandfather.'

\ea%x1037
\label{ex:x1037}
\gll Waaya  nain  me  maneka,  muuka  kia  gelemuta. \\
      \\
\glt
\z

pig  that1  not  big  son  white  small

`The pig wasn't big, it was a small white piglet.'

In certain cases the verb \textstyleStyleVernacularWordsItalic{ik}- `be' is required as a copula. This happens mainly in the future (\stepcounter{nx}{\thenx}) or sometimes in the past tense, or when the clause requires a medial form to indicate that it is a medial clause (\stepcounter{nx}{\thenx}). 

\ea%x986
\label{ex:x986}
\gll Ikoka  mua  eliwa  ne  mua  oona  ika-i-nan. \\
      \\
\glt
\z

later  man  good  and  man  bone  be-Np-FU.2p

`Later you will be a good and strong man.'

\ea%x987
\label{ex:x987}
\gll No  gelemuta  ik-eya  {\dots} \\
      \\
\glt
\z

2s.UNM  little  be-2/3s.DS

`When you were little, {\dots}'

\subsubsection{Equative and classifying clauses}
\hypertarget{RefHeading22181935131865}{}
Syntactically equative and classifying clauses are identical. The non-verbal predicate typically has contrastive focus marking -\textstyleStyleVernacularWordsItalic{ke}, even though it is not absolutely necessary. 

In an equative clause the subject and the non-verbal predicate have the same reference, so their order can be reversed with the basic meaning staying the same. 

\ea%x975
\label{ex:x975}
\gll Dogimaw  yiena  owow  saria=ke. \\
      \\
\glt
\z

Dogimaw  1p.GEN  village  headman=CF

`Dogimaw is our village headman.'

\ea%x976
\label{ex:x976}
\gll Yiena  owow  saria  Dogimaw(=ke). \\
      \\
\glt
\z

Our  village  headman  Dogimaw(=CF)

`Our village headman is Dogimaw.'

An equative clause is only negated with the verbal negator \textstyleStyleVernacularWordsItalic{me}:

\ea%x1752
\label{ex:x1752}
\gll Dogimaw  me  yiena  owow  saria=ke. \\
      \\
\glt
\z

Dogimaw  not  1p.GEN  village  headman=CF

`Dogimaw is not our village headman.'

In classifying clauses\footnote{\citet[233]{Dryer2007b} calls them ``clauses with a true nominal predicate''.} the reference of the subject is not identical with the reference of the predicate.

\ea%x977
\label{ex:x977}
\gll Yo  inasin  mua=ke. \\
      \\
\glt
\z

1s.UNM  spirit  man=CF

`I am a spirit man.'

\ea%x978
\label{ex:x978}
\gll Oo  Kululu  takira=ke,  o  me  amis-ar-e-k. \\
      \\
\glt
\z

oh  Kululu  young.person=CF  3s.UNM  not  knowledge-INCH-PA-3s

`Oh, Kululu is a youth (compared to us), he doesn't know.'

The classifying clauses are negated with the verbal negator \textstyleStyleVernacularWordsItalic{me} or with a clause-final negator \textstyleStyleVernacularWordsItalic{weetak/wia}.\footnote{Bergh\"all (2006:272) also gives \textit{marew} `no(ne)' as a possible negator for equative clauses, but actually the equative clauses do not use it, only the descriptive clauses.} 

\ea%x980
\label{ex:x980}
\gll Nain  me  inasin  mua=ke,  iperuma=ke. \\
      \\
\glt
\z

that1  not  spirit  man=CF  eel=CF

`That is not a spirit man, it is an eel.'

\ea%x984
\label{ex:x984}
\gll O  somek  mua  weetak/wia. \\
      \\
\glt
\z

he  song  man  no

`He is not a teacher (lit: song man).'

The predicates of both these clauses are more time-stable compared both with verbal predicates and those in the descriptive clauses (Giv\'on 1984:51, Stassen 1997:16). 

\subsubsection{Descriptive clauses}
\hypertarget{RefHeading22201935131865}{}
A descriptive clause is very much like an equative clause, but the predicate is an adjective phrase (\stepcounter{nx}{\thenx}), a noun phrase with an adjective (\stepcounter{nx}{\thenx}), or less frequently a numeral (\stepcounter{nx}{\thenx}) or an adverbial phrase (\stepcounter{nx}{\thenx}). On the time-stability scale these predicates are in between verbal and nominal predicates. 

\ea%x974
\label{ex:x974}
\gll Irak-owa  nain  kekanowa  akena. \\
      \\
\glt
\z

fight-NMZ  that1  strong  very

`The fighting was very fierce.'

\ea%x979
\label{ex:x979}
\gll Yiena  miiwa  kuisow. \\
      \\
\glt
\z

1p.GEN  land  one

`Our land is one.'

\ea%x985
\label{ex:x985}
\gll Nain  pun  sira  naap=iw,  mua  me  kerer-e-mik. \\
      \\
\glt
\z

that1  too  custom  thus=INST  man  not  appear-PA-1/3p

`That, too, was like that: people didn't arrive.'

A descriptive clause can use any of the negation strategies available in Mauwake (\sectref{sec:3.10}, 6.2). 

\ea%x990
\label{ex:x990}
\gll Biiris  me  eliwa,  damo-damola=ko. \\
      \\
\glt
\z

bridge  not  good  RDP-bad=NF

`The bridges were not good, they were bad.'

\ea%x988
\label{ex:x988}
\gll Yo  (mua)  maala  marew. \\
      \\
\glt
\z

1s.UNM  (man)  long  no(ne)

`I am not (a) tall (man).'

\ea%x989
\label{ex:x989}
\gll Awuliak  nain  eliwa  weetak/wia. \\
      \\
\glt
\z

sweet.potato  that1  good  no

`That sweet potato is not good.'

\subsubsection{Negated existential and possessive clauses}
\hypertarget{RefHeading22221935131865}{}
The existential (\sectref{sec:5.5.1}) and possessive clauses (\sectref{sec:5.5.2}) are different from the other verbal clauses with regard to negation. Besides the standard verbal negation (\stepcounter{nx}{\thenx}) they can use all the other negators as well (\sectref{sec:6.2.1}). The verb \textstyleStyleVernacularWordsItalic{ik}- `be' is retained only with the verbal negator \textstyleStyleVernacularWordsItalic{me}. With all the other negators the negator itself replaces the verb, and the clause becomes a verbless clause:

\ea%x982
\label{ex:x982}
\gll Iiriw  miiwa  muuta  nain  irak-owa  \textstyleEmphasizedVernacularWords{marew}. \\
      \\
\glt
\z

earlier  land  for  that1  fight-NMZ  no(ne)

`Earlier there was no fighting for land.'

\ea%x983
\label{ex:x983}
\gll I  urupa  \textstyleEmphasizedVernacularWords{weetak},  i  soomia  \textstyleEmphasizedVernacularWords{wia},  i  \\
      \\
\glt
\z

1p.UNM  cup  no  1p.UNM  spoon  no  1p.UNM

epira  \textstyleEmphasizedVernacularWords{marew}.

plate  no(ne)

`We had no cups, no spoons, no plates.'

\subsection{Nominalized clauses}
\hypertarget{RefHeading22241935131865}{}
Lexical nominalization, where an action nominal is a regular noun, was discussed in \sectref{sec:3.2.6.1}; in this section nominalization as an operation on the whole clause is described.

Action nominals and infinitives are usually assumed to be two separate non-finite categories.\footnote{Ylikoski (2003, 2009) discusses the similarities and differences between various non-finite verb forms and presents insightful definitions based mainly on their syntactic functions. Many of the details are not relevant to Mauwake, however, as there are no verb forms that would easily fit under the categories of converbs or participles in Mauwake, and because it seems that infinitives and action nominals may be collapsible into one category.} Cross-linguistically, the two often tend to be identical in form \citep[224]{Ylikoski2003}, and there is apparently a separate tendency for their functions to look rather similar as well (ibid. 196-197). It seems that the origin of the infinitive in many languages is in a nominalized verb \citep[69]{Noonan2007}. 

In Mauwake there is just one form, and rather than positing two homonymous forms with different functions, I maintain that action nominals function both like prototypical nouns or adjectives \textstyleEmphasizedWords{\textsc{and}} in functions typically associated with infinitives: as complements\footnote{Ylikoski widens the definition of complement to cover ``obligatory and argumental adverbials as well'' (2003:209).} of certain verbs, in goal/purpose and deontic structures among others.  

Structurally there are two kinds of nominalized clauses in Mauwake. They may occur as complements of the same verbs, with somewhat different semantics. The first type is what the term \textstyleEmphasizedWords{\textsc{nominal}}\textstyleEmphasizedWords{\textsc{ized clause}} most commonly refers to: the verbal predicate of a clause is nominalized, and consequently the whole clause becomes a noun phrase. The second type retains the form of a verbal clause, but the distal deictic \textstyleStyleVernacularWordsItalic{nain} `that' after the finite verb nominalizes it.  The first type has a wider distribution.

\subsubsection{Type1: with a nominalized verb}
\hypertarget{RefHeading22261935131865}{}
When verbs are nominalized, the action or event referred to still keeps some of its verbal characteristics (\textstyleBibliogBaseChar{Hopper and Thompson} 1985:177). Languages differ as to how verbal or nominal in character their nominalized verbs are, and also within one language the outcomes of different nominalization strategies may vary in regard to this \citep[344]{ComrieEtAl2007}%Thompson
. In this respect Mauwake is a very \textstyleEmphasizedWords{verbal} language: the nominalized verbs retain a number of their verbal characteristics.

Neutralization of tense or aspect distinctions, as well as the loss of other than just one argument are common features associated with nominalization (\textstyleBibliogBaseChar{Hopper and} \textstyleBibliogBaseChar{Thompson} 1984:737-738). In Mauwake the nominalized verb forms may keep all of the derivational suffixes but not the inflectional ones, which include tense and person/number marking.

\ea%x1226
\label{ex:x1226}
\gll Aakisa=ko  me  kerer-em-ika-i-ya,  wia  bala \\
      \\
\glt
\z

now=NF  not  appear-SS.SIM-be-Np-PR.3s  3p.ACC  decoration

op-aw-ap  wia  \textstyleEmphasizedVernacularWords{wiim-om-owa}  nain.\footnote{The long subject NP consisting of a nominalized clause has been right-dislocated.}

hold-CAUS-SS.SEQ  3p.ACC  escort-BEN-NMZ  that1

`Now it doesn't take place (any more), decorating them and escorting them for them (i.e. escorting girls to their prospective husbands).'

Verbal groups showing aspect may be nominalized as well, so the aspectual distinction is retained: 

\ea%x1841
\label{ex:x1841}
\gll [Mua  papako  maa  \textstyleEmphasizedVernacularWords{ik-em-ik-owa}]  nain  kawus  wiar \\
      \\
\glt
\z

man  some  food  roast-SS.SIM-be-NMZ  that1  smoke  3.DAT

uruf-i-kuan.

see-Np-FU.3p

`They will see the smoke from some men's roasting of food.'

The nominalized verb in itself is neutral in regard to modality, even if it often gets deontic interpretation. But it can be, and frequently is, used in cases where modality is intentionally left unspecified. In (\stepcounter{nx}{\thenx}) the reason for not coming may be that one is not allowed, or able, or willing, to come. 

\ea%x1257
\label{ex:x1257}
\gll Yo  \textstyleEmphasizedVernacularWords{ekap-owa}  wia. \\
      \\
\glt
\z

1s.UNM  come-NMZ  no

`I won't come.'

But note (\stepcounter{nx}{\thenx}) where the contrastive focus marker added to the nominalized verb forces a deontic interpretation. See also \sectref{sec:6.1.2}. 

The nominalized verb can keep all the arguments and peripherals that a corresponding finite verb would have. This sometimes results in very long noun phrases. (In the following example there is lexical nominalization of the verb \textstyleStyleVernacularWordsItalic{kookal}- `like' as well, besides the clausal nominalization.)

\ea%x1234
\label{ex:x1234}
\gll [\textstyleEmphasizedVernacularWords{Manin(a)  maneka,  ekina,  naisow  nena   \\
      \\
\glt
\z

garden    big    ekina  2s.ISOL  2s.GEN  

\textstyleEmphasizedVernacularWords{kookal-owa}\textstyleEmphasizedVernacularWords{=pa} \textstyleEmphasizedVernacularWords{perek-owa}]  weetak.

like-NMZ=LOC   pull.out-NMZ  no

`You are not allowed to harvest the big garden, called ekina, at your own liking.'

In the following two examples only the nominalized verb is within the scope of the negation. The nominalized clauses are in brackets.

\ea%x1235
\label{ex:x1235}
\gll [Maa  eneka  \textstyleEmphasizedVernacularWords{me  en-owa}]  maa  marew. \\
      \\
\glt
\z

thing  tooth  not  eat-NMZ  thing  none

`Not eating meat is all right / is not an issue.'

\ea%x1236
\label{ex:x1236}
\gll Wi  mua  [naap  \textstyleEmphasizedVernacularWords{me  on-owa}] \textstyleEmphasizedVernacularWords{} nain=ko  ik-e-mik=i? \\
      \\
\glt
\z

3p.UNM  man  thus  not  do-NMZ  that1=NF  be-PA-1/3p=QM

`Are there people who wouldn't do / keep doing like that?'

Any of the four negators (\sectref{sec:6.2.1}) may be used to negate the nominalized clause (\stepcounter{nx}{\thenx})-(\stepcounter{nx}{\thenx}). 

Cross-linguistically nominalized clauses also vary as to whether they retain a manner adverbial of the corresponding verbal clause or change it into an adjective \citep[374]{ComrieEtAl2007}%Thompson
. Mauwake keeps the adverbial: 

\ea%x1237
\label{ex:x1237}
\gll [Wiena  teeria  \textstyleEmphasizedVernacularWords{baliwep  wia  kakalt-owa}]  sira  nain  \\
      \\
\glt
\z

3p.GEN  family  well  3p.ACC  look.after-NMZ  custom  that1

wia  maak-e-k.

3p.ACC  tell-PA-3s

`He talked to them about the custom of looking after their families well.'

One common feature in nominalized clauses is that the arguments, instead of taking the morphology they would have in a finite clause, tend to follow typical \textstyleAcronymallcaps{NP} morphology in their marking \citep[738]{HopperEtAl1984}%Thompson
. This is perhaps clearest with the subject, which in many languages gets possessive/genitive marking in a nominalized clause. In Mauwake this criterion is not very helpful. The pronominal subject of this first type of nominalized clause, if present, is often genitive (\stepcounter{nx}{\thenx}), but may be nominative as well. But also the subject of a finite clause can be nominative or genitive in form, depending on whether it is neutral or emphatic; and if the subject of a nominalized clause is also the theme, it is nominative rather than genitive (\stepcounter{nx}{\thenx}). A pronominal object in a nominalized clause is in the accusative (\stepcounter{nx}{\thenx}). 

\ea%x1228
\label{ex:x1228}
\gll \textstyleEmphasizedVernacularWords{Yiena  owow  maneka  ikiw-owa  nain}  ma-i-yem. \\
      \\
\glt
\z

1p.GEN  village  big  go-NMZ  that1  say-Np-PR.1s

`I am talking about our going to town.'

The nominalized verb may take an adjective modifier:

\ea%x1240
\label{ex:x1240}
\gll \textstyleEmphasizedVernacularWords{Kema  suuw-owa  eliwa}  aaw-ep  kekan-e-k. \\
      \\
\glt
\z

liver  push-NMZ  good  get-SS.SEQ  get.strong-PA-3s

`He got good thinking and became strong.'

Another structural indicator of the nominal status of a nominalized clause is the focus marking, which can be attached to the verb. 

\ea%x1238
\label{ex:x1238}
\gll I  uuw-owa  yi-iwkin  \textstyleEmphasizedVernacularWords{baliwep  uuw-owa=ke} \\
      \\
\glt
\z

1p.UNM  work-NMZ  give.us-2/3p.DS  well  work-NMZ=CF

ik-ua.

be-PA.3s

`When they give us work, working well is our duty.'

Nominalized clauses, like other noun phrases, use the far deictic \textstyleStyleVernacularWordsItalic{nain} `that' as a determiner. 

\ea%x1842
\label{ex:x1842}
\gll [\textstyleEmphasizedVernacularWords{Ona}  \textstyleEmphasizedVernacularWords{epa  maneka  or-owa}]  \textstyleEmphasizedVernacularWords{nain}  fofa=pa  ...  unow-iya \\
      \\
\glt
\z

3s.GEN  place  big  descend-NMZ  that1  day=LOC  {\dots}  many=COM

taan-e-mik.

fill-PA-1/3p

`On the day of his coming down to the big place {\dots} all of them filled (the place).'

\ea%x1843
\label{ex:x1843}
\gll [\textstyleEmphasizedVernacularWords{Niena}  \textstyleEmphasizedVernacularWords{waaya  mik-owa}]  \textstyleEmphasizedVernacularWords{nain}  on-ami  kuep-i-man, \\
      \\
\glt
\z

2p.GEN  pig  spear-NMZ  that1  do-SS.SIM  break-Np-PR.2p

niena  maa=ke,  niena  wiowa=ke.

2p.GEN  thing=CF  2p.GEN  spear=CF

`If you break (the spears) (while) doing your pig-hunting (lit: pig-spearing), that is your business, they are your spears.'

An interesting structure, and not much described in Papuan languages, is one where a same-subject medial clause is in the scope of the nominalization. In Mauwake this tends to happen when the medial verb shares an object with the following verb and there is no, or very little, intervening material between the verbs. 

\ea%x1885
\label{ex:x1885}
\gll Dabe  wiawi  [\textstyleEmphasizedVernacularWords{maa  ik-ep  en-owa]}  na-ep  \\
      \\
\glt
\z

Dabe  3s/p.father  food  roast-SS.SEQ  eat-NMZ  say-SS.SEQ

manin(a)  koora  iw-a-k.

garden  house  go-PA-3s

`Dabe's father wanted to roast and eat food and went to the garden house.'

\ea%x1845
\label{ex:x1845}
\gll Toiyan  iiriw  maak-ep-pu-a-mik,  [\textstyleEmphasizedVernacularWords{uuriw  yia \\
      \\
\glt
\z

Toiyan  already  tell-SS.SEQ-CMPL-PA-1/3p  morning  1p.ACC  

\textstyleEmphasizedVernacularWords{aaw-ep}  \textstyleEmphasizedVernacularWords{Madang  ikiw-owa}] \textstyleEmphasizedVernacularWords{} nain.

take-SS.SEQ  Madang  go-NMZ  that1

`We already told Toiyan about taking us in the morning and going to Madang.'

Medial clauses preceding nominalized clauses do not automatically fall within the scope of the nominalization. Just looking at the linguistic form it would be possible to analyse the following examples so that the medial clause is outside the nominalization. In that case the free translation of (\stepcounter{nx}{\thenx}) would be `Having worked on the garden alone it is not acceptable to leave it there', and (\stepcounter{nx}{\thenx}) `Hold the planting stick and keep practising the making of planting holes'. But culturally these alternative interpretations are not valid. Even starting to work on a big garden without previous negotiations and proper rituals is not acceptable, and the holding of the planting stick and making planting holes form a cultural `expectancy chain' and belong together conceptually.

\ea%x1227
\label{ex:x1227}
\gll [\textstyleEmphasizedVernacularWords{Manina}  \textstyleEmphasizedVernacularWords{waisow}  \textstyleEmphasizedVernacularWords{mauw-ap  neeke \\
      \\
\glt
\z

garden  3s.ISOL  work-SS.SEQ  there.CF  

\textstyleEmphasizedVernacularWords{wafur-ap-pu-owa}]  nain  weetak.

throw-SS.SEQ-CMPL-NMZ  that1  no

`Working on the garden alone and leaving it there (=without proper rituals) is not (acceptable/customary).'

\ea%x1239
\label{ex:x1239}
\gll [\textstyleEmphasizedVernacularWords{Weria  op-ap}  \textstyleEmphasizedVernacularWords{wiinar-owa}] \textstyleEmphasizedVernacularWords{} nain  \\
      \\
\glt
\z

planting.stick  hold-SS.SEQ  make.planting.holes-NMZ  that1

akim-am-ik-e.

try-SS.SIM-be-IMP.2s

`Keep practising the making of planting holes with the planting stick.'

In the following example (\stepcounter{nx}{\thenx}) the medial clause has to be within the scope of the nominalization for the sentences to make sense. The speaker had seen a possum in a tree and would have liked to shoot it, but since he had not taken his bow and arrows with him, he did not climb up the tree either.

\ea%x1844
\label{ex:x1844}
\gll [\textstyleEmphasizedVernacularWords{Nomokowa}  \textstyleEmphasizedVernacularWords{ir-ap  mik-owa}]  nain  yena  amia  me \\
      \\
\glt
\z

tree  climb-SS.SEQ  shoot-NMZ  that  1s.GEN  bow  not  

aaw-e-m.

take-PA-1s

`(For) climbing the tree and shooting (an animal), I hadn't taken my bow (and arrows).'

An intervening overt object may block a same-subject medial clause from being within the scope of a following nominalized verb:

\ea%x1886
\label{ex:x1886}
\gll [Irak-ep]  \textstyleEmphasizedVernacularWords{luuwa}  niir-owa  piipu-a-mik. \\
      \\
\glt
\z

fight-SS.SEQ  ball  play-NMZ  leave-PA-1/3p

`We fought and stopped (lit: left) playing football.'

A different-subject medial clause does not fall within the scope of a nominalized verb. 

The nominalized clause has several different functions. Like any other noun phrase, it may function as an argument (\stepcounter{nx}{\thenx}) or in the periphery of a clause (\stepcounter{nx}{\thenx}), or in another noun phrase (\stepcounter{nx}{\thenx}).

\ea%x1230
\label{ex:x1230}
\gll [\textstyleEmphasizedVernacularWords{Epia}  \textstyleEmphasizedVernacularWords{wilin-owa}]\textsubscript{O}  uruf-ap  bom  yia  \\
      \\
\glt
\z

firewood  shine-NMZ  see-SS.SEQ  bomb  1p.ACC

wafur-om-i-kuan  na-e-mik.

throw-BEN-Np-FU.3p  say-PA-1/3p

`They\textsubscript{i}/we said that when they\textsubscript{j} see the light (lit: shining) of the fire they\textsubscript{j} will throw bombs at us.'

\ea%x1241
\label{ex:x1241}
\gll Wiena  oram  niir-emi  [\textstyleEmphasizedVernacularWords{wiam  kookal}\textstyleEmphasizedVernacularWords{-}\textstyleEmphasizedVernacularWords{owa=pa}]\textsubscript{Advl} \\
      \\
\glt
\z

3p.GEN  just  play-SS.SEQ  3p.REFL  like-NMZ=LOC

nan  wiam  aaw-i-mik.

there  3p.REFL  take-Np-PR.3s

`They just play together and (on the basis of) liking each other they marry each other.'

\ea%x1229
\label{ex:x1229}
\gll [[\textstyleEmphasizedVernacularWords{garanga}  \textstyleEmphasizedVernacularWords{oko  muuka  wiar  aaw-owa}]\textsubscript{NP}  sira]\textsubscript{NP} \\
      \\
\glt
\z

family  other  son/child  3.DAT  get-NMZ  custom

`the adoption custom' (Lit: the custom of getting a child from another family')

The following functions are often associated with infinitives in languages that distinguish between infinitives and nominalizations \citep[207]{Ylikoski2003}. 

Expressions of obligation (\sectref{sec:6.1.2}) use the nominalized form of the main verb. It is followed by the contrastive focus clitic, when it is either a non-verbal predicate (\stepcounter{nx}{\thenx}) or the subject of the verb \textstyleStyleVernacularWordsItalic{ikua} `is'(\stepcounter{nx}{\thenx}).

\ea%x1242
\label{ex:x1242}
\gll Yo  uurika  owow  maneka  \textstyleEmphasizedVernacularWords{ikiw-owa=ke}. \\
      \\
\glt
\z

1s.UNM  tomorrow  village  big  go-NMZ=CF

`I have to go to town tomorrow.'

\ea%x1243
\label{ex:x1243}
\gll Wi  iperowa  ekima  wia  op-ap  \textstyleEmphasizedVernacularWords{baliwep \\
      \\
\glt
\z

3p.UNM  middle-aged  forehead  3p.ACC  hold-SS.SEQ  well

\textstyleEmphasizedVernacularWords{ik-owa=ke  ik-ua}.

be-NMZ=CF  be-PA.3s

`One has to respect\footnote{The verbal expression for respecting someone is \textit{ekima opowa} `holding someone's forehead'.} the middle-aged/elderly and behave well.'

\ea%x1244
\label{ex:x1244}
\gll Inasina  wia  patir-a-mik  nain  \textstyleEmphasizedVernacularWords{me  wiar \\
      \\
\glt
\z

bush.spirit  3p.ACC  sacrifice-PA-1/3p  that1  not  3.DAT  

\textstyleEmphasizedVernacularWords{en-owa=ke}.

eat-NMZ=CF

`One must not eat what has been sacrificed to the bush spirits.'

Directional verbs (\sectref{sec:3.8.4.4.5}) may take a nominalized clause as the goal. In many of these cases it is hard to distinguish between goal and purpose, which can be expressed via nominalization as well.

\ea%x1245
\label{ex:x1245}
\gll Yo  \textstyleEmphasizedVernacularWords{emeria  aaw-owa}  urup-e-m. \\
      \\
\glt
\z

1s.UNM  woman  take-NMZ  ascend-PA-1s

`I came up to take my wife.'

Nominalized clauses are used as complements of various complement-taking verbs (\sectref{sec:8.3.2}). 

\ea%x1246
\label{ex:x1246}
\gll \textstyleEmphasizedVernacularWords{Miiw-aasa  muf-owa  ikiw-owa } na-em-ik-omkun  \\
      \\
\glt
\z

land-canoe  pull-NMZ  go-NMZ  say-SS.SIM-be-1s/p.DS

o  ar-e-k.

3s.UNM  die-PA-3s

`As we were talking about going to get a vehicle, she died.'

\ea%x1248
\label{ex:x1248}
\gll \textstyleEmphasizedVernacularWords{Maa  uup-owa}  paayar-ep  ep-a-n. \\
      \\
\glt
\z

food  cook-NMZ  know-SS.SEQ  come-PA-2s

`You know cooking and you came.'

A nominalized clause is sometimes used to express habituality. It indicates a more deliberate and permanent habit than that expressed by the continuous aspect, which is the default marking for the habitual (\sectref{sec:3.8.5.1.1.2}). 

\ea%x1249
\label{ex:x1249}
\gll Wi  mua  \textstyleEmphasizedVernacularWords{naap  me  onowa  nain=ko}  ik-e-mik=i? \\
      \\
\glt
\z

3p.UNM  man  thus  not  do-NMZ  that1=NF  be-PA-1/3p=QM

`Are there people who wouldn't keep doing like that?'

\ea%x1250
\label{ex:x1250}
\gll Mua  papako  \textstyleEmphasizedVernacularWords{opor(a)  makena  me  ookowa},  sira  samora  \\
      \\
\glt
\z

man  other  talk  true  not  follow-NMZ  custom  bad

on-am-ika-i-mik.

do-SS-SIM-be-Np-PR.1/3p

`Some people (as a rule) do not follow the true talk (but) keep doing bad things.'

Mauwake verbs may take a causative suffix, which often indicates causation (\sectref{sec:3.8.4.3.1}). When the causation is less mechanical and requires the cooperation of the object of the causation, the verb \textstyleStyleVernacularWordsItalic{suuw}- `push' is used together with a nominalized clause:

\ea%x1252
\label{ex:x1252}
\gll Mua  naareke  \textstyleEmphasizedVernacularWords{naap  on-owa}  nefa  suuw-a-k? \\
      \\
\glt
\z

man  who.CF  thus  do-NMZ  2s.ACC  push-PA-3s

`Who made you do like that?'

Ability is expressed via a nominalized clause followed by the intensity adverb \textstyleStyleVernacularWordsItalic{pepek} `enough, able'.

\ea%x1251
\label{ex:x1251}
\gll \textstyleEmphasizedVernacularWords{Ariwa  perek-owa}  me  pepek. \\
      \\
\glt
\z

arrow  pull.out-NMZ  not  enough/able

`(He was) not able to pull out the arrow.'

One strategy for purposives is to use the nominalized form of the main verb followed by the same-subject sequential form \textstyleStyleVernacularWordsItalic{naep} of the verb `say/think'. This strategy is used especially when the purpose understood to be somewhat generic (\stepcounter{nx}{\thenx}) or when the purpose clause is right-dislocated (\stepcounter{nx}{\thenx}). For purpose clauses, see \sectref{sec:8.3.2.1.4}.

\ea%x1253
\label{ex:x1253}
\gll Weniwa=pa  \textstyleEmphasizedVernacularWords{en-owa  na-ep}  uuw-i-mik. \\
      \\
\glt
\z

hunger.time=LOC  eat-NMZ  say-SS.SEQ  plant-Np-PR.1/3p

`We/they work in order to eat during the time of hunger.'

\ea%x1255
\label{ex:x1255}
\gll Ona  siowa  ikos  manina  ikiw-e-mik,  \textstyleEmphasizedVernacularWords{pika \\
      \\
\glt
\z

3s.GEN  dog  together.with  garden  go-PA-1/3p  fence

\textstyleEmphasizedVernacularWords{on}\textstyleEmphasizedVernacularWords{-}\textstyleEmphasizedVernacularWords{owa  na}\textstyleEmphasizedVernacularWords{-}\textstyleEmphasizedVernacularWords{ep}.

make-NMZ  say-SS.SEQ

`He went together with his dog to the garden (in order) to make a fence.'

Mauwake has an idiosyncratic clausal structure for the expression `not yet'. The negated verb is nominalized, and it is followed by an appropriate form of the verb \textstyleStyleVernacularWordsItalic{ik}- `be'.  The presence of the negative temporal adverb \textstyleStyleVernacularWordsItalic{eewuar} `not yet' indicates expectation that what hasn't happened yet will, or should, take place in not too distant future.  

\ea%x1254
\label{ex:x1254}
\gll Aakisa  baliwep  \textstyleEmphasizedVernacularWords{me  amis-ar-owa  ik-e-mik}. \\
      \\
\glt
\z

now  well  not  knowledge-INCH-NMZ  be-PA-1/3p

`Now we do not yet know it well.'

\ea%x1256
\label{ex:x1256}
\gll Iwera  popoka  wafur-am-ika-iwkin  or-op  `bulak',  \\
      \\
\glt
\z

coconut  unripe  throw-SS.SIM-be-2/3p.DS  descend-SS.SEQ  plop

eewuar,  eka  \textstyleEmphasizedVernacularWords{me  saan-ar-owa  ik-ua}.

not.yet  water  not  dry-INCH-NMZ  be-PA.3s

`They\textsubscript{i} kept throwing unripe coconuts\textsubscript{j} and when they\textsubscript{j} dropped they\textsubscript{j} said `plop' (so they\textsubscript{i} knew:) not yet, the water had not dried yet.'

Unlike many other languages, in Mauwake a nominalized clause does not function as a complement of an adjective. Rather, the nominalized clause functions as the subject and it takes the adjective as a non-verbal predicate: 

\ea%x1258
\label{ex:x1258}
\gll \textstyleEmphasizedVernacularWords{Maa  wiar  ikum  aaw-owa}  eliwa=ki? \\
      \\
\glt
\z

thing  3.DAT  illicitly  take-NMZ  good=CF.QM

`Is it good to steal?'

\ea%x1259
\label{ex:x1259}
\gll \textstyleEmphasizedVernacularWords{Galasim-owa}\footnote{The verb for spear-fishing is a loan from Tok Pisin, which refers to the goggles used when diving to spear fish.}  lawisiw  yoowa. \\
      \\
\glt
\z

spear.fish-NMZ  rather  hot/hard

`Spearing fish is rather hard.' Or: `It is rather hard to spear fish.'

\subsubsection{Type 2: with a finite verb}
\hypertarget{RefHeading22281935131865}{}
The second strategy for nominalizing a clause is to end an ordinary verbal clause with the far demonstrative \textstyleStyleVernacularWordsItalic{nain} `that' used as a determiner. The demonstrative is the only element marking the clause as nominalized. Comrie and Thompson call this type ``clausal nominalization'' (2007:376-377). Giv\'on (1990:506) suggests that there may be a correlation ``between the \textstyleEmphasizedWords{\textsc{degree of nounhood}} of a nominalized expression and its ability to take determiners''. In Mauwake this is clearly not the case, as the demonstrative is obligatory in this second type of nominalized clause but only optional in the first type, which is otherwise more like a \textstyleAcronymallcaps{NP}.

The distribution of finite clauses nominalized only with a demonstrative is more restricted than that of clauses with a nominalized verb. They function as relative clauses (\sectref{sec:8.3.1}), complement clauses (\sectref{sec:8.3.2}), or temporal clauses (\sectref{sec:8.3.3.1}), but not in the many other specific functions where the other type can occur. Forming complement clauses and relative clauses by adding a demonstrative pronoun after a finite verb is a common strategy in Papuan languages (Reesink 1983b and 1987:228, Farr 1999:77, Whitehead 2004:192). 

\ea%x1260
\label{ex:x1260}
\gll [\textstyleEmphasizedVernacularWords{Takira}  \textstyleEmphasizedVernacularWords{en-ow}(\textstyleEmphasizedVernacularWords{a})  \textstyleEmphasizedVernacularWords{gelemuta  wia  on-om-a-mik} \\
      \\
\glt
\z

child  eat-NMZ  small  3p.ACC  make-BEN-BNFY2.PA-1/3p

\textstyleEmphasizedVernacularWords{nain}]\textsubscript{CC}  ma-i-yem.

that1  say-Np-PR.1s

`I tell about our making/having made a feast for the children.'

\ea%x1261
\label{ex:x1261}
\gll [\textstyleEmphasizedVernacularWords{Akia}  \textstyleEmphasizedVernacularWords{ik-e-k  nain}]\textsubscript{RC}  me  en-e-k. \\
      \\
\glt
\z

banana  roast-PA-3s  that1  not  eat-PA-3s

`He did not eat the bananas that he roasted.'

\ea%x1940
\label{ex:x1940}
\gll [\textstyleEmphasizedVernacularWords{Koora}  \textstyleEmphasizedVernacularWords{ikiw}\textstyleEmphasizedVernacularWords{-}\textstyleEmphasizedVernacularWords{i}\textstyleEmphasizedVernacularWords{-}\textstyleEmphasizedVernacularWords{mik  nain}]\textsubscript{TempC}  mera  eka  me  enim-i-mik. \\
      \\
\glt
\z

house  go-Np-PR.1/3p  that1  fish  water  not  eat-Np-PR.1/3p

`When/After we go into the house, we do not eat fish soup.'

\section{Functional domains}
\hypertarget{RefHeading22301935131865}{}
This chapter describes various functional systems that affect the clause or the sentence as a whole.  Most of them are touched upon in various other parts of the grammar where they are relevant, but here they are treated in a more systematic manner.

\subsection{Modality}
\hypertarget{RefHeading22321935131865}{}
Modality, or mode -- expressing the speaker's attitude to a situation -- relates not just to the verb but to the whole proposition. Because of this it is typically not expressed via verbal inflection \citep[22]{Bybee1985}.  In Mauwake the counterfactual modality is manifested by a suffix on the verb (\sectref{sec:3.8.3.2}); more often the modality is conveyed via various other strategies outlined below.

Many Papuan languages make a distinction between realis and irrealis mode,\footnote{\citet[158]{Foley1986} calls it \textit{status}.} and tense. \citet[162]{Foley1986} estimates that, on the whole, tense is more prominent than mode, but there are also languages like Hua \citep{Haiman1980} and Maia \citep{Hardin2002} which do not have tense as a verbal category at all, only mode. But in Mauwake the realis-irrealis dichotomy is not grammatically relevant. 

\subsubsection{Epistemic modality}
\hypertarget{RefHeading22341935131865}{}
Epistemic modality has to do with certainty, probability and possibility: it ``relates to the speaker's {\dots} commitment to the probability that the situation is true'' \citep[234]{Payne1997}.  

The default  and unmarked mood in statements is indicative, when something is stated as a fact. If the speaker wants to strengthen the proposition more, the intensity adverb \textstyleStyleVernacularWordsItalic{akena} `truly, very' is added to the end of the statement.

\ea%x1050
\label{ex:x1050}
\gll Wi  owow  oko  oko  pun  wia  maake-miaw-i-yem  \\
      \\
\glt
\z

3p.UNM  place  other  other  also  3p.ACC  tell-wander-Np-PR.1s

\textstyleEmphasizedVernacularWords{akena}.

truly

`I really walk around telling people in many other places too.'

\ea%x1051
\label{ex:x1051}
\gll Wi  o  ook-owa  nain  me  pepek  \textstyleEmphasizedVernacularWords{akena}. \\
      \\
\glt
\z

3p.UNM  3s.UNM  follow.him-NMZ  that1  not  able  truly

`They really are not able to follow him.'

When the proposition is considered less than certain, either probable or just possible, the modal adverb clitic -\textstyleStyleVernacularWordsItalic{yon} `probably/perhaps/I think' (\sectref{sec:3.9.3}) is attached to the last word in the statement, usually either a verb or non-verbal predicate. An interjection can still follow the word with -\textstyleStyleVernacularWordsItalic{yon}.

\ea%x1052
\label{ex:x1052}
\gll Mua  Maneka=ke  lawisiw  wia  amukar-e-k=\textstyleEmphasizedVernacularWords{yon}. \\
      \\
\glt
\z

man  big=CF  somewhat  3p.ACC  scold-PA-3s-perhaps

`Perhaps God reproached/punished them a little.'

\ea%x1053
\label{ex:x1053}
\gll Nis  pun  kema  puk-owa  marewa=ke=\textstyleEmphasizedVernacularWords{yon}  aa! \\
      \\
\glt
\z

2p.UNM  also  liver  burst-NMZ  none=CF-perhaps  EXC

`Ah, I think you don't have any sense at all (lit: your liver hasn't burst)!'

\ea%x1071
\label{ex:x1071}
\gll Naap=\textstyleEmphasizedVernacularWords{yon}. \\
      \\
\glt
\z

thus-perhaps

`I think/suppose it is like that.'

The counterfactual form of the verb (\sectref{sec:3.8.3.2}) is used in speculative statements where the situation mentioned in the proposition \textstyleEmphasizedWords{\textsc{did not}} happen, although it could have. 

\ea%x1054
\label{ex:x1054}
\gll Lawisiw  akena  um-\textstyleEmphasizedVernacularWords{ek}-a-m. \\
      \\
\glt
\z

somewhat  very  die-CNTF-PA-1s

`I very nearly fell (but in reality didn't).'

\ea%x1055
\label{ex:x1055}
\gll Yena  kookal-owa=pa  uuw-\textstyleEmphasizedVernacularWords{ek}-a-m=na  sesa  na-ek-a-m. \\
      \\
\glt
\z

1s.GEN  like-NMZ=LOC  work-CNTF-PA-1s=TP  price  say-CNTF-PA-1s

`If I had worked on my own will, I would have required payment.'

\ea%x1056
\label{ex:x1056}
\gll Naap  wiar  amis-ar-\textstyleEmphasizedVernacularWords{ek}-a-mik  oo! \\
      \\
\glt
\z

thus  3.DAT  knowledge-INCH-CNTF-PA-1/3p  EXC

`Oh, if only we had known that about him/them!'

Abilitative is expressed by the adverb \textstyleStyleVernacularWordsItalic{pepek} `enough, correctly, able' as a non-verbal predicate. In affirmative statements the verb that the adverb refers to often occurs in the following clause:

\ea%x1089
\label{ex:x1089}
\gll No  \textstyleEmphasizedVernacularWords{pepek},  eliw  on-i-nan. \\
      \\
\glt
\z

2s.UNM  able  well  do-Np-FU.2s

`You are able, you can do it.'

The verb may also be in the same clause but in the nominalized form; this is more common in negative statements:

\ea%x1088
\label{ex:x1088}
\gll {\dots  mukuna  nain } \textstyleEmphasizedVernacularWords{umuk-owa}  \textstyleEmphasizedVernacularWords{me  pepek}. \\
      \\
\glt
\z

...  fire  that1  extinguish-NMZ  not  able

`{\dots} (we were) not able to extinguish the fire.'

Evidentials are an important feature in some Papuan languages, but Mauwake does not have them as a grammatical category.

\subsubsection{Deontic modality}
\hypertarget{RefHeading22361935131865}{}
The deontic modality indicates obligation or permission. Deontic expressions can vary from a statement of a strong obligation to a polite request or to expressions of permission or denying permission. 

The syntactic strategy for expressing strong obligation is to use the nominalized form of the verb followed by the contrastive focus clitic, and optionally an appropriate form of the verb `be'. 

\ea%x1077
\label{ex:x1077}
\gll Yo  uurika  owow  maneka  \textstyleEmphasizedVernacularWords{ikiw-owa=ke  (ik-ua)}. \\
      \\
\glt
\z

1s.UNM  tomorrow  village  big  go-NMZ=CF  be-PA.3s

`I have to go to town tomorrow.'

A nominalized clause structure may may be interpreted to express obligation even without the contrastive focus clitic, and in a medial clause. A dative pronoun is added if clarification is needed to state who is obligated to do something. 

\ea%x1079
\label{ex:x1079}
\gll \textstyleEmphasizedVernacularWords{Ekap-owa  efar  ika-eya}  ekap-e-m. \\
      \\
\glt
\z

come-NMZ  1s.DAT  be-2/3s.DS  come-PA-1s

`I had to come, so I came.'

A polite request can also take the form of a question.

\ea%x1163
\label{ex:x1163}
\gll No  maa  nain=ko  eliw  yi-i-nan=i? \\
      \\
\glt
\z

2s.UNM  thing  that=NF  well  give.me-Np-FUT.2s=QM

`Will/would you give that to me (please)?'

Permission is indicated by the adverb \textstyleStyleVernacularWordsItalic{eliw} `well/all right' placed before the verb, which is in the future form.

\ea%x1085
\label{ex:x1085}
\gll Yiena  miiwa  kuisow,  \textstyleEmphasizedVernacularWords{eliw}  feeke  soop-i-yen. \\
      \\
\glt
\z

1p.GEN  land  one  well  here.CF  bury-Np-FU.1p

`Our land is one, we may bury him here.'

\ea%x1086
\label{ex:x1086}
\gll \textstyleEmphasizedVernacularWords{Eliw}  ek-ap  fook-i-nan,  fook-ap  ep-i-nan. \\
      \\
\glt
\z

well  come-SS.SEQ  split-Np-FU.2s  split-SS.SEQ  go-Np-FU.2s

`You may come and split (coconuts), and having split them, go.'

Prohibition or denial of permission is done with a negated nominalized form of a verb.

\ea%x1087
\label{ex:x1087}
\gll Manin  maneka  na-isow  nena  kookal-owa=pa  \textstyleEmphasizedVernacularWords{perek-owa} \\
      \\
\glt
\z

garden  big  2s.ISOL  2s.GEN  like-NMZ=LOC  harvest-NMZ

\textstyleEmphasizedVernacularWords{weetak.}

no

`The big garden you are not allowed to harvest by yourself according to your liking.'

\ea%x1078
\label{ex:x1078}
\gll I  \textstyleEmphasizedVernacularWords{me}  sira  samora  \textstyleEmphasizedVernacularWords{on-owa=ke},  weetak. \\
      \\
\glt
\z

1p.UNM  not  custom  bad  do-NMZ=CF  no

`We must not do bad things.'

In sentences expressing disobedience to a prohibition, it is particularly common to have the prohibition in a relative clause where the nominalized verb is negated with the verbal negator \textstyleStyleVernacularWordsItalic{me}. Here the contrastive focus clitic is not used. 

\ea%x1887
\label{ex:x1887}
\gll Maa=ko  [\textstyleEmphasizedVernacularWords{me  on-owa}  nain]  nis=ke  on-i-man. \\
      \\
\glt
\z

thing=NF  not  do-NMZ  that1  2p.FC=CF  do-Np-PR.2p

`You do things that must not be done.'

\ea%x1888
\label{ex:x1888}
\gll Sabat  fofa=pa  [\textstyleEmphasizedVernacularWords{me  uuw-owa}  nain]  emeria  nain  saliw-a-k. \\
      \\
\glt
\z

sabbath  day=LOC  not  work-NMZ  that1  woman  that1  heal-PA-3s

`He healed the woman on a Sabbath day when it was forbidden to work.'

\subsection{Negation}\footnotemark{}
\hypertarget{RefHeading22381935131865}{}
\footnotetext{ The contents of this section is mostly based on Bergh\"all (2006).}
Mauwake has more variety in negation than many other Papuan languages. There are four negators in Mauwake instead of only one or two: \textstyleStyleVernacularWordsItalic{me}, \textstyleStyleVernacularWordsItalic{weetak}, \textstyleStyleVernacularWordsItalic{wia} and \textstyleStyleVernacularWordsItalic{marew}, which have somewhat overlapping functions (\sectref{sec:3.10}). Negation can also express frustration or be used as a verb root with certain suffixes; its scope can vary from one constituent to a whole sentence; and it may be emphasized. Double negation results in cancellation of the negation rather than emphasizing it. 

\subsubsection{Clausal negation}
\hypertarget{RefHeading22401935131865}{}
Verbal clauses are negated with the negator \textstyleStyleVernacularWordsItalic{me} `not', placed before the verb (\stepcounter{nx}{\thenx}), verbal group (\stepcounter{nx}{\thenx}) or verb phrase (\stepcounter{nx}{\thenx}). This type of negation, also called standard negation, is symmetric in Mauwake: the negative clause is similar to the corresponding affirmative clause apart from the presence of the negator \citep[61-67]{Miestamo2005}. The negation strategy is the same for transitive and intransitive, independent and dependent clauses, and for imperatives as well. 

\ea%x1090
\label{ex:x1090}
\gll I  iinan  aasa  \textstyleEmphasizedVernacularWords{me  kuuf-a-mik}. \\
      \\
\glt
\z

1p.UNM  sky  canoe  not  see-PA-1/3p

`We did not see the airplanes.'

\ea%x1091
\label{ex:x1091}
\gll Yo  \textstyleEmphasizedVernacularWords{me  keker  op-a-m},  Kedem=ke  makena. \\
      \\
\glt
\z

1s.UNM  not  fear  hold-PA-1s,  Kedem=CF  true

`I was not afraid, true, Kedem was.'

\ea%x1092
\label{ex:x1092}
\gll Mua  \textstyleEmphasizedVernacularWords{me  wia  kuuf-a-mik},  \textstyleEmphasizedVernacularWords{me  wia  furew-a-mik}, \\
      \\
\glt
\z

man  not  3p.ACC  see-PA-1/3p  not  3p.ACC  sense-PA-1/3p

ne  \textstyleEmphasizedVernacularWords{me  wia  imen-a-mik}.

and  not  3p.ACC  find-PA-1/3p

`We did not see, sense, or find the men.'

\ea%x1113
\label{ex:x1113}
\gll Ni \textstyleEmphasizedVernacularWords{} uf-ep=na  maadara  \textstyleEmphasizedVernacularWords{me  iirar-eka}. \\
      \\
\glt
\z

2p.UNM  dance-SS.SEQ=NF  forehead.ornament  not  remove-IMP.2p

`If/when you have danced, do not remove your forehead ornaments.'

The non-verbal predicate in equative and descriptive clauses can be negated with any of the four negators. 

\ea%x1093
\label{ex:x1093}
\gll O  somek  mua  \textstyleEmphasizedVernacularWords{weetak/wia}. \\
      \\
\glt
\z

3s.UNM  song  man  no

`He is not a teacher.'

\ea%x1095
\label{ex:x1095}
\gll O  \textstyleEmphasizedVernacularWords{me}  somek  mua=ke. \\
      \\
\glt
\z

3s.UNM  not  song  man=CF

`He is not a teacher.' 

However, \textstyleStyleVernacularWordsItalic{marew} is possible in these clauses only if the predicate contains an adjective.

\ea%x1096
\label{ex:x1096}
\gll Awuliak  fain  afila  \textstyleEmphasizedVernacularWords{weetak/wia/marew}. \\
      \\
\glt
\z

sweet.potato  this  sweet  no

`This sweet potato is not sweet.'

\ea%x1097
\label{ex:x1097}
\gll Awuliak  fain  \textstyleEmphasizedVernacularWords{me}  afila(=ke). \\
      \\
\glt
\z

sweet.potato  this  not  sweet=CF

`This sweet potato is not sweet.'

When the possessive and existential clauses are negated with the verbal negator \textstyleStyleVernacularWordsItalic{me}, they are like other verbal clauses. But if any of the other negators is used, the negator replaces the verb and becomes a non-verbal predicate, so these clauses become verbless clauses (\sectref{sec:5.6.3}).

\ea%x1098
\label{ex:x1098}
\gll I  sira  naap  \textstyleEmphasizedVernacularWords{me}  \textstyleEmphasizedVernacularWords{yiar  ik-ua}. \\
      \\
\glt
\z

1p.UNM  custom  thus  not  1p.DAT  be-PA.3s

`We do not have a custom like that.'

\ea%x1094
\label{ex:x1094}
\gll Wi  Yaapan  emeria  \textstyleEmphasizedVernacularWords{weetak},  mua  manek=iw. \\
      \\
\glt
\z

3p.UNM  Japan  woman  no,  man  big=LIM

`The Japanese had no women/wives, (they were) only men.'

\ea%x1099
\label{ex:x1099}
\gll Iiriw  miiwa  muuta  nain  irak-owa  \textstyleEmphasizedVernacularWords{me  ik-ua.} \\
      \\
\glt
\z

earlier  land  for  that1  fight-NMZ  not  be-PA.3s

`Earlier there was no fighting for land.'

\ea%x1100
\label{ex:x1100}
\gll Iiriw  miiwa  muuta  nain  irak-owa  \textstyleEmphasizedVernacularWords{marew}, {\dots} \\
      \\
\glt
\z

earlier  land  for  that1  fight-NMZ  no(ne)

`Earlier there was no fighting for land, ...'

With so many possible alternatives, the speaker has a choice of repeating the same negator or using different ones when several items are negated. Either strategy is used by good language users.

\ea%x1128
\label{ex:x1128}
\gll I  muuka  \textstyleEmphasizedVernacularWords{marew}  a,  i  wiipa  \textstyleEmphasizedVernacularWords{marew}  a. \\
      \\
\glt
\z

1p.UNM  son  no(ne)  oh  1p.UNM  daughter  no(ne)  oh

`We have no son, and we have no daughter.'

\ea%x1127
\label{ex:x1127}
\gll I  urupa  \textstyleEmphasizedVernacularWords{weetak},  i  soomia  \textstyleEmphasizedVernacularWords{wia},  i  \\
      \\
\glt
\z

1p.UNM  cup  no  1p.UNM  spoon  no  1p.UNM

epira  \textstyleEmphasizedVernacularWords{marew.}

plate  no(ne)

`We had no cups, we had no spoons, we had no plates.'

In a few cases the choice of a negator indicates a difference in meaning. The example (\stepcounter{nx}{\thenx}) is the Mauwake equivalent for the common Tok Pisin idiom \textstyleForeignWords{nogat tok } `I do not have anything against it'.

\ea%x1129
\label{ex:x1129}
\gll Yo  opora  \textstyleEmphasizedVernacularWords{weetak/wia}. \\
      \\
\glt
\z

1p.UNM  talk  no

`I have no talk. (= I do not have anything to say.)'

\ea%x1130
\label{ex:x1130}
\gll Yo  opora  \textstyleEmphasizedVernacularWords{marew}. \\
      \\
\glt
\z

1p.UNM  talk  no(ne)

`I have no talk. (= It is OK / I do not have anything against it.)'

The predicate function of the negators \textstyleStyleVernacularWordsItalic{weetak} and \textstyleStyleVernacularWordsItalic{marew} is also shown in the fact that they take a medial different-subject suffix, when the verbless negative possessive or existential-presentative clauses occur sentence-medially in a chaining structure. \textstyleStyleVernacularWordsItalic{Wia} cannot be suffixed with the medial verb suffix.

\ea%x1101
\label{ex:x1101}
\gll Maa  pela  \textstyleEmphasizedVernacularWords{marew-eya}  /  \textstyleEmphasizedVernacularWords{weetak-eya}  fofa  er-a-m. \\
      \\
\glt
\z

thing  leaf  no(ne)-2/3s.DS  /  no-2/3s.DS  market  go-PA-1s

`I had no greens and went to the market.'

\subsubsection{Constituent negation}
\hypertarget{RefHeading22421935131865}{}
Papuan languages typically do not have lexicalized constituent negation of the type `nothing', `nobody' etc., and even syntactic constituent negation may be lacking \citep[271-2]{Reesink1987}. But in Mauwake it is possible to negate various constituents within a clause, and, although very rarely, even inside a noun phrase. The basic constituent negator is \textstyleStyleVernacularWordsItalic{me} `not'. It precedes the negated element, which receives extra stress. Position of the negator, stress, and sometimes the neutral focus clitic all interact in constituent negation.

\ea%x1102
\label{ex:x1102}
\gll \textstyleEmphasizedVernacularWords{Me}  \textstyleEmphasizedVernacularWords{napuma=ke}  ifakim-o-k. \\
      \\
\glt
\z

not  sickness  kill-PA-3s

`It wasn't sickness that killed him.'

\ea%x1103
\label{ex:x1103}
\gll Maa  oposia  \textstyleEmphasizedVernacularWords{me  ewur}  enim-i-mik. \\
      \\
\glt
\z

thing  meat  not  soon  eat-Np-PR.1/3p

`Meat we will not eat soon (after spouse's death).'

\ea%x1108
\label{ex:x1108}
\gll \textstyleEmphasizedVernacularWords{Me  epa  fan}  irak-owa  uruf-a-mik. \\
      \\
\glt
\z

not  place  here  fight-NMZ  see-PA-1/3p

`It was not here that they saw the fighting.'

\ea%x1104
\label{ex:x1104}
\gll Nepa  opaimika  \textstyleEmphasizedVernacularWords{me  baliwep}  miim-a-mik. \\
      \\
\glt
\z

bird  talk  not  well  hear-PA-1/3p

`They did not hear (understand) Tok Pisin well.'

\ea%x1105
\label{ex:x1105}
\gll \textstyleEmphasizedVernacularWords{Me  nomokowa  eliwa } aaw-e\textstyleEmphasizedVernacularWords{-}mik. \\
      \\
\glt
\z

not  tree  good  take-PA-1/3p

`It wasn't good trees that they took.'

In clauses with \textstyleEmphasizedWords{\textsc{quantifiers}}, constituent negation has an important function disambiguating the meaning. If the subject or object noun phrase has a quantifier, the negation is done in different ways depending on whether the quantifier is in the scope of the negation or not. In (\stepcounter{nx}{\thenx}) the noun phrase with the particular quantifier \textstyleStyleVernacularWordsItalic{kuisow} `one' is not in the scope of the negation, but in (\stepcounter{nx}{\thenx}) it is. The neutral focus clitic is required to clarify the meaning; it can even be attached to some other constituent between the quantifier and the negator (\stepcounter{nx}{\thenx}).

\ea%x1142
\label{ex:x1142}
\gll Mua  \textstyleEmphasizedVernacularWords{kuisow}  \textstyleEmphasizedVernacularWords{me}  ekap-o-k. \\
      \\
\glt
\z

man  one  not  come-PA-3s

`One (particular) man did not come.'

\ea%x1143
\label{ex:x1143}
\gll Mua  \textstyleEmphasizedVernacularWords{kuisow}=\textstyleEmphasizedVernacularWords{ko}  \textstyleEmphasizedVernacularWords{me}  ekap-o-k. \\
      \\
\glt
\z

man  one=NF  not  come-PA-3s

`Not (even) one man came.'

\ea%x1147
\label{ex:x1147}
\gll Mua  \textstyleEmphasizedVernacularWords{kuisow}  owowa=pa=\textstyleEmphasizedVernacularWords{ko}  \textstyleEmphasizedVernacularWords{me}  ik-ua. \\
      \\
\glt
\z

man  one  village=LOC=NF  not  be-PA.3s

`Not (even) one man stayed in the village.'

The example (\stepcounter{nx}{\thenx}) is ambiguous as to whether only one man did not go down or whether it is negated that only one man went. The first alternative is the more likely meaning, and if one wants to make sure to give the second meaning, the standard strategy for constituent negation (\stepcounter{nx}{\thenx}) is used.

\ea%x1145
\label{ex:x1145}
\gll Mua  kuisow  muuta  \textstyleEmphasizedVernacularWords{me}  ekap-o-k. \\
      \\
\glt
\z

man  one  only  not  come-PA-3s

`Only one man did not come.' Or: `Not only one man came (but more).'

\ea%x1144
\label{ex:x1144}
\gll \textstyleEmphasizedVernacularWords{Me}  mua  kuisow  (muuta)  ekap-o-k. \\
      \\
\glt
\z

not  man  one  (only)  come-PA-3s

`Not only one man came.'

Similarly, with the universal quantifier \textstyleStyleVernacularWordsItalic{unowiya} `all' the scope of the negation may be ambiguous. The preferred interpretation for (\stepcounter{nx}{\thenx}) is that the statement about not following God's talk refers to all people, thus \textstyleEmphasizedWords{\textsc{no one}} follows it; but it may also be understood that even if all the people do not follow it, some do.

\ea%x1148
\label{ex:x1148}
\gll Emeria  mua  \textstyleEmphasizedVernacularWords{unowiya}  Mua  Maneka  opora  \textstyleEmphasizedVernacularWords{me}  ook-i-mik. \\
      \\
\glt
\z

woman  man  all  Man  Big  talk  not  follow-Np-PR.1/3p

`All the people do not follow God's talk.'

If the negator is in the constituent negation position, the statement is unambiguous. In this respect Mauwake behaves differently from Usan, which does not allow a constituent negation structure \citep[275-277]{Reesink1987}.

\ea%x1149
\label{ex:x1149}
\gll Nain  \textstyleEmphasizedVernacularWords{me}  mua  \textstyleEmphasizedVernacularWords{unowiya}  opora  wiar  op-i-mik. \\
      \\
\glt
\z

but  not  man  all  talk  3.DAT  hold-Np-PR.1/3p

`But not all men/people believe in him (= some do).'

Stress may also be employed to give a constituent negation interpretation to a negated clause. When the clausal stress is on the negator, the whole clause is negated (\stepcounter{nx}{\thenx}). In order to negate the universal quantifier rather than the verb, the main stress needs to be on the quantifier (\stepcounter{nx}{\thenx}). This type of negation is used in Usan as well (ibid. 277).

\ea%x669
\label{ex:x669}
\gll Mua \textstyleEmphasizedVernacularWords{} unow=iya  \textstyleEmphasizedVernacularWords{'me}  \textstyleEmphasizedVernacularWords{ikiw-e-mik.} \\
      \\
\glt
\z

man  many=COM  not  go-PA-1/3p

`All the men \textit{didn't go} (=none of them went).'

\ea%x671
\label{ex:x671}
\gll Mua  \textstyleEmphasizedVernacularWords{'unow=iya } \textstyleEmphasizedVernacularWords{me}  ikiw-e-mik. \\
      \\
\glt
\z

man  many=COM  not  go-PA-1/3p

`\textit{All} the men didn't go (=only some went).'

When a clause with the universal quantifier \textstyleStyleVernacularWordsItalic{unow onaiya} `all' is negated, it tends to be interpreted as a constituent negation of the quantifier, possibly because \textstyleStyleVernacularWordsxiiptItalic{unow onaiya} is a heavier structure than \textstyleStyleVernacularWordsItalic{unowiya} and as such more prominent (\stepcounter{nx}{\thenx}).

\ea%x668
\label{ex:x668}
\gll \textstyleEmphasizedVernacularWords{Unow  onaiya  me}  ikiw-e-mik. \\
      \\
\glt
\z

many  with  not  go-PA-1/3p

`Not all of them went (= only some went)'

The following example is not a case of \textstyleStyleVernacularWordsItalic{unowa} negated separately inside a \textstyleAcronymallcaps{NP}; instead, \textstyleStyleVernacularWordsItalic{mua} `man' is fronted as a theme (\sectref{sec:9.1}):

\ea%x1150
\label{ex:x1150}
\gll Mua  \textstyleEmphasizedVernacularWords{me  unowa}  ekap-e-mik. \\
      \\
\glt
\z

man  not  many  come-PA-1/3p

`There were not many men that came.' Or: `As for men, not many came.'

\textstyleStyleVernacularWordsItalic{Eliwa} `good' may be the only adjective that can be negated by itself inside a noun phrase. These structures are very rare and would need a more careful study. (\stepcounter{nx}{\thenx}) may also be a combination of a non-verbal clause and a transitive clause where the object \textstyleAcronymallcaps{NP} only retains the adjective; the noun is deleted because it occurs in the previous clause.

\ea%x1106
\label{ex:x1106}
\gll Maa  en-owa  \textstyleEmphasizedVernacularWords{eliw(a)  marew}  p-or-o-mik. \\
      \\
\glt
\z

thing  eat-NMZ  good  no(ne)  Bpx-descend-PA-1/3p

`They brought down not-good food.'

\ea%x1107
\label{ex:x1107}
\gll Biiris  \textstyleEmphasizedVernacularWords{me  eliwa},  damo-damola=ko  on-a-mik. \\
      \\
\glt
\z

bridge  not  good  RDP-bad=NF  make-PA-1/3p

`They did not make good bridges (but) bad ones.' (Or: The bridges were not good, they made bad ones.) 

The following looks like a constituent negation attached to the noun \textstyleStyleVernacularWordsItalic{mokoka} `eye(s)', but actually \textstyleStyleVernacularWordsItalic{me}  here is a clausal negator negating the whole idiomatic sentence of `keeping one's eyes shut' (i.e. being ignorant).

\ea%x1114
\label{ex:x1114}
\gll \textstyleEmphasizedVernacularWords{Me}  mokoka  op-ar-ep  ik-e. \\
      \\
\glt
\z

not  eye(s)  closed-CAUS-SS.SEQ  be-IMP.2s

`Do not have your eyes closed.'

Those cases of constituent negation where \textstyleStyleVernacularWordsItalic{me}  precedes a verb can be distinguished from clausal negation only in spoken language on the basis of extra stress on the verb.

\ea%x1110
\label{ex:x1110}
\gll Ni  iperuma  fain  \textstyleEmphasizedVernacularWords{me  e}\textstyleEmphasizedVernacularWords{{{\textprimstress}}}\textstyleEmphasizedVernacularWords{nim-eka},  wafur-eka! \\
      \\
\glt
\z

2p.UNM  eel  this  not  eat-IMP.2p  throw-IMP.2p

`\textit{Don't eat}  this eel, throw it away!'

\subsubsection{Negative interjection}
\hypertarget{RefHeading22441935131865}{}
A negative interjection is used as a one-word reply to a question or a statement. It stands as a complete sentence by itself or is preposed and syntactically independent of the rest of the sentence. Two of the negators are used as negative interjections: \textstyleStyleVernacularWordsItalic{weetak} and \textstyleStyleVernacularWordsItalic{wia}. They are synonymous and usually interchangeable, but in a few environments one or the other is preferred.

\ea%x1115
\label{ex:x1115}
\gll No  aaya  sesenar-e-n=i?  -\textstyleEmphasizedVernacularWords{Weetak/wia}  (me  sesenar-e-m). \\
      \\
\glt
\z

2s.UNM  sugar  buy-PA-2s=QM  -no  (not  buy-PA-1s)

`Did you buy sugar?' --`No (I didn't).'

\ea%x1116
\label{ex:x1116}
\gll Yomar  owora  efar  aaw-o-k.  \\
      \\
\glt
\z

1s/p.cousin  betelnut  1s.DAT  take-PA-3s

-\textstyleEmphasizedVernacularWords{Weetak/wia},  me  os=ke  aaw-o-k.

-no  not  3s.FC=CF  take-PA-3s

`My cousin took my betelnut. --No, it wasn't he who took it.'

For the use of \textstyleStyleVernacularWordsItalic{weetak/wia} as an answer to a negative question, see \sectref{sec:7.2.7}.

\subsubsection{Other cases of negation}
\hypertarget{RefHeading22461935131865}{}
When an affirmative clause is followed by a negative one, and the two only differ by the contrasted element, the whole clause apart from the contrasted element is replaced by \textstyleStyleVernacularWordsItalic{weetak} or \textstyleStyleVernacularWordsItalic{wia}. A full clause is possible instead of \textstyleStyleVernacularWordsItalic{weetak}/\textstyleStyleVernacularWordsItalic{wia}, but it is not as common.

\ea%x1119
\label{ex:x1119}
\gll Mua  bug  maala  nain=ke  mera  unowa  isak-i-non, \\
      \\
\glt
\z

man  wind  long  that1=CF  fish  many  spear-Np-FU.3s

mua  bug  iiwa  nain  \textstyleEmphasizedVernacularWords{weetak/wia}.

man  wind  short  that1  no.

`A man with long breath (=big lungs) will spear many fish, a man with short breath will not.'

\ea%x1120
\label{ex:x1120}
\gll Mera  papako  unowa,  papako  \textstyleEmphasizedVernacularWords{weetak/wia}. \\
      \\
\glt
\z

fish  some  many  some  no

`Some fish there are many, some not.'

Also when an affirmative question is followed by a negative alternative, \textstyleStyleVernacularWordsItalic{weetak} or \textstyleStyleVernacularWordsItalic{wia} is used. 

\ea%x1121
\label{ex:x1121}
\gll Sira  nain  piipua-i-nan=i  e  \textstyleEmphasizedVernacularWords{weetak}? \\
      \\
\glt
\z

habit  that  leave-Np-FU.2s=QM  or  no

`Will you stop that habit or not?'

\ea%x1122
\label{ex:x1122}
\gll Yo  emeria=ko  efar  uruf-a-man=i  e  \textstyleEmphasizedVernacularWords{weetak}? \\
      \\
\glt
\z

1s.UNM  woman=NF  1s.DAT  see-PA-2p=QM  or  no

`Have you seen my wife or not?'

If an action fails to have the expected result, again one of the two negative interjections is used either by itself or followed by a full clause. 

\ea%x1124
\label{ex:x1124}
\gll Marasin  wu-om-a-mik=na  \textstyleEmphasizedVernacularWords{weetak}. \\
      \\
\glt
\z

medicine  put-BEN-BNFY2.PA-1/3p=TP  no

`They put medicine in him but no (=with no result).'

\ea%x1126
\label{ex:x1126}
\gll Naap  ik-ok  uruf-am-ika-iwkin  \textstyleEmphasizedVernacularWords{wia}. \\
      \\
\glt
\z

thus  be-SS  see-SS.SIM-be-2/3p.DS  no

`They were thus watching him (but) no (he did not revive).'

\ea%x1123
\label{ex:x1123}
\gll I  unan  maa  en-e-mik  en-e-mik  \textstyleEmphasizedVernacularWords{wia},  \\
      \\
\glt
\z

1p.UNM  yesterday  food  eat-PA-1/3p  eat-PA-1/3p  no

ipoka  taan-ep  enakiwa  wu-a-mik.

stomach  become.full-SS.SEQ  half  put-PA-1/3p

`Yesterday we ate and ate, (but) no (=we could not finish the food), our stomachs were full and we put half of it aside.'

When the clause expressing frustration of an effort starts a new sentence and begins with the additive connective \textstyleStyleVernacularWordsItalic{ne} `and/but', the negator is always \textstyleStyleVernacularWordsItalic{wia}, and an explanatory clause follows.

\ea%x1125
\label{ex:x1125}
\gll \textstyleEmphasizedVernacularWords{Ne}  \textstyleEmphasizedVernacularWords{wia},  papako=ke  ma-e-mik,  ``Weetak,  moram  \\
      \\
\glt
\z

ADD  no  other=CF  say-PA-1/3p  no  why

owowa  p-ikiw-i-yan?''

village  Bpx-go-Np-FU.1p

`But no, others said, ``Why take him to the village?'' '

Mauwake has two different kinds of double negation. In both cases the negation is cancelled and the result is affirmative, but not an emphatic affirmative. A negated verb or an inherently negative verb may occur with the clausal negator \textstyleStyleVernacularWordsItalic{me} `not':

\ea%x1131
\label{ex:x1131}
\gll Ona  muuka  \textstyleEmphasizedVernacularWords{me}  sesek-owa=ke  \textstyleEmphasizedVernacularWords{me}  ma-e-k. \\
      \\
\glt
\z

3s.GEN  son  not  send-NMZ=CF  not  say-PA-3s

`He did not say that he wouldn't send (lit: say about not sending) his son.'

\ea%x1132
\label{ex:x1132}
\gll Maamuma  \textstyleEmphasizedVernacularWords{me  marew-ar-e-mik}. \\
      \\
\glt
\z

money  not  no(ne)-INCH-PA-1/3p

`We/They did not lack money.'

In the second type of double negation a speaker's negative statement is challenged by another speaker. In this case a different negator is used to challenge the original negation: 

\ea%x1133
\label{ex:x1133}
\gll Yo  episowa  weetak.  -\textstyleEmphasizedVernacularWords{Weetak  wia}. \\
      \\
\glt
\z

1s.UNM  tobacco  no.  -no  no

`I have no tobacco.'  `Don't say you don't have any.'

The negation can be emphasized with the intensity adverb \textstyleStyleVernacularWordsItalic{akena} `very':

\ea%x1134
\label{ex:x1134}
\gll \textstyleEmphasizedVernacularWords{Weetak  akena},  i  me  kuum-e-mik. \\
      \\
\glt
\z

no  very  1p.UNM  not  burn-PA-1/3s

`\textstyleEmphasizedWords{\textsc{No}}, we did not burn it.'

\ea%x1135
\label{ex:x1135}
\gll I  \textstyleEmphasizedVernacularWords{me}  kuum-e-mik  \textstyleEmphasizedVernacularWords{akena}. \\
      \\
\glt
\z

1p.UNM  not  burn-PA-1/3p  very

`We did \textstyleEmphasizedWords{\textsc{not}} burn it.'

Another possible strategy for emphasizing a negative statement or command is to attach the neutral focus clitic -\textstyleStyleVernacularWordsItalic{ko} to the verbal negator \textstyleStyleVernacularWordsItalic{me} `not'. In (\stepcounter{nx}{\thenx}) the neutral focus clitic appears twice, as the speaker wants both to emphasise the negation and to distance himself from the situation (without implying that someone else did see what he did not). 

\ea%x1152
\label{ex:x1152}
\gll I  \textstyleEmphasizedVernacularWords{me=ko}  miim-a-mik. \\
      \\
\glt
\z

1p.UNM  not=NF  hear-PA-1/3p

`We did \textstyleEmphasizedWords{\textsc{not}} hear it.'

\ea%x1136
\label{ex:x1136}
\gll Yo=ko  \textstyleEmphasizedVernacularWords{me=ko}  uruf-a-m. \\
      \\
\glt
\z

1s.UNM=NF  not=NF  see-PA-1s

`\textstyleEmphasizedWords{I} did \textstyleEmphasizedWords{\textsc{not}} see it.'

\ea%x1137
\label{ex:x1137}
\gll \textstyleEmphasizedVernacularWords{Me}\textstyleEmphasizedVernacularWords{=ko}  niir-e  sa,  kae  napum-ar-e-k. \\
      \\
\glt
\z

not=NF  laugh-IMP.2s  INTJ  grandfather  sick-INCH-PA-3s

`Do \textstyleEmphasizedWords{\textsc{not}} laugh, grandfather is sick.'

Negative spreading is fairly common in languages that have a medial verb system. The negation can spread forwards or backwards, or both, depending on the language. In Mauwake both forward (\stepcounter{nx}{\thenx}) and backward (\stepcounter{nx}{\thenx}) spreading is possible across medial clause boundaries, but only with the same-subject medial verbs.\footnote{In Usan, the negation of a final clause can spread backwards even with a different subject medial verb  \citep[282]{Reesink1987}, but Hua, like Mauwake, requires a same subject medial verb \citep[408]{Haiman1980}.} The spreading is not common, but it is more acceptable if the verbs form a logical sequence, an ``expectancy chain''.

\ea%x1138
\label{ex:x1138}
\gll Nain  yo  \textstyleEmphasizedVernacularWords{me  ep-ap}  \textstyleEmphasizedVernacularWords{nefa}  \textstyleEmphasizedVernacularWords{aaw-e-m}. \\
      \\
\glt
\z

but  1s.UNM  not  come-SS.SEQ  2s.ACC  get-PA-1s

`But I did not come and get you.'

\ea%x1140
\label{ex:x1140}
\gll Nainiw  \textstyleEmphasizedVernacularWords{ekap-ep}  \textstyleEmphasizedVernacularWords{maa}  \textstyleEmphasizedVernacularWords{me  sesek-a-mik}. \\
      \\
\glt
\z

again  come-SS.SEQ  food  not  sell-PA-1/3p

`They did not come back and sell food again.'

But negative spreading is not automatic; even with a same-subject medial verb two clauses \textstyleParagraphChari{can} have different polarity (\stepcounter{nx}{\thenx}), (\stepcounter{nx}{\thenx}). If the speaker wants to avoid ambiguity, finite clauses can be used when the polarity is different (\stepcounter{nx}{\thenx}).

\ea%x1156
\label{ex:x1156}
\gll Nepa  opaimika  \textstyleEmphasizedVernacularWords{me}  \textstyleEmphasizedVernacularWords{baliwep  amis-ar-ep}  wiena \\
      \\
\glt
\z

bird  talk  not  well  knowledge-INCH-SS.SEQ  3p.GEN

opaimik=iw  yia  maak-em-ik-e-mik.

talk=INST  1p.ACC  tell-SS.SIM-be-PA-1/3p

`They did not know Tok Pisin well and talked to us in their own language.'

\ea%x1763
\label{ex:x1763}
\gll Mua  lebuma  \textstyleEmphasizedVernacularWords{me  arim-ep}  takira  ik-ok  emeria \\
      \\
\glt
\z

man  lazy  not  grow-SS.SEQ  young  be-SS.SIM  woman

wia  aaw-i-mik.

3s.ACC  take-Np-PR.1/3p

`Lazy men, not having grown and (still) being young, take wives.'

\ea%x1153
\label{ex:x1153}
\gll Nainiw  ekap-e-mik,  nain  maa  \textstyleEmphasizedVernacularWords{me}  sesek-a-mik. \\
      \\
\glt
\z

again  come-PA-1/3p  that1  food  not  sell-PA-1/3p

`They came back again, but did not sell any food.'

If the context is not clear enough, the negator can be repeated for each negated verb in a medial verb construction. In (\stepcounter{nx}{\thenx}), if only the first verb is negated, the sentence could mean that many people do not know the person but follow him nevertheless; whereas if only the second verb is negated, the sentence might be taken to mean that many people do know the person but do not follow him. 

\ea%x1139
\label{ex:x1139}
\gll Mua  unowa  o  \textstyleEmphasizedVernacularWords{me}  amis-ar-ep \\
      \\
\glt
\z

man  many  3s.UNM  not  knowledge-INCH-SS.SEQ

\textstyleEmphasizedVernacularWords{me}  ook-i-mik.

not  follow-Np-PR.3p

`Many people do not know him and do not follow him.'

Different-subject marking blocks negative spreading in both directions. Thus in (\stepcounter{nx}{\thenx}) the polarity changes with each new clause:

\ea%x1141
\label{ex:x1141}
\gll Soomar-em-ika-iwkin  \textstyleEmphasizedVernacularWords{me  wia  far-eya} \\
      \\
\glt
\z

walk-SS.SIM-be-2/3p.DS  not  3p.ACC  call-2/3s.DS

nefa  ma-i-kuan,  {\dots}

2s.ACC  say-Np-FU.3p

`When they walk past, and you do not call them, they will say about you that {\dots}'

Negative transportation from a complement clause to a main clause does not take place in Mauwake.\footnote{This is true of Amele as well \citep[44]{Roberts1987}, but Usan allows it \citep[278-280]{Reesink1987}.} 

\subsection{Deixis} 
\hypertarget{RefHeading22481935131865}{}
Different parts in the grammar interact to produce the deictic system, the spatio-temporal and personal orientation related to the speech situation or another situation specified in the text. The default deictic centre is the speaker, the speaker's location and the present time.  

\subsubsection{Person deixis}
\hypertarget{RefHeading22501935131865}{}
Only the first and second person are inherently deictic, as they get their whole meaning, apart from the number, from the speech situation. The person marking is done by pronouns (\sectref{sec:3.5}) and by person/number suffixes on the verbs (\sectref{sec:3.8.3.4}, 3.8.3.5). The special status of the first person as against both the second and third persons shows in the imperative and the switch-reference marking. In the imperative the dual number is only possible in the first person (\sectref{sec:3.8.3.2.2}): 

\ea%x1262
\label{ex:x1262}
\gll Aria,  i  owowa=ko  urup-\textstyleEmphasizedVernacularWords{u}.  Auwa  aite \\
      \\
\glt
\z

alright  1p.UNM  village=NF  ascend-IMP.1d  1s/p.father  1s/p.mother

wia  karu-i-yan,  owowa=pa  wia  uruf-\textstyleEmphasizedVernacularWords{u}.

3p.ACC  visit-Np-FU.1p  village=LOC  3p.ACC  see-IMP.1d

`Alright, let's (dl) go up to the village. We'll visit father and mother, let's (dl) see them in the village.'

In the different-subject medial verbs the first person singular and plural share the same suffix, whereas the second and third persons are grouped together and the distinction is made according to number, between singular and plural (\sectref{sec:3.8.3.4.2}).

\ea%x1263
\label{ex:x1263}
\gll I  ikoka  urup-ep  nia  \textstyleEmphasizedVernacularWords{maak-omkun} \\
      \\
\glt
\z

1p.UNM  later  ascend-SS.SEQ  2p.ACC  tell-1s/p.DS

\textstyleEmphasizedVernacularWords{ora-iwkin,}  aria  owawiya  feeke  pok-ap  ik-ok  {\dots}

descend-2/3p.DS  alright  together  here.CF  sit-SS.SEQ  be-SS

`Later when we come up and tell you and (then) you come down and we sit down together here and {\dots}'

Even though the first and second person pronouns are already deictic in themselves, their unmarked plural forms can both co-occur with the proximate demonstrative \textstyleStyleVernacularWordsItalic{fain} 'this', and the second person also with the distal demonstrative \textstyleStyleVernacularWordsItalic{nain} `that'. As only one of the people referred to by these plural forms typically is a speech act participant and the others may or may not be present, the addition of the demonstrative  makes it clear that all the people referred to are present in the situation: 

\ea%x1269
\label{ex:x1269}
\gll Ikoka  Yaapan=ke  ekap-emi  \textstyleEmphasizedVernacularWords{ni}  emeria  unowa  \textstyleEmphasizedVernacularWords{fain} \\
      \\
\glt
\z

later  Japan=CF  come-SS.SIM  2p.UNM  woman  many  this

nia  aaw-urum-i-kuan.

2p.ACC  take-DISTR/A-Np-FU.3p

`Later the Japanese will come and take all of you many women [here in this village].'

Mauwake has no separate system of social deixis, as there are no honorifics, nor are there special pronouns used for particular kin or social groups or the like.

Emotional deixis, associating the speaker with the topic of conversation or distancing him from it, is a possible use for demonstratives in Papuan languages and worldwide (Farr and Whitehead 1982:72-78, Lakoff 1974:347-355). In Mauwake that possibility is not utilized: the demonstratives are neutral in this respect.

\subsubsection{Locative deixis} 
\hypertarget{RefHeading22521935131865}{}
Locative deixis, which relates the location to the speech act participants, utilizes several different word classes. The proximate demonstrative \textstyleStyleVernacularWordsItalic{fain} `this' (\sectref{sec:3.6.2}) and the corresponding locative adverb \textstyleStyleVernacularWordsItalic{fan} `here'(\sectref{sec:3.6.3})  are truly deictic, as their meaning is based on the location of the speaker. The distal-1 demonstrative \textstyleStyleVernacularWordsItalic{nain} `that' and the adverb \textstyleStyleVernacularWordsItalic{nan} `there' are more neutral, and the less common distal-2 and -3 deictics have other defining features besides the distance to the speaker. 

\ea%x1273
\label{ex:x1273}
\gll Ep-ap  owora  \textstyleEmphasizedVernacularWords{fain}  aaw-ep  enim-eka,  iwer(a)  eka  \\
      \\
\glt
\z

come-SS.SEQ  betelnut  this  take-SS.SEQ  eat-IMP.2p  coconut  water

\textstyleEmphasizedVernacularWords{fain}  enim-eka.

this  eat-IMP.2p

`Come and take this betelnut and eat it, (and) drink this coconut water.'

\ea%x1274
\label{ex:x1274}
\gll Yo  wia  wiim-urup-ep  \textstyleEmphasizedVernacularWords{fan}  wia  wu-ap  \\
      \\
\glt
\z

1s.UNM  3p.ACC  escort-ascend-SS.SEQ  here  3p.ACC  put-SS.SEQ

kiiriw  iw-a-m.

again  go-PA-1s

`I escorted them up here and went (back) again.'

In the location verbs \textstyleStyleVernacularWordsItalic{fan}- `arrive/be here' and \textstyleStyleVernacularWordsItalic{nan}- `arrive/be there' (\sectref{sec:3.8.4.4.3}) the deictic goal forms the verb root. 

\ea%x1275
\label{ex:x1275}
\gll Auwa  afura  \textstyleEmphasizedVernacularWords{fan-e-k}  a,  no=ko  wiar  \\
      \\
\glt
\z

1s/p.father  lime  here-PA-3s  INTJ  2s.UNM=NF  3.DAT

akim-ap=ko  uruf-e.

try-SS.SEQ=NF  see-IMP.2s

`Ah, father's lime is here, you try it and see.'

In the directional verbs (\sectref{sec:3.8.4.4.5}) as well as the related bring-verbs (\sectref{sec:3.8.2.4.2}) the verb root gives indication as to the direction of the movement. Only those directional verbs where the direction is clearly related to the speaker are deictic. The second person is not a possible alternative deictic centre for the verb \textstyleStyleVernacularWordsItalic{ekap}- `come'.  

\ea%x1278
\label{ex:x1278}
\gll Uurika  nefar  \textstyleEmphasizedVernacularWords{ikiw-i-nen}. \\
      \\
\glt
\z

tomorrow  2s.DAT  go-Np-FU.1s

`Tomorrow I'll come to you.' (Lit: `{\dots}I'll go (from my present place) ...')

\ea%x1279
\label{ex:x1279}
\gll Mua  imen-ap=na  feeke  wia  \textstyleEmphasizedVernacularWords{p-ekap-eka}. \\
      \\
\glt
\z

man  find-SS.SEQ=TP  here.CF  3p.ACC  BPx-come-IMP.2p

`If you find the men, bring them here.'

Although the prototypical deictic centre is close proximity to the speaker, it may be extended to quite a large area. In (\stepcounter{nx}{\thenx}) where the coming of the Japanese troops is described, it covers the whole North Coast of the New Guinea island: 

\ea%x1892
\label{ex:x1892}
\gll Ne  \textstyleEmphasizedVernacularWords{ekap-ep}  Numbia=pa  nan  urup-e-mik. \\
      \\
\glt
\z

ADD  come-SS.SEQ  Numbia=LOC  there  ascend-PA-1/3p

`And they came and landed at Numbia.'

In narratives it is more typical that the verbs \textstyleStyleVernacularWordsItalic{ekap}- `come' and \textstyleStyleVernacularWordsItalic{ikiw}- `go', as well as the related verbs for `bring' and `take', get their deictic centre from the main character, not the narrator, since the narrator often is not even a participant in the story. 

\ea%x1277
\label{ex:x1277}
\gll Sawur  emeria  nain  ikiw-eya  o  iikir-ami  owowa  \\
      \\
\glt
\z

spirit  woman  that  go-2/3s.DS  3s.UNM  get.up-SS.SIM  village

ekap-o-k.

come-PA-3s

`When the spirit woman went (away), he came to the/his village.'

\subsubsection{Temporal deixis}
\hypertarget{RefHeading22541935131865}{}
Temporal deixis relates time to the speech act, or alternatively to the time of a specific event. Tense marking (\sectref{sec:3.8.3.4}) is the most important device for this in Mauwake, as tense is an obligatory category in verbs.\footnote{In some Papuan languages tense markers and demonstratives are morphologically related (Cindi Farr, p.c.), but this is not the case in Mauwake.} The present tense marks the default deictic centre, the past tense refers to the time before that point, and the future tense to the time after it. The example (\stepcounter{nx}{\thenx}) is repeated here as (\stepcounter{nx}{\thenx}):

\ea%x1893
\label{ex:x1893}
\gll Unan  \textstyleEmphasizedVernacularWords{aakun-e-mik},  aakisa  \textstyleEmphasizedVernacularWords{aakun-i-mik}  ne \\
      \\
\glt
\z

yesterday  talk-PA-1/3p  now/today  talk-Np-PR.1/3p  ADD

uurika  nainiw  \textstyleEmphasizedVernacularWords{aakun-i-yen}.

tomorrow  again  talk-Np-FU.1p

`Yesterday we talked, now/today we talk and tomorrow we'll talk again.'

Papuan languages in general favour presenting a narrative in strictly chronological order, so a relative tense, where the deictic centre is shifted either to the past or to the future, is not utilized widely. This is true of Mauwake as well. When a shift to the past is needed, it can be done by right-dislocating a medial clause after a past-tense marked final clause: 

\ea%x1268
\label{ex:x1268}
\gll Wilkar  wia  muf-em-ik-om-a-mik,  \textstyleEmphasizedVernacularWords{mua \\
      \\
\glt
\z

cart  3p.ACC  pull-SS.SIM-be-BEN-BNFY2.PA-1/3p  man  

\textstyleEmphasizedVernacularWords{kui-kuisow}  \textstyleEmphasizedVernacularWords{wia  maak-iwkin}.

RDP-one  3p.ACC  tell-2/3p.DS

`They\textsubscript{1} pulled carts for them\textsubscript{2}, after they\textsubscript{2} had told the men\textsubscript{1} one by one.'

The same-subject sequential forms of the directional verbs \textstyleStyleVernacularWordsItalic{ekap}- `come' and \textstyleStyleVernacularWordsItalic{ikiw}- `go' also have temporal deictic use, the former referring to time extending to the present moment, the latter mainly to time from the present moment onwards. The examples (\stepcounter{nx}{\thenx}) and (\stepcounter{nx}{\thenx}) are repeated below as (\stepcounter{nx}{\thenx}) and (\stepcounter{nx}{\thenx}).

\ea%x1941
\label{ex:x1941}
\gll Naap  on-am-ik-e-mik,  \textbf{ekap-ep } aakisa. \\
      \\
\glt
\z

thus  do-SS.SIM-be-PA-1/3p  come-SS.SEQ  now

`We have been doing like that (all the time) up until now.'

\ea%x1942
\label{ex:x1942}
\gll No  naap  ik-ok  \textbf{iki(w-e)p}  mokoma  enuma  iiwawun \\
      \\
\glt
\z

2s.UNM  thus  be-SS  go-SS.SEQ  year  new  altogether

aakun-i-nan.

talk-Np-FU.2s

`You will be like that (long time) but next year you will talk.'

The two groups of deictic temporal adverbs (\sectref{sec:3.9.1.2}) behave differently as to what the deictic centre is. The specific temporal adverbs, which refer to a certain day in relation to the utterance, always take the time of the speech act as their deictic centre. 

\ea%x1889
\label{ex:x1889}
\gll {\dots}i  \textbf{uurika}  ora-i-yan,  ifera  un-owa \\
      \\
\glt
\z

{\dots}1p.UNM  tomorrow  descend-Np-FU.1p  sea(water)  fetch-NMZ  

ora-i-yan.

descend-Np-FU.1p

`{\dots}we will go down tomorrow, we will go down to fetch sea water.'

The non-specific temporals normally do this too: 

\ea%x1890
\label{ex:x1890}
\gll Nain  \textstyleEmphasizedVernacularWords{iiriw}  me  kerer-e-k,  \textstyleEmphasizedVernacularWords{aakisa} \textstyleEmphasizedVernacularWords{} \textstyleEmphasizedVernacularWords{fan}  {\O}. \\
      \\
\glt
\z

that1  earlier  not  appear  now  here  

`That didn't appear ealier/long ago but just now (lit: now here).'

But their time reference may also be relative, with the time of the event taken as the deictic centre. This is especially true of \textstyleStyleVernacularWordsItalic{aakisa} `now', which is used for perspectivization.\footnote{The ``WAS -NOW paradox'' occurs in ``free indirect style'' when ``[t]he deictic centre of the utterance is the writer/narrator, but certain deictic elements are relativized to give the impression of direct access to the character's mental states: these include temporal and spatial expressions such as \textit{now, here, today} {\dots} but not tense or person.'' (Mushin and Stirling 2000).} The temporal adverbs in the following two examples, \textstyleStyleVernacularWordsItalic{aakisa} `now' and \textstyleStyleVernacularWordsItalic{aakisa fan} `just now', do not refer externally to the time close to the speech event; instead, they are text-internal perspectivization devices to highlight the importance of the event to the main characters in the text. (\stepcounter{nx}{\thenx}) is from an old traditional story and (\stepcounter{nx}{\thenx}) tells about events that took place over four decades before the recording. 

\ea%x475
\label{ex:x475}
\gll Nain  or-op  ``buu''  (na-e-k),  \textstyleEmphasizedVernacularWords{aakisa}  eka  saanar-\textstyleEmphasizedVernacularWords{e}-k. \\
      \\
\glt
\z

that1  fall-SS.SEQ  buu  say-PA-3s  now  water  dry-PA-3s

`It fell with a thud (and they knew that) now the water had dried up.'

\ea%x1891
\label{ex:x1891}
\gll Ekap-ep,  ekap-ep,  \textstyleEmphasizedVernacularWords{aakisa  fan}  unowa  Wewak=pa \\
      \\
\glt
\z

come-SS.SEQ  come-SS.SEQ  now  here  many  Wewak=LOC  

nan  urup-\textstyleEmphasizedVernacularWords{e}-mik.

there  ascend-PA-1/3p

`They came and came, and just now many came up there in Wewak.'

For the deictic shift that takes place in indirect speech, see \sectref{sec:8.3.2.1.2}.

\subsection{Quantification}
\hypertarget{RefHeading22561935131865}{}
Nouns are not inflected for number in Mauwake, and in the whole noun phrase the number may be left unspecified (\stepcounter{nx}{\thenx}). The verbs are marked for either singular or plural, but the plural form can be used also for unspecified number (\stepcounter{nx}{\thenx}). The pronouns must be either singular or plural. Besides these two obligatory number marking devices the language has several other means for quantification.

\ea%x1284
\label{ex:x1284}
\gll Waaya  kiikir=iw  uruf-i-mik,  owowa=pa. \\
      \\
\glt
\z

pig  first=INST  see-Np-PR.1/3p  village=LOC

`First they look at the pig(s) in the village.'

\ea%x1285
\label{ex:x1285}
\gll Nain  pun  sira  naap=iw,  mua=ko  me  kerer-e-mik. \\
      \\
\glt
\z

that1  too  custom  thus=INST  man=NF  not  appear-PA-1/3p

`That was like that too, the (guilty) person/people did not appear.'

\subsubsection{Quantification in the noun phrase}
\hypertarget{RefHeading22581935131865}{}
Numerals (\sectref{sec:3.4.1}) are used when the exact number is relevant, non-numeral quantifiers (\sectref{sec:3.4.2}) are used elsewhere.  

\ea%x1286
\label{ex:x1286}
\gll \textstyleEmphasizedVernacularWords{Masin  erup}  nainiw  wu-owa  epa  ik-ua. \\
      \\
\glt
\z

engine  two  again  put-NMZ  place  be-PA.3s

`There is a place for putting two more engines.'

\ea%x1308
\label{ex:x1308}
\gll \textstyleEmphasizedVernacularWords{Waa}  \textstyleEmphasizedVernacularWords{muuka  arow}  ekap-o-k. \\
      \\
\glt
\z

pig  son  three  come-PA-1s

`Three piglets came.'

\ea%x1287
\label{ex:x1287}
\gll \textstyleEmphasizedVernacularWords{Emeria}  \textstyleEmphasizedVernacularWords{unow=iya}  ikiw-ep  eka  nain  imar-e-mik. \\
      \\
\glt
\z

woman  many=COM  go-SS.SEQ  river  that1  catch.fish-PA-1/3p

`All the women went and fished at the river.'

The third person plural unmarked pronoun functions as a pluraliser both in an ordinary \textstyleAcronymallcaps{NP} and with place names when the population of the place is referred to (4.1.1). 

\ea%x1288
\label{ex:x1288}
\gll Nain  \textstyleEmphasizedVernacularWords{wi  mua}  sira=ke,  emeria  soop-owa  sira. \\
      \\
\glt
\z

that1  3p.UNM  man  custom=CF  woman  bury-NMZ  custom

`That is the men's custom, the custom of burying wife/wives.'

\ea%x1289
\label{ex:x1289}
\gll Irak-owa  weeser-eya  aria  \textstyleEmphasizedVernacularWords{wi  Simbine} \\
      \\
\glt
\z

fight-NMZ  finish-2/3s.DS  alright  3p.UNM  Simbine

baurar-e-mik.

flee-PA-1/3p

`When the fighting was finished, alright the Simbine people fled.'

Even without the pluralizing pronoun the word for, or a name of, a village may occasionally, as a subject of a clause, refer to the population and thus be interpreted as plural. In the following example this shows in the plural marking in the verb. 

\ea%x1307
\label{ex:x1307}
\gll Ne  owowa  oko  nain=ke  maak-e-mik,  {\dots} \\
      \\
\glt
\z

ADD  village  other  that1=CF  tell-PA-1/3p

`And (the people of) that other village told him, {\dots}''

Reduplication is another pluralizing device used in the \textstyleAcronymallcaps{NP}. Only a small group of nouns can undergo reduplication (\sectref{sec:3.2.6.2}), but in adjectives it is somewhat more common (\sectref{sec:3.3}).

\ea%x1290
\label{ex:x1290}
\gll Waaya  pa-ep  \textstyleEmphasizedVernacularWords{kio-kiowa}  naap  uup-e-mik. \\
      \\
\glt
\z

pig  butcher-SS.SEQ  RDP-piece  thus  cook-PA-1/3p

`We butchered the pig and cooked the pieces like that.'

\ea%x1291
\label{ex:x1291}
\gll Owow(a)  saria=ke  kiikir  perek-i-mik, \\
      \\
\glt
\z

village  headman=CF  first  pull.out-Np-PR.1/3p

\textstyleEmphasizedVernacularWords{mua}  \textstyleEmphasizedVernacularWords{or-oram}  fain  weetak.

man  RDP-insignificant  this  no

`The village headmen harvest it first, not common people like this/us.'

Comitative noun phrases (\sectref{sec:4.1.3}) are used to indicate duality or plurality. 

\ea%x1292
\label{ex:x1292}
\gll \textstyleEmphasizedVernacularWords{(Yo/I)  auwa  ikos}  fan  ik-e-mik. \\
      \\
\glt
\z

1s/1p.UNM  1s/p.father  together.with  here  be-PA-1/3p

`I and my father are here.'

\ea%x1293
\label{ex:x1293}
\gll No  ikoka  \textstyleEmphasizedVernacularWords{mua  owawiya}  irak-ep  me  efar  \\
      \\
\glt
\z

2s.UNM  later  man  with  fight-SS.SEQ  not  1s.DAT

kerer-e.

appear-IMP.2s

`Later when you and your husband fight, don't come to me.'

\ea%x1294
\label{ex:x1294}
\gll Ne  \textstyleEmphasizedVernacularWords{bom=iya  kateres=iya  bom=iya  kateres=iya} {\O}\textstyleEmphasizedVernacularWords{.} \\
      \\
\glt
\z

and  bomb=COM  cartridge=COM  bomb=COM  cartridge=COM

 `And bombs and cartridges, bombs and cartridges (kept dropping).'

\ea%x1295
\label{ex:x1295}
\gll \textstyleEmphasizedVernacularWords{Pauli  ame}  era=pa  wia  uruf-ap  {\dots} \\
      \\
\glt
\z

Pauli  ASSOC  road=LOC  3p.ACC  see-SS.SEQ

`I saw Pauli and the others on the road, and {\dots}'

Personal pronouns have to mark the number,\footnote{Except for third person dative pronoun, which is \textit{wiar} for both singular and plural.} but in cases where the number is unknown or unspecified, plural is used.

\ea%x1309
\label{ex:x1309}
\gll Ikiw-ep  mua  \textstyleEmphasizedVernacularWords{wia}  uruf-a-k  na  weetak,  mua=ko  me  \\
      \\
\glt
\z

go-SS.SEQ  man  3p.ACC  see-PA-3s  but  no  man=NF  not

\textstyleEmphasizedVernacularWords{wia}  furew-a-k.

3p.ACC  sense-PA-3s

`She went and looked for anyone/people but no, she did not sense (there was) anyone (there).'

\subsubsection{Quantification devices in the verbs}
\hypertarget{RefHeading22601935131865}{}
The person/number suffix in the finite verbs (\sectref{sec:3.8.3.4}) is the most frequently used device to indicate quantification: it shows whether the subject is singular or plural. Often the person/number suffix in the verb is the only element in a clause overtly showing number. 

\ea%x1296
\label{ex:x1296}
\gll Mauw-am-ik-ok  ik-ok  mauw-owa  weeser-eya \\
      \\
\glt
\z

work-SS.SIM-be-SS  be-SS  work-NMZ  finish-2/3s.DS  

urera  ekap-e-\textstyleEmphasizedVernacularWords{mik}.

afternoon  come-PA-1/3p

`They came and landed there at Numbia.'

But if the subject noun is [-human], even the person/number suffix may not indicate the number, since plural marking is only used for humans and occasionally for large animals, and only very rarely for inanimates. In the following example, the verbs in both sentences refer to airplanes, but because the action in the first sentence is attributed to the soldiers inside the planes, the finite verb is in plural form. 

\ea%x1283
\label{ex:x1283}
\gll Amerika  irak-ow(a)  iinan  aasa  ekap-ep  Ulingan  nan  bom \\
      \\
\glt
\z

America  fight-NMZ  sky  canoe  come-SS.SEQ  Ulingan  there  bomb

\textstyleEmphasizedVernacularWords{fu-fuurk-ikiw-e-mik}.  Iinan=iw  iinan=iw  wu-ami  feenap

RDP-drop-go-PA-1/3p  sky=INST  sky=INST  put-SS.SIM  like.this

Wewak  kame  naap  \textstyleEmphasizedVernacularWords{ikiw-o-k}.

Wewak  side  thus  go-PA-3s

`American fighter planes came and went on dropping bombs there in Ulingan. They were really high up and went like this to Wewak.'

Reduplication is more common in verbs than in nouns or adjectives (\sectref{sec:3.8.2.4.1}). In transitive verbs the reduplication indicates plurality of the resulting object. 

\ea%x1298
\label{ex:x1298}
\gll Kau  nain  pa-ep,  gele-gelemuti-tik \\
      \\
\glt
\z

cow  that1  butcher-SS.SEQ  RDP-small-RDP

\textstyleEmphasizedVernacularWords{pu-puuk}\textstyleEmphasizedVernacularWords{-}\textstyleEmphasizedVernacularWords{ap}  uup-e-mik.

RDP-cut-SS.SEQ  cook-PA-1/3

`They butchered the cow and cut it into small pieces and cooked it/them.'

\ea%x1297
\label{ex:x1297}
\gll Aruf-irapar-emi  meren(a)  suuw-owa  wiar \\
      \\
\glt
\z

hit-to.and.fro-SS.SIM  leg  pull-NMZ  3.DAT

\textstyleEmphasizedVernacularWords{pere-perek-a-mik.}

RDP-tear-PA-1/3p

`They hit him all over and tore his trousers to pieces.'

Both the distributive suffixes (\sectref{sec:3.8.2.3.2}) mark plurality; the argument that the marking pluralizes depends on the type of verb. 

\ea%x1300
\label{ex:x1300}
\gll Iinan  aasa  fan  \textstyleEmphasizedVernacularWords{or-om-ik-omak-i-ya}. \\
      \\
\glt
\z

sky  canoe  here  descend-SS.SIM-be-DISTR/PL-Np-PR.3s

`Many planes are descending here.'

\ea%x1299
\label{ex:x1299}
\gll Koora  pun  ariwa=ke  kuum-eya  \textstyleEmphasizedVernacularWords{aw-omak-e-k}. \\
      \\
\glt
\z

house  also  arrow=CF  burn-2/3s.DS  burn-DISTR/PL-PA-3s

`Also many houses burned down when the ammunition burned them.'

\ea%x1301
\label{ex:x1301}
\gll Owowa  wia  \textstyleEmphasizedVernacularWords{wi-urum-e-p}  naap  ikiw-i-kuan. \\
      \\
\glt
\z

village  3p.ACC  give.them-DISTR/A-SS.SEQ  thus  go-Np-FU.3p

`They give villages to all of them and then they go like that.' (Certain villages are designated for certain people to go to.)

\ea%x1302
\label{ex:x1302}
\gll O  iiriw  maa  bala  wiar  \textstyleEmphasizedVernacularWords{aaw-urum-ep} \\
      \\
\glt
\z

3s.UNM  earlier  thing  ornament  3.ACC  get-DISTR/A-SS.SEQ

ona  mia=pa-r=iw  wu-a-k.

3s.GEN  body=LOC-{\O}=LIM  put-PA-3s

`Earlier he had received ornaments from all of them and (now) he put them on his own body only.'

In the object cross-referencing verbs (\sectref{sec:3.8.4.2.4}) the root shows singularity or plurality of the object that is cross-referenced.

\ea%x1303
\label{ex:x1303}
\gll Iperowa  opora  \textstyleEmphasizedVernacularWords{wiok-i-yan}. \\
      \\
\glt
\z

middle.aged  talk  follow.them-Np-FU.1p

`We'll follow/obey the talk of the middle-aged men.'

\ea%x1304
\label{ex:x1304}
\gll Maa  eneka  kes  mane-maneka  oram  \textstyleEmphasizedVernacularWords{iw-e-mik}. \\
      \\
\glt
\z

thing  tooth  case  RDP-big  just  give.him-PA-1/3p

`They gave him big meat (tin) cases for free.'

When a numeral follows a nominalized verb form and precedes the resultative verb \textstyleStyleVernacularWordsItalic{ar}- `become' (\sectref{sec:3.8.4.4.4}), that indicates how many times an action was performed. 

\ea%x1305
\label{ex:x1305}
\gll Ewar  maneka  \textstyleEmphasizedVernacularWords{muf-owa  erup  ar-e}. \\
      \\
\glt
\z

wind  big  pull-NMZ  two  become-IMP.2s

`Breathe deeply twice.'

\ea%x1306
\label{ex:x1306}
\gll Kiikir  iinan=pa  \textstyleEmphasizedVernacularWords{akim-owa  arow  ar-e-mik}. \\
      \\
\glt
\z

first  top=LOC  try-NMZ  three  become-PA-1/3p

`First they tried it three times on top.'

\subsection{Comparison}
\hypertarget{RefHeading22621935131865}{}
\subsubsection{Comparison of inequality: comparative constructions}
\hypertarget{RefHeading22641935131865}{}
As the inventory of adjectives is typically small in Papuan languages (Haiman 1980:268, Reesink 1987:63, MacDonald 1990:105-107), it is no surprise that regular morphological or syntactic forms to express comparative and superlative are rare, or even non-existent. In Mauwake comparison can be expressed in various ways, but there are no specific forms that could be called comparative or superlative. Since the overall frequency of comparative constructions is very low, it is not possible here to call any of them the preferred strategy.

One way to express comparison is to conjoin two structurally similar clauses, where the adjective in the first one functioning as the non-verbal predicate is unintensified, but in the second clause it has an intensifier. The first clause contains the standard of comparison. 

\ea%x1336
\label{ex:x1336}
\gll Poka  fain  \textstyleEmphasizedVernacularWords{maala},  ne  oko  \textstyleEmphasizedVernacularWords{maala  akena}. \\
      \\
\glt
\z

stilt  this  long  ADD  other  long  very

`This stilt is long but the other one is longer (lit: very long).'

Although the clauses usually are descriptive, as above, they do not have to be. In the following example the locative noun \textstyleStyleVernacularWordsItalic{iinan} `top', functioning like an adjective here, modifies the head noun in both the clauses.  

\ea%x1324
\label{ex:x1324}
\gll Ema  \textstyleEmphasizedVernacularWords{iinan } urup-e-m,  ne  no  ema  \\
      \\
\glt
\z

mountain  top  ascend-PA-1s  ADD  2s.UNM  mountain

\textstyleEmphasizedVernacularWords{iinan  akena}  urup-o-n.

top  very  ascend-PA-2s

`I climbed a high mountain, but you climbed a higher (lit: very high) mountain.'

Another way is to use adjectives that are antonymous. As a comparative structure this is problematic in that it is arbitrary to call the subject of one clause the standard and the subject of the other the object of comparison. 

\ea%x1325
\label{ex:x1325}
\gll Waaya  nain  \textstyleEmphasizedVernacularWords{gelemuta},  oko  nain  \textstyleEmphasizedVernacularWords{maneka}. \\
      \\
\glt
\z

pig  that1  small  other  that1  big

`That pig is smaller than the other one.' Or: `That other pig is bigger than that one.' (Lit: That pig is small, the other one is big.)'

The same caveat applies to the following structure, where the adjective is negated for comparison:

\ea%x1326
\label{ex:x1326}
\gll Auwa  uuw-owa  \textstyleEmphasizedVernacularWords{eliwa},  mua  oko  fain  \textstyleEmphasizedVernacularWords{wia}. \\
      \\
\glt
\z

1s/p.father  work-NMZ  good  man  other  this  no

`My father's work is better than this other man's. (Lit: My father's work is good, this other man's is not.) '

According to \citet{Stassen2008} this \textstyleEmphasizedWords{\textsc{Conjoined Comparative}} strategy, exemplified above, is prevalent in  Australia and New Guinea. But the sample of New Guinean languages used for the generalization is very small (and includes both Austronesian and Papuan languages), and I suggest that at least for \textstyleAcronymallcaps{TNG} languages the \textstyleEmphasizedWords{\textsc{Exceed Comparative}}, the strategy represented in that sample only by Amele \citep[134-135]{Roberts1987}, is a possible alternative and may actually be as common as, or more common than, the Conjoined Comparative strategy.\footnote{My opinion is mainly based on the experience of working with national translators. When searching for translation equivalents for comparison forms, they often start with the Conjoined Comparative pattern, but very soon after realising that they do not have to stay within the limits of stative clauses only or stick to the adjective class, many actually tend to prefer the Exceed Comparative as the more natural and accurate expression for comparison.  \citet[68]{Reesink1987} mentions both of these mechanisms for Usan. }   There are two clauses in this pattern too: one may be equative and contain an adjective, the other is a transitive clause containing the verb \textstyleStyleVernacularWordsItalic{nomak}- `exceed/surpass' as the predicate and the standard of comparison as the object. The order of the two clauses is free.

\ea%x1327
\label{ex:x1327}
\gll Maa  mane-maneka,  maa  fain  \textstyleEmphasizedVernacularWords{nomak-ep}  ik-ua. \\
      \\
\glt
\z

thing  RDP-big  thing  this  surpass-SS.SEQ  be-PA.3s

`They are big things, greater than these.'

\ea%x1328
\label{ex:x1328}
\gll No  yiena  nembesir  \textstyleEmphasizedVernacularWords{nomak-ep}  maneka \\
      \\
\glt
\z

2s.UNM  1p.GEN  ancestor  surpass-SS.SEQ  big

ar-ek-a-m  na-ep=i?

become-CNTF-PA-1s  say-SS.SEQ=QM

`Do you want to become greater than our ancestors?'

\ea%x1333
\label{ex:x1333}
\gll Nomokowa  kakawa  fain  iiwa,  oko  \textstyleEmphasizedVernacularWords{nomak-ep}  puuk-a-m. \\
      \\
\glt
\z

tree  strip  this  short  other  surpass-SS.SEQ  cut-PA-1s.

`This piece of timber is short, I cut the other one longer.'

A transitive clause with \textstyleStyleVernacularWordsItalic{nomak}- is also used, when a noun rather than an adjective describes the characteristic under comparison. 

\ea%x1329
\label{ex:x1329}
\gll O  kekan-owa=ke  yo  kekan-owa  efar \\
      \\
\glt
\z

3s.UNM  be.strong-NMZ=CF  1s.UNM  be.strong-NMZ  1s.DAT

\textstyleEmphasizedVernacularWords{nomak-e-k}.

surpass-PA-3s

`He is stronger than I. (Lit: His strength surpasses my strength.)'

\ea%x1894
\label{ex:x1894}
\gll Mua  oko=ke  ikiwosa/amisa  efar  \textstyleEmphasizedVernacularWords{nomak-e-k}. \\
      \\
\glt
\z

man  other=CF  head/knowledge  1s.DAT  surpass-PA-3s

`Someone else is more intelligent than I. (Lit: {\dots}surpasses my head/knowledge.)'

In the following example, \textstyleStyleVernacularWordsItalic{nomak}- is employed to compare arrival times:

\ea%x1895
\label{ex:x1895}
\gll ...wia  \textstyleEmphasizedVernacularWords{nomak-ep}  me  miim-ep  ... \\
      \\
\glt
\z

3p.ACC  surpass-SS.SEQ  not  precede-SS.SEQ  {\dots}  

urup-i-yen,  weetak.

ascend-Np-FU.1p  no

`{\dots} we'll not go up earlier than they, no.'

For superlatives, the quantifier \textstyleStyleVernacularWordsItalic{unowiya} `all' may be used in the object \textstyleAcronymallcaps{NP}.

\ea%x1330
\label{ex:x1330}
\gll No  unuma  nain  mua  \textstyleEmphasizedVernacularWords{unow=iya}  wia  \textstyleEmphasizedVernacularWords{nomakek}. \\
      \\
\glt
\z

2s.UNM  name  that1  man  many=COM  3p.ACC  surpass-PA-3s

`You are the most important of all people.' (Lit: `Your name surpasses all people.')

In the following example, the two comparison strategies are employed in the same sentence, and the intensifier \textstyleStyleVernacularWordsItalic{akena} `very' indicates superlative:

\ea%x1337
\label{ex:x1337}
\gll Poka  fain  \textstyleEmphasizedVernacularWords{maala},  nain  \textstyleEmphasizedVernacularWords{nomak-e-k},  ne  oko  nain  \textstyleEmphasizedVernacularWords{maala} \\
      \\
\glt
\z

stilt  this  long  that1  surpass-PA-3s  ADD  other  that1  long

\textstyleEmphasizedVernacularWords{akena}.

very

`This stilt is longer than that one, and/but that other one is the longest.'

When there is a difference between things that are compared but the difference is not graded, the phrase \textstyleStyleVernacularWordsItalic{sira oko} 'different (lit: another kind)' is used to modify the noun.

\ea%x1334
\label{ex:x1334}
\gll Iwakara  \textstyleEmphasizedVernacularWords{sira  oko}  miim-ap  baurar-e-mik. \\
      \\
\glt
\z

neck  kind  other  hear-SS.SEQ  flee-PA-1/3p

`They heard a different voice and ran away.'

\ea%x1335
\label{ex:x1335}
\gll Takira  opor(a)  \textstyleEmphasizedVernacularWords{sira  oko}=ko  me  wia  maak-e. \\
      \\
\glt
\z

youngster  talk  kind  other=NF  not  3p.ACC  tell-IMP.2s

`Don't tell different things to the youngsters (from what you are supposed to tell them).'

\subsubsection{Comparison of similarity: equative constructions}\footnotemark{}
\hypertarget{RefHeading22661935131865}{}
\footnotetext{ The term ``equative construction'' is not to be confused with equative clauses discussed in 5.6.1.}
A possible outcome of comparison is that the compared items, or actions, are identical or similar rather than different. Mauwake has several ways of expressing similarity. 

For an equivalent of  `as \textstyleAcronymallcaps{ADJ} as' structure, the intensity adverb \textstyleStyleVernacularWordsItalic{pepek} 'enough' is used, often together with another intensifier.

\ea%x1331
\label{ex:x1331}
\gll No  merena  \textstyleEmphasizedVernacularWords{maneka}  yo  merena  \textstyleEmphasizedVernacularWords{iiwawun  pepek}. \\
      \\
\glt
\z

2s.UNM  foot  big  1s.UNM  foot  altogether  enough

`Your feet are big, just as big as my feet.' Or: `Your big feet are just as big as mine.'

\ea%x1332
\label{ex:x1332}
\gll Urauwa  maala  Moro  owowa  \textstyleEmphasizedVernacularWords{maala  pepek  akena}. \\
      \\
\glt
\z

hole  long  Moro  village  long  enough  very

`The hole (is) as deep as Moro village is long.'

The two most common words used in equative constructions are the deictic manner adverb \textstyleStyleVernacularWordsItalic{naap} `thus, like that' (\sectref{sec:3.9.1.3}) and the postposition \textstyleStyleVernacularWordsItalic{saarik} `like' (\sectref{sec:3.12.3}). \textstyleStyleVernacularWordsItalic{Naap} is used to compare things that are essentially the same, even identical.

\ea%x1338
\label{ex:x1338}
\gll Auwa  mia  maneka,  muuka  pun  \textstyleEmphasizedVernacularWords{naap}. \\
      \\
\glt
\z

1s/p.father  body  big  son  also  like.that

`The father is big, (and) the son is like that too.'

\ea%x1339
\label{ex:x1339}
\gll Muuka  nain  (ona)  wiawi  \textstyleEmphasizedVernacularWords{naap}. \\
      \\
\glt
\z

boy  that1  3s/p.GEN  father  like.that

`The boy/son is like his father.'

\ea%x1348
\label{ex:x1348}
\gll I  maa  en-owa  \textstyleEmphasizedVernacularWords{naap}  nain  yienak-e. \\
      \\
\glt
\z

1p.UNM  food  eat-NMZ  like.that  that1  feed.us-IMP.2s

`Give us food like that.'

Also the corresponding proximal manner adverb \textstyleStyleVernacularWordsItalic{feenap} 'like this' is used occasionally:

\ea%x1347
\label{ex:x1347}
\gll Uura  \textstyleEmphasizedVernacularWords{feenap}  nain,  wi  wilkar  nain  muf-e-mik. \\
      \\
\glt
\z

night  like.this  that1  3p.UNM  cart  that1  pull-PA-1/3p

`On nights like this they pulled the carts.'

The postposition \textstyleStyleVernacularWordsItalic{saarik} `like' expresses some similarity between two essentially different things. The actual point of similarity may be expressed explicitly (\stepcounter{nx}{\thenx}) or left implied (\stepcounter{nx}{\thenx}).

\ea%x1341
\label{ex:x1341}
\gll Pon  oposia  eliwa,  aara  oposia  \textstyleEmphasizedVernacularWords{saarik}. \\
      \\
\glt
\z

turtle  meat  good  chicken  meat  like

`Turtle meat is good, like chicken meat.'

\ea%x1340
\label{ex:x1340}
\gll Mera  iperuma  ifa  \textstyleEmphasizedVernacularWords{saarik}. \\
      \\
\glt
\z

fish  eel  snake  like

`An eel is like a snake.'

The similarity may not be a particular quality, expressable with an adjective. In the following example it is the number of different things that is compared.

\ea%x1342
\label{ex:x1342}
\gll Ulingan  fa=na  iinan  aasa  nepa  \textstyleEmphasizedVernacularWords{saarik,}  unow(a)  akena. \\
      \\
\glt
\z

Ulingan  INTJ=TP  sky  canoe  bird  like  many  very

`Ulingan -- wow -- the airplanes were like birds, there were lots of them.'

When \textstyleStyleVernacularWordsItalic{saarik} is postposed after a nominalized verb, it indicates pretension. This is a case of a similarity of action, but not ``the real thing''.

\ea%x1343
\label{ex:x1343}
\gll Moram  era  \textstyleEmphasizedVernacularWords{paayar-owa  saarik}  fan  yia  \\
      \\
\glt
\z

why  road  understand-NMZ  like  here  1p.ACC

p-or-o-n?

BPx-descend-PA-2s

`Why did you bring us down here as if you knew the road?'

\ea%x1344
\label{ex:x1344}
\gll O  Menamura  \textstyleEmphasizedVernacularWords{or-owa  saarik}  iwera  fook-a-k. \\
      \\
\glt
\z

3s.UNM  Manam  descend-NMZ  like  coconut  split-PA-3s

`He split coconuts (for copra), as if he were going to Manam.'

In other cases it may not indicate pretension but a false or ungrounded expectation: 

\ea%x1345
\label{ex:x1345}
\gll Yo  \textstyleEmphasizedVernacularWords{efa  sesenar-owa  saarik}  oram  maneka  \\
      \\
\glt
\z

1s.UNM  1s.ACC  buy-NMZ  like  for.nothing  big

uuw-owa  yoowa  on-a-m.

work-NMZ  hot  do-PA-1s

`I worked hard for nothing, as if they would pay me for it (lit: buy me).'

The phrase \textstyleStyleVernacularWordsItalic{nainiw akena} `exactly like' is reserved for the cases of striking similarity: 

\ea%x1346
\label{ex:x1346}
\gll Wiipa  nain  onak  miikapura  \textstyleEmphasizedVernacularWords{nainiw  akena}. \\
      \\
\glt
\z

girl  that1  3s/p.mother  face  again  very

`The girl's face is exactly like her mother's.'

\section{Sentence types}
\hypertarget{RefHeading22681935131865}{}

The basic speech acts are mostly expressed by the functional sentence types typical of them: a statement by a declarative sentence, a question by an interrogative sentence and a command by an imperative sentence. 

\subsection{Statements}
\hypertarget{RefHeading22701935131865}{}
The declarative sentence, used to make a statement/assertion, is the unmarked sentence type, default in narrative, descriptive and procedural texts and common in other text types as well. The final verb has full tense and person/number marking. The intonation pattern in declarative sentences is falling (\sectref{sec:2.1.3.2}). 

\subsection{Questions}
\hypertarget{RefHeading22721935131865}{}
The basic function of questions, or interrogative sentences, is to request either information or some action from the addressee(s). Rhetorical questions have other functions as well. Structurally the two basic types are non-polar, or content questions and polar, or yes-no questions. Echo questions and confirmation questions are modifications of these. 

\subsubsection{Non-polar questions}
\hypertarget{RefHeading22741935131865}{}
Non-polar questions, or content questions, require the use of question words (\sectref{sec:3.7.1}). There is no question-word fronting: a question word occupies the same position that the questioned element would have in a statement.\footnote{This is typical of Papuan SOV languages \citep[294]{Reesink1987}.} The intonation is falling like in a statement, but the stressed syllable of the question word has a slightly higher pitch than the words before and/or after it (\sectref{sec:2.1.3.2}).

Any argument or peripheral in a clause can be questioned, as well their constituents. 

\ea%x1167
\label{ex:x1167}
\gll \textstyleEmphasizedVernacularWords{(Mua)  naareke}  koora  ku-am-ika-i-ya? \\
      \\
\glt
\z

(man)  who.CF  house  build-SS.SIM-be-Np-3s

`Who is building a house?'

\ea%x1166
\label{ex:x1166}
\gll Muuka  nain  \textstyleEmphasizedVernacularWords{maa  mauwa}  enim-i-non? \\
      \\
\glt
\z

son  that1  thing  what  eat-Np-FU.3s

`What will the son eat?'

\ea%x1164
\label{ex:x1164}
\gll No  muuka  wiipa  \textstyleEmphasizedVernacularWords{kamin}  (nefar  ik-ua)? \\
      \\
\glt
\z

2s.UNM  son  daughter  how.many  (2s.DAT  be-PA.3s)

`How many children (lit. son daughter) do you have?'

\ea%x1165
\label{ex:x1165}
\gll Mua  napuma  \textstyleEmphasizedVernacularWords{moram}  owowa  p-ikiw-i-yan? \\
      \\
\glt
\z

man  sick/body  why  village  Bpx-go-Np-FU.1p

`Why should we take the body to the village?'

\ea%x1378
\label{ex:x1378}
\gll Mukuna  aw-o-k  nain  \textstyleEmphasizedVernacularWords{kamenap}  umuk-i-yen? \\
      \\
\glt
\z

fire  burn-PA-3s  that1  how  extinguish-Np-FU.1p

`How could we extinguish the fire that was burning?'

\ea%x1173
\label{ex:x1173}
\gll Maa  nain  epa  \textstyleEmphasizedVernacularWords{kain=pa}  imenar-i-non? \\
      \\
\glt
\z

thing  that1  place  which=LOC  appear-Np-FU.3s

`Where (lit: in which place) will that thing appear?'

\ea%x1174
\label{ex:x1174}
\gll Wi  \textstyleEmphasizedVernacularWords{kaakew } mua=ke  uf-e-mik? \\
      \\
\glt
\z

3p.UNM  which.village  man=CF  dance-PA-1/3p

`The men of which village danced?'

The kind of ambiguity between a subject and an object that Usan has, which arises from the fronting of a topicalized element\footnote{A \textit{theme} in my terminology here.} \citep[294]{Reesink1987}, is not possible in Mauwake. This is because the question words take the contrastive focus marker \nobreakdash-\textstyleStyleVernacularWordsItalic{ke}  when functioning as a subject. Because of elision, and the merging of the contrastive focus marker with the question word, the word for `who' in Mauwake actually has a contrasted/nominative (\stepcounter{nx}{\thenx}), (\stepcounter{nx}{\thenx}) and an accusative form (\stepcounter{nx}{\thenx}). The object is fronted in (\stepcounter{nx}{\thenx}) as a theme; in (\stepcounter{nx}{\thenx}) the object is not fronted.

\ea%x1170
\label{ex:x1170}
\gll [Mua  nain]\textsubscript{S}  [\textstyleEmphasizedVernacularWords{naarew}]\textsubscript{O}  aruf-a-k? \\
      \\
\glt
\z

man  that1  who(ACC)  hit-PA-3s

`Who did that man hit?'

\ea%x1171
\label{ex:x1171}
\gll [Mua  nain]\textsubscript{O}  [\textstyleEmphasizedVernacularWords{naareke}]\textsubscript{S}  aruf-a-k? \\
      \\
\glt
\z

man  that1  who.CF  hit-PA-3s

`Who hit that man?'

\ea%x1172
\label{ex:x1172}
\gll [\textstyleEmphasizedVernacularWords{(Mua)  naareke}]\textsubscript{S}  [mua  nain]\textsubscript{O}  aruf-a-k? \\
      \\
\glt
\z

(man)  who.CF  man  that1  hit-PA-3s

`Who hit that man?'

It is most common to have the question in a main clause, but medial clauses also easily allow non-polar questions. The scope of the question word only extends to the clause which contains it. In (\stepcounter{nx}{\thenx}) the fact that the people ran away is not questioned.

\ea%x1168
\label{ex:x1168}
\gll \textstyleEmphasizedVernacularWords{Mua  naareke  wia  aruf-eya}  baurar-e-mik? \\
      \\
\glt
\z

man  who.CF  3p.ACC  hit-2/3s.DS  run.away-PA-1/3p

`Who hit them (so that) they ran away?'

A constituent in a complement clause (with a nominalized verb) can be questioned, but not in a relative clause. 

\ea%x1175
\label{ex:x1175}
\gll \textstyleEmphasizedVernacularWords{Ama  kamin  ikiw-owa}  ma-e-mik? \\
      \\
\glt
\z

sun/time  how.much  go-NMZ  say-PA-1/3p

`At what time did they say to go?'

\ea%x1176
\label{ex:x1176}
\gll \textstyleEmphasizedVernacularWords{*Wi  iikamin  ekap-e-mik  nain}  wia  uruf-a-n? \\
      \\
\glt
\z

3p.UNM  when  come-PA-1/3p  that1  3p.ACC  see-PA-2s

Multiple constituents in the same clause can be questioned with a question word. This is not common, but the following elicited sentences are considered completely natural. 

\ea%x1177
\label{ex:x1177}
\gll Emeria  \textstyleEmphasizedVernacularWords{naareke } ama  \textstyleEmphasizedVernacularWords{kamin=pa}  ekap-o-k? \\
      \\
\glt
\z

woman  who.CF  sun/time  how.much=LOC  come-PA-3s

`Who (woman) came at what time?'

\ea%x1178
\label{ex:x1178}
\gll Mua  \textstyleEmphasizedVernacularWords{kain=ke}  emeria  \textstyleEmphasizedVernacularWords{kain}  aaw-o-k? \\
      \\
\glt
\z

man  which=CF  woman  which  take-PA-3s

`Which man took/married which woman?'

When there is a lot of hesitation in the question, the question clitic -\textstyleStyleVernacularWordsItalic{i}, which normally marks a polar question, is added to the end of the question. This is the same form that the echo questions have (7.2.3). 

\ea%x1187
\label{ex:x1187}
\gll Auwa  efa  amukar-e-k  nain  yo  \textstyleEmphasizedVernacularWords{kamenap} \\
      \\
\glt
\z

father  1s.ACC  scold-PA-3s  that  1s.UNM  how

ar-i-nen=\textstyleEmphasizedVernacularWords{i}?

become-Np-FU.1s=QM

`(I wonder) what will happen to me because father scolded me?'

\subsubsection{  Polar questions}
\hypertarget{RefHeading22761935131865}{}
Polar questions\footnote{Also called nexus questions, or yes-no questions.} expect either confirmation or negation of the questioned proposition. According to \citet[63]{Wurm1982}, a polar question in \textstyleAcronymallcaps{TNG} languages is often marked by an affix which is part of the verb complex. In Mauwake it is coded by the question clitic -\textstyleStyleVernacularWordsItalic{i}  (\sectref{sec:3.12.8}) and slightly rising intonation (\sectref{sec:2.1.3.2}), both occurring sentence-finally.  Because Mauwake is an \textstyleAcronymallcaps{SOV} language, the clitic most often attaches itself to a verb (\stepcounter{nx}{\thenx}), but it can attach to another word class as well, when there is no final verb:

\ea%x1179
\label{ex:x1179}
\gll Ni  nain  me=ko  uruf-a-man\textstyleEmphasizedVernacularWords{=i}? \\
      \\
\glt
\z

2p.UNM  that1  not=NF  see-PA-2p=QM

`Didn't you see that?'

\ea%x1180
\label{ex:x1180}
\gll Nos=\textstyleEmphasizedVernacularWords{i}? \\
      \\
\glt
\z

2s.FC=QM

`You?'

\ea%x1181
\label{ex:x1181}
\gll Maa  nain  eliwa=\textstyleEmphasizedVernacularWords{ki}? \\
      \\
\glt
\z

thing  that1  good=CF.QM

`Is that thing good?'

When the polar question is in the negative, a one-word answer may be ambiguous. Traditionally the answer either affirmed or negated the affirmative or negative \textstyleEmphasizedWords{\textsc{polarity}} of the question, but because of the influence of Tok Pisin and English, Mauwake is changing so that the answer tends to either affirm or negate the \textstyleEmphasizedWords{\textsc{verb}} (\sectref{sec:6.2.4}).

Alternative questions can be closed or open.\footnote{\citet{Haspelmath2007} calls only the former an alternative (or disjunctive) question, and the latter a question with standard disjunction.} The former give two, or sometimes more, alternatives, one of which has to be chosen; the latter also allow the possibility that none of the alternatives is chosen. The two types differ in Mauwake as to what the last alternative is like.

The non-final alternatives in a closed question take the question marker -\textstyleStyleVernacularWordsItalic{i}. The final alternative, usually preceded by the disjunctive coordinator \textstyleStyleVernacularWordsItalic{e} `or' (\sectref{sec:3.11.2}), may be just a negation particle \textstyleStyleVernacularWordsItalic{weetak} or \textstyleStyleVernacularWordsItalic{wia} (\stepcounter{nx}{\thenx}), a full statement (\stepcounter{nx}{\thenx}), or an elliptical clause with only the questioned item (\stepcounter{nx}{\thenx}).

\ea%x1182
\label{ex:x1182}
\gll Yo  emeria=ko  efar  uruf-a-man=\textstyleEmphasizedVernacularWords{i  e}  weetak? \\
      \\
\glt
\z

1s.UNM  woman=NF  1s.DAT  see-PA-2p=QM  or  no

`Did you see my wife or not?'

\ea%x1183
\label{ex:x1183}
\gll Nain  kema  suuw-i-man=\textstyleEmphasizedVernacularWords{i  e}  kema  irin-ar-e-man? \\
      \\
\glt
\z

that1  liver  push-Np-PR.2p=QM  or  liver  stuck-INCH-PA-2p

`Do you remember (lit: push the liver) that, or have you forgotten (lit: liver is stuck) it?'

\ea%x1184
\label{ex:x1184}
\gll No  Matukar  ikiw-i-nan=\textstyleEmphasizedVernacularWords{i}  Dylup=\textstyleEmphasizedVernacularWords{i  e}  Sarang? \\
      \\
\glt
\z

2s.UNM  Matukar  go-Np-FU.2s=QM  Dylup=QM  or  Sarang

`Will you go to Matukar, Dylup, or Sarang?'

When the alternative question is open, the question marker -\textstyleStyleVernacularWordsItalic{i}  marks not only the non-final alternatives but also the final one. 

\ea%x1185
\label{ex:x1185}
\gll Matukar  ikiw-i-nan=\textstyleEmphasizedVernacularWords{i  e}  Dylup  ikiw-i-nan=\textstyleEmphasizedVernacularWords{i}? \\
      \\
\glt
\z

Matukar  go-Np-FU.2s=QM  or  Dylup  go-Np-FU.2s=QM

`Will you go to Matukar or Dylup (or perhaps neither)?'

\ea%x1186
\label{ex:x1186}
\gll Mukuna=ko  wu-a-man=\textstyleEmphasizedVernacularWords{i  e}  mua=ko  wia  uruf-a-man=\textstyleEmphasizedVernacularWords{i}? \\
      \\
\glt
\z

fire=NF  put-PA-2p=QM  or  man=NF  3.ACC  see-PA-2p=QM

`Did you light a fire or did you feel (that there was) a man?'

An alternative question is left open also when the last alternative is replaced with the question word \textstyleStyleVernacularWordsItalic{kamenion} '(or) what?'/ `(or) how is it?':

\ea%x1350
\label{ex:x1350}
\gll Maa  en-owa=ko  p-ekap-e-mik=\textstyleEmphasizedVernacularWords{i}  \textstyleEmphasizedVernacularWords{kamenion}? \\
      \\
\glt
\z

food  eat-NMZ=NF  BPx-come-PA-1/3p=QM  or.what

`Did they bring food, or what?'

\ea%x1351
\label{ex:x1351}
\gll Beel(a)-al-i-non=\textstyleEmphasizedVernacularWords{i}  \textstyleEmphasizedVernacularWords{kamenion},  naap  uruf-am-ik-ua. \\
      \\
\glt
\z

rotten-INCH-Np-FU.3s=QM  or.what  thus  see-SS.SIM-be-PA.3s

`He was watching whether it would rot or what would happen.'

Leading questions are another subtype of polar questions. The person asking wants to guide the answer in a certain direction. This is done in Mauwake by adding the epistemic modal adverb clitic -\textstyleStyleVernacularWordsItalic{yon} `perhaps' to the predicate of the question clause. The slightly rising intonation in the question distinguishes it from a statement. 

\ea%x1349
\label{ex:x1349}
\gll Me  ikiw-o-k=\textstyleEmphasizedVernacularWords{yon}? \\
      \\
\glt
\z

not  go-PA-3s-perhaps

`He didn't go, did he?'

\subsubsection{Echo questions}
\hypertarget{RefHeading22781935131865}{}
Echo questions are used when an original statement or question is questioned, either because it was not properly heard in the first place or because the addressee has some doubts about it. Structurally all echo questions are polar questions.

Echo question of a statement is like a normal polar question, except that the questioned element receives an extra stess.

\ea%x1189
\label{ex:x1189}
\gll A:Paapa  Goroka  ikiw-i-non.  -  B:  Gor\'oka  ikiw-i-non=i? \\
      \\
\glt
\z

A:elder.sibling  Goroka  go-Np-FU.3p  -  B:  Goroka  go-Np-FU.3p=QM

`A: Big brother is going to Goroka. B: Is he going to \textstyleEmphasizedWords{\textsc{Goroka}}?'

When the validity of a non-polar question (\stepcounter{nx}{\thenx}) is questioned, the question clitic is attached directly to the end of the question already containing a question word (\stepcounter{nx}{\thenx}). 

\ea%x1190
\label{ex:x1190}
\gll A:  Mua  naarew  wia  maak-e-k?  \\
      \\
\glt
\z

A:  man  who  3.ACC  tell-PA-3s

`Who did he tell?'

\ea%x1191
\label{ex:x1191}
\gll B:  Mua  naarew  wia  maak-e-k=\textstyleEmphasizedVernacularWords{i}? \\
      \\
\glt
\z

B:  man  who  3.ACC  tell-PA-3s=QM

`Who did he tell???'

But if the addressee wants to check if (s)he heard correctly, the echoed question is made into a complement of a sentence-final utterance verb, which gets a question clitic attached to it (\stepcounter{nx}{\thenx}).

\ea%x1192
\label{ex:x1192}
\gll B:  Mua  naarew  wia  maak-e-k  \textstyleEmphasizedVernacularWords{na-i-n=i}? \\
      \\
\glt
\z

B:  man  who  3.ACC  tell-PA-3s  say-Np-PR.2s=QM

`Are you asking who he told?'

Since polar questions already have a clause-final question clitic, an echo question cannot be formed by adding the same clitic a second time. Instead, the original question is made into a complement of the utterance verb \textstyleStyleVernacularWordsItalic{ma}- `say' or \textstyleStyleVernacularWordsItalic{na}- `say/think'. 

\ea%x1193
\label{ex:x1193}
\gll A:  Nain  eliwa=ki?  B:  Nain  eliwa-ki  \textstyleEmphasizedVernacularWords{ma-e-n=i}? \\
      \\
\glt
\z

A:  that1  good=CF.QM  B:  that1  good=CF.QM  say-PA-2s=QM

`A: Is that good?  B: Did you ask if that is good?'

\subsubsection{Confirmation questions}
\hypertarget{RefHeading22801935131865}{}
Confirmation questions are mainly used in argumentation. The question word \textstyleStyleVernacularWordsItalic{naap}\textstyleStyleVernacularWordsItalic{-i}  `(is it) like that?' is tagged to a statement, which may be preceded by another question. 

\ea%x1188
\label{ex:x1188}
\gll Ni  kema  maneka  naap  efa  wu-i-man=i,  \\
      \\
\glt
\z

2p.UNM  liver  big  thus  1s.ACC  put-Np-PR.2p=QM

yo  eliw  nia  saliw-i-nen,  \textstyleEmphasizedVernacularWords{naap=i}?

1s.UNM  well  2p.ACC  heal-Np-FU.1s  thus=QM

`Do you believe about me that I can heal you, is that so?'

\subsubsection{Indirect questions}
\hypertarget{RefHeading22821935131865}{}
Indirect questions are a subgroup of complement clauses and are discussed under \textstyleEmphasizedWords{\textsc{Indirect speech}} in  \sectref{sec:8.3.2.1.2}.

\ea%x1203
\label{ex:x1203}
\gll [Yo  maa  mauwa  uruf-a-m]  efa  na-e-k. \\
      \\
\glt
\z

1s.UNM  thing  what  see-PA-1s  1s.ACC  say-PA-3s

`He asked me what I saw.'

\ea%x1204
\label{ex:x1204}
\gll [Kamin  wu-a-mik(-yon)],  yo  me  wiar  \\
      \\
\glt
\z

how.much  put-PA-1/3p-perhaps  1s.UNM  not  3.DAT

amis-ar-e-m.

knowledge-INCH-PA-1s

`I don't know how much they put.'

\subsubsection{Rhetorical questions}
\hypertarget{RefHeading22841935131865}{}
Traditionally the Mauwake speakers lived in a society where everyone more or less knew everybody's business and there was not much need for eliciting information by asking questions. Consequently, many questions in normal speech are rhetorical in nature. The question form may be used to emphasise the opposite of what is said, or sometimes just to prompt the addressee to think more clearly, but very often rhetorical questions have an element of reproach or assigning blame as well.

\ea%x1205
\label{ex:x1205}
\gll Maamuma  kaaneke  ika-eya  ni-i-yan? \\
      \\
\glt
\z

money  where.CF  be-2/3s.DS  give.you-Np-FU.1p

`Where would we have that kind of money to give you? (=We do not have money to give you.)'

\ea%x1206
\label{ex:x1206}
\gll Yo  anane  niam=iya  ika-i-nen=i? \\
      \\
\glt
\z

1s.UNM  always  2p.REFL=COM  be-Np-FU.1s=QM

`Will I be with you forever? (= I will not.)'

\ea%x1202
\label{ex:x1202}
\gll No  moram  naap  om-em-ika-i-n? \\
      \\
\glt
\z

2s.UNM  why  thus  cry-SS.SIM-be-Np-PR.2s

`Why are you crying like that? (=You should not cry like that.)'

\ea%x1207
\label{ex:x1207}
\gll Mua  naareke  nia  maak-eya  ekap-e-man? \\
      \\
\glt
\z

man  who.CF  2p.ACC  say-2/3p.DS  come-PA-2p

`Who told you to come? (=You shouldn't have come)'

Implied reproach or accusation is particularly common with  questions including the word \textstyleStyleVernacularWordsItalic{moram} `why?', but it is not limited to them. Especially accusations of theft are couched in neutral-looking questions (\stepcounter{nx}{\thenx}).

\ea%x1208
\label{ex:x1208}
\gll Aa  muuka,  no  moram  naap  yia  on-a-n? \\
      \\
\glt
\z

oh  son  2s.UNM  why  thus  1s.ACC  do-PA-2s

`Oh son, why did you do this to us?'

\ea%x1209
\label{ex:x1209}
\gll Yo  seewa  gelemuta  uruma  or-o-k  nain  uruf-a-man=i? \\
      \\
\glt
\z

1s.UNM  rat  small  valley  descend-PA-3s  that1  see-PA-2p=QM

`Have you seen my ``little rat'' (pig) that went down to the valley? (implying: I have no doubt that you have stolen my pig.)'

Because questions are so easily understood as reproaches or accusations, real questions are often preceded by a preamble to prevent this interpretation.

\ea%x1210
\label{ex:x1210}
\gll [Ama  arow=pa  mauw-owa  weeser-eya]  maa  mauwa  on-a-man? \\
      \\
\glt
\z

sun  three=LOC  work-NMZ  finish-2/3s.DS  thing  what  do-PA-2p

`After your work finished at three, what did you do?'

\ea%x1211
\label{ex:x1211}
\gll [Yo  oram  nefa  nokar-i-yem],  soomia=ko  efar  \\
      \\
\glt
\z

1s.UNM  just  2s.ACC  ask-Np-PR.1s  spoon=NF  1s.DAT

uruf-a-n=i?

see-PA-2s=QM

`I'm just asking: have you seen my spoon?'

\ea%x1355
\label{ex:x1355}
\gll Anane  maneka  ewur  me  urup-i-n  nain  moram? \\
      \\
\glt
\z

always  big  quickly  not  ascend-Np-PR.2s  that1  why

`What is the reason why you never come up quickly?'

\subsubsection{Answers to questions}
\hypertarget{RefHeading22861935131865}{}
Apart from rhetorical questions, a verbal answer is often expected. An affirmative answer to a polar question (\stepcounter{nx}{\thenx}) may be just an affirmative interjection (\stepcounter{nx}{\thenx}) or the verb from the question by itself or preceded by the interjection (\stepcounter{nx}{\thenx}). A negative answer must have at least one negator, whether only a negative interjection (\sectref{sec:6.2.3}), or any of the other negators, or both (\stepcounter{nx}{\thenx}). Less commonly the answer may also be a full statement with or without a preceding affirmation (\stepcounter{nx}{\thenx})  or negation.

\ea%x1216
\label{ex:x1216}
\gll No  uurika  owow  maneka  ikiw-i-nan=i? \\
      \\
\glt
\z

2s.UNM  tomorrow  village  big  go-Np-FU.2s=QM

`Are you going to town tomorrow?'

\ea%x1217
\label{ex:x1217}
\gll Ae/Oo. \\
      \\
\glt
\z

yes

`Yes.'

\ea%x1218
\label{ex:x1218}
\gll (Ae,)  ikiw-i-nen. \\
      \\
\glt
\z

yes  go-Np-FU.1s

`(Yes,) I am going.'

\ea%x1220
\label{ex:x1220}
\gll (Weetak,)  me  ikiw-i-nen. \\
      \\
\glt
\z

no  not  go-Np-FU.1s

`(No,) I am not going.'

\ea%x1219
\label{ex:x1219}
\gll (Ae,)  yo  uurika  owow  maneka  ikiw-i-nen. \\
      \\
\glt
\z

yes  1s.UNM  tomorrow  village  big  go-Np-FU.1s

`(Yes,) I'll go to town tomorrow.'

The reply to a non-polar question most typically includes an answer to the questioned item and often the verb of the original question too.

\ea%x1221
\label{ex:x1221}
\gll Maa  sira  kamenap  nain  en-em-ik-e-man? \\
      \\
\glt
\z

thing/food  kind  how  that1  eat-SS.SIM-be-PA-2p

`What kind of food did you eat?'

\ea%x1222
\label{ex:x1222}
\gll Wi  mia  kia  en-owa  nain  (en-em-ik-e-mik). \\
      \\
\glt
\z

3p.UNM  body  white  eat-NMZ  that1  (eat-SS.SIM-be-PA-1/3p)

`(We ate) the white people's food.'

If the speaker wants to negate the presupposition in the question, (s)he begins with a negator, and then goes on to answer the question itself (\stepcounter{nx}{\thenx}).

\ea%x1223
\label{ex:x1223}
\gll Neremena  kamenap  nefa  on-a-k? \\
      \\
\glt
\z

2s/p.nephew  how  2s.ACC  do-PA-3s

`What did your nephew do to you?'

\ea%x1224
\label{ex:x1224}
\gll \textstyleEmphasizedVernacularWords{Weetak},  yo  mauw-a-m  ne  o  me  efa  \\
      \\
\glt
\z

no  1s.UNM  work-PA-1s  ADD  3s.UNM  not  1s.ACC  

uruf-a-k.

see-PA-3s

`I worked but he did not even look at me.' (Implying: Your presupposition is wrong; he did not do anything indecent to me.)

If the question or statement itself is negative, a one-word answer is ambiguous in present-day usage, and a full clause is needed to disambiguate it. Traditionally an answer to a question affirmed or negated the affirmative or negative \textstyleEmphasizedWords{\textsc{polarity}} of the question or statement:

\ea%x1151
\label{ex:x1151}
\gll O  aakun-owa  marew=yon.  -\textstyleEmphasizedVernacularWords{Wia},  aakun-owa  wiar  \\
      \\
\glt
\z

3s.UNM  talk-NMZ  no(ne)-perhaps  -no  talk-NMZ  3.DAT

ik-ua.

be-PA.3s

`Perhaps he doesn't have anything to say. --No, he \textstyleEmphasizedWords{\textsc{does}} have something to say.'

\ea%x1117
\label{ex:x1117}
\gll Auwa  me  ekap-o-k=i?  -\textstyleEmphasizedVernacularWords{Weetak}  (ekap-o-k). \\
      \\
\glt
\z

1s/p.father  not  come-PA-3s=QM  -no  (come-PA-3s)

`Didn't father come? --Yes (he \textstyleEmphasizedWords{\textsc{did}}).'

But Mauwake is changing to become more like English\footnote{A similar change is taking place in Tok Pisin, and it is likely that this is causing the development in Mauwake too.} in that the negative answer stands for a negative statement regardless of the polarity of the question or statement that it is a reply to: 

\ea%x1118
\label{ex:x1118}
\gll Auwa  me  ekap-o-k=i?  -\textstyleEmphasizedVernacularWords{Weetak}  (me  ekap-o-k). \\
      \\
\glt
\z

1s/p.father  not  come-PA-3s=QM  -no  (not  come-PA-3s)

`Didn't father come? --No (he didn't).'

\subsection{Commands}
\hypertarget{RefHeading22881935131865}{}
The simple imperative is the default way of expressing a command in Mauwake. It shows in the verb inflection (\sectref{sec:3.8.3.3.2}). In a prohibition the verbal negator \textstyleStyleVernacularWordsItalic{me} `not' precedes the simple imperative (\stepcounter{nx}{\thenx}).

\ea%x1072
\label{ex:x1072}
\gll Ni  Medebur  \textstyleEmphasizedVernacularWords{karu-eka},  \textstyleEmphasizedVernacularWords{baurar-eka}. \\
      \\
\glt
\z

2p.UNM  Medebur  run-IMP.2p  flee-IMP.2p

`Run(pl.) to Medebur, flee.'

\ea%x1075
\label{ex:x1075}
\gll Momora,  no  naap  \textstyleEmphasizedVernacularWords{me  ma-e}. \\
      \\
\glt
\z

fool  2s.UNM  thus  not  say-IMP.2s

`Fool, don't say like that.'

The simple imperative can be strengthened with the intensity adverb \textstyleStyleVernacularWordsItalic{akena} `very, truly' following the verb.

\ea%x1073
\label{ex:x1073}
\gll Ni  sira  samora  \textstyleEmphasizedVernacularWords{piipu-eka  akena}. \\
      \\
\glt
\z

2p.UNM  habit  bad  leave-IMP.2p  truly

`Really get rid of your bad habits.'

Another way to intensify it is with the clause-final interjection \textstyleStyleVernacularWordsItalic{nom} `\textstyleEmphasizedWords{\textsc{please}}!', which is only used when a person has already been told to do something at least once and has not complied.

\ea%x1074
\label{ex:x1074}
\gll \textstyleEmphasizedVernacularWords{Pootin-e,  nom}! \\
      \\
\glt
\z

stop.crying-IMP.2s  please

`Stop crying, \textstyleEmphasizedWords{\textsc{please}}!'

The imperative marking on verbs \textstyleStyleParagraphSILDoulosUnicodeIPAChar{shows} only in the finite forms. When a command or request is in a medial clause, and the final clause verb is in the indicative mood and future tense, there is nothing in the medial verb to indicate the mood. 

\ea%x1076
\label{ex:x1076}
\gll No  opaimika  pon  aaw-o-n  nain  \textstyleEmphasizedVernacularWords{ma-eya} \\
      \\
\glt
\z

2s.UNM  talk  turtle  get-PA-3s  that1  tell-2/3s.DS

i  miim-i-yen.

1p.UNM  hear-Np-FU.1p

`Tell us about your catching a turtle, and we'll listen.' (Or: `You will tell us about your catching a turtle and we'll listen.')

This type of clause combination has given rise to a softer, less direct command, which is given with a medial different-subject form of a verb; the final clause is left out altogether.\footnote{This fairly common usage of a medial verb form in Papuan languages is probably the origin of the use of \textit{pastaim} `first' in Tok Pisin commands, e.g. \textit{Kam pastaim} `Come!'}  This form is particularly common when commands are given to children. 

\ea%x1084
\label{ex:x1084}
\gll P-ekap-\textstyleEmphasizedVernacularWords{eya}! \\
      \\
\glt
\z

Bpx-come-2/3s.DS

`Bring it!'

The imperative of the final clause may have an influence on the medial clause(s) so that they, too, are interpreted as belonging within the scope of the command. This happens very easily with same-subject medial verbs (\stepcounter{nx}{\thenx}); it is also possible but much less likely when the subject changes (\stepcounter{nx}{\thenx})\footnote{This example may also be interpreted to have two commands, a ``soft'' one, expressed with a medial verb, and a regular one.}. (\stepcounter{nx}{\thenx}) is ambiguous: in the situation where was said, the medial clause was not in the scope of the final clause imperative; in some other situation it could be. When the medial verb has a first person form, imperative interpretation is not possible (\stepcounter{nx}{\thenx}). 

\ea%x1082
\label{ex:x1082}
\gll Emeria  manina  \textstyleEmphasizedVernacularWords{ikiw-ep}  en-owa  \textstyleEmphasizedVernacularWords{nop-ap  or-eka}. \\
      \\
\glt
\z

woman  garden  go-SS.SEQ  eat-NMZ  search-SS.SEQ  descend-IMP.2p

`Women, go to the garden, look for food and come down.'

\ea%x1364
\label{ex:x1364}
\gll Mua  emeria  wia  \textstyleEmphasizedVernacularWords{maak-eya}  me  efa  \textstyleEmphasizedVernacularWords{enim-uk}. \\
      \\
\glt
\z

man  woman  3p.ACC  tell-2/3s.DS  not  2s.ACC  eat-IMP.3p

`Tell the people and let them not eat me.'

\ea%x1846
\label{ex:x1846}
\gll Feeke  wiar  \textstyleEmphasizedVernacularWords{ik-ok}  kiiriw  mua  wiar  \textstyleEmphasizedVernacularWords{urup-e}. \\
      \\
\glt
\z

here.CF  3.DAT  be-SS  again  man  3.DAT  ascend-IMP.2s

`Having been here with him (=your brother), go up to your husband again.'

\ea%x1083
\label{ex:x1083}
\gll I  or-op  ununa  \textstyleEmphasizedVernacularWords{anum-amkun}  \textstyleEmphasizedVernacularWords{ma-eka},  ``{\dots''} \\
      \\
\glt
\z

1p.UNM  descend  slit.gong  beat-1s/p.DS  say-IMP.2p

`When we go down and beat the slit gong, say, ``{\dots}'' '

A special feature in Mauwake commands is that they occur with a pronominal  subject more often than statements do (\sectref{sec:3.5.2.1}, 3.5.11). 

Although a command is usually directed towards one or more people in the second person, it can also be directed towards self as part of a group of two (\stepcounter{nx}{\thenx}) or more (\stepcounter{nx}{\thenx}), or towards a third person in singular (\stepcounter{nx}{\thenx}) or plural (\stepcounter{nx}{\thenx}). 

\ea%x1157
\label{ex:x1157}
\gll Aria,  i  owowa=ko  \textstyleEmphasizedVernacularWords{or-u}. \\
      \\
\glt
\z

alright  1p.UNM  village=NF  descend-IMP.1d

`Alright, let's go down to the village.'

\ea%x1158
\label{ex:x1158}
\gll Ikiw-ep=ko  wia  \textstyleEmphasizedVernacularWords{uruf-ikua}. \\
      \\
\glt
\z

go-SS.SEQ=NF  3p.ACC  see-IMP.1p

`Let's go and see them.'

\ea%x1159
\label{ex:x1159}
\gll Womokowa  me  wia  \textstyleEmphasizedVernacularWords{maak-inok}. \\
      \\
\glt
\z

3s/p.brother  not  3p.ACC  tell-IMP.3s

`Let her not talk to her brothers.'

\ea%x1160
\label{ex:x1160}
\gll Ona  mua  owawiya  ek-ap  uruf-am-ik-ok  \\
      \\
\glt
\z

3s.GEN  man  with  go-SS.SEQ  see-SS.SIM-be-SS

\textstyleEmphasizedVernacularWords{ep-am-ika-uk.}

come-SS.SIM-be-IMP.3p

`Let her with her husband keep going, seeing him and coming back.'

Imperatives cannot have tense distinctions, but aspectual distinctions are possible. The continuous aspect form is used for habitual in (\stepcounter{nx}{\thenx}) and for continuous aspect in (\stepcounter{nx}{\thenx}). Completive aspect is used in (\stepcounter{nx}{\thenx}) and stative in (\stepcounter{nx}{\thenx}).

\ea%x1896
\label{ex:x1896}
\gll Sira naap \textstyleEmphasizedVernacularWords{on-am-ik-eka}. \\
      \\
\glt
\z

custom thus do-SS.SIM-be-IMP.2p

`freetranslation'

\ea%x1161
\label{ex:x1161}
\gll Aakisa  naap  \textstyleEmphasizedVernacularWords{on-ap-pu-e}. \\
      \\
\glt
\z

now  thus  do-SS.SEQ-CMPL-IMP.2s

`Now do that.'

\ea%x1162
\label{ex:x1162}
\gll No  me  mokoka  \textstyleEmphasizedVernacularWords{opar-ep-ik-e}. \\
      \\
\glt
\z

2s.UNM  not  eye  close-SS.SEQ-be-IMP.2s

`Don't have/keep your eyes closed.'

The second person future tense form is also used for a command, but this is not very common. It is used in a specific situation, not for giving generic commands or rules. The sentence (\stepcounter{nx}{\thenx}) was said to a person who was suspected of lying, and in (\stepcounter{nx}{\thenx}) parents give instructions to their daughter how to mourn.

\ea%x1080
\label{ex:x1080}
\gll No  \textstyleEmphasizedVernacularWords{me  sail-i-nan}! \\
      \\
\glt
\z

2s.UNM  not  lie-Np-FU.2s

`Don't lie!'

\ea%x1081
\label{ex:x1081}
\gll Naap  ma-emi  \textstyleEmphasizedVernacularWords{om-em-ika-i-nan}  na. \\
      \\
\glt
\z

thus  say-SS.SIM  cry-SS.SIM-be-Np-FU.2s  INTJ

`Say like that and wail.'

\section{Clause combinations}
\hypertarget{RefHeading22901935131865}{}
Some linguistic models, the mainstream generative grammar in particular, disregard the distinction between a clause and a sentence, but here the distinction is maintained. One of the main reasons is the medial clause system in Mauwake.  A simple sentence consists of one clause, but if that is a verbal clause, it must be a finite clause, not a medial one; medial clauses only function within a sentence in combination with other clauses. Their distribution is restricted to non-final position in a sentence -- they may occur sentence-finally only if they are dislocated. Medial clauses also add the chaining structure to the clause combination possibilities (\sectref{sec:8.2}), besides regular coordination (\sectref{sec:8.1}) and subordination (\sectref{sec:8.3}). 

A sentence consists of one or more clauses. The end of a sentence is marked in speech by a falling intonation, or by a slightly rising intonation in polar questions, and normally a pause. The sentence-final falling intonation is very clear, and can be distinguished from a less noticeable fall at the end of a non-final finite clause. In writing the end of a sentence is marked by a full stop, a question mark or an exclamation mark.

A simple sentence is the same as a clause, and was discussed in Chapter 5.

When two main clauses are joined in a coordinate sentence, they are independent of each other as to their functional sentence type. In (\stepcounter{nx}{\thenx}) the first clause is declarative and the second one interrogative; in (\stepcounter{nx}{\thenx}) the first clause is imperative and the second one declarative, but the order could also be reverse. 

\ea%x1352
\label{ex:x1352}
\gll Yo  owora=ko  me  aaw-e-m,  no  moram  efa \\
      \\
\glt
\z

1s.UNM  betelnut=NF  not  take-PA-1s  2s.UNM  why  1s.ACC

ma-i-n?

say-Np-PR.2s

`I didn't take the betelnut, why do you accuse me?'

\ea%x1358
\label{ex:x1358}
\gll Ni  uf-owa  ikiw-eka,  yo  miatin-i-yem. \\
      \\
\glt
\z

2p.UNM  dance-NMZ  go-IMP.2p  1s.UNM  dislike-Np-PR.1s

`(You) go to dance, I don't want to.'

In clause chaining (\sectref{sec:8.2}) and in complex clauses involving main and subordinate clauses (\sectref{sec:8.3}) the situation is more complicated. Formally almost all of the subordinate and medial clauses are neutral/declarative. A subordinate clause typically lacks an illocutionary force of its own \citep[32]{Cristofaro2003} and conforms to the functional sentence type of the main clause. In the following examples the subordinate clauses are in brackets.

\ea%x1357
\label{ex:x1357}
\gll [Ni  ifa  nia  keraw-i-ya  nain]  sira  kamenap \\
      \\
\glt
\z

2p.UNM  snake  2p.ACC  bite-Np-PR.3s  that1  custom  what.like

on-i-man?

do-Np-PR.2p

`When a snake bites you, what do you do?'

\ea%x1897
\label{ex:x1897}
\gll Ni  [yapen  ...  wiar  in-em-ik-e-man  nain]  \\
      \\
\glt
\z

2p.UNM  inland  {\dots}  3.DAT  sleep-SS.SIM-be-PA-2p  that1

kerer-omak-eka!

arrive-DISTR/PL-2p.IMP

`Those (many) of you, who have stayed inland, arrive (back in your villages)!'

\ea%x1898
\label{ex:x1898}
\gll [Ni  uf-ep-na]  ni  maadara  me  \\
      \\
\glt
\z

2p.UNM  dance-SS.SEQ=TP  2p.UNM  forehead.ornament  not  

iirar-eka.

remove-2p.IMP

`If/when you have danced, do not remove your forehead ornaments.'

The non-polar questions are an exception, since the question word may also be in a subordinate clause (\stepcounter{nx}{\thenx}). When a subordinate clause contains a question word, the illocutionary force of a question spreads to whole sentence. 

\ea%x1362
\label{ex:x1362}
\gll No  [\textstyleEmphasizedVernacularWords{kaaneke  ikiw-owa}]  efa  maak-i-n? \\
      \\
\glt
\z

2s.UNM  where.CF  go-NMZ  1s.UNM  tell-Np-PR.2s

`You are telling me to go where?'

A medial clause is coordinate with the main clause but dependent on it (\sectref{sec:8.2}). The imperative form is only possible in finite verbs, and the polar question marker only occurs sentence-finally.\footnote{As an alternative marker the QM is used in non-final clauses as well (\sectref{sec:3.1.2.8}, 8.1.2).} Because of these formal restrictions it is impossible to have an imperative or interrogative medial clause coordinated with a declarative main clause. A medial clause commonly conforms to the illocutionary force of the final clause, but it does not need to do so. In the examples (\stepcounter{nx}{\thenx}) and (\stepcounter{nx}{\thenx}) the bracketed medial clause is questioned with the main clause, in (\stepcounter{nx}{\thenx}) and (\stepcounter{nx}{\thenx}) it is not.  

\ea%x1899
\label{ex:x1899}
\gll [Maamuma  uruf-ap]  ma-i-n-i? \\
      \\
\glt
\z

money  see-SS.SEQ  say-PA-2s=QM

`Have you seen the money and (so) ask?'

\ea%x1900
\label{ex:x1900}
\gll [Yo  pina  on-amkun=ko]  efa  uruf-a-man=i? \\
      \\
\glt
\z

1s.UNM  guilt  do-1s/p.DS=NF  2s.ACC  see-PA-2p=QM

`Did I do wrong and you saw me?'

\ea%x1901
\label{ex:x1901}
\gll [Sande  erup  weeser-eya]  owowa  ekap-e-man=i? \\
      \\
\glt
\z

week  two  finish-2/3s.DS  village  come-PA-2p=QM

`When two weeks were finished, did you (then) come to the village?'

\ea%x1902
\label{ex:x1902}
\gll [...ikoka  ekap-ep]  sira  nain  piipua-i-nan=i  e  weetak? \\
      \\
\glt
\z

later  come-SS.SEQ  habit  that1  leave-Np-FU.2s=QM  or  no

`{\dots}later when you come, will you drop that habit or not?'

When a medial clause itself contains a question word, the illocutionary force spreads to the whole sentence. 

\ea%x1363
\label{ex:x1363}
\gll [\textstyleEmphasizedVernacularWords{No  maa  mauwa  uruf-ap}]  soran-ep  kirir-e-n? \\
      \\
\glt
\z

2s.UNM  thing  what  see-SS.SEQ  be.startled-SS.SEQ  shout-PA-2s

`What did you see and (then) got startled and shouted?'

\ea%x1903
\label{ex:x1903}
\gll [\textstyleEmphasizedVernacularWords{Naareke}  \textstyleEmphasizedVernacularWords{nia  maak-eya}]  ekap-e-man? \\
      \\
\glt
\z

who.CF  2p.ACC  tell-2/3s.DS  come-PA-2p

`Who told you to come?' (Lit: `Who told you and you came?)

When the final clause is in the imperative mood, the implication of a command often extends backwards to a medial verb marked for the same subject (\stepcounter{nx}{\thenx}), but not so easily to one marked for a different subject. In (\stepcounter{nx}{\thenx}) above the command /request extends to the medial clause, whereas in (\stepcounter{nx}{\thenx}) it does not. For more examples, see (\stepcounter{nx}{\thenx})-(\stepcounter{nx}{\thenx}) above.

\ea%x1365
\label{ex:x1365}
\gll [\textstyleEmphasizedVernacularWords{No}  \textstyleEmphasizedVernacularWords{nena  maa  fariar-ep}]  \textstyleEmphasizedVernacularWords{muuka  nain} \\
      \\
\glt
\z

2s.UNM  2s.GEN  food  abstain-SS.SEQ  son  that1

\textstyleEmphasizedVernacularWords{arim-ow-e}.

grow-CAUS-IMP.2s

`Abstain from (certain) food(s) and bring up the son.'  

\ea%x1356
\label{ex:x1356}
\gll [Nefa  war-iwkin]  \textstyleEmphasizedVernacularWords{naap  ma-e}. \\
      \\
\glt
\z

2s.ACC  shoot-2/3p.DS  thus  say-IMP.2s

`(If/when) they shoot you, (then) say like that.'

Although it is impossible to have an imperative verb form in a medial clause, a ``soft'' command/request (\sectref{sec:7.3}) may be used in medial clauses, as it takes the medial verb form. In (\stepcounter{nx}{\thenx}) the first clause is a request, the second one a statement.

\ea%x1366
\label{ex:x1366}
\gll Aite,  [\textstyleEmphasizedVernacularWords{i  aaya=ko  yia  aaw-om-aya}]  \\
      \\
\glt
\z

1s/p.mother  1p.UNM  sugarcane=NF  1p.ACC  get-BEN-BNFY2.2/3s.DS

enim-i-yan.

eat-Np-FU.1p

`Mother, get us sugarcane and we will eat it.'

\subsection{Coordination of clauses}
\hypertarget{RefHeading22921935131865}{}
Coordination links units of ``equivalent syntactic status'' \citep[93]{Crystal1997}. Clausal coordination commonly refers to the coordination of main clauses, as that is much more frequent than the coordination of subordinate clauses. In the following, too, it is assumed that the discussion is about main clause coordination unless stated otherwise.

The main clauses joined by coordination are independent in the sense that they could stand alone as individual sentences. And the examples (\stepcounter{nx}{\thenx}) and (\stepcounter{nx}{\thenx}) above show that they can even manifest different functional sentence types. But they are called clauses 1) because they are coordinated within one sentence and 2) for the sake of consistency, since the coordinated medial (\sectref{sec:8.2.1}) and subordinate clauses (\sectref{sec:8.3.6.1}) could not be called sentences.

As Giv\'on (1990:848) points out, no clause in a text is truly independent from its context. Likewise, the coordination vs. subordination of clauses is in many languages a matter of degree rather than a clear-cut distinction. 

Although chaining medial and final clauses (\sectref{sec:8.2}) is the main strategy for combining clauses in Mauwake, coordination of main clauses is also common. It is used not only for the cross-linguistically typical cases of conjunction, disjunction and adversative relations between clauses, but also for causal and consecutive relations.  

\subsubsection{Conjunction}
\hypertarget{RefHeading22941935131865}{}
Conjunction is the most neutral form of coordination: two or more clauses are joined in a sentence, with or without a link between them. If there is a link, it is a pragmatic additive that does not specify the semantic relationship between the clauses. This sometimes allows different interpretations for the relationship, but usually the context constrains the interpretation considerably. 

\paragraph[Juxtaposition]{Juxtaposition}
\hypertarget{RefHeading22961935131865}{}
In juxtaposition\footnote{Also called ``zero strategy'' by \citet[25]{Payne1985}.} two or more clauses are joined without any linking device at all. According to \citet[8]{Haspelmath2007} unwritten languages tend to lack their own coordinators and therefore use more juxtaposition and/or coordinators borrowed  from other, more prestigious languages. 

In Mauwake, juxtaposition is the most typical strategy for conjunction overall. Especially the coordination of verbless clauses is often symmetrical: the reversal of the conjuncts is possible without a change of meaning. 

\ea%x1367
\label{ex:x1367}
\gll Wi  Yaapan  emeria  weetak,  mua  manek=iw. \\
      \\
\glt
\z

3p.UNM  Japan  woman  no  man  big=LIM

`The Japanese didn't have any wives, (they were) just the men.'

\ea%x1390
\label{ex:x1390}
\gll Kuuten  wiawi  iperowa,  yo  auwa  kapa=ke. \\
      \\
\glt
\z

Kuuten  3s/p.father  firstborn  1s.UNM  1s/p.father  lastborn=CF

`Kuuten's father was the firstborn (son), my father was the lastborn.'

Also, when one of the conjuncts is a verbless clause and another is a verbal one, symmetrical conjunction is quite common:

\ea%x1391
\label{ex:x1391}
\gll I  uruwa  miim-i-mik,  ni  sosora=ke. \\
      \\
\glt
\z

1p.UNM  loincloth  precede-Np-PR.1/3p  2p.UNM  grass.skirt=CF

`We father's side of the family (lit: loincloth) go first, you are mother's side (lit: grass skirt).'

Symmetrical conjunction of verbal clauses may be used, when there is parallelism between the clauses:

\ea%x1368
\label{ex:x1368}
\gll Na-emi  wi  afa  ar-omak-e-mik,  \\
      \\
\glt
\z

say-SS.SIM  3p.UNM  flying.fox  become-DISTR/PL-PA-1/3p

osaiwa  ar-e-mik,  biri-birin-e-mik.

bird.of.paradise  become-PA-1/3p  RDP-fly-PA-1/3p

`Saying so, they became many flying foxes, they became birds of paradise, they flew (away).'

\ea%x1392
\label{ex:x1392}
\gll Aria  makera  miirifa  okaiwi  soo=pa  kaik-i-mik,  \\
      \\
\glt
\z

alright  cane  end  other.side  trap=LOC  tie-Np-PR.1/3p

okaiwi  pia  kaik-i-mik.

other.side  bamboo  tie-Np-PR.1/3p

`Alright we tie one end of the cane to the trap, the other to a (piece of) bamboo.'

In the following example the medial clause relates to both of the final clauses, not just to the first one:

\ea%x1851
\label{ex:x1851}
\gll Koora-pa  efa  uruf-am-ik-eya  \textstyleEmphasizedVernacularWords{ikiw}\textstyleEmphasizedVernacularWords{-}\textstyleEmphasizedVernacularWords{i}\textstyleEmphasizedVernacularWords{-}\textstyleEmphasizedVernacularWords{nen} \\
      \\
\glt
\z

house=LOC  1s.ACC  see-SS.SIM-be-2/3s.DS  go-Np-FU.1s  

\textstyleEmphasizedVernacularWords{ekap}\textstyleEmphasizedVernacularWords{-}\textstyleEmphasizedVernacularWords{i}\textstyleEmphasizedVernacularWords{-}\textstyleEmphasizedVernacularWords{nen}.

come-Np-FU.1s

`You see me from the house and/as I will go and come.'

When the coordination is not symmetrical, the clause in the second conjunct is an example or an explanation of the first clause (\stepcounter{nx}{\thenx}), or it follows the first one in a temporal sequence (\stepcounter{nx}{\thenx}).

\ea%x1370
\label{ex:x1370}
\gll Auwa  aite  wia  karu-i-yen,  owowa=pa  \\
      \\
\glt
\z

1s/p.father  1s/p.mother  3p.ACC  visit-Np-FU.1p  village=LOC

wia  uruf-u.

3p.ACC  see-1d.IMP

`We'll visit my parents, let's see them in the village.'

\ea%x1369
\label{ex:x1369}
\gll Miiw-aasa  um-eya  miiw-aasa  nain  on-am-ika-iwkin  \\
      \\
\glt
\z

land-canoe  die-2/3s.DS  land-canoe  that1  do-SS.SIM-be-2/3p.DS

\textstyleEmphasizedVernacularWords{epa  kokom(a)-ar-e-k,  epa  iimeka  tuun-e-k}.

place  dark-INCH-PA-3s  place  ten  count?-PA-3s

`The truck broke and while they were fixing the truck it became dark, (then) it was midnight.'

A fairly common structure is one where the first conjunct is not directly followed by another finite clause but by one or more medial clauses before the final clause:

\ea%x1371
\label{ex:x1371}
\gll \textstyleEmphasizedVernacularWords{Ikemika}  \textstyleEmphasizedVernacularWords{kaik-ow(a)  mua  nain  nop-a-mik},  imen-ap  \\
      \\
\glt
\z

wound  tie-NMZ  man  that1  search-PA-1/3p  find-SS.SEQ

maak-iwkin  \textstyleEmphasizedVernacularWords{o  miim-o-k}.

tell-2/3p.DS  3s.UNM  precede-PA-3s

`They looked for the medical orderly, and when they found him and told him, he went ahead of them.'

Juxtaposition in itself is neutral and only shows that the two or more clauses are somehow connected with each other, but it can be used when propositions joined by it have different semantic relationships with each other.

\ea%x1404
\label{ex:x1404}
\gll Waaya  maneka  marew  pun,  mua  unowa  me  wia  \\
      \\
\glt
\z

pig  big  no(ne)  also  man  many  not  3p.ACC

pepek-er-a-k.

enough-INCH-PA-3s

 `Also, the pig was not big, (so) it was not enough for many people.'

\ea%x1425
\label{ex:x1425}
\gll Ni  iperuma  fain  me  enim-eka,  inasin(a)  mua=ke. \\
      \\
\glt
\z

2p.UNM  eel  this  not  eat-IMP.2p  spirit  man=CF

`Don't eat this eel, (because) it is a spirit man.'

\paragraph[Conjunction with coordinating connectives]{Conjunction with coordinating connectives}
\hypertarget{RefHeading22981935131865}{}
Two of the three pragmatic connectives (\sectref{sec:3.11.1}) are used as clausal coordinators: the additive \textstyleStyleVernacularWordsItalic{ne}  and \textstyleStyleVernacularWordsItalic{aria} `alright' which marks a break in the topic chain. \textstyleStyleVernacularWordsItalic{Ne} can be used in some of the contexts where mere juxtaposition is also used, but it is less frequent. If the second conjunct is an explanation or example of the first one, conjoining the clauses with \textstyleStyleVernacularWordsItalic{ne} is not allowed. Example (\stepcounter{nx}{\thenx}) is a case of symmetrical coordination; but if the order of the two conjuncts were reversed, the adverbial \textstyleStyleVernacularWordsItalic{pun} `also', which has to be in the second conjunct, would not move to the first conjunct with the rest of the clause.

\ea%x1372
\label{ex:x1372}
\gll I  mua=ko  me  wia  furew-a-mik,  \textstyleEmphasizedVernacularWords{ne}  yiena  pun  \\
      \\
\glt
\z

1p.UNM  man=NF  not  3p.ACC  sense-PA-1/3p  ADD  1p.GEN  also

mukuna=ko  me  op-a-mik.

fire=NF  not  hold-PA-1/3p

`We didn't sense anyone there and we ourselves did not hold fire either.'

The following example (\stepcounter{nx}{\thenx}) is syntactically neutral, but semantically it is interpreted as both temporal and consecutive sequence.

\ea%x1373
\label{ex:x1373}
\gll ...maa  wiar  fe-feef-omak-e-mik,  \textstyleEmphasizedVernacularWords{ne}  wi  \\
      \\
\glt
\z

food  3.DAT  RDP-spill-DISTR/PL-PA-1/3p  ADD  3p.UNM

ikiw-e-mik ...

go-PA-1/3p

`{\dots} they\textsubscript{i} spilled their\textsubscript{j} food, and (so/then) they\textsubscript{j} went (away) {\dots}'

When there are more than two coordinated clauses in a sentence without any intervening medial clauses, it is common to have \textstyleStyleVernacularWordsItalic{ne}  joining the last two clauses:

\ea%x1374
\label{ex:x1374}
\gll Mua  kuum-e-mik  nain  me  wia  kuuf-a-mik,  me  \\
      \\
\glt
\z

man  burn-PA-1/3p  that1  not  3p.ACC  see-PA-1/3p  not

wia  furew-a-mik,  \textstyleEmphasizedVernacularWords{ne}  me  wia  imen-a-mik.

3p.ACC  sense-PA-1/3p  ADD  not  3p.ACC  find-PA-1/3p

`We didn't see the men who burned it, we didn't sense them and we didn't find them.'

The connective \textstyleStyleVernacularWordsItalic{ne}  is also used in sentences where an adversative interpretation can be applied.\footnote{Using Haspelmath's (2007:28) terms, \textit{ne} in the adversative function could be called an \textit{oppositive} coordinator, as the second coordinand does not cancel an expectation like it does in adversative clauses formed with either the demonstrative \textit{nain} or the topic marker -\textit{na} (\sectref{sec:8.3.4}).}  The example (\stepcounter{nx}{\thenx}) describes a couple that stayed in the village during the war and placed some of their belongings outside their house to show that there were people living in the village, while many others ran away into the rainforest. 

\ea%x1375
\label{ex:x1375}
\gll Amina,  wiowa,  eka  napia  koor(a)  miira=pa  \\
      \\
\glt
\z

pot  spear  water  bamboo  house  front=LOC

iimar-aw-ikiw-e-mik,  \textstyleEmphasizedVernacularWords{ne}  wi  unowa  baurar-e-mik.

stand-CAUS-go-PA-1/3p  ADD  3p.UNM  many  flee-PA-1/3p

`We placed the pots, spears and bamboo water containers in line in front of the house, but many ran away.'

The connective \textstyleStyleVernacularWordsItalic{aria} 'alright' may be used when there is a change of topic or an unexpected development within the sentence.

\ea%x1376
\label{ex:x1376}
\gll Epa  wii-wiim-ik-ua,  \textstyleEmphasizedVernacularWords{aria}  wi  sawur=ke  ekap-ep  \\
      \\
\glt
\z

place  RDP-dawn-be-PA.3s  alright  3p.UNM  spirit=CF  come-SS.SEQ

takira  nain  samapora  onaiya  akua  aaw-e-mik.

boy  that1  bed  with  shoulder  take-PA-1/3p

`It was getting light, and spirits came and carried the boy with his bed (away) on their shoulders.'

\ea%x1377
\label{ex:x1377}
\gll Iiriw  muuka  oko  wiawi  onak  urera  maa  \\
      \\
\glt
\z

earlier  boy  other  3s/p.father  3s/p.mother  afternoon  food

uup-e-mik,  \textstyleEmphasizedVernacularWords{aria}  maa  me  wu-om-a-mik  yon  {\dots}

cook-PA-1/3p  alright  food  not  put-BEN-BNFY2.PA-1/3p  perhaps

`Long ago, the parents of a boy cooked food in the afternoon, (but) perhaps they did not put any food for him {\dots}'

It is also the default coordinator when a non-verbal constituent in two or more otherwise very similar conjuncts are contrasted, or emphasized, in coordinated clauses.

\ea%x1379
\label{ex:x1379}
\gll Yo  Malala  mauw-owa  nia  asip-i-yem,  \textstyleEmphasizedVernacularWords{aria} \\
      \\
\glt
\z

1s.UNM  Malala  work-NMZ  2p.ACC  help-Np-PR.1s  alright

yena  owowa,  Moro  owowa  wia  asip-i-yem.

1s.GEN  village  Moro  village  3p.ACC  help-Np-PR.1s

`I help you Malala people with your work, and I help my village, Moro village.'

\ea%x1380
\label{ex:x1380}
\gll Eema  pun  ekap-ep  yia  maak-e-k,  \textstyleEmphasizedVernacularWords{aria}  buburia  ona  \\
      \\
\glt
\z

Eema  also  come-SS.SEQ  1p.ACC  tell-PA-3s  alright  bald  3s.GEN

pun  ekap-ep  yia  maak-e-k.

also  come-SS.SEQ  1p.ACC  tell-PA-3s

`Eema came and told us, and the bald man himself too came and told us.'

\subsubsection{Disjunction}
\hypertarget{RefHeading23001935131865}{}
The speech of the Mauwake people tends to be rather concrete in the sense that they do not speculate much on different abstract alternatives. So disjunction of clauses, although possible, is not common. Disjunction is marked by the connective \textstyleStyleVernacularWordsItalic{e} `or' placed between the conjuncts (\sectref{sec:3.11.2}). 

\ea%x1385
\label{ex:x1385}
\gll Nain=ke  napum-ar-i-mik  \textstyleEmphasizedVernacularWords{e}  um-i-mik,  mua  oko  \\
      \\
\glt
\z

that1=CF  sickness-INCH-Np-PR.1/3p  or  die-Np-PR.1/3p  man  other

napum-ar-e-k  nain  erewar-e-n.

sickness-INCH-PA-3s  that1  foresee-PA-2s

`That is about people becoming sick or dying, you foresaw (in a dream) that some man became sick.'

Sometimes the question marker -\textstyleStyleVernacularWordsItalic{i}  replaces the connective.

\ea%x1387
\label{ex:x1387}
\gll Aria  no  ikoka  mua  owawiya  irak-ep=\textstyleEmphasizedVernacularWords{i}  kamenap  \\
      \\
\glt
\z

alright  2s.UNM  later  man  with  fight-SS.SEQ=QM  how

on-ap  yo  me  efar  kerer-e,  no

do-SS.SEQ  1s.UNM  not  1s.DAT  arrive-IMP.2s  2s.UNM

nomokowa  Kululu  fan-e-k  a.

2s/p.brother  Kululu  here-PA-3s  INTJ

`Alright, later when you fight with your husband or do something like that, do not come to me, your brother Kululu is right here.'

Alternative questions (\sectref{sec:6.3.2}) have the question marker -\textstyleStyleVernacularWordsItalic{i} cliticized to the end of the clause at least in the first conjunct. Closed alternative questions leave the question mark out of the last conjunct. 

\ea%x1386
\label{ex:x1386}
\gll Ikoka  ekap-ep  feeke  sira  nain  piipua-i-nan=\textstyleEmphasizedVernacularWords{i} \\
      \\
\glt
\z

later  come-SS.SEQ  here.CF  habit  that1  leave-Np-FU.2s=QM

\textstyleEmphasizedVernacularWords{e}  weetak?

or  no

`Later when you come, will you here leave that habit or not?'

Open alternative questions have the question marker in all the conjuncts.

\ea%x1384
\label{ex:x1384}
\gll Mua  oko  miira  inawera=pa  uruf-ap  ma-i-mik, \\
      \\
\glt
\z

man  other  face  dream=LOC  see-SS.SEQ  say-Np-PR.1/3p

mua  oko=ke  napuma  aaw-o-k=\textstyleEmphasizedVernacularWords{i}  \textstyleEmphasizedVernacularWords{e}  um-o-k=\textstyleEmphasizedVernacularWords{i}?

man  other=CF  sickness  get-PA-3s=QM  or  die-PA-3s=QM

`When we see some man's face in a dream we say, ``Has some other man become sick or died (or possibly neither)?'' '

\subsubsection{Adversative coordination}
\hypertarget{RefHeading23021935131865}{}
There is no adversative coordinator in Mauwake. It was mentioned above  (\sectref{sec:3.11.1}, 8.1.1.2) that the pragmatic additive connective \textstyleStyleVernacularWordsItalic{ne}, which is semantically neutral, is possible also when there is a relationship between clauses that may be interpreted as contrastive. 

\ea%x1388
\label{ex:x1388}
\gll Iir  nain  Kedem  manek  akena  keker  op-a-k \\
      \\
\glt
\z

time  that  Kedem  big  very  fear  hold-PA-3s

\textstyleEmphasizedVernacularWords{ne}  Yoli  weetak.

ADD  Yoli  no

`That time Kedem was very scared but Yoli wasn't.'

There are two strategies that can be used when a strong adversative is needed. A `but'-protasis \citep[237]{Reesink1983b} may be marked by either the distal demonstrative \textstyleStyleVernacularWordsItalic{nain} `that', or the topic marker -\textstyleStyleVernacularWordsItalic{na} (\sectref{sec:8.3.4}), added to a finite clause. The adversative clauses with the demonstrative \textstyleStyleVernacularWordsItalic{nain} differ from the nominalized clauses functioning as complement clauses or relative clauses in the following respects. Intonationally \textstyleStyleVernacularWordsItalic{nain} is the initial element in the second one of the contrasted clauses, rather than a final element in a subordinate clause, and it is often preceded by a short pause. The protasis may even be a separate sentence (\stepcounter{nx}{\thenx}). 

\ea%x1395
\label{ex:x1395}
\gll Panewowa  nain,  wi  iiriw  eno-wa  en-e-mik,  \textstyleEmphasizedVernacularWords{nain}  me \\
      \\
\glt
\z

old.person  that1  3p.UNM  earlier  eat-NMZ  eat-PA-1/3p  that1  not

onak-e-mik.

give.3s-PA-1/3p

`As for the old woman, they (aready) ate the meal earlier but did not give (any of it) to her to eat.'

\ea%x728
\label{ex:x728}
\gll Yo  bom  koor  miira=pa  efar  or-om-ik-ua. \\
      \\
\glt
\z

1s.UNM  bomb  house  face=LOC  1s.DAT  fall-SS.SIM-be-PA.3s

\textstyleEmphasizedVernacularWords{Nain}  yo  me  baurar-em-ik-e-m.

that1  1s.UNM  not  flee-SS.SIM-be-PA-1s

`Bombs kept dropping in front of my house. But I didn't keep running away.'

The next two examples are structurally very similar to sentences with relative clauses (\sectref{sec:8.3.1.2}). But here the demonstrative \textstyleStyleVernacularWordsItalic{nain} is part of the adversative clause and is preceded by a pause. 

\ea%x1389
\label{ex:x1389}
\gll Mera  eka  enim-i-mik,  \textstyleEmphasizedVernacularWords{nain}  i  mangala \\
      \\
\glt
\z

fish  water  eat-Np-PR.1/3p  that1  1p.UNM  shellfish

me  enim-i-mik,  waaya  me  enim-i-mik.

not  eat-Np-PR.1/3p  pig  not  eat-Np-PR.1/3p

`We eat fish soup, but we don't eat shellfish, (and) we don't eat pork.'

\ea%x1394
\label{ex:x1394}
\gll I  nan  soomar-e-mik,  \textstyleEmphasizedVernacularWords{nain}  i  mukuna=ko  me \\
      \\
\glt
\z

1p.UNM  there  walk-PA-1/3p  that1  1p.UNM  fire=NF  not

op-a-mik.

hold-PA-1/3p

`We walked there, but we did not hold/have any fire.'

Compare (\stepcounter{nx}{\thenx}) with the relative clause (\stepcounter{nx}{\thenx}), where the demonstrative functions as a relative marker and comes at the end of the clause. This is shown by the slightly rising intonation on \textstyleStyleVernacularWordsxiiptItalic{nain}, as well as a pause following it in spoken text:\footnote{This similarity creates a problem with written texts that do not have adequate punctuation. Sometimes either interpretation is acceptable.} 

\ea%x1396
\label{ex:x1396}
\gll I  nan  soomar-e-mik  \textstyleEmphasizedVernacularWords{nain},  i  mukuna=ko  me  \\
      \\
\glt
\z

1p.UNM  there  walk-PA-1/3p  that1  1p.UNM  fire=NF  not

op-a-mik.

hold-PA-1/3p

`We who walked there didn't hold/have any fire.' (Or: `When we walked there, we didn't hold/have any fire.')

The adversative sentences formed with the topic marker -\textstyleStyleVernacularWordsItalic{na} are complex rather than coordinate sentences (\sectref{sec:8.3.4}).

\subsubsection{Consecutive coordination}
\hypertarget{RefHeading23041935131865}{}
Within a sentence, clauses are typically connected by one of the syntactically neutral strategies, which leave the semantic relationship implied. Some sentences using juxtaposition (\stepcounter{nx}{\thenx}), the pragmatic additive \textstyleStyleVernacularWordsItalic{ne} (\stepcounter{nx}{\thenx}) or clause chaining (\stepcounter{nx}{\thenx}) can be interpreted as having a consecutive relationship between the clauses, although this does not show in the syntax. This section deals with the cases where the consecutive relationship is marked overtly.

Relationships of cause and effect, or reason and result,\footnote{Reason-result relationship presupposes the presence of reasoning in the process, cause-effect relationship does not.}  are central in the discussion of causal and consecutive clauses. It seems that currently Mauwake may be developing a distinction between cause and reason on one hand, and between effect and result on the other. But the tendency, if there, is not very strong (\sectref{sec:3.11.2}). 

Both the clauses in a sentence expressing a cause-effect or reason-result relationship are main clauses and are in a coordinate relationship with each other. It is common for the two clauses to form separate sentences rather than be within the same sentence. 

The tendency to present events in the same order that they occur, common to languages in general, is very strong in Papuan languages. Consequently, there is a strong preference to present a cause clause before an effect clause (Haiman 1980:409, Roberts 1987:59, Reesink 1987). In Mauwake consecutive coordination is the default, unmarked strategy for those sentences that express cause-effect or reason-result relationships overtly, because their structure follows this principle, whereas in causal coordination sentences the effect is stated before the cause. 

\ea%x1400
\label{ex:x1400}
\gll Emar,  nos=ke  yo  efa  kemal-ep  iripuma  fain \\
      \\
\glt
\z

1s/p.friend  2s.CF=CF  1s.UNM  1s.ACC  pity-SS.SEQ  iguana  this

ifakim-o-n,  \textstyleEmphasizedVernacularWords{naapeya}  iripuma  fain  ik-ep  enim-e.

kill-PA-2s  therefore  iguana  this  roast-SS.SEQ  eat-IMP.2s

`Friend, it was you who pitied me and killed this iguana, therefore you roast and eat this iguana.' 

Effect and result clauses use \textstyleStyleVernacularWordsItalic{naapeya/naeya} `therefore, (and) so' (\sectref{sec:3.11.2}) as their connective:

\ea%x1401
\label{ex:x1401}
\gll Koora  fuluwa  unowa  marew,  \textstyleEmphasizedVernacularWords{naapeya}  in-i-mik  nain \\
      \\
\glt
\z

house  hole  many  no(ne)  therefore  sleep-Np-PR.1/3p  that1

dabela  me  senam  furew-i-mik.

cold  not  too.much  sense-Np-PR.1/3p

`The houses do not have many windows, so those who sleep (there) do not sense/feel the cold too much.'

\ea%x1402
\label{ex:x1402}
\gll Pita  weke  wiar  um-o-k,  \textstyleEmphasizedVernacularWords{naapeya}  o  suule  \\
      \\
\glt
\z

Pita  3s/p.grandfather  3.DAT  die-PA-3s  therefore  3s.UNM  school

me  iw-a-k.

not  go-PA-3s

`Pita's grandfather died, so he (Pita) didn't go to school.'

\ea%x1405
\label{ex:x1405}
\gll {\dots}pika  oona  me  kekan-ow-a-k,  \textstyleEmphasizedVernacularWords{naeya}  uura \\
      \\
\glt
\z

...wall  support  not  be.strong-CAUS-PA-3s  therefore  night

ewar  maneka=ke  kerer-emi  koora  nain  wiar

wind  big=CF  appear-SS-SIM  house  that1  3.DAT  

teek-a-k.

tear-PA-3s

`He did not strengthen the wall supports, so at night a big wind arose and tore down his house.'

\ea%x1408
\label{ex:x1408}
\gll No  nena  pun  pina  sira  naap  nain  on-i-n, \\
      \\
\glt
\z

2s.UNM  2s.GEN  also  guilt  custom  thus  that1  do-Np-PR.2s

\textstyleEmphasizedVernacularWords{naeya}  nos  pun  opora=pa  ika-i-nan.

therefore  2s.FC  also  talk=LOC  be-Np-FU.2s

`You yourself do bad things like that too, therefore you too will be under accusation.'

\textstyleStyleVernacularWordsItalic{Naapeya} can also co-occur with the conjunctive coordinator \textstyleStyleVernacularWordsItalic{ne}. 

\ea%x1403
\label{ex:x1403}
\gll Epa  nan  soomar-em-ik-ok  or-o-mik,  \\
      \\
\glt
\z

place  there  walk-SS.SIM-be-SS  descend-PA-1/3p

\textstyleEmphasizedVernacularWords{ne}  \textstyleEmphasizedVernacularWords{naapeya}  pina  wi  wiar  korin-e-k.

ADD  therefore  guilt  3p.UNM  3.ACC  stick-PA-3s

`They were walking there in that place and came down, and so the guilt (for starting a forest fire) stuck to them.'

The use of \textstyleStyleVernacularWordsItalic{naapeya} and \textstyleStyleVernacularWordsItalic{naeya} is both external and internal, i.e. they connect events in a situation and ideas in a text. The internal use of \textstyleStyleVernacularWordsItalic{ne naapeya} and \textstyleStyleVernacularWordsItalic{aria naapeya} is restricted to intersentential use. They refer to a longer stretch in the preceding text as their protasis.

\ea%x1407
\label{ex:x1407}
\gll \textstyleEmphasizedVernacularWords{Aria}  \textstyleEmphasizedVernacularWords{naapeya}  wi  inasina  ook-i-mik   \\
      \\
\glt
\z

alright  therefore  3p.UNM  spirit  follow-Np-PR.1/3p  

sira  nain  me  wiar  ook-eka.

custom  that1  not  3.DAT  follow-IMP.2p

`So therefore do not follow the behavior of those who follow/believe in spirits.'

As an internal connective \textstyleStyleVernacularWordsItalic{naeya} mainly connects full sentences (\stepcounter{nx}{\thenx}), only seldom clauses within a sentence (\stepcounter{nx}{\thenx}):

\ea%x1410
\label{ex:x1410}
\gll No  mua  woos  reen-owa=ke,  \textstyleEmphasizedVernacularWords{naeya}  no  kema  kir-owa \\
      \\
\glt
\z

2s.UNM  man  head  dry-NMZ=CF  therefore  2s.UNM  liver  turn-NMZ

miatin-i-n.

dislike-Np-PR.2s

`You are hard-headed, therefore you do not like to change your (bad) ways.'

\ea%x1411
\label{ex:x1411}
\gll Ni  sira-sira  naap  on-i-man.  \textstyleEmphasizedVernacularWords{Naeya}  opora  iiriw \\
      \\
\glt
\z

2p.UNM  RDP-custom  thus  do-Np-PR.2p  therefore  talk  earlier

ma-e-k  nain  pepek  akena  nia  ma-e-k.

say-PA-3s  that1  enough  very  2p.ACC  say-PA-3s

`You do (bad) things like that. Therefore the talk that he already said about you is very accurate.'

\textstyleStyleVernacularWordsItalic{Neemi}  is a consecutive coordinator that almost exclusively conjoins full sentences rather than clauses within a sentence: (\stepcounter{nx}{\thenx}) is from translated text but considered natural. (\stepcounter{nx}{\thenx}) is repeated here as (\stepcounter{nx}{\thenx}). \textit{Neemi} is an internal connective, only used in reasoning. It requires some point of similarity between the two conjuncts.

\ea%x1904
\label{ex:x1904}
\gll Teeria  fain  K10  wu-a-mik.  \textstyleEmphasizedVernacularWords{Neemi}  wi  teeria  nain  pun  \\
      \\
\glt
\z

group  this  K10  put-PA-1/3p  therefore  group  that1  too

K10  wu-a-mik.

K10  put-PA-1/3p

`This group put down ten kina. Therefore that group put down ten kina, too.'

\ea%x1409
\label{ex:x1409}
\gll Krais  sirir-owa  aaw-omak-e-k,  \textstyleEmphasizedVernacularWords{neemi}  is  pun \\
      \\
\glt
\z

Christ  hurt-NMZ  get-DISTR/PL-PA-3s  therefore  1p.FC  also

unowiya  naap  aaw-i-mik.

all  thus  get-Np-PR.1/3p

`Christ received a lot of pain, so we all too get (pain) like that.'

The connective \textstyleStyleVernacularWordsItalic{naap nain} is used almost only inter-sententially (\stepcounter{nx}{\thenx}). Between clauses in a sentence it is possible but rare (\stepcounter{nx}{\thenx}):

\ea%x1905
\label{ex:x1905}
\gll Naeya  nokar-e-mik,  ``\textstyleEmphasizedVernacularWords{Naap  nain}  no  naareke?'' \\
      \\
\glt
\z

therefore  ask-PA-1/3p  thus  that1  2s.UNM  who.CF

`Therefore they asked, ``So then, who are you?'' '

\ea%x1424
\label{ex:x1424}
\gll Wiam  arow  pepek  nan  urup-e-mik  nain,  \\
      \\
\glt
\z

3p.REFL  three  enough  there  ascend-PA-1/3p  that1

\textstyleEmphasizedVernacularWords{naap  nain}  yo  moram  urup-e-m.

thus  that1  1s.UNM  why/in.vain  ascend-PA-1s

`(Since it is the case that) those three are enough and came up, so then why did I have to come up? (or: {\dots}so then I came up in vain).'

\subsubsection{Causal coordination, ``afterthought reason''}
\hypertarget{RefHeading23061935131865}{}
The causal coordination is a very marked structure, which shows in the unusual ordering of the clauses: the causal clause follows rather than precedes the consequent clause. The causal clause in Mauwake begins with the connective \textstyleStyleVernacularWordsItalic{moram} `because' (\sectref{sec:3.1.1.2}), which is originally the interrogative word for `why'.  There are two possible origins for this untypical structure. It may be a recent calque on the Tok Pisin causal construction, which uses \textstyleForeignWords{bilong wanem} `why/because' as the connector and the same ordering of the two clauses. The ordering of the clauses shows that it may also have originated as an ``afterthought reason'',\footnote{The term suggested by Ger Reesink.} even though  currently it is used when the cause or reason is emphasized. 

\ea%x1417
\label{ex:x1417}
\gll Owowa  mamaiya  soora  weetak,  \textstyleEmphasizedVernacularWords{moram}  iwera \\
      \\
\glt
\z

village  near  forest  no  because  coconut

isak-omak-e-mik.

plant-DISTR/PL-PA-1/3p

`There is no forest near the village, because we have planted a lot of coconut palms.'

\ea%x1420
\label{ex:x1420}
\gll Poh  San  uruf-ap  kema  ten-e-mik,  \textstyleEmphasizedVernacularWords{moram}  i  kema \\
      \\
\glt
\z

Poh  San  see-SS.SEQ  liver  fall-PA-1/3p  because  1p.UNM  liver

naap  suuw-a-mik,  napuma  me  sariar-owa  ik-ua.

thus  push-PA-1/3p  sickness  not  heal-NMZ  be-PA.3s

`We saw Poh San and were relieved (lit: liver fell), because we had thought that (her) sickness hadn't healed yet (but it had).'

\textstyleStyleVernacularWordsItalic{Moram wia} is used almost exclusively between full sentences (\stepcounter{nx}{\thenx}); the example (\stepcounter{nx}{\thenx}) is the only intra-sentential instance of \textstyleStyleVernacularWordsItalic{moram wia} in the data. I have not noticed any semantic difference caused by the addition of the negator.

\ea%x1906
\label{ex:x1906}
\gll ...maamuma  senam  aaw-e-mik.  \textstyleEmphasizedVernacularWords{Moram  wia}, \\
      \\
\glt
\z

money  too/very.much  get-PA-1/3p  why  not  

maa  ele-eliwa  sesek-a-mik.

thing/food  RDP-good  sell-PA-1/3p

`{\dots}they got a lot of money. (That's) because they sold good food.'

\ea%x1421
\label{ex:x1421}
\gll Iir  nain  yo  owowa=pa=ko  me  mauw-a-m, \\
      \\
\glt
\z

time  that1  1s.UNM  village=LOC=NF  not  work-PA-1s  

\textstyleEmphasizedVernacularWords{moram  wia}  yo  Ukarumpa  urup-owa=ke  na-ep

because  not  1s.UNM  Ukarumpa  ascend-NMZ=CF  say-SS.SEQ  

mauw-owa  miatin-e-m.

work-NMZ  dislike-PA-1s

`That time I did not work in the village, because I thought that I was due to go up to Ukarumpa, and (so) I didn't like to work.'

Both a causative and a consecutive connective can co-occur in the same sentence. When that happens, the consecutive clause occurs twice: first without a connective and after the causal clause with a connective. This underlines the strong preference to keep the cause-effect (or reason-result) order.

\ea%x1422
\label{ex:x1422}
\gll I  epa  unowa=ko  me  soomar-e-mik,  \textstyleEmphasizedVernacularWords{moram}  owowa \\
      \\
\glt
\z

1p.UNM  place  many=NF  not  walk-PA-1/3p  because  village

maneka,  \textstyleEmphasizedVernacularWords{naapeya}  soomar-owa  lebum(a)-ar-e-mik.

big  therefore  walk-NMZ  lazy-INCH-PA-1/3p

`We didn't walk in many places, because the village/town was big, therefore we didn't care to walk.'

\ea%x1423
\label{ex:x1423}
\gll Mua  lebuma  emeria  me  wi-i-mik,  \textstyleEmphasizedVernacularWords{moram}  emeria  \\
      \\
\glt
\z

man  lazy  woman  not  give.them-Np-PR.1/3p  because  woman

muukar-eya  muuka  nain  maa  mauwa  enim-i-non,

give.birth-2/3s.DS  son  that1  food  what  eat-Np-FU.3s

\textstyleEmphasizedVernacularWords{naapeya}  mua  lebuma  emeria  me  wi-i-mik.

therefore  man  lazy  woman  not  give.them-Np-PR.1/3p

`We do not give wives to lazy men, because when the woman bears a child what would it eat, therefore we do not give wives to lazy men.'

\subsubsection{Apprehensive coordination}
\hypertarget{RefHeading23081935131865}{}
A less common clause type, that of apprehensive clauses \citep[61]{Roberts1987}, also called negative purpose clauses (Haiman 1980:444, Thompson and Longacre 1985:188), is perhaps more commonly subordinate than coordinate. But in Mauwake the apprehensive clauses are coordinated finite clauses, originally separate sentences (\stepcounter{nx}{\thenx}). The apprehension clause is introduced by the indefinite \textstyleStyleVernacularWordsItalic{oko} `other' (\sectref{sec:3.5.3}), which has also developed the meaning `otherwise'.  

\ea%x1426
\label{ex:x1426}
\gll Ni  maa  uru-uruf-ami  ik-eka,  \textstyleEmphasizedVernacularWords{oko}  mua  oko=ke \\
      \\
\glt
\z

2p.UNM  thing  RDP-see-SS.SIM  be-IMP.2p  other  man  other=CF

nia  peeskim-i-kuan.

2p.ACC  cheat-Np-FU.3p

`Watch out, otherwise/lest you get cheated.'

\ea%x1427
\label{ex:x1427}
\gll Naap  on-owa  weetak,  \textstyleEmphasizedVernacularWords{oko}  yiena  sira  puuk-i-yen. \\
      \\
\glt
\z

thus  do-NMZ  no  other  1p.GEN  custom  cut-NP-FU.1p

`We must not do like that, otherwise/lest we break our custom/law (or: {\dots}lest we ourselves break the custom/law).'

\ea%x1428
\label{ex:x1428}
\gll Naap  yo  aakisa  efa  uruf-i-n.  \textstyleEmphasizedVernacularWords{Oko}  neeke \\
      \\
\glt
\z

thus  1s.UNM  now  1s.ACC  see-Np-PR.2s  other  there.CF

soomar-ekap-em-ik-omkun  ma-i-nan,  ``  {\dots } ``

walk-come-SS-SIM-be-1s/p.DS  say-Np-FU.2s

`So you see me now. Otherwise I'll be walking there and you will say, `` {\dots} ''

\subsection{Clause chaining}
\hypertarget{RefHeading23101935131865}{}
Clause chaining is a typical feature in Papuan languages, and in the Trans-New Guinea languages in particular.\footnote{Wurm seems to consider clause chaining a genetic feature of the TNG languages (1982:36), but Haiman suggests that it is an areal feature (1980:xlvii). Roberts, with the largest data to date, suggests that there is a combination of both, but leaves the final decision open (1997:122).}  A sentence may consist of several medial clauses\footnote{The terms \textit{medial} and \textit{final} clauses are well established in Papuan linguistics. }   where the verbs have medial verb inflection (\sectref{sec:3.8.3.4}), and a final clause where the verb has ``normal'' finite inflection (\sectref{sec:3.8.3.3}). Clause chaining indicates either temporal sequence or simultaneity between adjacent clauses.

The division into just medial and final clauses is not adequate for describing the system. Haiman and Munro call the medial clauses\textit{} \textstyleEmphasizedWords{\textsc{marking clauses}} and the clauses following them \textstyleEmphasizedWords{\textsc{reference clauses}} (1983:xii).\footnote{\citet{Comrie1983} and \citet{Roberts1997} call them \textit{marked clauses} and \textit{controlling clauses}, respectively.} Marking clause is simply another name for a medial clause and will not be used here. But a reference clause may be medial or finite\footnote{I prefer the term \textit{finite} to \textit{final} clauses (and verbs), as it is the finiteness rather than the position in the sentence that is important in their relation with medial clauses. Subordinate clauses are the most typical \textit{non-final} finite clauses, and they may also have medial clauses preceding them and relating to them.} -- what is important is that both the temporal relationship of the medial verb, and the person reference, is stated in relation to the reference clause. When a reference clause for a preceding medial clause is also a medial clause, it again has its own reference clause following it.

The medial clauses linked by clause chaining are sometimes called \textstyleEmphasizedWords{\textsc{cosubordinate}} (Olson 1981, Foley and VanValin 1984:257)\footnote{This is cosubordination at the \textit{peripheral} level; verb serialization is cosubordination at core or nuclear level.} or \textstyleEmphasizedWords{\textsc{coordinate-}}\textstyleEmphasizedWords{\textsc{dependent}} \citep[177]{Foley1986}, because they share features with both coordinate and subordinate clauses. Their relationship with each other and with the following finite clause is essentially coordinate,\footnote{\citet{Roberts1988a} brings several syntactic arguments to show that basically switch reference is indeed coordination rather than subordination. But he also argues for a separate subordinate switch reference in Amele and some other languages.} but the medial clauses are dependent on the finite clause both for their absolute tense, and, in the case of ``same subject'' forms, also for their person/number specification. 

Another term commonly used for the chained clauses, \textstyleEmphasizedWords{\textsc{switch-reference clauses}} (\textstyleAcronymallcaps{SR}),\footnote{Clause chaining and switch reference are two separate strategies, but in Papuan languages the two very often  go together \citep[104]{Roberts1997}.} is related to their other function as a reference-tracking device (Haiman and Munro 1983:ix). They typically indicate whether their topic/subject is the same as, or different from, the topic/subject of the following clause. This is discussed below in \sectref{sec:8.2.3}. In this grammar the two terms are used interchangeably, as in Mauwake the medial verbs not only indicate a temporal relationship but are used for reference tracking as well.

\subsubsection{Chained clauses as coordinate clauses}
\hypertarget{RefHeading23121935131865}{}
It is widely accepted that the relationship of medial clauses to their reference clauses is basically coordinate, but with some special features and exceptions.\footnote{E.g. \citet[175,193]{Reesink1987}, Roberts (1988a:51, 1994:13).} In Mauwake medial clauses are subordinate only if subordinated with the topic/conditional marker -\textstyleStyleVernacularWordsItalic{na}; otherwise they are coordinate. 

Instead of giving background information like subordinate clauses do, medial clauses are predications that carry on the foreground story line. But they are also different from coordinate finite clauses. The similarities and differences are discussed in this section.

The pragmatic additives \textstyleStyleVernacularWordsItalic{ne} and \textstyleStyleVernacularWordsItalic{aria} (\sectref{sec:3.11.1}) can occur between a medial clause and its reference clause, as between normal coordinate clauses. This is uncommon, however.

\ea%x1444
\label{ex:x1444}
\gll Wiawi  ikiw-ep  maak-eya,  \textbf{ne}  wiawi=ke  maak-e-k  {\dots} \\
      \\
\glt
\z

3s/p.father  go-SS.SEQ  tell-2/3s.DS  ADD  3s/p.father=CF  tell-PA-3s

`She went to her father and told him, and her father told her {\dots}'

\ea%x1445
\label{ex:x1445}
\gll ...  wiena  en-emi,  epira  lolom  if-emi  \textbf{ne}  owowa \\
      \\
\glt
\z

...  3p.GEN  eat-SS.SIM  plate  mud  spread-SS.SIM  ADD  village

p-urup-em-ik-e-mik.

BPx-ascend-SS.SIM-be-PA-1/3p

`They ate it themselves, spread mud on the plates and brought them up to the village.'

\ea%x1442
\label{ex:x1442}
\gll I  ikoka  yien=iw  urup-ep  nia  maak-omkun \\
      \\
\glt
\z

1p.UNM  later  1p.GEN=LIM  ascend-SS.SEQ  2p.ACC  tell-1s/p.DS

ora-iwkin,  \textstyleEmphasizedVernacularWords{aria}  owawiya  feeke  pok-ap  ik-ok  eka

descend-2/3p.DS  alright  together  here.CF  sit-SS.SEQ  be-SS  water

liiwa  muuta  en-ep  \textstyleEmphasizedVernacularWords{aria}  ni  soomar-ek-eka.

little  only  eat-SS.SEQ  alright  2p.UNM  walk-go-2p.IMP

`Later we (by) ourselves will come up and tell you (to come), and when you come down we will sit here together and eat a little bit of soup and then you can walk back.'

Coordinated main clauses are free in regard to their mood and, related to that, their functional sentence type. The medial clauses do not have any marking for mood.  They usually conform to that of the finite clause, but this is a pragmatic matter, not a syntactic requirement. 

When either the medial clause or the finite clause is a question, the whole sentence is interrogative, even if the other clause is a statement. In (\stepcounter{nx}{\thenx}) the finite clause is a polar question, but the medial clause is not questioned. In the story that (\stepcounter{nx}{\thenx}) is taken from, the killing is not questioned, only the manner. But since a medial clause cannot take the question marker, the verb in the finite clause has to carry the marking.

\ea%x1449
\label{ex:x1449}
\gll Sande  erup  weeser-eya  owowa  ekap-e-man=i? \\
      \\
\glt
\z

week  two  finish-2/3s.DS  village  come-PA-2p=QM

`Two weeks were finished, and did you (then) come to the village?'\footnote{Another possible translation is `When the two weeks were finished, did you (then) come to the village?' but this does not reflect the coordinate relationship of the clauses in the original.}

\ea%x1452
\label{ex:x1452}
\gll Naap  on-ap  ifakim-i-nen=i? \\
      \\
\glt
\z

thus  do-SS.SEQ  kill-Np-FU.1s=QM

`Shall I do like that and kill her?' (Or: `Is it in that way that I shall kill her?')

A non-polar question can be in either a medial or a finite clause:

\ea%x1451
\label{ex:x1451}
\gll No  sira  kamenap  on-eya  napuma  fain \\
      \\
\glt
\z

2s.UNM  custom  how  do-2/3s.DS  sickness  this

nefar  kerer-e-k?

2s.DAT  appear-PA-3s

`What did you do (so that) this sickness came to you?'

\ea%x1450
\label{ex:x1450}
\gll No  karu-emi  kame  kaanek  ikiw-o-n? \\
      \\
\glt
\z

2s.UNM  run-SS.SIM  side  where  go-PA-2s

`You ran and where did you go?'

For more examples, see (\stepcounter{nx}{\thenx})-(\stepcounter{nx}{\thenx}) in \sectref{sec:7.3} and the introductory section to chapter 8. 

In regard to the scope of negation the same-subject medial clauses differ from all other clauses.  Negative spreading (\sectref{sec:6.2.4}) in both directions is allowed only between \textstyleAcronymallcaps{SS} medial clauses and their reference clauses; even there it is not very common. Especially backwards spreading is rare. In the following examples, negative spreading takes place in (\stepcounter{nx}{\thenx}) and (\stepcounter{nx}{\thenx}), but not in (\stepcounter{nx}{\thenx}) and (\stepcounter{nx}{\thenx}). Between other types of clauses negative spreading is not permitted at all. 

\ea%x1443
\label{ex:x1443}
\gll Nainiw  \textbf{ekap-ep}  maa  \textbf{me  sesenar-e-mik}. \\
      \\
\glt
\z

again  come-SS.SEQ  food  not  sell-PA-1/3p

`They did not come back and sell food.'

\ea%x1447
\label{ex:x1447}
\gll Ikiw-em-ik-ok  \textbf{me  kir-ep  uruf-e}, \\
      \\
\glt
\z

go-SS.SIM-be-SS  not  turn-SS.SEQ  look-IMP.2s

no  oram  woolal-ikiw-em-ik-e.

2s.UNM  just  paddle-go-SS.SIM-be-IMP.2s

`While going, don't turn and look back, just keep paddling.'

\ea%x1446
\label{ex:x1446}
\gll Yaapan=ke  urup-em-ika-iwkin  wi  Australia=ke \\
      \\
\glt
\z

Japan=CF  ascend-SS.SIM-be-2/3p.DS  3p.UNM  Australia=CF

wia  uruf-ap  baurar-emi  \textstyleEmphasizedVernacularWords{me  yia  maak-e-mik}.

3p.ACC  see-SS.SEQ  flee-SS.SIM  not  1p.ACC  tell-PA-1/3p

`When the Japanese were coming up the Australians saw them and ran away and/but did not tell us.'

\ea%x1448
\label{ex:x1448}
\gll Iiriw  auwa=ke  sira  fain  \textbf{me  paayar-ep} \\
      \\
\glt
\z

earlier  1s/p.father=CF  custom  this  not  understand-SS.SEQ

muuka  momor  wiar  aaw-em-ik-e-mik.

son  indiscriminately  3.DAT  get-SS.SIM-be-PA-1/3p

`Earlier our (fore)fathers didn't understand this custom, and (so) they adopted (lit: got/took) children indiscriminately.'

Like coordinated main clauses and unlike subordinate clauses, medial clauses are not embedded as constituents in other clauses. However, a medial clause may interrupt its reference clause and appear inside it, if the subject or object \textstyleAcronymallcaps{NP} of the reference clause is fronted as the theme and thus precedes the interrupting medial clause. For more examples, see (\stepcounter{nx}{\thenx}) and (\stepcounter{nx}{\thenx}). In the following examples the reference clause is bolded and the intervening medial clause is placed within square brackets.

\ea%x1464
\label{ex:x1464}
\gll Aria  \textstyleEmphasizedVernacularWords{yena  mua  pun}  [irak-owa  kerer-owa  epa  \\
      \\
\glt
\z

alright  1s.GEN  man  too  fight-NMZ  appear-NMZ  time

weeser-em-ik-eya]  \textstyleEmphasizedVernacularWords{iirar-iwkin}  owowa  ekap-o-k,

finish-SS.SIM-be-2/3s.DS  remove-2/3p.DS  village  come-PA-3s

o  amia  mua=pa  ik-ok.

3s.UNM  bow  man=LOC  be-SS

`Alright, the war was getting close and they dismissed my husband and he came to the village, after he had been a soldier.'

In (\stepcounter{nx}{\thenx}) both the object and the subject are fronted. After the first medial clause the object of the finite clause is fronted as the theme of the remainder of the sentence, and it pulls with it the subject, marked with the contrastive focus marker.  In the free translation passive is used, because the object is fronted as a theme.

\ea%x1465
\label{ex:x1465}
\gll Sisina=pa  wu-ap  \textstyleEmphasizedVernacularWords{papako}\textsubscript{O}  \textstyleEmphasizedVernacularWords{mua=ke}\textsubscript{S}  [mera  saa  \\
      \\
\glt
\z

shallow.water=LOC  put-SS.SEQ  some  man=CF  fish  sand

urup-eya]  \textstyleEmphasizedVernacularWords{patopat=iw  mik-i-mik}.

ascend-2/3s.DS  fishing.spear=INST  spear-Np-PR.1/3p

`They drive (lit: put) them to the shallow water and the fish ascend to the beach and (then) some are speared by men with a fishing spear.' 

The following three examples show that some of the same-subject medial clauses interrupting the reference clause, especially those that have a directional verb or the verb \textstyleStyleVernacularWordsItalic{aaw}- `take, get', may be in the process of grammaticalizing into serial verbs: 

\ea%x1466
\label{ex:x1466}
\gll \textstyleEmphasizedVernacularWords{I  iwer(a)  eka}  [iki(w-e)p]  \textstyleEmphasizedVernacularWords{nop-a-mik}. \\
      \\
\glt
\z

1p.UNM  coconut  water  go-SS.SEQ  fetch-PA-1/3p

`We went and fetched coconut water.'

\ea%x1467
\label{ex:x1467}
\gll \textstyleEmphasizedVernacularWords{Yo  merena}  [fura  aaw-ep]  \textstyleEmphasizedVernacularWords{puuk-a-m}. \\
      \\
\glt
\z

1s.UNM  leg  knife  take-SS.SEQ  cut-PA-1s

`I took a knife and cut (into) the leg. (Or: I cut into the leg with a knife.')

\ea%x1468
\label{ex:x1468}
\gll Um-eya  \textstyleEmphasizedVernacularWords{merena  ere-erup}  [ifara  aaw-ep]  \textstyleEmphasizedVernacularWords{kaik-ap} \\
      \\
\glt
\z

die-2/3s.DS  leg  RDP-two  rope  take-SS.SIM  tie-SS.SEQ

nabena  suuw-ap  akua  aaw-ep  or-o-m.

carrying.pole  push-SS.SEQ  shoulder  take-SS.SEQ  descend-PA-1s

`It (=a pig) died and I took a rope and tied its legs two and two together and pushed it to a carrying pole and carried it down on my shoulder.'

Right-dislocation of a medial clause is not unusual. One common reason for right-dislocations is an afterthought: the speaker notices something that should be part of the sentence and adds it to the end. Another reason is giving prominence to the dislocated clause, since the end of a sentence is a focal position. Especially the right-dislocation of same-subject sequential medial clauses breaks the iconicity between the events and the sentence structure, and has this effect. Consequently the right-dislocated \textstyleAcronymallcaps{SS} sequential clauses, like the ones in examples (\stepcounter{nx}{\thenx}) and (\stepcounter{nx}{\thenx}), are much more prominent than medial clauses in their normal position.

\ea%x1471
\label{ex:x1471}
\gll Or-op  naap  wia  uruf-a-mik,  [mua  oona,  eneka,  woosa  \\
      \\
\glt
\z

descend-SS.SEQ  thus  3p.ACC  see-PA-1/3p  man  bone  tooth  head

kia  kir-em-ik-eya].

white  turn-SS.SIM-be-2/3s.DS

`They went down and saw them like that, the people's bones, teeth and heads turning white.'

\ea%x1469
\label{ex:x1469}
\gll Aw-iki(w-e)m-ik-eya  wiena  mua  unowa  fiker(a)  epia \\
      \\
\glt
\z

burn-go-SS.SIM-be-2/3s.DS  3p.GEN  man  many  kunai.grass  fire

nain  ook-i-kuan,  [wiowa  aaw-ep].

that1  follow-Np-FU.3p  spear  take-SS.SEQ

`It keeps burning and many men follow the kunai grass fire, having taken spears.'

\ea%x1470
\label{ex:x1470}
\gll Aaya  muuna  kuisow  enim-i-mik,  [aite=ke    \\
      \\
\glt
\z

sugarcane  joint  one  eat-Np-PR.1/3p  1s/p.mother=CF  

manina=pa  yia  aaw-om-iwkin].

garden=LOC  1p.ACC  get-BEN-2/3p.DS

`We eat one joint of sugarcane, when/after our mothers have gotten it for us from the garden.'

\subsubsection{Temporal relations in chained clauses}
\hypertarget{RefHeading23141935131865}{}
Clause chaining in Mauwake distinguishes between sequential and simultaneous actions in the clauses joined by chaining, but only when the clauses have the same subject (\sectref{sec:3.8.3.5.1}). The sequential action verb (\stepcounter{nx}{\thenx}) indicates that one action is finished before the next one starts. 

\ea%x1431
\label{ex:x1431}
\gll No  nainiw  kir-\textstyleEmphasizedVernacularWords{ep}  ikiw-\textstyleEmphasizedVernacularWords{ep}  owow  mua  wia  \\
      \\
\glt
\z

2s.UNM  again  turn-SS.SEQ  go-SS.SEQ  village  man  3p.ACC

maak-eya  urup-\textstyleEmphasizedVernacularWords{ep}  mukuna  nain  umuk-uk.

tell-2/3s.DS  ascend-SS.SEQ  fire  that1  extinguish-IMP.3p

`Turn around, go and tell the village men and let them come up and extinguish the fire.'

When a clause has a simultaneous action medial verb (\stepcounter{nx}{\thenx}), it indicates at least some overlap with the action in the following clause. 

\ea%x1432
\label{ex:x1432}
\gll Or-\textstyleEmphasizedVernacularWords{omi}  yo  koka  koora=pa  nan  efa  \\
      \\
\glt
\z

descend-SS.SIM  1s.UNM  jungle  house=LOC  there  1s.ACC

wu-\textstyleEmphasizedVernacularWords{ami}  ma-e-k,  `` ... ''

put-SS.SIM  say-PA-3s

`As he went down he put me in the jungle house and said, `` ... '' '

Simultaneity vs. sequentiality is not always a choice between absolutes; sometimes it is a relative matter. The example (\stepcounter{nx}{\thenx}) refers to a situation where a man came back home from a period of labour elsewhere and got married upon arrival. In actual life there may have been a time gap of at least a number of days, possibly longer, but because the two events were so closely linked in the speaker's mind, the simultaneous action form was used when the story was told decades after the events took place.

\ea%x1433
\label{ex:x1433}
\gll Ekap-\textstyleEmphasizedVernacularWords{emi}  yo  efa  aaw-o-k. \\
      \\
\glt
\z

come-SS.SIM  1s.UNM  1s.ACC  take-PA-3s

`He came and married me.'

The simultaneous action form is less marked than the sequential action form: when the relative order of the actions or events is not relevant, the simultaneous action form is used. In the following example, the order of the preparations for a pighunt is not crucial, but the sequential action form on the last medial verb indicates that all the actions take place before leaving, rather than just at the time of leaving.  

\ea%x1437
\label{ex:x1437}
\gll Maa  en-ep-pu-\textstyleEmphasizedVernacularWords{ami}  top  aaw-\textstyleEmphasizedVernacularWords{emi}  moma  \\
      \\
\glt
\z

food  eat-SS.SEQ-CMPL-SS.SIM  trap  take-SS.SIM  taro

unukum-\textstyleEmphasizedVernacularWords{emi}  kapit,  wiowa  aaw-\textstyleEmphasizedVernacularWords{ep}  fikera

wrap-SS.SIM  trap.frame  spear  take-SS.SEQ  kunai.grass

iw-i-mik.

go-Np-PR.1/3p

`We eat, take the trap, wrap taro, take the the trap frame and spear(s) and go to the kunai grass area.'

A medial verb takes its temporal specification from the tense of the closest following finite clause, or in the case of a right-dislocated medial clause, from the preceding finite clause.

\ea%x1442
\label{ex:x1442}
\gll Nomokowa  maala  war-ep    ekap-ep  ifa  nain  ifakim-\textstyleEmphasizedVernacularWords{o}-k. \\
      \\
\glt
\z

tree  long  cut-SS.SEQ  come-SS.SEQ  snake  that1  kill-PA-3s

`He cut a long stick, came and killed the snake.'

\ea%x1444
\label{ex:x1444}
\gll Mua=ke  kais-ap  neeke  wu-ap  miiw-aasa  nop-ap  \\
      \\
\glt
\z

man=CF  husk-SS.SEQ  there.CF  put-SS.SEQ  land-canoe  fetch-SS.SEQ

miiw-aasa=ke  iwer(a)  ififa  nain  aaw-ep  p-ekap-ep

land-canoe=CF  coconut  dry  that1  take-SS.SEQ  BPx-come-SS.SEQ  

epia  koora  mamaiya=pa  wu-eya  fook-\textstyleEmphasizedVernacularWords{i-mik}.

fire  house  near=LOC  put-2/3s.DS  split-Np-PR.1/3p

`Men husk them (coconuts) and put them there and fetch a truck, and the truck takes the dry coconuts and brings them close to the drying shed (lit: fire house), and we split them.'

\ea%x1445
\label{ex:x1445}
\gll Ikoka  mua  ar-ep  emeria  aaw-ep  kamenap  on-\textstyleEmphasizedVernacularWords{i-nan}? \\
      \\
\glt
\z

later  man  become-SS.SEQ  woman  take-SS.SEQ  how  do-Np-FU.2s

`Later when you become a man and take a wife, what will you do?'

The \textstyleAcronymallcaps{DS} medial verbs (\sectref{sec:3.8.3.5.2}) do not differentiate between sequential and simultaneous action. Sequential action (\stepcounter{nx}{\thenx}) is the default interpretation for verbs other than \textstyleStyleVernacularWordsItalic{ik}- `be', which is interpreted as simultaneous with the verb in the reference clause (\stepcounter{nx}{\thenx}). So in order to specify that two or more actions by different participants took place at the same time, the speaker needs to use the continuous aspect form (\stepcounter{nx}{\thenx}):

\ea%x1502
\label{ex:x1502}
\gll Maa  unowa  ifer-aasa=ke  p-urup\textstyleEmphasizedVernacularWords{-eya}  miiw-aasa=ke  \\
      \\
\glt
\z

thing  many  sea-canoe=CF  Bpx-ascend-2/3s.DS  land-canoe=CF

fan  p-ir-am-ik-ua.

here  Bpx-come-SS.SIM-be-PA.3s

`The cargo was brought up (to the coast) by ship(s), and (then) trucks kept bringing it here.'

\ea%x1503
\label{ex:x1503}
\gll Wi  yapen=pa  \textstyleEmphasizedVernacularWords{ik-}omak\textstyleEmphasizedVernacularWords{-iwkin}  Amerika  kerer-e-mik. \\
      \\
\glt
\z

3p.UNM  inland=LOC  be-DISTR/PL-2/3p.DS  America  appear-PA-1/3p

`Many people were inland and the Americans arrived.'

\ea%x1472
\label{ex:x1472}
\gll Ek-ap  umuk-i-nen  na-ep  on-am\textstyleEmphasizedVernacularWords{-ik-eya} \\
      \\
\glt
\z

go-SS.SEQ  extinguish-Np-FU.1s  say-SS.SEQ  do-SS.SIM-be-2/3s.DS

ifa=ke  keraw-a-k,  ...

snake=CF  bite-PA-3s

`He went and as he was trying to extinguish it (a fire), a snake bit him, {\dots}'

Although the chaining structure itself only specifies the temporal relationship between the clauses and is neutral otherwise, it is open especially for causal/consecutive interpretation. \citet[237]{Reesink1983} notes this for different-subject medial verbs in Usan, and although not very common in Mauwake in general, it is more frequent with \textstyleAcronymallcaps{DS} predicates than with \textstyleAcronymallcaps{SS} verbs:

\ea%x1434
\label{ex:x1434}
\gll Yo  maamuma  marew\textstyleEmphasizedVernacularWords{-eya } maak-e-m, \\
      \\
\glt
\z

1s.UNM  money  no(ne)-2/3s.DS  tell-PA-1s

{\textquotedblIir  oko}=pa  ni-i-nen.''

time  other=LOC  give.you-Np-FU.1s

`I had no money and I told him (or: Because I had no money I told him), ``I'll give it to you another time.'' '

\ea%x1412
\label{ex:x1412}
\gll Iperowa=ke  kekan-\textstyleEmphasizedVernacularWords{iwkin}  ma-e-mik,  ``Aria,  ...'' \\
      \\
\glt
\z

middle.aged=CF  be.strong-2/3p.DS  say-PA-1/3p  alright

`The elders insisted, and (so) we said, ``All right, {\dots}'' '

The causal/consecutive interpretation is most common when the object of a transitive medial clause becomes the subject in an intransitive reference clause: in the following example `the son' is the object of the first two clauses and the subject of the final clause.

\ea%x1504
\label{ex:x1504}
\gll [\textstyleEmphasizedVernacularWords{Muuka}]\textsubscript{O}  p-or-op  \textstyleEmphasizedVernacularWords{p-er-iwkin}  \textstyleEmphasizedVernacularWords{yak-i-ya}. \\
      \\
\glt
\z

son  Bpx-descend-SS.SEQ  Bpx-go-2/3p.DS  bathe-Np-PR.3s

`They bring the son down (from the house) and take him (to the well) and (so) he bathes.'

Cognition verbs and feeling or experiential verbs seem to be the only ones that allow a causal/consecutive interpretation when a medial clause has a \textstyleAcronymallcaps{SS} verb:

\ea%x1440
\label{ex:x1440}
\gll Siiwa,  epa  maak-e-mik  nain  \textstyleEmphasizedVernacularWords{paayar-ep}  ma-e-k,  \\
      \\
\glt
\z

moon  place/time  tell-PA-1/3p  that1  understand-PA-SS.SEQ  say-PA-3s

``Amerika  aakisa  irak-owa  kerer-e-mik.''

America  now  fight-NMZ  appear-PA-1/3p

`He understood the month and time/place that they (had) told him, and (so) he said, ``Now the Americans have come to fight.'' '

\ea%x1441
\label{ex:x1441}
\gll ...  ne  wi  ikiw-e-mik,  \textstyleEmphasizedVernacularWords{kerewar-ep}  ikiw-e-mik. \\
      \\
\glt
\z

...  ADD  3p.UNM  go-PA-1/3p  become.angry-SS.SEQ  go-PA-1/3p

`{\dots} and they went; they were angry and (so) they went.'

\ea%x1484
\label{ex:x1484}
\gll Mua  oko=ko  \textstyleEmphasizedVernacularWords{napum-ar-}\textstyleEmphasizedVernacularWords{ep}  ikemika  kaik-ow(a)  mua \\
      \\
\glt
\z

man  other=CF  sickness-INCH-SS.SEQ  wound  tie-NMZ  man

wiar  ikiw-o-k.

3.DAT  go-PA-3s

`A man got sick and (so) he went to a doctor.'

\subsubsection{Person reference in chained clauses}
\hypertarget{RefHeading23161935131865}{}
The switch-reference marking tracks the referents in a different way from the person/number marking in finite verbs. The medial verb suffix indicates whether the clause has the same subject/topic as the reference clause that comes after it, and the \textstyleAcronymallcaps{DS} suffixes also have some specification of the subject (\sectref{sec:3.8.3.4.2}). In the following example the subjects are a man and his wife in the first two clauses and in the last one, and a spirit man in all the others:

\ea%x1436
\label{ex:x1436}
\gll Ikiw-\textstyleEmphasizedVernacularWords{ep}\textsubscript{i}  nan  ika-\textstyleEmphasizedVernacularWords{iwkin}\textsubscript{i}  inasina  mua\textsubscript{j}  ifa  \\
      \\
\glt
\z

go-SS.SEQ  there  be-2/3p.DS  spirit  man  snake

puuk-\textstyleEmphasizedVernacularWords{ap}\textsubscript{j}  solon-\textstyleEmphasizedVernacularWords{ep}\textsubscript{j}  urup-\textstyleEmphasizedVernacularWords{ep}\textsubscript{j}  manina=pa  waaya

change.into-SS.SEQ  crawl-SS.SEQ  ascend-SS.SEQ  garden=LOC  pig

puuk-\textstyleEmphasizedVernacularWords{ap}\textsubscript{j}  moma  wiar  en-em-ik-\textstyleEmphasizedVernacularWords{eya}\textsubscript{j}  uruf-a-mik\textsubscript{i}.

change.into-SS.SEQ  taro  3.DAT  eat-SS.SIM-be-2/3s.DS  see-PA-1/3s

`They went and were there, and a spirit man came and changed into a snake and crawled up and in the garden it changed into a pig and as it was eating their taro they saw it.'

Because the \textstyleAcronymallcaps{SR} marking relates to the subject/topic in two different clauses at the same time, this sometimes causes ambiguities that need to be solved. If the subjects in adjacent clauses are partially same and partially different, a choice has to be made whether they are marked as \textstyleAcronymallcaps{SS} or \textstyleAcronymallcaps{DS}; only a few Papuan languages have a choice of marking both \textstyleAcronymallcaps{SS} and \textstyleAcronymallcaps{DS} on the same verb \citep{Roberts1997}. Also, if the \textstyleAcronymallcaps{SR} marking is considered to track the syntactic subject, there are a number of apparent irregularities in the marking. These have been discussed in particular by \citet{Reesink1983a} and \citet{Roberts1988b} with reference to Papuan languages. The next three subsections describe how Mauwake deals with these questions. 

\paragraph[Partitioning of the participant set]{Partitioning of the participant set}
\hypertarget{RefHeading23181935131865}{}
When one of the subjects is plural and the other is singular included in the plural, this mismatch theoretically allows for a number of different choices in the switch-reference marking, but in practice each language limits this choice in a way peculiar to it.\textsuperscript{} \footnote{For a summary of how different Papuan languages treat this area of ambiguity, see Reesink (1983a, 1987:201-202) and \citet[87-91]{Roberts1988}.} The following table shows for Mauwake, where one alternative has to be chosen and which one, and where there is a choice. 





\begin{tabular}{llll}
\mytoprule


\multicolumn{2}{l}{{\bfseries Singular to plural}

 & \multicolumn{2}{l}{{\bfseries Plural to singular}

\\
1s  {\textgreater}  1p & SS & 1p  {\textgreater}  1s & SS\\
2s  {\textgreater}  1p & DS & 1p  {\textgreater}  2s & SS\\
2s  {\textgreater}  2p & SS/DS & 1p  {\textgreater}  3s & SS\\
3s  {\textgreater}  1p & SS/DS & 2p  {\textgreater}  2s & SS\\
3s  {\textgreater}  2p & SS/DS & 2p  {\textgreater}  3s & SS\\
3s  {\textgreater}  3p & SS/DS & 3p  {\textgreater}  3s & SS\\
\mybottomrule
\end{tabular}



\begin{table}
\caption{Switch-reference marking with partial overlap of subjects}
\label{tab:15}
\end{table}

When a plural subject changes into a singular, the suffix is always the one used for same subject. 

\ea%x1438
\label{ex:x1438}
\gll {\dots}owowa  urup-e-mik.  Owowa  urup-\textbf{ep } o  koora \\
      \\
\glt
\z

{\dots}village  ascend-PA-1/3p  village  ascend-SS.SEQ  3s.UNM  house

ikiw-o-k.

go-PA-3s

`{\dots}we came up to the village. After we came up to the village he went into the house.'

When a singular subject changes into a plural there is more variation. First person singular changing into plural calls for same-subject marking (\stepcounter{nx}{\thenx}), but second person singular switching into first person plural requires different-subject marking even when this second person singular is part of the group denoted by the first person plural (\stepcounter{nx}{\thenx}). 

\ea%x1435
\label{ex:x1435}
\gll Mik-ap,  patot=iw  mik-ap,  aaw-ep,  \\
      \\
\glt
\z

spear-SS.SEQ  fishing.spear=INST  spear-SS.SEQ  take-SS.SEQ

aasa=pa  wu-ap,  amap-urup-ep,  yena  koora=pa

canoe=LOC  put-SS.SEQ  BPx-ascend-SS.SEQ  1s.GEN  house=LOC

wu-ap,  uuriw  epa  wiim-eya  or-op,  

put-SS.SEQ  morning  place  get.light-2/3s.DS  descend-SS.SEQ 

saa=pa  pa-\textstyleEmphasizedVernacularWords{ep}  uup-e-mik.

sand=LOC  butcher-SS.SEQ  cook-PA-1/3p

`I speared it, I speared it with a fishing spear, and took it and put it in the canoe, brought it up and put it in my house, and in the morning when it was light I went down and butchered it on the beach, and \textstyleEmphasizedWords{we} cooked it.'

\ea%x1439
\label{ex:x1439}
\gll Ekap-\textbf{eya}  ikiw-i-yen. \\
      \\
\glt
\z

come-2/3s.DS  go-Np-FU.1p

`When you come we (including you) will go.'

When a second person plural switches into a first person plural (including the people indicated by the 2p), the marking has to be for different subject, but in the opposite case, the first person plural changing into the second person plural (again included in the 1p), the marking can be either for same or different subject. Both of these are exemplified in (\stepcounter{nx}{\thenx}), repeated below. Here the switch from first person plural to second person plural is marked with the \textstyleAcronymallcaps{SS} marking.

  \stepcounter{nx}{\thenx}x248)  I  ikoka  yien=iw  urup-ep  nia  maak-omkun\\
1p.UNM  later  1p.GEN=LIM  ascend-SS.SEQ  2p.ACC  tell-1s/p.DS

ora-\textstyleEmphasizedVernacularWords{iwkin}  aria  owawiya  feeke  pok-ap  ik-ok

descend-2/3p.DS  alright  together  here.CF  sit-SS.SEQ  be-SS

eka  liiwa  muuta  en-\textbf{ep}  aria  ni  soomar-ek-eka.

water  a.little  only  eat-SS.SEQ  alright  2p.UNM  walk-go-IMP.2p

`Later we (by) ourselves will come up and tell you (to come), and when you come down we will sit here together and eat a bit of something and then you (can) walk back.'

With the rest, the speaker has a choice between the two forms. This choice is probably pragmatic and depends on whether the speaker wants to stress the change or the continuity of the referents \citep[47]{Franklin1983}. 

\paragraph[Tracking a subject high in topicality ]{Tracking a subject high in topicality} 
\hypertarget{RefHeading23201935131865}{}
Haiman and Munro (1983:xi) claim that it is strictly the syntactic subject whose reference is tracked, but this statement has been challenged and modified by several others.\footnote{Giv\'on (1983), Reesink (1983a and 1987), Roberts (1988b and 1997), and \citet{Farr1999} among others.} If it is accepted as such, both Mauwake and other Papuan languages present a number of irregularities that have to be explained somehow.

\citet[242-3]{Reesink1983a} suggests that the switch-reference system does monitor the subject co-referentiality in the medial clause and its reference clause, but topicality considerations cause apparent ``anomalies'' to the basic system.  \citet{Roberts1988b} makes a well supported claim for Amele that in fact it is the topic that is tracked rather than the syntactic subject, or semantic agent, and he tentatively extends the claim to cover other Papuan languages as well. His later survey (1997) presents a more balanced view that \textstyleAcronymallcaps{SR} can be either agent-oriented or topic-oriented, while maintaining that in most Papuan languages it is topic-oriented. 

In a nominative-accusative language like Mauwake the syntactic subject, the semantic agent and the pragmatic topic coincide most of the time. The \textstyleAcronymallcaps{SR} marking tracks the subject, but when there is competition between a more topical and less topical subject in clause chains, it is the more topical one that is tracked. An object, even if it is the topic, does not participate in the \textstyleAcronymallcaps{SR} marking.

Competition between a more topical subject with a less topical one most commonly occurs when a clause with an inanimate subject intervenes between clauses where there is an animate/human subject. Even here the ``normal'' \textstyleAcronymallcaps{SR} strategy is used, if the inanimate subject is topical enough to control the \textstyleAcronymallcaps{SR} marking in the same way as animate subjects do. In the following examples, the drying of the soup (\stepcounter{nx}{\thenx}) and the bending of the coconut palm (\stepcounter{nx}{\thenx}) are important events in the development of the story and so the regular \textstyleAcronymallcaps{SR} marking is maintained. In (\stepcounter{nx}{\thenx}) the coconut palm can also be interpreted as a volitional participant, as it bends and straightens itself according to the need of the people.

\ea%x1474
\label{ex:x1474}
\gll Uup-em-ika-\textstyleEmphasizedVernacularWords{iwkin } maa  eka  saanar-em-ik-\textstyleEmphasizedVernacularWords{eya} \\
      \\
\glt
\z

cook-SS.SIM-be-2/3p.DS  food  water  dry-SS.SIM-be-2/3s.DS

iki(w-e)p  eka  un-ep  ekap-ep  amina=pa

go-SS.SEQ  water  draw-SS.SEQ  come-SS.SEQ  pot=LOC

feef-am-ik-e-mik.

pour-SS.SIM-be-PA-1/3p

`They were cooking it and the soup kept drying and they kept going and drawing water and coming and pouring it in the pot.'

\ea%x1480
\label{ex:x1480}
\gll Emeria  panewowa  nain  wiimasip  erup  wia  aaw-ep \\
      \\
\glt
\z

woman  old  that1  3s/p.grandchild  two  3p.ACC  take-SS.SEQ

owow  uruma  or-op  iimar-ep  ika-\textstyleEmphasizedVernacularWords{iwkin}

village  open.place  descend-SS.SEQ  stand.up-SS.SEQ  be-2/3p.DS

iwera  oko  mekemkar-ep  or-\textstyleEmphasizedVernacularWords{eya}  wi  iwera

coconut  other  bend-SS.SEQ  descend-2/3s.DS  3p.UNM  coconut

ir-\textstyleEmphasizedVernacularWords{iwkin}  nainiw  kaken  iimar-e-k.

climb-2/3p.DS  again  straight  stand.up-PA-3s

`The old woman took the two grandchildren and they went down to the village square and were standing there, and a coconut palm bent down and they climbed up the coconut palm and it stood up straight again.'

When an inanimate subject is topically low, the \textstyleAcronymallcaps{SR} marking of the previous clause disregards it and indicates same-subject continuation, but the verb in the inanimate clause has to indicate a change of subject, if a more topical subject follows. This structure in many Papuan languages is typical of temporal and climate expressions and other impersonal predications (Reesink 1983a, Roberts 1988b), which are often used for giving backgrounded\footnote{\citet[244]{Farr1999} calls this \textit{on-line background} to distinguish it from the off-line background information of subordinate clauses.} information. In the following examples the verb in the initial medial clause predicating the action of human participants is marked with same subject following even when the following clause mentions the coming of darkness or dawn. Returning to the main line action requires different-subject marking. In the examples, the ``skipped'' medial clauses are in brackets.

\ea%x1482
\label{ex:x1482}
\gll Aria  maa  en-ep  naap  ik-\textstyleEmphasizedVernacularWords{ok}  [kokom-ar-\textstyleEmphasizedVernacularWords{e}\textstyleEmphasizedVernacularWords{y}\textstyleEmphasizedVernacularWords{a}] \\
      \\
\glt
\z

alright  food  eat-SS.SEQ  thus  be-SS  dark-INCH-2/3s.DS

in-e-mik.

sleep-PA-1/3p

`Alright we ate and stayed like that and (then) it became dark and we slept.'

\ea%x1475
\label{ex:x1475}
\gll In-\textstyleEmphasizedVernacularWords{ep}  [epa  wiim-\textstyleEmphasizedVernacularWords{eya}]  onak  maak-e-mik,  ``{\dots''} \\
      \\
\glt
\z

sleep-SS.SEQ  place  dawn-2/3s.DS  3s/p.mother  tell-PA-1/3p

`They slept, and when it dawned they told their mother, ``{\dots}'' '

If the impersonal predicate is important for the main story line, rather than providing backgrounded information, the impersonal verb itself is placed as a final verb, and the verb in the preceding medial clause is marked for different subject. \citet[206]{Reesink1987} notes a similar rule for Usan. 

\ea%x1492
\label{ex:x1492}
\gll Kir-ep  ekap-em-ika-\textstyleEmphasizedVernacularWords{iwkin}  epa  wiim-o-k. \\
      \\
\glt
\z

turn-SS.SEQ  come-SS.SIM-be-2/3p.DS  place  dawn-PA-3s

`They turned and as they were coming, it dawned.'

In many Papuan languages the impersonal predications include a number of experiential verbs (Reesink 1987:204, Roberts 1997). In Mauwake most of the experiential expressions are adjunct plus verb constructions (\sectref{sec:3.8.5.2.1}), where the experiencer is a subject rather than an object; in chained clauses these behave in a regular\textstyleAcronymallcaps{} manner. But those few experiential expressions that are impersonal do not trigger \textstyleAcronymallcaps{DS} marking in the preceding medial clause, because the inanimate subject in the experiential clause is not topical enough to do it. In (\stepcounter{nx}{\thenx}) the first person singular subject of the medial clauses becomes the object of the final clause, but the medial clause has same subject marking:

\ea%x1491
\label{ex:x1491}
\gll Uuw-ap  uuw-\textstyleEmphasizedVernacularWords{ap}  oona=ke  efa  sirir-i-ya. \\
      \\
\glt
\z

work-SS.SEQ  work-SS.SEQ  bone=CF  1s.ACC  hurt-Np-PR.3s

`I worked and worked and my bones hurt.'

The verb \textstyleStyleVernacularWordsItalic{weeser}- `finish' is often used in chained clauses to indicate the finishing of an action. In this function its low-topicality subject, the nominalized form of the preceding verb, is never mentioned overtly, and the preceding medial clause has \textstyleAcronymallcaps{SS} marking:

\ea%x1483
\label{ex:x1483}
\gll Uup-\textstyleEmphasizedVernacularWords{ep}  [weeser-\textstyleEmphasizedVernacularWords{eya}]  aria  oposia  gelemuta  wiam  erup \\
      \\
\glt
\z

cook-SS.SEQ  finish-2/3s.DS  alright  meat  small  3p.REFL  two

fain  wia  wu-om-a-m.

this  3p.ACC  put-BEN-BNFY2.PA-1s

`I cooked it and when it was finished alright I put (aside) a little of the meat for these two (women).'

In (\stepcounter{nx}{\thenx}) there are two intervening clauses with different low-topicality inanimate subjects. The same-subject marking of the first clause ``jumps over'' these two clauses and refers to the subject in the last clause. The two clauses in between both have \textstyleAcronymallcaps{DS} marking. 

\ea%x1476
\label{ex:x1476}
\gll Maa  uup-\textstyleEmphasizedVernacularWords{ep } [fofola  urup-\textstyleEmphasizedVernacularWords{eya}]  [maa  op-\textstyleEmphasizedVernacularWords{iya}]  \\
      \\
\glt
\z

food  cook-SS.SEQ  foam  rise-2/3s.DS  food  be.done-2/3s.DS

iiw-o-k.

dish.out-PA-3s

`She cooked the food and when it boiled and was done she dished it out.'

Although human subjects are typically high in the topicality hierarchy (Giv\'on 1984:364), even a human subject may occasionally be so low in topicality that it gets overlooked in the \textstyleAcronymallcaps{SR} marking.\footnote{\citet[236-7]{Reesink1983a} gives similar examples from other Papuan languages.} What is particularly striking with the example (\stepcounter{nx}{\thenx}) is that the clause that is overlooked has a subject in first person singular, which is usually considered to be topically the highest possible subject. A plausible explanation is that politeness and hospitality requires the host of a big meal to downplay his own importance in this way. 

\ea%x1477
\label{ex:x1477}
\gll Efa  arew-\textstyleEmphasizedVernacularWords{ap}  [maa  eka  liiwa  muuta  on-\textstyleEmphasizedVernacularWords{amkun}]  \\
      \\
\glt
\z

1s.ACC  wait-SS.SEQ  food  water  little  only  make-1s/p.DS

en-ep-pu-ami  soomar-ek-eka.

eat-SS.SEQ-CMPL-SS.SIM  walk-go-IMP.2p

`Wait for me, and when I have made just some soup you eat it and then you (may) go.'

\ea%x1478
\label{ex:x1478}
\gll Ikiw-\textstyleEmphasizedVernacularWords{ep}  [mua  nain  urema  osarena=pa  iimar-ep  \\
      \\
\glt
\z

go-SS.SEQ  man  that1  bandicoot  path=LOC  stand-SS.SEQ  

ik-\textstyleEmphasizedVernacularWords{eya}]  ona  mua  nain  ifakim-o-k.

be-2/3s.DS  3s.GEN  man  that1  kill-PA-3s

`She went and as the man was standing on the bandicoot path she killed that husband of hers.'

In process descriptions the identity of people performing the actions is not important, and their topicality is low. In (\stepcounter{nx}{\thenx}) the person watching the fire in the coconut drying shed is not mentioned in any way. The example is also like (\stepcounter{nx}{\thenx}) above in that there are two clauses with a different low-topicality subject, here one of them [+human], intervening between the second \textstyleAcronymallcaps{SS} clause and the final clause, where the original subject is picked up.

\ea%x1481
\label{ex:x1481}
\gll Epia  wu-ap  ikiw-\textstyleEmphasizedVernacularWords{ep}  [iwera  kuuf-am-ik-\textstyleEmphasizedVernacularWords{eya}]  \\
      \\
\glt
\z

fire  put-SS.SEQ  go-SS.SEQ  coconut  watch-SS.SIM-be-2/3s.DS

[iwera  reen-\textstyleEmphasizedVernacularWords{eya}]  iwer  urupa  anum-i-mik.

coconut  dry-2/3s.DS  coconut  shell  knock-Np-PR.1/3p

`We/they put them (the coconuts) on the fire and go, and (someone) keeps watching the coconuts and they dry and (then) we/they knock the shells away.'

Even an inanimate subject may override an animate/human one in \textstyleAcronymallcaps{SR} marking, if its topicality is high enough. In the following example the subject/topic is \textstyleForeignWords{kunai} grass and the burning of the grass, which is such an important part of a pighunt that the hunt itself is called \textstyleStyleVernacularWordsItalic{fiker(a) kuumowa} `kunai-burning'.  The grass is a continuing topic from the previous several sentences, so a \textstyleAcronymallcaps{NP} is not used for marking it. 

\ea%x1479
\label{ex:x1479}
\gll Kuum-\textstyleEmphasizedVernacularWords{iwkin}  aw-\textstyleEmphasizedVernacularWords{emi}  [mua  unow  maneka  iiwawun  fikera \\
      \\
\glt
\z

burn-2/3p.DS  burn-SS.SIM  man  many  very  altogether  kunai.grass

kuum-emi  saawirin-ow-\textstyleEmphasizedVernacularWords{iwkin}]  aria  fiker  epia

burn-SS.SIM  round-CAUS-2/3p.DS  alright  kunai.grass  fire

aw-i-non.

burn-Np-FU.3s

`They burn it and it burns and all the men burn and surround the kunai grass, (and) alright the kunai fire will burn.'

\paragraph[Apparent mismatches of reference]{Apparent mismatches of reference}
\hypertarget{RefHeading23221935131865}{}
A medial verb with \textstyleAcronymallcaps{DS} marking is used in two instances where it does not indicate a change of subject. Both types have two or more clauses with identical \textstyleAcronymallcaps{DS} marking even though the subject is the same; only the last of those clauses really indicates a change of subject. One of them is recursion of a \textstyleAcronymallcaps{DS} verb, indicating continuity; the identification of the subject is suspended until the repetition ends \citep[201]{Reesink1987}. 

\ea%x1493
\label{ex:x1493}
\gll Wiawi  kuum-\textstyleEmphasizedVernacularWords{eya}  kuum-\textstyleEmphasizedVernacularWords{eya}  kuum-\textstyleEmphasizedVernacularWords{eya}  aw-ep  \\
      \\
\glt
\z

3s/p.father  burn-2/3s.DS  burn-2/3s.DS  burn-2/3s.DS  burn-SS.SEQ

eka  iw-a-k  na  wia,  eka=ke  saanar-e-k.

river  enter-PA-3s  but  no  river=CF  dry-PA-3s

`It kept burning and burning their father and he burned and entered the river but no, the river dried.'

A medial clause that has the same subject as the following medial clause may have \textstyleAcronymallcaps{DS} marking if both the medial clauses relate to the same finite clause as their reference clause, and the first of the medial clauses gets expanded or defined more closely in the second one. The \textstyleAcronymallcaps{DS} verbs may be identical (\stepcounter{nx}{\thenx}), but they do not need to be (\stepcounter{nx}{\thenx}).

\ea%x1494
\label{ex:x1494}
\gll Efa  uruf-am-ik-\textstyleEmphasizedVernacularWords{eya},  koora=pa  efa  uruf-am-ik-\textstyleEmphasizedVernacularWords{eya} \\
      \\
\glt
\z

1s.ACC  see-SS.SIM-be-2/3.DS  house=LOC  1s.ACC  see-SS.SIM-be-2/3.DS

ikiw-i-nen  ekap-i-nen.

go-Np-FU.1s  come-Np-FU.1s

`You will keep seeing me, you will keep seeing me from the house, and I will come and go.'

\ea%x1495
\label{ex:x1495}
\gll ...pon  sisina=pa  ik-\textstyleEmphasizedVernacularWords{eya},  piipa  unowa=pa  \\
      \\
\glt
\z

{\dots}turtle  shallow.water=LOC  be-2/3s.DS  seaweed  many=LOC

soomar-em-ik-\textstyleEmphasizedVernacularWords{eya}  mik-a-m.

walk-SS.SIM.be-2/3s.DS  spear-PA-1s

`{\dots} the turtle was in the shallow water, it was walking among a lot of seaweed and I speared it.'

\ea%x1496
\label{ex:x1496}
\gll No  ikoka  era=pa  wia  far-\textstyleEmphasizedVernacularWords{eya},  owora  wia  \\
      \\
\glt
\z

2s.UNM  later  road=LOC  3p.ACC  call-2/3s.DS  betelnut  3p.ACC

maak-\textstyleEmphasizedVernacularWords{eya},  aria  mua=ke  naap  me  nefa  ma-i-nok,  ``...''

tell-2/3s.DS  alright  man=CF  thus  not  2s.ACC  say-Np-FU.3s

`Later when you see them on the road, when you ask them for betelnut, alright let your husband not say about you that {\dots}'

The \textstyleAcronymallcaps{SS} medial form of the verb `be' is used in the expression \textstyleStyleVernacularWordsItalic{naap ikok}  `it is/was thus (and)', regardless of the following subject/topic. The construction seems to have grammaticalized as an expression of an indefinite time span.

\ea%x1500
\label{ex:x1500}
\gll \textstyleEmphasizedVernacularWords{Naap}  \textstyleEmphasizedVernacularWords{ik-ok}  wi  Saramun=ke  wiisa  uf-e-mik. \\
      \\
\glt
\z

thus  be-SS  3p.UNM  Saramun=CF  dance.name  dance-PA-1/3p

`It was like that and (then) the Saramun people danced \textstyleForeignWords{wiisa}.'

\ea%x1501
\label{ex:x1501}
\gll ...mua  me  wia  imen-a-mik.  \textstyleEmphasizedVernacularWords{Naap  ik-ok}  sarere  uura  \\
      \\
\glt
\z

{\dots}man  not  3p.ACC  find-PA-1/3p  thus  be-SS  Saturday  night

buburia  ona  amia  mua  wiar  kerer-ep

bald  3s.GEN  bow  man  3.DAT  appear-SS.SEQ  

opaimika=pa  yia  wu-a-k.

talk=LOC  1p.ACC  put-PA-3s

`{\dots} we didn't find the men. It was like that, and on Saturday evening the bald man himself went to the police and accused us.'

Even when the final clause is verbless (\stepcounter{nx}{\thenx}), (\stepcounter{nx}{\thenx}), or missing completely because of ellipsis (\stepcounter{nx}{\thenx}), a medial clause is still possible. In both cases the \textstyleAcronymallcaps{SR} marking is based on what the expected subject would be, if there were one. 

\ea%x1497
\label{ex:x1497}
\gll Naap  ik-\textstyleEmphasizedVernacularWords{ok}  uruf-am-ika-\textstyleEmphasizedVernacularWords{iwkin}  wia. \\
      \\
\glt
\z

thus  be-SS  see-SS.SIM-be-2/3p.DS  no

`He was like that and they were watching him, but no (he didn't get any better).'

\ea%x1498
\label{ex:x1498}
\gll Iinan  aasa  gurun-owa  miim-\textstyleEmphasizedVernacularWords{ap}  eka=iw  umuk-owa  \\
      \\
\glt
\z

sky  canoe  rumble-NMZ  hear-SS.SEQ  water=INST  extinguish-NMZ

ewur.

quickly

`We heard the rumble of the airplane(s) and quickly extinguished (the fires) with water (lit: and the extinguishing with water quickly).'

The two sentences preceding the example sentence (\stepcounter{nx}{\thenx}) mention American airplanes that flew over and dropped messages during the Second World War. The ``same subject'' needs to be picked from there -- as the story continues without another reference to the Americans for a while -- and the elliptical clause construed as something like \textstyleStyleVernacularWordsItalic{naap onamik} `and they did like that'.

\ea%x1499
\label{ex:x1499}
\gll Wi  Yaapan  nan  ik-e-mik  nain  wia  uruf-\textstyleEmphasizedVernacularWords{ap}.  \\
      \\
\glt
\z

3p.UNM  Japan  there  be-PA-1/3p  that1  3p.ACC  see-SS.SEQ

`They had seen that the Japanese were there (and so they [the Americans] did like that).'

\paragraph[Medial clauses as a complementation strategy for perception verbs]{Medial clauses as a complementation strategy for perception verbs}
\hypertarget{RefHeading23241935131865}{}
Perception verbs in Mauwake mostly use a medial clause as a complementation strategy \citep[371]{Dixon2010}, when the object of the perception verb is an \textstyleEmphasizedWords{\textsc{activity}}.\footnote{Reesink notes this for Usan too (1983:237).} Regular, nominalized complement clauses are only used with perception verbs when a \textstyleEmphasizedWords{\textsc{fact}} is reported (8.3.2.2).

\ea%x1509
\label{ex:x1509}
\gll Moma  wiar  \textstyleEmphasizedVernacularWords{en-em-ik-eya}  uruf-a-mik. \\
      \\
\glt
\z

taro  3.DAT  eat-SS.SIM-be-2/3s.DS  see-PA-1/3p

`It was eating their taro, and they saw it.' (Or: `They saw that it was eating their taro.')

\ea%x1510
\label{ex:x1510}
\gll Aara  \textstyleEmphasizedVernacularWords{muuk}\textstyleEmphasizedVernacularWords{-}\textstyleEmphasizedVernacularWords{ar}\textstyleEmphasizedVernacularWords{-ep  ik-}\textstyleEmphasizedVernacularWords{eya}  uruf-a-mik. \\
      \\
\glt
\z

hen  son-INCH-SS.SEQ  be-2/3s.DS  see-PA-1/3p

`The hen had laid an egg and we saw it.' (Or: `We saw that the hen had laid an egg.')

\ea%x1511
\label{ex:x1511}
\gll Yo  me  baliwep  paayar-e-m,  oram  iperowa=ke \\
      \\
\glt
\z

1s.UNM  not  well  understand-PA-1s  just  middle.aged=CF

\textstyleEmphasizedVernacularWords{nanar}\textstyleEmphasizedVernacularWords{-}\textstyleEmphasizedVernacularWords{iwkin}  miim-a-m.

tell.story-2/3p.DS  hear-PA-1s

`I do not understand it well, I have just heard the older people tell stories about it.'

\paragraph[Tail-head linkage]{Tail-head linkage}
\hypertarget{RefHeading23261935131865}{}
Tail-head linkage is a typical feature especially in oral texts\footnote{With the development of written style this feature is getting less prominent.}  in Papuan languages. It is an inter-sentential cohesive device and could be understood to belong outside ``syntax proper'', if syntax is defined very narrowly. It is mentioned here as it is an important linking device, and the chaining structure is used for it. In narratives and process descriptions tail-head linkage is utilized to tie sentences together within a thematic paragraph. 

 The tail-head link is formed when one sentence ends in a finite clause (``tail''), and the next sentence begins with a medial clause (``head'') that copies the verb but changes it into a medial one. The information in this medial clause is given rather than new, unlike in most other medial clauses. \citet[200-201]{Foley1986} claims for Yimas, and assumes for the rest of Papuan languages, that these medial clauses are subordinate, but at least in Mauwake they are not -- they are coordinate like the other medial clauses. In a narrative the final verbs, which then get recapitulated in the next sentence, carry the core of the story line.

\ea%x1505
\label{ex:x1505}
\gll Wafur-a-k  na  weetak,  \textstyleEmphasizedVernacularWords{ufer-a-k}.  \textstyleEmphasizedVernacularWords{Ufer-ap} \\
      \\
\glt
\z

throw-PA-3s  but  no  miss-PA-3s  miss-SS.SEQ

nainiw  burir  aaw-ep  woosa=pa  aruf-eya  waaya

again  axe  take-SS.SEQ  head=LOC  hit-2/3s.DS  pig

nain  \textstyleEmphasizedVernacularWords{in-e-k}.  \textstyleEmphasizedVernacularWords{In-eya}  yena  ikiw-emi  nainiw

that1  lie.down-PA-3s  lie.down-2/3s.DS  1s.GEN  go-SS.SIM  again

wiowa  erup  ar-ow-amkun  iiwawun  \textstyleEmphasizedVernacularWords{um-o-k}.  \textstyleEmphasizedVernacularWords{Um-eya}

spear  two  become-CAUS-1s/p.DS  altogether  die-PA-3s  die-2/3s.DS

merena  ere-erup  kaik-ap  {\dots}

leg  RDP-two  tie-SS.SEQ

`He threw it (=a spear) but no, he missed. He missed it and again took an axe and hit it on the head and the pig fell down. It fell down and I myself went and speared it twice and it died completely. It died and I tied its legs two and two together and {\dots}'

The repeated verb retains its arguments, but there is a choice in how overtly they and the peripherals are marked in the medial clause. Retaining them makes the medial clause more emphatic, and the first element becomes a theme for the new sentence (\sectref{sec:9.1}). In (\stepcounter{nx}{\thenx}) only the verbs are copied; (\stepcounter{nx}{\thenx}) copies the subject as well, (\stepcounter{nx}{\thenx}) the object and (\stepcounter{nx}{\thenx}) the locative adverbial.

\ea%x1506
\label{ex:x1506}
\gll \textstyleEmphasizedVernacularWords{Miiw-aasa  samor-ar-e-k. } \textstyleEmphasizedVernacularWords{Miiw-aasa  samor-ar-eya}  {\dots} \\
      \\
\glt
\z

land-canoe  bad-INCH-PA-3s  land-canoe  bad-INCH-2/3s.DS

`The car broke. The car broke and {\dots}'

\ea%x1507
\label{ex:x1507}
\gll Owowa  or-op,  wailal-ep  \textstyleEmphasizedVernacularWords{akia  ik-e-k}.  \\
      \\
\glt
\z

village  descend-SS.SEQ  be.hungry-SS.SEQ  banana  roast-PA-3s

\textstyleEmphasizedVernacularWords{Akia  ik-ep}  en-em-ik-ok,  {\dots}

banana  roast-SS.SEQ  eat-SS.SIM-be-SS

`He came down to the village, was hungry and roasted bananas. He roasted bananas and was eating them, and {\dots}'

\ea%x1508
\label{ex:x1508}
\gll P-ikiw-ep  \textstyleEmphasizedVernacularWords{Bogia=pa  nan  wu-a-mik}.  \\
      \\
\glt
\z

Bpx-go-SS.SEQ  Bogia=LOC  there  put-PA-1/3p

\textstyleEmphasizedVernacularWords{Bogia}\textstyleEmphasizedVernacularWords{=pa  nan  wu-ap}  i  kiiriw  ekap-e-mik.

Bogia=LOC  there  put-SS.SEQ  1p.UNM  again  come-PA-1/3p

`We took it (=his body) and put/buried it in Bogia. We put it in Bogia and came back again.'

Most commonly the derivational morphology in the two verbs is identical, but sometimes the derivation in the finite verb is dropped from the medial verb. In (\stepcounter{nx}{\thenx}) there is a good reason for dropping the benefactive marking from the repeated verb: the spear was thrown for someone's benefit, but it missed and consequently there was no benefit for anyone.

\ea%x1513
\label{ex:x1513}
\gll Olas=ke  ekap-emi  wiowa  \textstyleEmphasizedVernacularWords{wafur-om-a-k}. \\
      \\
\glt
\z

Olas=CF  come-SS.SIM  spear  throw-BEN-BNFY2.PA-3s

\textstyleEmphasizedVernacularWords{Wafur-a-k}  na  weetak,  ufer-a-k.

throw-PA-3s  but  no  miss-PA-3s

`Olas came and threw a spear for him. He threw it but no, he missed.'

\ea%x1514
\label{ex:x1514}
\gll Epa  wiim-eya  mua  \textstyleEmphasizedVernacularWords{karer-omak-e-mik}. \\
      \\
\glt
\z

place  dawn-2/3s.DS  man  gather-DISTR/PL-PA-1/3p

\textstyleEmphasizedVernacularWords{Karer-a-p}  ma-e-mik,  ``{\dots''}

gather-SS.SEQ  say-PA-1/3p

`It dawned and many men gathered. They gathered and said, ``{\dots}'' '

Adding new derivation to the medial verb is possible, but rare: the example (\stepcounter{nx}{\thenx}) is repeated below as (\stepcounter{nx}{\thenx}).

\ea%x1515
\label{ex:x1515}
\gll Ikiwosa  wiar  pepekim-ep  \textstyleEmphasizedVernacularWords{kaik-a-m}.  \textstyleEmphasizedVernacularWords{Kaik-om-ap}{\dots} \\
      \\
\glt
\z

head  3.DAT  measure-SS.SEQ  tie-PA-1s  tie-BEN-BNFY2.SS.SEQ

`I measured her head and tied it (=headdress). I tied it for her and {\dots}'

 Similarly, the aspect marking normally stays the same in both the verbs, but it is also possible to have aspect marking on the medial verb, although the finite verb is without any aspect marking (\stepcounter{nx}{\thenx}). When new information is added to the verb either by derivation or aspect marking, it is less clear if this still is a true case of tail-head linkage.  

\ea%x1516
\label{ex:x1516}
\gll ...nomokowa  maala  war-ep,  ekap-ep  ifa  nain  \\
      \\
\glt
\z

{\dots}tree  long  cut-SS.SEQ  come-SS.SEQ  snake  that1

\textstyleEmphasizedVernacularWords{ifakim-o-k}.  \textstyleEmphasizedVernacularWords{Ifakim-em-ik-eya}  ifa  nain=ke

beat-PA-3s  beat-SS.SIM-be-2/3s.DS  snake  that1=CF

siowa  wasirk-a-k.

dog  release-PA-3s

`{\dots}he cut a long stick, came and beat up the snake. As he was beating it, the snake released the dog.'

\ea%x1517
\label{ex:x1517}
\gll Moma  manina  mokomokoka  \textstyleEmphasizedVernacularWords{nop}\textstyleEmphasizedVernacularWords{-}\textstyleEmphasizedVernacularWords{i}\textstyleEmphasizedVernacularWords{-}\textstyleEmphasizedVernacularWords{mik}. \\
      \\
\glt
\z

taro  garden  first  clear-Np-PR.1/3p

\textbf{Nop}\textbf{-}\textbf{ap-pu}\textbf{-}\textbf{ap}  nomokowa  warimik.

clear-SS.SEQ-CMPL-SS.SEQ  tree  cut-Np-PR.1/3p

`First we clear (the undergrowth for) taro garden. When we have cleared it we cut the trees.'

The tail-head linkage disregards right-dislocated items that come between the two verbs. 

\ea%x1518
\label{ex:x1518}
\gll Ne  kiiriw  nan  Medebur  \textstyleEmphasizedVernacularWords{ek-a-mik},  mua  napuma  onaiya. \\
      \\
\glt
\z

ADD  again  there  Medebur  go-PA-1/3p  man  sick  with

\textstyleEmphasizedVernacularWords{Ek-ap}  Medebur=pa  neeke  {\dots}

go-SS.SEQ  Medebur=LOC  there.CF

`And again from there they went to Medebur, with the sick man. They went and there in Medebur {\dots}'

\ea%x1519
\label{ex:x1519}
\gll ...pok-ap  ika-iwkin  mua  wiar  \textstyleEmphasizedVernacularWords{ekap-e-mik},  \\
      \\
\glt
\z

sit.down-SS.SEQ  be-SS.SEQ  man  3.DAT  come-PA-1/3p

wiinar-ep.  \textstyleEmphasizedVernacularWords{Ekap-emi}  wia  maak-e-mik,

make.planting.holes-SS.SEQ  come-SS.SIM  3p.ACC  tell-PA-1/3p

``Maa  iiw-eka.''

food  dish.out-IMP.2p

`{\dots} they were sitting and their husbands came, having made the planting holes. They came and told them, ``Dish out the food.'' '

A summary tail-head linkage with a generic verb, a common feature in many \textstyleAcronymallcaps{TNG} languages, is used very little in Mauwake. 

\ea%x1520
\label{ex:x1520}
\gll Or-omi  \textstyleEmphasizedVernacularWords{ma-em-ik-e-mik},  ``Eka  mamaiya  akena  \\
      \\
\glt
\z

descend-SS.SIM  say-SS.SIM-be-PA-1/3p  river  near  very

i  yoowa  me  aaw-i-yen.''  \textstyleEmphasizedVernacularWords{Naap}

1p.UNM  hot  not  get-Np-FU.1p  thus

\textstyleEmphasizedVernacularWords{on-am-ika-iwkin}  eka  owowa  kerer-ep  {\dots}

do-SS.SIM-be-2/3p.DS  river  village  appear-SS.SEQ

`They went down and were saying, ``Very near the river we won't get hot.'' They were doing like that and (then) the river reached the village and {\dots}'

\subsection{Subordinate clauses: embedding and hypotaxis}
\hypertarget{RefHeading23281935131865}{}
Subordinate clauses are a problematic area to define both cross-linguistically (Haiman \& Thompson 1984, Matthiessen and Thompson 1988:317) and even within one language (Giv\'on 1990:848). It seems that there is a continuum from fully independent to embedded clauses (Reesink 1987:207, Lehmann 1988:189). 

Rather than treating subordinate clauses as one group it is helpful to differentiate between embedding and hypotaxis. Embedded clauses have a function in the main clause: relative clauses as qualifiers within a \textstyleAcronymallcaps{NP,} complement clauses as objects or subjects, and adverbial clauses as adverbials. Hypotactic clauses are also dependent on the main clause, but they do not function as a constituent in it (Halliday 1985:219, Lehmann 1988). Even though subordination is ``a negative term which lumps together all deviations from some `main clause' norm'' \citep[510]{HaimanEtAl1984}%Thompson
, the term still has limited usefulness, as there are some rules that affect both embedded and hypotactic clauses.  

In Mauwake, subordinate clauses usually precede the main clauses, and they have a non-final intonation pattern. The initial position is related to the pragmatic function of topic that these clauses often have \citep[187]{Lehmann1988}; but when the subordinate clause is right-dislocated, it does not have a topic function.\footnote{For a discussion on the topic function of subordinate clauses see e.g. Reesink (1983b, 1987), Matthiessen and \citet{Thompson1988}, \citet{Lehmann1988} and Thompson, Longacre and \citet{Hwang2007}.}  The semantic function varies according to the type of subordinate clause. 

The embedded clauses in Mauwake are nominalized clauses: relative clause nominalization (\textstyleAcronymallcaps{RC}) (\sectref{sec:8.3.1}) is always done with the demonstrative \textstyleStyleVernacularWordsItalic{nain}\textstyleStyleVernacularWordsItalic{} `that'added to a finite clause, whereas complement clauses (\textstyleAcronymallcaps{CC}) (\sectref{sec:8.3.2}) can use either one of the two nominalization strategies (\sectref{sec:5.7}). The locative and temporal adverbial clauses (\sectref{sec:8.3.3}), like the relative clauses, are type 2 nominalized clauses. All of these clauses bear out Reesink's (1983b:236) claim that ``subordinate clauses, especially in sentence-initial position, are natural vehicles for the speaker's presuppositions''.\footnote{``Presuppositions'' here refer to pragmatic, not logical-semantic presuppositions.}  Reesink (ibid. 230) also suggests that the origin of the relative clause is in a paratactic construction. At least in Mauwake this seems to be true not only of the relative clause but of the complement clause (\sectref{sec:8.3.2}) as well. 

The hypotactic conditional and concessive clauses are dependent on their main clause, but not embedded in it. 

\subsubsection{Relative clauses} 
\hypertarget{RefHeading23301935131865}{}
I define a restrictive relative clause (\textstyleAcronymallcaps{RC}),\footnote{This definition only applies to restrictive relative clauses; non-restrictive RCs (\sectref{sec:8.3.1.4}) are not real RCs although they are structurally similar to the real RCs.} following \citet[206]{Andrews2007b}, as a ``subordinate clause which delimits the reference of a \textstyleAcronymallcaps{NP} by specifying the role of the referent of that \textstyleAcronymallcaps{NP} in the situation described by the \textstyleAcronymallcaps{RC}''. 

The relative clause is a statement about some noun phrase\textstyleAcronymallcaps{} in the main clause. That \textstyleAcronymallcaps{NP} is here called the antecedent \textstyleAcronymallcaps{NP} (\textstyleAcronymallcaps{AntNP}),\footnote{This is often called Head NP, but because it is not grammatically a ``head'' of anything, I prefer to call it antecedent NP. The name ``antecedent'' is also somewhat of a misnomer, as in Mauwake it does not \textit{precede} the RelNP.}  since it is the unit that the the coreferential \textstyleAcronymallcaps{NP} in the relative clause derives its meaning from \citep[20]{Crystal1997}. The coreferential \textstyleAcronymallcaps{NP} in the \textstyleAcronymallcaps{RC} is called the relative \textstyleAcronymallcaps{NP} (\textstyleAcronymallcaps{RelNP}).\footnote{\citet[142]{Keenan1985} calls it a domain noun.}

Often the referent of the \textstyleAcronymallcaps{AntNP} is assumed to be known to the hearer but not necessarily easily accessible, so the \textstyleAcronymallcaps{RC} gives background information to help the hearer identify the referent. 

The relative marker is the distal demonstrative \textstyleStyleVernacularWordsItalic{nain} `that' (\sectref{sec:3.6.2}) occurring clause-finally in the relative clause (\stepcounter{nx}{\thenx}). It has a slightly rising non-final intonation indicating that the sentence continues; right-dislocated \textstyleAcronymallcaps{RC}s have sentence-final falling intonation. Givenness is an essential part of the meaning of the demonstrative, which is also used in \textstyleAcronymallcaps{NP}s (\stepcounter{nx}{\thenx}). The demonstrative in effect makes the \textstyleAcronymallcaps{RC} into a noun phrase. The similarity of the two structures can be seen in the examples below.

\ea%x1527
\label{ex:x1527}
\gll [Takira  gelemuta  nain]\textsubscript{NP}  uruf-a-m. \\
      \\
\glt
\z

boy  small  that1  see-PA-1s

`I saw that/the small boy.'

\ea%x1528
\label{ex:x1528}
\gll [Takira  me  arim-o-k  nain]\textsubscript{RC}  uruf-a-m. \\
      \\
\glt
\z

boy  not  grow-PA-3s  that1  see-PA-1s

`I saw the boy that has not grown.'

\paragraph[The type and position of the relative clause]{The type and position of the relative clause}
\hypertarget{RefHeading23321935131865}{}
In typological terms the relative clauses in Mauwake are mostly replacive, also called internal. A normal finite clause is made into a noun phrase by the addition of the demonstrative \textstyleStyleVernacularWordsItalic{nain}, and the \textstyleAcronymallcaps{RelNP} inside the \textstyleAcronymallcaps{RC} replaces the \textstyleAcronymallcaps{AntNP}. Pre-nominal \textstyleAcronymallcaps{RCs,} where the\textstyleAcronymallcaps{ RC} precedes the \textstyleAcronymallcaps{AntNP,} are cross-linguistically more typical of \textstyleAcronymallcaps{OV} languages than replacive ones \citep[144]{Keenan1985}, but the latter are also common in Papuan languages (Reesink 1983b:229 and 1987:219, Roberts 1987:49, Farr 1999:281, Whitehead 2004:193). Often both pre-nominal and replacive \textstyleAcronymallcaps{RC}s are possible, with one or the other being the dominant type.

\ea%x1529
\label{ex:x1529}
\gll [Ni  \textstyleEmphasizedVernacularWords{nomona}  unuf-a-man  nain],  aria  iimeka  kuisow  na-e-man. \\
      \\
\glt
\z

2p.UNM  stone  call-PA-2p  that1  alright  ten  one  say-PA-2p

`The money that you mentioned, alright you said ten (kina).'

\ea%x1530
\label{ex:x1530}
\gll Ne  [eka  opora  \textstyleEmphasizedVernacularWords{biiris}  marew  nain]  wiena  \\
      \\
\glt
\z

ADD  river  mouth  bridge  no(ne)  that1  3p.GEN

on-am-ik-e-mik.

do-SS.SIM-be-1/3p

`And they themselves kept making bridges to river channels that didn't have them.' 

\ea%x1545
\label{ex:x1545}
\gll [\textstyleEmphasizedVernacularWords{Mua}  kuum-e-mik  nain]  me  wia  kuuf-a-mik. \\
      \\
\glt
\z

man  burn-PA-1/3p  that1  not  3p.ACC  see-PA-1/3p

`We/They did not see the men that burned it.'

\ea%x1546
\label{ex:x1546}
\gll Ne  [\textstyleEmphasizedVernacularWords{akia}  ik-e-k  nain]  me  en-e-k. \\
      \\
\glt
\z

ADD  banana  roast-PA-3s  that1  not  eat-PA-3s

`And/but he did not eat the banana(s) that he roasted.'

It is possible to retain the \textstyleAcronymallcaps{AntNP}, in which case the relative clause is not replacive but pre-nominal. In Mauwake this is not common; it is used when the noun phrase that is relativized is given extra emphasis. 

\ea%x1532
\label{ex:x1532}
\gll [\textstyleEmphasizedVernacularWords{Fofa}  ikiw-e-mik  nain],  \textstyleEmphasizedVernacularWords{fofa}  nain  yo  me  paayar-e-m. \\
      \\
\glt
\z

day  go-PA-1/3p  that1  day  that1  1s.UNM  not  know-PA-1s

`The day that they went, I do not know the day/date.'

Even though the \textstyleAcronymallcaps{RC} is usually embedded in the main clause, it can be right-dislocated. In that case the main clause contains the \textstyleAcronymallcaps{AntNP}, and the \textstyleAcronymallcaps{RelNP} is deleted from the \textstyleAcronymallcaps{RC}. This way the first one of the coreferential \textstyleAcronymallcaps{NP}s  is retained for easier processing. Reasons for right-dislocation are: 1) a long \textstyleAcronymallcaps{RC}, which would be hard to process sentence-medially, 2) focusing on the \textstyleAcronymallcaps{RC}, or 3) an afterthought: something that the speaker still wants to add.

\ea%x1533
\label{ex:x1533}
\gll \textstyleEmphasizedVernacularWords{Wi  teeria  papako}  o  asip-a-mik,  [ona  eka  \\
      \\
\glt
\z

3p.UNM  group  other  3s.UNM  help-PA-1/3p  3s.GEN  water

sesenar-ep  wienak-e-k  nain].

buy-SS.SEQ  feed.them-PA-3s  that1

`Another group helped him, (those) that he had bought and given beer to.'

\ea%x1534
\label{ex:x1534}
\gll \textstyleEmphasizedVernacularWords{I  mua}  yiam  ikur,  [fikera  ikiw-e-mik  nain]. \\
      \\
\glt
\z

1p.UNM  man  1p.REFL  five  kunai.grass  go-PA-1/3p  that1

`There were five of us men that went to the kunai grass (=pig-hunting).'

In a very rare case the \textstyleAcronymallcaps{AntNP} is deleted and the \textstyleAcronymallcaps{RelNP} is retained in the right-dislocated \textstyleAcronymallcaps{RC}. What makes it possible in example (\stepcounter{nx}{\thenx}) may be that the verb in the main clause can only have some food (or medicine/poison) as its object, so the object, although usually present, may also be left out.

\ea%x1535
\label{ex:x1535}
\gll Wi  mua  ...  ekap-iwkin  wienak-e-mik, \\
      \\
\glt
\z

3p.UNM  man  ...  come-2/3p.DS  feed.them-PA-1/3p

[\textstyleEmphasizedVernacularWords{maa}  nop-a-mik  nain].

food  search-PA-1/3p  that1

`The men {\dots} came, and we gave it to them to eat, (that is,) the food that we had searched for.'

Comrie presents another typology based on how the role of the \textstyleAcronymallcaps{RelNP} is presented in the \textstyleAcronymallcaps{RC}. Basically Mauwake is of the ``gap type'', which ``does not provide any overt indication of the role of the head within the relative clause'' (1981:144-146). Noun phrases get very little case marking for their clausal role, and this is reflected in the \textstyleAcronymallcaps{RC} too. This results in ambiguous relative clauses when both a third person subject \textstyleAcronymallcaps{NP} and a third person object \textstyleAcronymallcaps{NP} are present in the \textstyleAcronymallcaps{RC} and the context does not make the meaning clear enough:

\ea%x1548
\label{ex:x1548}
\gll [Siowa  kasi  keraw-a-k  nain]  um-o-k. \\
      \\
\glt
\z

dog  cat  bite-PA-3s  that1  die-PA-3s

`The dog that bit the cat died.' Or: `The dog that the cat bit died.'

The ambiguity can be avoided by adding the contrastive focus marker to the subject when the object is fronted to the theme position.\footnote{For some reason this is done in relative clauses mainly with human subjects, although the contrastive focus marker can be added to non-human subjects as well.}  Although this is not case marking, it can function as such, because the subject is the best candidate for contrastive focus marking (\sectref{sec:3.12.7.2}).

\ea%x1549
\label{ex:x1549}
\gll [Mua  ona  emeria=ke  aruf-a-k  nain]  uruf-a-m. \\
      \\
\glt
\z

man  3s.GEN  woman=CF  hit-PA-3s  that1  see-PA-1s

`I saw the man whose wife hit him.'

Comrie's (1981:140) ``non-reduction type'' is exhibited in Mauwake by those few cases where the \textstyleAcronymallcaps{RelNP} keeps its oblique case marking. With overt case marking on the \textstyleAcronymallcaps{RelNP}, the \textstyleAcronymallcaps{AntNP} has to be retained:

\ea%x1544
\label{ex:x1544}
\gll [\textbf{Burir=iw}  nomokowa  war-em  nain,]  burir  nain  duduw-ar-e-k. \\
      \\
\glt
\z

axe=INST  tree  cut-PA-1s  that1  axe  that  blunt-INCH-PA-3s

`The axe with which I cut trees became blunt.'

But when the case marking does not appear in the \textstyleAcronymallcaps{RC,} the \textstyleAcronymallcaps{AntNP} is not present in the main clause either, and the \textstyleAcronymallcaps{RC} is a gapping-type relative clause:

\ea%x1541
\label{ex:x1541}
\gll [\textbf{Burir}  nomokowa  war-e-m  nain=ke]  duduw-ar-e-k. \\
      \\
\glt
\z

axe  tree  cut-PA-1s  that1=CF  blunt-INCH-PA-3s

`The axe with which I cut trees became blunt.'

\paragraph[The structure of the relative clause ]{The structure of the relative clause} 
\hypertarget{RefHeading23341935131865}{}
In Mauwake the most typical relative clause is syntactically like a finite main clause, plus the distal-1 deictic \textstyleStyleVernacularWordsItalic{nain} functioning as a clause final relative marker. It was mentioned in \sectref{sec:5.7.2} that this is one strategy for nominalizing clauses in Mauwake. The demonstrative as a possible origin of a relative marker is well attested cross-linguistically (e.g. Dixon 2010b:342). 

The verb of the relative clause is a fully inflected finite verb. But when a non-verbal clause is a relative clause, it has no verb and is structurally like other non-verbal clauses. 

\ea%x1943
\label{ex:x1943}
\gll Ne  [eka  opora  biiris  marew  nain]  wiena \\
      \\
\glt
\z

ADD  river  mouth  bridge  no(ne)  that1  3p.GEN  

on-am-ik-e-mik.

do-SS.SIM-PA-1/3p

`And they themselves kept making bridges to rivers that didn't have them.'

The \textstyleAcronymallcaps{RelNP} tends to be initial in the \textstyleAcronymallcaps{RC} regardless of its syntactic function. This is because it often has the pragmatic function of theme, which takes the clause-initial position. The initial position is easy to have also because a typical clause in Mauwake has so few noun phrases: in many \textstyleAcronymallcaps{RC}s the \textstyleAcronymallcaps{RelNP} is the only noun phrase. 

\ea%x1552
\label{ex:x1552}
\gll [Moma  p-or-o-mik  nain]  wiar  sesenar-e-mik. \\
      \\
\glt
\z

taro  Bpx-descend-PA-1/3p  that1  3.DAT  buy-PA-1/3p

`They\textsubscript{i} bought from them\textsubscript{j} the taro that they\textsubscript{j} brought down.'

When a personal pronoun functions as a subject and the \textstyleAcronymallcaps{RelNP} in some other syntactic role, the pronoun tends to keep its initial position, thus maintaining the basic constituent order. The personal pronouns are high in the topicality hierarchy (Giv\'on 1976:166), so it is natural that they tend to keep the clause-initial and also sentence-initial position.  Since the object \textstyleStyleVernacularWordsItalic{sirirowa} `pain' in (\stepcounter{nx}{\thenx}) is not fronted, a temporal adverbial also keeps a place it would have in a neutral main clause. 

The tense in the \textstyleAcronymallcaps{RC} can be past (\stepcounter{nx}{\thenx}) or present (\stepcounter{nx}{\thenx}), but not future. For future meaning, the present tense form has to be used (\stepcounter{nx}{\thenx}). 

\ea%x1531
\label{ex:x1531}
\gll [Yo  ikoka  sirir-owa  aaw-i-yem  nain],  nis  pun  eliw  \\
      \\
\glt
\z

1s.UNM  later  hurt-NMZ  get-Np-PR.1s  that1  2p.FC  also  well

aaw-owen=i?

get-FU.2p=QM

`Is it all right for you also to get the pain that I will later get?'

As was mentioned above, the \textstyleAcronymallcaps{AntNP} only rarely shows overtly. But a \textstyleAcronymallcaps{RelNP} can also be deleted if it is generic (\stepcounter{nx}{\thenx}), or recoverable from situational (\stepcounter{nx}{\thenx}) or textual context (\stepcounter{nx}{\thenx}). In the example (\stepcounter{nx}{\thenx}) the deleted \textstyleAcronymallcaps{RelNP} can either be generic `what/whatever' or it may be \textstyleStyleVernacularWordsItalic{opora} `talk'; in (\stepcounter{nx}{\thenx}) the speaker is describing the process of making a fishtrap, which has already been mentioned in previous sentences.

\ea%x1561
\label{ex:x1561}
\gll [Iinan  aasa=pa  or-omi  kiikir  furew-a-mik  nain]  dabela.  \\
      \\
\glt
\z

sky  canoe=LOC  descend-SS.SIM  first  sense-PA-1/3p  that1  cold

`What we first sensed/felt when we descended from the plane was the cold.'

\ea%x1555
\label{ex:x1555}
\gll [Kululu  ma-e-k  nain]  kirip-i-yem. \\
      \\
\glt
\z

Kululu  say-PA-3s  that1  turn/reply-Np-PR.1s

`I reply to what Kululu said.'

\ea%x1559
\label{ex:x1559}
\gll Aria  [malol=pa  ifemak-i-mik  nain]  aana  \\
      \\
\glt
\z

alright  deep.sea=LOC  press-Np-PR.1/3p  that1  cane

puuk-i-mik.

cut-Np-PR.1/3p

`Alright for those that we lower to the deep sea we cut cane.'

In (\stepcounter{nx}{\thenx}) there is no other indication of the  \textstyleAcronymallcaps{RelNP} than the person suffix of the verb. The group of women referred to were mentioned as a noun phrase only near the beginning of the story, whereas the example is from close to the end:

\ea%x1556
\label{ex:x1556}
\gll Domora=pa  or-omi  nan  ik-e-mik,  [afa  \\
      \\
\glt
\z

Domora=LOC  descend-SS.SIM  there  be-PA-1/3p  flying.fox

ar-e-mik  nain].

become-PA-1/3p  that1

`They went down from Domora and were there, those (women) who became flying foxes.'

In the following two examples the \textstyleAcronymallcaps{RCs} are identical, but they have a different \textstyleAcronymallcaps{RelNP}. The \textstyleAcronymallcaps{RelNP} of (\stepcounter{nx}{\thenx}) is \textstyleStyleVernacularWordsItalic{mukuruna} `noise', but the \textstyleAcronymallcaps{RelNP} of (\stepcounter{nx}{\thenx}), \textstyleStyleVernacularWordsItalic{wi} `they', only shows in the verbal suffix. The obligatory accusative pronoun in the main clause provides a key for the interpretation of (\stepcounter{nx}{\thenx}).

\ea%x1557
\label{ex:x1557}
\gll [Mukuruna  wua-i-mik  nain]  ikiw-ep  miim-eka. \\
      \\
\glt
\z

noise  put-Np-PR.1/3p  that1  go-SS.SEQ  hear-IMP.2p

`Go and listen to the noise that they are making.'

\ea%x1558
\label{ex:x1558}
\gll [Mukuruna  wua-i-mik  nain]  ikiw-ep  wia  miim-eka. \\
      \\
\glt
\z

noise  put-Np-PR.1/3p  that1  go-SS.SEQ  3p.ACC  hear-IMP.2p

`Go and listen to those who are making the noise.'

The antecedent in most relative clauses has a specific reference. When the reference is generic, in Mauwake a very generic noun is chosen as the head of the relativized \textstyleAcronymallcaps{NP} and that is modified by a question word. So called free \citep[213]{Andrews2007b} or condensed \citep[359]{Dixon2010b} relative clauses,\footnote{They are also called headless RCs; see \citet[317,360]{Dixon2010b} for a criticism of this.} which usually replace the whole \textstyleAcronymallcaps{NP} with a generic or interrogative pronoun, are not used in Mauwake. 

\ea%x1562
\label{ex:x1562}
\gll [Maa  mauwa  maak-i-n  nain]  me  nefa  miim-i-non. \\
      \\
\glt
\z

thing  what  tell-Np-PR.2s  that1  not  2s.ACC  hear-Np-FU.3s

`Whatever you tell him, he will not hear.'

\ea%x1563
\label{ex:x1563}
\gll [Mua  naareke  kema  enek-ar-i-ya  nain]  eka  dabela  \\
      \\
\glt
\z

man  who.CF  liver  tooth-INCH-Np-PR.3s  that1  water  cold

enim-i-nok.

eat-Np-IMP.3s

`Whoever is thirsty must drink (cold) water.'

When the antecedent is generic and human, there are two more possibilities for the RelNP: it may be \textstyleStyleVernacularWordsItalic{mua} `man, person' or the third person singular pronoun, plus the specifier \textstyleStyleVernacularWordsItalic{ena} (\sectref{sec:3.12.7.1}). 

\ea%x1564
\label{ex:x1564}
\gll [Mua  ena  kema  enek-ar-i-ya  nain]  ... \\
      \\
\glt
\z

man  SPEC  liver  tooth-INCH-Np-PR.3s  that1

`Whoever is thirsty{\dots}'

\ea%x1565
\label{ex:x1565}
\gll [O  ena  kema  enek-ar-i-ya  nain]  ... \\
      \\
\glt
\z

3s.UNM  SPEC  liver  tooth-INCH-Np-PR.3s  that1

`Whoever is thirsty{\dots}' 

Non-verbal descriptive clauses can be made into relative clauses, but it is only in the negative that they are recognizable as such. In the affirmative they are exactly like noun phrases with a demonstrative, and because the meanings are so similar, it can be questioned whether there is such a thing as an affirmative non-verbal descriptive \textstyleAcronymallcaps{RC} at all in Mauwake.  

\ea%x1550
\label{ex:x1550}
\gll [Mua  eliwa  nain]  kookal-i-yem. \\
      \\
\glt
\z

man  good  that1  like-Np-PR.1s

`I like the good man.' Or `I like the man that is good.'

In the negative these clauses are different from the noun phrases because the negation is placed before the non-verbal predicate.

\ea%x1551
\label{ex:x1551}
\gll [Koora  \textstyleEmphasizedVernacularWords{me}  maneka  nain]  uruf-a-m. \\
      \\
\glt
\z

house  not  big  that1  see-PA-1s

`I saw the house that is not big.'

\paragraph[Relativizable noun phrase positions]{Relativizable noun phrase positions}
\hypertarget{RefHeading23361935131865}{}
Several \textstyleAcronymallcaps{NP} functions can be relativized, and Mauwake conforms to Keenan and Comrie's (1977) Noun phrase accessibility hierarchy:\footnote{As Mauwake adjectives do not have comparative forms there can be no relativization for an object of comparison, which in Keenan and Comrie's hierarchy is the hardest to relativize.}  the higher up a \textstyleAcronymallcaps{NP} is in the hierarchy, the easier it is to relativise. Noun phrases with the following functions can be relativized: subject, object, recipient, beneficiary, instrument, comitative, object of genitive, temporal and locative. 

Subject (\stepcounter{nx}{\thenx}) and object (\stepcounter{nx}{\thenx}) are the most frequent functions of the \textstyleAcronymallcaps{RelNP}.

\ea%x1537
\label{ex:x1537}
\gll [\textstyleEmphasizedVernacularWords{Mesa  asia}  fiker(a)  gone=pa  ika-i-ya  nain]  \\
      \\
\glt
\z

wingbean  wild  kunai.grass  middle=LOC  be-Np-PR.3s  that1

aaw-em-ik-e-m.

get-SS.SIM-be-PA-1s

`I kept picking wild wingbeans that are/grow in the middle of the kunai grass.'

\ea%x1538
\label{ex:x1538}
\gll Muuka,  [yo  \textstyleEmphasizedVernacularWords{opora}  nefa  maak-i-yem  nain]  miim-ap \\
      \\
\glt
\z

son  1s.UNM  talk  2s.ACC  tell-Np-PR.1s  that1  hear-SS.SEQ

ook-e.

follow-IMP.2s

`Son, listen to and follow the talk that I am telling you.'

Recipient (\stepcounter{nx}{\thenx}) and beneficiary (\stepcounter{nx}{\thenx}) are possible to relativise, but in natural texts especially beneficiary is very infrequent.

\ea%x1539
\label{ex:x1539}
\gll [\textstyleEmphasizedVernacularWords{Takira}  iwoka  iw-e-m  nain]  yena  aamun=ke. \\
      \\
\glt
\z

boy  yam  give.him-PA-1s  that1  1s.GEN  1s/p.younger.sibling=CF

`The boy that I gave yam to is my younger brother.'

\ea%x1547
\label{ex:x1547}
\gll Ne  [\textstyleEmphasizedVernacularWords{wi}  \textstyleEmphasizedVernacularWords{emeria  papako}  iiriw  sawur  wia \\
      \\
\glt
\z

ADD  3p.UNM  woman  some  earlier  bad.spirit  3p.ACC

iirar-om-a-k  nain]  {\dots}

remove-BEN-BNFY2.PA-3s  that1

`And some women, from (lit: for) whom he had removed bad spirits, {\dots}'

When an instrument is relativized, the \textstyleAcronymallcaps{RelNP} either takes the instrumental case marking (\stepcounter{nx}{\thenx}) or has no case marking (\stepcounter{nx}{\thenx}): 

\ea%x1553
\label{ex:x1553}
\gll Aria  [\textstyleEmphasizedVernacularWords{maa  unowa}  wakesim-e-mik  nain]  sererk-a-mik. \\
      \\
\glt
\z

alright  thing  many  cover-PA-1/3p  that1  distribute-PA-1/3p

`Alright they distributed the many things with which they had covered her (body).'

A comitative \textstyleAcronymallcaps{NP} (4.1.3) containing a comitative postposition may be relativized (\stepcounter{nx}{\thenx}), but one formed with a comitative clitic may not. 

\ea%x1542
\label{ex:x1542}
\gll [\textbf{Mua  nain}  \textbf{ikos}  ikiw-e-mik  nain]  napum-ar-e-k. \\
      \\
\glt
\z

man  that1  with  go-PA-1/3p  that1  sick-INCH-PA-3s

`That man with whom I went became sick.'

The object of genitive, or object of \textstyleEmphasizedWords{\textsc{possessive}} as it should be called when describing Mauwake grammar, only uses the dative pronoun (\sectref{sec:3.5.5}) when relativized (\stepcounter{nx}{\thenx}), not the unmarked (\sectref{sec:3.5.2.1}) or genitive (\sectref{sec:3.5.4}) pronoun.

\ea%x1543
\label{ex:x1543}
\gll [\textbf{Mua}  emeria  \textbf{wiar}  um-o-k  nain=ke]  baurar-ep \\
      \\
\glt
\z

man  woman  3.DAT  die-PA-3s  that1=CF  flee-SS.SEQ

owowa  oko  ikiw-o-k.

village  other  go-PA-3s

`The man whose wife died went away\footnote{Moving to another village after some misfortune is quite common, and the verb `flee' is used in this context but here it does not have a strongly negative connotation; this is reflected in the free translation.} to another village.'

Temporal and locative \textstyleAcronymallcaps{RC}s are structurally identical to the other \textstyleAcronymallcaps{RC}s when the relativized temporal or locative \textstyleAcronymallcaps{NP} does not have an adverbial function in the main clause. 

\ea%x1554
\label{ex:x1554}
\gll [Fofa  ikiw-e-mik  nain]  me  paayar-e-m. \\
      \\
\glt
\z

day  go-PA-1/3p  that1  not  understand-PA-1s

`I don't know the day that they went.'

\ea%x1560
\label{ex:x1560}
\gll [Koora  maneka  wiena  opora  siisim-i-mik  nain]  uruf-a-mik. \\
      \\
\glt
\z

house  big  3p.GEN  talk  write-Np-1/3p  that1  see-PA-1/3p

`We saw the big house where they write their talk (=printshop).'

When the relativized temporal \textstyleAcronymallcaps{NP} is a temporal in the main clause as well, the relative marker can optionally be replaced by the locative deictic \textstyleStyleVernacularWordsItalic{nan} or \textstyleStyleVernacularWordsItalic{neeke} `there'.

\ea%x1625
\label{ex:x1625}
\gll [Aite  uroma  yaki-e-k  fofa  nain/nan/neeke] \\
      \\
\glt
\z

1s/p.mother  stomach  wash-PA-3s  day  that1/there/there.CF  

auwa  Madang  ikiw-o-k.

1s/p.father  Madang  go-PA-3s

`The day that mother gave birth, father went to Madang.'

When the relativized locative \textstyleAcronymallcaps{NP} is also a constituent in the main clause, the relative marker has to be replaced by \textstyleStyleVernacularWordsItalic{nan} or \textstyleStyleVernacularWordsItalic{neeke} (\stepcounter{nx}{\thenx}).

\ea%x1622
\label{ex:x1622}
\gll Or-op  [i  koora  ik-e-mik  neeke]  ekap-o-k. \\
      \\
\glt
\z

descend-SS.SEQ  1p.UNM  house  be-PA-1p  there.CF  come-PA-3s

`It descended and came to the house/building where we were.'

Temporal adverbial clauses, which are structurally close to relative clauses, are discussed below in \sectref{sec:8.3.3.1}, and locative adverbial clauses in \sectref{sec:8.3.3.2}.

\paragraph[Non-restrictive relative clauses]{Non-restrictive relative clauses}
\hypertarget{RefHeading23381935131865}{}
Non-restrictive, or appositional, relative clauses are structurally exactly like restrictive relative clauses, but their function is different. They do not delimit the reference of the antecedent \textstyleAcronymallcaps{NP}, instead, they give new information about it. Functionally they are like a coordinate clause added to the main clause.

Because of the structural and even intonational similarity it is sometimes difficult to tell if a particular \textstyleAcronymallcaps{RC} is restrictive or non-restrictive. When the \textstyleAcronymallcaps{AntNP} is a proper noun or when it includes a first or second person singular pronoun the \textstyleAcronymallcaps{RC} is usually non-restrictive:

\ea%x1567
\label{ex:x1567}
\gll Bang=ke  ekap-o-k,  [Ponkila  aaw-o-k  nain]. \\
      \\
\glt
\z

Bang=CF  come-PA-3s  Ponkila  get-PA-3s  that1

`Bang came, (he) who married Ponkila.'

\ea%x1568
\label{ex:x1568}
\gll Yo  [nena  owowa  moma  marew  nain]  miatin-i-yem. \\
      \\
\glt
\z

1s.UNM  2s.GEN  village  taro  no(ne)  that1  dislike-Np-PR.1s

`I don't like your village, which doesn't have taro.'

The proximate demonstrative \textstyleStyleVernacularWordsItalic{fain} `this' can also function as a relative marker in the non-restrictive \textstyleAcronymallcaps{RC}s but not in restrictive ones:

\ea%x1536
\label{ex:x1536}
\gll Nomokowa  unowa  fan-e-mik,  [Simbine \\
      \\
\glt
\z

2s/p.brother  many  here-PA-1/3p  Simbine

ekap-omak-e-mik  fain].

come-DISTR/PL-PA-1/3p  this

`Your many (clan) brothers are here, these Simbine people who came.'

When the \textstyleAcronymallcaps{AntNP} is a pronoun other than first or second singular, the \textstyleAcronymallcaps{RC} may be either restrictive or non-restrictive.

\ea%x1570
\label{ex:x1570}
\gll I  mua  yiam  ikur,  [fikera  ikiw-e-mik  nain]. \\
      \\
\glt
\z

1p.UNM  man  1p.REFL  five  kunai.grass  go-PA-1/3p  that1

`There were five of us men who went to the kunai grass (= pig-hunting).' Or: `We were five men, who went pig-hunting.'

\subsubsection{Complement clauses and other complementation strategies}
\hypertarget{RefHeading23401935131865}{}
The prototypical function of a complement clause is as a subject or object in a main clause. In Mauwake a complement clause proper functions as an object of a complement-taking verb (\textstyleAcronymallcaps{CTV}), and occasionally as a subject in a non-verbal clause. Structurally it is a type 2 nominalized clause: a finite clause that has the distal-1 demonstrative \textstyleStyleVernacularWordsItalic{nain} `that' occurring as a nominaliser clause-finally (\sectref{sec:5.7.2}). The complement clause precedes the complement-taking verb. The complement clause differs from the relative clause in that none of the \textstyleAcronymallcaps{NPs} inside it is an \textstyleAcronymallcaps{AntNP} or a \textstyleAcronymallcaps{RelNP}. 

The division of complements into different types, ``Fact, Activity and Potential'', that \citet[371]{Dixon2010b} provides, is crucial for the use of the different complementation strategies in Mauwake.  A complement clause is normally used when a \textstyleAcronymallcaps{CTV} needs a fact-type object complement.

Besides the regular complement clause described above, Mauwake has other complementation strategies. The indirect speech clauses are ordinary sentences embedded in the utterance clause (\sectref{sec:8.3.2.1.2}). Medial clauses are used as the main complementation strategy for activity-type complements with perception verbs (\sectref{sec:8.3.2.2}). Clauses with a nominalized verb are used for potential-type complements with various \textstyleAcronymallcaps{CTV}s. The regular complement clause and the clause with a nominalized verb may occur as a subject of a clause (\sectref{sec:8.3.2.5}). 

Since one \textstyleAcronymallcaps{CTV} can take more than one complementation strategy, the main grouping below is done according to the \textstyleAcronymallcaps{CTV}s.

\paragraph[Complements of utterance verbs]{Complements of utterance verbs}
\hypertarget{RefHeading23421935131865}{}
Some utterance verbs (\sectref{sec:3.8.4.4.6}) are also used for thinking, so speech and thought are discussed as one group. 

The status of direct quote clauses (\sectref{sec:8.3.2.1.1}) as complement clauses is questionable, but they are discussed here because of their co-occurrence with the utterance verbs and their similarity with the indirect quotes (\sectref{sec:8.3.2.1.2}), which are complement clauses.  

The most important of the utterance verbs is \textstyleStyleVernacularWordsItalic{na}- `say, think'. It is used as the utterance verb for indirect quote complements, which in turn have grammaticalized, together with the same subject sequential form of the verb, as desiderative (\sectref{sec:8.3.2.1.3}) and purpose clauses (\sectref{sec:8.3.2.1.4}) and the conative construction (\sectref{sec:8.3.2.1.5}) 

\subparagraph[Direct speech]{Direct speech}
\hypertarget{RefHeading23441935131865}{}
It seems to be a universal feature of direct quote clauses that they behave independently of their matrix clauses. If they are considered complement clauses of utterance verbs, their independence sets them apart from all the other complement clauses \citep[303]{Munro1982}. \citet[398]{Dixon2010} maintains that direct speech quotes are not any kind of complementation.

A direct quote may be a whole discourse on its own, not just a clause within a sentence.

It is rather typical in Papuan languages to have a strict quote formula both before and after a quotation, or at least before it (Franklin 1971:120, Davies 1981:1, Roberts 1987:12, Farr 1999, Hepner 2002:128). It is also common that either there is no separate structure for indirect speech \citep[2]{Davies1981} or that direct and indirect speech are so similar that they are often hard to distinguish from each other \citep[14]{Roberts1987}. 

In the use of quote formulas Mauwake is much freer than Papuan languages in general. A direct quotation in Mauwake is often preceded or followed by one of the utterance verbs. The verbs \textstyleStyleVernacularWordsItalic{na}- `say/think' and \textstyleStyleVernacularWordsItalic{naak}- `say/tell' are almost exclusively used after quotes. Enclosing a quote between two utterance verbs is not frequent:

\ea%x1571
\label{ex:x1571}
\gll Ne  ona  mua  pun  \textbf{ma-e-k},  ``Eka  maneka  nain=ke \\
      \\
\glt
\z

ADD  3s.GEN  man  also  say-PA-3s  river  big  that1=CF

iwa-mi  ifakim-o-k,''  \textstyleEmphasizedVernacularWords{na-e-k}.

come-SS.SIM  kill-PA-3s  say-PA-3s

`Her husband also said, ``The big river came and killed her,'' he said.'

Most commonly a speech verb only preceeds the quote:

\ea%x1578
\label{ex:x1578}
\gll Panewowa=ke  \textbf{ma-e-k},  ``Yo  nia  maak-emkun  opaimika \\
      \\
\glt
\z

old=CF  say-PA-3s  1s.UNM  2p.ACC  tell-2/3p.DS  talk

efa  fien-a-man.''

1s.ACC  disobey-PA-2p

`The old (woman) said, ``When I told you, you disobeyed me.'' '

\ea%x1579
\label{ex:x1579}
\gll Iiw-ep  wiipa  muuka  nain  wia  \textbf{maak-e-k}, \\
      \\
\glt
\z

dish.out-SS.SEQ  daughter  son  that1  3p.ACC  tell-PA-3s

``Auwa  maa  p-ikiw-om-aka.

1s/p.father  food  Bpx-go-BEN-BNFY2.IMP.2p

`She dished out (the food) and told the children, ``Take the food to father.'' '

An single utterance verb or a whole quote-closing clause may follow the quote. A quote-closing clause has to be used when the quotation consists of several sentences.

\ea%x1580
\label{ex:x1580}
\gll ``No  bom  fain=iw  mera  kuum-e,''  \textbf{naak-e-mik}. \\
      \\
\glt
\z

2s.UNM  bomb  this=INST  fish  burn-IMP.2s  tell-PA-1/3p

` ``Blast fish with this bomb,'' they told him.'

\ea%x1583
\label{ex:x1583}
\gll ``I  muuka  marew  a,  wiipa  marew  a,''  \textbf{naap  wia} \\
      \\
\glt
\z

1p.UNM  son  no(ne)  ah  daughter  no(ne)  ah  thus  3p.ACC

\textstyleEmphasizedVernacularWords{maak-e-k}.

tell-PA-3s

` ``We have no son, we have no daughter,'' he told them like that.'

In narratives where there are several exchanges between the participants, it is possible to leave out the utterance verb and even the \textstyleAcronymallcaps{NP} referring to the speaker of the utterance, if that is clear enough from the context. A good speaker creates variety to the text by utilizing all these different possibilities. 

\ea%x1581
\label{ex:x1581}
\gll Ne  onak=ke  \textstyleEmphasizedVernacularWords{{\O}},  ``A,  ifera  feeke  un-eka.'' \\
      \\
\glt
\z

ADD  3s/p.mother  {\O}  Ah,  salt.water  here.CF  fetch(water)-IMP.2p

Ne  wi  maak-e-mik,  ``Wia,  i  oro-or-op

ADD  3p.UNM  tell-PA-1/3p  No  1p.UNM  RDP-descend-SS.SEQ

un-i-yan.''  ``A,  neeke-r=iw  un-eka.''  ``Weetak,

fetch-Np-FU.1p  Ah  there.CF-{\O}=LIM  fetch-IMP.2p  no  

i  oro-ora-i-yan.''

1p.UNM  RDP-descend-Np-FU.1p

`And their mother said, ``Ah, fetch the sea water (from) here.'' But they told her, ``No, we'll go down (to the deep sea) and fetch it.'' ``Ah, fetch it right there.'' ``No, we'll go down a long way.'' '

\ea%x1582
\label{ex:x1582}
\gll ``Mauwa  ar-e-n,  amia=iya  nenar-e-mik=i?''  Sarak=ke  \textstyleEmphasizedVernacularWords{{\O}}. \\
      \\
\glt
\z

what  become-PA-2s  bow=COM  shoot.you-PA-1/3p=QM  Sarak=CF  {\O}

` ``What happened to you, did they shoot you with a gun?'' Sarak (asked).'

\subparagraph[Indirect speech]{Indirect speech}
\hypertarget{RefHeading23461935131865}{}
Indirect speech quotes, which report speech or thought, are objects of speech verbs. 

Most indirect quotes in Mauwake are syntactically identical to direct quotes. There is an intonational difference: the indirect quote is part of the intonation contour of the main clause, rather than having a contour of its own as a direct quote has. The quote is almost always followed by the utterance verb \textstyleStyleVernacularWordsItalic{na}- `say, think'; but it is also possible for the verb \textstyleStyleVernacularWordsItalic{ma}- `say' to precede it, in which case both the utterance verb and the quote have their own intonation contour (\stepcounter{nx}{\thenx}).\footnote{In Amele the absence of the speech verb before the quote is the main criterion for indirect speech \citep[14]{Roberts1987}. In Mauwake it cannot be used as a criterion, as the occurrence of speech verbs with direct quotes varies so much.} An indirect quote is never enclosed between two utterance verbs. 

\ea%x1585
\label{ex:x1585}
\gll Aria,  Kalina,  [Amerika  ekap-e-mik]  na-i-mik. \\
      \\
\glt
\z

alright  Kalina  America  come-PA-1/3p  say-Np-PR.1/3p

`Alright, Kalina, they say that the Americans have come.'

\ea%x1587
\label{ex:x1587}
\gll Ma-e-m,  [nena  owowa=pa  ik-o-n]. \\
      \\
\glt
\z

say-PA-1s  2s.GEN  village=LOC  be-PA-2s

`I said (to her\textsubscript{i}) that you\textsubscript{j} are in your own village.'

As direct quotes behave independently of their matrix clauses, they often have a separate deictic centre. But indirect quotes vary in this respect. Deictic elements, which get part or all of their interpretation from the situational context, are often the same in indirect quotes as they would be in direct quotes: 

\ea%x1584
\label{ex:x1584}
\gll Aite=ke  [manina  yook-e]  na-eya  o  \\
      \\
\glt
\z

1s/p.mother=CF  garden  follow.me-IMP.2s  say-2/3s.DS  3s.UNM

ook-e.

follow.her-IMP.2s

`When mother tells you to follow her to the garden, follow her.'

\ea%x1280
\label{ex:x1280}
\gll Ni  Krais  [yena  teeria  efar  ik-eka]  na-ep  \\
      \\
\glt
\z

2p.UNM  Christ  1s.GEN  family  1s.DAT  be-IMP.2p  say-SS.SEQ

nia  far-eya  ona  teeria  wiar  ik-e-man.

2p.ACC  call-2/3s.DS  3s.GEN  family  3.DAT  be-PA-2p

`Christ called you to be his family and (now) you are his family.'

But the deictic centre may also shift partly or completely towards that of the matrix clause. When this happens, the pronouns are the easiest to change, next the adverbs. In (\stepcounter{nx}{\thenx}) a second person pronoun has replaced the proper name or third person pronoun that would have been used in a direct quote.

\ea%x1586
\label{ex:x1586}
\gll Sarak  oo,  Amerika  ekap-ep  Ulingan  nan  ik-e-mik,  \\
      \\
\glt
\z

Sarak  INTJ  America  come-SS.SEQ  Ulingan  there  be-PA-1/3p

[\textstyleEmphasizedVernacularWords{nefa}  ikum-i-mik]  na-i-mik  oo.

2s.ACC  wonder.about-Np-PR.1/3p  say-Np-PR.1/3p  INTJ

`Sarak! The Americans have come and are in Ulingan and they say that they are wondering where you are!'

When reported by the addressee of the example clause (\stepcounter{nx}{\thenx}), only the pronoun in the reported clause (\stepcounter{nx}{\thenx}) is different:

\ea%x1281
\label{ex:x1281}
\gll No  owowa  ikiw-ep  buk  nain  sesek-om-e. \\
      \\
\glt
\z

2s.UNM  village  go-SS.SEQ  book  that1  send-BEN-BNFY1.IMP.2s

`When you go to the village, send the book to me.'

\ea%x1282
\label{ex:x1282}
\gll [\textstyleEmphasizedVernacularWords{Yo } owowa \textstyleEmphasizedVernacularWords{} ikiw-ep  buk  nain  sesek-om-e]  \\
      \\
\glt
\z

1s.UNM  village  go-SS.SEQ  book  that1  send-BEN-BNFY1.IMP.2s

efa  na-e-k.

1s.ACC  say-PA-3s

`He told me to send that book to him (lit: me) when I would go to the village.'

The verbs are most resistant to deictic shift. In (\stepcounter{nx}{\thenx}), even though the verb root changes, it still retains the tense and person marking of the direct quote (\stepcounter{nx}{\thenx}). Both the temporal adverb and the pronoun are shifted to reflect the deictic centre of the matrix clause.

\ea%x1264
\label{ex:x1264}
\gll Uurika  nefar  ikiw-i-nen. \\
      \\
\glt
\z

tomorrow  2s.DAT  go-Np-FU.1s

`Tomorrow I'll come (lit: go) to you.'

\ea%x1265
\label{ex:x1265}
\gll [\textstyleEmphasizedVernacularWords{Ikoka}  \textstyleEmphasizedVernacularWords{efar  ekap-}i-nen]  na-e-k  na  weetak. \\
      \\
\glt
\z

Later(today)  1s.DAT  come-Np-FU.1s  say-PA-3s  but  no

`He said that he would come to me today, but he hasn't.'

Below in (\stepcounter{nx}{\thenx}) also the person suffix is changed to reflect the situation of the new speech act participants. 

\ea%x1267
\label{ex:x1267}
\gll Ona  owowa=pa  ik-ua. \\
      \\
\glt
\z

3s.GEN  village=LOC  be-PA.3s

`She is in her own village.'

\ea%x1266
\label{ex:x1266}
\gll Ma-e-m,  [\textstyleEmphasizedVernacularWords{nena } owowa=pa  \textstyleEmphasizedVernacularWords{ik-o-n}]. \\
      \\
\glt
\z

say-PA-1s  2s.GEN  village=LOC  be-PA-2s

`I said (to her) that you are in your own village.'

The deictic shift would need more study to ascertain if there are specific rules governing this variation in indirect quotes.

When the verb \textstyleStyleVernacularWordsItalic{na}- `say, think' indicates thinking, the complement clause is usually an indirect quote rather than a direct one. 

\ea%x1588
\label{ex:x1588}
\gll [Muuka  ifera  me  enim-i-non]  na-ep  me  uruf-a-m. \\
      \\
\glt
\z

boy  salt.water  not  drink-Np-FU.3s  think-SS.SEQ  not  look-PA-1s

`Thinking that the boy wouldn't drown I didn't watch him.'

\ea%x1589
\label{ex:x1589}
\gll Mua  pepena=ke  [menat=ke  ek-i-ya]  na-ep  \\
      \\
\glt
\z

man  inexperienced=CF  tide=CF  go-Np-PR.3s  think-SS.SEQ

menat  ora-i-nan.

tide  descend-Np-FU.2s

`An inexperienced man will think that the tide is going down and will go to fish at low tide.'

Indirect non-polar questions are similar to the corresponding direct questions apart from possible adjustments to deictic elements.

\ea%x1592
\label{ex:x1592}
\gll [Wi  uf-ow(a)  epa  kaaneke  ik-ua]  na-e-k.  \\
      \\
\glt
\z

3p.UNM  dance-NMZ  place  where.CF  be-PA.3s  say-PA-3s

`He asked where their dancing place was.'

\ea%x1590
\label{ex:x1590}
\gll [O  ikoka  sesa  kamenap  aaw-i-non]  na-e-k. \\
      \\
\glt
\z

3s.UNM  later  price  what.like  get-Np-FU.3s  say-PA-3s

`He asked what kind of wages he would get later.'

Polar questions, when indirect, have to be alternative questions. The verb \textstyleStyleVernacularWordsItalic{naep} may be deleted, when the indirect question is sentence-final (\stepcounter{nx}{\thenx}). 

\ea%x1591
\label{ex:x1591}
\gll [Beel-al-i-non=i  kamenion]  na-ep  \\
      \\
\glt
\z

rotten-INCH-Np-FU.3s=QM  or.what  think-SS.SEQ

uruf-am-ik-ua.

see-SS.SIM-be-PA.3s

`He was watching (thinking) whether it would rot or what would happen.'

\ea%x1593
\label{ex:x1593}
\gll Wi  iwera  iinan=pa  ik-ok  iwer(a)  popoka  \\
      \\
\glt
\z

3p.UNM  coconut  top=LOC  be-SS  coconut  unripe

wafur-am-ik-e-mik,  [eka  saanar-e-k=i  eewuar] {\O}.

throw-SS.SIM-be-PA-1/3p  water  dry-PA-3s=QM  not.yet

`They were at the top of the coconut palm and threw unripe coconuts (thinking) whether the water had dried or not yet.'

\subparagraph[Desiderative clauses]{Desiderative clauses}
\hypertarget{RefHeading23481935131865}{}
It is very common in Papuan languages that an indirect quote construction with the intended action verb in future tense, imperative or irrealis form expresses a want/wish\footnote{In Mauwake the verb \textit{kookal}- `like, love, desire', is mostly used with a NP object, but it can take a clausal complement as well. It does not indicate intention or purpose. The complement is either type of the nominalized clauses (5.7).}, desire or intention to do something. (Reesink 1987:254-9, Foley 1986:157, Hardin 2003:112, Hepner 2002:76-7).

In Mauwake the future, imperative, counterfactual and nominalized forms of the main verb are possible in the complement. In the desiderative clauses the verb \textstyleStyleVernacularWordsItalic{na-} `say/think' is always in the medial same-subject sequential form \textstyleStyleVernacularWordsItalic{naep}; in purpose clauses other forms are possible as well. Historically, there probably always used to be a clause with a finite verb following the clause expressing intention or desire (\stepcounter{nx}{\thenx}); synchronically the finite clause is often missing (\stepcounter{nx}{\thenx}), especially when the verb would be the same as in the complement.

\ea%x367
\label{ex:x367}
\gll Niena  [maa  enim-u]  na-ep  iiw-eka. \\
      \\
\glt
\z

2p.GEN  food  eat-IMP.1d  say-SS.SEQ  dish.out-IMP.2p

`If you want/intend to eat food, dish it out (yourselves).'

\ea%x368
\label{ex:x368}
\gll Yo  [opora  gelemuta=ko  ma-i-nen]  na-ep. \\
      \\
\glt
\z

1s.UNM  talk  little=NF  say-Np-FU.1s  say-SS.SEQ

`I want to tell a little story.' Or: `I'm going to tell a little story.'

The main verb in the complement is either marked for first person\footnote{Purpose clauses may use other person forms as well (\sectref{sec:8.3.2.1.4}).} or is nominalized. Mauwake uses the future (\stepcounter{nx}{\thenx}) or imperative form (\stepcounter{nx}{\thenx}) of the main verb for intention or a clear/certain wish, and the counterfactual form for a wish that has less potential to be realized. The latter is also the most polite form to use, if the wish indicates a request (\stepcounter{nx}{\thenx}). 

\ea%x369
\label{ex:x369}
\gll [Haussik  p-ek-u]  na-ep  miiw-aasa  nop-a-mik. \\
      \\
\glt
\z

aidpost  BPx-go-IMP.1d  say-SS.SEQ  land-canoe  search-PA-1/3p

`We/they wanted to take her to the aidpost and looked for a vehicle.'

\ea%x370
\label{ex:x370}
\gll [Yo=ko  wia  uruf-ek-a-m]  na-ep.  \\
      \\
\glt
\z

1s.UNM=NF  3p.ACC  see-CNTF-PA-1s  say-SS.SEQ

`I would like to see them.'

The nominalized form is mostly used in complement clauses that can also be interpreted as purpose clauses. In ``pure'' desiderative clauses it is practical to use the nominalized form especially if the first person marking in the verb might make it harder to process the meaning:

\ea%x1610
\label{ex:x1610}
\gll Ne  [o  uruf-owa]  ne  [maa  en-owa  asip-owa] \\
      \\
\glt
\z

ADD  3s.UNM  see-NMZ  ADD  food  eat-NMZ  help-NMZ

na-ep=na  eliw  asip-uk.

say-SS.SEQ=TP  well  help-IMP.3p

`And if they want to see him and help him with food, let them help him.'

\subparagraph[Purpose clauses]{Purpose clauses}
\hypertarget{RefHeading23501935131865}{}
Purpose is both conceptually close and structurally similar to desiderative, and in Mauwake many of the desiderative clauses can be interpreted as purpose clauses. This is particularly so when the main verb is in the nominalized form. But a truly desiderative clause even with an action nominal is never right-dislocated, whereas a purpose clause (\stepcounter{nx}{\thenx}) often is. The nominalized form in the main verb is common:

\ea%x371
\label{ex:x371}
\gll [Weniwa=pa  en-owa]  na-ep  uuw-i-mik. \\
      \\
\glt
\z

famine=LOC  eat-NMZ  say-SS.SEQ  work-Np-PR.1/3p

 `We work in order to (be able to) eat during the time of hunger.'

\ea%x345
\label{ex:x345}
\gll [Wi  Amerika  wiam=iya  irak-owa]  na-ep  ikiw-e-mik. \\
      \\
\glt
\z

3p.UNM  America  3p=COM  fight-NMZ  say/think-SS.SEQ  go-PA-1/3p

`They went to fight with the Americans.'

\ea%x372
\label{ex:x372}
\gll Ona  siowa  ikos  manina  ikiw-e-mik,  [pika  on-owa] \\
      \\
\glt
\z

3s.GEN  dog  with  garden  go-PA-1/3p  fence  make-NMZ

na-ep.

say-SS.SEQ

`He went to the garden with his dog, in order to make a fence.'

Future and imperative forms are also used in the purpose clause. When the subject in the purpose clause is the same as the subject of the utterance verb and the main clause, the first person future form is used for singular and first person dual imperative for plural.

\ea%x1614
\label{ex:x1614}
\gll [Nain  nefa  maak-i-nen]  na-ep  yo  ep-a-m. \\
      \\
\glt
\z

that1  2s.ACC  tell-Np-FU.1s  say-SS.SEQ  1s.UNM  come-PA-1s

`I came to tell you that.'

\ea%x1616
\label{ex:x1616}
\gll No  [owora  sesenar-i-nen]  na-ep  Kainantu  fofa \\
      \\
\glt
\z

2s.UNM  betelnut  buy-Np-FU.1s  say-SS.SEQ  Kainantu  market

ikiw-ep  neeke  aaw-i-nan.

go-SS.SEQ  there.CF  get-Np-FU.2s

`To buy betelnut you will (need to) go to Kainantu marker and get it \textstyleEmphasizedWords{\textsc{there}}.'

\ea%x1620
\label{ex:x1620}
\gll Ne  [haussik  p-ek-u]  na-ep  miiw-aasa  nop-a-mik. \\
      \\
\glt
\z

ADD  aidpost  Bpx-go-IMP.1d  say-SS.SEQ  land-canoe  search-PA-1/3p

`And they searched for a truck (in order) to take him to the aidpost.'

When the subject of the verb in the main clause differs from that of the purpose clause, the verb inside the purpose clause has to be in the imperative. The whole purpose clause is structurally like a direct quote of the ``inner speech'' verb \textstyleStyleVernacularWordsItalic{naep}, so there is no deictic shift of the kind that may take place in indirect quotes.  

\ea%x1062
\label{ex:x1062}
\gll [Me  yiar-uk]  na-ep  koka=pa  ik-e-mik. \\
      \\
\glt
\z

not  shoot.us-IMP.3p  say-SS.SEQ  jungle=LOC  be-PA-1/3s

`We stayed in the jungle so that they would not shoot us.'

\ea%x346
\label{ex:x346}
\gll [Auwa=ke  o=ko  amukar-inok]  na-ep  maa  naap \\
      \\
\glt
\z

1s/p.father=CF  3s.UNM=NF  scold-IMP.3s  say-SS.SEQ  thing  thus

sirar-em-ik-e-mik.

make-SS.SIM-be-PA-1/3

`They kept doing things like that so that father would scold \textstyleEmphasizedWords{him} (and not them).'

\ea%x1615
\label{ex:x1615}
\gll Nain  [ni  amis-ar-eka]  na-ep  feenap \\
      \\
\glt
\z

that1  2p.UNM  knowledge-INCH-IMP.2p  say-SS.SEQ  like.this

on-i-yem. 

do-Np-PR.1s

`But I am doing this so that you would know.'

\ea%x1617
\label{ex:x1617}
\gll [Efa  asip-e]  na-ep  ekap-e-m. \\
      \\
\glt
\z

1s.ACC  help-IMP.2s  say-SS.SEQ  come-PA-1s

`I came so that you would help me.'

\ea%x1618
\label{ex:x1618}
\gll [Feenap  nokar-eka]  na-ep  yia  sesek-a-k. \\
      \\
\glt
\z

like.this  ask-IMP.2p  say-SS.SEQ  1p.ACC  send-PA-3s

`He sent us to ask (you) like this.'

\ea%x1619
\label{ex:x1619}
\gll [Yo  efa  miim-eka]  na-ep  wapena  wu-ami \\
      \\
\glt
\z

1s.UNM  1s.ACC  hear-IMP.2p  say-SS.SEQ  hand  put-SS.SIM

ma-e-k, ...

say-PA-3s

`He raised his hand for them to listen to him and said, {\dots}'

\ea%x1627
\label{ex:x1627}
\gll Ne  wi  popor-ar-urum-ep  ik-ok  ifana  muutiw \\
      \\
\glt
\z

ADD  3p.UNM  silent-INCH-DISTR/A-SS.SEQ  be-SS  ear  only

wu-am-ika-i-kuan,  [mua  unuma  wia  miim-u]

put-SS.SIM-be-Np-FU.3p  man  name  3p.ACC  hear-IMP.1d

na-ep.

say-SS.SEQ

`And they all will be quiet and listen carefully in order to hear the men's names.'

There is no raising of negative from the subordinate to the main clause. 

\ea%x1623
\label{ex:x1623}
\gll [Yo  me  pina=pa  nia  wu-ek-a-m]  na-ep \\
      \\
\glt
\z

1s.UNM  not  guilt=LOC  2p.ACC  put-CNTF-PA-1s  say-SS.SEQ

ma-i-yem.

say-Np-PR.1s

`I am not saying (this) to put guilt on you.' (=I am saying this, but not in order to put guilt on you.)

A purpose clause does not always have the auxiliary \textstyleStyleVernacularWordsItalic{naep}. A clause with just a nominalized verb is used especially with the directional verbs: 

\ea%x1659
\label{ex:x1659}
\gll [Yo  yena  emeria  aaw-owa]  urup-e-m. \\
      \\
\glt
\z

1s.UNM  1s.GEN  woman  take-NMZ  ascend-PA-1s

`I came up to take my wife.'

\ea%x1658
\label{ex:x1658}
\gll Bogia  ikiw-e-mik,  [opaimika  aakun-owa]. \\
      \\
\glt
\z

Bogia  go-PA-1/3p  talk  talk-NMZ

`We went to Bogia to talk.'

A clause with a nominalized verb plus a clause-final distal-1 demonstrative \textstyleStyleVernacularWordsItalic{nain} 'that' is also possible, but less common. I have not observed a functional difference between the different purpose structures.

\ea%x1633
\label{ex:x1633}
\gll Tunde=pa  [maa  muutitik  uruf-owa  nain]  soomar-e-mik. \\
      \\
\glt
\z

Tuesday=LOC  thing  all.kinds  see-NMZ  that1  walk-PA-1/3p

`On Tuesday we walked to see all kinds of things.'

\ea%x1634
\label{ex:x1634}
\gll Ifemak-ep  nomona  iinan=pa  wua-i-nan,  [ikoka  ifera  me  \\
      \\
\glt
\z

press-SS.SEQ  stone  on.top=LOC  put-Np-FU.2s  later  sea  not

p-ikiw-owa  nain].

Bpx-go-NMZ  that1

`You press it down and put stones on top ( or: put it on top of stones/corals) so that the sea would not later take it away.'

\subparagraph[Conative clauses: `try' ]{Conative clauses: `try'} 
\hypertarget{RefHeading23521935131865}{}
Instead of using a verbal construction with the verb `see' for conative modality -- expressing the attempt to do something -- which \citet[152]{Foley1986} claims as almost universal for Papuan languages, Mauwake makes use of a structure where the desiderative is followed by the verb \textstyleStyleVernacularWordsItalic{on-} `do' as the verb in its reference clause. Usan uses an identical construction for the same purpose \citep[258]{Reesink1987}. 

\ea%x373
\label{ex:x373}
\gll [Mukuna  umuk-u  na-ep  on-a-mik]=na  me  pepek. \\
      \\
\glt
\z

fire  extinguish-IMP.1d  say-SS.SEQ  do-PA-1/3p=TP  not  enough

`We tried to extinguish the fire but were not able to.'

When this structure is used, it is implied that somehow or other the effort fails:

\ea%x374
\label{ex:x374}
\gll [Emeria  aruf-i-nen  na-ep  on-am-ik-eya]  op-a-mik. \\
      \\
\glt
\z

woman  hit-Np-FU.1s  say-SS.SEQ  do-SS.SIM-be-2/3s.DS  hold-PA-1/3p

`When he was trying to hit the woman they grabbed him.'

\ea%x1606
\label{ex:x1606}
\gll [Wia  uruf-ek-a-m  na-ep  on-a-k  on-a-k]  weetak, \\
      \\
\glt
\z

3p.ACC  see-CNTF-PA-1s  say-SS.SEQ  do-PA-3s  do-PA-3s  no

o  me  wia  uruf-a-k.

3s.UNM  not  3p.ACC  see-PA-3s

`He tried and tried to see them but no, he didn't see them.'

The conative structure is not used when the effort is successful (\stepcounter{nx}{\thenx}), and also when the `trying' is not so much an effort to do something as experimenting (\stepcounter{nx}{\thenx}).  In these cases the verb \textstyleStyleVernacularWordsItalic{akim-} `try' is used. It requires a nominalized verb in the complement clause. 

\ea%x375
\label{ex:x375}
\gll [Aasa  keraw-owa]  akim-ap  akim-ap \\
      \\
\glt
\z

canoe  carve-NMZ  try-SS.SEQ  try-SS.SEQ

amis-ar-i-nan.

knowledge-INCH-Np-FU.2s

`After trying and trying to carve a canoe you will know (how to do it).'

\ea%x376
\label{ex:x376}
\gll [Weria  op-ap  wiinar-owa  \\
      \\
\glt
\z

planting.stick  hold-SS.SEQ  make.planting.holes-NMZ

nain]  akim-am-ik-e.

that1  try-SS.SIM-be-IMP.2s

`Keep trying/learning to make planting holes with the planting stick.'

\subparagraph[Complements of other utterance verbs ]{Complements of other utterance verbs} 
\hypertarget{RefHeading23541935131865}{}
The verb \textstyleStyleVernacularWordsItalic{ma}- `say, talk' can take a regular complement clause, which is of the fact type \citep[389]{Dixon2010b}. This clause functions as an object of the verb in the same way as a \textstyleAcronymallcaps{NP} with the head noun \textstyleStyleVernacularWordsItalic{opora} (or \textstyleStyleVernacularWordsItalic{opaimika}) `talk/story' in (\stepcounter{nx}{\thenx}):

\ea%x1595
\label{ex:x1595}
\gll [Opora  gelemuta=ko]\textsubscript{NP}  ma-i-nen  na-ep. \\
      \\
\glt
\z

talk  little=NF  say-Np-FU.1s  say/think-SS.SEQ

`I want to tell a little story.'

The complement clause says something about the contents of the story and functions as a kind of title. This type of structure is quite common in Papuan languages\footnote{\citet[231]{Reesink1987} treats them under relative clauses and considers them equivalents of English cleft sentences.} and is used mainly in an opening or closing formula in narrative texts: 

\ea%x1596
\label{ex:x1596}
\gll Aria  yo  aakisa  [takira  en-owa  gelemuta  wia  \\
      \\
\glt
\z

alright  1s.UNM  now  child  eat-NMZ  little  3p.ACC

on-om-a-mik  nain]\textsubscript{CC}  ma-i-yem.

make-BEN-BNFY2.PA-1/3p  that1  say-Np-PR.1s

`Alright now I tell about our making a little feast for the children.'

The complement ``clause'' may actually be a whole sentence, since it is possible to have medial clauses preceding the finite clause of the complement:

\ea%x1597
\label{ex:x1597}
\gll [Tunde=pa  fikera  kuum-iwkin  ikiw-ep  waaya  \\
      \\
\glt
\z

Tuesday=LOC  kunai.grass  burn-2/3p.DS  go-SS.SEQ  pig

mik-a-m  nain]\textsubscript{CC}  ma-i-yem.

spear-PA-1s  that1  say-Np-PR.1s

`I tell about that when they burned kunai grass on Tuesday and I went and speared a pig.'

Often the sentence has both a \textstyleAcronymallcaps{NP} containing a word for `story' and the complement clause (\stepcounter{nx}{\thenx}). The relationship of these two \textstyleAcronymallcaps{NP}s is not really appositional, because the nominalized clause modifies the other \textstyleAcronymallcaps{NP}. But the nominalized clause is not a prototypical \textstyleAcronymallcaps{RC} either, in spite of identical structure, because \textstyleStyleVernacularWordsItalic{opora} is neither an \textstyleAcronymallcaps{AntNP} nor a \textstyleAcronymallcaps{RelNP} that would have a function in the \textstyleAcronymallcaps{RC.} I consider the nominalized clause a modifier of the other \textstyleAcronymallcaps{NP}, and the whole comparable to the \textstyleAcronymallcaps{NP} in (\stepcounter{nx}{\thenx}).\footnote{Comrie and \citet{Horie1995} present another alternative: treating complement clauses like this and relative clauses as a single construction, where the structure only indicates that the subordinate clause is connected to a NP, and the interpretation of their relationship is done pragmatically. This possibility would need more investigation in Mauwake.} 

\ea%x1594
\label{ex:x1594}
\gll Aria  yo  aakisa  [fikera  ikum  kuum-e-mik  nain]\textsubscript{CC} \\
      \\
\glt
\z

alright  1s.UNM  now  kunai.grass  illicitly  burn-PA-1/3p  that1

opora  gelemuta=ko  ma-i-yem.

story  little=NF  say-Np-PR.1s

`Alright now I tell a little story about the kunai grass that was burned by arson.'

\ea%x1598
\label{ex:x1598}
\gll manina  uuw-owa  opora \\
      \\
\glt
\z

garden  work-NMZ  talk

`garden work talk / talk (n.) about garden work'

Another complementation strategy for utterance verbs is a clause with a nominalized verb. It is used when the event expressed in the clause is regarded as potential, rather than an actual activity or a fact. The following example has two levels of complementation, as the verb in the nominalized complement also takes a nominalized complement:

\ea%x1599
\label{ex:x1599}
\gll I  [yiena  [miiw-aasa  muf-owa]  ikiw-owa]  \\
      \\
\glt
\z

1p.UNM  1p.GEN  land-canoe  pull-NMZ  go-NMZ  

na-em-ik-omkun  o  ar-e-k.

say-SS.SIM-be-2/3p.DS  3s.UNM  become-PA-3s

`While we were talking about our going to fetch a vehicle, she died (lit: became).'

The same strategy is used with the verb \textstyleStyleVernacularWordsItalic{maak}- `tell' when it is used in the sense of ordering someone to do something: 

\ea%x1630
\label{ex:x1630}
\gll Emar,  [no  muut  fain  uf-owa]  nefa  maak-e-m. \\
      \\
\glt
\z

friend  2s.UNM  only  this  dance-NMZ  2s.ACC  tell-PA-1s

`Friend, I told you to dance only this.'

\paragraph[Complements of perception verbs]{Complements of perception verbs}
\hypertarget{RefHeading23561935131865}{}
It was mentioned above (\sectref{sec:8.2.3.4}) that the chaining structure is used with perception verbs in Mauwake as the main complementation strategy for perception verbs, when the complement is an activity or event. These are not genuine complement clauses, as they are not embedded in the main clause, but they perform the same function as regular complement clauses do. 

\ea%x1512
\label{ex:x1512}
\gll [Mukuruna  wu-am-ika-iwkin]  i  miim-a-mik. \\
      \\
\glt
\z

noise  put-SS.SIM-be-2/3p.DS  1p.UNM  hear-PA-1/3p

`We heard you making (the) noise.'

\ea%x1600
\label{ex:x1600}
\gll [Urema  maneka  um-ep  ika-eya]  uruf-a-mik. \\
      \\
\glt
\z

bandicoot  big  die-SS.SEQ  be-2/3s.DS  see-PA-1/3p

`They saw the big bandicoot dead (=having died).'

A regular complement clause is only used with perception verbs about a past activity, when the complement clause reports a fact rather than an activity. 

\ea%x1628
\label{ex:x1628}
\gll Iikir-ami  [iwera  nain  emeria  ar-e-p  ik-ua \\
      \\
\glt
\z

get.up-SS.SIM  coconut  that1  woman  become-PA-3s  be-PA.3s

nain]\textsubscript{CC}  uruf-ap  {\dots}

that1  see-SS.SEQ

`He got up and saw that the coconut had become a woman, and {\dots}'

\ea%x1629
\label{ex:x1629}
\gll [Yeesus  owow  iinan  urup-o-k  nain]\textsubscript{CC}  uruf-ap \\
      \\
\glt
\z

Jesus  village  above  ascend-PA-3s  that1  see-SS.SEQ  

kemel-a-mik.

rejoice-PA-1/3p

`They saw that Jesus ascended into heaven, and rejoiced.'

When a perception verb takes an indirect question as a complement, it has to be a regular complement clause:

\ea%x1631
\label{ex:x1631}
\gll Ni  [kakala  sira  kamenap  eliw-ar-i-ya  nain]\textsubscript{CC} \\
      \\
\glt
\z

2p.UNM  flower  custom  what.like  good-INCH-Np-PR.3s  that1

uruf-eka.

see-IMP.2p

`See how the flowers grow.'

\paragraph[Complements of cognitive verbs]{Complements of cognitive verbs}
\hypertarget{RefHeading23581935131865}{}
The verbs for knowing, \textstyleStyleVernacularWordsItalic{amisar}- and \textstyleStyleVernacularWordsItalic{paayar}- together cover the cognitive area of knowing facts and skills, coming to realize, and understanding. When the complement clause indicates contents of factual knowledge it is usually a regular complement clause.

\ea%x1602
\label{ex:x1602}
\gll O  [kaanek  aaw-ep  p-ekap-om-a-mik \\
      \\
\glt
\z

3s.UNM  where.CF  get-SS.SEQ  Bpx-come-BEN-BNFY2.PA-1/3p

nain]  me  amis-ar-e-k.

that1  not  knowledge-INCH-PA-3s

`He didn't know where they got it from and brought to him.'

It seems that a clause with a nominalized verb is also used as a ``fact'' complement but only when it refers to pre-knowledge of an event. It could also be understood as a ``potential'' type complement, in which case it is natural that it uses this complementation strategy. This requires more investigation. 

\ea%x1605
\label{ex:x1605}
\gll [O  ikiw-owa  nain]  amis-ar-e-n=i? \\
      \\
\glt
\z

3s.UNM  go-NMZ  that1  knowledge-INCH-PA-2s=QM

`Did you know about his going?'

When the complement is about knowing a skill, the verb in the complement clause is in nominalized form, or a medial clause is used: 

\ea%x1603
\label{ex:x1603}
\gll [Nain  on-owa  (nain)]  me  amis-ar-e-m. \\
      \\
\glt
\z

that1  do-NMZ  that1  not  knowledge-INCH-PA-1s

`I don't know how to do that.'

\ea%x1849
\label{ex:x1849}
\gll [Sawiter  inera  on-ap]  amis-ar-e-k. \\
      \\
\glt
\z

Sawiter  basket  make-SS.SEQ  knowledge-INCH-PA-3s

`Sawiter knows how to make baskets.'

When the complement indicates lack of some experience, a construction with a medial clause is used. In this case the main clause is in the negative, and the scope of the negation has to extend to the medial clause:

\ea%x1604
\label{ex:x1604}
\gll [Owora  en-ep]  me  paayar-e-m. \\
      \\
\glt
\z

betelnut  eat-SS.SEQ  not  understand-PA-1s

`I'm not used to eating betelnut.' Or: `I don't know how to eat betelnut.'

\paragraph[Complement clauses as subjects]{Complement clauses as subjects}
\hypertarget{RefHeading23601935131865}{}
Both types of a nominalized clause (\sectref{sec:5.7.1}, 5.7.2) may be used as subjects in verbless clauses, even though this function for complement clauses is not common. A clause with a nominalized verb is used when the activity is potential (\stepcounter{nx}{\thenx}), (\stepcounter{nx}{\thenx}). 

\ea%x1636
\label{ex:x1636}
\gll [Maa  wiar  ikum  aaw-owa]  eliwa=ki? \\
      \\
\glt
\z

thing  3.DAT  illicitly  take-NMZ  good=CF.QM

`Is stealing from others good?'

\ea%x1637
\label{ex:x1637}
\gll [Maa  eneka  me  en-owa]  maa  marew. \\
      \\
\glt
\z

thing  tooth  not  eat-NMZ  thing  no(ne)

`Not eating meat is all right.'

A regular complement clause with a finite verb is used when the activity is considered a fact:

\ea%x1639
\label{ex:x1639}
\gll [Ni  unuma  niam  p-ir-i-man  nain]  eliw(a) \\
      \\
\glt
\z

2p.UNM  name  2p.REFL  Bpx-ascend-Np-PR.2p  that1  good

marew.

no(ne)

`That you praise yourselves (lit: lift up your own name) is not good.'

\subsubsection{Adverbial clauses}
\hypertarget{RefHeading23621935131865}{}
Adverbial clauses are a very small group of subordinate clauses. They are type two nominalized clauses, and they perform the same function in a clause as a temporal or locative adverbial phrase. 

\paragraph[Temporal clauses ]{Temporal clauses} 
\hypertarget{RefHeading23641935131865}{}
The presence of the distal-1 demonstrative \textstyleStyleVernacularWordsItalic{nain} `that' indicates the pragmatic difference between the temporal clauses and those medial clauses that may get a temporal interpretation: the temporal clauses are presented as given information, whereas the medial clauses usually present new information (\stepcounter{nx}{\thenx}), except when they occur in Tail-Head constructions.

\ea%x1540
\label{ex:x1540}
\gll Ni  [ifa  nia  keraw-i-ya  nain]  sira  kamenap \\
      \\
\glt
\z

2p.UNM  snake  2p.ACC  bite-Np-PR.3s  that1  custom  what.like

on-i-man?

do-Np-PR.2p

`When a snake bites you, what do you do?'

\ea%x1569
\label{ex:x1569}
\gll [Maa  fain  pakak  na-e-k  nain]  yo  soran-e-m. \\
      \\
\glt
\z

thing  this  bang  say-PA-3s  that1  1s.UNM  be.startled-PA-1s

`When this thing went ``bang!'' I got startled.'

\ea%x1624
\label{ex:x1624}
\gll [Yo  napum-ar-e-m  nain]  eneka  maay-ar-e-m. \\
      \\
\glt
\z

1s.UNM  sick-INCH-PA-1s  that1  tooth  long-INCH-PA-1s

`When I got sick, I became hungry for meat (lit: my teeth got long).'

\ea%x1632
\label{ex:x1632}
\gll Yo  napum-ar-ep  eneka  maay-ar-e-m. \\
      \\
\glt
\z

1s.UNM  sick-INCH-SS.SEQ  tooth  long-INCH-PA-1s

`I got sick and became hungry for meat.'

\paragraph[Locative clauses]{Locative clauses}
\hypertarget{RefHeading23661935131865}{}
Locative adverbial clauses use a clause-final deictic locative \textstyleStyleVernacularWordsItalic{nan} or \textstyleStyleVernacularWordsItalic{neeke} `there' instead of the demonstrative  \textstyleStyleVernacularWordsItalic{nain} `that'. Note that in (\stepcounter{nx}{\thenx}) the locative noun \textstyleStyleVernacularWordsItalic{manina} `garden' is not a \textstyleAcronymallcaps{RelNP}; if there were one, that would be \textstyleStyleVernacularWordsItalic{epa} `place' immediately preceding \textstyleStyleVernacularWordsItalic{nan} `there'. 

\ea%x1621
\label{ex:x1621}
\gll I  naap  ikiw-ep  [yiena  manina  on-a-mik  nan] \\
      \\
\glt
\z

1p.UNM  thus  go-SS.SEQ  1p.GEN  garden  make-PA-1p  there

ik-e-mik.

be-PA-1p

`We went there and stayed where we had made our gardens.'

\ea%x1626
\label{ex:x1626}
\gll [Luuwa  niir-i-mik  neeke]  soomar-e-mik. \\
      \\
\glt
\z

ball  play-Np-PR.1/3p  there.CF  walk-PA-1/3p

`We walked (to) where they play football.'

The following example is actually a locative relative clause, since it has a \textstyleAcronymallcaps{RelNP} \textstyleStyleVernacularWordsItalic{kame} `side' that has a function in both clauses:

\ea%x1638
\label{ex:x1638}
\gll [No  in-i-n  kame  nan]  urup-ep  tepak  iw-a-mik. \\
      \\
\glt
\z

2s.UNM  sleep-Np-PR.2s  side  there  ascend-SS.SEQ  inside  go-PA-1/3p

`They climbed up on the side where you sleep and went inside.'

\subsubsection{Adversative subordinate clause } 
\hypertarget{RefHeading23681935131865}{}
The coordinate adversative clauses were discussed in \sectref{sec:8.1.3}.

The topic marker -\textstyleStyleVernacularWordsItalic{na} (\sectref{sec:3.12.7.1}) marks an adversative clause when the main clause cancels an expectation, either expressed in the text or assumed to be in the hearer's mind. Because of this, this construction is used when some effort is frustrated (\stepcounter{nx}{\thenx}), or when there is a strong element of surprise (\stepcounter{nx}{\thenx}) in the main clause. 

\ea%x729
\label{ex:x729}
\gll Mukuna  nain  umuk-a-mik=\textstyleEmphasizedVernacularWords{na}  me  pepek. \\
      \\
\glt
\z

fire  that1  quench-PA-1/3p=TP  not  able

`They tried to quench the fire, but couldn't.'

\ea%x730
\label{ex:x730}
\gll Ekap-ep  uruf-a-k=\textstyleEmphasizedVernacularWords{na}  ifa  maneka=ke  siowa  \\
      \\
\glt
\z

come-SS.SEQ  see-PA-3s=TP  snake  big=CF  dog  

wasi-ep-pu-eya {\dots}

tie.around-SS.SEQ-CMPL-2/3s.DS

`He came and looked, but a snake had tied itself around the dog, and/but {\dots}'

In (\stepcounter{nx}{\thenx}), what the boys expect to see is a crocodile, but it turns out to be a turtle.

\ea%x1393
\label{ex:x1393}
\gll Takir(a)  oko=ke  pon  muneka  wu-ek-a-m  na-ep \\
      \\
\glt
\z

boy  other=CF  turtle  egg  put-CNTF-PA-1s  say-SS.SEQ

urup-em-ika-eya  uruf-ap  tuar=ke  na-ep

ascend-SS.SIM-be-2/3s.DS  see-SS.SEQ  crocodile  say-SS.SEQ

alu-emi  baurar-e-k.  Takir(a)  unowa  ekap-ep

shout-SS.SIM  flee-PA-3s  boy  many  come-SS.SEQ  

uruf-a-mik=\textstyleEmphasizedVernacularWords{na}  pon=ke,  ne  unow=iya  op-ap

see-PA-1/3p=TP  turtle=CF  ADD  many=COM  hold-SS.SEQ  

kirip-a-mik.

turn-PA-1/3p

`A boy saw a turtle coming up (to the beach) to lay eggs and thought it was a crocodile, and shouted and fled. Many boys came and saw/looked, but it was a turtle, and they all together grabbed and turned it.'

In (\stepcounter{nx}{\thenx}), a man talks to his son whom he wanted and expected to be a good person:

\ea%x1397
\label{ex:x1397}
\gll Aakisa  yo  nefa  uruf-i-yem=\textstyleEmphasizedVernacularWords{na}  no  mua  eliw \\
      \\
\glt
\z

now  1s.UNM  2s.ACC  look-Np-PR.1s=TP  2s.UNM  man  good

marew.

no(ne)

`I now look at you but you are not a good man.'

Because these clauses express a cancellation or frustration of an expectation, a negator commonly follows as the first element in the main clause, and often the negator is the only element left of the main clause, as in (\stepcounter{nx}{\thenx}).

\ea%x1398
\label{ex:x1398}
\gll Marasin  wu-om-a-mik=\textstyleEmphasizedVernacularWords{na}  weetak. \\
      \\
\glt
\z

medicine  put-BEN-BNFY2.PA-1/3p=TP  no

`They injected medicine in him, but no (it had no effect).'

\subsubsection{Conditional clauses} 
\hypertarget{RefHeading23701935131865}{}
\citet{Haiman1978} was the first one to clearly describe the close connection between conditionals and topics, and it has since then been attested in various languages (Thompson, Longacre and Hwang 2007:292). In many Papuan languages the connection is very evident (Reesink 1987:235-244, MacDonald 1990:304-308, Farr 1999:263). The protasis -- the subordinate clause expressing the condition -- provides the presupposition for the apodosis, the asserted main clause. Or, ``it constitutes the framework which has been selected for the following discourse'' \citep[585]{Haiman1978}.

Conditional clauses in Mauwake can be grouped into three main groups on semantic and structural grounds: imaginative, predictive and reality conditionals.\footnote{The terminology is from Thompson, Longacre and \citet[255]{Hwang2007}.} The imaginative and predictive conditionals together belong to the unreality conditionals. The reality conditionals only include habitual/generic conditionals, as there are no present or past conditionals.

The protasis is placed before the apodosis. Right-dislocation of the protasis is possible but rare. The verb forms in the protasis and the apodosis depend on the type of conditional.

The topic marker -\textstyleStyleVernacularWordsItalic{na} is used as the conditional marker in the unreality conditional clauses, where it is cliticized to the last element of the protasis clause, usually the verb. The reality conditional clauses do not have a conditional marker, so structurally the protasis and apodosis are ordinary juxtaposed clauses. 

The intonation in the protasis has a slight rise towards the end.  

In imaginative conditional clauses the verb in both the protasis and the apodosis is in the counterfactual mood, which is marked by the suffix -\textstyleStyleVernacularWordsItalic{ek}. The conditional/topic marker -\textstyleStyleVernacularWordsItalic{na} is always present. The same form is used for semantically counterfactual and hypothetical conditionals. The counterfactual interpretation (\stepcounter{nx}{\thenx}) is more common, but especially if there is a reference to present (\stepcounter{nx}{\thenx}) or future time (\stepcounter{nx}{\thenx}), it forces hypothetical interpretation.

\ea%x1645
\label{ex:x1645}
\gll [Yo  Sek  haussik  ikiw-\textstyleEmphasizedVernacularWords{ek}-a-m=\textstyleEmphasizedVernacularWords{na}]  miiw-aasa=pa \\
      \\
\glt
\z

1s.UNM  Sek  hospital  go-CNTF-PA-1s=TP  land-canoe=LOC

uroma  yaki-\textstyleEmphasizedVernacularWords{ek}-a-m.

stomach  wash-CNTF-PA-1s

`If I had gone to the Sek hospital, I would have given birth in the truck.'

\ea%x1646
\label{ex:x1646}
\gll [Yena  aamun  aakisa  uruf-\textstyleEmphasizedVernacularWords{ek}-a-m=\textstyleEmphasizedVernacularWords{na}]  \\
      \\
\glt
\z

1s.GEN  1s/p.younger.sibling  now  see-CNTF-PA-1s=TP

kemel-\textstyleEmphasizedVernacularWords{ek}-a-m.

be.happy-CNTF-PA-1s

`If I saw my younger brother now, I would be happy.'

\ea%x1647
\label{ex:x1647}
\gll [Morauta  iimar-ow(a)  mua  ik-\textstyleEmphasizedVernacularWords{ek}-a-k=\textstyleEmphasizedVernacularWords{na},]  uurika \\
      \\
\glt
\z

Morauta  stand.up-NMZ  man  be-CNTF-PA-3s=TP  tomorrow

ikiw-ep  maak-\textstyleEmphasizedVernacularWords{ek}-a-mik.

go-SS.SEQ  tell-CNTF-PA-1/3p

`If Morauta were the leader, we would go and talk to him tomorrow.'

Usually the context determines the interpretation, but without a clear context the sentence may be ambiguous:  

\ea%x1648
\label{ex:x1648}
\gll [Inasin  napuma  ik-\textstyleEmphasizedVernacularWords{ek}-a-k=\textstyleEmphasizedVernacularWords{na}]  sariar-\textstyleEmphasizedVernacularWords{ek}-a-k. \\
      \\
\glt
\z

spirit/white.man  sickness  be-CNTF-PA-3s=TP  recover-CNTF-PA-3s

`If it were the white man's sickness\footnote{This is contrasted with \textit{owow napuma} `village sickness', caused by sorcery.} he would recover.' Or: `If it had been the white man's sickness, he would have recovered.'

The predictive conditionals are the most frequently used and show the greatest variation morphologically. The apodosis, and consequently the whole sentence, may be either a statement with a future tense verb, or a command with an imperative verb. The verb in the protasis may be in either present or future indicative, in imperative, or in medial form. The conditional/topic marker at the end of the protasis is obligatory. 

When the predictive conditional is a statement, usually the verb in both the protasis and in the apodosis is in the future tense.

\ea%x1652
\label{ex:x1652}
\gll [No  oram  mokok=iw  \textstyleEmphasizedVernacularWords{ika-i-nan=na}]  ikoka  mua  lebuma \\
      \\
\glt
\z

2s.UNM  just  eye=INST  be-Np-FU.2s=TP  later  man  lazy

\textstyleEmphasizedVernacularWords{ika}\textstyleEmphasizedVernacularWords{-}\textstyleEmphasizedVernacularWords{i}\textstyleEmphasizedVernacularWords{-}\textstyleEmphasizedVernacularWords{nan}.

be-Np-FU.2s

`If you just watch with your eyes (without joining the work) you will be(come) a lazy man.'

The protasis may have a medial verb form if the condition is likely to be fulfilled (\stepcounter{nx}{\thenx}), or when the protasis consists of two or more clauses that are in a medial-final relationship (\stepcounter{nx}{\thenx}).

\ea%x1654
\label{ex:x1654}
\gll [Emeria  \textstyleEmphasizedVernacularWords{sesenar}\textstyleEmphasizedVernacularWords{-}\textstyleEmphasizedVernacularWords{ek}\textstyleEmphasizedVernacularWords{-}\textstyleEmphasizedVernacularWords{a}\textstyleEmphasizedVernacularWords{-m  na-ep=na}]  waaya  ten  erup \\
      \\
\glt
\z

woman  buy-CNTF-PA-1s  say/think-SS.SEQ=TP  pig  ten  two

naap  wienak-i-non.

thus  feed.him-Np-FU.3s

`If/when he wants to buy a wife, he will give him (=the bride's father) twenty or so pigs.'

\ea%x1653
\label{ex:x1653}
\gll [Yaapan  me  \textstyleEmphasizedVernacularWords{piipu-ap=na}  anane  epaskun  ika-i-nan=na] \\
      \\
\glt
\z

Japan  not  leave-SS.SEQ=TP  always  together  be-Np-FU.2s=TP

no  iiwawun  weeser-i-nan.

2s.UNM  altogether  finish-Np-FU.2s

`If you don't leave the Japanese but are always together, you will be finished altogether.'

The predictive conditionals allow right-dislocation of the protasis, but it is uncommon: 

\ea%x1662
\label{ex:x1662}
\gll Owora  fain  aite  panewowa  onak-e,  [ekap-ep  \\
      \\
\glt
\z

betelnut  this  1s/p.mother  old  feed-IMP.2s  come-SS.SEQ

\textstyleEmphasizedVernacularWords{kerer-eya=na}].

arrive-2/3s.DS=TP

`Give these betelnuts to old mother to eat, if she comes and arrives here.'

Also those instances where the conditional marker is attached to a predicate that is not originally a verb, the predicate needs to have medial verb marking (\sectref{sec:3.8.3.4.2}).

\ea%x1660
\label{ex:x1660}
\gll [\textstyleEmphasizedVernacularWords{Weetak-eya}\textstyleEmphasizedVernacularWords{=na}]  weetak. \\
      \\
\glt
\z

no-2/3s.DS=TP  no

`If not, then not.'

\ea%x1661
\label{ex:x1661}
\gll [Mauw-owa  \textstyleEmphasizedVernacularWords{manek-aya=na}]  yia  maak-i-non. \\
      \\
\glt
\z

work-NMZ  big-2/3s.DS=TP  1p.ACC  tell-Np-FU.3s

`If the work is big, he will tell us.'

When the apodosis is in the imperative, there is normally some expectation that the the condition is to be fulfilled. When the likelihood is high, the medial form is used in the protasis (\stepcounter{nx}{\thenx}),(\stepcounter{nx}{\thenx}). Present tense (\stepcounter{nx}{\thenx}) and imperative (\stepcounter{nx}{\thenx}) indicate less, and future tense (\stepcounter{nx}{\thenx}) the least likelihood for the condition to be fulfilled.

\ea%x1650
\label{ex:x1650}
\gll [Wia  \textstyleEmphasizedVernacularWords{uruf-ap=na}]  wia  maak-e. \\
      \\
\glt
\z

3p.ACC  see-SS.SEQ=TP  3p.ACC  tell-IMP.2s

`If/when you see them, tell them.'

\ea%x1649
\label{ex:x1649}
\gll [Maa  mauwa  nefa  \textstyleEmphasizedVernacularWords{maak-iwkin=na}]  opaimika  miim-e. \\
      \\
\glt
\z

thing  what  2s.ACC  tell-2/3p.DS=TP  talk  listen-IMP.2s

`Whatever they may tell you, listen to the talk.' (Lit: `If they tell you what(ever), listen to the talk.')

\ea%x1651
\label{ex:x1651}
\gll Koora  pun  naap:  [mua  oko  naareke  koora  \textstyleEmphasizedVernacularWords{kua-i-ya=na}] \\
      \\
\glt
\z

house  also  thus  man  other  who.CF  house  build-Np-PR.3s=TP

o  asip-e.

3p.UNM  help-IMP.2p

`A house is like that too: if/when any man builds a house, help him.'

\ea%x1656
\label{ex:x1656}
\gll [Ni  kirip-owa  \textstyleEmphasizedVernacularWords{ika-inok}=\textstyleEmphasizedVernacularWords{na}]  kirip-eka. \\
      \\
\glt
\z

2p.UNM  reply-NMZ  be-IMP.3s=TP  reply-IMP.2p

`If you have something to reply, then reply.'

\ea%x1657
\label{ex:x1657}
\gll [Wia  \textstyleEmphasizedVernacularWords{uruf-i-nan=na}]  wia  maak-e. \\
      \\
\glt
\z

3p.ACC  see-Np-FU.2s=TP  3p.ACC  tell-IMP.2s

`If you (happen to) see them, tell them.'

The reality conditional clauses are morpho-syntactically different from the other conditional clauses in that they are not marked with the topic marker. The protasis and apodosis are juxtaposed main clauses in future tense, but this construction is mainly used to encode habitual or generic conditions. The protasis can never be right-dislocated, since it does not have the topic marker. 

\ea%x1644
\label{ex:x1644}
\gll [No  inasin(a)  unuma  me  unuf-i-nan],  mua  oko=ke  waaya \\
      \\
\glt
\z

2s.UNM  spirit  name  not  call-Np-FU.2s  man  other=CF  pig

nain  mik-ap  nefar  aaw-i-non.

that1  spear-SS.SEQ  2s.DAT  take-Np-FU.3s

`If you don't call the spirit name, another man will spear the pig and take it from you.'

If there are two protasis clauses, they may be juxtaposed without a connective (\stepcounter{nx}{\thenx}) or joined with the pragmatic additive \textstyleStyleVernacularWordsItalic{ne} (\stepcounter{nx}{\thenx}).

\ea%x1635
\label{ex:x1635}
\gll [Nena  kuuf-i-nan,  parew-i-non],  eliw  perek-i-nan.  \\
      \\
\glt
\z

2s.GEN  see-Np-FU.2s  mature-Np-FU.3s  well  harvest-Np-FU.2s

`If you see it yourself and it is matured you may harvest it.'

\ea%x1643
\label{ex:x1643}
\gll [Yo  um-i-nen  ne  yena  emeria  mua  oko  \\
      \\
\glt
\z

1s.UNM  die-Np-FU.1s  ADD  1s.GEN  woman  man  other

aaw-i-non],  muuka  onaiya  me  ikiw-i-non.

take-Np-FU.3s  son  with  not  go-Np-FU.3s

`If I die and my wife takes another husband, she will not go (to him) with the son.'

When a sentence contains alternatives expressed by two sets of reality conditional constructions, these are joined by the pragmatic additive \textstyleStyleVernacularWordsItalic{ne}.

\ea%x1642
\label{ex:x1642}
\gll [Yo  auwa  miiwa=pa  mauw-i-nen],  irak-owa  marew,  ne \\
      \\
\glt
\z

1s  1s/p.father  land=LOC  work-Np-FU.1s  fight-NMZ  no(ne)  ADD

[yo  aite  miiwa=pa  mauw-i-nen],  irak-owa  ika-i-non.

1s  1s/p.moher  land=LOC  work-Np-FU.1s  fight-NMZ  be-Np-FU.3s

`If I work on my father's land there is no fighting (over land), but if I work on my mother's land there will be fighting.'

The same construction can encode a simple coordinate relationship, but it is less common. In spoken text a slightly falling intonation at the end of the first clause indicates a coordinate sentence.

\ea%x1850
\label{ex:x1850}
\gll Oko-ke  pusun-emi  feeke  \textstyleEmphasizedVernacularWords{ikiw}\textstyleEmphasizedVernacularWords{-}\textstyleEmphasizedVernacularWords{i}\textstyleEmphasizedVernacularWords{-}\textstyleEmphasizedVernacularWords{non},  a  mua   \\
      \\
\glt
\z

other=CF  run.loose-SS.SIM  here.CF  go-Np-FU.3s  ah  man 

oko-ke  \textstyleEmphasizedVernacularWords{mik}\textstyleEmphasizedVernacularWords{-}\textstyleEmphasizedVernacularWords{i}\textstyleEmphasizedVernacularWords{-}\textstyleEmphasizedVernacularWords{non}.

other=CF  spear-Np-FU.3s

`Another (pig) will run loose and run this way, ah, another man will spear it.'

\subsubsection{Concessive clauses}
\hypertarget{RefHeading23721935131865}{}
Concessive clauses may look exactly like the predictive conditional clauses. If the context is not clear enough, the phrase \textstyleStyleVernacularWordsItalic{nain pun} `that too' may be added between the protasis and the apodosis for clarification.  

\ea%x1655
\label{ex:x1655}
\gll [Naapeya  aara=ki  e  kasi=ke  um-inok=na]  ni  nain \\
      \\
\glt
\z

therefore  hen=CF.QM  or  cat=CF  die-IMP.3s=TP  2p.UNM  that1

kema  bagiw-ir-ap  malaria  sevis  me  wia

liver  hatred-rise-SS.SEQ  malaria  service  not  3p.ACC

iirar-eka.

remove-IMP.2p

`Therefore, (even) if hens or cats die, do not get angry and drive away the Malaria Service people.'

\ea%x1430
\label{ex:x1430}
\gll [Naap  yia  ma-ikuan=na]  \textstyleEmphasizedVernacularWords{nain  pun}  ni  kekan-ep \\
      \\
\glt
\z

thus  1p.ACC  say-FU.3p=TP  that  too  2p.UNM  be.strong-SS.SEQ

sira  eliwa  ook-eka.

custom  good  follow-IMP.2p

`Even if they talk about us like that, be strong and follow the good custom/ways.'

\subsubsection{Coordination of subordinate clauses} 
\hypertarget{RefHeading23741935131865}{}
Subordinate clauses may also be coordinated with each other, although in normal speech the frequency of these constructions is low. The only subordinate clauses in the natural text data conjoined either by juxtaposition or with the additive \textstyleStyleVernacularWordsItalic{ne}  are relative clauses. The distal demonstrative \textstyleStyleVernacularWordsItalic{nain}, functioning as a relative marker, is attached to the end of each relative clause.

\ea%x1381
\label{ex:x1381}
\gll ...[\textstyleEmphasizedVernacularWords{waaya  koka=pa  ika-i-ya  nain}]\textsubscript{RC}\textstyleEmphasizedVernacularWords{, } [\textstyleEmphasizedVernacularWords{sokowa  maneka=pa} \\
      \\
\glt
\z

pig  jungle=LOC  be-Np-PR.3s  that1  grove  big=LOC

\textstyleEmphasizedVernacularWords{ika-i-ya}  \textstyleEmphasizedVernacularWords{nain}]\textsubscript{RC}  kanu-ep  aap-ekap-ep

be-Np-PR.3s  that1  chase-SS.SEQ  BPx-come-SS.SEQ

fikera=pa-r=iw  fiirim-eka.

kunai.grass=LOC-{\O}=LIM  gather-IMP.2p

`{\dots}chase the pigs that are in the jungle (and) that are in the big grove(s) and bring them and gather them right inside the kunai grass (area).'

\ea%x1382
\label{ex:x1382}
\gll Ne \textstyleEmphasizedVernacularWords{} [\textstyleEmphasizedVernacularWords{o  maa  kamenap  on-eya  wiar \\
      \\
\glt
\z

ADD  3s.UNM  thing  how  do-SS.SEQ  3.DAT  

\textstyleEmphasizedVernacularWords{uruf-i-n}  \textstyleEmphasizedVernacularWords{nain}]\textsubscript{RC}  \textstyleEmphasizedVernacularWords{ne } [\textstyleEmphasizedVernacularWords{wiar  miim-i-n  nain}]\textsubscript{RC}

see-Np-PR.2s  that1  ADD  3.DAT  hear-Np-PR.2s  that1  

wia  maak-em-ika-i-nan.

3p.ACC  tell-SS.SIM-be-Np-FU.2s

`And you will keep telling them that which you see and which you hear him do.'

The chaining structure is also used to coordinate relative clauses (\stepcounter{nx}{\thenx}) and complement clauses that have a nominalized verb (\stepcounter{nx}{\thenx}), copied as (\stepcounter{nx}{\thenx}) below: 

\ea%x1463
\label{ex:x1463}
\gll [\textstyleEmphasizedVernacularWords{Ni  manina  urup-ep  episowa  perek-a-man }  \\
      \\
\glt
\z

2p.UNM  garden  ascend-SS.SEQ  tobacco  pick-PA-2p  

\textstyleEmphasizedVernacularWords{nain}]\textsubscript{RC}  auwa  p-ikiw-om-aka.

that  1s/p.father  BPx-go-BEN-BNFY2.IMP.2p

`Take to father the tobacco that you went up to the garden and picked.'

\ea%x1848
\label{ex:x1848}
\gll Toiyan  iiriw  maak-ep-pu-a-mik,  [\textbf{uuriw  yia} \\
      \\
\glt
\z

Toiyan  already  tell-SS.SEQ-CMPL-PA-1/3p  morning  1p.ACC

\textstyleEmphasizedVernacularWords{aaw-ep  Madang  ikiw-owa]}\textsubscript{CC} \textstyleEmphasizedVernacularWords{} nain

take-SS.SEQ  Madang  go-NMZ  that1

`We already told Toiyan about taking us in the morning and going to Madang.' 

\section{Theme, topic and focus}
\hypertarget{RefHeading23761935131865}{}
Three features of textual prominence, the pragmatic functions theme, topic and focus, are discussed in this chapter.  All of them play an important role in Mauwake, and they show up in morphology and/or syntax. They are not mutually exclusive: a clausal constituent may have more than one pragmatic function. 

Theme, topic and focus have been defined in linguistic literature in several different and sometimes conflicting ways, so they need a definition of how they are used here. 

The definitions of topic are mainly divided along two questions: whether the topic needs to be an entity -- more specifically an argument -- or not, and whether it functions on clause or discourse level, or both. 

One classic definition describes the topic as ``the entity about which something is said, whereas the further statement made about this entity is the comment'' (Crystal 1997, see also Dik 1978:19). It treats the topic as a clause-level function and an entity, but does not specify whether this entity needs to be an argument of the verb or not.  Chafe's (1976:50) well-known definition also discusses the topic on the sentence level only. Any constituent may be a topic, in fact it need not even be an entity: ``the topic sets a spatial, temporal or individual framework within which the main predication holds'' (see also Li and Thompson 1976:461). Haiman's (1978) analysis of conditionals as topics is based on this definition, as the protasis in the conditional clauses provides the presupposition for the assertion in the apodosis.

The definitions above do not touch upon topic continuity, which is an important  object of study for those linguists who consider topic mainly from discourse point of view (Giv\'on 1976, 1983a, 1990).  In this case the topic function can only be assigned to an argument of a clause. Also \citet[340]{Dixon2010a} defines topic as ``an argument which occurs in a succession of clauses in a discourse and binds them together''.\footnote{In some other approaches this discourse topic has also been called \textit{theme} or \textit{global topic}.} A single sentence can be said to have a topic only if the sentence constitutes at least a clause chain or a paragraph (Giv\'on 1990:902). 

In the following, \textstyleEmphasizedWords{\textsc{topic}} is understood in the sense that Giv\'on advocates, whereas the term \textstyleEmphasizedWords{\textsc{theme}} is used to refer to Chafe's ``topic'', for which a sentence-initial position is crucial.\footnote{Considering ``topicality'' in Chafe's sense, Mauwake is basically a subject-prominent language (Li \& Thompson 1976). It has the following characteristics: surface coding for the subject as the first argument and as the argument that governs verb agreement; scarcity of ``double subject'' constructions, even though they are possible; the subject controls co-referential constituent deletion; there are constraints on the ``topic'' constituent; and the frequency of topic-comment clauses is low. But Mauwake shares the following  features with topic-prominent languages: there is no passivization, neither are there any empty or dummy subjects.} What \citet[19]{Dik1978} calls a theme is here called a \textstyleEmphasizedWords{\textsc{left-dislocated theme}}.

Mauwake is a \textstyleAcronymallcaps{SOV} language and the default topic is also the syntactic subject and the semantic agent/actor, and yet the first \textstyleAcronymallcaps{NP} in a clause often is not the topic. This is because once the topic has been established, it is normally only marked by verbal suffixes, and the clause-initial position is taken by another constituent.  

\subsection{Theme}
\hypertarget{RefHeading23781935131865}{}
The position as the leftmost non-verb constituent in a sentence defines the theme in Mauwake. It may be an argument or a peripheral. When a sentence -- or the first clause in a multi-clause sentence -- consists of a verb only, there is no theme in the sense adopted here. When the theme is an argument, it introduces what the sentence is about (\stepcounter{nx}{\thenx}). When it is not an argument but a peripheral, it provides a circumstancial setting for the sentence, most commonly a locative or temporal setting (\stepcounter{nx}{\thenx}). A theme forms one intonation contour with the rest of the clause. 

\ea%x1908
\label{ex:x1908}
\gll \textstyleEmphasizedVernacularWords{Wi}  \textstyleEmphasizedVernacularWords{owow  mua=ke}  wilkar  wia  \\
      \\
\glt
\z

3p.UNM  village  man=CF  cart  3p.ACC  

muf-em-ik-om-a-mik.

pull-SS.SIM-be-BEN-BNFY2.PA-1/3p

`The village men pulled carts for them.'

\ea%x1698
\label{ex:x1698}
\gll Ne  \textstyleEmphasizedVernacularWords{fraide=pa}  maapora  puk-o-k,  urera.  \\
      \\
\glt
\z

ADD  Friday=LOC  party  burst-PA-3s  afternoon

`And on Friday the party started, in the afternoon.'

When the theme coincides with the subject/topic (\stepcounter{nx}{\thenx}), the clause has the default word order. But when another argument is the theme, it takes the initial position, and if there is also a subject NP in the same clause, it follows the theme  NP: 

\ea%x1473
\label{ex:x1473}
\gll [\textstyleEmphasizedVernacularWords{I  yar}]\textsubscript{O}  [i]\textsubscript{S}  uruf-am-ik-omkun \\
      \\
\glt
\z

1p.UNM  1s/p.brother-in-law  1p.UNM  see-SS.SIM-be-1s/p.DS

o  koora=pa  pok-ap  ik-ua.

3s.UNM  house=LOC  sit.down-SS.SEQ  be-PA.3s

`Our brother-in-law, as we are seeing him, is sitting in his house.' (Lit: `Our brother-in-law we are seeing and he is sitting in his house.')

The sentence (\stepcounter{nx}{\thenx}) is from the middle of a description about the arrival of the Japanese troops, and the goods that they brought are only mentioned in this one sentence, so the theme is neither the subject nor the topic. 

\ea%x1701
\label{ex:x1701}
\gll [\textstyleEmphasizedVernacularWords{Maa}  \textstyleEmphasizedVernacularWords{unowa}]\textsubscript{O}  ifer  aasa=ke  p-urup-eya  \\
      \\
\glt
\z

thing  many  sea  canoe=CF  Bpx-ascend-2/3s.DS

miiw-aasa=ke  fan  p-ir-am-ik-ua.

land-canoe=CF  here  Bpx-come-SS.SIM-be-PA.3s

`The many things were brought up (to the coast) by ships and brought here by trucks.'

In a text about a school party, dancing is first mentioned in a final verb, and then the dance becomes the theme for the following sentence:

\ea%x1702
\label{ex:x1702}
\gll Naap  ik-ok  wi  Saramun=ke  wiisa  uf-e-mik. \\
      \\
\glt
\z

thus  be-SS  3p.UNM  Saramun=CF  dance.name  dance-PA-1/3p

[\textbf{Uf-owa  eliwa}]\textsubscript{O}  i  wiar  uruf-a-mik.

dance-NMZ  good  1p.UNM  3.DAT  see-PA-1/3p

`Then the Saramun people danced \textit{wiisa}. It was a good dance we saw from them.'

Very commonly the subject only shows in the verbal suffixation, and the theme position is taken either by an object -- which may or may not be a topic -- or by an adverbial phrase. In (\stepcounter{nx}{\thenx}) the theme \textstyleStyleVernacularWordsItalic{auwa ame} `father and the others' becomes a topic that continues for the next five clauses, whereas the themes of (\stepcounter{nx}{\thenx}) and (\stepcounter{nx}{\thenx}) do not become topics and are not mentioned any more. (In the free translation is not often possible to reflect the theme naturally.)

\ea%x1909
\label{ex:x1909}
\gll \textstyleEmphasizedVernacularWords{Auwa}  \textstyleEmphasizedVernacularWords{ame}  wia  maak-eya  res  aaw-ep  \\
      \\
\glt
\z

1s/p.father  ASSOC  3p.ACC  tell-2/3s.DS  razor  take-SS.SEQ  

merena  ifa  keraw-a-k  nain  puuk-a-mik.

leg  snake  bite-PA-3s  that1  cut-PA-1/3p

`He told my father and the others, and they took  a razor and made a cut into the leg that the snake had bitten.'

\ea%x1910
\label{ex:x1910}
\gll \textstyleEmphasizedVernacularWords{Maa}  \textstyleEmphasizedVernacularWords{en-owa}  nopa-yiaw-ep  wailal-ep  \\
      \\
\glt
\z

thing  eat-NMZ  search-move.around-SS.SEQ  hunger-SS.SEQ  

naap  ma-e-mik...

thus  say-PA-1/3p

`They searched around for food and were hungry and said like that{\dots}'

\ea%x1911
\label{ex:x1911}
\gll \textstyleEmphasizedVernacularWords{Emeria}  naap  wia  aruf-i-nen  na-ep  on-a-k. \\
      \\
\glt
\z

woman  thus  3p.ACC  hit-Np-FU.1s  say/think-SS.SEQ  do-PA-3s

`He tried to hit the women like that.'

In a tail-head linkage construction (\sectref{sec:8.2.3.5}) the final verb of a sentence is repeated in the beginning of the following sentence, but in a medial form. An argument (\stepcounter{nx}{\thenx}), (\stepcounter{nx}{\thenx}), or occasionally a peripheral (\stepcounter{nx}{\thenx}), (\stepcounter{nx}{\thenx}), from the final clause may be picked as the theme of the new sentence.

\ea%x1912
\label{ex:x1912}
\gll Owowa  or-op,  wailal-ep  akia  ik-e-k.  \\
      \\
\glt
\z

village  descend-SS.SEQ  hunger-SS.SEQ  banana  roast-PA-3s  

\textstyleEmphasizedVernacularWords{Akia}  ik-ep  en-em-ik-ok{\dots}

banana  roast-SS.SEQ  eat-SS.SIM-be-SS

`He came down to the village, was hungry and roasted bananas. He roasted bananas and was eating them and {\dots}'

\ea%x1913
\label{ex:x1913}
\gll Aria,  wi  kiiriw  neeke  {\O}  miiw-aasa  um-o-k.  \\
      \\
\glt
\z

alright  3p.UNM  again  there.CF  {\O}  land-canoe  die-PA-3s

\textstyleEmphasizedVernacularWords{Miiw-a}\textstyleEmphasizedVernacularWords{asa}  um-eya  miiw-aasa  nain  on-am-ika-iwkin...

land-canoe  die-2/3s.DS  land-canoe  that1  do-SS.SIM-be-2/3p.DS

`Alright, again when they (were) there the truck broke down (lit: died). The truck broke down, and while they were working on the truck{\dots}'

  (\stepcounter{nx}{\thenx}x1914)  Yaki-ep  weeser-eya  owowa  urup-e-mik.

bathe-SS.SEQ  finish-2/3s.DS  village  ascend-PA-1/3p  

\textstyleEmphasizedVernacularWords{Owowa}  urup-ep  o  koora  ikiw-o-k.

village  ascend-SS.SEQ  3s.UNM  house  go-PA-3s

`They bathed and when it was finished they came up to the village. They came to the village and he went into the house.'

\ea%x1915
\label{ex:x1915}
\gll ...siowa  wiawi  nain=ke  alu-owa  miim-ap  \\
      \\
\glt
\z

dog  3s/p.father  that1=CF  make.noise-NMZ  hear-SS.SEQ  

karu-(o)w=iw  ekap-o-k.  \textstyleEmphasizedVernacularWords{Karu-(o)w=iw}  ekap-ep

run-NMZ=INST  come-PA-3s  run-NMZ=INST  come-SS.SEQ  

uruf-a-k=na {\dots}

see-PA-3s=TP

`The dog's owner heard the noise and came running. He came running and saw{\dots}'

It is more common to have a tail-head linkage where only the final verb is repeated; when the speaker repeats an argument or a peripheral as well, there is a reason for it: to give it prominence as the theme in the new sentence. 

When the theme position is taken by a temporal or locative adverbial phrase, it normally gives a setting for the the whole sentence:  

\ea%x1699
\label{ex:x1699}
\gll \textstyleEmphasizedVernacularWords{Eka  mamaiya  akena}  i  yoowa  me  aaw-i-yen. \\
      \\
\glt
\z

river  close  very  1p.UNM  hot  not  get-Np-FU.1p

`Very close to the river we won't get hot.'

\ea%x1916
\label{ex:x1916}
\gll \textstyleEmphasizedVernacularWords{Ikoka  kuisow}  miiw-aasa=ke  karu-eya  ku-ku-ep  \\
      \\
\glt
\z

later  one  land-canoe=CF  run-2/3s.DS  RDP-break-SS.SEQ  

or-om-ik-ua.

descend-SS.SIM-be-PA.3s

`Straight away when the trucks ran (over them) they kept breaking and falling down.'

But in the following example the first temporal phrase is a setting for only the first clause, and the final clause has another temporal phrase. Also, in the second sentence the object is both a new topic (\sectref{sec:9.2.1}) and fronted as the theme before the temporal adverbial. In neutral constituent order a temporal adverbial precedes the object. 

\ea%x1703
\label{ex:x1703}
\gll \textstyleEmphasizedVernacularWords{Uura  feenap  nain}  i  me  in-em-ik-e-mik, \\
      \\
\glt
\z

night  like.this  that1  1p.UNM  not  sleep-SS.SIM-be-PA-1/3p

amirika  maa  me  en-em-ik-e-mik.

day  food  not  eat-SS.SIM-be-PA-1/3p

\textstyleEmphasizedVernacularWords{Maa}  uura  uup-ep  en-em-ik-e-mik.

food  night  cook-SS.SEQ  eat-SS.SIM-be-PA-1/3p

`On nights like this we did not sleep, in the daytime we did not eat food. The food we used to cook and eat at night.'

Other adverbial phrases may also be used in the theme position to provide a circumstantial setting. In particular the deictic manner adverbial \textstyleStyleVernacularWordsItalic{naap} `thus, like that' is relatively common.

\ea%x1917
\label{ex:x1917}
\gll \textstyleEmphasizedVernacularWords{Naap}  maak-iwkin  naap  ik-ua.  \textstyleEmphasizedVernacularWords{Naap}  ik-ok  \\
      \\
\glt
\z

thus  tell-2/3p.DS  thus  be-PA.3s  thus  be-SS  

uruf-am-ika-iwkin  wia.

see-SS.SIM-be-2/3p.DS  no

`Like that they told him and like that he was. Like that he was and they watched him, but no (he did not get better).'

\ea%x1918
\label{ex:x1918}
\gll \textstyleEmphasizedVernacularWords{Wiena}  \textstyleEmphasizedVernacularWords{merena  ne  wapen=iw}  era  akup-ami  owowa  \\
      \\
\glt
\z

3p.GEN  foot  ADD  hand=INST  road  search-SS.SIM  village  

ikiw-e-mik.

go-PA-1/3p

`With their feet and hands they searched the road and went to the village.'

A sentence-initial adverbial phrase that is syntactically outside the clause and also has its own slightly rising intonation contour on the last syllable is here called a \textstyleEmphasizedWords{\textsc{left-dislocated theme}}. In the written text it is separated from the rest of the clause by a comma.This clause-external pragmatic function is called theme by \citet[19]{Dik1978}. He defines its function as ``specif[ying] the universe of discourse with respect to which the subsequent predication is presented as relevant''.

 In the following example, the left-dislocated theme consists of a relative clause where the antecedent noun \textstyleStyleVernacularWordsItalic{soo} `fish trap' has been deleted:

\ea%x1704
\label{ex:x1704}
\gll Aria  [\textstyleEmphasizedVernacularWords{{\O}}  \textstyleEmphasizedVernacularWords{malol=pa  ifemak-i-mik  nain}]\textsubscript{RC},  aana  \\
      \\
\glt
\z

alright  {\O}  deep.sea=LOC  press-Np-PR.1/3p  that1  cane  

puuk-i-mik  ...

cut-Np-PR.1/3p

`Alright, as for those (=fishtraps) that we let down in the deep sea, we cut cane...'

There can be more than one dislocated theme for the same clause. In (\stepcounter{nx}{\thenx}) there are two dislocated themes -- a temporal and a locative phrase -- plus a clause-internal theme \textstyleStyleVernacularWordsItalic{moma} `taro', which is the syntactic object of the clause:

\ea%x1700
\label{ex:x1700}
\gll \textstyleEmphasizedVernacularWords{Iiriw},  \textstyleEmphasizedVernacularWords{owow(a)  oko  mua  manina},  moma    \\
      \\
\glt
\z

earlier  village  other  man  garden  taro    

waaya=ke  anane  wiar  en-ow(a)=iw  ika-i-ya.

pig=CF  always  3.DAT  eat-NMZ=INST  be-Np-PR.3s

`Earlier, (in) the garden of a man from another village, his taro was always being eaten by a pig.'

\subsection{Topic} 
\hypertarget{RefHeading23801935131865}{}
Giv\'on (1976:152) posited a universal topicality hierarchy, which shows features affecting the likelihood of \textstyleAcronymallcaps{NP}s becoming discourse topics: 

a.  human {\textgreater} non-human

b.  definite {\textgreater} indefinite

c.  more involved participant {\textgreater} less involved participant

d.  1\textsuperscript{st} person {\textgreater} 2\textsuperscript{nd} person {\textgreater} 3\textsuperscript{rd} person

This hierarchy can be observed in Mauwake as well: the prototypical topic refers to a referent that is human and definite, and if the first person is involved in the text, it is often the topic. And the most involved participant, the grammatical subject, is typically also the pragmatic topic. 

The following three sections discuss how a new topic is introduced and maintained in a narrative text, and how it is brought back after it has been absent for a while. 

\subsubsection{Introducing a new topic}
\hypertarget{RefHeading23821935131865}{}
Even when a new topic is introduced for the first time, it is often definite,\footnote{Definiteness is not an obligatory category in Mauwake. A NP may be marked as definite or indefinite when this  feature is considered important enough, but often it is left unspecified.} identifiable to the addressee: a personal pronoun, a proper name or a relationship term, or a noun phrase. 

\ea%x1663
\label{ex:x1663}
\gll \textstyleEmphasizedVernacularWords{I}  me  amis-ar-em-ik-omkun  iinan  aasa  \\
      \\
\glt
\z

1p.UNM  not  knowledge-INCH-SS.SIM-be-1s/p.DS  sky  canoe

iinan=pa  fan  ekap-emi  ...

sky=LOC  here  come-SS.SIM

`We were not aware (that anything would happen) and planes came here on the sky and {\dots}'

\ea%x1664
\label{ex:x1664}
\gll \textstyleEmphasizedVernacularWords{Muakura=ke}  ma-e-k,  ``  {\dots''} \\
      \\
\glt
\z

Muakura=CF  say-PA-3s

`Muakura said, `` ... '' '

\ea%x1665
\label{ex:x1665}
\gll Aria  \textstyleEmphasizedVernacularWords{yena  mua } pun  ...  iirar-iwkin  owowa  ekap-o-k. \\
      \\
\glt
\z

alright  1s.GEN  man  also  ...  remove-2/3p.DS  village  come-PA-3s

`Alright they also dismissed my husband {\dots} and he came to the village.'

\ea%x1667
\label{ex:x1667}
\gll Iiriw  \textstyleEmphasizedVernacularWords{wi}  \textstyleEmphasizedVernacularWords{mua  iperowa=ke}  feenap  \\
      \\
\glt
\z

earlier  3p.UNM  man  middle.aged=CF  like.this

ma-em-ik-e-mik,  emeria=ke  osaiwa  ar-e-mik.

say-SS.SIM-be-PA-1/3p  woman=CF  bird.of.paradise  become-PA-1/3p

`Earlier the elders kept telling this story that women had changed into birds of paradise.'

When a potential topic is introduced it is indefinite -- the addressee is not expected to be able to identify it -- and one of the following strategies is used. The new topic may  first be an object in a clause before becoming the subject in the following clause. (\stepcounter{nx}{\thenx}) is repeated here: 

\ea%x1668
\label{ex:x1668}
\gll Uura  feenap  nain  i  me  in-em-ik-e-mik, \\
      \\
\glt
\z

night  like.this  that1  1p.UNM  not  sleep-SS.SIM-be-PA-1/3p

amirika  \textstyleEmphasizedVernacularWords{maa}  me  en-em-ik-e-mik.

noon  food  not  eat-SS.SIM-be-PA-1/3p

\textstyleEmphasizedVernacularWords{Maa}  uura  uup-ep  en-em-ik-e-mik.

food  night  cook-SS.SEQ  eat-SS.SIM.be-PA-1/3p

`On nights like this we did not sleep, at noon we did not eat food. Food we used to cook and eat at night.'

Most commonly, the new topic is already a subject in the clause where it is introduced, and the \textstyleAcronymallcaps{NP} is either modified with the indefinite \textstyleStyleVernacularWordsItalic{oko} `other' or marked by the neutral focus clitic \textit{-}\textstyleStyleVernacularWordsItalic{ko}, which also has its origin in \textstyleStyleVernacularWordsItalic{oko}. Occasionally both of them are used on the same \textstyleAcronymallcaps{NP} (\stepcounter{nx}{\thenx}). 

\ea%x1669
\label{ex:x1669}
\gll Iiriw  Malala  suule  maneka  \textstyleEmphasizedVernacularWords{uuw-owa  mua  oko}  unuma  Kila. \\
      \\
\glt
\z

earlier  Malala  school  big  work-NMZ  man  other  name  Kila

`Earlier there was a workman at the big Malala school whose name was Kila.'

\ea%x1670
\label{ex:x1670}
\gll \textstyleEmphasizedVernacularWords{Emer(a)}  \textstyleEmphasizedVernacularWords{en-ow(a)  mua=ko}  emeria  fan  aaw-o-k. \\
      \\
\glt
\z

sago  eat-NMZ  man=NF  woman  here  take-PA-3s

`A Sepik man married a wife here.'

\ea%x1671
\label{ex:x1671}
\gll Iiriw  akena  \textstyleEmphasizedVernacularWords{mua  oko=ko}  fura  aaw-ep  koka  iw-a-k. \\
      \\
\glt
\z

earlier  very  man  other=NF  knife  take-SS.SEQ  jungle  go-PA-3s

`Long ago a man took a knife and went into the jungle.'

The indefinite \textstyleAcronymallcaps{NP} may even have contrastive focus marking:

\ea%x1666
\label{ex:x1666}
\gll Pika  ifara  mufe-wiaw-ik-ok  \textstyleEmphasizedVernacularWords{ifa  maneka=ke}  siowa  \\
      \\
\glt
\z

fence  vine  pull-move.around-be-SS  snake  big=CF  dog

wiar  aaw-o-k.

3.DAT  take-PA-3s

`As he was pulling around vines for the fence, a big snake grabbed his dog.'

Existential clauses, which Giv\'on (1990:741) mentions as one of the major strategies for introducing important topics, are possible but not very commonly employed for this function in Mauwake. Note that the neutral focus clitic -\textstyleStyleVernacularWordsItalic{ko} is also present.

\ea%x1672
\label{ex:x1672}
\gll \textstyleEmphasizedVernacularWords{Iiriw}  \textstyleEmphasizedVernacularWords{mua  iperowa=ko}  nan  Wakoruma  owowa=pa  ik-ua. \\
      \\
\glt
\z

earlier  man  middle.aged=NF  there  Wakoruma  village=LOC  be-PA.3s

`Earlier there was a middle-aged man in Wakoruma village.'

\subsubsection{Maintaining an established topic} 
\hypertarget{RefHeading23841935131865}{}
When a potential topic, which is indefinite when first introduced, becomes established the next mention is often made with a \textstyleAcronymallcaps{NP} marked as definite by the distal-1 demonstrative \textstyleStyleVernacularWordsItalic{nain} `that'. The sentence  following (\stepcounter{nx}{\thenx}) in the text is (\stepcounter{nx}{\thenx}). 

\ea%x1673
\label{ex:x1673}
\gll \textstyleEmphasizedVernacularWords{Mua  nain}  emeria  ne  muuka  wiipa  marew. \\
      \\
\glt
\z

man  that1  woman  ADD  son  daughter  no(ne).

`The man had no wife or children.'

Another possibility is the mere subject marking on the verb: sentence (\stepcounter{nx}{\thenx}) below is continuation to the sentence (\stepcounter{nx}{\thenx}) above. 

\ea%x1674
\label{ex:x1674}
\gll Ne  manina  ikiw-o-\textstyleEmphasizedVernacularWords{k}. \\
      \\
\glt
\z

ADD  garden  go-PA-3s.

`And he went to the garden.'

When the topic is already definite when introduced, it is possible to make a second mention with a personal pronoun. (This whole story is Text 2 in Appendix 2.) 

\ea%x1919
\label{ex:x1919}
\gll Yena  yaiya  Tup  ifa  ku-o-k  nain  opaimika  \\
      \\
\glt
\z

1s.GEN  1s/p.uncle  Tup  snake  bite-PA-3s  that1  talk  

ma-i-yem.  Ae,  \textstyleEmphasizedVernacularWords{o}  fiker(a)  gone  urup-o-k.

say-Np-PR.1s  yes  3s.UNM  kunai.grass  middle  ascend-PA-3s

`I tell a story about that when my uncle Tup was bitten by a snake. Yes, he went up to the middle of the \textit{kunai} grass (area).'

In (\stepcounter{nx}{\thenx}) the personal pronoun \textstyleStyleVernacularWordsItalic{wi}  `they' in the second sentence refers to the ``weria-relatives'' (see \sectref{sec:1.3.6}), introduced as the object in the preceding sentence; a mere verbal suffix would indicate continuation with the old topic, i.e. those who sent the message. 

\ea%x1868
\label{ex:x1868}
\gll Wiena  mua  weria  ...  opaimika  wia  \\
      \\
\glt
\z

3p.GEN  man  planting.stick  ...  talk  3p.ACC

sesek-omak-e-mik.  Ne  \textstyleEmphasizedVernacularWords{wi}  ekap-e-mik.

send-DISTR/PL-PA-1/3p  ADD  3p.UNM  come-PA-1/3p

`They\textsubscript{i} sent word to their\textsubscript{i} many weria-relatives\textsubscript{j}. And they\textsubscript{j} came.' 

More commonly the topic, once established, is maintained as a continuing topic without an overt \textstyleAcronymallcaps{NP} or a pronoun, only via subject marking on the verb. This minimal marking conforms to Giv\'on's (1983a:67) claim that the heaviness of the topic marking is in inverse relation to topic continuity/predictability. The following example is a section of a text where \textstyleStyleVernacularWordsItalic{sawur} `spirits', introduced earlier, think that there is a boy on the bed they are carrying, but the boy has already escaped. The reference to the spirits is only made by medial and final verb suffixes. 

\ea%x1675
\label{ex:x1675}
\gll Ne  aria,  samapora  oram  akua  aaw-\textstyleEmphasizedVernacularWords{ep}  ikiw-e-\textstyleEmphasizedVernacularWords{mik}. \\
      \\
\glt
\z

ADD  alright,  bed  just  shoulder  take-SS.SEQ  go-PA-1/3p

Ikiw-\textstyleEmphasizedVernacularWords{ep}  wiena  owowa=pa  uruf-a-\textstyleEmphasizedVernacularWords{mik}=na  weetak,

go-SS.SEQ  3p.GEN  village=LOC  see-PA-1/3p=TP  no

samapora  muutiw  akua  aaw-e-\textstyleEmphasizedVernacularWords{mik}.

bed  only  shoulder  take-PA-1/3p

Aria  nainiw  kir-e-\textstyleEmphasizedVernacularWords{mik}.  Kir-\textstyleEmphasizedVernacularWords{ep}  ekap-em-ika-\textstyleEmphasizedVernacularWords{iwkin}

alright  again  turn-PA-1/3p  turn-SS.SEQ  come-SS.SIM-be-2/3p.DS

epa  wiim-o-k.

place  dawn-PA-3s

`Alright, they carried just the bed and went. They went and in their village they looked but (to their surprise) they only carried the bed. Alright they turned back again. They turned and as they were coming, it dawned.'

The switch-reference system (\sectref{sec:3.8.3.4}, 8.2.3) together with the person/number marking in the finite verbs can easily keep track of two active topics alternating with each other. 

\ea%x1676
\label{ex:x1676}
\gll O  iiwawun  samor  aaw-o-k.  Ne  nan  ik-e-\textstyleEmphasizedVernacularWords{mik}.  \\
      \\
\glt
\z

3s.UNM  altogether  badly  get-PA-3s  ADD  there  be-PA-1/3p

Nan  ik-\textstyleEmphasizedVernacularWords{ok}  ik-\textstyleEmphasizedVernacularWords{ok}  neeke  pu-o-\textstyleEmphasizedVernacularWords{k}.  Neeke  pu-\textstyleEmphasizedVernacularWords{eya}  oram

there  be-SS  be-SS  there.CF  die-PA-3s.  there.CF  die-2/3s.DS  just

akua  aaw-e-\textstyleEmphasizedVernacularWords{mik}.

shoulder  take-PA-1/3p

`He got really bad. They stayed and stayed there and he died there. He died there and they just carried him on their shoulders.'

In the example above one of the topics is in the singular, the other in the plural. But in (\stepcounter{nx}{\thenx}), repeated below as (\stepcounter{nx}{\thenx}), there are two third person singular topics alternating, with only the verbal marking to indicate who is doing what:

\ea%x1920
\label{ex:x1920}
\gll Ifakim-\textbf{eya}  \textstyleEmphasizedVernacularWords{\textmd{pu-ep-ik}}\textstyleEmphasizedVernacularWords{-eya } om-em-ik-\textbf{ua}. \\
      \\
\glt
\z

kill-2/3s.DS  die-SS.SEQ-be-2/3s.DS  cry-SS.SIM-be-PA.3s

`When she killed him and he was dead, she was crying.'

\subsubsection{Re-activating an earlier topic}
\hypertarget{RefHeading23861935131865}{}
When a topic has not been active for some time in the text, there are two main strategies to re-activate it. A personal pronoun is mainly used for the major participants. For the third person singular pronoun this is the most common usage (\sectref{sec:3.5.11}). The following example is from a text where the main participant, a man, has been killed by his spirit lover. For several sentences the topic position is taken by the spirit woman and her parents, but in the sentence (\stepcounter{nx}{\thenx}) the man, as a re-activated topic, gets up and goes to his village.

\ea%x1921
\label{ex:x1921}
\gll Epa  wiim-eya  sawur  emeria  nain  ikiw-eya  \\
      \\
\glt
\z

place  dawn-2/3s.DS  spirit  woman  that1  go-2/3s.DS  

\textstyleEmphasizedVernacularWords{o}  iikir-ami  owowa  ekap-o-k.

3s.UNM  get.up-SS.SIM  village  come-PA-3s

`It dawned and the spirit woman went, and he got up and came to the village.'

The sentence (\stepcounter{nx}{\thenx}) re-activates the topic, a grandmother and two grandchildren, after a gap of five clauses:

\ea%x1923
\label{ex:x1923}
\gll Iwera  mekemkar-ep  or-eya  \textstyleEmphasizedVernacularWords{wi}  pikin-ep  \\
      \\
\glt
\z

coconut  bend-SS.SEQ  descend-2/3s.DS  3p.UNM  jump-SS.SEQ

miiwa  or-o-mik.

ground  descend-PA-1/3p

`The coconut tree bent down and they jumped down to the ground.'

In the following stretch the health officer, who is one of the main participants in this section of the story, is mentioned as an object NP, and after two clauses he becomes the topic for just one clause. Afterwards the men accompanying the sick man again resume as the topic. 

\ea%x1924
\label{ex:x1924}
\gll ...ikemika  kaik-owa  mua  nain  nop-a-mik,  imen-ap  \\
      \\
\glt
\z

wound  tie-NMZ  man  that1  search-PA-1/3p  find-SS.SEQ  

maak-iwkin  \textstyleEmphasizedVernacularWords{o}  miim-o-k.  Aria,  \textstyleEmphasizedVernacularWords{wi}  kiiriw  neeke

tell-2/3p.DS  3s.UNM  precede-PA-3s  alright  3p.UNM  again  there.CF

{\O}  miiw-aasa  um-o-k.

{\O}  land-canoe  die-PA-3s

`{\dots} they searched for the health officer, and when they found him and told him, he went ahead of them (to the aidpost). Alright, again when they were there the truck broke down.'

A full noun phrase is used for reactivating major participants when there are several of them and pronouns are not adequate for disambiguating between them. A \textstyleAcronymallcaps{NP} is always used when minor participants and props\footnote{Participants are typically human and active in a narrative, props are non-human and inactive. }  are brought back to the stage. 

The following example is an extract from a story about the speaker's uncle, who was introduced with a kinship term at the very beginning of the story and only referred to by a verbal suffix or an occasional pronoun afterwards. An  important prop, snake poison, which becomes a topic for a short stretch, is referred to by a full \textstyleAcronymallcaps{NP}.

\ea%x1677
\label{ex:x1677}
\gll Akia  ik-ep  en-em-ik-ok  \textstyleEmphasizedVernacularWords{ifa  marasin } \\
      \\
\glt
\z

banana  roast-SS.SEQ  eat-SS.SIM.be-SS  snake  poison  

\textstyleEmphasizedVernacularWords{nain=ke}  kema  wiar  iw-a-k.

that1=CF  liver  3.DAT  go-PA-3s

Iw-aya  nan  miira  saawirin-e-k.

go-2/3s.DS  there  face  become.round-PA-3s

Ne  auwa  ame  wia  maak-eya  res  aaw-ep

ADD  1s/p.father  ASS  3p.ACC  tell-2/3s.DS  razor  take-SS.SEQ

merena  ...  nain  puuk-a-mik.  Puuk-ap  marasin

leg  {\dots}  that1  cut-PA-1/3p  cut-SS.SEQ  medicine

wu-om-a-mik.  Marasin  wu-om-a-mik=na

put-BEN-BNFY2.PA-1/3p  medicine  put-BEN-BNFY2.PA-1/3p=TP  

weetak.  Iiriw  \textstyleEmphasizedVernacularWords{ifa  marasin=ke}  kekan-e-k.

no  earlier  snake  poison=CF  be.strong-PA-3s  

Ne  akia  ik-e-k  nain  me  en-e-k.  Nan

ADD  banana  roast-PA-3s  that1  not  eat-PA-3s  there  

mukuna=pa  ik-eya \textstyleEmphasizedVernacularWords{} o  nan  samor  aaw-o-k.

fire=LOC  be-2/3s.DS  3s.UNM  there  badly  get-PA-3s

`He roasted bananas and when he was eating them the snake poison entered his liver. It entered and he felt dizzy there. And when he told my father and others, they took a razor and made a cut into the leg{\dots} They made a cut and put medicine into it. They put medicine but no (it didn't help). The snake poison was already strong. And he didn't eat the bananas that he roasted. They were there on the fire and he really got bad there.'

\subsubsection{Highlighted topic}
\hypertarget{RefHeading23881935131865}{}
The topic marker -(\textstyleStyleVernacularWordsItalic{e})\textstyleStyleVernacularWordsItalic{na} (\sectref{sec:3.12.7.1}) is only used with a new topic that the speaker wants to highlight. The constituent may have been briefly mentioned in an earlier clause (\stepcounter{nx}{\thenx}), (\stepcounter{nx}{\thenx}), or it is known from the outset as a future topic (\stepcounter{nx}{\thenx}), and now it is specified as the topic for the following section of text. 

\ea%x1680
\label{ex:x1680}
\gll Mauw-owa  kamenap  nain  on-a-man?  \textstyleEmphasizedVernacularWords{Mauw-owa}\textstyleEmphasizedVernacularWords{=na} \\
      \\
\glt
\z

work-NMZ  what.like  that1  do-PA-2p  work-NMZ=TP

sira  yia  nokar-e-mik,  yiena  kae  sira

custom  1p.ACC  ask-PA-1/3p  1p.GEN  1s/p.grandfather  custom

nain.

that1

`What kind of work did you do? - The work (was such that) they they asked us about customs, our ancestral customs.'

\ea%x1678
\label{ex:x1678}
\gll ...Filip  uruf-ap  maak-e-k,  ``Ikos  ikiw-u.''  \textstyleEmphasizedVernacularWords{Filip=na} \\
      \\
\glt
\z

...Filip  see-SS.SEQ  tell-PA-3s  together  go-IMP.1d  Filip=TP

ona  owowa  Suaru,  {\dots}

3s.GEN  village  Suaru

`He saw Filip and told him, ``We'll go together.'' Filip is/was from Suaru, {\dots}'

\ea%x1679
\label{ex:x1679}
\gll Yo  efa  aaw-eya  i  owawiya  ik-omkun  \\
      \\
\glt
\z

1s.UNM  1s.ACC  take-2/3s.DS  1p.UNM  together  be-1s/p.DS

\textstyleEmphasizedVernacularWords{Yaapan}\textstyleEmphasizedVernacularWords{=ena}  Wewak  kame=pa  nan  ir-a-mik.

Japan=TP  Wewak  side=LOC  there  come-PA-1/3p

`He married me, and while we were together the Japanese came from the Wewak side.'

The highlighted topic is more frequent in conversations than in narratives. Since the first and second persons are more topical than the third person, they may even get marked as a topic when the discussion itself is about something else. The second person singular pronoun with the topic marker, \textstyleStyleVernacularWordsItalic{nos-na,} has acquired the meaning somewhat like `you know' in English (\stepcounter{nx}{\thenx}).

\ea%x1684
\label{ex:x1684}
\gll \textstyleEmphasizedVernacularWords{Nos=na}? \\
      \\
\glt
\z

2s.FC=TP

`What about you?' or: `What do you want?' or: `Where are you going?'

\ea%x1685
\label{ex:x1685}
\gll Emar,  \textstyleEmphasizedVernacularWords{yos=na}  amina=ke  weetak. \\
      \\
\glt
\z

friend  1s.FC=TP  saucepan=CF  no

`Friend, I do not have a saucepan.'

\ea%x1686
\label{ex:x1686}
\gll \textstyleEmphasizedVernacularWords{Is}\textstyleEmphasizedVernacularWords{=na}  yoo,  takira  fain  ifa=ke  ku-eya  akua  \\
      \\
\glt
\z

1p.FC=TP  INTJ  boy  this  snake=CF  bite-2/3s.DS  shoulder

aaw-ep  ekap-em-ika-i-mik  yoo!

take-SS.SEQ  come-SS.SIM-be-Np-PR.1/3p  INTJ

`We - this boy was bitten by a snake and we are coming carrying him on our shoulders.'

\ea%x1687
\label{ex:x1687}
\gll Mauwa  ar-e-n,  amia=iya  nenar-e-mik=i?  \\
      \\
\glt
\z

what  become-PA-2s  bow=COM  shoot.you-PA-1/3p=QM

-Wia,  \textstyleEmphasizedVernacularWords{nos=na},  yo  fiker  fufa  iw-ap  nefa

No  2s.FC=TP  1s.UNM  kunai.grass  base  go-SS.SEQ  2s.ACC

far-i-yem.

call-Np-PR.1s

`What happened to you, did they shoot you with a gun? -No, you know, I went inside the \textit{kunai} grass and am calling you.'

\subsection{Focus constructions}\footnotemark{}
\hypertarget{RefHeading23901935131865}{}
\footnotetext{ This section is based on my earlier paper (J\"arvinen 1988b).}
The focus discussed here refers to special prominence given to some constituent in a clause \citep[174]{Dixon2010a}. In Mauwake it is not the same as new information, as the focused element can be either new or given information. And it cannot be contrasted with topic, as the topic may receive focus marking as well. 

The main discussion concentrates on the two focus clitics, but syntactic and phonological focusing devices are also briefly touched on.

Since focus refers to special prominence, it is possible to have clauses and sentences without any focus marking; in fact, most of the clauses do not have any. But it is also possible to have more than one focused item in a clause. 

\subsubsection{Contrastive focus}
\hypertarget{RefHeading23921935131865}{}
\citet{Chafe1976} lists the following factors as necessary for focus of contrast: the knowledge that someone did something, a set of possible candidates in the addressee's mind, and the assertion as to which of these candidates is the correct one. Thus there can be no contrast if the number of candidates is either unlimited or one. A contrastive sentence often contradicts the addressee's expectation, but this is not crucial; what is essential is that there is a set of possible candidates in the addressee's mind. Also, as  \citet[348]{Linde1979} remarks, marking contrast is not obligatory. Even if there are more candidates than one in the addressee's mind, it is still the speaker who decides whether to make something overtly contrastive or not.  

The contrastive focus marker in Mauwake is -\textstyleStyleVernacularWordsItalic{ke}, attached as a clitic to a noun phrase (\sectref{sec:3.12.7.2}). It can also follow a temporal or location word, although this is rare. The domain of the contrastive focus is one constituent. It is used when the \textstyleAcronymallcaps{NP} is in focus and is contrasted with something else. 

\ea%x1688
\label{ex:x1688}
\gll \textstyleEmphasizedVernacularWords{Os=ke}  ikiw-o-k. \\
      \\
\glt
\z

3s.FC=CF  go-PA-3s

`It was he who went.'

\ea%x1689
\label{ex:x1689}
\gll \textstyleEmphasizedVernacularWords{Mua  bug  maala  nain}\textstyleEmphasizedVernacularWords{=ke}  mera  unowa  isak-i-non, \\
      \\
\glt
\z

man  wind  long  that1=CF  fish  many  spear-Np-FI.3s

mua  bug  iiwa  nain  weetak.

man  wind  short  that1  no

`A man with big lungs will spear many fish, a man with small lungs no. 

Sometimes the use or non-use of the contrastive marker makes a difference in the interpretation of the meaning of a word. In (\stepcounter{nx}{\thenx}) \textstyleStyleVernacularWordsItalic{maneka} `big' refers to the size of the man as a neutral quality, but in (\stepcounter{nx}{\thenx}) either his big size is contrasted with the size of other people, or he is set apart as one of a limited set of big men, i.e. chiefs. 

\ea%x1690
\label{ex:x1690}
\gll Mua  nain  maneka. \\
      \\
\glt
\z

man  that1  big

`That man is big.'

\ea%x1691
\label{ex:x1691}
\gll Mua  nain  \textstyleEmphasizedVernacularWords{maneka=ke}. \\
      \\
\glt
\z

man  that1  big=CF

`That man is \textstyleStyleVernacularWordsItalic{\textsc{big}}\textsc{'} or: `That man is a big man/chief.'

The main placement for the contrastive focus marking is the subject (\stepcounter{nx}{\thenx}),\footnote{Waskia has an identical morpheme \textit{ke}, labeled as a subject marker, which is very similar in function \citep[36]{RossEtAl1978}%Paol
.} (\stepcounter{nx}{\thenx}), or the non-verbal predicate of a verbless clause (\stepcounter{nx}{\thenx}). In a few cases some other constituent is marked: a contrasted object or adverbial phrase.

\ea%x1692
\label{ex:x1692}
\gll Ne  \textstyleEmphasizedVernacularWords{erepam  nain=ke}  wiena  skul  stua  on-a-mik. \\
      \\
\glt
\z

ADD  four  that1=CF  3p.GEN  school  store  make-PA-1/3p

`And the fourth one they made into their school store.'

\ea%x1697
\label{ex:x1697}
\gll Maa  en-owa  iw-e-mik,  \textstyleEmphasizedVernacularWords{rais=ke}  weetak. \\
      \\
\glt
\z

food  eat-NMZ  give.him-PA-1/3p  rice=CF  no

`They gave him food (root crops), but not rice.'

\ea%x1693
\label{ex:x1693}
\gll \textstyleEmphasizedVernacularWords{Amirika}\textstyleEmphasizedVernacularWords{=ke}  eliw  ika-i-yem,  \textstyleEmphasizedVernacularWords{uura=ke}  napum-ar-i-yem. \\
      \\
\glt
\z

noon=CF  well  be-Np-PR.1s  night=CF  sick-INCH-Np-PR.1s

`At noon I am well, at night I am sick.'

\ea%x1694
\label{ex:x1694}
\gll \textstyleEmphasizedVernacularWords{Amiten}\textstyleEmphasizedVernacularWords{=ke}  ikiw-i-yem,  \textstyleEmphasizedVernacularWords{Susure=ke}  me  ikiw-i-yem. \\
      \\
\glt
\z

Amiten=CF  go-Np-PR.1s  Susure=CF  not  go-Np-PR.1s

`I go to Amiten but I don't go to Susure.'

The contrastive focus marker is also used for subject disambiguation. When an object of a clause is fronted as the theme, the constituent order changes from \textstyleAcronymallcaps{SOV} to \textstyleAcronymallcaps{OSV}. If both the subject and object are in the third person and are realized as overt NPs, this creates a potential ambiguity as to which one is which argument. The contrastive focus marker, added to the subject, may first have been used to disambiguate clauses like (\stepcounter{nx}{\thenx}) and later spread as an optional marking even to clauses where the verbal suffix distinguishes the two arguments and which would not need this extra marking (\stepcounter{nx}{\thenx}). Without the contrastive focus marking the example (\stepcounter{nx}{\thenx}) would mean `and my younger sibling got/took an arrow/a bullet'.

\ea%x1695
\label{ex:x1695}
\gll Ne  yena  aamun  \textstyleEmphasizedVernacularWords{ariwa=ke}  aaw-o-k. \\
      \\
\glt
\z

ADD  1s.GEN  1s/p.younger.sibling  arrow=CF  get-PA-3s

`And my younger sibling was killed by a bullet (lit: arrow).'

\ea%x1696
\label{ex:x1696}
\gll Fofa=pa  maa  mauwa  on-i-mik  nain  (\textstyleEmphasizedVernacularWords{yos=ke})  \\
      \\
\glt
\z

market=LOC  thing  what  do-Np-PR.1/3p  that1  1s.FC=CF

ma-i-yem.

say-Np-PR.1s

`I tell what we do at the market.' 

Although the contrastive focus marking is very common when both the subject and object in an \textstyleAcronymallcaps{OSV} clause are \textstyleAcronymallcaps{NP}s, it is not obligatory. See (\stepcounter{nx}{\thenx}) for an example. 

Also in other situations where there is ambiguity about the subject, the \textstyleAcronymallcaps{CF} marker is used. If (\stepcounter{nx}{\thenx}) did not have \textstyleAcronymallcaps{CF} marking, Pita would be interpreted as the possessor of the betelnuts and the meaning would be `He stole Pita's betelnuts from me'.

\ea%x1705
\label{ex:x1705}
\gll \textstyleEmphasizedVernacularWords{Pita}\textstyleEmphasizedVernacularWords{=ke}  owora  efar  ikum  aaw-eya  {\dots} \\
      \\
\glt
\z

Pita=CF  betelnut  1s.DAT  illicitly  take-2/3s.DS

`Pita stole my betelnuts, and {\dots}'

In (\stepcounter{nx}{\thenx}) the second \textstyleAcronymallcaps{CF} is necessary, because otherwise `this other one' would be interpreted as an object; the real object in the second clause clause is marked by zero.

\ea%x1706
\label{ex:x1706}
\gll Ikoka  \textstyleEmphasizedVernacularWords{masin  kaanin=ke}  samor-ar-eya  \textstyleEmphasizedVernacularWords{oko  fain=ke} \\
      \\
\glt
\z

later  engine  which=CF  bad-INCH-2/3s.DS  other  this=CF

asip-i-non.

help-Np-FU.3s

`Later when any of the engines breaks this other one will help it.'

A clause can only have one constituent with contrastive focus. In a verbless clause  either the subject or the non-verbal predicate can be marked with it, but not both.

\ea%x1707
\label{ex:x1707}
\gll Yo  \textstyleEmphasizedVernacularWords{owow(a)  saria=ke}. \\
      \\
\glt
\z

1s.UNM  village  headman=CF

`I am the village headman.'

\ea%x1708
\label{ex:x1708}
\gll Wia,  \textstyleEmphasizedVernacularWords{yos=ke}  owow  saria  ika-i-yem. \\
      \\
\glt
\z

no  1s.FC=CF  village  headman  be-Np-PR.1s

`No, \textit{I} am the village headman.'

Contrastive focus can be assigned to a constituent regardless of whether it is given or new, definite or indefinite. Non-verbal predicates with \textstyleAcronymallcaps{\textup{CF}} are mostly new information, whereas for subjects neither givenness nor definiteness matters.

\ea%x1709
\label{ex:x1709}
\gll Iperuma  nain  me  enim-eka,  \textstyleEmphasizedVernacularWords{inasin  mua=ke}. \\
      \\
\glt
\z

eel  that1  not  eat-IMP.2p  spirit  man=CF

`Don't eat the eel, it is a spirit man.'

\ea%x1710
\label{ex:x1710}
\gll Iir  oko  \textstyleEmphasizedVernacularWords{mua  oko=ke}  koora  ku-ek-a-m  na-ep \\
      \\
\glt
\z

time  other  man  other=CF  house  build-CNTF-PA-1s  say-SS.SEQ

maakara  war-ep  {\dots}

timber  cut-SS.SEQ

`Another time a man wanted to build a house and he cut timber and {\dots}'

\ea%x1711
\label{ex:x1711}
\gll \textstyleEmphasizedVernacularWords{Aaya}  \textstyleEmphasizedVernacularWords{nain=ke}  ifa  puuk-a-k. \\
      \\
\glt
\z

sugarcane  that1=CF  snake  change.into-PA-3s

`The sugarcane changed into a snake.'

In non-polar questions it is the questioned element that is in focus. This is reflected in the question words and in the answers. When the question word is the subject or the non-verbal predicate, it takes the \textstyleAcronymallcaps{CF} clitic, and the corresponding constituent in the answer usually gets the focus marking as well.

\ea%x1714
\label{ex:x1714}
\gll \textstyleEmphasizedVernacularWords{Mua  naareke}  nefa  maak-e-k? \\
      \\
\glt
\z

man  who.CF  2s.ACC  tell-PA-3s

`Who told you?'

\ea%x1715
\label{ex:x1715}
\gll \textstyleEmphasizedVernacularWords{Mua}\textstyleEmphasizedVernacularWords{=ke}  me  efa  maak-e-mik,  yena  mokok=iw  uruf-a-m. \\
      \\
\glt
\z

man=CF  not  1s.ACC  tell-PA-1/3p  1s.GEN  eye=INST  see-PA-1s

`It wasn't people that told me, I saw it with my own eyes. '

\ea%x1712
\label{ex:x1712}
\gll Maa  nain  \textstyleEmphasizedVernacularWords{mauwa=ke}? \\
      \\
\glt
\z

thing  that1  what=CF

`What is that thing?'

\ea%x1713
\label{ex:x1713}
\gll Maa  nain  \textstyleEmphasizedVernacularWords{posa=ke}. \\
      \\
\glt
\z

thing  that1  turban.shell

`That thing is a turban shell.'

The contrastive focus clitic and the question clitic -\textstyleStyleVernacularWordsItalic{i} merge into -\textstyleStyleVernacularWordsItalic{ki} when both are used with the same constituent. This happens when the non-verbal predicate of a verbless clause is questioned, in alternative questions, and sometimes in alternative statements.

\ea%x1716
\label{ex:x1716}
\gll Emeria  fain  \textstyleEmphasizedVernacularWords{Eema=ki}? \\
      \\
\glt
\z

woman  this  Eema=CF.QM

`Is this woman Eema?'

\ea%x1718
\label{ex:x1718}
\gll Emeria  fain  \textbf{Eema=ki } e  \textbf{emeria  oko=ke?} \\
      \\
\glt
\z

woman  this  Eema=CF.QM  or  woman  other=CF

`Is this woman Eema or another woman?'

\ea%x1717
\label{ex:x1717}
\gll \textstyleEmphasizedVernacularWords{Iwer(a)  eka}\textstyleEmphasizedVernacularWords{=ki}  e  \textstyleEmphasizedVernacularWords{mauwa=ki},  \textstyleEmphasizedVernacularWords{owora=ki},  \\
      \\
\glt
\z

coconut  water=CF.QM  or  what=CF.QM  betelnut=CF.QM

\textstyleEmphasizedVernacularWords{episowa}\textstyleEmphasizedVernacularWords{=ki}  ika-i-non  ...

tobacco=CF.QM  be-Np-FU.3s

`If there is coconut water, or something else, betelnut, or tobacco {\dots}'

\subsubsection{Neutral focus} 
\hypertarget{RefHeading23941935131865}{}
The neutral focus clitic (\sectref{sec:3.12.7.2}) most commonly occurs in irrealis-type clauses, i.e. questions, commands, negated clauses or those with future tense, hence its original name in J\"arvinen (1988b). But it is also used in a some realis-type clauses. Although the clitic has probably developed from the indefinite \textstyleStyleVernacularWordsxiiptItalic{oko} `a (certain), (an)other', it is added to definite noun phrases as well. 

\ea%x1719
\label{ex:x1719}
\gll \textstyleEmphasizedVernacularWords{Aaya}\textstyleEmphasizedVernacularWords{=ko}  niar  ik-ua=i? \\
      \\
\glt
\z

sugar=NF  2p.DAT  be-PA.3s=QM

`Do you have (any) sugar?'

\ea%x1720
\label{ex:x1720}
\gll Ikiw-ep  \textstyleEmphasizedVernacularWords{maa  en-owa=ko}  nop-aka. \\
      \\
\glt
\z

go-SS.SEQ  food  eat-NMZ=NF  fetch-IMP.2s

`Go and fetch (some) food.'

\ea%x1721
\label{ex:x1721}
\gll \textstyleEmphasizedVernacularWords{Owowa}  \textstyleEmphasizedVernacularWords{oko=ko}  me  uf-e-mik. \\
      \\
\glt
\z

village  other=NF  not  dance-PA-1/3p

`Other villages did not dance.'

\ea%x1722
\label{ex:x1722}
\gll Yo  aakisa  \textstyleEmphasizedVernacularWords{opaimika=ko}  ma-i-nen. \\
      \\
\glt
\z

1s.UNM  now  talk/story=NF  say-Np-FU.1s

`Now I will tell a story.'

In most of those few instances where the \textstyleAcronymallcaps{NF} clitic marks a constituent in a clearly realis-type clause, that constituent is a new, indefinite \textstyleAcronymallcaps{NP} introduced as a subject:

\ea%x1733
\label{ex:x1733}
\gll ...\textstyleEmphasizedVernacularWords{emer  en-ow  mua=ko}  eka  en-ep  momor-ar-ep  {\dots} \\
      \\
\glt
\z

sago  eat-NMZ  man=NF  water  eat-SS.SEQ  fool-INCH-SS.SEQ

`{\dots} a Sepik man had drunk beer and became drunk and {\dots}'

\ea%x1732
\label{ex:x1732}
\gll Nan  iimar-ep  ika-eya  \textstyleEmphasizedVernacularWords{urema=ko}  ekap-eya \\
      \\
\glt
\z

there  stand.up-SS.SEQ  be-2/3s.DS  bandicoot=NF  come-2/3s.DS

miim-a-k.

hear-PA-3s

`He was standing there and he heard a bandicoot coming.' (Lit: `{\dots}a bandicoot came and he heard it').

Any constituent in a clause can be marked as focused with the neutral focus clitic. The subject and object have been exemplified above, but a recipient (\stepcounter{nx}{\thenx}), adverbial (\stepcounter{nx}{\thenx}), comitative (\stepcounter{nx}{\thenx}), (\stepcounter{nx}{\thenx}) and instrument (\stepcounter{nx}{\thenx}) are also possible:

\ea%x1723
\label{ex:x1723}
\gll \textstyleEmphasizedVernacularWords{Mua}  \textstyleEmphasizedVernacularWords{nain=ko } onak-e. \\
      \\
\glt
\z

man  that1=NF  feed.him-IMP.1s

`Give it to that man to eat.'

\ea%x1724
\label{ex:x1724}
\gll Miiw-aasa  \textstyleEmphasizedVernacularWords{era=pa=ko}  me  yiar  samor-ar-e-k. \\
      \\
\glt
\z

land-canoe  road=LOC=NF  not  1p.DAT  bad-INCH-PA-3s

`Our truck did not break on the road.'

\ea%x1730
\label{ex:x1730}
\gll Ne  \textstyleEmphasizedVernacularWords{samor  akena=ko}  aruf-a-mik. \\
      \\
\glt
\z

ADD  badly  very=NF  hit-PA-1/3p

`And they beat him \textsc{very badly}.'

\ea%x1725
\label{ex:x1725}
\gll \textstyleEmphasizedVernacularWords{Ikos}\textstyleEmphasizedVernacularWords{=ko}  niir-u. \\
      \\
\glt
\z

together=NF  play-IMP.1d

`Let's play \textsc{together}.'

\ea%x1726
\label{ex:x1726}
\gll \textstyleEmphasizedVernacularWords{Fura}\textstyleEmphasizedVernacularWords{=iw=ko}  me  puuk-a-mik. \\
      \\
\glt
\z

knife=INST=NF  not  cut-PA-1/3p

`They didn't cut it \textsc{with a knife}\textsc{.}'

In a sentence the final, fully inflected verbs are already on the basis of their position more prominent than other verbs, and they cannot receive focus marking. But the medial verbs may be given extra prominence with the neutral focus clitic:

\ea%x1727
\label{ex:x1727}
\gll Amerika  kerer-e-mik  na-i-ya,  \textstyleEmphasizedVernacularWords{ikiw-ep=ko} \\
      \\
\glt
\z

America  appear-PA-1/3p  say-Np-PR.3s  go-SS.SEQ=NF

wia  uruf-ik-ua.

3p.ACC  see-be-PA.3s

`He says that the Americans have arrived, let's \textsc{go} and see them.'

Even the verbal negation particle \textstyleStyleVernacularWordsItalic{me} `not' can be marked with the \textstyleAcronymallcaps{NF} clitic, in which case the focus is on negating the whole proposition.

\ea%x1728
\label{ex:x1728}
\gll Takira  \textstyleEmphasizedVernacularWords{me=ko}  wia  aruf-a-mik. \\
      \\
\glt
\z

boy  not=NF  3p.ACC  hit-PA-1/3p

`It is \textsc{not} the case that we hit the boys.'

A clause can only have one contrastive focus,\footnote{The clauses with a locative adverb \textit{neeke} or \textit{feeke} (\sectref{sec:3.6.3}) are an exception.}  but negations and especially polite requests may contain two or even three constituents marked with the neutral focus:

\ea%x1729
\label{ex:x1729}
\gll \textstyleEmphasizedVernacularWords{No=ko  era=ko  imen-ap=ko}  yia  asip-e. \\
      \\
\glt
\z

2s.UNM=NF  way=NF  find-SS.SEQ=NF  1p.ACC  help-IMP.2s

`If you find a way, please help us.' Or: `Please find a way to help us.' 

Especially in spoken language, it is possible to reduplicate the \textstyleAcronymallcaps{NF} clitic in a word for extra prominence:

\ea%x1731
\label{ex:x1731}
\gll Wi  kema  ma-e-mik,  ``\textstyleEmphasizedVernacularWords{O=ko=ko}  amukar-ek-a-n   \\
      \\
\glt
\z

3p.UNM  liver  say-PA-1/3p  3s.UNM=NF=NF  scold-CNTF-PA-2s  

nom.  Moram  me  amukar-e-n?''

please  why  not  scold-PA-2s

 `They said in their hearts, ``C'mon, you should have scolded \textstyleEmphasizedWords{\textsc{him}}. Why didn't you scold him?'' '

The two focus markers are not mutually exclusive, and consequently they can co-occur in one clause:

\ea%x1737
\label{ex:x1737}
\gll \textstyleEmphasizedVernacularWords{Yos=ke  maa  nain=ko}  me  aaw-e-m. \\
      \\
\glt
\z

1s.FC=CF  thing  that1=NF  not  take-PA-1s

`It wasn't I who took that thing.'

The constituent with focus marking retains the same position in a clause that it has in a non-focused clause. When a personal pronoun in some other case than nominative receives neutral focus, an unmarked pronoun is added as a pronoun copy and marked with the \textstyleAcronymallcaps{NF} clitic:

\ea%x1743
\label{ex:x1743}
\gll Mua  nain  \textstyleEmphasizedVernacularWords{i=ko}  me  \textstyleEmphasizedVernacularWords{yia}  far-e-k. \\
      \\
\glt
\z

man  that1  1p.UNM=NF  not  1p.ACC  call-PA-3s

`The man didn't call \textsc{us}.'

When this pronoun gets also fronted as a theme, it is the pronoun copy that is fronted:

\ea%x1744
\label{ex:x1744}
\gll \textstyleEmphasizedVernacularWords{I}\textstyleEmphasizedVernacularWords{=ko}  mua  nain-(ke)  me  \textstyleEmphasizedVernacularWords{yia}  far-e-k. \\
      \\
\glt
\z

1p.UNM=NF  man  that1-(CF)  not  1p.ACC  call-PA-3s

`\textsc{Us}  the man didn't call.'

Exactly what kind of prominence the \textstyleAcronymallcaps{NF} clitic marks is difficult to pin down, and more research is needed on that. Of the following three examples, (\stepcounter{nx}{\thenx}) is a low-prominence clause with no item marked for extra prominence, in (\stepcounter{nx}{\thenx}) `I' is contrasted with other people, and in (\stepcounter{nx}{\thenx}) the prominence is neutral: the speaker emphasizes that (s)he didn't see, but there is no implied contrast. 

\ea%x1734
\label{ex:x1734}
\gll Yo  me  uruf-a-m. \\
      \\
\glt
\z

1s.UNM  not  see-PA-1s

`I didn't see it.'

\ea%x1735
\label{ex:x1735}
\gll \textstyleEmphasizedVernacularWords{Yos}\textstyleEmphasizedVernacularWords{=ke}  me  uruf-a-m. \\
      \\
\glt
\z

1s.FC=CF  not  see-PA-1s

`It wasn't I who saw it (but someone else).'

\ea%x1736
\label{ex:x1736}
\gll \textstyleEmphasizedVernacularWords{Yo}\textstyleEmphasizedVernacularWords{=ko}  me  uruf-a-m. \\
      \\
\glt
\z

1s.UNM=NF  not  see-PA-1s

`\textit{I}  didn't see it (regardless of whether anyone else did or not)'

Introduction of an indefinite topic has already been mentioned as one of the functions of neutral focus. In questions and requests the focus marking indicates politeness. And especially in many negated clauses with \textstyleAcronymallcaps{NF} marking there is a sense of distancing oneself from the situation. 

\ea%x1738
\label{ex:x1738}
\gll I  \textstyleEmphasizedVernacularWords{mua=ko}  me  wia  furew-a-mik  ne  yiena  pun  \\
      \\
\glt
\z

1s.UNM  man=NF  not  3p.ACC  sense-PA-1/3p  ADD  1p.GEN  also

\textstyleEmphasizedVernacularWords{mukuna}\textstyleEmphasizedVernacularWords{=ko}  me  op-a-mik.

fire=NF  not  hold-PA-1/3p

`We didn't sense any people (there) and we ourselves didn't carry fire either.'

\subsubsection{Other focusing devices}
\hypertarget{RefHeading23961935131865}{}
Cross-linguistically possibly the most common focusing device is stress. Stress in Mauwake is not only a word-level feature (\sectref{sec:2.1.3.1}); it can be employed to give prominence to a word or phrase in a clause. Default, or neutral, clause stress is on the verb or the non-verbal predicate. An extra heavy stress is used for contrastive focus especially for those constituents that seldom or never take \textstyleAcronymallcaps{CF} marking:  

\ea%x1739
\label{ex:x1739}
\gll \textstyleEmphasizedVernacularWords{O}\textstyleEmphasizedVernacularWords{{{\textprimstress}}}\textstyleEmphasizedVernacularWords{wowa=pa}  emeria  unowa  wia  maak-e-mik. \\
      \\
\glt
\z

village=LOC  woman  many  3p.ACC  tell-PA-1/3p

`\textsc{In the village} they told it to many women.'

\ea%x1740
\label{ex:x1740}
\gll Owowa=pa  \textstyleEmphasizedVernacularWords{e}\textstyleEmphasizedVernacularWords{{{\textprimstress}}}\textstyleEmphasizedVernacularWords{meria}  unowa  wia  maak-e-mik. \\
      \\
\glt
\z

village=LOC  woman  many  3p.ACC  tell-PA-1/3p

`In the village they told it to many \textstyleEmphasizedWords{\textsc{women}}.'

\ea%x1741
\label{ex:x1741}
\gll Owowa=pa  emeria  \textstyleEmphasizedVernacularWords{u}\textstyleEmphasizedVernacularWords{{{\textprimstress}}}\textstyleEmphasizedVernacularWords{nowa}  wia  maak-e-mik. \\
      \\
\glt
\z

village=LOC  woman  many  3p.ACC  tell-PA-1/3p

`In the village they told it to \textstyleEmphasizedWords{\textsc{many}} women.'

\ea%x1742
\label{ex:x1742}
\gll Owowa=pa  emeria  unowa  wia  \textstyleEmphasizedVernacularWords{{{\textprimstress}}}\textstyleEmphasizedVernacularWords{maak-e-mik}. \\
      \\
\glt
\z

village=LOC  woman  many  3p.ACC  tell-PA-1/3p

`In the village they \textstyleEmphasizedWords{\textsc{told}} it to many women (instead of hiding it from them).'

Note that in (\stepcounter{nx}{\thenx}) it is only the loudness/intensity in the stressed word that distinguishes it from the neutral clausal stress.

Right-dislocation is often called a topicalizing device, but in Mauwake it can't be that, since only a few of the right-dislocated constituents are topics (\stepcounter{nx}{\thenx}). 

\ea%x1745
\label{ex:x1745}
\gll Maa  nain  aaw-ep  iima=pa  wu-om-ap  \\
      \\
\glt
\z

thing  that1  take-SS.SEQ  chest=LOC  put-BEN-BNFY2.SS.SEQ

om-em-ik-ua,  \textstyleEmphasizedVernacularWords{sawur} \textstyleEmphasizedVernacularWords{} \textstyleEmphasizedVernacularWords{emeria} \textstyleEmphasizedVernacularWords{} \textstyleEmphasizedVernacularWords{nain}\textstyleEmphasizedVernacularWords{=ke}.

cry-SS.SIM-be-3s.PA  spirit  woman  that1=CF

`She took the thing and put it on his chest, the spirit woman (did).'

Most of the right-dislocated elements are not topics. Right-dislocation seems to be a focusing device of a special kind: the speaker decides that some constituent needs clarification or more prominence than it received, and adds it as an afterthought after the clause.

\ea%x1746
\label{ex:x1746}
\gll Saapara=pa  nan  suusa  iw-e-mik,  \textstyleEmphasizedVernacularWords{wiena  ifa} \\
      \\
\glt
\z

Saapara=LOC  there  needle  give.him-PA-1/3p  3p.GEN  snake

\textstyleEmphasizedVernacularWords{suusa  nain}.

needle  that1

`There in Saapara they gave him an injection, their snake (antivenene) injection.'

\ea%x1747
\label{ex:x1747}
\gll Aaya  puuk-ap  iimar-ep  ik-ua,  \\
      \\
\glt
\z

sugarcane  change.into-SS.SEQ  stand.up-SS.SEQ  be-PA.3s  

\textstyleEmphasizedVernacularWords{manin(a)} \textstyleEmphasizedVernacularWords{} \textstyleEmphasizedVernacularWords{afua=pa}.

garden  old(garden)=LOC

`It had changed into a sugarcane and was standing in the old garden.'

\ea%x1748
\label{ex:x1748}
\gll Ne  fraide-pa  maapora  puk-o-k,  \textstyleEmphasizedVernacularWords{urera}. \\
      \\
\glt
\z

ADD  Friday  party  burst-PA-3s  afternoon

`And on Friday the party started, in the afternoon.'


\section*{Appendix 1: List of main texts used}


The texts marked with an asterisk (*) appear interlinearised in Appendix 2.
{(S) indicates spoken, (W) written text.}

\begin{tabular}{llllll}
No &  Name & Code & Type & Author & Sent./Cl\\
1 & Uncle Tup* &  NASRAB & Narrative (S) &  Saror Aduna &  65/146\\
2 & Turtle*  & NAECAB & Narrative (S) & Kuuten  & 9/33 \\
3 & World War 2*  & HIKCCH & Narrative (S) & Kalina Sarak  & 134/336 \\
4 & Boika's kunai  & NASRGG & Narrative (S) & Saror Aduna  & 62/149 \\
5 & Dog and snake  & NASCAI & Narrative (S) & Saror Aduna  & 13/30 \\
6 & School party  & NASRBI & Narrative (S) & Saror Aduna  & 56/101 \\
7 & Piglet  & NASREJ & Narrative (S) & Saror Aduna  & 31/90 \\
8 & Komori's wife  & NACRDC & Narrative (S) & Saror Aduna  & 30/57 \\
9 & Man's lover  & LEHCAN & Trad.story (S) & Kinangir Saror  & 16/50 \\
10 & Flood  & LEYREG & Trad.story (S) & Yaura  & 33/89 \\
11 & Eel  & LESW09 & Trad.story (W) & Saror Aduna  & 49/144 \\
12 & Rubaruba  & LEHCAO & Trad.story (S) & Kinangir Saror  & 63/201 \\
13 & Boy and flies  & LESCAM & Trad.story (S) & Saror Aduna  & 20/62 \\
14 & Suun story  & LEPRDA & Trad.story (S) & Paul  & 28/89 \\
15 & Marus' wedding  & HOURGC & Hortatory (S) & Kululu Sarak  & 116/274 \\
16 & Copra  & DESW01 & Process (W) & Saror Aduna  & 12/37 \\
17 & Garden work  & DESW02 & Process (W) & Saror Aduna  & 35/68 \\
18 & Fishing customs*  & DESW03 & Process (W) & Saror Aduna  & 59/116 \\
19 & Headdress  & DESW15 & Process (W) & Saror Aduna  & 17/38 \\
20 & Pighunt  & DEWCCA & Process (W) & Muandilam  & 31/100 \\
21 & Girls' initiation  & DEKRDE & Descriptive (S) & Kalina Sarak  & 21/58 \\
22 & Funeral customs  & DEKRDF & Descriptive (S) & Kalina Sarak  & 41/109 \\
23 & Adoption  & DESW14 & Descriptive (W) & Saror Aduna  & 40/89 \\
24 & My family  & DESCAH & Descriptive (S) & Saror Aduna  & 23/38 \\
25 & Dreams  & DESW16 & Descriptive (W) & Saror Aduna  & 15/30 \\
26 & Tidal wave  & DESRBK & Descriptive (S) & Saror Aduna  & 17/36 \\
\end{tabular}