%5

\chapter{Clause}
\hypertarget{RefHeading21901935131865}{}
A clause,\footnote{I use the separate terms \textit{clause} and \textit{sentence} to avoid confusion.  A simple sentence consists of just one clause, but most of the sentences in Mauwake have more than one clause in either coordinate, chaining or subordinate relationship.} or simple sentence, typically expresses one predication and is a minimal utterance that can stand alone.

In Mauwake the predicate is the only obligatory element in those clauses that have a verbal predicate. Verbless clauses need to have both an overt subject and a predicate. The different clause types are discussed in {\S}\ref{sec:5.3}-5.6.

Instead of the common two-way distinction between main and subordinate clauses, in Trans New Guinea languages it is practical to talk about main, medial and subordinate clauses. Main clauses have a finite verb, and most commonly it is the last element in a sentence. Medial clauses (\sectref{sec:8.2.1}) are coordinate with the main clauses but dependent on them, and the verbs are in medial form (\sectref{sec:3.8.3.4}). The default position for a medial clause is non-final, but for pragmatic purposes it may be postposed to follow the main clause.  Also a subordinate clause (\sectref{sec:8.3}) usually precedes the main clause. 

\section{Order of constituents}
\hypertarget{RefHeading21921935131865}{}
Two seemingly conflicting statements about the clausal constituent order in Papuan languages have been given by \citet{Wurm1982} and \citet{Foley1986}. \citet[64]{Wurm1982} maintains that they have a rigid \textstyleAcronymallcaps{SOV} order; \citet[167]{Foley1986} claims that the order in most Papuan languages is relatively free, and therefore he prefers to call them just verb-final (ibid. 10). But it seems that the two linguists are talking about somewhat different things, and both of them are correct in what they say.  The \textstyleEmphasizedWords{\textsc{default}} constituent order in neutral sentences is \textstyleAcronymallcaps{SOV}, as Wurm claims, but Foley is right in that the \textstyleEmphasizedWords{\textsc{interpretation}}\textstyleEmphasizedWords{} of the arguments of a verb as subject or object does not rely heavily on the constituent order. Especially in languages with extensive verb morphology marking the syntactic roles on the verb itself the order of the nominals can be relatively free, and is mainly constrained by pragmatic factors. 

The basic constituent order in Mauwake clauses is quite rigid \textstyleAcronymallcaps{SOV}, even if the verb morphology cross-references the syntactic roles to some extent. Although only a fraction of the clauses in the text corpus -- less than 10\% -- have an overt subject and object \textstyleAcronymallcaps{NP,} it is possible to establish the dominant order. About nine out of ten of those clauses that do have an overt subject and object \textstyleAcronymallcaps{NP} manifest \textstyleAcronymallcaps{SOV} order.\footnote{SOV: 210 clauses, OSV: 22 clauses} They are also pragmatically neutral (\stepcounter{nx}{\thenx}), whereas the other possible order, \textstyleAcronymallcaps{OSV}, only occurs when the object is fronted as a theme (\stepcounter{nx}{\thenx}). 

\ea%x896
\label{ex:x896}
\gll [Owow  mua]\textsubscript{S}  [kau  kuisow]\textsubscript{O}  aaw-e-mik. \\
     village  man  cow  one  get-PA-1/3p \\
\glt `The village men got one cow.'
\z
 

\ea%x897
\label{ex:x897}
\gll [Yena  aamun]\textsubscript{O}  [ariwa=ke]\textsubscript{S}  aaw-o-k. \\
      \\
\glt
\z

1s.POSS  1s/p.younger.sibling  arrow=CF  get-PA-3s

`My younger brother was killed by an arrow.'

As was described in \sectref{sec:1.4.2.2}, Mauwake exhibits many typological characteristics associated with \textstyleAcronymallcaps{SOV} languages.

The basic constituent order is always based on the structure of a transitive clause. Intransitive clauses (\sectref{sec:5.3}) do not have objects, but otherwise the structure is the same as that in the transitive clauses. The structure of other types of clauses is described in the relevant sections.

The constituent order in an extended predication is harder to establish, because a clause typically has very few constituents, the average being only 1.2 non-verb constituents per clause; because any non-verbal element can be fronted as topic; and because the subject is often shown only by a verbal suffix and the object by an accusative pronoun in the \textstyleAcronymallcaps{VP}. A clause formula for a maximally extended predication is hypothetical, and mainly shows the order of the constituents on the basis of their attested orders in relation to each other: 

S    X\textsubscript{1}  O\textsubscript{1}  X\textsubscript{2}  O\textsubscript{2}  X\textsubscript{3}  V

There are two object positions\footnote{The two objects are discussed further in the next sections 5.2 and 5.3.}  and three X-positions for adverbial phrases.\footnote{Depending on the grammatical model, these may be called peripherals, obliques, satellites or adjuncts. I call them \textit{peripherals} and reserve the term \textit{adjunct} for the non-verb part of an adjunct plus verb construction.}  If a clause has only one object, it occupies the O\textsubscript{2} position immediately preceding the verb regardless of whether the semantic function is that of a patient, a recipient or a beneficiary. When there are two objects, their position is dictated mainly by their relative topicality. A [+human] argument tends to be more topical than a [\nobreakdash-human] one, so an object that is semantically a recipient (\stepcounter{nx}{\thenx}), (\stepcounter{nx}{\thenx}), or a beneficiary (\stepcounter{nx}{\thenx}), typically occupies the first object position, and the other object, typically a [\nobreakdash-human] patient, fills the second object position. 

\ea%x928
\label{ex:x928}
\gll [Muuka]\textsubscript{O1}  [sira]\textsubscript{O2}  iw-i-mik. \\
      \\
\glt
\z

son  custom  give.him-Np-PR.1/3p

`They teach the right behaviour to the son.'

\ea%x933
\label{ex:x933}
\gll Sarak=ke  [wi  takira]\textsubscript{O1}  [inglis]\textsubscript{O2}  [wia]\textsubscript{O1}  ofakow-i-ya.{\footnotemark} \\
      \\
\glt
\z

Sarak=CF  3p.UNM  child  English  3p.ACC  teach-Np-PR.3s

`Sarak teaches the children English.'

\footnotetext{Compare this with: \textit{Sarak=ke inglis wia ofakowiya} `Sarak teaches (them) English' and \textit{Sarak=ke wi takira wia ofakowiya} `Sarak teaches the children.' Both the recipient and the patient are coded in the same way as an object.}

\ea%x916
\label{ex:x916}
\gll Ni  [auwa]\textsubscript{O1}  [maa]\textsubscript{O2}  p-ikiw-om-aka.  \\
      \\
\glt
\z

2p.UNM  1s/p.father  food  BPx-go-BEN-BNFY2.IMP.2p

`Take food to/for father.'

If a [-human] patient object is more topical than a [+human] object, it can occupy the first object position. A more topical [+human] object in (\stepcounter{nx}{\thenx}) would have an unmarked third person plural pronoun before the [-human] object.\footnote{More examples of can be found in \sectref{sec:5.3.2}-5.3.4.}

\ea%x917
\label{ex:x917}
\gll Onak=ke  [aaya]\textsubscript{O1}  [wia]\textsubscript{O2}  aaw-om-aya  \\
      \\
\glt
\z

3s/p.mother  sugarcane  3p.ACC  get-BEN-BNFY2.2/3s  

enim-or-om-ik-e-mik.

eat-descend-SS.SIM-be-PA-1/3p

`Their mother got sugarcane for them and they went down eating it.'

If both the objects are [-human], the one that is more clearly the patient, i.e. more profoundly affected by the action, occupies the O\textsubscript{2} position. Usually the object in O\textsubscript{1} position has a more locative-type meaning.

\ea%x927
\label{ex:x927}
\gll [Epira]\textsubscript{O1}  [lolom]\textsubscript{O2}  if-e-mik. \\
      \\
\glt
\z

plate  mud  smear-PA-1/3p

`They smeared the plate with mud' or: `They smeared mud on the plate.'

\ea%x934
\label{ex:x934}
\gll [Wut  makena  nain]\textsubscript{O1}  [ona]\textsubscript{O2}  puuk-a-m. \\
      \\
\glt
\z

Derris.root.tree  seed  that1  hole  cut-PA-1s

`I cut a hole in the seed of a derris root tree.'

The normal position of the peripherals is between the subject and the object \textstyleAcronymallcaps{NP,} if any (\stepcounter{nx}{\thenx}), or between the first and second object (\stepcounter{nx}{\thenx}). 

\ea%x895
\label{ex:x895}
\gll Yo  \textstyleEmphasizedVernacularWords{uura}  arua  isim-ap  ... \\
      \\
\glt
\z

1s.UNM  night  torch  light-SS.SEQ

`I lighted a torch in the night and ...'

\ea%x913
\label{ex:x913}
\gll [Wiipa  nain]\textsubscript{O}1  [\textstyleEmphasizedVernacularWords{samapora  iinan=pa}]\textsubscript{AdvP}  [epia]\textsubscript{O}2  \\
      \\
\glt
\z

daughter  that  floor  top=LOC  fire  

ururum-om-ap{\dots}

light-BEN-BNFY2.SS.SEQ

`They lighted a fire for the daughter on top of the floor, and ...'

A locative adverbial can also come between an object \textstyleAcronymallcaps{NP} and a verb. A deictic locative phrase or another short locative phrase is common in this position:

\ea%x914
\label{ex:x914}
\gll Emer  en-ow(a)  mua=ko  [emeria]\textsubscript{O}  [\textstyleEmphasizedVernacularWords{fan}]\textsubscript{AdvP}  aaw-o-k. \\
      \\
\glt
\z

sago  eat-NMZ  man=NF  woman  here  get-PA-3s

`A Sepik man got a wife here.'

\ea%x931
\label{ex:x931}
\gll Yo  [maa  unowa]\textsubscript{O}  [\textstyleEmphasizedVernacularWords{koora=pa}]\textsubscript{AdvP}  wu-a-m. \\
      \\
\glt
\z

1s.UNM  thing  many  house=LOC  put-PA-1s

`I put (the) many things in the house.'

The position immediately before the verb is also the only possible place for a [+human] locative adverbial, manifested by a dative pronoun (\sectref{sec:3.5.5}). In both (\stepcounter{nx}{\thenx}) and (\stepcounter{nx}{\thenx}) there are two locative adverbials, a [-human] and a [+human] one. The [+human] locative adverbial refers to the people of the location. If it is left out, the other locative refers to the location but not the people. 

\ea%x854
\label{ex:x854}
\gll [Ni  [koka-pa]\textsubscript{AdvP}  [\textstyleEmphasizedVernacularWords{wiar}]\textsubscript{AdvP} \textstyleEmphasizedVernacularWords{} in-em-ik-e-man  \\
      \\
\glt
\z

2p.UNM  jungle=LOC  3.DAT  sleep-SS.SIM-be-PA-2p  

nain]\textsubscript{RC} kerer-omak-eka.

that1 appear-DISTR/PL-IMP.2p

`You(pl.) who have slept in the jungle (villages), come!' 

\ea%x855
\label{ex:x855}
\gll I  amirk=iw  [Gawar]\textsubscript{AdvP}  [\textstyleEmphasizedVernacularWords{wiar}]\textsubscript{AdvP} \textstyleEmphasizedVernacularWords{} urup-e-mik. \\
      \\
\glt
\z

1p.UNM  day=INST  Gawar  3.DAT  ascend-PA-1/3p

`During the day we went to Gawar.'

If there are more adverbial phrases than one, a temporal phrase normally precedes any others (\stepcounter{nx}{\thenx}).  The relative order of the other adverbial phrases is syntactically quite free and depends on their relative topicality.

\ea%x915
\label{ex:x915}
\gll I  \textstyleEmphasizedVernacularWords{amirika  owowa  ewur}  me  ekap-em-ik-e-mik. \\
      \\
\glt
\z

1p.UNM  day  village  quickly  not  come-SS.SIM-be-PA-1/3p

`In the daytime we didn't come quickly to the village.'

\ea%x918
\label{ex:x918}
\gll Niena  \textstyleEmphasizedVernacularWords{ikoka  oram  neeke}  ika-i-non. \\
      \\
\glt
\z

2s/p.mother  later  for.nothing  there.CF  be-Np-FU.3s

`Your mother will later just be there (without you).'

\ea%x919
\label{ex:x919}
\gll \textstyleEmphasizedVernacularWords{Mokoma  kuisow}  \textstyleEmphasizedVernacularWords{naap  fan  yiam=iya}  ik-e-mik. \\
      \\
\glt
\z

year  one  thus  here  1p.REFL=COM  be-PA-1/3p

`They were here with us for about a year.' 

Both transitive and intransitive clauses are negated with the verbal negator \textstyleStyleVernacularWordsItalic{me} `not' placed immediately before the verb phrase.\footnote{For the placement of \textit{me} as a constituent negator, see \sectref{sec:6.2.2}.} 

\ea%x981
\label{ex:x981}
\gll I  \textstyleEmphasizedVernacularWords{me}  wia  amukar-e-mik. \\
      \\
\glt
\z

1p.UNM  not  3.ACC  scold-PA-1/3p

`We didn't scold them.'

\ea%x1057
\label{ex:x1057}
\gll Nain  yo  \textstyleEmphasizedVernacularWords{me}  baurar-em-ik-e-m. \\
      \\
\glt
\z

but  1s.UNM  not  run.away-SS.SIM-be-PA-1s

`But I didn't keep running away.'

As was mentioned above, pragmatic factors influence the constituent order.  A constituent that is fronted as a theme to the beginning of the clause is still part of the constituent structure of the clause (for theme, see \sectref{sec:9.1}).\footnote{In Amele the pre-verbal position is a focus position \citep[142]{Roberts1987}, but in Mauwake this does not seem to be the case. Focus is indicated by a heavier stress and sometimes by focus markers.} 

\ea%x930
\label{ex:x930}
\gll [\textstyleEmphasizedVernacularWords{Oposia}  \textstyleEmphasizedVernacularWords{gelemuta}]\textsubscript{O}1  [wiam  erup  fain  wia]\textsubscript{O}2  \\
      \\
\glt
\z

meat  little  3p.REFL  two  this  3p.ACC  

wu-om-a-m.

put-BEN-BNFY2.PA-1s

`A bit of the meat I put (aside) for these two.'

A left-dislocated theme (\sectref{sec:9.1}) and an afterthought are outside the clause proper.  A left-dislocated theme (\stepcounter{nx}{\thenx}) is separated from the clause by a short pause and a comma intonation, slightly rising pitch at the end of the utterance.  An afterthought (\stepcounter{nx}{\thenx}), right-dislocated, is also separated from the rest of the clause by a short pause.

\ea%x935
\label{ex:x935}
\gll \textstyleEmphasizedVernacularWords{Irak-owa  fa},  opora  unowa  akena. \\
      \\
\glt
\z

fight-NMZ  EXC  talk  much  very

`The war, now - there is much to talk about.'

\ea%x929
\label{ex:x929}
\gll Inasin  opaimika  eliwa  me  yia  maak-e-mik,  \textstyleEmphasizedVernacularWords{wi} \\
      \\
\glt
\z

spirit  talk  good  not  1p.ACC  tell-PA-1/3p  3p.UNM  

\textstyleEmphasizedVernacularWords{Yaapan=ke}.

Japan=CF

`They didn't speak good Pidgin to us, the Japanese (didn't).'

\section{Syntactic arguments}
\hypertarget{RefHeading21941935131865}{}
Syntactic arguments together with the verb form the core of a clause. They differ from the peripherals in that they have a grammatical relation to the verb (Foley and Van Valin 1984:77-80), and therefore have to do with the valence of the clause. The basic syntactic structure is influenced by the arguments but not by the peripherals.  In Mauwake the only syntactic arguments are subject and object. 

Since Mauwake is very clearly a nominative-accusative type language, the grammatical role of \textstyleEmphasizedWords{\textsc{subject}}\textstyleEmphasizedWords{\textsc{} }and the semantic role of \textstyleEmphasizedWords{\textsc{agent}}\textstyleEmphasizedWords{\textsc{} }or\textstyleEmphasizedWords{\textsc{} }\textstyleEmphasizedWords{\textsc{actor}} normally converge on the same constituent, which usually, but not always, also has the pragmatic role of \textstyleEmphasizedWords{\textsc{topic}}. Another semantic role the subject may have is that of \textstyleEmphasizedWords{\textsc{experiencer}}, and in verbless clauses that of ``\textstyleEmphasizedWords{\textsc{theme}}''\footnote{This semantic role ``theme'' is different from the pragmatic function and refers to the participant which is said to be in some state, or located in some place \citep[140]{Andrews2007a}. Because of a possible confusion with the pragmatic role of theme, the term for the semantic role is written inside double quotes.}.

The syntactic coding of the subject includes both the clausal constituent order and cross-referencing on the verb.  In pragmatically neutral clauses the subject is the first of two argument noun phrases.  It is also obligatorily marked on the person/number suffix of the verb. The same distinctions are made in the subject marking of the verb as in the personal pronouns: first, second or third person and singular or plural number.

\ea%x936
\label{ex:x936}
\gll Komori  emeria  wu-a-k. \\
      \\
\glt
\z

Komori  woman  put-PA-3s

`Komori buried his wife.'

The subject governs reflexivization.  A noun phrase itself is marked as subject only when the subject \textstyleAcronymallcaps{NP} is a pronoun: then it has to be unmarked (\sectref{sec:3.5.2.1}) or in the genitive case (\sectref{sec:3.5.4}).

\ea%x920
\label{ex:x920}
\gll Tirinde  uura  \textstyleEmphasizedVernacularWords{i}  nainiw  \textstyleEmphasizedVernacularWords{yiam}  fiirim-e-\textstyleEmphasizedVernacularWords{mik}. \\
      \\
\glt
\z

Wednesday  evening  1p.UNM  again  1p.REFL  gather-PA-1/3p

`On Wednesday evening we gathered again.'

The switch-reference system (\sectref{sec:8.2.3}) basically tracks the subject.\footnote{Roberts maintains that in Amele and most other Papuan languages the switch reference system tracks the thematic notion of topic across clauses (1988b:105, 1997). But his definition of topic (1988b:96) is such that in Mauwake it practically excludes all other clause constituents except the subject.} 

\ea%x921
\label{ex:x921}
\gll {\dots}imen-\textstyleEmphasizedVernacularWords{ap}  maak-\textstyleEmphasizedVernacularWords{iwkin}  o  miim-o-k. \\
      \\
\glt
\z

find-SS.SEQ  tell-2/3p.DS  3s.UNM  precede-PA-3s

`{\dots}when they found him and told him, he went ahead.'

However, in the following example, the initial unmarked pronoun \textstyleStyleVernacularWordsItalic{wi}  pluralizes the object/theme, the Australians; in the second clause the Australians are the subject.

\ea%x922
\label{ex:x922}
\gll Wi  Australia  \textstyleEmphasizedVernacularWords{Amerika=ke}  wia  asip-\textstyleEmphasizedVernacularWords{iwkin} \\
      \\
\glt
\z

3p.UNM  Australia  America=CF  3p.ACC  help-2/3p.DS

irak-owa  nomak-e-mik.

fight-NMZ  win-PA-1/3p

`Australians were helped by Americans and won the war.'

Syntactic operations like passivization and dative shift do not apply to Mauwake and consequently cannot be used to define either the subject or other syntactic arguments. 

Although the prototypical subject is a [+human] agent, also an instrument that is unable to initiate an action can become a metaphorical agent, thus the subject (Giv\'on 1984:106). This is very common in expressions describing cases where one involuntarily hurts oneself with some instrument. (In the following example there is also possessor raising (\sectref{sec:5.3.2.3}) resulting in two objects.)

\ea%x958
\label{ex:x958}
\gll [\textstyleEmphasizedVernacularWords{Fura=ke}]\textsubscript{S}  [merena]\textsubscript{O1}  [efa]\textsubscript{O2}  puuk-a-k. \\
      \\
\glt
\z

knife=CF  leg  1s.ACC  cut-PA-3s

`I cut my leg / myself in the leg.' (Lit: `A knife cut my leg.')

Because the subject is so often marked by only a verbal suffix, it would be possible to treat the subject marking on the verb as the real subject, as Van Valin and LaPolla suggest for those languages that mark the core arguments on the verb (1997:33-34). Although this approach would have some advantages,\footnote{The main advantage would be having an overt subject in every clause, regardless of the presence of a separate subject NP.} I choose the more traditional way of treating the \textstyleAcronymallcaps{NP} as the subject, both because 1) an object is not marked in the verb inflection but requires either a separate \textstyleAcronymallcaps{NP} or an accusative pronoun outside the verb proper,\footnote{The small group of object cross-referencing verbs (\sectref{sec:3.8.4.2.4}) are an exception.} and because 2) the constituent order, based on the position of the subject and object \textstyleAcronymallcaps{NP}s, has many interconnections with various parts of the syntax. 

The \textstyleEmphasizedWords{\textsc{object}} as a syntactic role is not coded on the verb word except in the few object cross-referencing verbs (\sectref{sec:3.8.4.2.4}). A [+human] object must be referenced by an accusative pronoun (\sectref{sec:3.5.3}) preceding the verb in the verb phrase (\sectref{sec:4.5}), even if the object is also expressed by a full noun phrase earlier in the clause (\stepcounter{nx}{\thenx}). The position of the object \textstyleAcronymallcaps{NP} in the argument structure of the clause is between the subject and the verb (\stepcounter{nx}{\thenx}), but this syntactic definition is not very useful, as subject and object do not co-occur very often, and sometimes when they do, the object is fronted to the clause-initial theme position (\stepcounter{nx}{\thenx}). 

\ea%x923
\label{ex:x923}
\gll \textstyleEmphasizedVernacularWords{Emeria}  naap  \textstyleEmphasizedVernacularWords{wia}  aruf-i-nen  na-ep  on-a-k. \\
      \\
\glt
\z

woman  thus  3p.ACC  hit-Np-FU.1s  say-SS.SEQ  do-PA-3s

`He tried to hit women that way.'

\ea%x924
\label{ex:x924}
\gll Amia  mua=ke  \textstyleEmphasizedVernacularWords{wiam  erup  nain  wia}  nokar-e-k. \\
      \\
\glt
\z

bow  man=CF  3p.REFL  two  that1  3p.ACC  ask-PA-3s

`The policeman asked those two.'

\ea%x925
\label{ex:x925}
\gll \textstyleEmphasizedVernacularWords{Mua  emeria  muuka  wiipa}  eka=ke  \textstyleEmphasizedVernacularWords{wia}  mu-o-k. \\
      \\
\glt
\z

man  woman  son  daughter  river=CF  3p.ACC  swallow-PA-3s

`A man and his wife and children were drowned by the river.'

There is not enough basis in Mauwake for positing a separate syntactic category \textstyleEmphasizedWords{\textsc{indirect object}}. In many languages the most typical verb requiring an indirect object for the semantic role of recipient is `give'.  But in Mauwake the verbs `give' and `feed/give to eat' are among the few object cross-referencing verbs (\sectref{sec:3.8.4.2.4}), which change their stem according to the patient or recipient object. The verb `send', another cross-linguistically typical verb taking an indirect object, in Mauwake requires the benefactive suffix on the verb (\sectref{sec:3.8.4.3.2}), rather than a marking on the \textstyleAcronymallcaps{NP}.  In (\stepcounter{nx}{\thenx}) the verb \textstyleStyleVernacularWordsItalic{maak}- `tell' has two objects, the patient object \textstyleStyleVernacularWordsItalic{moma} `taro' and the recipient object \textstyleStyleVernacularWordsItalic{yia} `us', which is marked by an accusative pronoun in the \textstyleAcronymallcaps{VP} in the same way as a [+human] undergoer/patient. 

\ea%x932
\label{ex:x932}
\gll Wi  [\textstyleEmphasizedVernacularWords{moma}]\textsubscript{O}  [\textstyleEmphasizedVernacularWords{yia}]\textsubscript{O}  maak-i-mik. \\
      \\
\glt
\z

3p.UNM  taro  1p.ACC  tell-Np-PR.1/3p

`They are telling us (to get them) taro.'

This is consistent with Whitehead's (1981:51) survey results showing that ``a large number of [Papuan] languages ... do not differentiate between Patient and Recipient''. Rather, there are verbs that are capable of taking two objects (ibid. 52).\footnote{Usan behaves in the same way as Mauwake \citep[160]{Reesink1987}.} Amele is one of those Papuan languages that clearly have indirect object as a syntactic category \citep[69]{Roberts1987}.

It could be argued that a locative adverbial is an argument rather than a peripheral with the directional verbs\footnote{For Usan, where motion verbs have either the goal or locative as a nuclear argument, see \citet[130]{Reesink1987}.} (\sectref{sec:3.8.4.4.5}) and with the verb \textstyleStyleVernacularWordsItalic{ik}- `be/live (somewhere)' (\sectref{sec:3.8.4.4.1}), as these verbs so often co-occur with a locative. But these verbs also occur without a locative so often that it would be both unnatural to interpret all of those instances as elliptical constructions and sometimes difficult to posit the ``deleted'' locative. 

\ea%x1456
\label{ex:x1456}
\gll Maak-e-mik,  {\textquotedbl}No  ikiw-e,  irak-owa  maneka \\
      \\
\glt
\z

tell-PA-1/3p  2s.UNM  go-IMP.2s  fight-NMZ  big  

fan-e-k  a.''

here-PA-3s  INTJ

`They told him, ``Go, the war is here.'' '

\ea%x1457
\label{ex:x1457}
\gll Iwera  uruk-am-ika-iwkin  wi  ikiw-emi \\
      \\
\glt
\z

coconut  drop-SS.SIM-be-2/3p.DS  3p.UNM  go-SS.SIM  

aaw-em-ik-e-mik ...

get-SS.SIM-be-PA-1/3p

`When they\textsubscript{i} kept dropping coconuts, they\textsubscript{j} kept going and getting them {\dots}'

The verb \textstyleStyleVernacularWordsItalic{ik}- `be' seldom occurs alone (\stepcounter{nx}{\thenx}), repeated below as (\stepcounter{nx}{\thenx}). This is probably due to its very neutral semantic character. When it denotes being or living somewhere, it is accompanied by a locative adverbial phrase (\stepcounter{nx}{\thenx}). Another very common adverbial phrase accompanying the verb is a manner phrase, especially the adverb \textstyleStyleVernacularWordsItalic{naap} `thus' (\stepcounter{nx}{\thenx}). Rather than positing separate clause types with adverbials as arguments it seems reasonable to subsume clauses like these under intransitive clauses.

\ea%x1458
\label{ex:x1458}
\gll Ika-i-nen. \\
      \\
\glt
\z

be-Np-FU.1s

`I will just be like this.'

\ea%x1459
\label{ex:x1459}
\gll Siiwa  erepam  naap  \textstyleEmphasizedVernacularWords{nan  ik-ok}  napuma  sariar-e-k. \\
      \\
\glt
\z

moon  four  thus  there  be-SS  sickness  heal-PA-3s

`He was there about four months and his sickness was healed.'

\ea%x1460
\label{ex:x1460}
\gll Komor(a)  muuka  nain  memel-am-ik-emkun \\
      \\
\glt
\z

cuscus  son  that1  tame-SS.SIM-be-1s/p.DS

\textstyleEmphasizedVernacularWords{naap  ik}\textstyleEmphasizedVernacularWords{-}\textstyleEmphasizedVernacularWords{ok}  iir  oko  uura  baurar-e-k.

thus  be-SS  time  other  night  escape-PA-3s

`I was taming the cuscus and it was like that and then one night it escaped.'

\section{Transitive clauses}
\hypertarget{RefHeading21961935131865}{}
Transitivity is an important characteristic of both a verb and a clause; which of these is primary has been an object of a great deal of discussion.\footnote{In transformational grammar (EST) verbs had to be subcategorized in the lexicon according to whether they allowed a NP-object or not \citep[120]{Radford1981}. Also Van Valin and La\citet[147-157]{Polla1997} consider transitivity essentially a characteristic of a verb, distinguishing between the semantic, syntactic and macrorole transitivity of each verb. Giv\'on (1995:76), Kittil\"a (2002:25) and \citet[115]{Dixon2010}, among others, maintain that transitivity is primarily a characteristic of a whole clause. Taking still another angle, Hopper and \citet[294]{Thompson1980} claim that transitivity is very closely bound with discourse features, namely background and foreground.} This may be language-specific. In languages like English where an intransitive verb like \textstyleForeignWords{sneeze} can be made transitive in a construction \textstyleForeignWords{He sneezed the napkin off the table} \citep[9]{Goldberg1995}, it makes sense to say that the verb combines with a transitive argument structure construction \citep[6]{Goldberg2006}. But in Mauwake it can be claimed that the transitivity of the verb is primary. The claim is supported by the clear distinction between transitive and intransitive verbs and the fact that transitive verbs often require a dummy object when there is no real object available.

Clauses are here looked at from the point of view of \textstyleEmphasizedWords{\textsc{syntactic transitivity}}: clauses that have an overt object are treated as transitive clauses, regardless of the semantic role of the object. 

Linguistically the most interesting transitive clauses are those that have two or possibly even three objects. These can be divided into three different groups: clauses where a transitive verb can take more than one object without requiring any morphological or syntactic operation (\sectref{sec:5.3.2.1}) and those where an object has been added by a valence-increasing operation (\sectref{sec:5.3.2.2}) or by possessor raising (\sectref{sec:5.3.2.3}).

\subsection{Monotransitive clauses}
\hypertarget{RefHeading21981935131865}{}
Monotransitive clauses have a transitive verb and one object, which is prototypically a patient. 

\ea%x937
\label{ex:x937}
\gll [Sawur  emeria  nain=ke]\textsubscript{S}  [\textstyleEmphasizedVernacularWords{ona  soma  mua  nain}]\textsubscript{O}  ifakim-o-k. \\
      \\
\glt
\z

spirit  woman  that1=CF  3s.GEN  lover  man  that1  kill-PA-3s

`The spirit woman killed her lover.'

\ea%x939
\label{ex:x939}
\gll Amirika  [i]\textsubscript{S}  [\textstyleEmphasizedVernacularWords{maa  eneka  fain}]\textsubscript{O}  uup-ep  \\
      \\
\glt
\z

noon  1p.UNM  thing  tooth  this  cook-SS.SEQ 

enim-i-yen.

eat-Np-FU.1p

`At noon we'll cook and eat this (edible) animal.'

If there is only one \textstyleAcronymallcaps{NP} argument in a transitive clause, it is usually the object rather than subject (\stepcounter{nx}{\thenx}), unless marked with the contrastive focus marker -\textstyleStyleVernacularWordsxiiptItalic{ke} (\stepcounter{nx}{\thenx}).

\ea%x941
\label{ex:x941}
\gll [\textstyleEmphasizedVernacularWords{Wiawi}]\textsubscript{O}  kuum-eya  aw-ep  eka  iw-a-k=na  wia. \\
      \\
\glt
\z

3s/p.father  burn-2/3.DS  burn-SS.SEQ  river  go-PA-3s=TP  no

`(It) burned their father and when he burned he went into the river but it didn't help.'

\ea%x938
\label{ex:x938}
\gll Ufer-iwkin  urup-em-ik-eya  [yos=ke]\textsubscript{S}  mik-a-m. \\
      \\
\glt
\z

miss-2/3p.DS  ascend-SS.SIM-be-2/3s.DS  1s.FC=CF  shoot-PA-1s

`When they missed and it was going up, I shot it.'

But in the following clause the contrastive focus marker is not needed to disambiguate the subject from the object: \textstyleStyleVernacularWordsItalic{wiam arow} `the three of them' has to be the subject; if it were the object, it would require the third person plural accusative pronoun \textstyleStyleVernacularWordsItalic{wia} in the \textstyleAcronymallcaps{VP}.

\ea%x940
\label{ex:x940}
\gll [Ne  wiam  arow]\textsubscript{S}  miim-ap  {\dots} \\
      \\
\glt
\z

and  3p.REFL  three  hear-SS.SEQ

`And the three of them heard it, and {\dots}'

Clauses with an impersonal experience verb (\sectref{sec:3.8.4.4.7}) as the predicate are also transitive. The subject is inanimate, usually a body part where the pain is felt, and the human experiencer is the object. The possibility of adding the contrastive focus clitic to the noun indicating body part shows that it is the subject rather than a second object (\stepcounter{nx}{\thenx}).

\ea%x1012
\label{ex:x1012}
\gll Uuw-ap  uuw-ap  [oona=ke]\textsubscript{S}  [efa]\textsubscript{O}  sirir-i-ya. \\
      \\
\glt
\z

work-SS.SEQ  work-SS.SEQ  bone=CF  1s.ACC  hurt-Np-3s

`I have worked and worked, and my bones hurt (me).'

\ea%x1015
\label{ex:x1015}
\gll Yo  [uroma]\textsubscript{S}  [efa]\textsubscript{O}  op-am-ik-eya  yo  haussik  \\
      \\
\glt
\z

1s.UNM  stomach  1s.ACC  hold-SS.SIM-be-2/3s.DS  1s.UNM  aidpost

me  ikiw-e-m.

not  go-PA-1s

`I was having birth pains (lit: My stomach was holding/grabbing me) but I did not go to the aidpost.'

\subsection{Ditransitive clauses}
\hypertarget{RefHeading22001935131865}{}
A number of ditransitive clauses (\stepcounter{nx}{\thenx})-(\stepcounter{nx}{\thenx}) were already listed under \sectref{sec:5.1}. They belong to the three different groups below.

\subsubsection[Inherent ditransitivity ]{Inherent ditransitivity} 
\hypertarget{RefHeading22021935131865}{}
Some  ditransitive clauses are called inherently ditransitive, because they do not require a morphological or syntactic process to make them ditransitive. The most common verbs in ditransitive clauses of this type are the object cross-referencing verbs and the utterance verb \textstyleStyleVernacularWordsItalic{maak}- `tell', and the verb \textstyleStyleVernacularWordsItalic{ofakow}- `show, teach'.

The (recipient) object is marked in the verb root of the object cross-referencing verbs denoting giving and feeding (\sectref{sec:3.8.4.2.4}), but it may appear as a separate \textstyleAcronymallcaps{NP} as well (\stepcounter{nx}{\thenx}):

\ea%x950
\label{ex:x950}
\gll [Mua  yiar  ekap-e-mik  nain]\textsubscript{O}  [pura  kui-kuisow\textbf{]}\textsubscript{O} \\
      \\
\glt
\z

man  1p.DAT  come-PA-1/3p  that  bunch  RDP-one

wi-e-mik.

give.them-PA-1/3p

`We gave a bunch each to the men who came to us.'

The section on utterance verbs (\sectref{sec:3.8.4.4.6}) describes in some detail how these verbs behave in clauses. \textstyleStyleVernacularWordsItalic{Maak}- `tell' requires the addressee/recipient to be a [+human] obligatory object, and as a second object it often has a \textstyleAcronymallcaps{NP} denoting the speech itself or the contents of that speech. 

\ea%x1839
\label{ex:x1839}
\gll I  \textbf{[}\textbf{opora  muut  nain}\textbf{]}\textsubscript{O}\textbf{ } \textbf{[}\textbf{nefa}\textbf{]}\textsubscript{O}\textbf{  maak-u  na-ep} \\
      \\
\glt
\z

1p.UNM  talk  only  thus  2s.ACC  tell-IMP.2d  say-SS.SEQ

ep-a-mik.

come-PA-1/3p

`We came wanting to tell you just that (talk).'

\ea%x955
\label{ex:x955}
\gll Wi  \textbf{[}\textstyleEmphasizedVernacularWords{moma}\textbf{]}\textsubscript{O} \textstyleEmphasizedVernacularWords{} \textbf{[}\textstyleEmphasizedVernacularWords{yia}\textbf{]}\textsubscript{O}  \textstyleEmphasizedVernacularWords{maak-i-mik},  moma=ko  \\
      \\
\glt
\z

3p.UNM  taro  1p.ACC  tell-Np-PR.1/3p  taro=NF

wi-i-yan.

give.them-Np-FU.1p

`They tell us (to get them) taro, (so) we'll give them taro.'

\textstyleStyleVernacularWordsItalic{Na}- `say, speak, call, think' most commonly has the quotation as a speech complement, but it may also have up to two nominal objects instead.

\ea%x956
\label{ex:x956}
\gll [\textstyleEmphasizedVernacularWords{Waaya}\textbf{]}\textsubscript{O} \textstyleEmphasizedVernacularWords{} \textbf{[}\textstyleEmphasizedVernacularWords{yia}\textbf{]}\textsubscript{O}  \textstyleEmphasizedVernacularWords{na-iwkin}  waaya  wienak-em-ik-e-mik. \\
      \\
\glt
\z

pig  1p.ACC  say-2/3p.DS  pig  feed.them-SS.SIM-be-PA-1/3p

`They spoke about pigs to us and we kept giving them pigs to eat.'

There are a few verbs that are ordinary transitive verbs but which can take semantically different objects. It is also possible to have one of each kind in the same clause. The verb \textstyleStyleVernacularWordsItalic{if}- `paint, spread' can have a patient or goal object; the sentence in example (\stepcounter{nx}{\thenx}) includes both. Another such verb is \textstyleStyleVernacularWordsItalic{mik}\nobreakdash- `spear, hit' see (\stepcounter{nx}{\thenx}).  

\ea%x944
\label{ex:x944}
\gll [Yena  aasa\textbf{]}\textsubscript{O}  \textbf{[}ofa\textbf{]}\textsubscript{O}  if-e-m. \\
      \\
\glt
\z

1s.GEN  canoe  colour  paint-PA-1s

`I painted my canoe with paint.' Or: `I spread paint on my canoe.'

\subsubsection[Derived ditransitivity ]{Derived ditransitivity} 
\hypertarget{RefHeading22041935131865}{}
When the transitivity is increased by one of the valence-increasing strategies (\sectref{sec:3.8.4.3}), a recipient or beneficiary (\stepcounter{nx}{\thenx}) becomes a second object. The linear order of the two objects depends on their relative topicality.

\ea%x947
\label{ex:x947}
\gll [Moma  pura  oko]\textsubscript{O}  [Kuuten]\textsubscript{O}  amap-urup-om-a-mik. \\
      \\
\glt
\z

taro  bunch  other  Kuuten  BPx-ascend-BEN-BNFY2.PA-1/3p

`They took another bunch of taro up for Kuuten.'

\ea%x1840
\label{ex:x1840}
\gll Ne  [mua  nain]\textsubscript{O}  [waaya]\textsubscript{O}  mik-om-a-mik. \\
      \\
\glt
\z

ADD  man  that1  pig  spear-BEN-BNFY2.PA-1/3p

`And they speared that man a pig.'

\subsubsection[Possessor raising]{Possessor raising}
\hypertarget{RefHeading22061935131865}{}
There are also cases with two patient-type objects, either one of which could be the single patient of the same verb. One of these objects has resulted from possessor raising: the possessor of the initial object \textstyleAcronymallcaps{NP,} which has to be a semantic patient, has been ``raised'' to become a second object (Van Valin and LaPolla 1997:258, Payne 1997:194-6). Especially when something is done to a body part or name, or something closely identified with a person, both the person and the other noun occur as objects.

\ea%x951
\label{ex:x951}
\gll [\textstyleEmphasizedVernacularWords{Merena}]\textsubscript{O} \textstyleEmphasizedVernacularWords{} [\textstyleEmphasizedVernacularWords{efa}\textbf{]}\textsubscript{O} \textstyleEmphasizedVernacularWords{} keraw-a-k. \\
      \\
\glt
\z

leg  1s.ACC  bite-PA-3s

`It bit me in the leg.' Or: `It bit my leg.'

\ea%x952
\label{ex:x952}
\gll [\textstyleEmphasizedVernacularWords{No  unuma}]\textsubscript{O}\textbf{ } [\textstyleEmphasizedVernacularWords{nefa}\textbf{]}\textsubscript{O} \textstyleEmphasizedVernacularWords{} faker-i-kuan. \\
      \\
\glt
\z

2s.UNM  name  2s.ACC  raise-Np-FU.3p

`They will praise (lit: lift up) your name.'

\ea%x957
\label{ex:x957}
\gll [\textstyleEmphasizedVernacularWords{Opaimika}\textbf{]}\textsubscript{O}\textbf{ } [\textstyleEmphasizedVernacularWords{efa}\textbf{]}\textsubscript{O} \textstyleEmphasizedVernacularWords{} fien-a-man. \\
      \\
\glt
\z

talk  1s.ACC  push.aside-PA-2p

`You disregarded/disobeyed my talk.'

\ea%x948
\label{ex:x948}
\gll Era=pa  [\textstyleEmphasizedVernacularWords{ekera  wiam  erup}]\textsubscript{O1} \textstyleEmphasizedVernacularWords{} [\textstyleEmphasizedVernacularWords{kukusa}]\textsubscript{O2} \textstyleEmphasizedVernacularWords{} [\textstyleEmphasizedVernacularWords{wia}]\textsubscript{O1} \\
      \\
\glt
\z

way=LOC  1s/p.sister  3p.REFL  erup  picture  3p.ACC

aaw-o-k.

take-PA-3s

`On the way he took a picture of my two sisters.'

\ea%x949
\label{ex:x949}
\gll Mua  papako=ke  [\textstyleEmphasizedVernacularWords{irak-owa}]\textsubscript{O} \textstyleEmphasizedVernacularWords{} [\textstyleEmphasizedVernacularWords{wia}]\textsubscript{O} \textstyleEmphasizedVernacularWords{} puuk-a-mik. \\
      \\
\glt
\z

man  some=CF  fight-NMZ  3p.ACC  cut-PA-1/3p

`Some men\textsubscript{i} stopped their\textsubscript{j} fight.'

Even three objects are allowed, but this is rare (\stepcounter{nx}{\thenx}): the verb \textstyleStyleVernacularWordsItalic{mik}- `spear, hit' itself allows two different objects, and the third one is added via possessor raising. The objects have to be in this order. Note that in the English translation, only one direct object is allowed, and the other two phrases have to be either possessive or oblique.

\ea%x953
\label{ex:x953}
\gll [\textstyleEmphasizedVernacularWords{Keema-muuna,  umakuna}\textbf{]}\textsubscript{O} \textstyleEmphasizedVernacularWords{} [\textstyleEmphasizedVernacularWords{meta}]\textsubscript{O} \textstyleEmphasizedVernacularWords{} [\textstyleEmphasizedVernacularWords{yia}]\textsubscript{O}  \textstyleEmphasizedVernacularWords{mik-i-mik}. \\
      \\
\glt
\z

knee-joint  neck  ritual.paste  1p.ACC  hit-Np-PR.1/3p

`They stick the \textstyleForeignWords{meta} paste on our knees and necks' or: `They mark our knees and necks with the \textstyleForeignWords{meta} paste.'

As the preferred clause structure in Mauwake is short and because it is harder to process a verb with many arguments, a common strategy is to divide the arguments between more than one clause, so that each clause has only one or two arguments:

\ea%x967
\label{ex:x967}
\gll I  dabuela  aaw-ep  Yaapan  wi-em-ik-e-mik. \\
      \\
\glt
\z

1p.UNM  pawpaw  take-SS.SEQ  Japan  give.them-SS.SIM-be-PA-1/3p

`We took pawpaws and gave them to the Japanese' or: `We gave pawpaws to the Japanese.'

Even if having more than one \textstyleAcronymallcaps{NP} in non-subject argument or peripheral positions in the same clause is not preferred, it is still reasonably common. But having more than one pronoun as arguments or peripherals is unusual. In the rare case that that does happen, the accusative pronoun occupies the position closest to the verb, next the dative pronoun, then the others. The first two of the following three examples have been elicited.

\ea%x1574
\label{ex:x1574}
\gll Mua  nain  teeria  muutiw  \textstyleEmphasizedVernacularWords{wame  wia}  ofakow-a-k. \\
      \\
\glt
\z

man  that  group  only  3s.REFL  3p.ACC  show-PA-3s

`He only showed himself to that man's group.'

\ea%x1577
\label{ex:x1577}
\gll O  \textstyleEmphasizedVernacularWords{wiar  nefa}  sesek-i-yem. \\
      \\
\glt
\z

3s.UNM  3.DAT  2s.ACC  send-Np-PR.1s

`I am sending you to him.'

\ea%x1575
\label{ex:x1575}
\gll Emeria  ikoka  Yaapan  \textbf{wiena}  \textbf{niar}  aaw-i-kuan \\
      \\
\glt
\z

woman  later  Japan  3p.GEN  2p.DAT  take-Np-FU.3p

`Later the Japanese will take your wives as their own.'

\section{Intransitive clauses}
\hypertarget{RefHeading22081935131865}{}
An intransitive clause in Mauwake is a verbal clause that does not have an object. It normally indicates an event of some kind (action or process), or a state. This differs from the definition used for typological studies of an intransitive predication consisting of ``a one-place predicate and its argument'' \citep[9]{Stassen1997} in that in Mauwake those predications that indicate some property or quality, or designate a class, are not treated as intransitive but as verbless clauses (\sectref{sec:5.6}). Any of the intransitive verbs (\sectref{sec:3.8.4.2.1}) can be the predicate in an intransitive clause, whereas a verbless clause characteristically has no verb. The only negation strategy for clauses with a verbal predicate is the negator \textstyleStyleVernacularWordsItalic{me} `not', whereas verbless clauses have more negator options. 

The following clauses are typical intransitive clauses:

\ea%x961
\label{ex:x961}
\gll Epa  wiim-eya  mua  karer-omak-e-mik. \\
      \\
\glt
\z

place  dawn-2/3s.DS  man  gather-DISTR/PL-PA-1/3p

`When it got light a lot of people gathered.'

\ea%x964
\label{ex:x964}
\gll O  koora=pa  naap  ik-ok  um-o-k. \\
      \\
\glt
\z

3s.UNM  house=LOC  thus  be-SS  die-PA-3s

`She was in the house like that and died.'

\ea%x959
\label{ex:x959}
\gll Uuriw  akena  mukuna  nain  kerer-e-k. \\
      \\
\glt
\z

morning  truly  fire  that1  appear-PA-3s

`The fire started early in the morning.'

\ea%x960
\label{ex:x960}
\gll Iiwawun  iwera  pun  wiar  aw-omak-e-k. \\
      \\
\glt
\z

altogether  coconut  also  3.DAT  burn-DISTR/PL-PA-3s

`His many coconut trees too burned altogether.'

\ea%x965
\label{ex:x965}
\gll I  Sarak  ikos  owowa  ekap-em-ik-e-mik. \\
      \\
\glt
\z

1p.UNM  Sarak  with  village  come-SS.SIM-be-PA-1/3p

`Sarak and I kept coming back to the village.'

\ea%x962
\label{ex:x962}
\gll Fikera  mamaiya=pa  nan  pok-ap  ik-e-mik. \\
      \\
\glt
\z

kunai.grass  close=LOC  there  sit-SS.SEQ  be-PA-1/3p

`We were sitting near the \textstyleForeignWords{kunai} grass.'

\ea%x968
\label{ex:x968}
\gll Ne  kiiriw  miiw-aasa  nan  ik-eya  {\dots} \\
      \\
\glt
\z

and  again  land-canoe  there  be-2/3s.DS

`And again the car was/stayed there, and {\dots}'

Many climate expressions are normal intransitive clauses.

\ea%x1020
\label{ex:x1020}
\gll Moram  \textstyleEmphasizedVernacularWords{ewar } pun  \textstyleEmphasizedVernacularWords{wuun-e-k}  ne  {\dots} \\
      \\
\glt
\z

why  west.wind  too  blow-PA-3s  and

`Because wind blew too, and {\dots}' Or: `Because it was windy too, and {\dots}'

\ea%x1022
\label{ex:x1022}
\gll \textstyleEmphasizedVernacularWords{Ipia  or-om-ik-eya}  owora  aaw-ep  up-o-k. \\
      \\
\glt
\z

rain  descend-SS.SIM-be-2/3s.DS  betelnut  take-SS.SEQ  plant-PA-3s

`When it was raining he took betelnuts and planted them.'

\ea%x1021
\label{ex:x1021}
\gll \textstyleEmphasizedVernacularWords{Epa  kokom-ar-eya}  in-e-mik. \\
      \\
\glt
\z

place  dark-INCH-2/3s.DS  sleep-PA-1/3p

`It became dark and we slept.'

The resultative verbs (\sectref{sec:3.8.4.4.4}) require a nominal argument expressing the result of change: 

\ea%x963
\label{ex:x963}
\gll Mua  eneka,  woosa  \textstyleEmphasizedVernacularWords{kia  kir-em-ik-ua}. \\
      \\
\glt
\z

man  tooth  head  white  turn-SS.SIM-be-PA.3s

`The people's teeth and skulls were turning white.'

\ea%x966
\label{ex:x966}
\gll Arim-emi  \textstyleEmphasizedVernacularWords{emeria  ar-e-k}. \\
      \\
\glt
\z

grow-SS.SIM  woman  become-PA-3s

`She grew and became a woman.'

A few intransitive verbs can occur with a syntactic object or object-like element whose semantic role is not a patient. These differ from true patient objects in that the range of possible ``objects'' for those verbs is very restricted, they cannot be substituted with an accusative pronoun, and the verb cannot occur with the dummy object \textstyleStyleVernacularWordsItalic{maa} `thing'. The first type can be called a ``content object'' \citep[179]{HakulinenEtAl1979}%Karlsson
: 

\ea%x308
\label{ex:x308}
\gll Wis  pun  wiisa  uf-e-mik. \\
      \\
\glt
\z

3p.FC  too  wiisa  dance-PA-1/3p

`They, too, danced ``wiisa''.'

The second type is an object-like adverbial, as it functions in the same way as an adverbial phrase.

\ea%x307
\label{ex:x307}
\gll Era  maala  soomar-e-mik\textstyleEmphasizedVernacularWords{\textmd{\textit{.}}} \\
      \\
\glt
\z

way  long  walk-PA-1/3p

`We walked a long way.'

\section{Existential and possessive clauses}
\hypertarget{RefHeading22101935131865}{}
Existential clauses and possessive clauses are distinguished from the intransitive clauses. Only the verb \textstyleStyleVernacularWordsItalic{ik}- `be' is used as the predicate in both of them. 

\subsection{Existential clauses}
\hypertarget{RefHeading22121935131865}{}
Existential clauses are not very common. Giv\'on (1990:741) names these clauses as one of the main devices for introducing a new topic into a discourse, but in Mauwake they are not used very much in that function (\sectref{sec:9.1.2.1}). Existential clauses use the verb \textstyleStyleVernacularWordsItalic{ik}- `be' as their predicate, and they often contain a locative phrase (\stepcounter{nx}{\thenx}), but it is not necessary (\stepcounter{nx}{\thenx}), (\stepcounter{nx}{\thenx}). 

\ea%x970
\label{ex:x970}
\gll \textstyleEmphasizedVernacularWords{Aaya=ko}  \textstyleEmphasizedVernacularWords{feeke  ik-eya}  nefa  aaw-ep  enim-i-yen. \\
      \\
\glt
\z

sugarcane=NF  here.CF  be-2/3s.DS  2s.ACC  take-SS.SEQ  eat-Np-FU.1p

`If there is (any) sugarcane here, we'll take and eat you (the sugarcane).'

\ea%x971
\label{ex:x971}
\gll Aakisa  Malala  suule  ik-ua,  {\dots} \\
      \\
\glt
\z

now  Malala  school  be-PA.3s

`Now there is the Malala school, {\dots}'

Both the past and future tense forms can be used; the past tense may be used for both present and past meaning. 

\ea%x1068
\label{ex:x1068}
\gll Kuisow  owowa=pa=ko  me  ik-ua. \\
      \\
\glt
\z

one  village=LOC=NF  not  be-PA.3s

`There was/is not even one in the village.'

\ea%x1067
\label{ex:x1067}
\gll Waaya  ika-i-non-(na)  waaya  uup-i-nan. \\
      \\
\glt
\z

pig  be-Np-FU.3s-(TP)  pig  cook-Np-FU.2s

`If there is a pig, you will cook a pig.'

When an existential clause of this type is negated with a negator other than \textstyleStyleVernacularWordsItalic{me}, it becomes a verbless clause (\sectref{sec:5.6.3}).

A special type of existential clause has one of the two location verbs (\sectref{sec:3.8.4.4.3}) as the predicate. These verbs are only used in the past tense, even with the present tense meaning.

\ea%x1154
\label{ex:x1154}
\gll Nomokowa  unowa  \textstyleEmphasizedVernacularWords{fan-e-mik},  aakisa  wia  uruf-i-n.  \\
      \\
\glt
\z

2s/p.brother  many  here-PA-1/3p  now  3p.ACC  see-Np-PR.2s

 `Many of your brothers are here, now you see them.'

\ea%x1155
\label{ex:x1155}
\gll No  niawi  akena  \textstyleEmphasizedVernacularWords{nan-e-k},  no  fain  \\
      \\
\glt
\z

2s.UNM  2s/p.father  real  there-PA-3s  2s.UNM  this

me  nena  niawi  akena=ke.

not  2s.GEN  2s/p.father  real=CF

`Your real father is there, this isn't your real father.'

\subsection{Possessive clauses}
\hypertarget{RefHeading22141935131865}{}
Possessive clauses, or so-called `have' clauses, are formed with a dative pronoun and the verb \textstyleStyleVernacularWordsItalic{ik}- `be'. This is a grammaticalization from [+human] locative constructions with the semantic function of goal or locative \citep[50-61]{Heine1997}, as was briefly mentioned in \sectref{sec:3.5.5}.  

The possessee is the patient-of-state subject, which is shown by the fact that it may take the contrastive focus clitic -\textstyleStyleVernacularWordsItalic{ke} (\stepcounter{nx}{\thenx}) and it determines the person inflection on the verb as well (\stepcounter{nx}{\thenx}). The possessor is a \textstyleEmphasizedWords{\textsc{habitive adverbial}}, like  a corresponding construction in Finnish is called \citep[209]{HakulinenEtAl1979}%Karlsson
. Giv\'on calls it a dative object (1984:104), but I prefer to keep the term ``object'' for those arguments in a transitive clause that can take an accusative form when they are [+human].\footnote{\citet[302]{Dixon2010b} calls the initial argument position copula subject (CS) and the second one copula complement (CC), regardless of whether the position is filled by the possessor or the possessee.}

\ea%x595
\label{ex:x595}
\gll Aaya  \textstyleEmphasizedVernacularWords{efar}  \textstyleEmphasizedVernacularWords{ikua},  ifera  wia. \\
      \\
\glt
\z

sugar  1s.DAT  be-PA.3s  salt  no

`I have sugar, but no(t) salt.'

\ea%x1065
\label{ex:x1065}
\gll Apu  maa  epira  marok  maneka=ke  \textstyleEmphasizedVernacularWords{wiar  ik-ua}. \\
      \\
\glt
\z

Apu  food  plate  prawn  big=CF  3.DAT  be-PA.3s

`Apu has/had big prawns on his food plate.' (Lit: `Apu's food plate\textsubscript{Theme} he has/had big prawns.')

\ea%x1323
\label{ex:x1323}
\gll Woos(a)  mua  \textstyleEmphasizedVernacularWords{yiar  ik-e-mik},  wis=ke  eliw  nia \\
      \\
\glt
\z

head  man  1p.DAT  be-PA-1/3p  3p.FC=CF  well  2p.ACC

kaken-i-kuan.

straight-Np-FU.3p

`We have leaders, they can straighten you out.'

Because the possessee is typically inanimate and often indefinite whereas the possessor is human and definite, this causes a violation to the universal discourse-pragmatic principle, according to which animate/human and definite participants tend to precede inanimate and indefinite participants \citep[135]{Heine1997}. In order to follow the principle, Mauwake often makes the possessor a theme by moving the possessor \textstyleAcronymallcaps{NP} to sentence-initial position; only the dative pronoun keeps its position immediately preceding the verb (\stepcounter{nx}{\thenx}). If there is no other possessor \textstyleAcronymallcaps{NP}, an unmarked pronoun is used as a theme (\stepcounter{nx}{\thenx}). In these two sentences, moving part of the \textstyleAcronymallcaps{NP} to the theme position causes the \textstyleAcronymallcaps{NP} to be non-contiguous. In the example (\stepcounter{nx}{\thenx}) the possessee subject \textstyleStyleVernacularWordsItalic{aaya} `sugar' is also the theme, and in (\stepcounter{nx}{\thenx}) the possessee is animate/human, so in those clauses there is less pressure to make the possessor into the theme.

\ea%x973
\label{ex:x973}
\gll [\textstyleEmphasizedVernacularWords{I}]\textsubscript{T}heme  sira  naap  me  \textstyleEmphasizedVernacularWords{yiar  ik-ua}. \\
      \\
\glt
\z

1p.UNM  custom  thus  not  1p.DAT  be-PA.3s

`We do not have a custom like that.'

\ea%x972
\label{ex:x972}
\gll [\textstyleEmphasizedVernacularWords{Mua}  \textstyleEmphasizedVernacularWords{oko}]\textsubscript{T}heme  ona  koor  miira=pa]  [nan]  [waaya  \\
      \\
\glt
\z

man  other  3s.GEN  house  face=LOC  there  pig  

unowa]  \textstyleEmphasizedVernacularWords{wiar  ik-ua}.

many  3.DAT  be-PA.3s

`Another man has many pigs there in front of his house.'

Clauses like the example (\stepcounter{nx}{\thenx}), where the possessed noun is [+human] and [+plural], triggering the plural form of the verb, are quite rare, and it seems that the singular verb form is also becoming possible in these cases:

\ea%x1321
\label{ex:x1321}
\gll Mua  nain  pun  muuka  wiipa  \textstyleEmphasizedVernacularWords{wiar  ik-ua}. \\
      \\
\glt
\z

man  that1  also  son  daughter  3.DAT  ik-PA.3s

`That man also has children/son(s) and daughter(s).'

All the tenses are possible. The past tense form normally covers both present and past meaning.

\ea%x1066
\label{ex:x1066}
\gll Naap  on-i-non=na  pina  \textstyleEmphasizedVernacularWords{wiar  ika-i-non}. \\
      \\
\glt
\z

thus  do-Np-FU.3s=TP  guilt  3.DAT  be-Np-FU.3s

`If he does like that he will have guilt.'

When the present tense form is used, it indicates a more transitory possession:

\ea%x1201
\label{ex:x1201}
\gll Wis  pun  maa  eliwa=ko  wiar  \textstyleEmphasizedVernacularWords{ika-i-ya=na} \\
      \\
\glt
\z

3p.UNM  too  thing  good=NF  3.DAT  be-Np-PR.3s=TP

iw-i-mik.

give.him-Np-PR.1/3p

`They too, if they (happen to) have good things, give to him.'

A possessive clause may be elliptical, with the verb deleted, in cases where the possessed \textstyleAcronymallcaps{NP} has at least one post-modifier, which most commonly is a quantifier.

\ea%x1322
\label{ex:x1322}
\gll Yo  muuka  arow,  wiipa  kuisow  muuta  {\O}. \\
      \\
\glt
\z

1s.UNM  son  three,  daughter  one  only

`I have three sons, (and/but) only one daughter.'

When the possessor is not human, the possessive clause is made with the existential verb \textstyleStyleVernacularWordsItalic{ik}- `be' plus a comitative construction rather than the dative pronoun; and the possessor always precedes the possessee. These are cross-linguistically typical features for the grammaticalization strategy that uses a comitative phrase for a possessive predication \citep[53-57]{Heine1997}. As was noted in \sectref{sec:3.5.4}, in this case the third person singular genitive pronoun \textstyleStyleVernacularWordsItalic{ona} is used for a non-human possessor.

\ea%x1807
\label{ex:x1807}
\gll Parina  ona  wakesim-owa  \textstyleEmphasizedVernacularWords{onaiya}  ika-i-ya. \\
      \\
\glt
\z

lamp  3s.GEN  cover-NMZ  with  be-Np-PR.3s

`The lamp has a cover.'

\ea%x1810
\label{ex:x1810}
\gll Miiwa  ona  mua  \textstyleEmphasizedVernacularWords{onaiya}  ik-ua. \\
      \\
\glt
\z

land  3s.GEN  man  with  be-PA.3s

`The land has its men.' (Each piece of ground ``has'' men whose responsibility it is to see how the land is allocated for gardens.)

Possessive clauses are similar to existential clauses in that when a possessive clause is negated with a negator other than \textstyleStyleVernacularWordsItalic{me}, it becomes a verbless clause (\sectref{sec:5.6.3} and \sectref{sec:6.2.1}).

\section{Verbless clauses}
\hypertarget{RefHeading22161935131865}{}
The predicate of a verbless clause belongs to some other phrase class than the verbs. The two subtypes below, equative and descriptive clauses, are very similar syntactically; their differences are mainly in the semantics of the predicates. Their negation strategies are also slightly different from each other. 

\ea%x1036
\label{ex:x1036}
\gll Mua  nain  yena  kae  panewowa=ke. \\
      \\
\glt
\z

man  that1  1s.GEN  1s/p.grandfather  old=CF

`That man is my old grandfather.'

\ea%x1037
\label{ex:x1037}
\gll Waaya  nain  me  maneka,  muuka  kia  gelemuta. \\
      \\
\glt
\z

pig  that1  not  big  son  white  small

`The pig wasn't big, it was a small white piglet.'

In certain cases the verb \textstyleStyleVernacularWordsItalic{ik}- `be' is required as a copula. This happens mainly in the future (\stepcounter{nx}{\thenx}) or sometimes in the past tense, or when the clause requires a medial form to indicate that it is a medial clause (\stepcounter{nx}{\thenx}). 

\ea%x986
\label{ex:x986}
\gll Ikoka  mua  eliwa  ne  mua  oona  ika-i-nan. \\
      \\
\glt
\z

later  man  good  and  man  bone  be-Np-FU.2p

`Later you will be a good and strong man.'

\ea%x987
\label{ex:x987}
\gll No  gelemuta  ik-eya  {\dots} \\
      \\
\glt
\z

2s.UNM  little  be-2/3s.DS

`When you were little, {\dots}'

\subsection{Equative and classifying clauses}
\hypertarget{RefHeading22181935131865}{}
Syntactically equative and classifying clauses are identical. The non-verbal predicate typically has contrastive focus marking -\textstyleStyleVernacularWordsItalic{ke}, even though it is not absolutely necessary. 

In an equative clause the subject and the non-verbal predicate have the same reference, so their order can be reversed with the basic meaning staying the same. 

\ea%x975
\label{ex:x975}
\gll Dogimaw  yiena  owow  saria=ke. \\
      \\
\glt
\z

Dogimaw  1p.GEN  village  headman=CF

`Dogimaw is our village headman.'

\ea%x976
\label{ex:x976}
\gll Yiena  owow  saria  Dogimaw(=ke). \\
      \\
\glt
\z

Our  village  headman  Dogimaw(=CF)

`Our village headman is Dogimaw.'

An equative clause is only negated with the verbal negator \textstyleStyleVernacularWordsItalic{me}:

\ea%x1752
\label{ex:x1752}
\gll Dogimaw  me  yiena  owow  saria=ke. \\
      \\
\glt
\z

Dogimaw  not  1p.GEN  village  headman=CF

`Dogimaw is not our village headman.'

In classifying clauses\footnote{\citet[233]{Dryer2007b} calls them ``clauses with a true nominal predicate''.} the reference of the subject is not identical with the reference of the predicate.

\ea%x977
\label{ex:x977}
\gll Yo  inasin  mua=ke. \\
      \\
\glt
\z

1s.UNM  spirit  man=CF

`I am a spirit man.'

\ea%x978
\label{ex:x978}
\gll Oo  Kululu  takira=ke,  o  me  amis-ar-e-k. \\
      \\
\glt
\z

oh  Kululu  young.person=CF  3s.UNM  not  knowledge-INCH-PA-3s

`Oh, Kululu is a youth (compared to us), he doesn't know.'

The classifying clauses are negated with the verbal negator \textstyleStyleVernacularWordsItalic{me} or with a clause-final negator \textstyleStyleVernacularWordsItalic{weetak/wia}.\footnote{Bergh\"all (2006:272) also gives \textit{marew} `no(ne)' as a possible negator for equative clauses, but actually the equative clauses do not use it, only the descriptive clauses.} 

\ea%x980
\label{ex:x980}
\gll Nain  me  inasin  mua=ke,  iperuma=ke. \\
      \\
\glt
\z

that1  not  spirit  man=CF  eel=CF

`That is not a spirit man, it is an eel.'

\ea%x984
\label{ex:x984}
\gll O  somek  mua  weetak/wia. \\
      \\
\glt
\z

he  song  man  no

`He is not a teacher (lit: song man).'

The predicates of both these clauses are more time-stable compared both with verbal predicates and those in the descriptive clauses (Giv\'on 1984:51, Stassen 1997:16). 

\subsection{Descriptive clauses}
\hypertarget{RefHeading22201935131865}{}
A descriptive clause is very much like an equative clause, but the predicate is an adjective phrase (\stepcounter{nx}{\thenx}), a noun phrase with an adjective (\stepcounter{nx}{\thenx}), or less frequently a numeral (\stepcounter{nx}{\thenx}) or an adverbial phrase (\stepcounter{nx}{\thenx}). On the time-stability scale these predicates are in between verbal and nominal predicates. 

\ea%x974
\label{ex:x974}
\gll Irak-owa  nain  kekanowa  akena. \\
      \\
\glt
\z

fight-NMZ  that1  strong  very

`The fighting was very fierce.'

\ea%x979
\label{ex:x979}
\gll Yiena  miiwa  kuisow. \\
      \\
\glt
\z

1p.GEN  land  one

`Our land is one.'

\ea%x985
\label{ex:x985}
\gll Nain  pun  sira  naap=iw,  mua  me  kerer-e-mik. \\
      \\
\glt
\z

that1  too  custom  thus=INST  man  not  appear-PA-1/3p

`That, too, was like that: people didn't arrive.'

A descriptive clause can use any of the negation strategies available in Mauwake (\sectref{sec:3.10}, 6.2). 

\ea%x990
\label{ex:x990}
\gll Biiris  me  eliwa,  damo-damola=ko. \\
      \\
\glt
\z

bridge  not  good  RDP-bad=NF

`The bridges were not good, they were bad.'

\ea%x988
\label{ex:x988}
\gll Yo  (mua)  maala  marew. \\
      \\
\glt
\z

1s.UNM  (man)  long  no(ne)

`I am not (a) tall (man).'

\ea%x989
\label{ex:x989}
\gll Awuliak  nain  eliwa  weetak/wia. \\
      \\
\glt
\z

sweet.potato  that1  good  no

`That sweet potato is not good.'

\subsection{Negated existential and possessive clauses}
\hypertarget{RefHeading22221935131865}{}
The existential (\sectref{sec:5.5.1}) and possessive clauses (\sectref{sec:5.5.2}) are different from the other verbal clauses with regard to negation. Besides the standard verbal negation (\stepcounter{nx}{\thenx}) they can use all the other negators as well (\sectref{sec:6.2.1}). The verb \textstyleStyleVernacularWordsItalic{ik}- `be' is retained only with the verbal negator \textstyleStyleVernacularWordsItalic{me}. With all the other negators the negator itself replaces the verb, and the clause becomes a verbless clause:

\ea%x982
\label{ex:x982}
\gll Iiriw  miiwa  muuta  nain  irak-owa  \textstyleEmphasizedVernacularWords{marew}. \\
      \\
\glt
\z

earlier  land  for  that1  fight-NMZ  no(ne)

`Earlier there was no fighting for land.'

\ea%x983
\label{ex:x983}
\gll I  urupa  \textstyleEmphasizedVernacularWords{weetak},  i  soomia  \textstyleEmphasizedVernacularWords{wia},  i  \\
      \\
\glt
\z

1p.UNM  cup  no  1p.UNM  spoon  no  1p.UNM

epira  \textstyleEmphasizedVernacularWords{marew}.

plate  no(ne)

`We had no cups, no spoons, no plates.'

\section{Nominalized clauses}
\hypertarget{RefHeading22241935131865}{}
Lexical nominalization, where an action nominal is a regular noun, was discussed in \sectref{sec:3.2.6.1}; in this section nominalization as an operation on the whole clause is described.

Action nominals and infinitives are usually assumed to be two separate non-finite categories.\footnote{Ylikoski (2003, 2009) discusses the similarities and differences between various non-finite verb forms and presents insightful definitions based mainly on their syntactic functions. Many of the details are not relevant to Mauwake, however, as there are no verb forms that would easily fit under the categories of converbs or participles in Mauwake, and because it seems that infinitives and action nominals may be collapsible into one category.} Cross-linguistically, the two often tend to be identical in form \citep[224]{Ylikoski2003}, and there is apparently a separate tendency for their functions to look rather similar as well (ibid. 196-197). It seems that the origin of the infinitive in many languages is in a nominalized verb \citep[69]{Noonan2007}. 

In Mauwake there is just one form, and rather than positing two homonymous forms with different functions, I maintain that action nominals function both like prototypical nouns or adjectives \textstyleEmphasizedWords{\textsc{and}} in functions typically associated with infinitives: as complements\footnote{Ylikoski widens the definition of complement to cover ``obligatory and argumental adverbials as well'' (2003:209).} of certain verbs, in goal/purpose and deontic structures among others.  

Structurally there are two kinds of nominalized clauses in Mauwake. They may occur as complements of the same verbs, with somewhat different semantics. The first type is what the term \textstyleEmphasizedWords{\textsc{nominal}}\textstyleEmphasizedWords{\textsc{ized clause}} most commonly refers to: the verbal predicate of a clause is nominalized, and consequently the whole clause becomes a noun phrase. The second type retains the form of a verbal clause, but the distal deictic \textstyleStyleVernacularWordsItalic{nain} `that' after the finite verb nominalizes it.  The first type has a wider distribution.

\subsection{Type1: with a nominalized verb}
\hypertarget{RefHeading22261935131865}{}
When verbs are nominalized, the action or event referred to still keeps some of its verbal characteristics (\textstyleBibliogBaseChar{Hopper and Thompson} 1985:177). Languages differ as to how verbal or nominal in character their nominalized verbs are, and also within one language the outcomes of different nominalization strategies may vary in regard to this \citep[344]{ComrieEtAl2007}%Thompson
. In this respect Mauwake is a very \textstyleEmphasizedWords{verbal} language: the nominalized verbs retain a number of their verbal characteristics.

Neutralization of tense or aspect distinctions, as well as the loss of other than just one argument are common features associated with nominalization (\textstyleBibliogBaseChar{Hopper and} \textstyleBibliogBaseChar{Thompson} 1984:737-738). In Mauwake the nominalized verb forms may keep all of the derivational suffixes but not the inflectional ones, which include tense and person/number marking.

\ea%x1226
\label{ex:x1226}
\gll Aakisa=ko  me  kerer-em-ika-i-ya,  wia  bala \\
      \\
\glt
\z

now=NF  not  appear-SS.SIM-be-Np-PR.3s  3p.ACC  decoration

op-aw-ap  wia  \textstyleEmphasizedVernacularWords{wiim-om-owa}  nain.\footnote{The long subject NP consisting of a nominalized clause has been right-dislocated.}

hold-CAUS-SS.SEQ  3p.ACC  escort-BEN-NMZ  that1

`Now it doesn't take place (any more), decorating them and escorting them for them (i.e. escorting girls to their prospective husbands).'

Verbal groups showing aspect may be nominalized as well, so the aspectual distinction is retained: 

\ea%x1841
\label{ex:x1841}
\gll [Mua  papako  maa  \textstyleEmphasizedVernacularWords{ik-em-ik-owa}]  nain  kawus  wiar \\
      \\
\glt
\z

man  some  food  roast-SS.SIM-be-NMZ  that1  smoke  3.DAT

uruf-i-kuan.

see-Np-FU.3p

`They will see the smoke from some men's roasting of food.'

The nominalized verb in itself is neutral in regard to modality, even if it often gets deontic interpretation. But it can be, and frequently is, used in cases where modality is intentionally left unspecified. In (\stepcounter{nx}{\thenx}) the reason for not coming may be that one is not allowed, or able, or willing, to come. 

\ea%x1257
\label{ex:x1257}
\gll Yo  \textstyleEmphasizedVernacularWords{ekap-owa}  wia. \\
      \\
\glt
\z

1s.UNM  come-NMZ  no

`I won't come.'

But note (\stepcounter{nx}{\thenx}) where the contrastive focus marker added to the nominalized verb forces a deontic interpretation. See also \sectref{sec:6.1.2}. 

The nominalized verb can keep all the arguments and peripherals that a corresponding finite verb would have. This sometimes results in very long noun phrases. (In the following example there is lexical nominalization of the verb \textstyleStyleVernacularWordsItalic{kookal}- `like' as well, besides the clausal nominalization.)

\ea%x1234
\label{ex:x1234}
\gll [\textstyleEmphasizedVernacularWords{Manin(a)  maneka,  ekina,  naisow  nena}   \\
      \\
\glt
\z

garden    big    ekina  2s.ISOL  2s.GEN  

\textstyleEmphasizedVernacularWords{kookal-owa}\textstyleEmphasizedVernacularWords{=pa} \textstyleEmphasizedVernacularWords{perek-owa}]  weetak.

like-NMZ=LOC   pull.out-NMZ  no

`You are not allowed to harvest the big garden, called ekina, at your own liking.'

In the following two examples only the nominalized verb is within the scope of the negation. The nominalized clauses are in brackets.

\ea%x1235
\label{ex:x1235}
\gll [Maa  eneka  \textstyleEmphasizedVernacularWords{me  en-owa}]  maa  marew. \\
      \\
\glt
\z

thing  tooth  not  eat-NMZ  thing  none

`Not eating meat is all right / is not an issue.'

\ea%x1236
\label{ex:x1236}
\gll Wi  mua  [naap  \textstyleEmphasizedVernacularWords{me  on-owa}] \textstyleEmphasizedVernacularWords{} nain=ko  ik-e-mik=i? \\
      \\
\glt
\z

3p.UNM  man  thus  not  do-NMZ  that1=NF  be-PA-1/3p=QM

`Are there people who wouldn't do / keep doing like that?'

Any of the four negators (\sectref{sec:6.2.1}) may be used to negate the nominalized clause (\stepcounter{nx}{\thenx})-(\stepcounter{nx}{\thenx}). 

Cross-linguistically nominalized clauses also vary as to whether they retain a manner adverbial of the corresponding verbal clause or change it into an adjective \citep[374]{ComrieEtAl2007}%Thompson
. Mauwake keeps the adverbial: 

\ea%x1237
\label{ex:x1237}
\gll [Wiena  teeria  \textstyleEmphasizedVernacularWords{baliwep  wia  kakalt-owa}]  sira  nain  \\
      \\
\glt
\z

3p.GEN  family  well  3p.ACC  look.after-NMZ  custom  that1

wia  maak-e-k.

3p.ACC  tell-PA-3s

`He talked to them about the custom of looking after their families well.'

One common feature in nominalized clauses is that the arguments, instead of taking the morphology they would have in a finite clause, tend to follow typical \textstyleAcronymallcaps{NP} morphology in their marking \citep[738]{HopperEtAl1984}%Thompson
. This is perhaps clearest with the subject, which in many languages gets possessive/genitive marking in a nominalized clause. In Mauwake this criterion is not very helpful. The pronominal subject of this first type of nominalized clause, if present, is often genitive (\stepcounter{nx}{\thenx}), but may be nominative as well. But also the subject of a finite clause can be nominative or genitive in form, depending on whether it is neutral or emphatic; and if the subject of a nominalized clause is also the theme, it is nominative rather than genitive (\stepcounter{nx}{\thenx}). A pronominal object in a nominalized clause is in the accusative (\stepcounter{nx}{\thenx}). 

\ea%x1228
\label{ex:x1228}
\gll \textstyleEmphasizedVernacularWords{Yiena  owow  maneka  ikiw-owa  nain}  ma-i-yem. \\
      \\
\glt
\z

1p.GEN  village  big  go-NMZ  that1  say-Np-PR.1s

`I am talking about our going to town.'

The nominalized verb may take an adjective modifier:

\ea%x1240
\label{ex:x1240}
\gll \textstyleEmphasizedVernacularWords{Kema  suuw-owa  eliwa}  aaw-ep  kekan-e-k. \\
      \\
\glt
\z

liver  push-NMZ  good  get-SS.SEQ  get.strong-PA-3s

`He got good thinking and became strong.'

Another structural indicator of the nominal status of a nominalized clause is the focus marking, which can be attached to the verb. 

\ea%x1238
\label{ex:x1238}
\gll I  uuw-owa  yi-iwkin  \textstyleEmphasizedVernacularWords{baliwep  uuw-owa=ke} \\
      \\
\glt
\z

1p.UNM  work-NMZ  give.us-2/3p.DS  well  work-NMZ=CF

ik-ua.

be-PA.3s

`When they give us work, working well is our duty.'

Nominalized clauses, like other noun phrases, use the far deictic \textstyleStyleVernacularWordsItalic{nain} `that' as a determiner. 

\ea%x1842
\label{ex:x1842}
\gll [\textstyleEmphasizedVernacularWords{Ona}  \textstyleEmphasizedVernacularWords{epa  maneka  or-owa}]  \textstyleEmphasizedVernacularWords{nain}  fofa=pa  ...  unow-iya \\
      \\
\glt
\z

3s.GEN  place  big  descend-NMZ  that1  day=LOC  {\dots}  many=COM

taan-e-mik.

fill-PA-1/3p

`On the day of his coming down to the big place {\dots} all of them filled (the place).'

\ea%x1843
\label{ex:x1843}
\gll [\textstyleEmphasizedVernacularWords{Niena}  \textstyleEmphasizedVernacularWords{waaya  mik-owa}]  \textstyleEmphasizedVernacularWords{nain}  on-ami  kuep-i-man, \\
      \\
\glt
\z

2p.GEN  pig  spear-NMZ  that1  do-SS.SIM  break-Np-PR.2p

niena  maa=ke,  niena  wiowa=ke.

2p.GEN  thing=CF  2p.GEN  spear=CF

`If you break (the spears) (while) doing your pig-hunting (lit: pig-spearing), that is your business, they are your spears.'

An interesting structure, and not much described in Papuan languages, is one where a same-subject medial clause is in the scope of the nominalization. In Mauwake this tends to happen when the medial verb shares an object with the following verb and there is no, or very little, intervening material between the verbs. 

\ea%x1885
\label{ex:x1885}
\gll Dabe  wiawi  [\textstyleEmphasizedVernacularWords{maa  ik-ep  en-owa]}  na-ep  \\
      \\
\glt
\z

Dabe  3s/p.father  food  roast-SS.SEQ  eat-NMZ  say-SS.SEQ

manin(a)  koora  iw-a-k.

garden  house  go-PA-3s

`Dabe's father wanted to roast and eat food and went to the garden house.'

\ea%x1845
\label{ex:x1845}
\gll Toiyan  iiriw  maak-ep-pu-a-mik,  [\textstyleEmphasizedVernacularWords{uuriw  yia} \\
      \\
\glt
\z

Toiyan  already  tell-SS.SEQ-CMPL-PA-1/3p  morning  1p.ACC  

\textstyleEmphasizedVernacularWords{aaw-ep}  \textstyleEmphasizedVernacularWords{Madang  ikiw-owa}] \textstyleEmphasizedVernacularWords{} nain.

take-SS.SEQ  Madang  go-NMZ  that1

`We already told Toiyan about taking us in the morning and going to Madang.'

Medial clauses preceding nominalized clauses do not automatically fall within the scope of the nominalization. Just looking at the linguistic form it would be possible to analyse the following examples so that the medial clause is outside the nominalization. In that case the free translation of (\stepcounter{nx}{\thenx}) would be `Having worked on the garden alone it is not acceptable to leave it there', and (\stepcounter{nx}{\thenx}) `Hold the planting stick and keep practising the making of planting holes'. But culturally these alternative interpretations are not valid. Even starting to work on a big garden without previous negotiations and proper rituals is not acceptable, and the holding of the planting stick and making planting holes form a cultural `expectancy chain' and belong together conceptually.

\ea%x1227
\label{ex:x1227}
\gll [\textstyleEmphasizedVernacularWords{Manina}  \textstyleEmphasizedVernacularWords{waisow}  \textstyleEmphasizedVernacularWords{mauw-ap  neeke} \\
      \\
\glt
\z

garden  3s.ISOL  work-SS.SEQ  there.CF  

\textstyleEmphasizedVernacularWords{wafur-ap-pu-owa}]  nain  weetak.

throw-SS.SEQ-CMPL-NMZ  that1  no

`Working on the garden alone and leaving it there (=without proper rituals) is not (acceptable/customary).'

\ea%x1239
\label{ex:x1239}
\gll [\textstyleEmphasizedVernacularWords{Weria  op-ap}  \textstyleEmphasizedVernacularWords{wiinar-owa}] \textstyleEmphasizedVernacularWords{} nain  \\
      \\
\glt
\z

planting.stick  hold-SS.SEQ  make.planting.holes-NMZ  that1

akim-am-ik-e.

try-SS.SIM-be-IMP.2s

`Keep practising the making of planting holes with the planting stick.'

In the following example (\stepcounter{nx}{\thenx}) the medial clause has to be within the scope of the nominalization for the sentences to make sense. The speaker had seen a possum in a tree and would have liked to shoot it, but since he had not taken his bow and arrows with him, he did not climb up the tree either.

\ea%x1844
\label{ex:x1844}
\gll [\textstyleEmphasizedVernacularWords{Nomokowa}  \textstyleEmphasizedVernacularWords{ir-ap  mik-owa}]  nain  yena  amia  me \\
      \\
\glt
\z

tree  climb-SS.SEQ  shoot-NMZ  that  1s.GEN  bow  not  

aaw-e-m.

take-PA-1s

`(For) climbing the tree and shooting (an animal), I hadn't taken my bow (and arrows).'

An intervening overt object may block a same-subject medial clause from being within the scope of a following nominalized verb:

\ea%x1886
\label{ex:x1886}
\gll [Irak-ep]  \textstyleEmphasizedVernacularWords{luuwa}  niir-owa  piipu-a-mik. \\
      \\
\glt
\z

fight-SS.SEQ  ball  play-NMZ  leave-PA-1/3p

`We fought and stopped (lit: left) playing football.'

A different-subject medial clause does not fall within the scope of a nominalized verb. 

The nominalized clause has several different functions. Like any other noun phrase, it may function as an argument (\stepcounter{nx}{\thenx}) or in the periphery of a clause (\stepcounter{nx}{\thenx}), or in another noun phrase (\stepcounter{nx}{\thenx}).

\ea%x1230
\label{ex:x1230}
\gll [\textstyleEmphasizedVernacularWords{Epia}  \textstyleEmphasizedVernacularWords{wilin-owa}]\textsubscript{O}  uruf-ap  bom  yia  \\
      \\
\glt
\z

firewood  shine-NMZ  see-SS.SEQ  bomb  1p.ACC

wafur-om-i-kuan  na-e-mik.

throw-BEN-Np-FU.3p  say-PA-1/3p

`They\textsubscript{i}/we said that when they\textsubscript{j} see the light (lit: shining) of the fire they\textsubscript{j} will throw bombs at us.'

\ea%x1241
\label{ex:x1241}
\gll Wiena  oram  niir-emi  [\textstyleEmphasizedVernacularWords{wiam  kookal}\textstyleEmphasizedVernacularWords{-}\textstyleEmphasizedVernacularWords{owa=pa}]\textsubscript{Advl} \\
      \\
\glt
\z

3p.GEN  just  play-SS.SEQ  3p.REFL  like-NMZ=LOC

nan  wiam  aaw-i-mik.

there  3p.REFL  take-Np-PR.3s

`They just play together and (on the basis of) liking each other they marry each other.'

\ea%x1229
\label{ex:x1229}
\gll [[\textstyleEmphasizedVernacularWords{garanga}  \textstyleEmphasizedVernacularWords{oko  muuka  wiar  aaw-owa}]\textsubscript{NP}  sira]\textsubscript{NP} \\
      \\
\glt
\z

family  other  son/child  3.DAT  get-NMZ  custom

`the adoption custom' (Lit: the custom of getting a child from another family')

The following functions are often associated with infinitives in languages that distinguish between infinitives and nominalizations \citep[207]{Ylikoski2003}. 

Expressions of obligation (\sectref{sec:6.1.2}) use the nominalized form of the main verb. It is followed by the contrastive focus clitic, when it is either a non-verbal predicate (\stepcounter{nx}{\thenx}) or the subject of the verb \textstyleStyleVernacularWordsItalic{ikua} `is'(\stepcounter{nx}{\thenx}).

\ea%x1242
\label{ex:x1242}
\gll Yo  uurika  owow  maneka  \textstyleEmphasizedVernacularWords{ikiw-owa=ke}. \\
      \\
\glt
\z

1s.UNM  tomorrow  village  big  go-NMZ=CF

`I have to go to town tomorrow.'

\ea%x1243
\label{ex:x1243}
\gll Wi  iperowa  ekima  wia  op-ap  \textstyleEmphasizedVernacularWords{baliwep} \\
      \\
\glt
\z

3p.UNM  middle-aged  forehead  3p.ACC  hold-SS.SEQ  well

\textstyleEmphasizedVernacularWords{ik-owa=ke  ik-ua}.

be-NMZ=CF  be-PA.3s

`One has to respect\footnote{The verbal expression for respecting someone is \textit{ekima opowa} `holding someone's forehead'.} the middle-aged/elderly and behave well.'

\ea%x1244
\label{ex:x1244}
\gll Inasina  wia  patir-a-mik  nain  \textstyleEmphasizedVernacularWords{me  wiar} \\
      \\
\glt
\z

bush.spirit  3p.ACC  sacrifice-PA-1/3p  that1  not  3.DAT  

\textstyleEmphasizedVernacularWords{en-owa=ke}.

eat-NMZ=CF

`One must not eat what has been sacrificed to the bush spirits.'

Directional verbs (\sectref{sec:3.8.4.4.5}) may take a nominalized clause as the goal. In many of these cases it is hard to distinguish between goal and purpose, which can be expressed via nominalization as well.

\ea%x1245
\label{ex:x1245}
\gll Yo  \textstyleEmphasizedVernacularWords{emeria  aaw-owa}  urup-e-m. \\
      \\
\glt
\z

1s.UNM  woman  take-NMZ  ascend-PA-1s

`I came up to take my wife.'

Nominalized clauses are used as complements of various complement-taking verbs (\sectref{sec:8.3.2}). 

\ea%x1246
\label{ex:x1246}
\gll \textstyleEmphasizedVernacularWords{Miiw-aasa  muf-owa  ikiw-owa } na-em-ik-omkun  \\
      \\
\glt
\z

land-canoe  pull-NMZ  go-NMZ  say-SS.SIM-be-1s/p.DS

o  ar-e-k.

3s.UNM  die-PA-3s

`As we were talking about going to get a vehicle, she died.'

\ea%x1248
\label{ex:x1248}
\gll \textstyleEmphasizedVernacularWords{Maa  uup-owa}  paayar-ep  ep-a-n. \\
      \\
\glt
\z

food  cook-NMZ  know-SS.SEQ  come-PA-2s

`You know cooking and you came.'

A nominalized clause is sometimes used to express habituality. It indicates a more deliberate and permanent habit than that expressed by the continuous aspect, which is the default marking for the habitual (\sectref{sec:3.8.5.1.1.2}). 

\ea%x1249
\label{ex:x1249}
\gll Wi  mua  \textstyleEmphasizedVernacularWords{naap  me  onowa  nain=ko}  ik-e-mik=i? \\
      \\
\glt
\z

3p.UNM  man  thus  not  do-NMZ  that1=NF  be-PA-1/3p=QM

`Are there people who wouldn't keep doing like that?'

\ea%x1250
\label{ex:x1250}
\gll Mua  papako  \textstyleEmphasizedVernacularWords{opor(a)  makena  me  ookowa},  sira  samora  \\
      \\
\glt
\z

man  other  talk  true  not  follow-NMZ  custom  bad

on-am-ika-i-mik.

do-SS-SIM-be-Np-PR.1/3p

`Some people (as a rule) do not follow the true talk (but) keep doing bad things.'

Mauwake verbs may take a causative suffix, which often indicates causation (\sectref{sec:3.8.4.3.1}). When the causation is less mechanical and requires the cooperation of the object of the causation, the verb \textstyleStyleVernacularWordsItalic{suuw}- `push' is used together with a nominalized clause:

\ea%x1252
\label{ex:x1252}
\gll Mua  naareke  \textstyleEmphasizedVernacularWords{naap  on-owa}  nefa  suuw-a-k? \\
      \\
\glt
\z

man  who.CF  thus  do-NMZ  2s.ACC  push-PA-3s

`Who made you do like that?'

Ability is expressed via a nominalized clause followed by the intensity adverb \textstyleStyleVernacularWordsItalic{pepek} `enough, able'.

\ea%x1251
\label{ex:x1251}
\gll \textstyleEmphasizedVernacularWords{Ariwa  perek-owa}  me  pepek. \\
      \\
\glt
\z

arrow  pull.out-NMZ  not  enough/able

`(He was) not able to pull out the arrow.'

One strategy for purposives is to use the nominalized form of the main verb followed by the same-subject sequential form \textstyleStyleVernacularWordsItalic{naep} of the verb `say/think'. This strategy is used especially when the purpose understood to be somewhat generic (\stepcounter{nx}{\thenx}) or when the purpose clause is right-dislocated (\stepcounter{nx}{\thenx}). For purpose clauses, see \sectref{sec:8.3.2.1.4}.

\ea%x1253
\label{ex:x1253}
\gll Weniwa=pa  \textstyleEmphasizedVernacularWords{en-owa  na-ep}  uuw-i-mik. \\
      \\
\glt
\z

hunger.time=LOC  eat-NMZ  say-SS.SEQ  plant-Np-PR.1/3p

`We/they work in order to eat during the time of hunger.'

\ea%x1255
\label{ex:x1255}
\gll Ona  siowa  ikos  manina  ikiw-e-mik,  \textstyleEmphasizedVernacularWords{pika} \\
      \\
\glt
\z

3s.GEN  dog  together.with  garden  go-PA-1/3p  fence

\textstyleEmphasizedVernacularWords{on}\textstyleEmphasizedVernacularWords{-}\textstyleEmphasizedVernacularWords{owa  na}\textstyleEmphasizedVernacularWords{-}\textstyleEmphasizedVernacularWords{ep}.

make-NMZ  say-SS.SEQ

`He went together with his dog to the garden (in order) to make a fence.'

Mauwake has an idiosyncratic clausal structure for the expression `not yet'. The negated verb is nominalized, and it is followed by an appropriate form of the verb \textstyleStyleVernacularWordsItalic{ik}- `be'.  The presence of the negative temporal adverb \textstyleStyleVernacularWordsItalic{eewuar} `not yet' indicates expectation that what hasn't happened yet will, or should, take place in not too distant future.  

\ea%x1254
\label{ex:x1254}
\gll Aakisa  baliwep  \textstyleEmphasizedVernacularWords{me  amis-ar-owa  ik-e-mik}. \\
      \\
\glt
\z

now  well  not  knowledge-INCH-NMZ  be-PA-1/3p

`Now we do not yet know it well.'

\ea%x1256
\label{ex:x1256}
\gll Iwera  popoka  wafur-am-ika-iwkin  or-op  `bulak',  \\
      \\
\glt
\z

coconut  unripe  throw-SS.SIM-be-2/3p.DS  descend-SS.SEQ  plop

eewuar,  eka  \textstyleEmphasizedVernacularWords{me  saan-ar-owa  ik-ua}.

not.yet  water  not  dry-INCH-NMZ  be-PA.3s

`They\textsubscript{i} kept throwing unripe coconuts\textsubscript{j} and when they\textsubscript{j} dropped they\textsubscript{j} said `plop' (so they\textsubscript{i} knew:) not yet, the water had not dried yet.'

Unlike many other languages, in Mauwake a nominalized clause does not function as a complement of an adjective. Rather, the nominalized clause functions as the subject and it takes the adjective as a non-verbal predicate: 

\ea%x1258
\label{ex:x1258}
\gll \textstyleEmphasizedVernacularWords{Maa  wiar  ikum  aaw-owa}  eliwa=ki? \\
      \\
\glt
\z

thing  3.DAT  illicitly  take-NMZ  good=CF.QM

`Is it good to steal?'

\ea%x1259
\label{ex:x1259}
\gll \textstyleEmphasizedVernacularWords{Galasim-owa}{\footnotemark}  lawisiw  yoowa. \\
      \\
\glt
\z

spear.fish-NMZ  rather  hot/hard

`Spearing fish is rather hard.' Or: `It is rather hard to spear fish.'

\footnotetext{The verb for spear-fishing is a loan from Tok Pisin, which refers to the goggles used when diving to spear fish.}

\subsection{Type 2: with a finite verb}
\hypertarget{RefHeading22281935131865}{}
The second strategy for nominalizing a clause is to end an ordinary verbal clause with the far demonstrative \textstyleStyleVernacularWordsItalic{nain} `that' used as a determiner. The demonstrative is the only element marking the clause as nominalized. Comrie and Thompson call this type ``clausal nominalization'' (2007:376-377). Giv\'on (1990:506) suggests that there may be a correlation ``between the \textstyleEmphasizedWords{\textsc{degree of nounhood}} of a nominalized expression and its ability to take determiners''. In Mauwake this is clearly not the case, as the demonstrative is obligatory in this second type of nominalized clause but only optional in the first type, which is otherwise more like a \textstyleAcronymallcaps{NP}.

The distribution of finite clauses nominalized only with a demonstrative is more restricted than that of clauses with a nominalized verb. They function as relative clauses (\sectref{sec:8.3.1}), complement clauses (\sectref{sec:8.3.2}), or temporal clauses (\sectref{sec:8.3.3.1}), but not in the many other specific functions where the other type can occur. Forming complement clauses and relative clauses by adding a demonstrative pronoun after a finite verb is a common strategy in Papuan languages (Reesink 1983b and 1987:228, Farr 1999:77, Whitehead 2004:192). 

\ea%x1260
\label{ex:x1260}
\gll [\textstyleEmphasizedVernacularWords{Takira}  \textstyleEmphasizedVernacularWords{en-ow}(\textstyleEmphasizedVernacularWords{a})  \textstyleEmphasizedVernacularWords{gelemuta  wia  on-om-a-mik} \\
      \\
\glt
\z

child  eat-NMZ  small  3p.ACC  make-BEN-BNFY2.PA-1/3p

\textstyleEmphasizedVernacularWords{nain}]\textsubscript{CC}  ma-i-yem.

that1  say-Np-PR.1s

`I tell about our making/having made a feast for the children.'

\ea%x1261
\label{ex:x1261}
\gll [\textstyleEmphasizedVernacularWords{Akia}  \textstyleEmphasizedVernacularWords{ik-e-k  nain}]\textsubscript{RC}  me  en-e-k. \\
      \\
\glt
\z

banana  roast-PA-3s  that1  not  eat-PA-3s

`He did not eat the bananas that he roasted.'

\ea%x1940
\label{ex:x1940}
\gll [\textstyleEmphasizedVernacularWords{Koora}  \textstyleEmphasizedVernacularWords{ikiw}\textstyleEmphasizedVernacularWords{-}\textstyleEmphasizedVernacularWords{i}\textstyleEmphasizedVernacularWords{-}\textstyleEmphasizedVernacularWords{mik  nain}]\textsubscript{TempC}  mera  eka  me  enim-i-mik. \\
      \\
\glt
\z

house  go-Np-PR.1/3p  that1  fish  water  not  eat-Np-PR.1/3p

`When/After we go into the house, we do not eat fish soup.'

