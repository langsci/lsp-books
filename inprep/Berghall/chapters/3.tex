%3

\chapter{Morphology}
\hypertarget{RefHeading19221935131865}{}
\section{Introduction}
\hypertarget{RefHeading19241935131865}{}
A grammatical word in Mauwake is defined on the basis of the following main criteria quoted from \citet[12-14]{Dixon2010b}:

A grammatical word


\begin{itemize}
\item has as its base one or more lexical roots to which morphological processes apply;

\item has a conventionalized coherence and meaning.


\end{itemize}
When a grammatical word involves compounding or affixation, its component grammatical elements 


\begin{itemize}
\item always occur together;

\item generally occur in a fixed order


\end{itemize}
The following supplementary criteria also apply. A word only allows one inflectional affix of any one type (ibid. 15). Also in derivation recursiveness is blocked except in the case of causatives (ibid. 16-17). Even here the recursion is more ostensible than real, as it does not add another argument into the clause (\sectref{sec:3.8.2.3.1}). Person/number suffixes act as word-final boundary markers in finite verbs (ibid. 17). Many words, especially those belonging to the major word classes, ``may constitute a complete utterance'' (ibid. 19) by themselves. 

The boundaries of the grammatical and phonological words coincide, except in the case of clitics. Grammatically a clitic is a word but phonologically it is bound to the preceding word.

The classes of nouns, adjectives, personal pronouns, quantifiers, verbs and adverbials can be reasonably clearly defined both morpho-syntactically and semantically. The classes of question words and deictics include words with  heterogeneous syntactic behaviour; question words have semantic and functional, and some morphological similarities as a group, whereas the category of deictics is based on strong morphological and semantic similarities. Connectives share the function of conjoining elements on the same level. As ``functor words'' postpositions and especially clitics are dependent on the preceding phrase. Interjections are different from all the other word classes in that they operate outside the normal syntax and often constitute a whole expression by themselves.

Nouns are naturally the largest category, but verbs are morphologically the most complex and interesting word class.

Although the great majority of the words in Mauwake can be assigned to one of the categories above, there is some indeterminacy with regard to some words that seem to belong to two or more word classes and the meanings which are clearly related.\footnote{In Austronesian languages it is common to have pre-categorial stems that may combine with affixation belonging to various word classes; only the whole word may be assigned to a particular word class.} They are not homonyms, since they are semantically related. Some transitive verbs have been derived by zero derivation from nouns and adjectives, and even from adverbs (\sectref{sec:3.8.2.2.1}, 3.8.4.4.3). Nominalized verbs (\sectref{sec:3.2.6.1}) function as nouns or adjectives. At the end of section 3.2.2 there is a list of words that are originally nouns but have become adjectives as well. Some non-numeral quantifiers (\sectref{sec:3.4.2}) also function as intensity adverbs (\sectref{sec:3.9.2}). Besides these, there are individual words that function in more than one word class; these are mentioned where they occur.

\section{Nouns}
\hypertarget{RefHeading19261935131865}{}
\subsection{General discussion}
\hypertarget{RefHeading19281935131865}{}
Although the traditional semantic definition of the noun as the ``name of a person, place or thing'' is not valid as a basis for assigning members to the class, it still gives a good general description of the prototypical members of the class in Mauwake. In Frawley's (1992:63) words, ``when the traditional definition is reversed, the definition turns out to be true. Nouns are not always persons, places or things, but persons, places and things always turn out to be nouns.''.\footnote{See also \textstyleBibliogBaseChar{Sapir 1921:117}, Jespersen 1924:60, Lyons 1977:449 and Schachter 1985:7.} Recognizing the semantic motivation of the class does not eliminate the need to define the class by its formal or functional  properties.

No good morphological definition of nouns is possible in Mauwake, as there is no inflection for number (\stepcounter{nx}{\thenx}), gender or class,\footnote{Gender or class systems are widespread in Papuan languages \citep[77]{Foley1986}. Especially in the TNG languages a covert system is common \citep[58]{Wurm1982}, where the noun class determines what existential verb is used with each noun.}  or case, in the noun itself. Especially the lack of plural marking is typical of the nouns in Trans-New Guinea languages \citep[36]{Wurm1982}. The glosses in the following example indicate a singular/plural alternative in the nouns, but the singular form in the glosses of other examples is to be understood as neutral regarding the number. 

\ea%x1
\label{ex:x1}
\gll siowa  wiawi \\
      \\
\glt
\z

dog(s)  father(s)

`The dog's/dogs' owner(s)'

Nouns are usually monomorphemic, with the exception of a small group of inalienably possessed nouns (\sectref{sec:3.2.4}), nouns derived from verbs (\sectref{sec:3.2.6.1}), reduplicated nouns (\sectref{sec:3.2.6.2}) and compound nouns (\sectref{sec:3.2.5}). The division into \textstyleEmphasizedWords{count} and \textstyleEmphasizedWords{mass} nouns is not very noticeable. It is mainly shown in the choice between the quantifiers \textstyleStyleVernacularWordsItalic{unowa} `many' and \textstyleStyleVernacularWordsItalic{maneka} `big, much', and to some extent in verb agreement morphology (\sectref{sec:3.4}).

The syntactic function provides the best criterion for defining a noun in Mauwake. Nouns function mainly as the head of a noun phrase, often the head being the only element in the \textstyleAcronymallcaps{NP}.\footnote{Sometimes an adjective, a quantifier or a genitive pronoun looks like a head of a NP, but those cases are elliptical, and the head noun is recoverable.} They can also function as a qualifier or, more rarely, as a modifier in a \textstyleAcronymallcaps{NP}. In (\stepcounter{nx}{\thenx}) \textstyleAcronymallcaps{NP}s, in this case manifested by just nouns, function as subject and object.

\ea%x2
\label{ex:x2}
\gll Emeria=ke  iwera  fiirim-i-mik. \\
      \\
\glt
\z

woman=CF  coconut  gather-Np-PR.1/3p

`(The) women gather coconuts.'

Hopper and \citet[710]{Thompson1984} also maintain that \textstyleBibliogCitationAAAstyleChar{``from the discourse point of view, nouns function to introduce participants and `props' and to deploy them''}\footnote{Actually this is the function of a NP rather than a noun.}\textstyleBibliogCitationAAAstyleChar{.} This is true in Mauwake as well, but it is not used as a criterion for defining the nouns.

\subsection{Nouns and adjectives: one or two word classes?}
\hypertarget{RefHeading19301935131865}{}
Since adjectives in Mauwake are phonologically, morphologically and syntactically very similar to nouns, the question must be asked whether the two form just one class of nominals or whether they belong to two separate word classes. In the following discussion they are treated on a semantic basis as if they were separate classes, i.e. certain words are called nouns and others adjectives, but a final conclusion as to their status is not drawn until the end of the section.

A \textstyleEmphasizedWords{\textsc{phonologically}} interesting feature common to nouns and adjectives is that the majority of both end in the vowel /a/.\footnote{In the other word classes words ending in /a/ do occur but they are very infrequent.} Inside noun phrases this vowel, when unstressed, is usually elided preceding a vowel and often also preceding a consonant. In cases like (\stepcounter{nx}{\thenx}), where there are two or more possible places for elision, the vowel most easily drops at the end of an adjective preceding an intensifier. Elision is also acceptable in two or more sites within one \textstyleAcronymallcaps{NP} (\stepcounter{nx}{\thenx}), (\stepcounter{nx}{\thenx}). 

\ea%x3
\label{ex:x3}
\gll koora  eliw(a)  akena,   also:  koor(a)  eliw(a)  akena  \\
      \\
\glt
\z

house  good  very

`a very good house'

\ea%x4
\label{ex:x4}
\gll koor(a)  kemena  manek(a)  akena  nain \\
      \\
\glt
\z

house  inside  big  very  that1

`the very big room'

\textstyleEmphasizedWords{\textsc{Morphologically}} nouns and adjectives resemble each other in that they lack inflection. There is no number, case, or gender marking in the adjectives, nor is there any inflection for comparison. (For comparison of adjectives, see \sectref{sec:6.5}). 

Both nouns (\stepcounter{nx}{\thenx}) and adjectives (\stepcounter{nx}{\thenx}) may be derived from verbs with the nominaliser suffix \nobreakdash-\textstyleStyleVernacularWordsItalic{owa}.

\ea%x7
\label{ex:x7}
\gll mua  \textstyleEmphasizedVernacularWords{soop-owa}  sira \\
      \\
\glt
\z

man  bury-NMZ  custom

`the burial custom (lit: the custom of burying men)'

\ea%x8
\label{ex:x8}
\gll Emi  \textstyleEmphasizedVernacularWords{kekan-owa}  nain  puuk-a-mik. \\
      \\
\glt
\z

taboo  be.strong-NMZ  that1  cut-PA-1/3p

`They broke the strong taboo rule.'

Verbs can be derived from both adjectives and nouns by zero verb formation (\stepcounter{nx}{\thenx}), (\stepcounter{nx}{\thenx}) or by the inchoative verbaliser \nobreakdash-\textstyleStyleVernacularWordsItalic{ar}  (\textstyleParagraphCharChar{\stepcounter{nx}{\thenx}}). (See \sectref{sec:3.8.2.2} for these processes and more examples.)

\ea%x482
\label{ex:x482}
\gll Miiw-aasa  samor-a-k. \\
      \\
\glt
\z

land-canoe  bad-PA-3s

`He broke/ruined the car.'

\ea%x484
\label{ex:x484}
\gll Iwer(a)  ififa  palis-i-ya. \\
      \\
\glt
\z

coconut  dry  pair.of.coconuts-Np-PR.3s

`He is tying dry coconuts into pairs.'

\ea%x483
\label{ex:x483}
\gll Miiw-aasa  samor-ar-e-k. \\
      \\
\glt
\z

land-canoe  bad-INCH-PA-3s

`The car broke.'

A clear morphological \textstyleEmphasizedWords{difference} between nouns and adjectives is that adverbs may be formed from some adjectives by deleting the word-final /a/, but they cannot be formed from nouns in the same way.

\ea%x19
\label{ex:x19}
\gll samora    {{\textgreater}  samor} \\
      \\
\glt
\z

`bad'      `badly'

\textstyleEmphasizedWords{\textsc{Syntactically}} there are a few similarities between nouns and adjectives. Both can function as a modifier following the head noun in a \textstyleAcronymallcaps{NP}, although adjectives (\stepcounter{nx}{\thenx}) are much more common in this position. In \textstyleBibliogBaseChar{Hopper and Thompson's (1985:161)} terms, it is nouns whose categorial status has been reduced, i.e. nouns that are not fully individuated in the discourse (\stepcounter{nx}{\thenx}), that can function in this modifier position.

\ea%x9
\label{ex:x9}
\gll aasa  \textstyleEmphasizedVernacularWords{awona}  fain \\
      \\
\glt
\z

canoe  old  this

`this old canoe'

\ea%x10
\label{ex:x10}
\gll mua  \textstyleEmphasizedVernacularWords{sira  eliwa} \\
      \\
\glt
\z

man  manner  good

`a well-mannered man (=a good man)'

The intensifier \textstyleStyleVernacularWordsItalic{akena} `real(ly), very' can also modify both adjectives (\stepcounter{nx}{\thenx}) and nouns (\stepcounter{nx}{\thenx}).

\ea%x11
\label{ex:x11}
\gll mua  \textstyleEmphasizedVernacularWords{akena} \\
      \\
\glt
\z

man  real/true

`a real man'

Complete or partial reduplication of adjectives is a common strategy for indicating plurality in Austronesian languages \citep[62]{Wurm1982}, and it also occurs to some extent in many Papuan languages, including Mauwake.  Reduplication is a more productive process in the adjectives (\stepcounter{nx}{\thenx}), (\stepcounter{nx}{\thenx}), but it is possible for a few  nouns too (\stepcounter{nx}{\thenx}), (\stepcounter{nx}{\thenx}) (\sectref{sec:3.2.6.2}).  

\ea%x12
\label{ex:x12}
\gll ifa  \textstyleEmphasizedVernacularWords{samo-samora} \\
      \\
\glt
\z

snake  RDP-bad

`bad snakes'

\ea%x481
\label{ex:x481}
\gll Maa  \textstyleEmphasizedVernacularWords{ele-eliwa}  sesek-a-mik. \\
      \\
\glt
\z

thing/food  RDP-good  sell-PA-1/3p

`They sold good foods (different kinds).'

\ea%x13
\label{ex:x13}
\gll \textstyleEmphasizedVernacularWords{Owow-owowa}  ikiw-e-mik. \\
      \\
\glt
\z

RDP-village  go-PA-1/3p

`They went to many villages.'

\ea%x1859
\label{ex:x1859}
\gll \textstyleEmphasizedVernacularWords{sira-sira} \\
      \\
\glt
\z

custom-custom

`many customs', `different kinds'

The syntactic \textstyleEmphasizedWords{\textsc{differences}} between nouns and adjectives are as follows. Adjectives do not function as the head of a noun phrase. The cases where they would seem to do so are in fact cases of ellipsis, and the head noun must be recoverable from the context, either linguistic or extra-linguistic. 

\ea%x14
\label{ex:x14}
\gll {\O  awona  nain  p-ekap-e!} \\
      \\
\glt
\z

{\O}  old  that1  BPx-come-IMP.2s

`Bring the old one!'

Only a noun may occur as a qualifier in a noun phrase, preceding the head noun (\stepcounter{nx}{\thenx}).  In some of these cases it is difficult to decide whether they are really \textstyleAcronymallcaps{NP}s with a qualifier and a head noun, or compound nouns. But if the latter is the case, then the restriction applies that an adjective cannot be the first element in a compound noun.

\ea%x15
\label{ex:x15}
\gll \textstyleEmphasizedVernacularWords{mera}  eka \\
      \\
\glt
\z

fish  water

`fish soup'

\ea%x16
\label{ex:x16}
\gll [[\textstyleEmphasizedVernacularWords{mera  eka}]  \textstyleEmphasizedVernacularWords{en-owa}]  sira \\
      \\
\glt
\z

fish  water  eat-NMZ  custom

`the custom of eating fish soup'

An adjective cannot be the only element following a genitive pronoun, but a noun can. Even in elliptical expressions an adjective following a genitive pronoun is not very acceptable (\stepcounter{nx}{\thenx}). 

\ea%x17
\label{ex:x17}
\gll ?Yiena  {\O } \textstyleEmphasizedVernacularWords{awona}  nain  p-ekap-e! \\
      \\
\glt
\z

1p.GEN  {\O}  old  that1  BPx-come-IMP.2s

`Bring our old one(s)!'

An exception to this rule is the adjective \textstyleStyleVernacularWordsItalic{maneka} `big'. The expression \textstyleStyleVernacularWordsItalic{yiena Maneka} `our Lord' (literally: our Big one), is probably formed following Tok Pisin \textstyleForeignWords{Bikpela bilong yumi.}\footnote{Non-prototypical adjectives are discussed later in this section; `big' is a prototypical adjective, so its use in a typically nominal position is an exception.}  

\ea%x105
\label{ex:x105}
\gll wi  Amerika  \textstyleEmphasizedVernacularWords{maneka},  unuma  Magerka \\
      \\
\glt
\z

3p.UNM  America  big  name  MacArthur

`the leader of the Americans, whose name was MacArthur'

Only an adjective functions as the head of an adjective phrase. In that position it may be modified by intensity adverbs (\sectref{sec:3.9.2}).  Of these, \textstyleStyleVernacularWordsItalic{lawisiw} `rather' does not modify nouns at all (\stepcounter{nx}{\thenx}); \textstyleStyleVernacularWordsItalic{akena} `very' and \textstyleStyleVernacularWordsItalic{pepek} `enough' may modify nouns as well; \textstyleStyleVernacularWordsItalic{wenup} `very'can do that too, but as a noun modifier it has a somewhat restricted use and a different meaning, `many'.

\ea%x18
\label{ex:x18}
\gll Mera  nain  \textstyleEmphasizedVernacularWords{lawisiw  maneka  akena}. \\
      \\
\glt
\z

fish  that1  rather  big  very

`That fish is rather huge.'

What further obscures the area of nouns and adjectives is the fact that there are a number of words that sometimes function like nouns (\stepcounter{nx}{\thenx}), sometimes like adjectives (\stepcounter{nx}{\thenx}), and also semantically could be like either.

\ea%x20
\label{ex:x20}
\gll \textstyleEmphasizedVernacularWords{Pina}  maneka  kamenap? \\
      \\
\glt
\z

weight  big  what.like

`What is the weight like?', `How big is the weight?'

\ea%x21
\label{ex:x21}
\gll Maa  nain  lawisiw  \textstyleEmphasizedVernacularWords{pina}. \\
      \\
\glt
\z

thing  that1  rather  heavy

`The thing is rather heavy.

The prototype view offers a plausible solution for the problem. Starting from the study of basic colour terms (Berlin and Kay 1969) it has been applied to other areas of semantics and also to linguistic categorization (e.g. Wierzbicka 1986, Taylor 1989 and Frawley 1992). The main idea that categories have more central, or focal, members as well as more marginal members was also recognized by \citet{Crystal1967} in his description of English word classes. The prototype approach allows for stability as well as flexibility \citep[53]{Taylor1989}, both of which are needed in an attempt to describe a human language.

If prototypical linguistic categories are focal, or optimal, instances on a continuum \citep[321]{Seiler1978} and maximally distinct from one another \citep[709]{HopperEtAl1984}%Thompson
, what are prototypical nouns like as opposed to prototypical adjectives? According to \citet{Wierzbicka1986}, noun indicates \textstyleEmphasizedWords{\textsc{categor}}\textstyleEmphasizedWords{\textsc{ization}}: most prototypical nouns identify a certain kind of person, thing or animal. Relative \textstyleEmphasizedWords{\textsc{temporal stability}} is for Giv\'on what characterizes nouns, and the most prototypical nouns denote concrete, physical, compact entities (1984:51). Instead of time stability, \citet[66]{Frawley1992} claims it is relative \textstyleEmphasizedWords{\textsc{atemporality}}\textstyleEmphasizedWords{} that makes an entity an entity.  Adjectives, or property concepts, indicate \textstyleEmphasizedWords{\textsc{description}}, and they denote single properties unlike nouns which denote a cluster of properties \citep{Wierzbicka1986}.

In Mauwake, a prototypical noun occurs as a head in a \textstyleAcronymallcaps{NP}, as a pre-modifier or, less frequently, as a post-modifier in a \textstyleAcronymallcaps{NP}, or as any element in a compound noun. It does not occur as the head in an \textstyleAcronymallcaps{AP}. It can be modified by adjectives or genitive pronouns but not by the intensity adverbs \textstyleStyleVernacularWordsItalic{lawisiw} `rather' and \textstyleStyleVernacularWordsItalic{wenup} `very'. Prototypical \textstyleEmphasizedWords{\textsc{adjectives}}\textstyleEmphasizedWords{\textsc{} }occur\textstyleEmphasizedWords{\textsc{} }as the head of an adjective phrase. They do not pre-modify nouns or function as the first element in a compound noun.

It turns out that in Mauwake the most prototypical nouns include names of concrete \textstyleEmphasizedWords{\textsc{non}}-human rather than human objects, when one would expect words referring to human beings to be nouns \textstyleEmphasizedWords{\textsc{par excellence}} (see Taylor 1989:192). Some human nouns may be used as post-modifiers in a \textstyleAcronymallcaps{NP}: from the cluster of properties denoted by the noun one has been picked out, and the noun is used like an adjective (\stepcounter{nx}{\thenx}), (\stepcounter{nx}{\thenx}). The adjectival use of \textstyleStyleVernacularWordsItalic{mua} `man' in (\stepcounter{nx}{\thenx}) is particularly interesting, because the adjectives \textstyleStyleVernacularWordsItalic{morena} `male' and \textstyleStyleVernacularWordsItalic{suwina} `female' are used for animals.

\ea%x23
\label{ex:x23}
\gll labuel(a)  mua \\
      \\
\glt
\z

pawpaw  man

`male pawpaw'

\ea%x24
\label{ex:x24}
\gll donki  takira \\
      \\
\glt
\z

donkey  young.person

`young donkey'

The less prototypical status of human nouns also shows in words like \textstyleStyleVernacularWordsItalic{apura} `widow' and \textstyleStyleVernacularWordsItalic{oosa} `widower' which may occur by themselves as heads of a \textstyleAcronymallcaps{NP}, but which are most typically used as post-modifiers of \textstyleStyleVernacularWordsItalic{emeria} `woman' and \textstyleStyleVernacularWordsItalic{mua} `man', respectively.\footnote{Other words in this group are \textstyleFootnoteBaseChar{\textit{muupera}}\textbf{\textit{} }`visitor, guest' and especially \textstyleFootnoteBaseChar{\textit{weria}}, which as a human noun only occurs in the combination \textstyleFootnoteBaseChar{\textit{mua weria}}, `uncle/ male cross cousin/ nephew'. The \textstyleFootnoteBaseChar{\textit{mua weria'}}s are responsible for burying a dead person and dispensing of his/her belongings (1.3.6).} As age in human beings tends to be to be treated as a crucial determinant of \textstyleEmphasizedWords{\textsc{kind}}, even languages with large adjective classes often have special nouns for referring to old persons \citep[368]{Wierzbicka1986}. In Mauwake, adjectives that indicate age in humans are non-prototypical, more noun-like than most adjectives: both \textstyleStyleVernacularWordsItalic{iperowa} `middle-aged' and \textstyleStyleVernacularWordsItalic{panewowa} `old' are used as the head of a \textstyleAcronymallcaps{NP} besides the typical adjectival use.

\ea%x25
\label{ex:x25}
\gll \textstyleEmphasizedVernacularWords{Iperowa}  opora  wiar  miim-i-yen. \\
      \\
\glt
\z

middle-aged  talk  3.DAT  hear-Np-FU.1p

`We will listen to the talk of the middle-aged (men).'

According to \citet[56]{Dixon1977}, if a language has adjectives at all, words expressing age, dimension, value and colour are likely to belong to the adjective class, however small the class. The most prototypical adjectives in Mauwake belong to these groups, with the exception of adjectives denoting human age, discussed above. In the group of adjectives denoting either physical property or human propensity, some are ambiguous as to their basic category: \textstyleStyleVernacularWordsItalic{anima} is both `blade' and `sharp', and \textstyleStyleVernacularWordsItalic{pina}  both `weight, burden' and `heavy'. Different groups of adjectives, as well as the use of adjectives, are discussed below in Section 3.3.

With the rules given above it is fairly straightforward to distinguish the nouns and adjectives in Mauwake. But a small group remains that seems to have a membership in both classes. Originally they are are nouns that have now been employed as adjectives as well. The claim is based on the fact that the noun category is the more basic and universally  recognized, whereas the existence of the adjective category is disputed in some languages; and in Mauwake the noun class is clearly established, large, and more easily definable.  Also, there are at least two nouns in Mauwake that currently seem to be in the process of becoming regular adjectives: the meaning of the phrase stays the same with the pre-modifying noun and the post-modifying adjective. 

  (\stepcounter{nx}{\thenx}x108)  \textstyleEmphasizedVernacularWords{napum(a)}  mua

sickness  man

`a sick man'

\ea%x107
\label{ex:x107}
\gll mua  \textstyleEmphasizedVernacularWords{napuma} \\
      \\
\glt
\z

man  sick

`a sick man', also: `human (lit: man's) sickness'

\ea%x1822
\label{ex:x1822}
\gll \textstyleEmphasizedVernacularWords{wadol(a)}  opora \\
      \\
\glt
\z

lie/false  talk

`a lie'

\ea%x1823
\label{ex:x1823}
\gll opor(a)  \textstyleEmphasizedVernacularWords{wadola} \\
      \\
\glt
\z

talk  lie/false

`a lie'

Below is a list of the most common of the words functioning both as nouns and as adjectives:

anima  `blade, point, edge'  `sharp'

afila  `grease'  `greasy, sweet'

foma  `ashes'  `grey'

ikina  `smell'  `smelly'

irauwa  `hole'  `deep'

makena  `true'  `truth, essential nature'

napuma  `sickness, corpse'  `sick'

pina  `weight, burden, guilt'  `heavy'

siisia  `design, pattern'  `spotted, patterned'

tumina  `dirt'  `dirty'

wadola  `lie'  `false, fake'

\subsection{Common  vs.  proper nouns}
\hypertarget{RefHeading19321935131865}{}
There is very little difference between common and proper nouns in Mauwake, and it can be questioned whether the two should be grouped separately as is traditionally often done in language descriptions. Proper nouns are sometimes classified separately because they are said to be unable to have modifiers \citep[152]{Roberts1987}, and in practice, they usually occur without any modifiers. This is related to the fact that they normally only have a referent, but no intension. In most of the cases where a proper noun is modified, ``it lacks a unique reference and is being used as a common noun'' (Van Valin and LaPolla 1997:59):

  (\stepcounter{nx}{\thenx}x26)  I mean the old and cranky Joe Smith, not the younger one. 

The most common type of a proper noun is a name of a \textstyleEmphasizedWords{\textsc{person}}. A proper noun may also become a true common noun, when one or more of the qualities of a person are used to characterise some other being \citep[66]{Jespersen1924}. For example, the name of a well-known expatriate, Jooren, was borrowed by Mauwake speakers to mean `a stingy shopkeeper' (that is, one who does not sell things on credit and does not give discount to relatives). 

In Mauwake proper names can be modified without difficulty, especially by the demonstrative \textstyleStyleVernacularWordsItalic{nain} `that', but also by adjectives. In a culture where there are several namesakes, and surnames are rarely used, modifiers are occasionally needed to distinguish between people (\stepcounter{nx}{\thenx}).

\ea%x27
\label{ex:x27}
\gll \textstyleEmphasizedVernacularWords{Adek  panewowa  nain}  ma-i-yem. \\
      \\
\glt
\z

Adek  old  that1  say-Np-PR.1s

`I am talking about the \textstyleEmphasizedWords{old} Adek.'

But even proper names that have a unique reference and do not need to be distinguished from any other referent can be modified:

\ea%x106
\label{ex:x106}
\gll \textstyleEmphasizedVernacularWords{Dabe  fain}  uuw-ow(a)  mua=ke. \\
      \\
\glt
\z

Dabe  this  work-NMZ  man=CF

`Dabe here is a hard worker.'

In this case the behaviour of proper names is similar to that of the personal pronouns, which also have unique reference, but can be modified nevertheless. \textstyleBibliogBaseChar{Van Valin and LaPolla} (ibid. 59-60) note that languages may vary in how freely they allow proper nouns and pronouns to take modifiers.

Name taboos influence the use of personal names in several ways. A person is given many different names: at least one from each parents' side (as in-laws may not mention each others' names), a baptismal name, and possibly others as well. These names are used by different people. Name taboos may be avoided by calling someone by a teknonym like `Sarak's father', or by calling a wife by the husband's name when she is with the in-laws and the husband is not around. Nicknames, often referring to physical properties, are also very common: \textstyleStyleVernacularWordsItalic{buburia} `bald', \textstyleStyleVernacularWordsItalic{mua kuuma} `lame' (literally `stick-man'). The term `namesake' is very common and even used of people who have been named after different names of the same person.  Two boys, Yoli and Wangali, were called namesakes of each other, as they were both named after the same ancestor. 

Perhaps the most characteristic feature of personal names is \textstyleEmphasizedWords{\textsc{discourse-pragmatic}}: in a text their token frequency is very low. Especially the main participant, once (s)he has been mentioned by name -- if (s)he ever is -- (s)he is then usually referred to by other means: a \textstyleAcronymallcaps{NP}, pronoun, or just person marking on the verb.  

Besides the names of people, \textstyleEmphasizedWords{\textsc{place names}} form another large group of proper names. In Mauwake, the proper name often modifies a generic noun:  \textstyleStyleVernacularWordsItalic{Moro} (\textstyleStyleVernacularWordsItalic{owowa}) `Moro (village), \textstyleStyleVernacularWordsItalic{Siburten} (\textstyleStyleVernacularWordsItalic{ema}) `Siburten (mountain/hill)', \textstyleStyleVernacularWordsItalic{Nemuru} (\textstyleStyleVernacularWordsItalic{eka}) `Nemuru (river)' (\sectref{sec:4.1}). 

The place name is also used when the inhabitants are referred to. When reference is made to an individual or a select group, the place name is used as a qualifier in the noun phrase: 

\ea%x421
\label{ex:x421}
\gll \textstyleEmphasizedVernacularWords{Amiten } mua  oko  ekap-o-k. \\
      \\
\glt
\z

Amiten  man  other  come-PA-3s

`A man from Amiten came.'

When the whole group is referred to, a plural pronoun is added to the place name:

\ea%x422
\label{ex:x422}
\gll \textstyleEmphasizedVernacularWords{I}  \textstyleEmphasizedVernacularWords{Moro=ke}  uf-e-mik. \\
      \\
\glt
\z

1p.UNM  Moro=CF  dance-PA-1/3p

`We Moro people danced.'

\ea%x423
\label{ex:x423}
\gll \textstyleEmphasizedVernacularWords{(Wi)  Lasen  wia}  nokar-e-k.\footnotemark{} \\
      \\
\glt
\z

3p.UNM  Lasen  3p.ACC  ask-PA-3s

`He asked the Lasen people'

\footnotetext{The optional initial pronoun \textit{wi} is part of the object here, not a subject pronoun.}

\subsection{Alienable and inalienable possession}
\hypertarget{RefHeading19341935131865}{}
The Austronesian languages in Melanesia tend to have very elaborate semantically based possessive systems that indicate the relationship between the ``possessor'' and the ``possession'': kin relation, body part, food etc.  Inalienable possession is indicated by affixation on the noun, alienable possession by a separate possessive pronoun. Because of this, the simpler inalienable possession marking also evident in many \textstyleAcronymallcaps{TNG} languages could easily be attributed to influence from Austronesian languages.  But \citet[28]{Ross1996} claims it is likely that even Proto \textstyleAcronymallcaps{TNG} had inalienable nouns before there was any contact with Austronesian languages.\footnote{On the time frames of TNG occupation and Austronesian migration, see e.g. \citet[39-41]{Ross2005}.} In Mauwake the division into alienably and inalienably possessed nouns is along the lines of kinship terms (see \sectref{sec:1.3.6} for a kinship chart). Most kin terms obligatorily indicate who the {\textquotedbl}possessor{\textquotedbl} is:

  1s/p  2s/p  3s/p  possessor  

a.  auwa  niawi  wiawi  `father'

b.  aite  niena  onak  `mother'

c.  paapa  neepe  weepe  `elder sibling'

d.  (y)aamun  niamun  wiamun  `younger sibling'

e.  yaaya  nie  wie  `uncle'

f.  paapan  noopan  woopan  `aunt'

g.  kae  neke  weke  `grandfather'

h.  kome  nokome  wokome  `grandmother'

i.  eremena  neremena  weremena  `nephew, niece'

j.  emar, yomar  nomar  womar  `(cross-)cousin'

k.  yomokowa  nomokowa  womokowa  `brother'\footnote{Among siblings, age is more important than sex: \textstyleFootnoteBaseChar{\textit{paapa}} and \textstyleFootnoteBaseChar{\textit{aamun}} are used very frequently and for siblings of either gender. When the gender is in focus, \textstyleFootnoteBaseChar{\textit{yomokowa}} is used for `my brother' and \textstyleFootnoteBaseChar{\textit{ekera}} for `my sister' especially by siblings of the opposite sex.}

l.  (y)ekera  nekera  wekera  `sister'

m.  (y)emi  nemi  wemi  `(man's) brother-in-law'

n.  epua  nepua  wepua  `(woman's) brother-in-law\footnote{A woman calls her elder sister's husband \textit{auwa} `father', but the other brothers-in-law are \textit{epua}.}

o.  yomora  nomora  womora  `sister-in-law'

p.  yopariw  nopariw  wopariw  `husband's brother's wife'

q.  yamekua  namekua  wamekua  `daughter-in-law'\footnote{Some in-law relations are non-symmetrical: even though there are special terms for sons- and daughters-in-law,  \textstyleFootnoteBaseChar{\textit{auwa}} `(my) father' and \textstyleFootnoteBaseChar{\textit{aite}} `(my) mother' are used for `(my) mother-in-law' and `(my) father-in-law'.}

r.  yar  nar  war  `son-in-law'

s.  yookati  nookati  wookati  `co-wife'\footnote{This term dates back to the time when polygamy was practiced; it was used for the wives of the same man.}

t.  yomawa  nomawa  womawa  `namesake'

The possessive prefixes  \textstyleStyleVernacularWordsItalic{y}-, \textstyleStyleVernacularWordsItalic{n}- and \textstyleStyleVernacularWordsItalic{w}- in the inalienably possessed nouns developed from the first, second, and third person pronouns. These prefixes are in the process of merging with the root. The terms in (a-j) above are somewhat more lexicalized than the ones in (k-s): the first person prefix is mostly lost, and in some cases there is suppletion in the stem. These are some of the socially most important and frequently used kinship terms.  The frequent use probably accounts for the omission of the possession prefix in the first person: these terms are used more as terms of address, whereas the other kinship nouns are only needed as terms of reference. Also, there is a tendency to drop the first person prefix before the front vowel /e/ regardless of the closeness of the kinship relation.

The ``possessors'' are differentiated as first, second or third person but not as single vs. plural. An unmarked (\stepcounter{nx}{\thenx}) or a genitive (\stepcounter{nx}{\thenx}), (\stepcounter{nx}{\thenx}) pronoun may be used to either make this number distinction or to emphasise the kin relationship, when the relationship  is used as a term of reference rather than as a term of address.

\ea%x1311
\label{ex:x1311}
\gll Kuuten  \textstyleEmphasizedVernacularWords{wiawi}  iperowa,  \textstyleEmphasizedVernacularWords{yo}  \textstyleEmphasizedVernacularWords{auwa}  kapa=ke. \\
      \\
\glt
\z

Kuuten  3s/p.father  firstborn  1s.UNM  1s/p.father  lastborn=CF

`Kuuten's father was the firstborn, my father the lastborn.'\footnote{Both of these fathers could be called \textit{auwa} `my/our father(s)' by the two men.}

\ea%x28
\label{ex:x28}
\gll Aakisa  \textstyleEmphasizedVernacularWords{yena} \textstyleEmphasizedVernacularWords{} \textstyleEmphasizedVernacularWords{auwa}  kapa  fain=ke  yia  uruf-i-ya. \\
      \\
\glt
\z

now  1s.GEN  1s/p.father  lastborn  this=CF  1p.ACC  see-Np-PR.3s

`Now this lastborn of my ``fathers'' watches over us.'

\ea%x1312
\label{ex:x1312}
\gll Sa,  a  \textstyleEmphasizedVernacularWords{nena  nie=ke},  \textstyleEmphasizedVernacularWords{nena  nepua=ke,} \\
      \\
\glt
\z

INTJ  INTJ  2s.GEN  2s/p.uncle=CF  2s.GEN  2s/p.brother-in-law

niawi=ke.

2s/p.father

`(Don't you understand,) those are \textit{your} uncle(-in-law), \textit{your} brother-in-law and father(-in-law).'

When a neutral, ``non-possessed'', kinship term is needed, the first person form is used. This is interesting, as the third person singular is typically considered the neutral, or unmarked, form. The terms `(my) mother' and `(my) father' are also used as respectful terms of address for almost any stranger regardless of age, or for anyone whose status in the kinship system is uncertain.\footnote{I have been addressed as {\textquotedbl}mother{\textquotedbl} by an old man who temporarily forgot what my status according to their kinship system was - I was actually his granddaughter!}

Four alienably possessed nouns, namely those for `man', `woman', `boy' and `girl', have been taken into the kinship system for terms of some nuclear family members: 

mua  `man, husband'

emeria  `woman, wife'

muuka  `boy, child, son'

wiipa  `girl, daughter'

Also the term \textstyleStyleVernacularWordsItalic{nembesir} `ancestor (beyond grandparents)' or `descendant (beyond grandchildren)' is an alienably possessed noun, possibly because relatives so far removed in time are considered less relevant. It is used both for males and females. But the term for `namesake', \textstyleStyleVernacularWordsItalic{yomawa}, is included in the inalienably possessed kinship terms, as a child is named after some relatives, and the namesake relation forms an additional bond between them.

\subsection{Noun compounding}
\hypertarget{RefHeading19361935131865}{}
The distinction between compound nouns and noun phrases is a problematic area in many languages, including Mauwake. Both are formed by combining independent elements into larger units, and their form and meaning are largely based on the form and meaning of those elements (\textstyleBibliogBaseChar{Anderson 1985a}:40). Phonological, morphological, syntactic as well as semantic criteria have been called upon to differentiate between compounds and phrases.

In many languages, ``word accent'' \citep[204]{Lyons1968}, i.e. stress and/or pitch, helps to distinguish compounds. In Mandarin Chinese, contrastive stress can only fall on the ``stress center'' of a word, including compounds (\textstyleBibliogBaseChar{Anderson 1985a}:41). In Finnish, the primary stress is on the first, and only on the first, syllable of even very long compound words like \textstyleForeignWords{kuluttajansuoja-asiamiesverkostokysymys} `the question of consumer ombudsman network', but even in Finnish there are unclear cases like \textstyleForeignWords{valveillaolo} vs. \textstyleForeignWords{valveilla olo}\textstyleEmphasizedWords{} `being awake'. In the latter, the varying writing convention reflects the ambiguity. 

Linguists differ in their views about the importance of stress placement in interpreting English compounds. \citet[228]{Bloomfield1935} and \citet[41]{Anderson1985a} consider it criterial, and so do \citet[1330]{QuirkEtAl1989}, although more cautiously. \citet[120]{Lees1968} takes it as one premise for his study of compounds while admitting that the case is not very well substantiated. Others, like \citet[31]{Jespersen1933}, \citet{Downing1977} and \citet{Bauer1983} do not consider a single primary stress essential for compounds. According to \citet[105]{Bauer1983}, Lyons' (1968:202) criteria for judging ``wordness'' in English, i.e. positional mobility and uninterruptability (or internal stability) do not distinguish between single- and double-stressed compounds.

Morphology may place constraints on compounding. In English, the genitive is common in phrases but rare in compounds: duck's egg vs. duck-egg (Anderson\textstyleBibliogBaseChar{ 1985a}:41).\footnote{But note also women's lib(eration), a compound.} In Finnish, the first part of a compound is often in the nominative or genitive case, whereas the other cases are infrequent in this position. In German, certain elements may serve as morphological ``glue'' between the parts of a compound (ibid. 42).

The two criteria for wordness by \citet[202]{Lyons1968} mentioned above are syntactic in nature: a word, hence also a compound, is moved as one unit, and cannot be interrupted by other words as a phrase often can. These criteria do not apply to all, and only, compound words, but they are useful in trying to establish the difference between compounds and phrases in a given language. \citet[232]{Bloomfield1933} adds another one: a member of a compound generally cannot serve as a constituent in a syntactic construction. One can say \textstyleEmphasizedWords{a very black bird} but not \textstyleEmphasizedWords{* a very blackbird}.

The semantic interpretation of phrases is generally quite compositional: the meaning of the whole can be deduced from the meanings of the words. Compounds are more heterogeneous in their interpretation: some are compositional, whereas others involve special interpretive principles not applicable to phrases. Also, compounds as words are subject to changes of meaning, so many compounds may have meanings that are only vaguely or metaphorically related to that which is predicted on the basis of the parts (Anderson\textstyleBibliogBaseChar{ 1985a}:42). Knowledge of the pragmatics of the situation may be needed for the interpretation of many compound words (Bauer\textstyleBibliogBaseChar{ 1983}:58). The more fully lexicalized the compounds are, the more the meaning of the whole may deviate from the meaning of the parts. The same compound word may also be fully lexicalized in a certain context, and still be open for other interpretations in other contexts (Andrew Pawley, p. c.). 

While there are languages where it is easy to distinguish between compound nouns and noun phrases, in others there is an intermediate area between the two. Thus \citet[810]{Downing1977} doubts that the dividing line is always well-defined, and Quirk et al\textstyleBibliogBaseChar{.} (1989:1569) suggest the concept of ``partial compounding'' to account for the formal and semantic gradience between compounds and phrases in English. Bringing a historical viewpoint to the question, citing developments in English both from phrase to compound and from compound to phrase, \citet[102]{Jespersen1924} offers a very liberal view: ``it is of no consequence whether we reckon [the] doubtful cases as one word or two words, for ... a word group (like a single word) may be either primary or an adjunct or a subjunct''. 

None of the criteria mentioned above can be easily applied in Mauwake. \textstyleEmphasizedWords{\textsc{Semantically}} there is a continuum between fully compositional noun phrases and fully lexicalized compounds. But \textstyleBibliogBaseChar{Bloomfield} (1933:227) warns that the greater specialization in meaning in the compound words as against phrases should not be used as a criterion, as ``we cannot gauge meanings accurately enough, and many a phrase is as specialized in meaning as any compound''. This warning is all the more relevant when one studies a language not one's own. 

The basic \textstyleEmphasizedWords{\textsc{stress pattern}} of noun phrases and compounds is similar, as one of the modifiers usually receives the phrase stress rather than the head noun (\stepcounter{nx}{\thenx}), (\stepcounter{nx}{\thenx}). Likewise, in compound nouns the modifying formative receives the main stress and the main formative is only weakly stressed (\stepcounter{nx}{\thenx}), (\stepcounter{nx}{\thenx}): the ``stress centre'' \citep[45]{Anderson1985a} is on another element than the head. 

\ea%x29
\label{ex:x29}
\gll yo  'auwa  aasa\footnotemark{} \\
      \\
\glt
\z

1s.UNM  1s/p.father  canoe

`my father's canoe'

\footnotetext{In the examples (\stepcounter{nx}{\thenx})-(\stepcounter{nx}{\thenx}) only the phrase stress is marked by ' preceding the stressed syllable.}

\ea%x30
\label{ex:x30}
\gll aas(a)  ge'lemuta \\
      \\
\glt
\z

canoe  small

`a small canoe'

\ea%x31
\label{ex:x31}
\gll 'miiw(a)-aasa\footnotemark{} \\
      \\
\glt
\z

\footnotetext{In Mauwake orthography, the parts of a compound word are usually written separately to help the new readers to identify the parts; \textit{miiw-aasa} `vehicle' is one of the exceptions.}

land-canoe

`vehicle, car'

\ea%x32
\label{ex:x32}
\gll enow(a)  ge'lemuta{\footnotemark} \\
      \\
\glt
\z

food/meal  small

`feast'
\footnotetext{\textit{Enow gelemuta} is not used with its literal meaning `small meal'.}
However, the head noun in a \textstyleAcronymallcaps{NP} may receive the phrase stress if it is emphasized for contrast, clarification or some other reason, whereas the stress centre in a compound stays the same. 

Since there is hardly any \textstyleEmphasizedWords{\textsc{morphology}} in nouns and noun phrases, one would not expect to find much help here in distinguishing between compounds and phrases. But there is a minor factor that is relevant in this respect: a phrase containing a noun and an adjective can be pluralized by adjectival reduplication when the adjective allows reduplication (\stepcounter{nx}{\thenx}), whereas a compound noun with a similar structure usually cannot (\stepcounter{nx}{\thenx}), even if it is possible in some rare cases (\stepcounter{nx}{\thenx}). 

\ea%x33
\label{ex:x33}
\gll maa  gelemuti-tik \\
      \\
\glt
\z

thing  small-RDP

`small things'

\ea%x34
\label{ex:x34}
\gll *enow(a)  gelemuti-tik \\
      \\
\glt
\z

food/meal  small

\ea%x35
\label{ex:x35}
\gll owow(a)  mane-maneka \\
      \\
\glt
\z

village  RDP-big

`towns', `big villages'

Uninterruptibility is more typical of compounds than phrases. The noun phrase \textstyleStyleVernacularWordsItalic{owow maneka} means `a big village', as a compound it means `a town/city'. As a phrase it is interruptible (\stepcounter{nx}{\thenx}), as a compound it is not.

\ea%x1768
\label{ex:x1768}
\gll owowa  lawisiw  maneka \\
      \\
\glt
\z

village  rather  big

`a rather big village'

Likewise, as a compound \textstyleStyleVernacularWordsItalic{kae sira} 'ancestral custom' (literally: `grandfather's custom') is uninterruptible. When  a genitive pronoun is inserted between the two parts, the meaning cannot be `ancestral custom':

\ea%x1860
\label{ex:x1860}
\gll kae  ona  sira \\
      \\
\glt
\z

grandfather  3s.GEN  custom

`grandfather's custom/habit'

In Mauwake word combinations are treated as compounds if they 1) have a specialized meaning, 2) have a stress centre not affected by contrastive stress, and 3) tend to be uninterruptible.  However, this distinction is very tentative in some cases. Some examples are provided where the same combination may be either a compound noun or a noun phrase.

Morphologically there are four compound noun types in Mauwake: \textstyleAcronymallcaps{N}+\textstyleAcronymallcaps{N,}  \textstyleAcronymallcaps{V}\textsubscript{NMZ}\textstyleAcronymallcaps{} +\textstyleAcronymallcaps{N,  N+V}\textsubscript{NMZ}\textstyleAcronymallcaps{ } and \textstyleAcronymallcaps{N}+\textstyleAcronymallcaps{ADJ}.  Syntactically these correspond to a head noun with a nominal pre- or post-modifier in a \textstyleAcronymallcaps{NP} or a head noun with an adjective post-modifier in the \textstyleAcronymallcaps{NP}. In most compound nouns the last noun is the head.  But in generic-specific compounds as well as the \textstyleAcronymallcaps{N}+\textstyleAcronymallcaps{ADJ} and\textstyleAcronymallcaps{} \textsc{N+V}\textsubscript{NMZ} compounds the first part is the main element and the scope of its meaning is restricted by the second part. In coordinate compounds the two parts are equally important.

On the basis of the semantic relations between the parts the \textstyleAcronymallcaps{N}+\textstyleAcronymallcaps{N} compounds can be divided into a few main groups. In the first one the relationship can be said to be characterized by \textstyleEmphasizedWords{\textsc{origin}} understood very widely, e.g. in the sense of place of origin (\stepcounter{nx}{\thenx}), source (\stepcounter{nx}{\thenx}), or ``possession'' (\stepcounter{nx}{\thenx}), (\stepcounter{nx}{\thenx}). 

\ea%x37
\label{ex:x37}
\gll piip(a)  mera \\
      \\
\glt
\z

seaweed  fish

`rainbow fish'

\ea%x40
\label{ex:x40}
\gll emeria  napuma \\
      \\
\glt
\z

woman  sick(ness)

`menstruation'

\ea%x41
\label{ex:x41}
\gll ibiamun  sama \\
      \\
\glt
\z

dove  ladder

`cross-beam (in a roof)'

The compound noun (\stepcounter{nx}{\thenx}) has the stress centre on the first part, but the noun phrase \textstyleStyleVernacularWordsItalic{emeria napuma}, with the phrase stress on \textstyleStyleVernacularWordsItalic{napuma}, may be used to mean either `a sick woman', or more commonly `a (dead) woman's body', a euphemistic expression.  

The second relationship is a \textstyleEmphasizedWords{\textsc{whole-part}} relationship: the first element states the whole, the second its part.

\ea%x42
\label{ex:x42}
\gll mokok(a)  oposia \\
      \\
\glt
\z

eye  meat

`pupil (of the eye)'

\ea%x43
\label{ex:x43}
\gll ekek(a)  muuna \\
      \\
\glt
\z

branch  joint/projection

`bud'

The third relationship is that of \textstyleEmphasizedWords{\textsc{container}}. As a compound \textstyleStyleVernacularWordsItalic{muuk(a) sia} (\stepcounter{nx}{\thenx}) has the stress centre on the first word, in a noun phrase (\stepcounter{nx}{\thenx}) the phrase stress may also be on the second item if it is emphasized; a third person singular genitive pronoun may be added between the parts as well. Example (\stepcounter{nx}{\thenx}) is an extended compound: \textstyleStyleVernacularWordsItalic{iinan aasa}  is a ``sky canoe'', or vehicle, for flying in the sky, and \textstyleStyleVernacularWordsItalic{iinan aasa epa} a place for those vehicles.

\ea%x45
\label{ex:x45}
\gll muuk(a)  sia \\
      \\
\glt
\z

son  netbag

`womb', `pouch (of a marsupial) 

\ea%x1770
\label{ex:x1770}
\gll muuk(a)  sia \\
      \\
\glt
\z

son  netbag

`a son's/child's netbag (used for carrying the baby)'

\ea%x46
\label{ex:x46}
\gll iinan  aasa  epa \\
      \\
\glt
\z

sky  canoe  place

`airstrip, airport'

As was mentioned above, the \textstyleEmphasizedWords{\textsc{generic-specific}} relationship is different in that the modifying part follows rather than precedes the main part. In this respect these compounds resemble phrases where the head noun has an adjective rather than a noun modifier.  A particularly common word for the first part in these compounds is the maximally generic word in Mauwake, \textstyleStyleVernacularWordsItalic{maa} `\textstyleFreeTranslationChar{thing'}(\stepcounter{nx}{\thenx}).\footnote{The scope of meaning for \textstyleFootnoteBaseChar{\textit{maa}} is like that of `thing\textit{'} in its widest sense in English.}

\ea%x47
\label{ex:x47}
\gll mera  nepa \\
      \\
\glt
\z

fish  bird

`eagle ray'

\ea%x48
\label{ex:x48}
\gll oon(a)  tiretira \\
      \\
\glt
\z

bone  horizontal.cane  (in  roof  structure)

`rib'

\ea%x49
\label{ex:x49}
\gll maa  pela \\
      \\
\glt
\z

thing  leaf

`(edible) greens'

There are two compound types with nominalized verbs. When the nominalized verb follows the other noun, it behaves like an adjective and receives the phrase stress.

\ea%x1521
\label{ex:x1521}
\gll maa  en-owa \\
      \\
\glt
\z

thing/food  eat-NMZ

`food'

\ea%x1522
\label{ex:x1522}
\gll emer(a)  ik-owa \\
      \\
\glt
\z

sago  roast-NMZ

`bread, roasted sago'

A compound type where the nominalized verb precedes the other noun is more common than the one above. When the second part is a human noun, it usually has to be the \textstyleEmphasizedWords{\textsc{agent}} of the verb (\stepcounter{nx}{\thenx}), but when the noun is non-human, it is harder to find a common denominator for the semantic relationships between the parts in different compounds. Quite often the meaning centers around function, purpose or ``typical'' action, place, time etc.

\ea%x52
\label{ex:x52}
\gll uuw-ow(a)  mua \\
      \\
\glt
\z

work-NMZ  man

`worker'

\ea%x53
\label{ex:x53}
\gll in-ow(a)  koora \\
      \\
\glt
\z

sleep-NMZ  house

`bedroom'

\ea%x54
\label{ex:x54}
\gll om-ow(a)  eka \\
      \\
\glt
\z

cry-NMZ  water

`tear'

This compound type particularly easily allows compounds with more than two roots: 

\ea%x55
\label{ex:x55}
\gll ikemik(a)  kaik-ow(a)  mua \\
      \\
\glt
\z

wound  tie-NMZ  man

`doctor'

\ea%x56
\label{ex:x56}
\gll emer(a)  en-ow(a)  mua \\
      \\
\glt
\z

sago  eat-NMZ  man

`a Sepik man (lit: a sago eater)'\footnote{Sepik province is known for its main staple, sago starch.}

\ea%x60
\label{ex:x60}
\gll ama  urup-ow(a)  (epa/kame) \\
      \\
\glt
\z

sun  rise-NMZ  place/side

`east'

In the example (\stepcounter{nx}{\thenx}) the main noun \textstyleStyleVernacularWordsItalic{epa}/\textstyleStyleVernacularWordsItalic{kame} can be dropped, and this happens in some other compounds as well:

\ea%x61
\label{ex:x61}
\gll epir(a)  suruk-ow(a)  (tetelka) \\
      \\
\glt
\z

plate  wipe-NMZ  finger

`forefinger' 

The \textstyleEmphasizedWords{\textsc{coordinate}} compounds are different from the other compounds in that neither of the parts modifies the other. The meaning of the whole is derived from the combined meaning of the two terms. Also, there is no stress centre: both parts of the compound are stressed equally. The number of these compounds is small.

\ea%x50
\label{ex:x50}
\gll emeria  mua \\
      \\
\glt
\z

woman  man

`people'

\ea%x51
\label{ex:x51}
\gll muuka  wiipa \\
      \\
\glt
\z

son  daughter

`children'

The \textstyleAcronymallcaps{N}+\textstyleAcronymallcaps{ADJ} compounds are as hard to distinguish from phrases as some of the other groups mentioned above. Again the uninterruptibility and lexicalized meaning are the main criteria. If the adjective \textstyleStyleVernacularWordsItalic{sepa} `black' is added between the two words in (\stepcounter{nx}{\thenx}), the meaning changes into `a small black man'.

\ea%x57
\label{ex:x57}
\gll mua  gelemuta \\
      \\
\glt
\z

man  small

`a little boy'

\ea%x58
\label{ex:x58}
\gll mia  yoowa \\
      \\
\glt
\z

body/skin  hot

`fever'

\ea%x59
\label{ex:x59}
\gll maa  samora \\
      \\
\glt
\z

thing  bad

`mosquito'

Compounding is a productive process in Mauwake, and it is the most common language-internal means used for adding new lexical items to the language. 

\subsection{Derived nouns}
\hypertarget{RefHeading19381935131865}{}
In this section I will discuss derivations where the \textstyleEmphasizedWords{\textsc{end result}} is a noun. There are only two of these: nouns made out of verbs, and noun reduplications. 

\subsubsection[Action nominals]{Action nominals}
\hypertarget{RefHeading19401935131865}{}
The process of nominalizing verbs is a straightforward and fully productive process of adding the nominalizing suffix -\textstyleStyleVernacularWordsItalic{owa} to the verb stem. The nominalized verbs most commonly function as nouns, sometimes also as adjectives (\stepcounter{nx}{\thenx}).\footnote{In the Mauwake dictionary some of these nominalized forms have their own entry as if they were fully lexicalized as nouns, but this is to some extent a concession to other languages, where separate nouns may be required for the action nominals and more lexicalized deverbal nouns (for the distinction, see Ylikoski 2003: 193). In Mauwake it is often difficult to establish which of the nominalizations are lexicalized.}

\ea%x62
\label{ex:x62}
\gll uf-\textstyleEmphasizedVernacularWords{owa} \\
      \\
\glt
\z

dance-NMZ

`(the act of) dancing', `(traditional) dance'

\ea%x63
\label{ex:x63}
\gll irak-\textstyleEmphasizedVernacularWords{owa} \\
      \\
\glt
\z

fight-NMZ

`fighting', `fight/war' 

\ea%x1231
\label{ex:x1231}
\gll Fiirim-\textstyleEmphasizedVernacularWords{owa}=pa  opaimika  aakun-e-mik. \\
      \\
\glt
\z

gather-NMZ=LOC  talk  talk-PA-1/3p

`In the meeting we talked.'

\ea%x1247
\label{ex:x1247}
\gll Amina  puk\textstyleEmphasizedVernacularWords{-owa}  eliw(a)  marewa=ke. \\
      \\
\glt
\z

pot  break-NMZ  good  none=CF

`The pot is broken (and) not good' or: `The broken pot is not good.'

Action nominals function like any regular nouns in Mauwake. They can be, for example, a head (\stepcounter{nx}{\thenx}) or a qualifier (\stepcounter{nx}{\thenx}) in a \textstyleAcronymallcaps{NP}, and a first (\stepcounter{nx}{\thenx}) or last element (\stepcounter{nx}{\thenx}) in a compound noun.

\ea%x64
\label{ex:x64}
\gll Siowa  \textstyleEmphasizedVernacularWords{alu-owa}  miim-ap  ekap-o-k. \\
      \\
\glt
\z

dog  make.noise-NMZ  hear-SS.SEQ  come-PA-3s

`He heard the dog's noise and came' or: `The dog heard noise and came.'

\ea%x65
\label{ex:x65}
\gll \textstyleEmphasizedVernacularWords{Irak-owa}  \textstyleEmphasizedVernacularWords{kerer-owa}  epa  weeser-em-ik-eya  {\dots}{\footnotemark} \\
      \\
\glt
\z


fight-NMZ  appear-NMZ  time  finish-SS.SIM-be-2/3s.DS

`As the time of the war was getting close{\dots}' (Lit: `As the war-appearing time was coming to an end{\dots}')

\footnotetext{\textit{Kererowa} is both the head of \textit{irakowa kererowa} and part of the qualifier phrase in \textit{irakowa kererowa epa.} }

\ea%x66
\label{ex:x66}
\gll Oram  \textstyleEmphasizedVernacularWords{niir-ow(a)}  opora  ma-e-m. \\
      \\
\glt
\z

just  laugh-NMZ  talk  say-PA-1s

`I just said it as a joke.'

\ea%x67
\label{ex:x67}
\gll Kaul  \textstyleEmphasizedVernacularWords{wafur-owa } mera  \textstyleEmphasizedVernacularWords{aaw-owa}  eliw. \\
      \\
\glt
\z

hook  throw-NMZ  fish  get-NMZ  all.right

`As for throwing a hook, it is a good way of catching fish.' (Lit: `Hook-throwing is all right for fish-catching.') 

The following expressions form an interesting pair, as (\stepcounter{nx}{\thenx}) is a \textstyleAcronymallcaps{NP} with a nominalized verb as a head, and (\stepcounter{nx}{\thenx}) is a compound noun with a nominalized verb as the first part.

\ea%x424
\label{ex:x424}
\gll mua  aakun-\textstyleEmphasizedVernacularWords{owa} \\
      \\
\glt
\z

man  talk-NMZ

`talk(ing) of man/people', `people's talk'

\ea%x425
\label{ex:x425}
\gll aakun-\textstyleEmphasizedVernacularWords{ow}(\textstyleEmphasizedVernacularWords{a})  mua \\
      \\
\glt
\z

talk-NMZ  man

`a talker', `a spokesman'

Action nominals keep their verb-like property of  being able to take the same arguments and peripherals as the verb serving as the root of the noun. The result is a  nominalized clause, which functions like a noun phrase. This is discussed further in \sectref{sec:5.7} and \sectref{sec:8.3.2}.

Comrie and \citet[334-342]{Thompson2007} list various kinds of other nominalization possibilities,\footnote{Giv\'on calls all of these \textit{lexical nominalizations} (1990:500), and Ylikoski calls them \textit{deverbal nouns} (2003:193) to distinguish them from action nominals.}  but in Mauwake the corresponding expressions are compound nouns or noun phrases consisting of the nominalized verb (or clause) plus another noun, rather than simple nominalizations. 

\ea%x1232
\label{ex:x1232}
\gll ikemika  kaik-\textstyleEmphasizedVernacularWords{ow(a)}  mua \\
      \\
\glt
\z

wound  tie-NMZ  man

`doctor, nurse'

\ea%x1233
\label{ex:x1233}
\gll maa  eneka  teek-\textstyleEmphasizedVernacularWords{ow(a)}  (maa)\footnotemark{} \\
      \\
\glt
\z

thing  tooth  open-NMZ  (thing)

`can opener'

\footnotetext{\textit{Maa eneka} is a compound referring to edible animals; the very generic noun \textit{maa} `thing' may be omitted from the end.}

\subsubsection[Noun reduplication]{Noun reduplication}
\hypertarget{RefHeading19421935131865}{}
Reduplication of nouns to denote plurality is a very marginal process in Mauwake, whereas reduplication of verbs (\sectref{sec:3.8.2.4.1}) is much more frequent, and that of adjectives (\sectref{sec:3.3}) also more common. Usually the whole noun is reduplicated; final /a/ is deleted in the reduplicated part of words that are longer than two syllables (\stepcounter{nx}{\thenx}). 

\ea%x68
\label{ex:x68}
\gll Dabuel  \textstyleEmphasizedVernacularWords{poka-poka}  nain=iw  biiris  on-am-ik-e-mik. \\
      \\
\glt
\z

pawpaw  RDP-trunk  that1=INST  bridge  make-SS.SIM-be-PA-1/3p

`They kept making the bridge with pawpaw trunks.'

\ea%x69
\label{ex:x69}
\gll Waaya  pa-ep  \textstyleEmphasizedVernacularWords{kio-kiowa}  naap  uup-e-mik. \\
      \\
\glt
\z

pig  butcher-SS.SEQ  RDP-piece  thus  cook-PA-1/3p

`We butchered the pig and cooked the pieces like that.'

\ea%x426
\label{ex:x426}
\gll \textstyleEmphasizedVernacularWords{Owow-owowa}  ikiw-e-mik. \\
      \\
\glt
\z

RDP-village  go-PA-1/3p

`We went to several villages.'

\section{Adjectives}
\hypertarget{RefHeading19441935131865}{}
The existence of noun and verb as universal categories is generally acknowledged, but the status of adjectives is less clear. There is considerable variation among languages as to what belongs to the adjective class, and sometimes a question is posed whether the class exists at all.  But when there is a class of adjectives, the following tendencies emerge: languages that have a small class of adjectives show a lot of similarity in what kinds of concepts they express through this class; and similarly, in languages where the adjective class is large the semantic content of the class is fairly constant \citep[20]{Dixon1977}. Semantically it is somewhat of an in-between category sharing similarities with both nouns and verbs \citep[447]{Lyons1977}. Nouns ``connote the possession of a complex of qualities, and [adjectives] the possession of one single quality'' (Jespersen 1924:81; see also Wierzbicka 1986:362). Nouns have reference, adjectives do not \citep[77]{HakulinenEtAl1979}%Karlsson
. Instead of categorizing like nouns do, adjectives describe \citep[357]{Wierzbicka1986}. They may also code transitory states, and in Giv\'on's (1984:52) time-stability scale they occupy the middle area between nouns and verbs.\footnote{But see \textstyleFootnoteBaseChar{Thompson}'s (1988) criticism on \textstyleFootnoteBaseChar{Giv\'on}'s placing of adjectives on the time-stability scale.} 

The morphological and syntactic coding of ``property concepts'' reflects their semantically ambivalent status: especially in languages which have either no adjectives or only a small adjective class, the concepts are usually expressed via verbs and/or nouns, sometimes by other means \citep[20]{Dixon1977}.

The adjective class in Mauwake is a relatively small open class when compared with nouns and verbs. But compared with some other Papuan languages \citep[50-51]{Dixon1977} it is a fairly large class: the number of non-derived adjectives currently in the dictionary is about 80.\footnote{Usan also has a relatively large adjective inventory (\textstyleFootnoteBaseChar{Reesink 1987}:63).}  The morphological and syntactic similarities and differences between nouns and adjectives were discussed above in \sectref{sec:3.2.2}.  Adjectives do not inflect at all.  

A prototypical adjective functions as the head of an adjective phrase\footnote{Often the head is the sole constituent of the adjective phrase.} (\sectref{sec:4.2}) and may be modified by different intensity adverbs (\sectref{sec:3.9.2}), including the pre-modifier \textstyleStyleVernacularWordsItalic{lawisiw} `rather' (\stepcounter{nx}{\thenx}) and various post-modifiers (\stepcounter{nx}{\thenx}). 

\ea%x70
\label{ex:x70}
\gll Nomokowa  \textstyleEmphasizedVernacularWords{maala}  war-e-k. \\
      \\
\glt
\z

tree  long  cut-PA-3s

`He cut a tall tree.'

\ea%x71
\label{ex:x71}
\gll Waaya  me  \textstyleEmphasizedVernacularWords{maneka},  muuka,  \textstyleEmphasizedVernacularWords{kia  gelemuta}. \\
      \\
\glt
\z

pig  not  big  son  white  small

`The pig was not big, it was a piglet, white (and) small.'

\ea%x72
\label{ex:x72}
\gll Malol  \textstyleEmphasizedVernacularWords{lawisiw  yoowa}. \\
      \\
\glt
\z

open.sea  rather  hard

`(Fishing in the) open sea is rather hard.'

\ea%x73
\label{ex:x73}
\gll Koora  nain  \textstyleEmphasizedVernacularWords{maneka  wenup}. \\
      \\
\glt
\z

house  that1  big  very

`That house is very big.'

Only the following adjectives have been found to be non-scalar:

morena  `male'

suwina  `female'

emi    `taboo(ed)'

enuma\footnote{\textit{Enuma} also means `new' and `green'.}  `alive'

The typical adjectives in Mauwake are all non-derived, and among them are all those listed by \citet[23]{Dixon1977} as the most common adjectives cross-linguistically: large, small, long, short, old, new, good, bad, black, white and red.

Of the various adjective groups mentioned by \citet{Dixon1977}, those of \textstyleEmphasizedWords{\textsc{age}} and value are quite small in Mauwake. Only two of the age adjectives are non-derived, the other two are derived:

awona  `old'  -  enuma  `new'

panewowa  `old'

iperowa  `middle-aged, elder'

 The adjective \textstyleStyleVernacularWordsItalic{awona} `old' refers to the age of things, not people; when used of people, the meaning is `previous' (\stepcounter{nx}{\thenx}). Correspondingly, its antonym \textstyleStyleVernacularWordsItalic{enuma} `new' refers to age of things or recency in humans (\stepcounter{nx}{\thenx}). The adjective referring to age in people,  \textstyleStyleVernacularWordsItalic{panewowa} `old'\footnote{\textstyleFootnoteBaseChar{\textit{Panewowa}} is derived from the verb \textstyleFootnoteBaseChar{\textit{pan}}- `grow old'.} does not have any adjective as an antonym; the noun \textstyleStyleVernacularWordsItalic{takira} `youth' is used instead. \textstyleStyleVernacularWordsItalic{Panewowa} `old' and \textstyleStyleVernacularWordsItalic{iperowa} `middle-aged' do  not indicate age only, but social status as well: it is the middle-aged men, rather than young or old, that have most power and make the important decisions in the community. \textstyleStyleVernacularWordsItalic{Iperowa} is also used for older siblings when the age of siblings is compared.

\ea%x74
\label{ex:x74}
\gll Emeria  \textstyleEmphasizedVernacularWords{panewowa}  nain  Kait  emeria  \textstyleEmphasizedVernacularWords{awona}=ke. \\
      \\
\glt
\z

woman  old  that1  Kait  woman  old=CF

`The old woman is Kait's old (=previous) wife.'

\ea%x75
\label{ex:x75}
\gll Ona  mua  \textstyleEmphasizedVernacularWords{enuma}  iiriw  pani-e-k. \\
      \\
\glt
\z

3s.GEN  man  new  already  grow.old-PA-3s

`Her new husband is (already) old.'

\textstyleEmphasizedWords{\textsc{Value}} adjectives are the following: 

eliwa  `good'  -  samora  `bad'

makena  `true'  -  wadola  `false'

emi    `taboo, forbidden'

\ea%x1760
\label{ex:x1760}
\gll Inasin  opaimika  \textstyleEmphasizedVernacularWords{eliwa}  me  yia  maak-e-mik. \\
      \\
\glt
\z

spirit  talk  good  not  1p.ACC  tell-PA-1/3p

`They did not speak good Tok Pisin (lit: spirit talk) to us.'

\ea%x1759
\label{ex:x1759}
\gll Iiriw  sira  nain  \textstyleEmphasizedVernacularWords{emi}  maneka  wiar  ik-ua. \\
      \\
\glt
\z

earlier  custom  that1  forbidden  big  3.DAT  be-PA.3s

`Earlier that custom was completely forbidden to them.'

The list of \textstyleEmphasizedWords{\textsc{colour}} terms is also very limited; only the first three terms in the list are purely colour terms, all the others have their origin elsewhere:

sepa  `black'

kia    `white'

oka    `red', `brown'

enuma  `green'  {{\textless}}  `new'

ligam  `yellow'  {{\textless}}  `turmeric'

ekapina  `blue'  {{\textless}}  `shrub sp. (used for blue dye)'

foma  `grey'  {{\textless}}  `ashes'\footnote{cf. Berlin and Kay 1969:4.}

\ea%x1753
\label{ex:x1753}
\gll Aalbok  mia  \textstyleEmphasizedVernacularWords{sepa} \textstyleEmphasizedVernacularWords{} \textstyleEmphasizedVernacularWords{akena}  kerer-e-k. \\
      \\
\glt
\z

black.cuckoo.shrike  body  black  very  become-PA-3s

`The body of the black cuckoo-shrike became very black.'

\ea%x109
\label{ex:x109}
\gll Konima  nain  \textstyleEmphasizedVernacularWords{sepa  kia}. \\
      \\
\glt
\z

cloth  that1  black  white

`The cloth is black-and-white.'

\ea%x1754
\label{ex:x1754}
\gll Mia  afif(a)  \textstyleEmphasizedVernacularWords{oka},  \textstyleEmphasizedVernacularWords{oka}  gelemuta. \\
      \\
\glt
\z

body  hair  red,  red  small

`The feathers were red, (it was) red and small.'

\ea%x1755
\label{ex:x1755}
\gll Komora  nain  \textstyleEmphasizedVernacularWords{kia  ne  maneka}  wenup. \\
      \\
\glt
\z

cuscus  that1  white  ADD  big  very

`That cuscus is/was white and very big.'

In (\stepcounter{nx}{\thenx}) the dimensional adjective for `small' may follow directly after the colour adjective, whereas the adjective \textstyleStyleVernacularWordsItalic{maneka} `big' needs a connective between the two adjectives in (\stepcounter{nx}{\thenx}), because \textstyleStyleVernacularWordsItalic{maneka} is used as an intensifier when immediately following a colour term, and \textstyleStyleVernacularWordsItalic{kia maneka} would mean `completely white'.

The darkness of a colour is expressed through the adjectives \textstyleStyleVernacularWordsItalic{sepa} `black' and \textstyleStyleVernacularWordsItalic{kia} `white' used as modifiers of the main colour adjective (\stepcounter{nx}{\thenx}).

\ea%x110
\label{ex:x110}
\gll ifa  \textstyleEmphasizedVernacularWords{enuma  lawisiw  sepa} \\
      \\
\glt
\z

leaf  new/green  rather  black

`a dark green leaf'

Among the adjectives denoting \textstyleEmphasizedWords{\textsc{dimension}} there are a number of terms describing various kinds of thinness and thickness, as well as shortness. 

maneka  `large'  -  gelemuta  `small'

maala  `long'  -  iiwa  `short'

kuruma  `thick'  -  gawela  `thin'

fula(kia)  `fat'  -  bebeta  `slim, skinny'

teena  `thin'

komosia  `small, short'

\ea%x1756
\label{ex:x1756}
\gll Epa  dabela=pa  mia  suuw-owa  \textstyleEmphasizedVernacularWords{gawela}  suuw-ap  \\
      \\
\glt
\z

place  cold=LOC  body  push-NMZ  thin  push-SS.SEQ  

mia  fulil-i-nan.

body  feel.cold-Np-FU.2s

`When you wear thin clothes (mia suuwowa) in a cold place you will feel cold.' 

\ea%x76
\label{ex:x76}
\gll Owor(a)  ara  \textstyleEmphasizedVernacularWords{teena } nain  ku-i-non. \\
      \\
\glt
\z

betelnut.palm  trunk  thin  that1  break-Np-FU.3s

`The thin betelnut palm trunk will break.'

\ea%x77
\label{ex:x77}
\gll Epa  dabel-al-eya  mia  suuw-owa  \textstyleEmphasizedVernacularWords{kuruma } wu-e. \\
      \\
\glt
\z

place  cold-INCH-2/3s.DS  body  push-NMZ  thick  put-IMP.2s

`When it gets cold, put thick clothes on.'

The group of adjectives denoting \textstyleEmphasizedWords{\textsc{physical property}} is larger than any of the other groups and includes several antonym pairs. The list below is just a sample:

yoowa  `hot, hard'  -  dabela  `cold'

supuka  `wet'  -  ififa  `dry'

pina  `heavy'  -  efefa  `light'

kaken  `straight'  -  meka  `crooked'

melina  `clear'  -  wiwisa  `murky'

anima  `sharp'  -  duduwa  `blunt'

dubila  `slippery, smooth'

itita  `soft'

masia  `bitter (taste)'

siina  `tight'

\ea%x78
\label{ex:x78}
\gll Iwera  \textstyleEmphasizedVernacularWords{ififa}  ora-eya  fiirim-i-mik. \\
      \\
\glt
\z

coconut  dry  descend-2/3s.DS  gather-Np-PR.1/3p

`When the dry coconuts drop we gather them'.

\ea%x1758
\label{ex:x1758}
\gll {\dots}epia  foma  lawisiw  \textstyleEmphasizedVernacularWords{yoowa}  ik-ua. \\
      \\
\glt
\z

fire(wood)  ashes  rather  hot  be-PA.3s

`{\dots} the ashes were rather hot.'

\textstyleEmphasizedWords{\textsc{Human propensity}}  adjectives is the second largest group. 

lebuma  `lazy'  -  topia  `diligent'

asia  `wild'  -  memela  `tame'

lebuma  `lazy'

momora  `foolish'

popora  `quiet'

yamunsia  `stingy'

\ea%x1757
\label{ex:x1757}
\gll Takira=ke  keker  op-ap  \textstyleEmphasizedVernacularWords{popor(a)}  maneka  ik-e-mik. \\
      \\
\glt
\z

boy=CF  fear  hold-SS.SEQ  quiet  big  be-PA-1/3p

`The boys were afraid and very quiet.'

\ea%x1418
\label{ex:x1418}
\gll Mua  \textstyleEmphasizedVernacularWords{lebuma}  nain  emeria  me  wi-i-mik. \\
      \\
\glt
\z

man  lazy  that1  woman  not  give.them-Np-PR.1/3p

`We do not give wives to lazy men.'

Although Mauwake has a considerable inventory of adjectives for a Papuan language, in actual use they are rather infrequent.\footnote{Their frequency in the text material is about 1.5\% of all the words.} Especially physical property and human propensity are frequently expressed through verbs which have been verbalized from adjectives. A true adjective is a more likely candidate to indicate a stable or essential quality of the head noun (\stepcounter{nx}{\thenx}), whereas the verbalized form is used for more temporary characteristics (\stepcounter{nx}{\thenx})-(\stepcounter{nx}{\thenx}).

\ea%x1419
\label{ex:x1419}
\gll Sama=pa  or-owa  nain  eliw,  nain  ikoka  or-op \\
      \\
\glt
\z

stairs=LOC  descend-NMZ  that1  well  that1  later  descend-SS.SEQ

or-op  or-op  \textstyleEmphasizedVernacularWords{lebum(a)-ar-i-nan},  epasia  akena.

descend-SS.SEQ  descend-SS.SEQ  lazy-INCH-Np-FU.2s  far  very

`Descending on the stairs is all right, but later when you have gone down and down and down you will be lazy/tired, (as) it is very far.'

\ea%x79
\label{ex:x79}
\gll Moma  \textstyleEmphasizedVernacularWords{kasu(a)-ar-eya } me  enim-i-mik. \\
      \\
\glt
\z

taro  hard-INCH-2/3s.DS  not  eat-Np-PR.1/3p

`We don't eat hard taro.' (Lit: `When taro is hardened, we don't eat it.')

\ea%x80
\label{ex:x80}
\gll \textstyleEmphasizedVernacularWords{Yamunsi(a)-ar-iwkin}  me  wia  nokar-e-m. \\
      \\
\glt
\z

stingy-INCH-2/3p.DS  not  3p.ACC  ask-PA-1s

`They were (being) stingy, (so) I didn't ask them.'

\textsc{Speed} is expressed through adverbs or verbs rather than adjectives.

\textstyleEmphasizedWords{\textsc{Comparison}} of adjectives is an area where there is very little differentiation in many Papuan languages, including Mauwake.\footnote{See \textstyleBibliogBaseChar{Roberts} (1987:134-5), \textstyleBibliogBaseChar{Reesink} (1987:68), \textstyleBibliogBaseChar{Hardin} (2002:63-4); \textstyleBibliogBaseChar{Haiman} reports only three or four true adjectives for Hua, and does not mention comparison (1980:268).} Intensifiers are used for this function, as well as the verb \textstyleStyleVernacularWordsItalic{nomak}\textstyleEmphasizedVernacularWords{-} `overcome, surpass'. 

\ea%x81
\label{ex:x81}
\gll Poka  fain  maala,  nain  \textstyleEmphasizedVernacularWords{nomak-e-k},  ne  oko  nain  \textstyleEmphasizedVernacularWords{maala} \\
      \\
\glt
\z

stilt  this   long  that1  surpass-PA-3s  ADD  other  that1  long

\textstyleEmphasizedVernacularWords{akena}.

very

`This stilt is longer than that, and/but the other one is the longest (lit: very long).'

Two adjectives can also be compared by contrasting them: 

\ea%x441
\label{ex:x441}
\gll Nomokow(a)  kakawa  fain  \textstyleEmphasizedVernacularWords{iiwa},  oko  \textstyleEmphasizedVernacularWords{maala}  puuk-a-n. \\
      \\
\glt
\z

tree  part  this   short  other  long  cut-PA-2s

`You cut this plank shorter than the other one.' (Lit: `You cut this plank short, the other long.')

Adjectives denoting size form a scale of three: \textstyleStyleVernacularWordsItalic{gelemuta} `small', \textstyleStyleVernacularWordsItalic{manisiri} `biggish', \textstyleStyleVernacularWordsItalic{maneka} `big'. Usually, if three degrees of comparison are needed, it is possible to express them periphrastically, but that is seldom necessary.  Comparison as a functional domain is discussed in \sectref{sec:6.5}. 

Like nouns, adjectives can also be \textstyleEmphasizedWords{\textsc{reduplicated}} for plural (\sectref{sec:2.3.3.2}). Reduplication of adjectives is not very common, but it is more frequent than that of nouns. 

\ea%x85
\label{ex:x85}
\gll Maa  eneka  kes  \textstyleEmphasizedVernacularWords{mane-maneka}  oram  iw-e-mik. \\
      \\
\glt
\z

thing  tooth  case  RDP-big  just  give.him-PA-1/3p

`They just gave him big cases of meat tins.'

The adjective \textstyleStyleVernacularWordsItalic{gelemuta} `small' has several reduplicated forms: \textstyleStyleVernacularWordsItalic{gelemuti-tik}\textstyleEmphasizedVernacularWords{,} \textstyleStyleVernacularWordsItalic{gelemutu-mut}\textstyleEmphasizedVernacularWords{,} \textstyleStyleVernacularWordsItalic{gele-gelemuti-tik}\textstyleEmphasizedVernacularWords{.} 

\ea%x486
\label{ex:x486}
\gll Waaya  \textstyleEmphasizedVernacularWords{gelemutu-mut}  pu-puuk-e. \\
      \\
\glt
\z

pig  small-RDP  RDP-cut-IMP.2s

`Cut the pig into small pieces.'

Occasionally reduplication can be used for an intensifying function as well. The noun modified by the reduplicated adjective in (\stepcounter{nx}{\thenx}) is either singular or plural, in (\stepcounter{nx}{\thenx}) it is definitely singular.

\ea%x485
\label{ex:x485}
\gll Biiris  eliwa  me  on-a-mik,  \textstyleEmphasizedVernacularWords{damo-damola}=ko. \\
      \\
\glt
\z

bridge  good  not  make-PA-1/3p  RDP-bad=NF

`They didn't make a good bridge (but) very bad.' (or: `{\dots}good bridges but bad.')

\ea%x86
\label{ex:x86}
\gll {\dots ifa}=ke  keraw-a-k,  mamepaperuma  \textstyleEmphasizedVernacularWords{gele-gelemuti-tik}  nain=ke. \\
      \\
\glt
\z

{\dots}snake=CF  bite-PA-3s  death.adder  RDP-small-RDP  that1=CF

`{\dots} a snake bit him, a very small death adder.'

\textstyleEmphasizedWords{\textsc{New adjectives}} are derived from verbs with the nominalizing suffix \nobreakdash-\textstyleStyleVernacularWordsItalic{owa}. This is not a very productive process.

kekanowa  `strong'  {{\textless}}  kekan-  `be strong' 

panewowa  `old'  {{\textless}}  pan-  `become old'

kainowa  `high (voice)'  {{\textless}}  kain-  `be high (voice)'

bolonowa  `slack'  {{\textless}}  bolon-  `be slack'

\ea%x1766
\label{ex:x1766}
\gll No  mua  samora,  mua  emin(a)  \textstyleEmphasizedVernacularWords{kekan}\textstyleEmphasizedVernacularWords{-}\textstyleEmphasizedVernacularWords{owa} \\
      \\
\glt
\z

2s.UNM  man  bad  man  occiput  be.strong-NMZ  

nefa  na-i-kuan.

2s.ACC  say-Np-FU.3p

`They will call you a bad man, a pig-headed (lit: strong occiput) man.'

\ea%x1765
\label{ex:x1765}
\gll Someka  aw-i-ya  nain  iwakara  \textstyleEmphasizedVernacularWords{kain-owa}  maneka  \\
      \\
\glt
\z

song  weave-Np-PR.3s  that1  neck  be.high-NMZ  big  

aw-i-ya.

weave-Np-PR.3s

`When (s)he sings, (s)he sings with a very high voice.'

\ea%x1767
\label{ex:x1767}
\gll Makera  \textstyleEmphasizedVernacularWords{saawirin}\textstyleEmphasizedVernacularWords{-}\textstyleEmphasizedVernacularWords{owa}  kaik-a-m. \\
      \\
\glt
\z

cane  surround-NMZ  tie-PA-1s

`I tied the cane round.'

Adjectives can be made into verbs by zero verb formation (\sectref{sec:3.8.2.2.1}) or by the inchoative verbaliser \nobreakdash-\textit{ar}  (\sectref{sec:3.8.2.2.2}).

\section{Quantifiers}
\hypertarget{RefHeading19461935131865}{}
Quantifiers are a small closed class of words. The group can be divided into numeral and non-numeral quantifiers. The reasons for treating them as a group of their own, separate from adjectives, are the following. Their position is after the adjectives in a \textstyleAcronymallcaps{NP}.\footnote{Actually it is the Quantifier Phrase that comes after the Adjective Phrase, but usually the phrases consist of only one word, a quantifier in the former and an adjective in the latter.} Some of the numerals consist of a phrase or even a clause, but they still function as a single unit. And semantically quantifiers are quite different from adjectives.

A quantifier is the only obligatory element in a quantifier phrase (\textstyleAcronymallcaps{QP,} \sectref{sec:4.3}). These are used as post-modifiers in a \textstyleAcronymallcaps{NP}, where their position is between an adjective phrase (\textstyleAcronymallcaps{AP}) and demonstrative (\stepcounter{nx}{\thenx}), or by themselves as a non-verbal predicate (\stepcounter{nx}{\thenx}). 

\ea%x87
\label{ex:x87}
\gll I  koora  maneka  \textstyleEmphasizedVernacularWords{kuisow}  nain  yiar  aw-o-k. \\
      \\
\glt
\z

1p.UNM  house  big  one  that1  1p.DAT  burn-PA-3s

`That one big house of ours burned.'

\ea%x442
\label{ex:x442}
\gll Mua  iperowa  \textstyleEmphasizedVernacularWords{arow}  \textstyleEmphasizedVernacularWords{muutiw.} \\
      \\
\glt
\z

man  middle-aged  three  only.

`There are/were only three middle-aged men.' (Lit: `The middle-aged men (are) only three.') 

The numerals, especially \textit{erup} `two', may be added to a pronoun to quantify it: the numeral occurs following a reflexive (or occasionally unmarked) form of the pronoun, but the pronoun is used like an unmarked pronoun.

\ea%x89
\label{ex:x89}
\gll Ne  \textstyleEmphasizedVernacularWords{wiam  erup}  pun  epa  neeke  or-o-mik. \\
      \\
\glt
\z

ADD  3p.REFL  two  too  place  there.CF  descend-PA-1/3p

`And the two of them too went down there.'

\subsection{Numerals}
\hypertarget{RefHeading19481935131865}{}
The traditional counting system in Mauwake is quinary, i.e. based on five\footnote{In New Guinea languages, there are counting systems based on two, five ten and twenty, as well as systems that use different body parts as tallies. All of these systems are present in the Madang area languages as well (\textstyleBibliogBaseChar{Lean 1991}).}, and counting is gestured using the fingers.\footnote{To count, the fingers are bent down one by one, starting from the little finger of the right hand, and proceeding towards the thumb, then on to the little finger of the other hand etc.}  

kuisow  `one'

erup  `two'

arow  `three'

erepam  `four'

ikur / wapen inawiya  `five' / `a hand sleeps'

(ikur) okai(wi)=pa kuisow  `six' (lit: `(five) one on/from the other side')

(ikur) okai(wi)=pa erup  `seven'

(ikur) okai(wi)=pa arow  `eight'

(ikur) okai(wi)=pa erepam  `nine'

iimeka kuisow / okaipa okaipa inek  `ten' / `both sides sleep'

\ea%x90
\label{ex:x90}
\gll Uura  ama  \textstyleEmphasizedVernacularWords{ikur  okai}(\textstyleEmphasizedVernacularWords{wi})\textstyleEmphasizedVernacularWords{=pa  arow}  naap  in-e-mik. \\
      \\
\glt
\z

night  sun  five  other.side=LOC  three  thus  sleep-PA-1/3p

`In the evening we slept at around eight o'clock.'

Nowadays the borrowed Tok Pisin numerals have largely superseded the vernacular numerals, especially those indicating numbers ten and above (\stepcounter{nx}{\thenx}).  There are no terms for `hundred', `thousand' or bigger numbers in the vernacular system.

\ea%x91
\label{ex:x91}
\gll Mokoma  \textstyleEmphasizedVernacularWords{ten  arow}  aaw-o-k. \\
      \\
\glt
\z

year  ten  three  get-PA-3s

`He became 30 years old.'

Numerals can be modified with the intensity adverbs \textstyleStyleVernacularWordsItalic{kakeniw}  `\textstyleFreeTranslationChar{correctly, exactly}',  \textstyleStyleVernacularWordsItalic{akena}\textstyleEmphasizedVernacularWords{}  `really, truly' or \textstyleStyleVernacularWordsItalic{muutiw} `only'.

\ea%x443
\label{ex:x443}
\gll \textstyleEmphasizedVernacularWords{Erepam  kaken}\textstyleEmphasizedVernacularWords{=iw}  mik-a-mik. \\
      \\
\glt
\z

four  straight-ISOL  spear-PA-1/3p

`We speared exactly four.'

\ea%x661
\label{ex:x661}
\gll Mua  \textstyleEmphasizedVernacularWords{arow  akena}  epa  nain  iimar-e-mik. \\
      \\
\glt
\z

man  three  truly  place  that  stand.up-PA-1/3p

`Exactly three men stood at that place.'

When the number is somewhat uncertain and the disjunctive connective \textstyleStyleVernacularWordsItalic{e} `\textstyleFreeTranslationChar{or}' and/or the question marker -\textstyleStyleVernacularWordsItalic{i}  is used, either the smaller or the bigger number may be mentioned first.

\ea%x1416
\label{ex:x1416}
\gll Waaya  maneka  wiowa  \textstyleEmphasizedVernacularWords{erup=i  e  arow}  naap  mik-iwkin  \\
      \\
\glt
\z

pig  big  spear  two=QM  or  three  thus  hit-2/3p.DS  

um-i-ya.

die-Np-PR.3s

`When a big pig is hit with two or three spears it dies.'

\ea%x92
\label{ex:x92}
\gll Mua  wiam  \textstyleEmphasizedVernacularWords{ikur=i  erepam}  naap  wia  aaw-e-mik. \\
      \\
\glt
\z

man  3p.REFL  five=QM  four  thus  3p.ACC  get-PA-1/3p

`They took/got those four or five men.'

Repetition (\stepcounter{nx}{\thenx}) or reduplication (\stepcounter{nx}{\thenx}) of the numerals indicates manner: `so and so many \textstyleEmphasizedWords{\textsc{at a time}}'. The reduplicated form of \textstyleStyleVernacularWordsItalic{kuisow} `\textstyleFreeTranslationChar{one}', \textstyleStyleVernacularWordsItalic{kui-kuisow},  has two meanings: `one by one' and `a few'. 

\ea%x93
\label{ex:x93}
\gll Naap  \textstyleEmphasizedVernacularWords{kuisow  kuisow}  aaw-ikiw-e-mik. \\
      \\
\glt
\z

thus  one  one  get-go-PA-1/3p

`They kept getting them one at a time as they went.

\ea%x94
\label{ex:x94}
\gll Waaya  merena  \textstyleEmphasizedVernacularWords{ere-erup}  kaik-ap{\dots} \\
      \\
\glt
\z

pig  leg  RDP-two  tie-SS.SEQ

`I tied the pig's legs two and two together and {\dots}'

Money is counted using different nouns indicating certain amounts:

maamuma ({{\textless} maa mumua)}   '10 toea', also generic `money', lit: `seed'

fuluwa  `1 kina', lit: `hole' (the coin has a hole)

ifa    `2 kina', lit: `leaf'

ifa oka  `5 kina', lit: `red leaf'

kuuma  `10 kina', lit: `stick'\footnote{From a stick of tobacco, used for payment in the colonial days.} 

\ea%x97
\label{ex:x97}
\gll \textstyleEmphasizedVernacularWords{Kuuma  kuisow  ifa  erup}  naap  yia  sesenar-e-mik. \\
      \\
\glt
\z

stick  one  leaf  two  thus  1p.ACC  buy-PA-1/3p

`They paid to us (lit: bought us for) 14 kina.' 

Mauwake has no separate words for \textstyleEmphasizedWords{ordinal} numbers. To indicate numerical order, various structures are employed.  In many cases the cardinal numbers can be used:

\ea%x96
\label{ex:x96}
\gll Mua  arow  epa  nain  iimar-e-mik,  yos=ke  \textstyleEmphasizedVernacularWords{erepam}. \\
      \\
\glt
\z

man  three  place  that1  stand-PA-1/3p  1s.FC=CF  four

`Three men were standing there, and I was the fourth.'

\ea%x428
\label{ex:x428}
\gll Koora  tuun-e:  \textstyleEmphasizedVernacularWords{kuisow  iki}(\textstyleEmphasizedVernacularWords{w})\textstyleEmphasizedVernacularWords{-}(\textstyleEmphasizedVernacularWords{e})\textstyleEmphasizedVernacularWords{p  erepam},  ne  \textstyleEmphasizedVernacularWords{oko } nain \\
      \\
\glt
\z

house  count-IMP.2s  one  go-SS.SEQ  four  ADD  other  that1

ona  koora.

3s.GEN  house

`His house is the fifth one' (Lit: `Count the houses: one to four, and the other/next is his house.')

In the case of time units, cardinal numbers are combined with the verb \textstyleStyleVernacularWordsItalic{ikiw}- `go':

\ea%x427
\label{ex:x427}
\gll \textstyleEmphasizedVernacularWords{Fofa  okai}(\textstyleEmphasizedVernacularWords{wi})\textstyleEmphasizedVernacularWords{=pa  arow  ikiw-eya}  ekap-i-non. \\
      \\
\glt
\z

day  other.side=LOC  three  go-2/3s.DS  come-Np-FU.3s

`He will come on the ninth day.' (Lit: `When eight days have gone he will come').

Order can also be indicated through verbs like \textstyleStyleVernacularWordsItalic{murar}- and \textstyleStyleVernacularWordsItalic{ook}- `follow'. 

\ea%x98
\label{ex:x98}
\gll Wi  Ulingan=ke  nomak-e-mik.  Ne  i  Moro  \\
      \\
\glt
\z

3p.UNM  Ulingan=CF  win-PA-1/3p  ADD  1p.UNM  Moro

\textstyleEmphasizedVernacularWords{murar-e-mik}.

follow-PA-1/3p

`The Ulingan people/team won. And (we from) Moro came second.'

Numbers are \textstyleEmphasizedWords{not} used when listing one's children. The terms \textstyleStyleVernacularWordsItalic{iperowa} `firstborn', \textstyleStyleVernacularWordsItalic{ookap onarowa} `following' (used repeatedly, if necessary) and \textstyleStyleVernacularWordsItalic{kapa} `lastborn' are employed for that.

\subsection{Non-numeral quantifiers}
\hypertarget{RefHeading19501935131865}{}
Some non-numeral quantifiers can only be used with either count or mass nouns, others occur with both. Those that can be used with both are:

senam  `too much/too many'

unowiya  `all'  (from: \textstyleStyleVernacularWordsxiiptItalic{unowa} `\textstyleFreeTranslationChar{many}' plus comitative clitic =\textstyleStyleVernacularWordsxiiptItalic{iya})

iiwawun  `all/altogether'

\ea%x665
\label{ex:x665}
\gll Moma  \textstyleEmphasizedVernacularWords{senam}  en-e-mik. \\
      \\
\glt
\z

taro  too.much  eat-PA-1/3p

`We ate too much taro.'

\ea%x666
\label{ex:x666}
\gll Nomokowa  \textstyleEmphasizedVernacularWords{senam}  war-e-man. \\
      \\
\glt
\z

tree  too.many  cut.PA-2p

`You cut too many trees.'

\ea%x99
\label{ex:x99}
\gll Yagin  eka=pa  \textstyleEmphasizedVernacularWords{unow=iya}  nan  yaki-e-mik. \\
      \\
\glt
\z

Yagin  water=LOC  many=COM  there  bathe-PA-1/3p

`We all bathed there at Yagin together.'

The following are only used with \textstyleEmphasizedWords{\textsc{count}} nouns:

papako\footnote{\textstyleFootnoteBaseChar{\textit{Papako}} is actually a plural indefinite `other'(\sectref{sec:3.7.2}), but it has a secondary function as a quantifier.}  `other/\textstyleFreeTranslationChar{some/a few}'

unowa  `\textstyleFreeTranslationChar{many}'

unow onaiya  `all' (from \textstyleStyleVernacularWordsItalic{unowa} plus \textstyleStyleVernacularWordsItalic{onaiya} `together with')

wenup  `\textstyleFreeTranslationChar{lots of}'

\ea%x100
\label{ex:x100}
\gll Mua  \textstyleEmphasizedVernacularWords{unowa},  emeria  \textstyleEmphasizedVernacularWords{papako}  um-e-mik. \\
      \\
\glt
\z

man  many  woman  some  die-PA-1/3p

`Many men and some women died.'

\ea%x667
\label{ex:x667}
\gll Ipia  saana=pa  iina  \textstyleEmphasizedVernacularWords{wenup}. \\
      \\
\glt
\z

rain  season=LOC  mosquito  lots.of

`In the rainy season there are lots of mosquitoes.'

Both \textstyleStyleVernacularWordsItalic{wenup} and\textstyleEmphasizedVernacularWords{} \textstyleStyleVernacularWordsItalic{unowa}\textstyleEmphasizedVernacularWords{} can be intensified with \textstyleStyleVernacularWordsItalic{akena} `very'; \textstyleStyleVernacularWordsItalic{unowa}  may also be intensified with \textstyleStyleVernacularWordsItalic{wenup}\textstyleEmphasizedVernacularWords{} `\textstyleFreeTranslationChar{very}' (\stepcounter{nx}{\thenx}); or with\textstyleEmphasizedVernacularWords{} \textstyleStyleVernacularWordsItalic{maneka} (lit: `\textstyleFreeTranslationChar{big}') that gives it the meaning `\textstyleFreeTranslationChar{all}' (\stepcounter{nx}{\thenx}).

\ea%x809
\label{ex:x809}
\gll Siipepe  kokora  maroka  \textstyleEmphasizedVernacularWords{wenup  akena}  ika-i-ya. \\
      \\
\glt
\z

Siipepe  riverbed  prawn  lots.of  very  be-Np-PR.3s

`There are lots of prawns in the Siipepe riverbed.'

\ea%x101
\label{ex:x101}
\gll Iinan  aasa  nepa  saarik,  \textstyleEmphasizedVernacularWords{unow(a)  akena/wenup}. \\
      \\
\glt
\z

sky  canoe  bird  like  many  very

`The planes were like birds, very many.'

\ea%x102
\label{ex:x102}
\gll Emeria  \textstyleEmphasizedVernacularWords{unow}(\textstyleEmphasizedVernacularWords{a})  \textstyleEmphasizedVernacularWords{maneka}  sosora  bee-beela  a-e-mik. \\
      \\
\glt
\z

woman  many  big  grass.skirt  RDP-rotten  tie-PA-1/3p

`All the women put on rotten grass skirts.'

The negation of the universal quantifier is discussed below in \sectref{sec:6.2.2}.

The following quantifiers only occur with \textstyleEmphasizedWords{\textsc{mass}} nouns:

maneka   `\textstyleFreeTranslationChar{a lot/much}' (lit: `big')

gelemuta  `\textstyleFreeTranslationChar{little}'

lawiliw    `\textstyleFreeTranslationChar{somewhat/a little}'

\ea%x103
\label{ex:x103}
\gll Eka  yoowa=pa  aaya  \textstyleEmphasizedVernacularWords{maneka/gelemuta}  wu-e. \\
      \\
\glt
\z

water  hot=LOC  sugar  big/little  put-IMP.2s

`Put a lot of/a little sugar in the tea.'

The following non-numeral quantifiers also function as degree/intensity adverbs, modifying a verb: \textstyleStyleVernacularWordsItalic{iiwawun}, \textstyleStyleVernacularWordsItalic{lawiliw}, \textstyleStyleVernacularWordsItalic{senam} and \textstyleStyleVernacularWordsItalic{wenup} (\sectref{sec:3.9.2}).

  \stepcounter{nx}{\thenx}x511)  Yos=ke  \textstyleEmphasizedVernacularWords{lawiliw}  asip-i-yem.\\
1s=CF  somewhat  help-Np-PR.1s

`I am helping her somewhat/a little.'

\ea%x512
\label{ex:x512}
\gll Iperowa=ke  \textstyleEmphasizedVernacularWords{senam}  kekan-e-mik. \\
      \\
\glt
\z

middle.aged=CF  too.much  be.strong-PA-1/3p

`The middle-aged men were very strong (in their opinion).'

\ea%x513
\label{ex:x513}
\gll Waaya  mik-amkun  \textstyleEmphasizedVernacularWords{iiwawun}  um-o-k.  \\
      \\
\glt
\z

pig  spear-1s/p.DS  altogether  die-PA-3s

`When I speared the pig it died completely.'

\textstyleEmphasizedWords{\textsc{Fractions}} are hard to express in Mauwake.\footnote{I have not seen fractions treated in grammars of Papuan languages, but know from discussions with colleagues that translating fractions is a major problem not only in Mauwake but in other Papuan languages as well.} The noun \textstyleStyleVernacularWordsItalic{enakiwa} `half' is sometimes also used for unspecified `part', and \textstyleStyleVernacularWordsItalic{okaiwi} `one/other side' can be used for `half', when a clearly bounded entity is divided in half (\stepcounter{nx}{\thenx}). I have not found other terms indicating fractions. Longer expressions are needed for them e.g. `divide into ten parts and take one part'.

\ea%x104
\label{ex:x104}
\gll Yabuela  \textstyleEmphasizedVernacularWords{okaiwi}  enak-e. \\
      \\
\glt
\z

pawpaw  one.side  feed.me-IMP.2s

`Give me half of the pawpaw to eat.'

\section{Pronouns}\footnotemark{}
\hypertarget{RefHeading19521935131865}{}
\footnotetext{ Most of the material in this section has been published in my earlier paper (J\"arvinen 1991).}
\subsection{Introduction}
\hypertarget{RefHeading19541935131865}{}
Pronouns are a closed class of words. According to traditional grammar, pronouns can substitute for nouns, but actually they substitute for full noun phrases. 

Pronouns in Mauwake only include personal pronouns. Demonstratives, which are like pronouns in some respects, are discussed under deictics (\sectref{sec:3.6.2}). The indefinites, which are used as modifiers in a noun phrase, are closely related to question words and are treated in \sectref{sec:3.7.2}.

In principle all the pronouns in Mauwake are used for humans only. In legends also spirits can be referred to by these pronouns since they sometimes act like humans and can take human form. There is no third person singular pronoun for non-humans. 

\citet{Wurm1982} posited three typological sets of personal pronouns for Papuan languages, and mentioned Madang province as an area where set III is particularly widespread. The basic forms of Wurm's set III pronouns are:

  singular  plural

1  da\~{ta\~{}ya}  ki\~{ti}

2  na  nik

3  nu\footnote{The third person plural form is not included in Wurm's typology because of gaps in the material and greater variability than in the other person forms.}  \citep[40-42]{Wurm1982}

In all the three pronoun sets fronting of vowels often goes together with plurality (ibid. 78), the non-singular forms in Papuan languages being derived from the singular forms \citep[361]{Franklin1979}. 

With more data and after more rigorous and detailed work on the \textstyleAcronymallcaps{TNG} pronouns, \citet[5]{Ross1995} gives the following as reconstructions of Proto Madang and Proto Croisilles free pronouns:





\begin{tabular}{lllllll}
\mytoprule
 & 1s & 2s & 3s & 1p & 2p & 3p\\
Proto Madang & *ya & *na & *ua/*nu & *i- & *ni-/*ta- & {\dots}\\
Proto Croisilles & *ya & *na/*ni & *ua/*nu & *i[ge]/*i[na] & *ni[ge] & *ua[ge]/*ua[na]\\
\mybottomrule
\end{tabular}



\begin{table}
\caption{Proto Madang and Proto Croisilles free pronouns}
\label{tab:8}
\end{table}

For different functions in the clause Papuan languages often have one or two classes, or functional sets, of pronouns with or without prepositions or suffixes to mark the appropriate cases. Amele \citep{Roberts1987}, Maia \citep[71]{Hardin2002}, Hua \citep[215]{Haiman1980}, Waskia \citep[53]{RossEtAl1978}%Paol
 and Bargam \citep[29]{Hepner2002} have only one basic set each, to which postpositions or suffixes are added. Usan \citep{Reesink1987} and Siroi \citep{Wells1979} each have a nominative and a possessive set. Most Finisterre-Huon languages have different sets for regular and emphatic pronouns (McElhanon 1973).

Person is the more basic category than number in the pronoun systems of Papuan languages \citep[69]{Foley1986}. As for number, it is most common just to have a two-way distinction between singular and plural, but dual forms are also quite widespread in \textstyleAcronymallcaps{TNG} languages, and trial forms are found in some areas as well. An inclusive-exclusive distinction in the first person plural form is not common \citep[60]{Wurm1982} like it is in Austronesian languages, but according to \citet[56]{Ross2005} it has probably been an areal feature for a long time, even before the Austronesians arrived. 

Morphological resemblance between free pronouns and some verbal affixes, most commonly subject markers, is fairly widespread in Papuan languages \citep{Franklin1979}. It is not unusual to find that verbal affixes, e.g. object markers, make fewer person/number distinctions than free pronouns \citep[67]{Foley1986}.

In the following respects Mauwake manifests general typological features of \textstyleAcronymallcaps{TNG} Madang pronouns. There is no gender or noun class system that would be indicated through concord and marking with nouns and/or pronouns. Also, the morphology is suffixal rather than prefixal. There is no inclusive-exclusive distinction. Possession is marked through suffixation on the personal pronouns \citep[40-42]{Wurm1982}.  

The basic unmarked pronouns in Mauwake reflect the Proto Croisilles forms rather closely, apart from the third person plural form \textstyleStyleVernacularWordsItalic{wi}, which \citet[23]{Ross1996} mentions as an innovation *\textstyleStyleVernacularWordsItalic{u-i}- shared by the Kumil languages and the neighbouring Kaukombar languages. The ending -\textstyleStyleVernacularWordsItalic{fa} in the first and second person singular accusative pronouns is an innovation in the Kumil languages only. 

Some features in the Mauwake pronoun system not typical of Papuan languages are the existence of dative pronouns and also their use as possessives, and the distinction between the unmarked pronouns and the focal pronouns. 

The personal pronoun system in Mauwake is very regular, including the first, second and third persons both in singular and plural. Normally the plural form can also be used for dual; the dual number is only marked in one group, and there by adding a numeral rather than through affixation (\sectref{sec:3.5.8}). Since dual number does not occur in verb person marking either, apart from the first person imperative form, it is not very significant in the category of number. Spatial deixis is not marked in the personal  pronoun system in Mauwake. The case is marked to some extent. \tableref{tab:9} lists the personal pronouns in Mauwake:



\begin{table}


\begin{tabular}{llllllllll}
\mytoprule
 & \multicolumn{2}{l}{{\bfseries FREE}}

 & {\bfseries ACC} & {\bfseries GEN} & {\bfseries DAT} & {\bfseries ISOL} & {\bfseries RESTR} & {\bfseries REFL} & {\bfseries COM}\\
 & {\bfseries UNM} & {\bfseries Focal} &  &  &  &  &  &  & \\
1s & yo & yo-s & efa & y-ena & efa-r & ya-isow & yena-iw/yos-iw & y-ame & efa-m-iya\\
2s & no & no-s & nefa & n-ena & nefa-r & na-isow & nena-iw/nos-iw & n-ame & nefa-m-iya\\
3s & (w)o & (w)o-s & {\O} & o-na & wi-ar & wa-isow & ona-iw/os-iw & w-ame & wama-iya\\
1p & (y)i & (y)i-s & yia & yi-ena & yi-ar & (y)i-isow & yien-iw/is-iw & yi-am & yiam-iya\\
2p & ni & ni-s & nia & ni-ena & ni-ar & ni-isow & nien-iw/nis-iw & ni-am & niam-iya\\
3p & wi & wi-s & wia & wi-ena & wi-ar & wi-isow & wien-iw/wis-iw & wi-am & wiam-iya\\
\mybottomrule
\end{tabular}



\caption{Personal pronouns}
\label{tab:9}
\end{table}

Mauwake is a so-called pro-drop language, and a complete sentence can consist of a verb alone. The person of the subject is marked fully in the final verbs and partially in the medial verbs, so that besides the pragmatic clues there are also grammatical means for tracing the participants. But the pronouns are not completely optional: their use is rather strictly dictated by textual factors.

It is a fairly common feature in languages that pronouns can either modify a noun in a \textstyleAcronymallcaps{NP} or replace a full \textstyleAcronymallcaps{NP}, but cannot be the head of a \textstyleAcronymallcaps{NP} taking modifiers (e.g. Hakulinen and Karlsson 1979, Saari 1985, Roberts 1987). In Mauwake the personal pronouns usually occur without modifiers, but they \textstyleEmphasizedWords{\textsc{can}} be modified by a demonstrative, provided there is no collocational clash between the demonstrative and the personal pronoun. 

\ea%x530
\label{ex:x530}
\gll \textstyleEmphasizedVernacularWords{Ni  fain=ke}  ekap-eka! \\
      \\
\glt
\z

2s.UNM  this=CF  come-IMP.2p

`You here (or: This group of you), come!'

\ea%x531
\label{ex:x531}
\gll \textstyleEmphasizedVernacularWords{O  nain}  fan  me  ik-ua. \\
      \\
\glt
\z

3s.UNM  that1  here  not  be-PA.3s

`He is not here.'

A pronoun copy after a full \textstyleAcronymallcaps{NP} is hardly ever used in Mauwake for the subject. The rare example (\stepcounter{nx}{\thenx}) is from a hortatory text and may show rhetoric style:

\ea%x683
\label{ex:x683}
\gll Maneka  fain  [wie  \textstyleEmphasizedVernacularWords{wi}]  eliw  wiar  op-i-kuan. \\
      \\
\glt
\z

big  this  uncle  3p.UNM  well  3.DAT  grab-NpFU.3p

`These big ones the uncles may take from her.'

The example (\stepcounter{nx}{\thenx}) is not a  case of a genuine pronoun copy, since the genitive pronoun \textstyleStyleVernacularWordsItalic{wiena} adds the emphasizing meaning `themselves':

\ea%x532
\label{ex:x532}
\gll \textstyleEmphasizedVernacularWords{Wi } iperowa  \textstyleEmphasizedVernacularWords{wi-ena}  ekap-e-mik. \\
      \\
\glt
\z

3p.UNM  middle.aged  3p-GEN  come-PA-1/3p

`The middle-aged (people) themselves came.' 

For a pronoun copy of the genitive in a possessive \textstyleAcronymallcaps{NP}, see \sectref{sec:4.1.1}.

Pronouns as deictic elements are discussed in \sectref{sec:6.3.1}.

\subsection{Free pronouns}
\hypertarget{RefHeading19561935131865}{}
There are two sets of free pronouns:  the unmarked pronouns, and the slightly longer focal pronouns.

\subsubsection[Unmarked pronouns]{Unmarked pronouns}
\hypertarget{RefHeading19581935131865}{}
The unmarked pronouns are as follows:

  singular  plural

1  yo  (y)i

2  no  ni

3  (w)o  wi

The main use of the unmarked pronouns is as subjects. In narratives only the person marking on the verb, rather than a pronoun, is used for an established, continuing subject/topic (\sectref{sec:9.2.2}). Especially third person unmarked pronouns marking the subject are quite rare in narrative texts; first person pronouns are relatively much more common (J\"arvinen 1991:79-80). 

  \stepcounter{nx}{\thenx}x533)  Irak-owa=ke  kerer-eya  \textstyleEmphasizedVernacularWords{wi } puk-omak-e-mik.\\
fight-NMZ=CF  appear-2/3p.DS  3p.UNM  disperse-DISTR.PL-PA-1/3p

`When the fight started they (many) dispersed.'

\ea%x534
\label{ex:x534}
\gll \textstyleEmphasizedVernacularWords{O}  koora=pa  naap  ik-ok  um-o-k. \\
      \\
\glt
\z

3s.UNM  house=LOC  thus  be-SS  die-PA-3s

`She was like that in the house and died.'

\ea%x1803
\label{ex:x1803}
\gll Bogia=pa  nan  wu-ap  \textstyleEmphasizedVernacularWords{i}  kiiriw  ekap-e-mik. \\
      \\
\glt
\z

Bogia=LOC  there  put-SS.SEQ  1p.UNM  again  come-PA-1/3p

`We buried his body (lit: put it) there in Bogia and came (back) again.'

However, with imperative verbs the subject pronoun is common (\sectref{sec:3.5.11}, \sectref{sec:3.8.3.3.2}, \sectref{sec:7.3}). In this Mauwake provides an interesting exception to a very strong cross-linguistic tendency of dropping subject pronouns in imperative clauses (Giv\'on 1979:80, Sadock and Zwicky 1985:173-174).  In this position the pronoun is usually unstressed, unless it is contrasted with the subject of another  clause coordinated with the imperative clause (\stepcounter{nx}{\thenx}). 

\ea%x1771
\label{ex:x1771}
\gll ``\textstyleEmphasizedVernacularWords{No}  me  baurar-e,''  naap  maak-e-k.  \\
      \\
\glt
\z

2s.UNM  not  flee-IMP.2s  thus  tell-PA-3s

` ``Don't run away,'' he told her.'

\ea%x1772
\label{ex:x1772}
\gll \textstyleEmphasizedVernacularWords{I}  or-u. \\
      \\
\glt
\z

1p.UNM  descend-IMP.1d

`Let's go down.'

\ea%x1780
\label{ex:x1780}
\gll \textstyleEmphasizedVernacularWords{No}  feeke  ik-e,  yo  Amerika  wia  \\
      \\
\glt
\z

2s.UNM  here.CF  be-IMP.2s  1s.UNM  America  3p.ACC  

akup-ikiw-i-yem.

search-go-Np-PR.1s

`You stay here, I will go searching the Americans.'

In an imperative clause the subject pronoun may also be used appositionally with a \textstyleAcronymallcaps{NP} that has vocative function, to address a person (\stepcounter{nx}{\thenx}). 

\ea%x627
\label{ex:x627}
\gll Muuka,  \textstyleEmphasizedVernacularWords{no}  aakisa  emeria  aaw-e! \\
      \\
\glt
\z

son  2s.UNM  now  woman  take-IMP.2s

`Son, take a wife now!' (i.e. It is time for you to get married.)

There are some cases where the imperative clauses tend not to have a subject pronoun. When the clause has a theme (9.1) different from the subject, and especially when the theme is another pronoun (\stepcounter{nx}{\thenx}), the imperative subject is blocked:

\ea%x1773
\label{ex:x1773}
\gll A,  ifera\textsubscript{T}H  feeke  un-eka. \\
      \\
\glt
\z

ah  salt.water  here.CF  draw/fetch-IMP.2p

`Ah, fetch the salt water (right) here.'

\ea%x1774
\label{ex:x1774}
\gll Yo\textsubscript{TH, TP}  momor  me  yook-e. \\
      \\
\glt
\z

1s.UNM  foolishly  not  follow.me-IMP.2s

`Don't be foolish and follow me.' (Lit: `Don't follow me foolishly.')

When an imperative final clause is preceded by a different-subject medial clause, it does not have a subject pronoun either:

\ea%x1775
\label{ex:x1775}
\gll Nefa  war-iwkin  naap  ma-e. \\
      \\
\glt
\z

2s.ACC  shoot-2/3p.DS  thus  say-IMP.2s

`When they shoot you, (then) say like that.'

A sentence-initial subject pronoun is quite common, when one or more same-subject medial clauses precede the imperative final clause and the scope of the imperative extends backwards over all the verbs:

\ea%x628
\label{ex:x628}
\gll \textstyleEmphasizedVernacularWords{Ni}  ikiw-ep  moma  perek-eka! \\
      \\
\glt
\z

2p.UNM  go-SS.SEQ  taro  pull.out-IMP.2p

`Go and pull out (i.e. harvest) taro!'

The only example in the text data of a subject pronoun repeated in the final clause is a case where the medial clause is subordinated with the topic marker -\textstyleStyleVernacularWordsItalic{na}: 

\ea%x1776
\label{ex:x1776}
\gll \textstyleEmphasizedVernacularWords{Ni}  uf-ep-na  \textstyleEmphasizedVernacularWords{ni}  maadara  \\
      \\
\glt
\z

2p.UNM  dance-SS.SEQ=TP  2s.UNM  forehead.ornament  

me  iirar-eka  ...

not  remove-IMP.2p

`If/when you have danced, do not remove the forehead ornaments {\dots}' 

When the level of politeness is reduced, the subject pronoun is less common. Some acceptable reasons for this are urgency (\stepcounter{nx}{\thenx}), or speech by an official that is expected to be brusque (\stepcounter{nx}{\thenx}). Example (\stepcounter{nx}{\thenx}) is from a situation where the behaviour of some men has been offensive to their wives, and when the men return home and give a blunt command, their wives react to this additional insult by repeating the command and then stating their own grievance and their revenge. 

\ea%x1777
\label{ex:x1777}
\gll Karu-eka,  ikoka  Yaapan  ir-ami  {\dots} \\
      \\
\glt
\z

run-IMP.2p  later  Japan  come-SS.SIM

`Run, later the Japanese will come and {\dots}'

\ea%x1778
\label{ex:x1778}
\gll ...amia  mua=ke  ma-e-mik,  ``Nainiw  owowa  ikiw-eka.'' \\
      \\
\glt
\z

bow  man=CF  say-PA-1/3p  again  village  go-IMP.2p

`{\dots}the policemen said, ``Go back to the village.'' '

\ea%x1779
\label{ex:x1779}
\gll Ekap-emi  wia  maak-e-mik,  ``Maa  iiw-eka.''  \\
      \\
\glt
\z

come-SS.SIM  3p.ACC  tell-PA-1/3p  food  dish.out-IMP.2p

```Maa  iiw-eka.'  Nis=ke  sira  oko  on-ami...''

food  dish.out-IMP.2p  2p.FC=CF  custom  other  do-SS.SIM

`They came and told them, ``Dish out the food.''  `` `Dish out the food!' You acted offensively (lit: did another custom) and{\dots}'' '

There is some tendency to have a pronominal form to occupy the sentence-initial theme position (\sectref{sec:9.1}), especially when the pronoun refers to the main participant of the sentence. In some cases this results in the restructuring of the sentence so that a medial clause appears in the middle of the finite clause, instead of coming before it as would be more normal. In (\stepcounter{nx}{\thenx}) and (\stepcounter{nx}{\thenx}) the medial clauses are enclosed in square brackets. 

\ea%x539
\label{ex:x539}
\gll \textstyleEmphasizedVernacularWords{Yo}  [eka  yoowa  Magidar=ke  kirip-ap  yi-eya]  \\
      \\
\glt
\z

1s.UNM  water  hot  Magidar=CF  mix-SS.SEQ  give.me-2/3s.DS  

en-e-m.

eat-PA-1s

`Magidar made tea and gave it to me, and I drank it.'

\ea%x540
\label{ex:x540}
\gll \textstyleEmphasizedVernacularWords{No}  [um-eya]  or-o-n. \\
      \\
\glt
\z

2s.UNM  die-2/3s.DS  descend-PA-2s

`After he died you went down.'

Sentence-initial unmarked pronouns are also used when they are not subjects but rather mark a pronoun with other than subject function as the theme. The first person pronoun in particular is placed in the theme position very frequently, the second person less so and the third person least of all.

\ea%x535
\label{ex:x535}
\gll \textstyleEmphasizedVernacularWords{Yo  efa}  uruf-e! \\
      \\
\glt
\z

1s.UNM  1s.ACC  look-IMP.2s

`Look at me!'

\ea%x536
\label{ex:x536}
\gll \textstyleEmphasizedVernacularWords{I}  \textstyleEmphasizedVernacularWords{yiena}  mua  opora  \textstyleEmphasizedVernacularWords{yia}  asip-owa  ekap-e-mik  nain \\
      \\
\glt
\z

1p.UNM  1p.GEN  man  talk  1p.ACC  help-NMZ  come-PA-1/3p  that

`Our men who have come to help us with the language {\dots}'

Especially in spoken language the unmarked pronouns may also be used, instead of genitive pronouns, to indicate possession. This is most commonly done with kinship terms and body parts, sometimes with other nouns\footnote{The following list covers most of them: \textstyleFootnoteBaseChar{\textit{opora}} `talk, speech', \textit{opaimika} `mouth, speech', \textit{unuma} `name', \textstyleFootnoteBaseChar{\textit{koora}} `house, home', \textit{manina} `garden', \textit{siowa} `dog' and \textstyleFootnoteBaseChar{\textit{amina}} `saucepan'.} too, referring to things closely associated with a person. This usage can be seen as a kind of widening of the range of inalienably possessed nouns beyond the kinship terms (\sectref{sec:3.2.4}) to other nouns that would be inalienably possessed in related languages or some other languages in the area. 

\ea%x537
\label{ex:x537}
\gll \textstyleEmphasizedVernacularWords{Yo}  auwa  nan  ik-ua. \\
      \\
\glt
\z

1s.UNM  1s/p.father  there  be-PA.3s

`My father is there.'

\ea%x538
\label{ex:x538}
\gll Ikoka  Yaapan=ke  \textstyleEmphasizedVernacularWords{ni } umakuna  nia  puuk-i-kuan.  \\
      \\
\glt
\z

Later  Japan=CF  2p.UNM  neck  2p.ACC  cut-Np-FU.3s

`Later the Japanese will cut your necks.'

\ea%x1804
\label{ex:x1804}
\gll Aria,  \textstyleEmphasizedVernacularWords{yo}  opora  muut  nan-e-k. \\
      \\
\glt
\z

alright  1s.UNM  talk  only  there-PA-3s

`Alright, there is my talk.'

The third person plural unmarked pronoun is used to pluralise a noun phrase (\stepcounter{nx}{\thenx}). It is also often used with a place name to refer to the inhabitants of the place collectively (\stepcounter{nx}{\thenx}).

\ea%x625
\label{ex:x625}
\gll \textstyleEmphasizedVernacularWords{Wi}  sawur  nain=ke  kuura  puuk-a-mik. \\
      \\
\glt
\z

3p.UNM  spirit  that1=CF  fly  cut-PA-3s

`Those spirits changed into flies.'

\ea%x626
\label{ex:x626}
\gll \textstyleEmphasizedVernacularWords{Wi}  Lasen=ke  kuum-e-mik. \\
      \\
\glt
\z

3p.UNM  Lasen=CF  burn-PA-1/3p

`The Lasen people burned it.' (Or: `It was the Lasen people who burned it.')

The neutral focus marker -\textstyleStyleVernacularWordsItalic{ko} attaches itself to the unmarked pronoun rather than the focal pronoun. I do not know the reason for this.\footnote{Kwan Poh San suggests as a possible reason that as the irrealis focus does not give as strong an emphasis as the contrastive focus, it also attaches itself to a less emphasized form of the pronoun (p.c.).}

\ea%x547
\label{ex:x547}
\gll Waaya  en-e-man  nain  \textstyleEmphasizedVernacularWords{yo=ko}  me  uruf-a-m. \\
      \\
\glt
\z

pig  eat-PA-2p  that1  1s.UNM=NF  not  see-PA-1s  

`I didn't (get to even) see the pig that you ate.'

The unmarked pronouns are used as the basic form for focal, genitive, reflexive-reciprocal and isolative pronouns.

\subsubsection[Focal pronouns]{Focal pronouns}
\hypertarget{RefHeading19601935131865}{}
The focal pronouns are similar to the unmarked pronouns but have final \textstyleStyleVernacularWordsItalic{-s: yos, nos, (w)os, (y)is, nis, wis.} These pronouns are never used for a neutral, non-focused subject. They are used in isolation and in lists (\stepcounter{nx}{\thenx}), as well as with the topic marker -\textstyleStyleVernacularWordsItalic{na} (\stepcounter{nx}{\thenx}), the contrastive focus marker -\textstyleStyleVernacularWordsItalic{ke} (\stepcounter{nx}{\thenx}), the question marker -\textstyleStyleVernacularWordsItalic{i} (\stepcounter{nx}{\thenx}) and the adverb \textstyleStyleVernacularWordsItalic{pun} `also' (\stepcounter{nx}{\thenx}). With the limiter -\textstyleStyleVernacularWordsItalic{iw} (\stepcounter{nx}{\thenx}) the focal pronoun forms one of the two kinds of restrictive pronoun.  (See \sectref{sec:3.5.7}.)

\ea%x541
\label{ex:x541}
\gll \textstyleEmphasizedVernacularWords{Yos},  yena  emeria,  ne  Yoli  gelemuta  {\dots} \\
      \\
\glt
\z

1s.FC  1s.GEN  woman  ADD  Yoli  little

`I, my wife and little Yoli {\dots}'

\ea%x542
\label{ex:x542}
\gll \textstyleEmphasizedVernacularWords{Nos}=na? \\
      \\
\glt
\z

2s.FC=TP

`What about you?'

\ea%x543
\label{ex:x543}
\gll \textstyleEmphasizedVernacularWords{Is}=ke  me  kuum-e-mik. \\
      \\
\glt
\z

1p.FC=CF  not  burn-PA-1/3p

`\textstyleEmphasizedWords{\textsc{We}}  didn't burn it.'

\ea%x544
\label{ex:x544}
\gll \textstyleEmphasizedVernacularWords{Yos}=i? \\
      \\
\glt
\z

1sg.FC=QM

`I?' 

\ea%x546
\label{ex:x546}
\gll \textstyleEmphasizedVernacularWords{Os}  pun  opora  kuisow  naap=iw  ma-e-k. \\
      \\
\glt
\z

3s.FC  also  talk  one  thus=LIM  say-PA-3s

`\textstyleEmphasizedWords{\textsc{He}}  also said the same thing.'

\ea%x545
\label{ex:x545}
\gll Anane  \textstyleEmphasizedVernacularWords{nos=iw}  nefa  maak-i-ya. \\
      \\
\glt
\z

always  2s.FC=LIM  2s.ACC  tell-Np-PR.3s

`He always talks to you only.'

When the subject of an imperative clause is contrasted with some other possible subject, the focal pronoun with contrastive focus clitic is employed: 

\ea%x629
\label{ex:x629}
\gll \textstyleEmphasizedVernacularWords{Nos=ke}  ikiw-e! \\
      \\
\glt
\z

2s.FC=CF  go-IMP.2s

`\textstyleEmphasizedWords{\textsc{You}} go (not someone else)!'

\subsection{Accusative pronouns}
\hypertarget{RefHeading19621935131865}{}
The accusative pronouns may have been derived from the unmarked pronouns, but because at present there is little similarity between the singular forms of the two sets, the accusative pronouns are treated as a set of their own. Their main use is to mark the syntactic object of a clause, which is typically the semantic patient but with a few verbs may be a recipient (\sectref{sec:5.2}). The plural forms are also used for the beneficiary, as the beneficiary suffix -\textstyleStyleVernacularWordsItalic{a} in the verb (\sectref{sec:3.8.3.1}) does not distinguish between singular and plural. The accusative pronouns serve as a basis for some other pronoun forms with different functions as well. The form of the accusative pronouns is reflected very closely in the plural stems of the object cross-referencing verbs but not in the singular stems (\sectref{sec:3.8.4.2.4}). 

The accusative pronouns are:

  singular  plural

1  efa  yia

2  nefa  nia

3  {\O} (zero)  wia

Only objects that are [+human] are marked with the pronoun. As there is no other case marking in \textstyleAcronymallcaps{NP}s, except for oblique case marking like locative and instrument for [\nobreakdash-human] \textstyleAcronymallcaps{NP}s, the accusative pronouns provide some of this case marking, when the object is a [+human] \textstyleAcronymallcaps{NP}. Much of the time there is no overt pronoun, as the third person singular form is zero.\footnote{Zero pronoun for the third person singular is not exceptional cross-lingustically (Lyons 1968:278, Foley 1986:66, Giv\'on 1976:166), and in Papuan languages it is common especially for the object pronoun. All the 25 Northern Adelbert Range languages compared by Z'\citet[9,160]{Graggen1980} have zero as object pronoun or object marking on the verb for the third person singular form.} 

The position of the accusative pronouns in Mauwake is immediately preceding the verb. This is probably the main reason why Z'\citet{Graggen1971} treats them as verbal prefixes. Likewise, \citet[108]{Reesink1987} states that Usan has object prefixes, even if they have a rather loose status and can be detached from the verb. But I consider the object pronouns in Mauwake independent words, as they all have two syllables and follow the normal stress pattern of the language. They are, however, very closely bound to the verb, and it seems that a cliticization process is going on.\footnote{In J\"arvinen (1991) I discussed this question whether Mauwake pronouns are full words, clitics or affixes, at some length.}  

The accusative pronouns are used for encoding semantic patient (\stepcounter{nx}{\thenx}), or recipient (\stepcounter{nx}{\thenx}), both of which are syntactic objects (\sectref{sec:5.2}, 5.3). 

\ea%x548
\label{ex:x548}
\gll Irakowa=pa  \textstyleEmphasizedVernacularWords{wia}  war-e-mik. \\
      \\
\glt
\z

fight=LOC  3p.ACC  kill-PA-1/3p

`In the fight they killed them.'

\ea%x550
\label{ex:x550}
\gll Opora  nain  \textstyleEmphasizedVernacularWords{efa}  maak-e-k. \\
      \\
\glt
\z

talk  that  1s.ACC  tell-PA-3s

`He told me the story.' 

The plural forms of the accusative pronouns are used together with the beneficiary form in the verb to disambiguate between the persons (\stepcounter{nx}{\thenx}) (\sectref{sec:3.8.3.1}).

\ea%x549
\label{ex:x549}
\gll Aite  maa  \textstyleEmphasizedVernacularWords{yia}  p-or-om-a-k. \\
      \\
\glt
\z

mother  food  1p.ACC  Bpx-descend-BEN-BNFY2.PA-3s

`Mother brought food down for us.'

The only grammatical difference between the semantic roles of patient and beneficiary is shown in the verb, which can incorporate the benefactive suffix; and between patient and recipient there is no syntactic or morphological difference. The following hierarchy is followed: if there is a recipient not incorporated in the verb root,\footnote{Verbs like `give' and `feed' incorporate the recipient object in the verb root itself (\sectref{sec:3.8.4.2.4}).} the accusative pronoun refers to it (\stepcounter{nx}{\thenx}), if there is no recipient but a plural beneficiary, the pronoun refers to the latter (\stepcounter{nx}{\thenx}). And if there is neither recipient nor beneficiary, the accusative pronoun refers to the patient (\stepcounter{nx}{\thenx}).\footnote{Cf. a rather similar hierarchy for the distributive suffix in verbs (\sectref{sec:3.8.2.3.2})} 

Transitive verbs in Mauwake usually require an overt object, and verbs like `teach', `tell', `ask', which can take two objects, require the presence of at least the human object, whether patient (\stepcounter{nx}{\thenx}), or recipient (\stepcounter{nx}{\thenx}). In (\stepcounter{nx}{\thenx}) the pronoun \textstyleStyleVernacularWordsItalic{wia} `3p.ACC' may be definite or indefinite, hence the alternative free translations.

\ea%x552
\label{ex:x552}
\gll \textstyleEmphasizedVernacularWords{Nefa}  nokar-i-yem. \\
      \\
\glt
\z

2s.ACC  ask-Np-PR.1s

`I'm asking you.'

\ea%x551
\label{ex:x551}
\gll Inglis  \textstyleEmphasizedVernacularWords{wia}  ofakow-i-ya. \\
      \\
\glt
\z

English  3p.ACC  teach-Np-PR.3s

`(S)he teaches them English.' (Or: `(S)he teaches English.')

In rare cases the human object may be left out:

\ea%x553
\label{ex:x553}
\gll Oram  nokar-i-yem. \\
      \\
\glt
\z

just  ask-Np-PR.1s

`I'm just asking.' (Asking nobody in particular, or for no particular reason.)

Transitive verbs with [+human] objects require pronouns even when the object is mentioned as a noun or a noun phrase.

\ea%x554
\label{ex:x554}
\gll Emeria  \textstyleEmphasizedVernacularWords{wia}  amukar-e-k. \\
      \\
\glt
\z

woman  3p.ACC  scold-PA-3s

`He/she scolded the women.'

\ea%x555
\label{ex:x555}
\gll Emeria  \textstyleEmphasizedVernacularWords{nia}  amukar-e-k. \\
      \\
\glt
\z

woman  2p.ACC  scold-PA-3s

`He/she scolded (you) women.'

Since the third person singular form is zero, all the cases with [+human] object noun without overt object pronoun by default indicate the third person singular (\stepcounter{nx}{\thenx}). Because there is no number or case distinction in the nouns for the arguments of the verb, without this indication by pronouns it would often be ambiguous whether the \textstyleAcronymallcaps{NP} was subject or object, or whether the object was singular or plural. 

\ea%x556
\label{ex:x556}
\gll Emeria  amukar-e-k. \\
      \\
\glt
\z

woman  scold-PA-3s

`He scolded his wife.'

In theory, the example (\stepcounter{nx}{\thenx}) could also mean `The woman scolded him/her' but in practice it does not. For when the subject is old/established information it is usually left out rather than marked by a \textstyleAcronymallcaps{NP}, and when it is new information, it is marked by the contrastive focus marker -\textstyleStyleVernacularWordsItalic{ke}.\footnote{To have the meaning `He/she scolded a/the \textstyleEmphasizedWords{woman'}, the noun would be followed by the non-numeral quantifier \textstyleFootnoteBaseChar{\textit{oko}} `a, a certain' or the demonstrative \textstyleFootnoteBaseChar{\textit{nain}} `that'.}

It must be clearly indicated whether the speaker or addressee is included in the object (\stepcounter{nx}{\thenx}), (\stepcounter{nx}{\thenx}), (\stepcounter{nx}{\thenx}).

\ea%x557
\label{ex:x557}
\gll Mua  \textstyleEmphasizedVernacularWords{yia}  aaw-i-kuan. \\
      \\
\glt
\z

man  1p.ACC  take-Np-FU.3p

`They will take (us) men.'

\citet[52-53]{Reesink1987} mentions that Usan, another Pihom Stock language, has object prefixes, but a free pronoun can also occupy the object position in the third person singular. This is not the case in Mauwake; in (\stepcounter{nx}{\thenx}) the free pronoun \textstyleStyleVernacularWordsItalic{o}  is a re-activated topic (\sectref{sec:9.2.3}). The negative clause (\stepcounter{nx}{\thenx}) shows that the position of the free pronoun is not directly preceding the verb. The clauses (\stepcounter{nx}{\thenx}), (\stepcounter{nx}{\thenx}) have a similar structure with pronouns in non-third person marking a theme. When the pronoun is fronted as a theme (9.1), it is this unmarked pronoun that is used in the theme position. 

\ea%x1354
\label{ex:x1354}
\gll Wi  teeria  papako  \textstyleEmphasizedVernacularWords{o}  {\O}  asip-a-mik... \\
      \\
\glt
\z

3p.UNM  group  other  3s.UNM  {\O}  help-PA-1/3p

`Another group helped him{\dots}' (Or: `He was helped by another group{\dots}')

\ea%x1353
\label{ex:x1353}
\gll \textstyleEmphasizedVernacularWords{O}  me  {\O  aaw-e-mik.} \\
      \\
\glt
\z

3s.UNM  not  {\O}  take-PA-1/3p

`They did not take/choose him.' (Or: `He was not taken by them.')

\ea%x684
\label{ex:x684}
\gll \textstyleEmphasizedVernacularWords{Yo}  me  \textstyleEmphasizedVernacularWords{efa}  aaw-e-mik. \\
      \\
\glt
\z

1s.UNM  not  1s.ACC  take-PA-1/3p

`They didn't take/choose me.' (Or: `I wasn't taken by them.')

\ea%x560
\label{ex:x560}
\gll \textstyleEmphasizedVernacularWords{Yo}  \textstyleEmphasizedVernacularWords{efa}  aaw-e-mik. \\
      \\
\glt
\z

1s.UNM  1s.ACC  take-PA-1/3p

`They took/chose me.' (Or: `I was taken by them.')

There is one instance where the free third person singular pronoun does occur after the negator and immediately preceding the verb, just like accusative pronouns. This is when there is constituent negation (\sectref{sec:6.2.2}.) on the object, which then also receives clausal stress (\stepcounter{nx}{\thenx}) (\sectref{sec:2.1.3.1}, 9.2.3). Here it is the negator that moves to precede the constituent it negates. The same process is also seen in (\stepcounter{nx}{\thenx}) where the negator has moved in front of the whole object \textstyleAcronymallcaps{NP}. 

\ea%x561
\label{ex:x561}
\gll Me  \textstyleEmphasizedVernacularWords{o}  uruf-a-m. \\
      \\
\glt
\z

not  3s.UNM  see-PA-1s

`It wasn't him/her that I saw.'

\ea%x562
\label{ex:x562}
\gll Me  \textstyleEmphasizedVernacularWords{wi}  owow  mua  \textstyleEmphasizedVernacularWords{wia}  arew-a-mik{\dots} \\
      \\
\glt
\z

not  3p.UNM  village  man  3p.ACC  wait-PA-1/3p

`It wasn't the village people that we waited for{\dots}'

There are situations where it is impossible to determine whether the unmarked  third person singular pronoun is marking a topic/subject or an object fronted as a theme (\sectref{sec:9.1}). The context would be needed to disambiguate between the slightly different meanings of (\stepcounter{nx}{\thenx}), which do not come out well in the English translation. The first meaning is likely if the context mentions some other people seeing something; the second meaning is more probable elsewhere. 

\ea%x563
\label{ex:x563}
\gll \textstyleEmphasizedVernacularWords{O}  me  uruf-a-k. \\
      \\
\glt
\z

3s.UNM  not  see-PA-3s

`\textit{(S)he}  didn't see him/her/it.' (Or: `(S)he didn't see \textit{him/her}.')

There are a few verbs in Mauwake that cross-reference the patient or recipient object in the verb root (\sectref{sec:3.8.4.2.4}). These verbs do not allow a separate accusative pronoun for the function that is already expressed by the verb root (\stepcounter{nx}{\thenx}), (\stepcounter{nx}{\thenx}), but it is possible to have a separate accusative pronoun for the patient when the verb cross-references the recipient rather than the patient (\stepcounter{nx}{\thenx}).

\ea%x564
\label{ex:x564}
\gll Ipia=ke  \textstyleEmphasizedVernacularWords{yiar-eya}  ekap-e-mik. \\
      \\
\glt
\z

rain=CF  hit.us-2/3s.DS  come-PA-1/3p

`The rain hit us and we/they came.'

\ea%x1526
\label{ex:x1526}
\gll Yomar,  no  uurika  \textstyleEmphasizedVernacularWords{yook}\textstyleEmphasizedVernacularWords{-}\textstyleEmphasizedVernacularWords{ap}  urup-e. \\
      \\
\glt
\z

friend  2s.UNM  tomorrow  follow.me-SS.SEQ  ascend-IMP.2s

`Friend, follow me up tomorrow.'

\ea%x565
\label{ex:x565}
\gll Iiriw  \textstyleEmphasizedVernacularWords{nefa}  \textstyleEmphasizedVernacularWords{wi-e-mik}. \\
      \\
\glt
\z

already  2s.ACC  give.them-PA-1/3p

`We have already given you to them.'

When other verbs require both a [+human] recipient and a [+human] patient, it is encoded as a clause chain. The first verb then takes one of the arguments and the second the other.

\ea%x566
\label{ex:x566}
\gll Uuriw  \textstyleEmphasizedVernacularWords{wia}  aaw-ep  \textstyleEmphasizedVernacularWords{nia}  p-ekap-om-i-yen.  \\
      \\
\glt
\z

morning  3p.ACC  take-SS.SEQ  2p.ACC  Bpx-come-BEN-Np-FU.1p

`In the morning we will bring them (people) to you.'

\subsection{Genitive pronouns}
\hypertarget{RefHeading19641935131865}{}
Since possession can be expressed by means of three different kinds of personal pronouns in Mauwake, I call the function \textstyleEmphasizedWords{\textsc{possessive}} and the different grammatical forms \textstyleEmphasizedWords{\textsc{genitive}}, \textstyleEmphasizedWords{\textsc{dative}}\textstyleDefinedWords{} and \textstyleEmphasizedWords{\textsc{unmarked pronoun}}. All these forms have other functions besides possessive, as has already been shown for the unmarked pronoun. 

The genitive pronouns are derived from the unmarked pronouns by the ending \nobreakdash-\textstyleStyleVernacularWordsItalic{ena}:\footnote{This ending is probably related to the specifier -\textit{ena.}}

  singular  plural

1  y-ena  yi-ena

2  n-ena  ni-ena

3  o-na  wi-ena

The main function of the genitive pronoun is to indicate the possessor in a \textstyleAcronymallcaps{NP,} and the main strategy for expressing the possessor in a \textstyleAcronymallcaps{NP} is to use either the genitive pronoun or a possessive noun phrase. Unlike most other modifiers of the noun, the genitive pronoun precedes the head noun. This is in accord with Giv\'on's (1984:202) implicational hierarchy of conformity to basic word order, as well as Dryer's (2007a:62) statement about word order correlations. In Mauwake only the nominal and genitive modifiers and noun complements, which are also at the top of Giv\'on's (1984) hierarchy, precede the head noun in the \textstyleAcronymallcaps{NP}s; all the other modifiers follow the head noun.

The genitive pronoun is used when the possessor is coreferential with the subject,\footnote{It does not have to be used when the possessive relationship is clear from the context; see (234)} and its meaning is often close to English `own'.

\ea%x1805
\label{ex:x1805}
\gll Sawur  emeria  nain=ke  \textstyleEmphasizedVernacularWords{ona}  soma  mua  nain  ifakim-o-k. \\
      \\
\glt
\z

spirit  woman  that1=CF  3s.GEN  lover  man  that1  kill-PA-3s

`The spirit woman killed her (own) lover.'

\ea%x1806
\label{ex:x1806}
\gll Mua  me  wia  imen-ap=na  \textstyleEmphasizedVernacularWords{niena}  maa=ke  ... \\
      \\
\glt
\z

man  not  3p.ACC  find-SS.SEQ=TP  2p.GEN  thing=CF

`If you don't find the men, it's your (own) business {\dots}'

\ea%x567
\label{ex:x567}
\gll \textstyleEmphasizedVernacularWords{Niena}  unuma  maifa  feeke  siisim-eka. \\
      \\
\glt
\z

2p.GEN  name  paper  here.CF  write-IMP.2p

`Write your names on the paper here/ on this paper.'

In descriptive or equative clauses genitive pronouns can modify both the subject \textstyleAcronymallcaps{NP} (\stepcounter{nx}{\thenx}) and the non-verbal predicate \textstyleAcronymallcaps{NP} (\stepcounter{nx}{\thenx}), whereas the dative pronouns can modify neither. 

\ea%x568
\label{ex:x568}
\gll \textstyleEmphasizedVernacularWords{Yena}  koora  maneka  wenup. \\
      \\
\glt
\z

1s.GEN  house  big  very

`My house is very big.'

\ea%x569
\label{ex:x569}
\gll Mua  fain  me  \textstyleEmphasizedVernacularWords{nena}  niawi  akena=ke. \\
      \\
\glt
\z

man  this   not  2s.GEN  2s/p.father  true=CF

`This man is not your real father.'

It is possible for a genitive pronoun to co-occur with a dative pronoun to modify the same noun which is not coreferential with the subject. (See \sectref{sec:3.5.5} for a further discussion on the differences between genitive and dative possessives.)

\ea%x570
\label{ex:x570}
\gll \textstyleEmphasizedVernacularWords{Yena}  koora  \textstyleEmphasizedVernacularWords{efar}  aw-o-k. \\
      \\
\glt
\z

1s.GEN  house  1s.DAT  burn-PA-3s

`My house burned.'

Even when the possessor is expressed by a noun or \textstyleAcronymallcaps{NP}, the genitive pronoun is sometimes explicit, \textstyleParagraphContinuationChar{occurring either between the possessor and the possessed} \textstyleAcronymallcaps{NP}\textstyleParagraphContinuationChar{ (}\textstyleParagraphContinuationChar{\stepcounter{nx}{\thenx}}\textstyleParagraphContinuationChar{) or, quite frequently, preceding both (}\textstyleParagraphContinuationChar{\stepcounter{nx}{\thenx}}\textstyleParagraphContinuationChar{).} 

\ea%x573
\label{ex:x573}
\gll Om-em-ik-eya  sawur  emeria  \textstyleEmphasizedVernacularWords{ona}  wiawi  \\
      \\
\glt
\z

cry-SS.SIM-be-2/3s.DS  spirit  woman  3s.GEN  3s/p.father

onak=ke  ekap-emi  maak-e-mik{\dots}

3s/p.mother=CF  come-SS.SIM  tell-PA-1/3p

`While she was crying, the spirit woman's father and mother came and told her, {\dots}'

\ea%x574
\label{ex:x574}
\gll \textstyleEmphasizedVernacularWords{Wiena}  mia  kia  maa=iw  on-a-mik. \\
      \\
\glt
\z

3p.GEN  skin  white  thing=INST  do-PA-1/3p

`They did it with the Europeans' things.'

The reason for this addition of a pronoun may be the lack of case marking in nouns, which makes the processing of possessed \textstyleAcronymallcaps{NP}s more difficult when there are modifying nouns in the \textstyleAcronymallcaps{NP}. But it is also quite common for a possessive \textstyleAcronymallcaps{NP} to occur without a genitive pronoun. 

\ea%x575
\label{ex:x575}
\gll Mua  oko  miira  inawera=pa  uruf-ap  ma-i-mik,  {\dots} \\
      \\
\glt
\z

man  other  face  dream=LOC  see-SS.SEQ  say-Np-PR.1/3p

`When we see another man's face in a dream we say, {\dots}'

The third person singular possessive pronoun provides an exception to the rule that the personal pronouns are only used for the humans. However, the cases where \textstyleStyleVernacularWordsItalic{ona} `3s. possessive' refers to a non-human possessor are few and seem to require the connotation `own':

\ea%x1808
\label{ex:x1808}
\gll {\dots}\textstyleEmphasizedVernacularWords{ona}  pia=pa  nan  karu-emi  {\dots} \\
      \\
\glt
\z

3s.GEN  bamboo=LOC  there  run-SS.SIM  

`{\dots}it (molten copper) runs there in its pipe (lit:bamboo) and {\dots}'

In those instances where the possessed \textstyleAcronymallcaps{NP} in the predicative position lacks an overt head noun, three different strategies may be used. I have not observed any  difference in meaning. The genitive pronoun may occur by itself, without a head noun, which can either be deleted completely (\stepcounter{nx}{\thenx}) or substituted by \textstyleStyleVernacularWordsItalic{nain} `that' (\stepcounter{nx}{\thenx}), or the \textstyleAcronymallcaps{NP} can be expressed by a possessive phrase (\stepcounter{nx}{\thenx}) (4.4). In all these instances the head noun occurs earlier in the same sentence, or occasionally in the preceding sentence. 

\ea%x576
\label{ex:x576}
\gll Ikiwosa  \textstyleEmphasizedVernacularWords{yena},  wapena  \textstyleEmphasizedVernacularWords{yena}{\dots} \\
      \\
\glt
\z

head  1s.GEN,  hand  1s.GEN

`The head is mine (to eat), the hands are mine{\dots}'

\ea%x577
\label{ex:x577}
\gll Fikera  pun  \textstyleEmphasizedVernacularWords{wiena}  nain=ke. \\
      \\
\glt
\z

kunai.grass  too  3p.GEN  that1=CF

`The kunai grass is theirs, too.'

\ea%x578
\label{ex:x578}
\gll Maa  nain  \textstyleEmphasizedVernacularWords{yo/yena}  \textstyleEmphasizedVernacularWords{efarik}. \\
      \\
\glt
\z

thing  that1  1s.UNM/1s.GEN  1s.DAT

`That thing is mine.'

Like possessives in many other languages, the genitive pronoun may function as the subject of a nominalized clause (\sectref{sec:5.7}). The unmarked pronoun is used in the same position too; I have not found any difference in their use.

\ea%x571
\label{ex:x571}
\gll \textstyleEmphasizedVernacularWords{Yiena}  owow  maneka  ikiw-owa  nain  ma-i-yem. \\
      \\
\glt
\z

1p.GEN  village  big  go-NMZ  that1  say-Np-PR.1s

`I'm telling about our going to town.'

As ordinary main clause subjects the genitive pronouns are more emphatic than the unmarked pronouns.\footnote{Usan \citep[55]{Reesink1984}, Siroi \citep[20]{Wells1979} and Maia \citep[73]{Hardin2002} also use the same pronoun forms for possessive and emphatic pronouns, whereas Waskia (Ross and Paol 1978) does not.} The pronunciation reflects the emphasis too: these pronouns receive a stronger stress than the unmarked pronouns when used as a subject.

\ea%x572
\label{ex:x572}
\gll Aasa  enuma  \textstyleEmphasizedVernacularWords{yena}  me  suuw-i-yem. \\
      \\
\glt
\z

canoe  new  1s.GEN  not  push-Np-PR.1s

`I don't take a new canoe down myself.'

The following example has two identical genitive pronouns, the first one functioning as an emphatic subject pronoun and the second one as a possessive pronoun:

\ea%x686
\label{ex:x686}
\gll \textstyleEmphasizedVernacularWords{Yiena}  iisow,  \textstyleEmphasizedVernacularWords{yiena}  garanga  muutiw  aaw-ep \\
      \\
\glt
\z

1p.GEN  1p.ISOL  1p.GEN  family  only  take-SS.SEQ  

uup-ep  en-e-mik.

cook-SS.SEQ  eat-PA-1/3p

`Only our family by ourselves (lit: we ourselves we only, our family only) took it, cooked and ate it.'

A genitive pronoun is also possible as the subject of a relative clause, when the subject is emphatic:

\ea%x1809
\label{ex:x1809}
\gll Wi  teeria  papako  o  asip-a-mik, \\
      \\
\glt
\z

3p.UNM  group  other  3s.UNM  help-PA-1/3p

[\textstyleEmphasizedVernacularWords{ona}  eka  sesenar-ep  wienak-e-k  nain]\textsubscript{RC}.

3s.GEN  water  buy-SS.SEQ  feed.them-PA-3s  that1

`Another group helped him, those for whom \textit{he} had bought and given beer.'

When the limiting clitic -\textstyleStyleVernacularWordsItalic{iw} `only' is added to the genitive pronoun, the result is a restrictive pronoun (\sectref{sec:3.5.7}):

\ea%x604
\label{ex:x604}
\gll Yo  me  nia  maak-i-nen,  \textstyleEmphasizedVernacularWords{nien=iw}  ma-eka. \\
      \\
\glt
\z

1s.UNM  not  2p.ACC  tell-Np-FU.1s  2p.GEN=LIM  say-IMP.2p

`I will not tell you (what to do); discuss it on your own (among yourselves/as a group).'

\subsection{Dative pronouns}
\hypertarget{RefHeading19661935131865}{}
The dative case is typically associated with the semantic function of goal. The pronouns called dative in Mauwake do sometimes function as goals, but mostly they  have a locative or source function. So the term here is to be understood more as a [\textstyleEmphasizedWords{\textsc{+}}human]\textstyleEmphasizedWords{\textsc{} }\textstyleEmphasizedWords{\textsc{locative}}, which includes not only locative but goal and source as well. The dative pronouns have also grammaticalized as possessives to form possessive predicate construction (\sectref{sec:5.5.2}) and as attributive possessives to indicate that the possessor is non-coreferential with the subject. 

The dative pronouns are formed by adding -r to the accusative pronouns, with the exception of third person singular, which is identical with the plural.\footnote{The third person singular form probably used to be \textstyleFootnoteBaseChar{\textit{wo-ar}},  which is still currently used by a few people.}

  singular  plural

1  efa-r  yia-r

2  nefa-r  nia-r

3  wia-r  wia-r

The syntactic function of a dative pronoun may be clausal (a locative adverbial phrase, see \sectref{sec:4.6.1}), or \textstyleAcronymallcaps{NP}-internal (a possessive modifier, see \sectref{sec:4.1.1}). Regardless of its function, the dative pronoun is always in immediately preverbal position. 

The semantic function of a dative pronoun is related to the verb of the clause. With motion verbs it has goal function: 

\ea%x1781
\label{ex:x1781}
\gll Pok-ap  ika-iwkin  mua  \textstyleEmphasizedVernacularWords{wiar}  ekap-e-mik. \\
      \\
\glt
\z

sit-SS.SEQ  be-2/3p.DS  man  3.DAT  come-PA-1/3p

`They were sitting and (their) husbands came to them.'

\ea%x580
\label{ex:x580}
\gll Mia  kokas-owa=ke  \textstyleEmphasizedVernacularWords{wiar}  kerer-e-k. \\
      \\
\glt
\z

skin  itch-NMZ=CF  3.DAT  appear/arrive-PA-3s

`Her skin started to itch.' (Lit: `Skin itch appeared to her.)'  

With stative verbs the pronouns indicate location. (Note that the free translation needs to use a comitative expression, since English does not have a [+human] locative expression equivalent to the Mauwake dative.)  

\ea%x1782
\label{ex:x1782}
\gll Feeke  \textstyleEmphasizedVernacularWords{wiar}  ik-ok  kiiriw  mua  wiar  urup-e. \\
      \\
\glt
\z

here.CF  3.DAT  be-SS  again  man  3.DAT  ascend-IMP.2s

`Having been here with him, go (back) to your husband again.'

\ea%x1783
\label{ex:x1783}
\gll Wi  sawur  nain  ir-ami  fan  \textstyleEmphasizedVernacularWords{yiar}  pok-a-mik.{\footnotemark} \\
      \\
\glt
\z

3p.UNM  spirit  that1  go.east-SS.SIM  here  1p.DAT  sit-PA-1/3p

`The spirits, going eastward, sat here with us.'

\footnotetext{Although the most natural free translation is `{\dots}with us', comitative connotation should not be read into the Mauwake text; this is a locative.}

With verbs that indicate receiving something (take, get, buy, etc.) the dative has the semantic function of source:

\ea%x579
\label{ex:x579}
\gll Yo  emeria  Lasen=pa  \textstyleEmphasizedVernacularWords{wiar}  aaw-e-m. \\
      \\
\glt
\z

1s.UNM  woman  Lasen=LOC  3.DAT  get/take-PA-1s

`I got (my) wife from (the) Lasen (people).'

\ea%x1784
\label{ex:x1784}
\gll Kuisow  akena  ika-eya  yos=ke  \textstyleEmphasizedVernacularWords{wiar}  sesenar-ep  aaw-e-m. \\
      \\
\glt
\z

one  very  be-2/3s.DS  1s.FC=CF  3.DAT  buy-SS.SEQ  get-PA-1s

`There was only one and (it was) I (who) bought it from them.'

\ea%x1785
\label{ex:x1785}
\gll Mua  oko=ke  waaya  nain  mik-ap  \textstyleEmphasizedVernacularWords{nefar}  aaw-i-non. \\
      \\
\glt
\z

man  other=CF  pig  that1  spear-SS.SEQ  2s.DAT  get/take-Np-FU.3s

`Another man will spear the pig and take it from you.'

The ``source'' can also be more abstract. I have observed this use only with verbs indicating hearing or speaking.

\ea%x1786
\label{ex:x1786}
\gll Naap  \textstyleEmphasizedVernacularWords{wiar}  miim-a-m. \\
      \\
\glt
\z

thus  3.DAT  hear-PA-1s

`I heard thus about him/her/them.'

A locative phrase referring to a village or village area including its inhabitants is commonly used with a dative pronoun as well, otherwise it refers to just the location rather than the inhabitants. The pronoun may be used with towns or bigger areas as well, but the bigger the location, the less probable the pronoun is. In (\stepcounter{nx}{\thenx}) the people ran away to the people in the Bogia area, whereas in (\stepcounter{nx}{\thenx}) the people of Bogia town may not have been involved in the burial at all. In (\stepcounter{nx}{\thenx}) the speaker was going to the Highlands, not in order to meet the Highlanders but  to work in a location there. 

\ea%x586
\label{ex:x586}
\gll Lasen  \textstyleEmphasizedVernacularWords{wiar}  ek-a-mik. \\
      \\
\glt
\z

Lasen  3.DAT  go.east-PA-1/3p

`We went to Lasen (village).'

\ea%x587
\label{ex:x587}
\gll Baurar-ep  Bogia  kame  \textstyleEmphasizedVernacularWords{wiar}  ikiw-e-mik. \\
      \\
\glt
\z

run.away-SS.SEQ  Bogia  area  3.DAT  go-PA-1/3p

`They ran away to the Bogia area.'

\ea%x1802
\label{ex:x1802}
\gll P-ikiw-ep  Bogia=pa  nan  wu-a-mik. \\
      \\
\glt
\z

Bpx-go-SS.SEQ  Bogia=LOC  there  put-PA-1/3p

`We/They took it (a body) and buried it in Bogia.'

\ea%x1800
\label{ex:x1800}
\gll Uuriw  iinan  aasa  aaw-ep  Epa  Dabela  urup-e-mik. \\
      \\
\glt
\z

morning  sky  canoe  take-SS.SEQ  place  cold  ascend-PA-1/3p

`In the morning we took an airplane and went up to the Highlands.'

Cross-linguistically a \textstyleEmphasizedWords{\textsc{possessive predicate}} construction, a `have' construction,  has often been derived from a locative or a goal/dative construction, plus a verb of existence \citep[50-61]{Heine1997}. In the possessive predicates in Mauwake the dative pronoun precedes the verb \textstyleStyleVernacularWordsItalic{ik}\textstyleStyleVernacularWordsItalic{-} `be'. 

\ea%x1788
\label{ex:x1788}
\gll I  sira  naap  \textbf{yiar}  ik-ua. \\
      \\
\glt
\z

1p.UNM  custom  thus  1p.DAT  be-PA.3s

`We have a custom like that.' (Lit: `A custom like that is to us.)'

The possessive predicate construction is discussed in more detail in \sectref{sec:5.5.2}. 

The same dative pronoun  has also grammaticalized as a possessive attribute in a noun phrase , but here it is the semantic function of [+\textstyleEmphasizedWords{\textsc{human}}] \textstyleEmphasizedWords{\textsc{source}} that is behind the development. Conceptually the structures `\textstyleAcronymallcaps{X} took \textstyleAcronymallcaps{Y} from me' and `\textstyleAcronymallcaps{X} took my \textstyleAcronymallcaps{Y}' are very close. In (\stepcounter{nx}{\thenx}) \textstyleStyleVernacularWordsItalic{efar} can mean either `my' or `from me'. 

\ea%x581
\label{ex:x581}
\gll Nos=ke  anane  urema  \textstyleEmphasizedVernacularWords{efar}  ikum-ar-i-n. \\
      \\
\glt
\z

2s.FC=CF  always  bandicoot  1s.DAT  illicitly-INCH-Np-PR.2s

`You always steal bandicoots from me / my bandicoots.'

That it is difficult to distinguish between the roles of possessor and source is not unusual.\footnote{Sometimes it is hard to distinguish even between a possessor and a goal. In the following sentence \textit{efar} could also mean `to my place/house', with the head noun deleted: \textit{Yo me efar ekap-e}!  [1s.UNM not 1s.DAT come-IMP.2s] `Don't come to me!'           }  \citet[133]{Heine1997} mentions that early in the grammaticalization process ``these expressions can simultaneously be interpreted with reference to either their non-possessive source or to possession.'' In the following examples the source interpretation is not possible.  The example (\stepcounter{nx}{\thenx}) describes a situation in future when the speaker will already be dead and his son is made to lose his inheritance.

\ea%x1861
\label{ex:x1861}
\gll A,  yo  aamun  nan  \textstyleEmphasizedVernacularWords{efar}  ik-ua. \\
      \\
\glt
\z

ah  1s.UNM  younger.sibling  there  1s.DAT  be-PA.3s

`Ah, there is my younger brother.'

\ea%x1862
\label{ex:x1862}
\gll Ikoka  yena  yeepa  muuka=ke  yo  muuka  \textstyleEmphasizedVernacularWords{efar} \\
      \\
\glt
\z

later  1s.GEN  elder.sibling  son=CF  1s.UNM  son  1s.DAT

iirar-ep  maak-i-non  {\dots}

remove-SS.SEQ  tell-Np-FU.3s

`Later my elder brother's son will remove/displace/drive away my son and tell him, {\dots}'

This grammaticalization probably started with the verbs denoting taking and getting, but it is only a short step from there to interpreting the dative as a possessor with other verbs as well, especially as it is likely that the dative pronoun was already earlier established in the possessive predicate structure. 

\ea%x1789
\label{ex:x1789}
\gll Owowa  \textstyleEmphasizedVernacularWords{yiar}  kuuf-owa  ekap-e-mik. \\
      \\
\glt
\z

village  1p.DAT  see-NMZ  come-PA-1/3p

`They came to see our village.'

\ea%x1791
\label{ex:x1791}
\gll Auwa  afura  \textstyleEmphasizedVernacularWords{wiar}  akim-ap=ko  uruf-e. \\
      \\
\glt
\z

1s/p.father  lime  3.DAT  try-SS.SEQ=NF  see-IMP.2s

`Try father's lime and see (what it is like).'

\ea%x1790
\label{ex:x1790}
\gll Ikiwosa  \textstyleEmphasizedVernacularWords{wiar}  pepekim-ep  kaik-a-m. \\
      \\
\glt
\z

head  3.DAT  measure-SS.SEQ  tie-PA-1s

`I measured her head and tied it (a cane).'

\ea%x1795
\label{ex:x1795}
\gll No  me  emeria  \textstyleEmphasizedVernacularWords{nefar}  maak-i-mik. \\
      \\
\glt
\z

2s.UNM  not  woman  2s.DAT  tell-Np-PR.1/3p

`We are not telling/talking to your wife.'

Although the possessive is often associated with malefactive overtones as in (\stepcounter{nx}{\thenx}) and (\stepcounter{nx}{\thenx}), this is not part of its meaning (\stepcounter{nx}{\thenx}), (\stepcounter{nx}{\thenx}). 

\ea%x1787
\label{ex:x1787}
\gll Buburia  koora  \textstyleEmphasizedVernacularWords{wiar}  aw-o-k. \\
      \\
\glt
\z

bald  house  3.DAT  burn-PA-3s  

`The bald man's house burned (on him).'

\ea%x1792
\label{ex:x1792}
\gll Irak-emi  amina  \textstyleEmphasizedVernacularWords{wiar}  fo-fook-omak-e-mik.  \\
      \\
\glt
\z

fight-SS.SIM  pot  3.DAT  RDP-split-DISTR/PL-PA-1/3p

`They\textsubscript{i} fought and split their\textsubscript{j} pots.'

But since Mauwake already had genitive pronouns to indicate possession, why did another possessive strategy develop? The answer may lie in the original source function of the dative pronoun. The referent of the participant with the source function is normally another than the referent of the clausal subject, and it is this feature of non-coreferentiality with the subject that became the distinctive feature for the new possessive. 

The dative possessive construction is particularly useful for disambiguating between the subject and the possessor, if both of them are in third person. The following two pairs of examples show this clearly. The corresponding English sentences are ambiguous, whereas the Mauwake sentences are not: 

\ea%x1797
\label{ex:x1797}
\gll Yena  eremena=ke  \textstyleEmphasizedVernacularWords{ona }   siowa  aruf-eya  kepura  ku-o-k. \\
      \\
\glt
\z

1s.GEN  nephew=CF  3s.GEN  dog  hit-2/3s.DS  leg  break-PA-3s

`My nephew\textsubscript{i} hit his\textsubscript{i} dog and its leg broke.'

\ea%x1796
\label{ex:x1796}
\gll Yena  eremena=ke  siowa  \textstyleEmphasizedVernacularWords{wiar}  aruf-eya  kepura  ku-o-k. \\
      \\
\glt
\z

1s.GEN  nephew=CF  dog  3.DAT  hit-2/3s.DS  leg  break-PA-3s

`My nephew\textsubscript{i} hit his/her\textsubscript{j} dog and its leg broke.'

\ea%x1798
\label{ex:x1798}
\gll Wis=ke  wiawi  maak-e-mik. \\
      \\
\glt
\z

3p.FC=CF  3s/p.father  tell-PA-1/3p

`(It was) they\textsubscript{i} (who) told their\textsubscript{i} father.'

\ea%x1799
\label{ex:x1799}
\gll Wis=ke  wiawi  \textstyleEmphasizedVernacularWords{wiar}  maak-e-mik. \\
      \\
\glt
\z

3p.FC=CF  3s/p.father  3.DAT  tell-PA-1/3p

`(It was) they\textsubscript{i} (who) told their\textsubscript{j} father.'

Currently the dative possessive has to be used when the possessor is non-coreferential with the subject or recipient of the clause.

\ea%x588
\label{ex:x588}
\gll Marasin  nain=ke  kema  \textstyleEmphasizedVernacularWords{wiar}  iw-a-k. \\
      \\
\glt
\z

medicine  that1=CF  liver  3.DAT  go-PA-3s

`The medicine went into his liver.'

\ea%x853
\label{ex:x853}
\gll Wiowa  nain  o  wapena=pa  \textstyleEmphasizedVernacularWords{wiar } ku-o-k. \\
      \\
\glt
\z

spear  that1  3s.UNM  hand=LOC  3.DAT  break-PA-3s

`The spear broke in his hand.'

\ea%x1794
\label{ex:x1794}
\gll Pina  ...  \textstyleEmphasizedVernacularWords{nefar}  kaken-ami  welaw-i-kuan. \\
      \\
\glt
\z

guilt  {\dots}  2s.DAT  straighten-SS.SIM  finish-Np-FU.3p

`They will straighten your(sg) {\dots} guilt and finish it.'

It follows from the non-coreferentiality restriction that a possessed \textstyleAcronymallcaps{NP} with the possessive pronoun in the dative cannot be the subject of a clause.  

In the possessor function the dative pronoun does not co-occur with the accusative pronoun in the same clause (\stepcounter{nx}{\thenx}). In the rare occasion where there would be rivalry for the position immediately preceding the verb, the accusative is chosen (\stepcounter{nx}{\thenx}) rather than the dative (\stepcounter{nx}{\thenx}). 

\ea%x584
\label{ex:x584}
\gll *Yena  muuka  erup  \textstyleEmphasizedVernacularWords{efar}  \textstyleEmphasizedVernacularWords{wia}  aaw-o-k. \\
      \\
\glt
\z

1s.GEN  son  two  1s.DAT  3p.ACC  take-PA-3s

\ea%x583
\label{ex:x583}
\gll Yena  muuka  erup  \textstyleEmphasizedVernacularWords{wia}  aaw-o-k. \\
      \\
\glt
\z

1s.GEN  son  two  3p.ACC  take-PA-3s

`He took my two sons.'

\ea%x1928
\label{ex:x1928}
\gll ?Yena  muuka  erup  \textstyleEmphasizedVernacularWords{efar}  aaw-o-k. \\
      \\
\glt
\z

1s.GEN  son  two  3.DAT  take-PA-3s

But if the dative pronoun has the semantic role of goal, it may co-occur with an accusative pronoun; in this case it precedes the accusative pronoun.

\ea%x1576
\label{ex:x1576}
\gll O  \textstyleEmphasizedVernacularWords{wiar}  \textstyleEmphasizedVernacularWords{nefa}  sesek-i-yem. \\
      \\
\glt
\z

3s.UNM  3.DAT  2s.ACC  send-Np-PR.1s

`I am sending you to him.'

The use of the genitive possessive pronoun is much less restricted. Besides being employed where the possessor is coreferential with the subject (\stepcounter{nx}{\thenx}) or recipient (\stepcounter{nx}{\thenx}), it can also be used when a possessed \textstyleAcronymallcaps{NP} is the subject or non-verbal predicate of a descriptive or equative clause (\stepcounter{nx}{\thenx}). 

\ea%x589
\label{ex:x589}
\gll Eema=ke  \textstyleEmphasizedVernacularWords{ona}  kolos  Garamin  iw-o-k. \\
      \\
\glt
\z

Eema=CF  3s.GEN  dress  Garamin  give.him/her-PA-3s

`Eema\textsubscript{i} gave her\textsubscript{i} dress to Garamin.'

\ea%x590
\label{ex:x590}
\gll Eema=ke  Garamin  \textstyleEmphasizedVernacularWords{ona}  kolos  iw-o-k. \\
      \\
\glt
\z

Eema=CF  Garamin  3s.GEN  dress  give.him/her-PA-3s

`Eema\textsubscript{i} gave Garamin\textsubscript{j} her\textsubscript{j} dress.'

\ea%x591
\label{ex:x591}
\gll \textstyleEmphasizedVernacularWords{Yena}  koora  maneka  wenup. \\
      \\
\glt
\z

1s.GEN  house  big  very

`My house is very big.'

The genitive or unmarked pronoun may co-occur together with the dative pronoun referring to the same person, thus emphasizing the possessive function of the dative (\stepcounter{nx}{\thenx}), (\stepcounter{nx}{\thenx}).

\ea%x1863
\label{ex:x1863}
\gll \textstyleEmphasizedVernacularWords{Yo}  emeria  \textstyleEmphasizedVernacularWords{efar}  uruf-a-man=i  e  wia? \\
      \\
\glt
\z

1s.UNM  woman  1s.DAT  see-PA-2p=QM  or  no

`Have you seen my wife or not?'

\ea%x593
\label{ex:x593}
\gll \textstyleEmphasizedVernacularWords{Ona}  koora=pa  \textstyleEmphasizedVernacularWords{wiar}  wu-a-mik. \\
      \\
\glt
\z

3s.GEN  house=LOC  3.DAT  put-PA-1/3p

`They put it in his (own) house.'

Example (\stepcounter{nx}{\thenx}) shows how the genitive and dative possessives, in \textstyleEmphasizedWords{\textsc{different}} person forms, can modify the same noun. The dative pronoun can here be interpreted either as a possessive `your (wives)' or as a source `(wives) from you'.

\ea%x594
\label{ex:x594}
\gll Emeria  ikoka  Yaapan  \textstyleEmphasizedVernacularWords{wiena}  \textstyleEmphasizedVernacularWords{niar}  aaw-i-kuan. \\
      \\
\glt
\z

woman  later  Japanese  3p.GEN  2p.DAT  take-Np-FU.3p

`Later the Japanese will take your wives as their own.'

In the following example, where there are several possessive \textstyleAcronymallcaps{NP}s, the two genitive pronouns both refer to the man who is identified in the preceding text. In the second clause the possessor is a modifier in the subject \textstyleAcronymallcaps{NP}, so it has to be in the genitive. The subject in the third clause is the lover's spirit, and because only one dative possessive is possible in one clause, here it is naturally assigned to the man's wife whose things were thrown around, and the man is referred to by a genitive possessive. In this case the genitive possessive also underlines the fact that one of the women was the man's own wife. The clauses are separated by brackets.

\ea%x1318
\label{ex:x1318}
\gll [Ikiw-ep-ik-eya]  [\textstyleEmphasizedVernacularWords{ona}  soma  emeria  nain  kukusa  nain=ke \\
      \\
\glt
\z

go-SS.SEQ-be-2/3s.DS  3s.GEN  lover  woman  that1  spirit  that1=CF

ekap-ep]  [\textstyleEmphasizedVernacularWords{ona}  emeria  nain  maa  \textstyleEmphasizedVernacularWords{wiar}

come-SS.SEQ  3s.GEN  woman  that1  thing  3.DAT  

wafufur-eya]  [naap  maak-e-k,]  {\dots}

throw.around-2/3s.DS  thus  tell-PA-3s

`When he\textsubscript{i} was gone, his\textsubscript{i} lover-woman's\textsubscript{j} spirit came and threw around his\textsubscript{i} (own) wife's\textsubscript{k} things, and she\textsubscript{k} told her like this, {\dots}'

Dative pronouns also have a longer form, with the suffix -\textstyleStyleVernacularWordsItalic{ik}: \textstyleStyleVernacularWordsItalic{efarik}, \textstyleStyleVernacularWordsItalic{nefarik} etc. The pronoun is a contracted form of the `have' construction, with just the root left of the verb \textstyleStyleVernacularWordsItalic{ik}- `be', which has been suffixed to the pronoun. In natural text the frequency of these pronouns is extremely low. They have to be used when the dative pronoun is clause final (\stepcounter{nx}{\thenx})-(\stepcounter{nx}{\thenx}), as the regular dative pronoun only occurs pre-verbally. The longer form is often accompanied by either the genitive pronoun (\stepcounter{nx}{\thenx}) or the unmarked pronoun (\stepcounter{nx}{\thenx}), which suggests that it is more emphasized than the simple dative.

\ea%x597
\label{ex:x597}
\gll Miiw  ara  gelemuta  nain  \textstyleEmphasizedVernacularWords{yiena  yiarik}. \\
      \\
\glt
\z

land  piece  small  that1  1p.GEN  1p.DAT

`That small piece of ground is ours.'

\ea%x598
\label{ex:x598}
\gll Wiawi=ke  amap-or-o-k=i,  weke  \textstyleEmphasizedVernacularWords{wiarik}? \\
      \\
\glt
\z

3s/p.father=CF  Bpx-descend-PA-3s=QM  3s/p.grandfather  3.DAT

`Did her father take her down to her grandfather?'

The long dative with a ``receive'' type verb in the following  example can be traced back to \textstyleStyleVernacularWordsxiiptItalic{niar ikeya} `you had it, and{\dots}':

\ea%x596
\label{ex:x596}
\gll Yo  mesa  up-owa  fain  \textstyleEmphasizedVernacularWords{ni}  \textstyleEmphasizedVernacularWords{niarik} \\
      \\
\glt
\z

1s.UNM  winged.bean  plant-NMZ  this  2p.UNM  2p.DAT

aaw-ep  isak-e-m.

get-SS.SEQ  plant-PA-1s

`I got these winged bean seeds from you and planted them.'

\subsection{Isolative pronouns}
\hypertarget{RefHeading19681935131865}{}
The isolative pronoun forms are based on the unmarked pronouns. The ending \nobreakdash-\textstyleStyleVernacularWordsItalic{isow}\textstyleStyleVernacularWordsItalic{,} which the numeral \textstyleStyleVernacularWordsItalic{kuisow} `one' shares with these pronouns, may be an earlier morpheme possibly meaning `alone'. The meaning of the isolative pronouns is roughly `X alone' or `by -self'. In the singular forms the vowel /o/ is replaced by /a/, since  /oi/ is not a permissible vowel sequence in Mauwake. 

singular  plural

1  ya-isow  (y)i-isow

2  na-isow  ni-isow

3  wa-isow  wi-isow

When an isolative pronoun functions as a subject, which is \textstyleEmphasizedWords{\textsc{not}} theme (\sectref{sec:9.1}), it is alone (\stepcounter{nx}{\thenx}); but more commonly it is both theme and subject, and is preceded by the unmarked pronoun also showing the case marking overtly (\stepcounter{nx}{\thenx}). 

\ea%x599
\label{ex:x599}
\gll Manina  \textstyleEmphasizedVernacularWords{waisow}  mauw-ap  neeke  wu-a-k. \\
      \\
\glt
\z

garden  3s.ISOL  work-SS.SEQ  there.CF  put-PA-3s

`He made his garden alone/by himself and left it there.'

\ea%x600
\label{ex:x600}
\gll \textstyleEmphasizedVernacularWords{No}  \textstyleEmphasizedVernacularWords{naisow}  or-op  kaul  wafur-e. \\
      \\
\glt
\z

2s.UNM  2s.ISOL  descend-SS.SEQ  hook  throw-IMP.2s

`Go down alone/by yourself and do fishing (lit: throw the hook).' 

The example (\stepcounter{nx}{\thenx}) has an accusative pronoun to show the case and an initial unmarked pronoun \textstyleStyleVernacularWordsxiiptItalic{yo} `I' to mark the object as theme.

\ea%x601
\label{ex:x601}
\gll \textstyleEmphasizedVernacularWords{Yo}  \textstyleEmphasizedVernacularWords{yaisow}  me  \textstyleEmphasizedVernacularWords{efa}  keraw-a-k. \\
      \\
\glt
\z

1s.UNM  1s.ISOL  not  1s.ACC  bite-PA-3s

`It didn't bite only me.' (Or: `It wasn't only me that it bit.')

When the isolative pronoun is preceded by the genitive/emphatic pronoun it is intensified:

\ea%x1813
\label{ex:x1813}
\gll Aakisa  mua  iperowa  nain  \textstyleEmphasizedVernacularWords{ona}  \textstyleEmphasizedVernacularWords{waisow}  soor  owowa=pa \\
      \\
\glt
\z

now  man  middle.aged  that1  3s.GEN  3s.ISOL  jungle  village=LOC

ika-i-ya.

be-Np-PR.3s

`Now that middle-aged man is staying all by himself in a jungle hamlet.'

In the plural the meaning is `\textstyleEmphasizedWords{\textsc{only}} we/you/they (as a \textstyleEmphasizedWords{\textsc{group}})'. 

\ea%x602
\label{ex:x602}
\gll Wi  feeke  ika-uk,  \textstyleEmphasizedVernacularWords{i  iisow}  ikiw-i-yen.  \\
      \\
\glt
\z

3p.UNM  here.CF  be-IMP.3p  1p.UNM  1p.ISOL   go-Np-FU.1p

`Let them stay here, only we will go.'

When the first syllable of a plural isolative pronoun is reduplicated, the pronoun refers to \textstyleEmphasizedWords{\textsc{individuals}} in the group:

\ea%x603
\label{ex:x603}
\gll \textstyleEmphasizedVernacularWords{Ii-iisow}  pok-ap  opora  siisim-ep  weeser-eya  \\
      \\
\glt
\z

RDP-1p.ISOL  sit-SS.SEQ  talk  write-SS.SEQ  finish-2/3s.DS  

unow=iya  aakun-e-mik.

many=COM  talk-PA-1/3p

`We sat and wrote separately, and then talked together.'

Although the pronouns in (\stepcounter{nx}{\thenx}) and (\stepcounter{nx}{\thenx}) sound rather similar, there is a stress difference between them. In the former, \textstyleStyleVernacularWordsItalic{iisow} gets stronger stress than \textstyleStyleVernacularWordsItalic{i}, in the latter the first syllable of the reduplicated word is stressed.

\subsection{Restrictive pronouns}
\hypertarget{RefHeading19701935131865}{}
The restrictive pronouns are formed by adding the limiting clitic -\textstyleStyleVernacularWordsItalic{iw} `only' either to a genitive pronoun or to a focal pronoun (\sectref{sec:3.12.6}). When it is added to a genitive pronoun it means `on one's own':

\ea%x605
\label{ex:x605}
\gll No  \textstyleEmphasizedVernacularWords{nena=iw}  ma-i-n=i? \\
      \\
\glt
\z

2s.UNM  2s.GEN=LIM  say-Np-PR.2s=QM

`Do you say it on your own?' (i.e. `Did you think of it yourself?')

\ea%x606
\label{ex:x606}
\gll O=ko  me  efa  maak-e-k,  \textstyleEmphasizedVernacularWords{yena=iw} \\
      \\
\glt
\z

3s.UNM=NF  not  1s.ACC  tell-PA-3s  1s.GEN=LIM  

amis-ar-e-m.

knowledge-INCH-PA-1s

`He/she didn't tell me, I learned it on my own.'

\ea%x607
\label{ex:x607}
\gll \textstyleEmphasizedVernacularWords{Yien=iw}  ikiw-ik-ua. \\
      \\
\glt
\z

1p.GEN=LIM  go-be-PA.3s

`Let's go on our own (as a group, or one by one).'

When the limiting clitic is added to the focal form of the free pronoun it adds the meaning of exclusiveness to the pronoun:

\ea%x608
\label{ex:x608}
\gll Anane  \textstyleEmphasizedVernacularWords{nos=iw}  nefa  maak-i-ya. \\
      \\
\glt
\z

always  2s.FC=LIM  2s.ACC  tell-Np-PR.3s

`He always talks to you only.'

\ea%x609
\label{ex:x609}
\gll Wi  anane  \textstyleEmphasizedVernacularWords{is=iw}  yiam=iya  irak-i-mik. \\
      \\
\glt
\z

3p.UNM  always  1p.FC=LIM  1p.REFL=COM  fight-Np-PR.1/3p

`They always fight with us only.'

\subsection{Reflexive-reciprocal pronouns}
\hypertarget{RefHeading19721935131865}{}
The reflexive-reciprocal pronouns have the unmarked pronouns as their basis, but the derivative suffix is slightly different for singular and plural.  They are as follows:

  singular  plural

1  y-ame\footnote{In the coastal dialect the singular suffix is -\textit{ama}.}  yi-am

2  n-ame  ni-am

3  w-ame  wi-am

The singular forms are used as reflexives only (\stepcounter{nx}{\thenx}), (\stepcounter{nx}{\thenx}),  the plural forms both as reflexives (\stepcounter{nx}{\thenx}), (\stepcounter{nx}{\thenx})  and as reciprocals (\stepcounter{nx}{\thenx}), (\stepcounter{nx}{\thenx}).

\ea%x610
\label{ex:x610}
\gll Naap  on-ap  \textstyleEmphasizedVernacularWords{yame}  amukar-e-m. \\
      \\
\glt
\z

thus  do-SS.SEQ  1s.REFL  scold-PA-1s

`Having done so I scolded myself (i.e. was angry at myself).'

\ea%x1864
\label{ex:x1864}
\gll Iinan  akena  ikiw-ep  \textstyleEmphasizedVernacularWords{wame}  pipilim-ep  \\
      \\
\glt
\z

on.top  very  go-SS.SEQ  3s.REFL  hide-SS.SEQ  

aakun-em-ika-i-non.

speak-SS.SIM-be-Np-FU.3s

`It (= a bird) will go very high up and hide itself and keep making its calls.'

\ea%x611
\label{ex:x611}
\gll \textstyleEmphasizedVernacularWords{Niam}  tuun-ap  teeria  erup  wu-eka. \\
      \\
\glt
\z

2p.REFL  count-SS.SEQ  group  two  put-IMP.2p

`Count yourselves and form two groups.'

\ea%x1865
\label{ex:x1865}
\gll Nainiw  sande  uura  \textstyleEmphasizedVernacularWords{yiam}  fiirim-e-mik. \\
      \\
\glt
\z

again  Sunday  night  1p.REFL  gather-PA-1/3p

`Again on Sunday night we gathered.'

\ea%x612
\label{ex:x612}
\gll \textstyleEmphasizedVernacularWords{Wiam}  fook-ap  irak-e-mik. \\
      \\
\glt
\z

3p.REFL  split-SS.SEQ  fight-PA-1/3p

`They split from each other and fought.'

\ea%x1866
\label{ex:x1866}
\gll Sarir-ap  {\dots } \textstyleEmphasizedVernacularWords{yiam}  far-i-mik. \\
      \\
\glt
\z

surround-SS.SEQ  {\dots}  1p.REFL  call-Np-PR.1/3p

`We surround (the fish) {\dots} and call each other.'

In many contexts only the reflexive or the reciprocal interpretation is natural. But a potential ambiguity in some contexts is resolved by adding a genitive pronoun to mark the reflexive (\stepcounter{nx}{\thenx}) and an unmarked or restrictive pronoun to mark the reciprocal pronoun (\stepcounter{nx}{\thenx}).

\ea%x614
\label{ex:x614}
\gll \textstyleEmphasizedVernacularWords{Niena  niam}  kookal-eka. \\
      \\
\glt
\z

2p.GEN  2p.REFL  like-IMP.2p

`Like/love yourselves.'

\ea%x613
\label{ex:x613}
\gll \textstyleEmphasizedVernacularWords{Ni/nieniw  niam}  kookal-eka. \\
      \\
\glt
\z

2p.UNM/2p.LIM  2p.REFL  like-IMP.2p

`Like/love each other.'

The reflexives are not very frequent in Mauwake, because they seem to be fairly strongly connected with [+Control]. If one hurts oneself unintentionally, the cause(r) or instrument occupies the subject position instead of the person hurt. Thus, (\stepcounter{nx}{\thenx}) is a semantically appropriate equivalent for the English clause `I cut myself with a knife':

\ea%x617
\label{ex:x617}
\gll Fura=ke  efa  puuk-a-k. \\
      \\
\glt
\z

knife=CF  1s.ACC  cut-PA-3s

`A knife cut me.'

But a reflexive pronoun is used especially in expressions involving \textstyleParagraphChari{body parts}  when one does something to oneself, and the instrument is not known or mentioned (\stepcounter{nx}{\thenx}). In corresponding expressions English often uses possessive rather than reflexive pronouns.

\ea%x618
\label{ex:x618}
\gll Merena  \textstyleEmphasizedVernacularWords{yame}  puuk-a-m. \\
      \\
\glt
\z

leg  1s.REFL  cut-PA-1s

`I cut my leg.' (Or: `I cut myself in the leg.')

The plural forms of the reflexive pronouns have another, quite different use: when they are followed by numerals, especially by `two' or `three', they function as dual/trial etc. forms for the personal pronouns. They are considered to be in the nominative case when not followed by other pronoun forms (\stepcounter{nx}{\thenx}). Other cases need to be shown by appropriate additional pronouns (\stepcounter{nx}{\thenx}).

\ea%x615
\label{ex:x615}
\gll \textstyleEmphasizedVernacularWords{Yiam  arow}  nain  miim-ap  soran-e-mik. \\
      \\
\glt
\z

1p.REFL  three  that1  hear-SS.SEQ  be.startled-PA-1/3p

`The three of us heard that and were startled.'

\ea%x616
\label{ex:x616}
\gll Amia  mua=ke  \textstyleEmphasizedVernacularWords{wiam  erup}  nain  \textstyleEmphasizedVernacularWords{wia}  nokar-e-k,  {\dots} \\
      \\
\glt
\z

bow  man=CF  3p.REFL  two  that1  3p.ACC  ask-PA-3s

`The policeman asked those two {\dots}'

\subsection{Comitative pronouns}
\hypertarget{RefHeading19741935131865}{}
The comitative set is a mixture as far as the basic forms are concerned. The first and second person singular forms have accusative pronouns, all the others have the reflexive pronouns as their roots. The ending is the comitative clitic -\textstyleStyleVernacularWordsItalic{iya} (\sectref{sec:3.12.1}), which can also be added to nouns and is one of several ways of expressing accompaniment in Mauwake. The first and second person singular forms have a transition consonant -\textstyleStyleVernacularWordsItalic{m}- preceding the comitative clitic.

  singular  plural

1  efa-m-iya  yiam-iya

2  nefa-m-iya  niam-iya

3  wama-iya  wiam-iya

\ea%x619
\label{ex:x619}
\gll Lasen  mua  emeria  \textstyleEmphasizedVernacularWords{wiam=iya}  me  aakun-e-mik. \\
      \\
\glt
\z

Lasen  man  woman  3p.REFL=COM  not  talk-PA-1/3p

`We didn't talk with the Lasen people.'

\ea%x620
\label{ex:x620}
\gll Liisa  Poh  San  ikos  \textstyleEmphasizedVernacularWords{yiam=iya}  soomar-emi  {\dots} \\
      \\
\glt
\z

Liisa  Poh  San  with  1p.REFL=COM  walk-SS.SIM

`Liisa and Poh San walked with us and {\dots}'

\subsection{Primary and secondary reference of personal pronouns}
\hypertarget{RefHeading19761935131865}{}
Typically pronouns refer to the persons the form indicates: first person singular to the speaker, second person singular to the addressee etc.  Besides this primary, or default, reference some pronouns may also have a secondary reference, if the person and/or number of the referent(s) is different from that indicated by the pronoun.

In Mauwake both the first and second person singular forms as well as the third person plural marking on verbs can be used for non-specific, or generic, reference. They occur particularly in explanations of customs or general principles, and in examples. The sentences are usually in the future tense and therefore hypothetical. In these texts the second person singular pronoun and the third person verb marking can alternate quite freely. Example (\stepcounter{nx}{\thenx}) is from a text describing the adoption process in general, and example (\stepcounter{nx}{\thenx}) was said to a person who does not even have a spirit name to call upon, nor does know how to spear pigs. Here the pronouns have acquired a non-deictic role: their correct interpretation does not depend on the non-linguistic context \citep[260]{AndersonEtAl1985}%Keenan
.

\ea%x621
\label{ex:x621}
\gll \textstyleEmphasizedVernacularWords{Yo}  muuka  kookal-ep  \textstyleEmphasizedVernacularWords{yena}  samapora  wia  \\
      \\
\glt
\z

1s.UNM  son  like-SS.SEQ  1s.GEN  clan  3p.ACC

maak-i-nen.

tell-Np-FU.1s

`When I like to have a son/child I will tell my clan.' (Or: `When \textit{one} wants a child he will tell his own clan.')

\ea%x622
\label{ex:x622}
\gll \textstyleEmphasizedVernacularWords{No}  waaya  mik-ap  inasina  unuma  me  unuf-i-nan=na \\
      \\
\glt
\z

2s.UNM  pig  spear-SS.SEQ  spirit  name  not  call-Np-FU.2s=TP

mua  oko=ke  nainiw  mik-ap  \textstyleEmphasizedVernacularWords{nefar}  aaw-i-non.

man  other=CF  again  spear-SS.SEQ  2s.DAT  take-Np-FU.3s

`If you spear a pig and don't call your spirit name, another man will spear it again and take it from you.' (Or: `If \textit{one} spears a pig{\dots}')

When a maximally generic object is needed for a transitive verb, or when there is no overt object available, the first person plural accusative form is used. 

\ea%x623
\label{ex:x623}
\gll Ifa  nain=ke  \textstyleEmphasizedVernacularWords{yia}  keraw-i-ya. \\
      \\
\glt
\z

snake  that1=CF  1p.ACC  bite-Np-PR.3s

`That snake bites.'

\ea%x624
\label{ex:x624}
\gll Marasin  fain  \textstyleEmphasizedVernacularWords{yia}  girin-i-ya. \\
      \\
\glt
\z

medicine  this   1p.ACC  smart-Np-PR.3s

`This medicine smarts.'

\subsection{Use of personal pronouns in text}
\hypertarget{RefHeading19781935131865}{}
In Mauwake it is possible to leave the subject pronoun out, as the person and number of the subject are marked on the verb suffix. And this is not only possible but very common: approximately only 6\% of all the clauses in narrative and descriptive texts have a pronominal subject, compared to about 30\% of the clauses having a subject \textstyleAcronymallcaps{NP} of any kind. As the other arguments are not marked on the verb, except for a two-way distinction for beneficiary (\sectref{sec:3.7.3.1}), other than subject pronouns need to be used for them if there is no full \textstyleAcronymallcaps{NP}, and they are often employed even when there is a \textstyleAcronymallcaps{NP}.

The frequency of subject pronouns depends on whether the person referred to is first, second or third, and on the type of text as well. The first person, both in singular and plural, is commonly referred to with a pronoun, instead of just a verb suffix.  Second person pronouns are very frequent in hortatory texts and are used somewhat in conversations. Most narratives in the data have their main participants in third person, but pronouns are used to refer to them quite rarely. 

A pronoun may be used for the second mention of a newly established topic (\sectref{sec:9.2.2}). In particular when an important participant has been introduced by a proper name, in the next sentence (s)he can be referred to by a personal pronoun. 

\ea%x1867
\label{ex:x1867}
\gll Eema=ke  waisow  amis-ar-e-k.  \textstyleEmphasizedVernacularWords{Os=ke}  uuriw  \\
      \\
\glt
\z

Eema=CF  3s.ISOL  knowledge-INCH-PA-3s  3s.FC=CF  morning

urup-emi{\dots}

rise-SS.SIM

`Only Eema knew. She got up in the morning and {\dots}'

When a participant has been established as the topic, (s)he is referred to with a verb suffix only, or with a \textstyleAcronymallcaps{NP} if a better identification is needed. A pronoun is used mainly when the topic is re-activated after being inactive for a while (\sectref{sec:9.2.3}). The example (\stepcounter{nx}{\thenx}) is from a text where a couple goes down to the husband's village and then returns to the wife's village. The wife's relatives, inactive as a topic for the span of five clauses, are re-assigned the topic status with the pronoun \textstyleStyleVernacularWordsItalic{wi}  `they'. 

\ea%x1922
\label{ex:x1922}
\gll Or-op  ik-ok  nainiw  urup-e-mik.  Aria  \textstyleEmphasizedVernacularWords{wi} \\
      \\
\glt
\z

descend-SS.SEQ  be-SS  again  ascend-PA-1/3p  alright  3p.UNM

samapora  maneka  fook-ap  {\dots}

floor  big  split-SS.SEQ  

`They (=the couple) went down and after a while they came up again. Alright they (=the wife's relatives) split (wood for) a big floor and {\dots}'

In commands (\sectref{sec:7.3}) the subject pronouns are more frequent than in statements.\footnote{As many as 39\% of commands in the text material have a pronoun subject, as against 6\% in statements.} The pronoun here is not a vocative; that would be separated from the rest of the clause by a pause, whereas a subject is not. The following is a fairly typical command:

\ea%x685
\label{ex:x685}
\gll Ni  ikiw-eka! \\
      \\
\glt
\z

2p  go-IMP.2p

`Go (2p)!'

This is an unusual feature cross-linguistically, as languages tend to drop the subject pronoun in imperative clauses (Giv\'on 1979:80).\footnote{The relatively high frequency of subject pronouns in imperative clauses may not be a peculiarity of Mauwake only. The grammatical descriptions of Papuan languages often state that the subject pronoun is optional in these clauses, but give no information as to their actual frequency. Personal communication with other field linguists working on Papuan languages gives reason to suggest that an overt personal pronoun with the imperative may be more common than is generally assumed.}  

\section{Spatial deictics}
\hypertarget{RefHeading19801935131865}{}
This section brings together what are often called demonstrative pronouns and deictic locative adverbs. What is common to them is the spatial orientation based on the location of the speaker, as well as morphological similarity. The whole deictic system, which also includes personal and temporal deixis, is discussed briefly in 6.3. 

Deictics operate on the scale of proximity, making reference to something else on the basis of location (Halliday and Hasan 1976:57-58). The relative proximity may be measured either from the speaker or from the speaker and addressee. Papuan languages manifest both these types as well as a combination of the two. Elevation and visibility may be additional parameters, so the demonstrative systems range from a simple and rather common two-term system to quite complicated ones \citep[75-77]{Foley1986}. Two-way distinctions are found in Siroi \citep[20]{Wells1979} and Golin \citep{Bunn1974}, three-way distinctions in Waskia \citep[59]{RossEtAl1978}%Paol
, Bine \citep{Saari1985} and Korafe \citep[65]{FarrEtAl1981}%Whitehead
. Usan has four basic deictics, but derivations extend the system into an elaborate one \citep[76-81]{Reesink1987}. \citet[38-39]{Murane1974} reports 19 locatives in Daga that are also used as demonstrative pronouns. 

\subsection{The basic spatial deixis in Mauwake}
\hypertarget{RefHeading19821935131865}{}
The main factors dividing the deictic space in Mauwake are the relative proximity to the speaker, and visibility. There are four deictic roots, one of them proximal and three distal. They are as follows: 

fa-  `here' (close to speaker, visible)  proximate

na-  `there' (away from the speaker; generic)  distal-1

eef-  `here/there'  (rather close, usually visible)  distal-2

een-  `there' (far away, usually not visible)  distal-3

The proximal deictic \textstyleStyleVernacularWordsItalic{fa}- indicates close proximity to the speaker: prototypically the referent marked with \textstyleStyleVernacularWordsItalic{fain} `this' can be touched by the speaker, and \textstyleStyleVernacularWordsItalic{fan} `here' indicates the speaker's location or close proximity to it. The distal-1 deictic \textstyleStyleVernacularWordsItalic{na}\textit{-} indicates a distance that is out of touching distance to the speaker; the distance to the addressee is irrelevant. \textstyleStyleVernacularWordsItalic{Na}- is the most neutral and the least restricted of the three distal deictics, and its frequency is extremely high because of the various functions that the demonstrative \textstyleStyleVernacularWordsItalic{nain} has. On the other hand, the words formed with both the distal\nobreakdash-2 root \textstyleStyleVernacularWordsItalic{eef}\textit{\nobreakdash-} and the distal\nobreakdash-3 root \textstyleStyleVernacularWordsItalic{een}\nobreakdash-, although available, are rarely used. They may be employed when the pragmatic situation meets the semantic specification for their occurrence, and they are needed when more than one far deictic is called for. Often the distance is a relative matter, and the speaker has a subjective choice between the different deictics.

The deictic roots suffixed with -\textstyleStyleVernacularWordsItalic{in}, marking given information, are used as demonstratives. When the roots are suffixed with -\textstyleStyleVernacularWordsItalic{an} `locative', the words function as locative adverbs. The distribution of  both these suffixes is very restricted: they are only attached to deictic or question word (\sectref{sec:3.7.1}) roots.

The deictic manner adverbs (\sectref{sec:3.6.4}) are also based on the same roots.

\subsection{Demonstratives}
\hypertarget{RefHeading19841935131865}{}
The four demonstratives in Mauwake are formed by one of the deictic roots plus the suffix -\textstyleStyleVernacularWordsItalic{in} indicating given information. 

In Mauwake the demonstratives are like the personal pronouns in that they can function as the sole head of a \textstyleAcronymallcaps{NP}. But they differ from the personal pronouns in that they do not have the case forms typical of the latter. In this respect the demonstratives are more like adjectives. Another feature that they share with adjectives is that they mainly function as modifiers in a \textstyleAcronymallcaps{NP}. But unlike the adjectives, which only occur alone in complement position (unless the \textstyleAcronymallcaps{NP} is elliptical), the demonstratives occur by themselves in several clause positions. 

The numeral modifiers are positioned between an adjective and a demonstrative in a \textstyleAcronymallcaps{NP} (\stepcounter{nx}{\thenx}), but never between two adjectives (\stepcounter{nx}{\thenx}).

\ea%x631
\label{ex:x631}
\gll koora  maneka  arow  \textstyleEmphasizedVernacularWords{nain} \\
      \\
\glt
\z

house  big  three  that1

`those three big houses'

\ea%x632
\label{ex:x632}
\gll siowa  sepa  gelemuta  erup \\
      \\
\glt
\z

dog  black  small  two

`two small black dogs'

There is a clear distinction in Mauwake between human and non-human reference, which shows in the choice of a pronoun vs. a demonstrative. A third person pronoun is not used for non-humans, whereas demonstratives in isolation\footnote{Demonstratives are common as \textit{modifiers} of NPs referring to humans.} are normally only used for non-humans. The only exception in my data is example (\stepcounter{nx}{\thenx}); \textstyleStyleVernacularWordsItalic{nain} `that' would not be acceptable even here.

\ea%x633
\label{ex:x633}
\gll No{\footnotemark} \textstyleEmphasizedVernacularWords{fain}  me  nena  niawi  akena=ke. \\
      \\
\glt
\z

2s.UNM  this  not  2s.GEN  2s/p.father  true=CF

`This is not your true father.'

\footnotetext{\textit{No} `you' is an extra-clausal theme, not part of the subject.} 

Apart from the proximal demonstrative \textstyleStyleVernacularWordsItalic{fain} `this', the other demonstratives are not mutually exclusive. The distal-1 demonstrative \textstyleStyleVernacularWordsItalic{nain} `that' is the least restricted of the three, and it is extremely frequent, whereas both \textstyleStyleVernacularWordsItalic{eefin} `this/that' and \textstyleStyleVernacularWordsItalic{eenin} `that' are very rarely used. In (\stepcounter{nx}{\thenx}) the distances of the two mountains fit the specifications for \textstyleStyleVernacularWordsItalic{eefin} and \textstyleStyleVernacularWordsItalic{eenin} , and more than one distal demonstrative is needed for contrastive purposes:

\ea%x1749
\label{ex:x1749}
\gll Ema  \textstyleEmphasizedVernacularWords{eenin}  fikera=ke  aw-o-k,  aria  \textstyleEmphasizedVernacularWords{eefin} \\
      \\
\glt
\z

mountain  that3  kunai.grass=CF  burn-PA-3s,  alright  that2  

fikera=ke  me  aw-o-k.

kunai.grass=CF  not  burn-PA-3s

`The kunai grass on that mountain (far away, invisible) burned, but the grass on this/that one (somewhat closer) did not burn.'

There is no number distinction in demonstratives. When they modify a [+human] noun, plurality is shown in the person/number marking of the verb and optionally by an additional personal pronoun.

\ea%x635
\label{ex:x635}
\gll (\textstyleEmphasizedVernacularWords{Wi})  takira  \textstyleEmphasizedVernacularWords{fain=ke}  niir-e-mik. \\
      \\
\glt
\z

3p.UNM  boy  this=CF  play-PA-1/3p

`It was these boys that played.'

With [-human] nouns, a quantifier in the \textstyleAcronymallcaps{NP} may be used (\stepcounter{nx}{\thenx}), or distributive suffix on the verb (\stepcounter{nx}{\thenx}) to indicate plurality, or the number may be left unspecified (\stepcounter{nx}{\thenx}).

\ea%x636
\label{ex:x636}
\gll Mera  \textstyleEmphasizedVernacularWords{arow  nain}  aaw-e-m. \\
      \\
\glt
\z

fish  three  that1  get-PA-1s

`I caught those three fish.'

\ea%x637
\label{ex:x637}
\gll Mera  \textstyleEmphasizedVernacularWords{nain}  aaw-\textstyleEmphasizedVernacularWords{omak}-e-m. \\
      \\
\glt
\z

fish  that1  get-DISTR.PL-PA-1s

`I caught those (many) fish.'

\ea%x638
\label{ex:x638}
\gll Amina  \textstyleEmphasizedVernacularWords{fain}  p-ekap-e-mik. \\
      \\
\glt
\z

pot  this  Bpx-come-PA-1/3p

`We brought this pot / these pots.'

Besides the exophoric (text-external) deictic use described above, another common function for demonstratives cross-linguistically is endophoric, or text-internal anaphoric and cataphoric reference. The proximity in the case of demonstratives relates to the participants in the text, rather than the speech situation \citep[278]{Lyons1968}. 

Mauwake follows the typical pattern: the neutral distal demonstrative \textstyleStyleVernacularWordsItalic{nain} `that' is anaphoric: it only refers to the text preceding it, as in (\stepcounter{nx}{\thenx}), where the example sentence comes after the description of fishing with a fish trap. The proximal \textstyleStyleVernacularWordsItalic{fain} `this'is cataphoric, referring to the text following it (\stepcounter{nx}{\thenx}). The other two demonstratives, \textstyleStyleVernacularWordsItalic{eefin} and \textstyleStyleVernacularWordsItalic{eenin}, are not used for text-internal reference at all.

\ea%x639
\label{ex:x639}
\gll \textstyleEmphasizedVernacularWords{Nain}  soo  era=ke. \\
      \\
\glt
\z

that1  fish.trap  way=CF

`That is the way (to catch fish) with a fish trap.'

\ea%x640
\label{ex:x640}
\gll Mua  arow  \textstyleEmphasizedVernacularWords{fain}:  Kuuten,  Dogimaw,  aria  Olas  {\dots} \\
      \\
\glt
\z

man  three  this:  Kuten,  Dogimaw,  alright  Olas

`These three men: Kuuten, Dogimaw and Olas {\dots}'

The demonstrative \textstyleStyleVernacularWordsItalic{nain} `that' marks given/established information, and often has a similar function to a definite article (cf. Dryer 2007c:154). It has an important pragmatic function of marking topic continuity in Mauwake. A continuing [+human] topic, especially the main participant, is usually marked only by person/number inflection on the verb, whereas a minor participant or a [-human] established topic uses \textstyleAcronymallcaps{NP}s modified by \textstyleStyleVernacularWordsItalic{nain}.

Still another function for the demonstrative \textstyleStyleVernacularWordsItalic{nain} `that' is that of a nominaliser of otherwise finite verbal clauses (\sectref{sec:5.7.2}). A nominalized clause of this type may be a relative clause (\stepcounter{nx}{\thenx}) (\sectref{sec:8.3.1}), a complement clause (\stepcounter{nx}{\thenx}) (\sectref{sec:8.3.2}) or a temporal subordinate clause (\stepcounter{nx}{\thenx}) (\sectref{sec:8.3.3.1}).\footnote{All these clauses have a function that is consistent with the core meaning of `givenness' \citep{Haiman1978} or presupposition \citep{Reesink1987}.}

\ea%x687
\label{ex:x687}
\gll [Merena  ifa  keraw-a-k  \textstyleEmphasizedVernacularWords{nain}]\textsubscript{RC}  puuk-a-mik. \\
      \\
\glt
\z

leg  snake  bite-PA-3s  that1  cut-PA-1/3p

`They cut the leg that the snake had bitten.'

\ea%x689
\label{ex:x689}
\gll [Mukuna  kerer-e-k  \textstyleEmphasizedVernacularWords{nain}]\textsubscript{CC}  i  me  paayar-e-mik. \\
      \\
\glt
\z

fire  start-PA-3s  that1  1p.UNM  not  understand-PA-1/3p

`We didn't know that a fire had started.'

\ea%x688
\label{ex:x688}
\gll [Goron-ep  ora-i-ya  \textstyleEmphasizedVernacularWords{nain},]  maa  muutitik  \\
      \\
\glt
\z

fall-SS.SEQ  descend-Np-PR.3s  that1  thing  all.kinds  

iiwawun  lalat-i-ya.

altogether  sweep-Np-3s

`When it goes down, it sweeps everything with it.'

The same demonstrative is also used as a strong adversative `but' (\stepcounter{nx}{\thenx}) (\sectref{sec:8.1.3}). In that function it is placed clause-initially rather than clause-finally.

\ea%x690
\label{ex:x690}
\gll Wiawi  eliw  naak-e-k,  \textstyleEmphasizedVernacularWords{nain}  me  ikiw-o-k. \\
      \\
\glt
\z

3s/p.father  all.right  say-PA-3s  that1  not  go-PA-3s

`He said yes (lit: all right) to his father, but didn't go.'

\subsection{Deictic locative adverbs} 
\hypertarget{RefHeading19861935131865}{}
The undebatable locative adverbs in Mauwake are all deictic (\sectref{sec:3.9.1.1}). For each of the four deictic roots there are two corresponding locative adverbs. The first set contains the deictic root and the locative suffix -\textstyleStyleVernacularWordsItalic{an}. The homorganic vowels in the root and affix have merged into one. The second set is suffixed with the contrastive focus clitic -(\textstyleStyleVernacularWordsItalic{e})\textstyleStyleVernacularWordsItalic{ke}. When the clitic is added, the deictic adverb is in focus, but not necessarily contrastive. The morphophonological change that has taken place in the root is unusual: the vowel /a/ has assimilated with the initial /e/ of the contrastive focus clitic.  

Adv  Adv + CF

fa-an{{\textgreater}fan  fa-eke{{\textgreater}}feeke } `here' (close to speaker, visible)  

na-an{{\textgreater}}nan  na-eke{{\textgreater}}neeke  `there' (away from the speaker; generic) 

eef-an  eef-eke  `here'  (rather close, usually visible) 

een-an  een-eke  `there' (far away, usually not visible) 

The difference in the usage between the neutral and focused member of each pair is that the first is \textstyleEmphasizedWords{\textsc{only}} used with realis-type verb forms, i.e. past (\stepcounter{nx}{\thenx}), (\stepcounter{nx}{\thenx}) and present tense (\stepcounter{nx}{\thenx}), whereas the second one is \textstyleEmphasizedWords{\textsc{mainly}} used with future (\stepcounter{nx}{\thenx}), imperative (\stepcounter{nx}{\thenx}), and counterfactual (\stepcounter{nx}{\thenx}), i.e. irrealis-type forms. Yet Mauwake does not differentiate between realis and irrealis in verbs, and a possible explanation here is that only locative adverbs that are in focus can make it into a future, imperative or counterfactual clause, whereas past or present clauses are less restrictive and use either focal or non-focal form. 

\ea%x463
\label{ex:x463}
\gll Owowa=pa  \textstyleEmphasizedVernacularWords{fan}  ik-emkun  aasa  maneka  ekap-o-k. \\
      \\
\glt
\z

village=LOC  here  be-1s/p.DS  canoe  big  come-PA-3s

`As I was here in the village the big ship came.'

\ea%x464
\label{ex:x464}
\gll Eliw  \textstyleEmphasizedVernacularWords{feeke}  soop-i-yen. \\
      \\
\glt
\z

well  here.CF  bury-Np-FU.1p

`We can bury him \textstyleEmphasizedWords{\textsc{here}}.'

\ea%x1213
\label{ex:x1213}
\gll Yo  fura  belemuta  \textstyleEmphasizedVernacularWords{eefan}  piipu-a-m. \\
      \\
\glt
\z

1s.UNM  knife  small  there2  leave-PA-1s

`I left the small knife (somewhere) here.'

\ea%x465
\label{ex:x465}
\gll Ni  koora  epa  \textstyleEmphasizedVernacularWords{eefeke}  ku-eka. \\
      \\
\glt
\z

2p.UNM  house  place  there2.CF  build-IMP.2p

`Build a/the house \textstyleEmphasizedWords{\textsc{over here}} in this place.'

\ea%x1214
\label{ex:x1214}
\gll Wi  aakisa  fain  manina  \textstyleEmphasizedVernacularWords{eenan}  on-i-mik. \\
      \\
\glt
\z

3p.UNM  now  this  garden  there3  make-Np-PR.1/3p

`Nowadays they make the garden(s) there (far away).'

\ea%x1573
\label{ex:x1573}
\gll Ni  \textstyleEmphasizedVernacularWords{eeneke}  ikiw-ep  momor  naap  niir-eka. \\
      \\
\glt
\z

2p  there3.CF  go-SS.SEQ  foolish  thus  play-IMP.2p

`Go \textstyleEmphasizedWords{\textsc{there}} (out of my sight) and play your foolish game.'

\ea%x466
\label{ex:x466}
\gll \textstyleEmphasizedVernacularWords{Neeke}  ik-ek-a-k=na  iwer(a)  ififa=ke  ifakim-ek-a-k. \\
      \\
\glt
\z

there1.CF  be-CNTF-PA-3s=TP  coconut  dry=CF  kill-CNTF-PA-3s

`If he had been \textstyleEmphasizedWords{\textsc{there}} a (falling) dry coconut would have killed him.'

\ea%x1197
\label{ex:x1197}
\gll Soo  nainiw  muf-owa  pun  naap,  aana=pa  \textstyleEmphasizedVernacularWords{neeke}  muf-i-mik. \\
      \\
\glt
\z

trap  again  pull-NMZ  too  thus  rattan=LOC  there1.CF  pull-Np-PR.1/3p

`Pulling the trap again is also like that, we/they pull it \textstyleEmphasizedWords{\textsc{there}} by the rattan.'

\ea%x1198
\label{ex:x1198}
\gll Malol=pa  \textstyleEmphasizedVernacularWords{neeke}  nainiw  suuw-urup-i-ya. \\
      \\
\glt
\z

open.sea  there1.CF  again  push-ascend-Np-PR.3s

`\textstyleEmphasizedWords{\textsc{There}} from the open sea it (= tsunami wave) again pushes up (to the coast).'

In the following examples \textstyleStyleVernacularWordsItalic{neeke} and\textstyleStyleVernacularWordsItalic{ feeke} are used with past or present tense verbs and indicate a temporary rather than permanent location, but this is probably secondary, or related, to the adverbs being focal: there is less need to focus on a permanent location than on a temporary one. Note that in these clauses it is possible to have two constituents with contrastive focus marking.

\ea%x1146
\label{ex:x1146}
\gll Miiw(a)  aasa  fa-ow(a)  mua=ke  \textstyleEmphasizedVernacularWords{neeke}  wia  aaw-o-k. \\
      \\
\glt
\z

land  canoe  drive-NMZ  man=CF  there1.CF  3p.ACC  take-PA-3s

`\textstyleEmphasizedWords{\textsc{There}} the truck driver picked them up.'

\ea%x1199
\label{ex:x1199}
\gll Or-op  \textstyleEmphasizedVernacularWords{neeke}  ika-iwkin  kokom-ar-e-k. \\
      \\
\glt
\z

descend-SS.SEQ  there1.CF  be-2/3p.DS  dark-INCH-PA-3s

`When they had gone down and were \textstyleEmphasizedWords{\textsc{there}} it became dark.'

\ea%x1200
\label{ex:x1200}
\gll Nainiw  mukuna  mamaiya  \textstyleEmphasizedVernacularWords{neeke}  ikiw-o-k. \\
      \\
\glt
\z

again  fire  close  there1.CF  go-PA-3s

`Again he went \textstyleEmphasizedWords{\textsc{there}} close to the fire.'

The following example is a comment from a man after he sees Japanese bombers in the sky:

\ea%x1572
\label{ex:x1572}
\gll Fa,  Yaapan=ke  \textstyleEmphasizedVernacularWords{feeke}  ik-e-mik! \\
      \\
\glt
\z

INTJ  Japan=CF  here.CF  be-PA-1/3p

`Damn, the Japanese are \textstyleEmphasizedWords{\textsc{here}}!'

\subsection{Deictic manner adverbs}
\hypertarget{RefHeading19881935131865}{}
The four deictic manner adverbs are based on the deictic roots, but their derivation is less regular than that of either the demonstratives or the deictic locatives, due to the restriction that a geminate vowel is only possible in an initial syllable. Again, the proximate and especially the distal-1 adverbs are common but the others are very infrequent.

feenap  `like this'  proximate  

naap  `like that, thus'  distal-1

eefenap  `like that (further away)'  distal-2

eenap  `like that (far away)'  distal-3

\ea%x701
\label{ex:x701}
\gll Ikiw-e-mik=na  \textstyleEmphasizedVernacularWords{feenap}  ma-em-ik-e-mik  {\dots} \\
      \\
\glt
\z

go-PA-1/3p=TP  like.this  say-SS.SIM-be-PA-1/3p

`They went and (unexpectedly) kept saying like this {\dots}'

\ea%x702
\label{ex:x702}
\gll \textstyleEmphasizedVernacularWords{Naap}  maak-iwkin  \textstyleEmphasizedVernacularWords{naap}  ik-ua. \\
      \\
\glt
\z

thus  tell-2/3.DS  thus  be-PA.3s

`They told him like that, and he was like that.'

In (\stepcounter{nx}{\thenx}) there is a long temporal distance between the hearing and the recounting of the story, which is apparently reflected in the choice of the adverbial.

\ea%x1857
\label{ex:x1857}
\gll Iiriw  auwa-ke  ma-iwkin  \textstyleEmphasizedVernacularWords{eefenap}  miim-a-m. \\
      \\
\glt
\z

earlier  1s/p.father=CF  say-2/3p.DS  thus2  hear-PA-1s

`The fathers spoke (about this) long ago and I heard it like that.'

In (\stepcounter{nx}{\thenx}) there is both some temporal and a considerable locative distance between the original time and place of the quote and that of the rest of the example: 

\ea%x1858
\label{ex:x1858}
\gll ``Mua  nain  opora=pa  wu-ami  ifakim-e,''  \textstyleEmphasizedVernacularWords{eenap} \\
      \\
\glt
\z

man  that1  talk=LOC  put-SS.SIM  kill-IMP.2s  thus3

efa  maak-e-mik.

1s.ACC  tell-PA-1/3p

` ``Accuse (lit: put to talk) that man and kill him,'' they told me like that.'

Location verbs (\sectref{sec:3.8.4.4.3}) are also based on the deictic roots, but directional verbs (\sectref{sec:3.8.4.4.5}), which also participate in the spatial deictic system in Mauwake, have different roots. 

\section{Question words and indefinites}
\hypertarget{RefHeading19901935131865}{}
Most of the indefinites in Mauwake are also question words, hence the treatment of both in the same subsection.

\subsection{Question words}
\hypertarget{RefHeading19921935131865}{}
The question words are here grouped together because of their shared semantic features and their function and position in content questions, although on the basis of their syntactic function on clause level some are pronouns, others adjectives or adverbs. 

The majority of the question words have an initial morpheme \textstyleStyleVernacularWordsItalic{ka}-, which indicates a question and is below in the derivations given the gloss `what', although it is unrelated to the question word \textstyleStyleVernacularWordsItalic{mauwa} `what'. The morphemes that make up the question words in the list below are given in parentheses when they can be reasonably clearly established.  

The question words are:

iikamin  `when?'\footnote{\textstyleFootnoteBaseChar{\textit{Ama kamin}} `sun how much' is used when time measured by clock is inquired; \textstyleFootnoteBaseChar{\textit{iikamin}} is less specific.}   ({{\textless}}iir-kamin  `time-how.much')

kaakew(e)  `of what place?'

kaan  `where'  ({{\textless}}ka-an  `what=LOC')

kaaneke  `where?'   ({{\textless}}ka-an-eke   `what=LOC=CF\textstyleAcronymallcaps{'})

kaanin  `which (of two)?'   ({{\textless}}ka-an-in   `what=LOC-\textstyleAcronymallcaps{GIVEN'})

kain  `which?'   ({{\textless}}ka-in   `what-\textstyleAcronymallcaps{GIVEN'})

kamin  `how many?', `how much?'

kamenap  `how?', `what {\dots} like?' ({{\textless}kamin-naap `how.much-thus')}

mauwa  `what?'

moram  `why?'

naarew(e)  `who?'

kamenion  `(or) what/how?'  ({{\textless}}kamin-yon `how.much-perhaps')

naap-i  `like that?'

Both the words translated with\textit{} `which', \textstyleStyleVernacularWordsItalic{kain} and \textstyleStyleVernacularWordsItalic{kaanin}, have the suffix -\textstyleStyleVernacularWordsItalic{in} marking givenness. They are both morphologically and semantically related to the demonstratives \textstyleStyleVernacularWordsxiiptItalic{fain} `this' and \textstyleStyleVernacularWordsxiiptItalic{nain} `that' (\sectref{sec:3.6.2}). 

\textstyleStyleVernacularWordsItalic{Kaan} `where' is formed by the question root \textstyleStyleVernacularWordsItalic{ka}- and the same locative affix -\textstyleStyleVernacularWordsItalic{an} that is used in the deictic locative adverbs \textstyleStyleVernacularWordsItalic{fan} `here' and \textstyleStyleVernacularWordsItalic{nan} `there' (\sectref{sec:3.6.3}). The derivation with the contrastive focus marker \textstyleStyleVernacularWordsItalic{-(e)}\textstyleStyleVernacularWordsItalic{ke} is more frequently used than the non-focused form, possibly because the the other two most frequent question words, \textstyleStyleVernacularWordsItalic{mauwa} `what' and \textstyleStyleVernacularWordsItalic{naarewe} `who', so often take the contrastive focus clitic. 

\ea%x1852
\label{ex:x1852}
\gll Mua  nain  unuf-ami  ma-i-kuan,  {\textquotedbl}Mua  nain  \textstyleEmphasizedVernacularWords{kaan}  ik-ua?{\textquotedbl} \\
      \\
\glt
\z

man  that1  call-SS.SIM  say-Np-FU.3p  man  that1  where  be-PA.3s

`They call the man's name and say, ``Where is that man?'' '

\ea%x1854
\label{ex:x1854}
\gll Oo  Sarak,  no  \textstyleEmphasizedVernacularWords{kaan=eke}  ik-ok  kerer-e-n  a? \\
      \\
\glt
\z

INTJ  Sarak  2s.UNM  where=CF  be-SS  arrive-PA-2s  INTJ

`Oh Sarak, where have you been (lit: where were you and arrived)?'

\textstyleStyleVernacularWordsItalic{Kaanin} `which of two' also shares the locative morpheme \textstyleStyleVernacularWordsItalic{an}- with \textstyleStyleVernacularWordsItalic{kaan}- `where' as well as \textstyleStyleVernacularWordsItalic{fan} `here' and \textstyleStyleVernacularWordsItalic{nan} `there', although in its present meaning it is not a locative question.

There is also a morphological relationship between \textstyleStyleVernacularWordsItalic{kamenap} `how/ what{\dots}like?' and \textstyleStyleVernacularWordsItalic{kamin} `how many/much?' and the deictic adverb \textstyleStyleVernacularWordsItalic{naap} `thus', but synchronically their semantic relationship is opaque. \textstyleStyleVernacularWordsItalic{Kamenion} `or what? / how is it?' has obviously developed from \textstyleStyleVernacularWordsxiiptItalic{kamin} `how many/much?' and the modal clitic \textstyleStyleVernacularWordsItalic{\nobreakdash-yon} `perhaps' (\sectref{sec:3.9.3}), but again, the relationship is not transparent any more.

The question words, except for \textstyleStyleVernacularWordsItalic{kamenion} and \textstyleStyleVernacularWordsItalic{naap-i}, occupy the same syntactic position and clausal function as the corresponding non-interrogative element would have:

\ea%x520
\label{ex:x520}
\gll Mua  nain  \textstyleEmphasizedVernacularWords{iikamin}  ekap-o-k? \\
      \\
\glt
\z

man  that  when  come-PA-3s

`When did that/the man come?'

\ea%x647
\label{ex:x647}
\gll Mua  nain  \textstyleEmphasizedVernacularWords{unan}  ekap-o-k. \\
      \\
\glt
\z

man  that  yesterday  come-PA-3s

`That/the man came yesterday.'

\ea%x521
\label{ex:x521}
\gll Maa  \textstyleEmphasizedVernacularWords{mauwa}  en-e-n? \\
      \\
\glt
\z

thing/food  what  eat-PA-2s

`What did you eat?'

\ea%x648
\label{ex:x648}
\gll Maa  \textstyleEmphasizedVernacularWords{oposia}  en-e-m. \\
      \\
\glt
\z

thing/food  meat  eat-PA-1s

`I ate meat.'

Neither number nor case is marked on the interrogative words themselves. If either marking is required, it is done through personal pronouns, but for [+human] \textstyleAcronymallcaps{NP}s only.

\ea%x522
\label{ex:x522}
\gll Mua  \textstyleEmphasizedVernacularWords{naarew  wia}  uruf-a-n? \\
      \\
\glt
\z

man  who  3p.ACC  see-PA-2s

`Whom (pl) did you see?'

\ea%x523
\label{ex:x523}
\gll \textstyleEmphasizedVernacularWords{Naarew  wiar}  aaw-o-k? \\
      \\
\glt
\z

who  3.DAT  get-PA-3s

`Who did he get it from?'

When an interrogative word is used as a subject, the contrastive focus marker \nobreakdash-\textstyleStyleVernacularWordsItalic{ke} is added. This is natural since it is the question word that is the focal element in questions. 

\ea%x524
\label{ex:x524}
\gll \textstyleEmphasizedVernacularWords{Mauwa}\textstyleEmphasizedVernacularWords{=ke}  nefa  aruf-a-k? \\
      \\
\glt
\z

what=CF  2s.ACC  hit-PA-3s

`What hit you?'

\ea%x525
\label{ex:x525}
\gll Mua  \textstyleEmphasizedVernacularWords{kain=ke}  nomak-e-k? \\
      \\
\glt
\z

man  which=CF  win-PA-3s

`Which man won?'

\ea%x526
\label{ex:x526}
\gll Masin  \textstyleEmphasizedVernacularWords{kaanin=ke}  samor-ar-e-k? \\
      \\
\glt
\z

engine  which.of.2=CF  bad-INCH-PA-3s

`Which engine (of the two) broke?'

\textstyleStyleVernacularWordsItalic{Naarew}\textstyleStyleVernacularWordsItalic{(e)} `who?' is only used for [+human] referents. When the contrastive focus maker -\textstyleStyleVernacularWordsItalic{ke} is suffixed to the question word, the last syllable is normally deleted. \textstyleStyleVernacularWordsItalic{Mauwa} `what', on the other hand, is used almost solely for [-human] nouns. The only natural expression with \textstyleStyleVernacularWordsItalic{mauwa} referring to humans that I have encountered is of the type (\stepcounter{nx}{\thenx}). When a person's name is inquired, either \textstyleStyleVernacularWordsItalic{naarewe} (\stepcounter{nx}{\thenx}) or \textstyleStyleVernacularWordsItalic{kamenap} (\stepcounter{nx}{\thenx}) is used rather than \textstyleStyleVernacularWordsItalic{ma}\textstyleStyleVernacularWordsItalic{uwa}.

\ea%x649
\label{ex:x649}
\gll Emeria  nain  no/nena  \textstyleEmphasizedVernacularWords{mauwa=ke}? \\
      \\
\glt
\z

woman  that  1s.UNM/1s.GEN  what=CF

`What (relation) of yours is that woman?'

\ea%x1855
\label{ex:x1855}
\gll O  unuma  \textstyleEmphasizedVernacularWords{naare=ke}? \\
      \\
\glt
\z

3s.UNM  name  who=CF

`What is his/her name?'

\textstyleStyleVernacularWordsItalic{Kaanin} `which of two?' is specified for number (\stepcounter{nx}{\thenx}), but \textstyleStyleVernacularWordsItalic{kain} `which?' is not.

\ea%x691
\label{ex:x691}
\gll No  \textstyleEmphasizedVernacularWords{kain}  kookal-i-n? \\
      \\
\glt
\z

2s.UNM  which  like-Np-PR.2s

`Which one (of two or many) do you like?'

The locative question word \textstyleStyleVernacularWordsItalic{kaan}(\textstyleStyleVernacularWordsItalic{eke}) `where' is often used as a phrase by itself (\stepcounter{nx}{\thenx}), (\stepcounter{nx}{\thenx}). but it is also employed as a modifier of a locative noun phrase rather than \textstyleStyleVernacularWordsItalic{kain} or \textstyleStyleVernacularWordsItalic{kaanin}: 

\ea%x1853
\label{ex:x1853}
\gll [Epa  ara  \textstyleEmphasizedVernacularWords{kaan=eke}]\textsubscript{NP}  ikiw-e-mik? \\
      \\
\glt
\z

place  section  where=CF  go-PA-1/3p

`What/which area did they go to?'

\textstyleStyleVernacularWordsItalic{Kamenap} is a question word both for manner `how?' (\stepcounter{nx}{\thenx}) and for adjectives `what {\dots} like?'. In the latter sense it usually modifies the noun \textstyleStyleVernacularWordsItalic{sira} `custom, kind' (\stepcounter{nx}{\thenx}).

\ea%x527
\label{ex:x527}
\gll No  \textstyleEmphasizedVernacularWords{kamenap}  ik-o-n? \\
      \\
\glt
\z

2s.UNM  how  be-PA-2s

`How are/were you?'

\ea%x528
\label{ex:x528}
\gll O  koora  \textstyleEmphasizedVernacularWords{sira}  \textstyleEmphasizedVernacularWords{kamenap}  ku-a-k? \\
      \\
\glt
\z

3s.UNM  house  custom/kind  what.like  build-PA-3s

`What kind of house did he build?'

It is also used with the noun \textstyleStyleVernacularWordsItalic{unuma} `name' when the name of someone or something is inquired:

\ea%x650
\label{ex:x650}
\gll O  unuma  \textstyleEmphasizedVernacularWords{kamenap}? \\
      \\
\glt
\z

3s.UNM  name  what.like?

`What is his/her name?'

\ea%x651
\label{ex:x651}
\gll Nomokowa  fain  unuma  \textstyleEmphasizedVernacularWords{kamenap}? \\
      \\
\glt
\z

tree  this   name  what.like

`What is the name of this tree?'

In example (\stepcounter{nx}{\thenx}) \textstyleStyleVernacularWordsItalic{kamenap} is interchangeable with \textstyleStyleVernacularWordsItalic{naare}(\textstyleStyleVernacularWordsItalic{we})-\textstyleStyleVernacularWordsItalic{ke} `who', but in (\stepcounter{nx}{\thenx}) it is not interchangeable with \textstyleStyleVernacularWordsItalic{mauwa}\textstyleStyleVernacularWordsItalic{-ke} `what'.

The interrogative \textstyleStyleVernacularWordsItalic{kamenion} forms a clause by itself and only occurs after the question clitic -\textstyleStyleVernacularWordsItalic{i} and/or the connective \textstyleStyleVernacularWordsItalic{e} `or'. 

\ea%x529
\label{ex:x529}
\gll Maa  en-owa=ko  p-ekap-e-mik=i  \textstyleEmphasizedVernacularWords{kamenion}? \\
      \\
\glt
\z

thing  eat-NMZ=NF  Bpx-come-PA-1/3p=QM  or.what

`Did they bring food, or what (happened)?'

The question word \textstyleStyleVernacularWordsItalic{naap-i} `like that?' is different from the other question words. It is formed by adding the question marker -\textstyleStyleVernacularWordsItalic{i}  to the demonstrative \textstyleStyleVernacularWordsItalic{naap} `thus, like that', and it occurs by itself or sentence-finally after a statement, which often follows another question. It is mainly used in argumentation. 

\ea%x1194
\label{ex:x1194}
\gll Siiwa  arow  ikiw-eya  maa  en-owa  perek-i-mik.  \textstyleEmphasizedVernacularWords{Naap}\textstyleEmphasizedVernacularWords{=i}? \\
      \\
\glt
\z

moon  three  go-2/3s.DS  thing  eat-NMZ  harvest-Np-1/3p  thus=QM

`After three months we'll harvest the food, right?'

\ea%x1195
\label{ex:x1195}
\gll Feenap  eliw  ma-i-yen=i?  Sira  nain  eliw  marew,  \\
      \\
\glt
\z

like.this  well  say-Np-FU.1p=QM  custom  that1  good  none

\textstyleEmphasizedVernacularWords{naap}\textstyleEmphasizedVernacularWords{=i}?

thus=QM

`Should we say that that custom is not good -- is that what you are saying?'

Questions are discussed in \sectref{sec:7.2}, which has more examples as well.

\subsection{Indefinites}
\hypertarget{RefHeading19941935131865}{}
Indefinites are sometimes classified as pronouns, although they often are not very pronoun-like; sometimes they are grouped together with quantifiers \citep[81]{HakulinenEtAl1979}%Karlsson
. By definition they lack definiteness which is typical of other pronouns (Quirk et al. 1985:376). Also their status as \textstyleAcronymallcaps{NP} substitutes is questionable.

In Mauwake, the indefinites behave syntactically very much like quantifiers. The position of the indefinites in the \textstyleAcronymallcaps{NP} is after the adjective phrase and immediately preceding the demonstrative. They rarely co-occur with a quantifier phrase, but if they do, they follow the \textstyleAcronymallcaps{QP}.

The number of indefinites in Mauwake is very small. The last four in the list are actually question words (\sectref{sec:3.7.1}) that also function as indefinites:

oko    `a certain, (an)other'

papako  `some, other'

naarew(e)  `whoever, someone, one'

mauwa  `whatever, something'

kain  `whichever'

kaanin  `whichever (of two)'

\ea%x641
\label{ex:x641}
\gll Iiriw  muuka  \textstyleEmphasizedVernacularWords{oko}  wiawi  onak  urera  \\
      \\
\glt
\z

long.ago  boy  other  3s/p.father  3s/p.mother  evening  

maa  uup-e-mik.

food  cook-PA-1/3p

`Long ago, a certain boy's father and mother cooked food.'

\ea%x642
\label{ex:x642}
\gll Ne  wia,  \textstyleEmphasizedVernacularWords{papako=ke}  ma-e-mik,  {\dots} \\
      \\
\glt
\z

ADD  no,  some/other=CF  say-PA-1/3p

`But no, some/others said, {\dots}'

The indefinite \textstyleStyleVernacularWordsItalic{oko} `a certain, (an)other' also has the meaning `otherwise' when it introduces an apprehensive clause (8.1.6).

\ea%x741
\label{ex:x741}
\gll Gurun-owa  epasia=pa  miim-am-ika-i-kuan,  \textstyleEmphasizedVernacularWords{oko}  mua  \\
      \\
\glt
\z

rumble-NMZ  far=LOC  hear-SS.SIM-be-Np-FU.3p  other  man

papako  maa  ik-em-ik-owa  nain  kawus  wiar

some  thing/food  roast-SS.SIM-be-NMZ  that1  smoke  3.DAT

uruf-i-kuan.

see-Np-FU.3p

`They (villagers) keep listening to the rumble from far away, otherwise/lest they (pilots) see the smoke from some men's/people's food-roasting fire.'

Those question words (\sectref{sec:3.7.1}) that may function as indefinites behave similarly to question words as \textstyleAcronymallcaps{NP} constituents, but on the sentence level there are differences between them. The interrogatives occur either in a simple interrogative sentence or occasionally in a medial clause (\stepcounter{nx}{\thenx}). The indefinites can occur in a medial clause (\stepcounter{nx}{\thenx}), but they are more common in subordinate clauses, especially relative clauses (\stepcounter{nx}{\thenx}). 

\ea%x643
\label{ex:x643}
\gll \textstyleEmphasizedVernacularWords{Naarew}  wia  far-ep  ekap-o-n? \\
      \\
\glt
\z

who  3p.ACC  call-SS.SEQ  come-PA-2s

`Who did you call, and then came?'\footnote{A more natural translation into English would be `Who did you call before you came?', but it would hide the fact that medial clauses are coordinate.}

\ea%x644
\label{ex:x644}
\gll Masin  \textstyleEmphasizedVernacularWords{kaanin=ke}  samor-ar-eya  oko  fain=ke  \\
      \\
\glt
\z

engine  which.of.2=CF  bad-INCH-2/3s.DS  other  this=CF  

asip-i-non.

help-Np-FU.3s

`Whichever engine breaks down, this other one will help/substitute.'\footnote{With question intonation it would mean: `Which engine\textsubscript{i} will this other one\textsubscript{j} help, if it\textsubscript{i} breaks down?'}

\ea%x645
\label{ex:x645}
\gll Prais  aaw-ep  [\textstyleEmphasizedVernacularWords{uf-owa}  \textstyleEmphasizedVernacularWords{kain=ke}  nomak-e-k  nain]\textsubscript{RC}   \\
      \\
\glt
\z

prize  take-SS.SEQ  dance-NMZ  which=CF  win-PA-3s  that1  

wi-e-mik.

give.them-PA-1/3p

`They took the prize and, whichever dance won, they gave it (the prize) to them (the dancers).'

The indefinite \textstyleStyleVernacularWordsItalic{mauwa} `what' is also used as a generic substitute for any [\nobreakdash-human] \textstyleAcronymallcaps{NP} that is left unmentioned because the name of the particular thing is not known or is temporarily forgotten, like \textstyleForeignWords{whatchamacallit} in English.

\ea%x646
\label{ex:x646}
\gll Mua  nain  \textstyleEmphasizedVernacularWords{mauwa}  nain  akim-a-k=na  weetak,  \textstyleEmphasizedVernacularWords{mauwa } nain  \\
      \\
\glt
\z

man  that1  what  that1  try-PA-3s=TP  no,  what  that1  

me  or-o-k.

not  descend-PA-3s

`The man tried the thing (press button), but the thing (lift) didn't go down.'

The locative question word \textstyleStyleVernacularWordsxiiptItalic{kaaneke} is also used as an indefinite locative adverb:

\ea%x1869
\label{ex:x1869}
\gll No  \textstyleEmphasizedVernacularWords{kaaneke}  ikiw-i-nan=na,  yos  pun  nook-i-nen. \\
      \\
\glt
\z

2s.UNM  where.CF  go-Np-FU.2s=TP  1s.FC  too  follow.you-Np-FU.1s

`Wherever you go, I will follow you.'

\section{Verbs}
\hypertarget{RefHeading19961935131865}{}
\subsection{General discussion}
\hypertarget{RefHeading19981935131865}{}
\subsubsection[Definition]{Definition}
\hypertarget{RefHeading20001935131865}{}
The verb category can be defined morphologically, syntactically, semantically and pragmatically. Of these, the first criterion is the most critical in Mauwake and covers the whole class; the others are less definitive, but help define a \textstyleEmphasizedWords{\textsc{prototypical}} verb.

According to the \textstyleEmphasizedWords{\textsc{morphological}}, or structural, criterion, a verb is a word that can be inflected for tense as well as the person and number of the subject. The derivational suffix categories of verbaliser, distributive and benefactive are not as useful in defining the class of verbs, as these can be used in the nominalized forms of verbs as well. \citet[190]{Anderson1985b} also adds aspect and mood into inherent verbal inflections, but in Mauwake aspect is coded syntactically (see verbal groups in \sectref{sec:3.8.5.1}), and modal categories either morphologically, syntactically or lexically.

\textstyleEmphasizedWords{\textsc{Syntactically}} a verb functions as the nucleus of a predication independently or as part of a verbal cluster (\sectref{sec:3.8.5}). Since single verbs and verbal clusters have such similar functions, the latter are described in the morphology chapter immediately after the verbs, and not in the chapter on phrase. Also, the term \textstyleEmphasizedWords{\textsc{verb}} is often used below as a generic term to cover both a single verb and a verbal cluster, unless specifically the verbal cluster is meant. The verb is the last element in a pragmatically neutral clause.

The verbal predicate is the only obligatory element in an intransitive clause. A transitive clause does require an object, but even it can often consist of a verb only, as the third person singular accusative  pronoun, used for object, is zero (\stepcounter{nx}{\thenx}). The directional verbs (\sectref{sec:3.8.4.4.5}) often co-occur with a goal, but when it is left implied the verb can be the only element (\stepcounter{nx}{\thenx}). In a verbless clause the predicate is a noun, adjective, possessive pronoun or adverb. 

\ea%x177
\label{ex:x177}
\gll \textstyleEmphasizedVernacularWords{Aaw-e-m.} \\
      \\
\glt
\z

get-PA-1s

`I got it.'

\ea%x178
\label{ex:x178}
\gll \textstyleEmphasizedVernacularWords{Urup-e-mik}.  \\
      \\
\glt
\z

go/come.up-PA-1/3p 

`We went/came up.' 

The predicate verb selects the arguments in a predication. This argument selection can be used as an important basis for the division into different verb classes (\sectref{sec:3.8.4}).

\textstyleEmphasizedWords{\textsc{Semantically}}, according to \textstyleBibliogBaseChar{Giv\'on} (1984:64), a prototypical verb encodes ``\textstyleBibliogCitationAAAstyleChar{less time-stable experiences, primarily transitory states, events and actions}''. In Mauwake this lack of time-stability feature shows in the strong tendency to use inchoative verbs (\stepcounter{nx}{\thenx}) (\sectref{sec:3.8.2.2.2}) instead of adjectives to describe non-permanent states. 

\ea%x179
\label{ex:x179}
\gll \textstyleEmphasizedVernacularWords{supuk-ar-e}\textstyleEmphasizedVernacularWords{-k}   vs.  \textstyleEmphasizedVernacularWords{supuka}  `wet' \\
      \\
\glt
\z

wet-INCH-PA-3s  

`(it) is wet' (lit: `has become wet') 

But it is also possible to express less prototypical, time-stable states and events with verbs. In \textstyleBibliogBaseChar{Frawley}'s (1992:66) words\textstyleBibliogCitationAAAstyleChar{,} ``\textstyleBibliogCitationAAAstyleChar{verbs {\dots} require temporal fixing''}, when compared with the ``\textstyleBibliogCitationAAAstyleChar{relative atemporality}'' of an entity. So the \textstyleEmphasizedWords{\textsc{relative temporality}} is the main defining factor for verbs, regardless of the time-stability.

\textstyleBibliogBaseChar{Hopper and Thompson} (1984:726) add a \textstyleEmphasizedWords{\textsc{discourse}} perspective to the definition of verbs by suggesting that ``\textstyleBibliogCitationAAAstyleChar{verbs which do not report discourse events fail to show the range of oppositions characteristic of those which do}'', and are therefore less prototypical. According to them, categoriality is only weakly associated with the root forms, and the discourse use determines how clearly the verbhood manifests itself (ibid. 747). Theirs is an important viewpoint for the study of language in general and of those languages in particular that have plenty of root forms that can be used for different word classes. But for Mauwake I assume the existence of rather discrete categories of noun and verb, which the root forms belong to, rather than just having \textstyleBibliogCitationAAAstyleChar{``a propensity or predisposition to become} \textstyleAcronymallcaps{\textup{N}}\textstyleBibliogCitationAAAstyleChar{'s or} \textstyleAcronymallcaps{\textup{V}}\textstyleBibliogCitationAAAstyleChar{'s''} (ibid. 747). The number of roots that can be used across categories without special derivational suffixes is small.

\subsubsection[General characteristics of verbs in Mauwake]{General characteristics of verbs in Mauwake}
\hypertarget{RefHeading20021935131865}{}
Mauwake is a strongly verb-oriented language, and often a verb is the only element in the clause. In running text, there are roughly three words per clause, so approximately one word in three is a verb, as most of the clauses are verbal clauses.

The verb morphology is agglutinative; this shows mainly in the structure of the verbs. Suffixing is the basic strategy, but a few prefixes are used as well. Reduplication is of the prefixing type, with few exceptions. 

Although the verb morphology in Mauwake is quite extensive, for a Papuan language it is not very complex, and the patterns are quite transparent.  The verb morphology marks features of the event itself: tense, mood, sequentiality vs. simultaneity of actions, but also features related to the participants in the clause: subject, beneficiary, and distributive indicating the number of \textstyleAcronymallcaps{S}, \textstyleAcronymallcaps{O} or \textstyleAcronymallcaps{REC}. Aspect is expressed through verbal groups (\sectref{sec:3.8.5.1.1}).

To enlarge its verb inventory, Mauwake uses serial verbs (\sectref{sec:3.8.5.1.2}) or adjunct\footnote{Adjunct is here used in the sense of ``a secondary element in a construction [,~which] may be removed without the structural identity of the rest of the construction being affected'' \citep[9]{Crystal1997}.} plus verb constructions (\sectref{sec:3.8.5.2}). The serial verbs are mostly formed by a productive process, whereas the adjunct plus verb constructions tend to be lexicalized forms. 

Some verbs have roots that are very similar to nouns. Especially in Austronesian languages the question arises whether these roots are originally nouns, verbs, or unspecified as to the grammatical category (\textstyleBibliogBaseChar{Bugenhagen 1995}:162-5). This question for Mauwake is discussed in the section on verb derivation (\sectref{sec:3.8.2}).

Mauwake has no passive voice.  The subject demotion strategy is described in \sectref{sec:3.8.4.3.3}. 

There is a distinction in Mauwake between medial (\sectref{sec:3.8.3.5}) and final verbs (\sectref{sec:3.8.3.4}).\footnote{Sometimes they are also called dependent and independent verbs (e.g. \textstyleBibliogBaseChar{Foley 1986}:11).} This distinction is very important on both sentence and discourse levels. 

The verbs can be divided into two conjugation classes based on the past tense suffix vowel. Semantically these classes are arbitrary; the division is made on the basis of morphophonology and is discussed in \sectref{sec:2.3.3.3}.  But the classification done according to transitivity (\sectref{sec:3.8.4.2}) and that based on semantic characteristics (\sectref{sec:3.8.4.4}) are more interesting grammatically and reveal more of the nature of the language. 

\subsubsection[Verb structure]{Verb structure}
\hypertarget{RefHeading20041935131865}{}
A verb consists of a root optionally preceded by a derivational prefix and followed by various derivational and inflectional suffixes, as shown in the diagram below (\figref{fig:1}). Only tense and person/number suffixes are obligatory in a finite verb in the \textstyleEmphasizedWords{\textsc{indicative}} mood. The obligatory elements are bolded in the diagrams.

{\bfseries
Deriv.        Derivation              Inflection}

[Warning: Draw object ignored][Warning: Draw object ignored][Warning: Draw object ignored]

Prefix -- \textbf{ROOT} -- INCH -- CAUS -- DISTR -- BEN -- BNFY -- CNTF -- \textbf{TNS -- PRS/NUM}

[Warning: Draw object ignored]

{\bfseries
S  T  E  M}


\begin{figure}
\caption{Verb derivation and finite inflection (indicative)}
\label{fig:1}
\end{figure}

\ea%x180
\label{ex:x180}
\gll Soomia  wia  \textstyleEmphasizedVernacularWords{amap-ep-om-i-ya.} \\
      \\
\glt
\z

spoon  3p.ACC  Bpx-go-BEN-Np-PR.3s

`He takes spoons to them.'

\ea%x181
\label{ex:x181}
\gll Iwera  pun  wiar  \textstyleEmphasizedVernacularWords{aw-omak-e-k}. \\
      \\
\glt
\z

coconut  too  3.DAT  burn-DISTR/PL-PA-3s

`Many of his coconut palms burned too.'

\ea%x182
\label{ex:x182}
\gll Lawiliw  akena  \textstyleEmphasizedVernacularWords{um-ek-a-m.} \\
      \\
\glt
\z

nearly  very  die-CNTF-PA-1s  

`I very nearly died.'

The \textstyleEmphasizedWords{\textsc{imperative}} verb structure is understandably different in that it cannot take counterfactual, tense or indicative person/number suffixes. Instead, an imperative person/number suffix needs to be attached as the final suffix of the verb.

Prefix -- \textbf{ROOT} -- \textstyleAcronymallcaps{INCH} -- \textstyleAcronymallcaps{CAUS} -- \textstyleAcronymallcaps{DISTR} -- \textstyleAcronymallcaps{BEN} -- \textstyleAcronymallcaps{BNFY} -- \textbf{IMP.PRS/NUM}

\ea%x183
\label{ex:x183}
\gll Ni  \textstyleEmphasizedVernacularWords{ekap-omak-eka.} \\
      \\
\glt
\z

2p.UNM  come-DISTR/PL-IMP.2p

`Come!'  (said to several people together) 

\ea%x184
\label{ex:x184}
\gll Muuka  \textstyleEmphasizedVernacularWords{arim-ow-e.} \\
      \\
\glt
\z

son  grow-CAUS-IMP.2s  

`Bring up the boy.' 

\textstyleEmphasizedWords{\textsc{Medial}} verbs likewise can have only the medial suffix after the derivational suffixes, if there are any. The medial suffix distinguishes between sequentiality (\stepcounter{nx}{\thenx}) and simultaneity (\stepcounter{nx}{\thenx}) of the actions when the subject stays the same; with a different subject (\stepcounter{nx}{\thenx}) the actions are understood to be sequential, and simultaneity needs to be marked through continuous aspect form (\sectref{sec:3.8.5.1.1.2}).

{\bfseries
[Warning: Draw object ignored]                                    SEQ}

{\bfseries
                              SS}

{\bfseries
[Warning: Draw object ignored][Warning: Draw object ignored]                                   } 

[Warning: Draw object ignored]Prefix - \textbf{Root} - INCH - CAUS - DISTR - BEN - BNFY -             \textbf{SIM}

                                \textbf{DS}


\begin{figure}
\caption{Medial verb inflection}
\label{fig:2}
\end{figure}

\ea%x185
\label{ex:x185}
\gll Oposia  \textstyleEmphasizedVernacularWords{pu-puuk-ap}  uup-e-mik.  \\
      \\
\glt
\z

meat  RDP-cut-SS.SEQ  cook-PA-1/3p  

`They cut the meat in many pieces and cooked it.'

\ea%x186
\label{ex:x186}
\gll Ewar=ke  \textstyleEmphasizedVernacularWords{wuun-ow-ami}  epia  faker-a-k,  mukuna.  \\
      \\
\glt
\z

wind=CF  blow-CAUS-SS.SIM  firewood  raise-PA-3s  fire  

`The wind blew and raised the fire(wood),  the fire.' 

\ea%x187
\label{ex:x187}
\gll \textstyleEmphasizedVernacularWords{Kees-om-a-ya}  en-ek. \\
      \\
\glt
\z

spit-BEN-BNFY2-2/3s.DS  eat-PA-3s  

`He spat/regurgitated it for her and she ate.'

\subsection{Verb derivatives}
\hypertarget{RefHeading20061935131865}{}
This section deals with derivational processes in which the end result is always a verb. Verbs can be derived from other word classes through two category-\textstyleEmphasizedWords{\textsc{changing}} strategies. In category-\textstyleEmphasizedWords{\textsc{maintaining} }derivations affixes are added to the verb root to change the semantics of the root. Among the latter, the semantic changes can be considerable especially in cases where the valence changes, whereas in category-changing derivations the semantic difference is not always so great (\textstyleBibliogBaseChar{Bybee 1985}:83).  

\subsubsection[Derivation vs. inflection]{Derivation vs. inflection}
\hypertarget{RefHeading20081935131865}{}
According to \textstyleBibliogBaseChar{Bybee} (1985:81), ``\textstyleBibliogCitationAAAstyleChar{an inflectional morpheme {\dots} is a bound non-root morpheme whose appearance in a particular position is compulsory}.'' It is ``\textstyleBibliogCitationAAAstyleChar{required by syntax}''. In contrast, derivational affixes are non-obligatory (\textstyleBibliogBaseChar{Greenberg 1954}:191). In Mauwake, all the derivations are non-obligatory. Of the inflections, the beneficiary suffix and the counterfactual suffix as such are not required by syntax like the tense and person marking, but they have an interdependence relationship with other suffixes: the beneficiary suffix has to occur in a past tense or imperative form of a verb that also has the benefactive suffix, and the counterfactual suffix restricts the tense marking to past tense.

In Mauwake verb structure the derivational suffixes always precede the inflectional ones. This agrees with one of \textstyleBibliogBaseChar{Greenberg}'s universals: ``\textstyleBibliogCitationAAAstyleChar{If the derivation and inflection follow the root {\dots} the derivation is always between the root and the inflection}'' (1966:93).

Inflectional suffixes in Mauwake form paradigms, even if in some cases the paradigms only have two members.

The greater syntagmatic freedom of derivational affixes (\textstyleBibliogBaseChar{Malkiel 1978}:128-9) is shown in Mauwake by the fact that a verb with any of the derivations can be nominalized with the nominalizing suffix \nobreakdash-\textstyleStyleVernacularWordsItalic{owa}, whereas one with inflectional suffixes cannot.\footnote{It is possible to nominalize whole \textit{clauses} where the main verb has inflectional suffixes, by adding the demonstrative \textstyleFootnoteBaseChar{\textit{nain}} `that' after the clause, but this strategy is not available for individual verbs (5.7.2).} This ability of verb stems with derivational suffixes to be nominalized is the main distinction between derivation and inflection in Mauwake. 

A special feature in Mauwake is the dividing point between the derivational and inflectional suffixes: the benefactive suffix is derivational, whereas the beneficiary suffix is inflectional. The latter can only be present when there are other verbal suffixes following, whereas the former can also be followed by a nominaliser suffix. Other differences between the two suffixes are described below (\sectref{sec:3.8.2.3.3}), (\sectref{sec:3.8.3.1}). In the following section, the derivational suffixes are introduced in the order that they occur following the verb root; the prefixes are discussed last.

There is clear iconicity in the linear ordering of the derivational suffixes: the closer the suffix is to the root, the more profound the change it effects on it. The verbalizing suffixes change the word class; the causative adds an argument; the distributive pluralizes an argument, and the benefactive adds a peripheral.





\begin{tabular}{llllll}
\mytoprule


PREFIX & ROOT & VERBALISER & CAUSATIVE & DISTRIBUTIVE & BENEFACTIVE\\
p\textbf{-} &  & \textbf{-}{\O} & \textbf{-}ow & \textbf{-}omak & \textbf{-}om\\
amap\textbf{-} &  & \textbf{-}ar &  & \textbf{-}urum & \\
aap\textbf{-} &  &  &  &  & \\
\textstyleAcronymallcaps{RDP} &  &  &  &  & \\
\mybottomrule
\end{tabular}



\begin{table}
\caption{Verbal derivation}
\label{tab:10}
\end{table}

\subsubsection[Category-changing derivation: verb formation]{Category-changing derivation: verb formation}
\hypertarget{RefHeading20101935131865}{}
There are two strategies in Mauwake whereby words from other word classes can be changed into verbs.  Zero verb formation is less productive than the inchoative. Also the meanings of the verbs resulting from zero verb formation are in some cases more lexicalized, or less transparent, than the meanings of the verbs formed with the inchoative suffix. Often roots can be used for both the strategies, but not always: words like \textstyleStyleVernacularWordsItalic{amisa} `\textstyleFreeTranslationChar{knowledge'} and \textstyleStyleVernacularWordsItalic{ewur} `\textstyleFreeTranslationChar{quickly, fast'} only allow the inchoative suffix.

\paragraph[Zero verb formation]{Zero verb formation}
\hypertarget{RefHeading20121935131865}{}
Mauwake has a number of verbs where the root is originally a noun, an adjective or an adverb, and the verb is formed without any overt morpheme to mark the category change. Hopper and \citet[745]{Thompson1984} remark that ``\textstyleBibliogCitationAAAstyleChar{languages often possess rather elaborate morphology whose sole function is to convert verbal roots into} \textstyleAcronymallcaps{N}\textstyleBibliogCitationAAAstyleChar{'s, but no morphology whose sole function is to convert nominal roots into} \textstyleAcronymallcaps{V}\textstyleBibliogCitationAAAstyleChar{'s}''. Zero verb formation is here understood, not as adding a zero morpheme, but as a lexical process (following Payne 1997:224). A noun, adjective or adverb is used as a root for the verb, and in this process it becomes a true verb, unlike nominalizations which are nouns but retain a lot of their verbal nature as well \citep[747]{HopperEtAl1984}%Thompson
.\footnote{\textstyleBibliogBaseChar{Hopper and Thompson} (1984) propose that a word root is unspecified as to the grammatical category, and that discourse function assigns categoriality. This fits many Austronesian languages in which there are plenty of words where only the non-root morphology, or else syntactic behaviour, shows what class the word belongs to. Mauwake has relatively few forms like this and it is reasonable to assign words to specific word classes even without reference to discourse function.}

The resulting verb is usually transitive, with a few exceptions. The final vowel of a noun or an adjective, usually  /a/, is deleted before the verbal inflection. 

From nouns:

akuwa  `knot'  akuw-  `knot/bind/tie with a knot'

anima  `blade'  anim-  `sharpen'

eneka  `tooth, flame'  enek-  `light (a fire)'

ilen  `sign'  ilen-  `recognise sign'

nanar  `story'  nanar-  `tell a story'

From adjectives:

dubila  `smooth'  dubil-  `smoothen'

enuma  `new'  enum-  `renew'

iiwa  `short'  iiw-  `shrink' 

itita  `soft'  itit-  `smash'

kaken  `straight'  kaken-  `straighten'

maneka  `big'  manek-  `enlarge'

momora  `fool'  momor-  `confuse'

samora  `bad'  samor-  `destroy'

siina  `tight'  siin-  \textstyleTableEntryChar{`diminish' (intr.)}\footnote{Another intransitive verb can be derived from \textstyleFootnoteBaseChar{\textit{siina}} with the inchoative suffix: \textstyleFootnoteBaseChar{\textit{siin-ar}}- `become tight/narrow'.}

From adverbs:

bilik  `mixed'  bilik-  `mix'

ikum  `illicitly'  ikum-  `speculate'

kerew  `strongly'  kerew-  `be angry at'

fan    `here'  fan-  `be/come here'

nan  `there'  nan-  `be/come here'

  \stepcounter{nx}{\thenx}x188)  Yo  aakisa  inasina  Rubaruba  \textstyleEmphasizedVernacularWords{nanar-i-yem}.\\
1s.UNM  now  spirit  Rubaruba  story-Np-PR.1s  

`Now I tell about spirit Rubaruba.'

  \stepcounter{nx}{\thenx}x189)  Aruf-ami  me  \textstyleEmphasizedVernacularWords{samor-eka}!\\
hit-SS.SIM  not  bad-IMP.2p  

`Don't hit/beat and destroy it.' 

Semantically the resulting verb is usually very close to the word that serves as the root, but in a few instances like (\stepcounter{nx}{\thenx}) the semantic link is not very strong.

\ea%x190
\label{ex:x190}
\gll Nefa  \textstyleEmphasizedVernacularWords{ikum-am-ika-iwkin}  nan  kerer-e-n. \\
      \\
\glt
\z

2s.ACC  illicitly-SS.SIM-be-2/3p.DS  there  appear-PA-2s  

`They were just speculating about you when you arrived.' 

\paragraph[Inchoative suffix ]{Inchoative suffix} 
\hypertarget{RefHeading20141935131865}{}
The second verb formation process in Mauwake takes a noun, adjective or adverb root and adds an inchoative suffix -\textstyleStyleVernacularWordsItalic{ar}  (\sectref{sec:2.3.3.4}) to form a new verb usually meaning `become n'.\footnote{The term `inchoative' is used for  derivation; `inceptive' for aspect, following  \textstyleBibliogBaseChar{Payne} (1997:95).} Although in the majority of the cases a word from one of the other word classes is made into a verb, the basic meaning is inchoative rather than verbalizing, as the same suffix can also be added to a few verbs. The suffix has been grammaticalized from the verb \textstyleStyleVernacularWordsItalic{ar}\textstyleEmphasizedVernacularWords{-} `become', `enter into a state', and there are a few cases where it is difficult to decide with certainty which one it is. The differences between the full verb and the suffix are listed below. 

The full resultative verb \textstyleStyleVernacularWordsItalic{ar}\textstyleEmphasizedVernacularWords{-} `become' is more common with nouns, and the meaning of the verb is transparent (\stepcounter{nx}{\thenx}). It is also used with numerals (\stepcounter{nx}{\thenx}). Both words retain their word stress.

\ea%x191
\label{ex:x191}
\gll Arim-emi  mu'a  \textstyleEmphasizedVernacularWords{ar-'e-k}. \\
      \\
\glt
\z

grow-SS.SIM  man  become-PA-3s  

`He grew up and became man/adult.'

\ea%x192
\label{ex:x192}
\gll Aruf-owa  e\textstyleEmphasizedVernacularWords{'}repam  \textstyleEmphasizedVernacularWords{ar-'e-m}. \\
      \\
\glt
\z

hit-NMZ  four  become-PA-1s  

`I hit it four times.' (Lit: `Hitting it I became four.')

The inchoative suffix \textstyleEmphasizedVernacularWords{\nobreakdash-}\textstyleStyleVernacularWordsItalic{ar}  can occur with nouns (\stepcounter{nx}{\thenx}), but is more common with adjectives (\stepcounter{nx}{\thenx}) and adverbs (\stepcounter{nx}{\thenx}), and can attach to a few verb roots too (\stepcounter{nx}{\thenx}). Since the result is one word it only has one word stress. 

\ea%x193
\label{ex:x193}
\gll Yiena  opaimika  me  baliwep  \textstyleEmphasizedVernacularWords{a'mis-ar-e-mik}. \\
      \\
\glt
\z

1p.GEN  talk  not  well  knowledge-INCH-PA-1/3p  

`They don't know our language well.' 

\ea%x194
\label{ex:x194}
\gll Miiw-aasa  \textstyleEmphasizedVernacularWords{sa'mor-ar-ek}.{\footnotemark} \\
      \\
\glt
\z

land-canoe  bad-INCH-PA-3s  

`The car broke.'

\footnotetext{\textstyleFootnoteBaseChar{\textit{Miiw-aasa samor-a-k}}  `He broke the car' would be a corresponding sentence with zero verbalization. }

\ea%x195
\label{ex:x195}
\gll Kau  pun  weeser-owa  \textstyleEmphasizedVernacularWords{e'wur-ar-ek.} \\
      \\
\glt
\z

cow  too  finish-NMZ  quickly-INCH-PA-3s  

`The beef finished quickly too.'

\ea%x196
\label{ex:x196}
\gll Mua  \textstyleEmphasizedVernacularWords{i'men-ar-ep}  opora  pun  \textstyleEmphasizedVernacularWords{i'men-ar-ek}. \\
      \\
\glt
\z

man  find-INCH-SS.SEQ  talk  too  find-INCH-PA-3s

`When man appeared, talk/language appeared too.'

Verbs derived from adjectives are often used rather than adjectives in the predicative position (\stepcounter{nx}{\thenx}), (\stepcounter{nx}{\thenx}).  And instead of a modifying adjective, a whole relative clause with a verb derived from an adjective may be used (\stepcounter{nx}{\thenx}). This happens especially when the property denoted by the adjective is not static. 

\ea%x82
\label{ex:x82}
\gll Sia  nain  senam  \textstyleEmphasizedVernacularWords{pin(a)-ar-e-k}.  \\
      \\
\glt
\z

netbag  that1  too.much  heavy-INCH-PA-3s

`The netbag is/was (lit: became) very heavy.' 

\ea%x1764
\label{ex:x1764}
\gll Muuka  nain  op-iya  \textstyleEmphasizedVernacularWords{dubil}\textstyleEmphasizedVernacularWords{(a)-}\textstyleEmphasizedVernacularWords{al}\textstyleEmphasizedVernacularWords{-}\textstyleEmphasizedVernacularWords{e}\textstyleEmphasizedVernacularWords{-}\textstyleEmphasizedVernacularWords{k}. \\
      \\
\glt
\z

boy  that1  hold-2/3s.DS  slippery-INCH-PA-3s

`When he\textsubscript{1} held the boy\textsubscript{2}, he\textsubscript{2} was slippery.'

\ea%x83
\label{ex:x83}
\gll [Konima  \textstyleEmphasizedVernacularWords{supuk(a)-ar-e-k  nain}]  yasuw-e. \\
      \\
\glt
\z

cloth  wet-INCH-PA-3s  that1  wash-IMP.2s

`Wash the wet cloth.' (Lit: `Wash the cloth that has become wet.')

The consonant /r/ in the suffix is lateralized into /l/ when the root has /l/ in the immediately preceding syllable.  Lateralization takes place arbitrarily in a few other cases as well (\stepcounter{nx}{\thenx}).

\ea%x197
\label{ex:x197}
\gll Yo  damol(a)-\textstyleEmphasizedVernacularWords{al}-e-m  oo.  \\
      \\
\glt
\z

1s.UNM  bad-INCH-PA-1s  oh  

`I feel terrible.' (Lit: `I'm destroyed/ruined.') 

\ea%x198
\label{ex:x198}
\gll Epa  dabel(a)-\textstyleEmphasizedVernacularWords{al}-ek. \\
      \\
\glt
\z

place  cold-INCH-PA-3s  

`It is cold.'

\ea%x199
\label{ex:x199}
\gll Opaimika  efa  masi(a)-\textstyleEmphasizedVernacularWords{al}-i-ya. \\
      \\
\glt
\z

mouth  1s.ACC  bitter-INCH-Np-PR.3s  

`It tastes bitter to me / in my mouth.'

If two preceding syllables contain /l/, the consonant in the verbaliser is not lateralized.\footnote{This rule is very tentative, as \textstyleFootnoteBaseChar{\textit{ilelar}}- is the only example found so far.}

\ea%x200
\label{ex:x200}
\gll Aasa  puuk-ap  ilel(a)-\textstyleEmphasizedVernacularWords{ar}-i-ya. \\
      \\
\glt
\z

canoe  cut-SS.SEQ  gouge-INCH-Np-PR.3s  

`He has cut the canoe (length from a tree) and is gouging/carving it' 

The suffix \textstyleStyleVernacularWordsItalic{ar}- often retains its original verbal meaning `become' when adjectives are made into verbs (\stepcounter{nx}{\thenx}), but when the other word classes are used as the root the original meaning tends to become more opaque or get lost (\stepcounter{nx}{\thenx}). 

\ea%x201
\label{ex:x201}
\gll \textstyleEmphasizedVernacularWords{Dubil}\textstyleEmphasizedVernacularWords{(a)-al-e-k.} \\
      \\
\glt
\z

slippery/smooth-INCH-PA-3s  

`It became slippery/smooth.' 

\ea%x202
\label{ex:x202}
\gll No  \textstyleEmphasizedVernacularWords{wadol}\textstyleEmphasizedVernacularWords{(a)-al-i-n.} \\
      \\
\glt
\z

2s.UNM  lie-INCH-Np-PR.2s  

`You are lying.' 

Most of the verbs formed with the inchoative suffix are intransitive, but some are active, transitive verbs:

\ea%x203
\label{ex:x203}
\gll Muuka  kuisow  \textstyleEmphasizedVernacularWords{muuk}\textstyleEmphasizedVernacularWords{(a)-ar-e-k}. \\
      \\
\glt
\z

son  one  son-INCH-PA-3s  

`She gave birth to one son.'

\ea%x204
\label{ex:x204}
\gll Epa  \textstyleEmphasizedVernacularWords{mores-ar-ep}  ikiw-o-k. \\
      \\
\glt
\z

place  like(ADV)-INCH-SS.SEQ  go-PA-3s  

`He made the place ready and went.'

The inchoative verb formation is also used with verb loans from other languages, especially Tok Pisin.\footnote{The Tok Pisin loans often originally come from English.} Both of the loan words below also have a vernacular synonym.

\ea%x487
\label{ex:x487}
\gll Muuka  wia  \textstyleEmphasizedVernacularWords{was-ar-e-mik}.\textstyleParagraphChari{   (from Tok Pisin} \textstyleForeignWords{was} \textstyleParagraphChari{`look after')} \\
      \\
\glt
\z

son  3p.ACC  look.after-INCH-PA-1/3p

`They were looking after the boys/children'

\ea%x488
\label{ex:x488}
\gll \textstyleEmphasizedVernacularWords{Nading-ar-ep}  uf-e-mik.  (from Mala \textstyleForeignWords{nading} `decoration') \\
      \\
\glt
\z

decoration-INCH-SS.SEQ  dance-PA-1/3p

`We decorated ourselves and danced.'

\subsubsection[Category-maintaining derivation: suffixes]{Category-maintaining derivation: suffixes}
\hypertarget{RefHeading20161935131865}{}
\paragraph[Causative suffix]{Causative suffix}
\hypertarget{RefHeading20181935131865}{}
The causative suffix \nobreakdash-\textstyleStyleVernacularWordsItalic{ow} (or \nobreakdash-\textstyleStyleVernacularWordsItalic{aw}) transitivizes an intransitive verb \citep[2]{Peterson2007}: the clause gets a new subject, and the subject of the intransitive verb becomes the direct object. Usually it adds a causative meaning `cause someone to do something', or `cause something to happen'. The object of a causative construction has no control, or only minimal control, over the action or event indicated by the verb.

In many verbs there is free variation between \nobreakdash-\textstyleStyleVernacularWordsItalic{ow} and \nobreakdash-\textstyleStyleVernacularWordsItalic{aw}. Some verbs seem to prefer one or the other, but there is no clear pattern. There is also some dialectal and possibly age-based variation depending on the speaker. \nobreakdash-\textstyleStyleVernacularWordsItalic{ow}  is taken here as the basic form, since it is the more common of the two, and because in ``double causatives'' it is always used at least as the first one.  

CAUSATIVE    FROM:  

arim-ow-  `bring up / raise'  arim-  `grow'

in-aw-  `put to bed'  in-  `lie down'

bagiwir-ow  `cause to be angry'  bagiwir-  `be angry'

iimar-ow-  `make sg. stand up'  iimar-  `stand up'

imenar-ow-  `create/cause to appear'  imenar-  `appear'

waki-ow-aw-  `cause to stumble'  waki-  `stumble'

ook-ow-  `place alongside'  ook-  `follow'

Sometimes the causative suffix occurs reduplicated as a ``double causative'', but these still add only one argument. Many of the short directional verbs (\sectref{sec:3.8.4.4.5}) take a double causative instead of a single one. 

\ea%x205
\label{ex:x205}
\gll Eewua  ir\textstyleEmphasizedVernacularWords{-ow-aw-}ap  osaiwa  ar-e-k. \\
      \\
\glt
\z

wing  climb-CAUS-CAUS-SS.SEQ  bird.of.paradise  become-PA-3s  

`She put the wing up (on herself) and became a bird of paradise.'

A single or double causative can be added to the intransitive verb \textstyleStyleVernacularWordsItalic{reen}- `(become) dry' with the result of two different meanings, but both of these still only add one more argument: \textstyleStyleVernacularWordsItalic{reenow}- `dry (something)', \textstyleStyleVernacularWordsItalic{reenowaw}- `smoke (something)'.

The only two transitive verbs that have been found to take the causative are \textstyleStyleVernacularWordsItalic{mik}- `spear/hit' and \textstyleStyleVernacularWordsItalic{op}- `hold/grab', with the causative forms \textstyleStyleVernacularWordsItalic{mik-ow-aw}- `join (the ends of two long items)' and \textstyleStyleVernacularWordsItalic{op-aw}- `accuse falsely'. 

Verbs that do \textstyleEmphasizedWords{\textsc{not}} have any causative meaning include the following:

aakun-ow-  `grumble (at)'  from:  aakun-  `speak'

baun-ow-  `bark (at)'  baun-  `bark'

kirir-ow-  `shout (about)'  kirir-  `shout'

op-aw-  `accuse falsely'  op-  `hold'

\ea%x991
\label{ex:x991}
\gll Mukuna  kuuf-ap  kirir-e-k. \\
      \\
\glt
\z

fire  see-SS.SEQ  shout-PA-3s

`She saw the fire and shouted.'

\ea%x489
\label{ex:x489}
\gll Yiok-ami  naap  \textstyleEmphasizedVernacularWords{yia}  \textstyleEmphasizedVernacularWords{kirir-ow-am-ik-ua.} \\
      \\
\glt
\z

follow.us-SS.SIM  thus  1p.ACC  shout-CAUS-SS.SIM-be-PA.3s

`She was following us and shouting about us like that.'

The causative as a valence-increasing device is discussed in \sectref{sec:3.8.4.3.1}. 

\paragraph[Distributive suffix]{Distributive suffix}
\hypertarget{RefHeading20201935131865}{}
A distributive suffix pluralizes one of the verbal arguments. There are two distributive suffixes: \nobreakdash-\textstyleStyleVernacularWordsItalic{urum}\textstyleEmphasizedVernacularWords{} `all' and \nobreakdash-\textstyleStyleVernacularWordsItalic{omak} `many'. They are fully productive in the whole verb class, as long as the semantics of the verb allows multiple arguments. 

The hierarchy of which argument the distributive applies to is as follows: if there is a recipient (\stepcounter{nx}{\thenx}) or beneficiary (\stepcounter{nx}{\thenx}), the distributive applies to that; if there is no recipient or beneficiary but an object, the distributive applies to the object (\stepcounter{nx}{\thenx}); and in case the clause has neither a recipient or beneficiary nor an object, the distributive applies to the subject (\stepcounter{nx}{\thenx}). Since transitive verbs need an object, the subject can be pluralized with the distributive only when the verb is intransitive.

{\bfseries
REC/BEN  {{\textgreater}}  O  {{\textgreater}}  S
}

\ea%x209
\label{ex:x209}
\gll Mua  teeria  opaimika  wia  sesek-\textstyleEmphasizedVernacularWords{omak}-e-mik. \\
      \\
\glt
\z

man  family  talk  3p.ACC  send-DISTR/PL-PA-1/3p  

`They sent word to (many members of) the man's family.'

\ea%x429
\label{ex:x429}
\gll Wiena  wiawi=ke  amia  wia  keraw-om-\textstyleEmphasizedVernacularWords{omak}-e-mik. \\
      \\
\glt
\z

3p.GEN  3s/p.father=CF  spear  3p.ACC  carve-BEN-DISTR/PL-PA-1/3p

`Their fathers carved spears for them (\textit{many} beneficiaries).'

\ea%x492
\label{ex:x492}
\gll Emeria  unowa  fain  nia  aaw-\textstyleEmphasizedVernacularWords{urum}-i-kuan. \\
      \\
\glt
\z

woman  many  this   2s.ACC  take-DISTR/A-Np-FU.3p

`They will take all of you women.'

\ea%x491
\label{ex:x491}
\gll Emeria  teeria  koka  ikiw-\textstyleEmphasizedVernacularWords{urum}-e-mik. \\
      \\
\glt
\z

woman  group  jungle  go-DISTR/A-PA-1/3p

`The whole group of women / all the women went to the jungle.'

In verbal groups (\sectref{sec:3.8.5.1}) the distributive suffix usually attaches to the last verb root, but it can occasionally also attach to the first root, i.e. the main verb  in  a verb+\textstyleAcronymallcaps{AUX} combination (\stepcounter{nx}{\thenx}).

\ea%x207
\label{ex:x207}
\gll Iinan  aasa  ikiw-emi  paran-em-\textstyleEmphasizedVernacularWords{mi-omak-e-k}. \\
      \\
\glt
\z

sky  canoe  go-SS.SIM  rumble-SS.SIM-go.around-DISTR/PL-PA-3s  

 `Many planes went rumbling around.'

\ea%x490
\label{ex:x490}
\gll Iinan  aasa  fan  or-om\textstyleEmphasizedVernacularWords{-ik-omak-eya}  {\dots} \\
      \\
\glt
\z

sky  canoe  here1  descend-SS.SIM-be-DISTR/PL-2/3s.DS

`When many planes were coming down here {\dots}'

\ea%x208
\label{ex:x208}
\gll Wi  ifa  saarik  \textstyleEmphasizedVernacularWords{in-urum-ep}-ik-e-mik.  \\
      \\
\glt
\z

3p.UNM  snake  like  sleep-DISTR/A-SS.SEQ-be-PA-1/3p  

`They all slept/lay like snakes.'

Both suffixes can be attached to the same verb but it is rare. In that case \textstyleEmphasizedVernacularWords{\nobreakdash-}\textstyleStyleVernacularWordsItalic{urum} precedes \textstyleEmphasizedVernacularWords{\nobreakdash-}\textstyleStyleVernacularWordsItalic{omak}.

\ea%x206
\label{ex:x206}
\gll Wia  ifakim\textstyleEmphasizedVernacularWords{-urum}-\textstyleEmphasizedVernacularWords{omak}-e-mik. \\
      \\
\glt
\z

3p.ACC  kill-DISTR/A-DISTR/PL-PA-1/3p  

`They killed each and every one of them.' (There were many of those killed.) 

\paragraph[Benefactive suffix ]{Benefactive suffix} 
\hypertarget{RefHeading20221935131865}{}
The benefactive suffix, indicating the fact that the action of the verb is done \textstyleEmphasizedWords{\textsc{for someone}}, for their benefit or detriment, is a borderline case among the derivations. It is the last one of the derivational suffixes, and the \textstyleEmphasizedWords{\textsc{beneficiary suffix}} (\sectref{sec:3.8.3.1}) following it and marking the person that the action is done for, is inflectional even if the two suffixes go together semantically. The position of the benefactive is not as stable as that of the other suffixes: it comes after the distributive when the beneficiary is first person singular (\stepcounter{nx}{\thenx}), (\stepcounter{nx}{\thenx}) but occurs preceding it with the other persons (\stepcounter{nx}{\thenx}), (\stepcounter{nx}{\thenx}). 

\ea%x210
\label{ex:x210}
\gll Mua  Maneka=ke  maa  maneka  on-omak-\textstyleEmphasizedVernacularWords{om}-e-k. \\
      \\
\glt
\z

Man  Big=CF  thing  big  do-DISTR/PL-BEN-BNFY1.PA\footnote{The vowel of the beneficiary suffix deletes the vowel of the past tense suffix. The relationship between the beneficiary suffix and the suffix following it is discussed in detail in \sectref{sec:3.8.3.1}, and the medial suffix forms are discussed in \sectref{sec:3.8.3.5}.} -3s  

`God did great things to/for me.'

\ea%x1925
\label{ex:x1925}
\gll Buk  aaw-omak-\textstyleEmphasizedVernacularWords{om}-e! \\
      \\
\glt
\z

book  get-DISTR/PL-BEN-BNFY1.IMP.2s

`Get the books for me!'

\ea%x211
\label{ex:x211}
\gll Buk  aaw-\textstyleEmphasizedVernacularWords{om}-omak-e! \\
      \\
\glt
\z

book  get-BEN-DISTR/PL-IMP.2s  

`Get the books for him!'

\ea%x1926
\label{ex:x1926}
\gll Wiena  wiawi=ke  amia  wia  keraw-\textstyleEmphasizedVernacularWords{om}-omak-e-mik. \\
      \\
\glt
\z

3p.GEN  3s/p.father=CF  bow  3p.ACC  carve-BEN-DISTR/PL-PA-1/3p  

`Their fathers carved bows for them.'

In verbal groups the benefactive suffix is usually attached to the finite verb or auxiliary (\stepcounter{nx}{\thenx}) but can occasionally occur on the non-finite root (\stepcounter{nx}{\thenx}) or even on both of the two (\stepcounter{nx}{\thenx}).

\ea%x212
\label{ex:x212}
\gll Iwera  wia  uruk-am-ik-\textstyleEmphasizedVernacularWords{om}-a-mik.  \\
      \\
\glt
\z

coconut  3p.ACC  drop-SS.SIM-be-BEN-BNFY2.PA-1/3p

`We kept dropping coconuts for them.' 

\ea%x213
\label{ex:x213}
\gll Maamuma  wia  p-ikiw-\textstyleEmphasizedVernacularWords{om}-ap-pu-ap  {\dots} \\
      \\
\glt
\z

money  3p.ACC  BPf-go-BEN-BNFY2.SS.SEQ-CMPL-SS.SEQ  

`Having taken money to them, {\dots} 

\ea%x214
\label{ex:x214}
\gll Moro  mua  wia  wu-\textstyleEmphasizedVernacularWords{om}-am-ik-\textstyleEmphasizedVernacularWords{om}-a-mik. \\
      \\
\glt
\z

Moro  man  3p.ACC  put-BEN-BNFY2.SS.SIM-be-BEN-BNFY2.PA-1/3p

`They put them (=carts) for the Moro men.' 

The benefactive form does not always mean that something happens for someone's\textstyleEmphasizedWords{} \textstyleEmphasizedWords{\textsc{benefit}}. The benefactive may be strengthened with the adverb \textstyleStyleVernacularWordsItalic{orawin} `for the benefit' (\stepcounter{nx}{\thenx}), which makes it unambiguous.

\ea%x215
\label{ex:x215}
\gll Iwera  \textstyleEmphasizedVernacularWords{orawin}  kais-\textstyleEmphasizedVernacularWords{om}-e-mik. \\
      \\
\glt
\z

Coconut  for.the.benefit  husk-BEN-BNFY1.PA-1/3p  

`They husked coconuts for me (for free).'

By using a suffix completely unrelated to the verb `give', Mauwake shows itself different from all of those reasonably closely related languages that have grammatical descriptions available. A serial verb construction involving the verb `give' is a very common way of expressing benefactive in Papuan languages (\textstyleBibliogBaseChar{Foley 1986}:141).  Waskia (\textstyleBibliogBaseChar{Ross and Paol 1978}:45) and Maia (\textstyleBibliogBaseChar{Hardin 2002}:125) employ this strategy, and in Usan it is behind one of the two strategies: the benefactive verb form has been grammaticalized from a serial verb with the verb `give' (\textstyleBibliogBaseChar{Reesink 1987}:110-1). The other strategy for Usan is to use a postposition with the appropriate noun phrase (ibid. 154). Bargam is similar to it (\textstyleBibliogBaseChar{Hepner 2002}:65-6, 99), but Amele utilizes an indirect object clitic attached to the verb to express the beneficiary as well as other semantic relations (\textstyleBibliogBaseChar{Roberts 1987}:167). 

\subsubsection[Derivational prefixes]{Derivational prefixes}
\hypertarget{RefHeading20241935131865}{}
Although Mauwake is very strongly a suffixing language, it makes use of some derivational prefixes as well. Reduplication is the most common among these. 

\paragraph[Reduplication]{Reduplication}
\hypertarget{RefHeading20261935131865}{}
The morphophonological aspect of reduplication was already described in \sectref{sec:2.3.3.2}. In \sectref{sec:6.4.2} reduplication is discussed as one of the many quantification strategies in Mauwake.

Reduplication in verbs is used in Mauwake to indicate continuity or iterativity of action and/or plurality of the resulting object. Mostly the reduplication is done only once, but especially motion verbs can have several identical reduplicative prefixes. 

In verbs of motion reduplication means continuity (\stepcounter{nx}{\thenx}), and the passing of time may be shown by the number of reduplications (\stepcounter{nx}{\thenx}).

\ea%x218
\label{ex:x218}
\gll \textstyleEmphasizedVernacularWords{Biri-birin-emi}  wia  akim-omak-e-mik.  \\
      \\
\glt
\z

RDP-fly-SS.SIM  3p.ACC  try-DISTR/PL-PA-1/3p  

`They were flying and teasing them.' 

\ea%x216
\label{ex:x216}
\gll Ne  \textstyleEmphasizedVernacularWords{oro-oro-oro}-oro-mi  \textstyleEmphasizedVernacularWords{oro-oro}-or-o-k,  \\
      \\
\glt
\z

and  RDP-RDP-RDP-descend-SS.SIM  RDP-RDP-descend-PA-3s

onoma.

horizon.

`And it went down and down and down all the way to the horizon.'

In other intransitive verbs reduplication indicates either iterative action (\stepcounter{nx}{\thenx}) or occasionally continuity (\stepcounter{nx}{\thenx}).

\ea%x217
\label{ex:x217}
\gll Nomokowa  \textstyleEmphasizedVernacularWords{ku-ku-ep}  or-om-ik-ua. \\
      \\
\glt
\z

tree  RDP-break-SS.SEQ  descend-SS.SIM-be-PA.3s

`The timber (in a bridge) kept breaking and falling down.' 

\ea%x692
\label{ex:x692}
\gll Epa  \textstyleEmphasizedVernacularWords{wii-wiim-ik-ua},  {\dots} \\
      \\
\glt
\z

place  RDP-dawn-be-PA.3s

`It was dawning, {\dots}'

Both of these meanings fit in well with \textstyleBibliogBaseChar{Moravcsik}'s description of the various meanings that reduplication in verbs can have (1978:319). In transitive verbs reduplication indicates iterative action as well as the plurality of an inanimate object (\stepcounter{nx}{\thenx}). The form is used especially when the action \textstyleEmphasizedWords{results} in a plural object (\stepcounter{nx}{\thenx}).

\ea%x219
\label{ex:x219}
\gll Iinan  aasa=ke  maifa  \textstyleEmphasizedVernacularWords{fu-fuurk-ikiw-o-k}. \\
      \\
\glt
\z

sky  canoe=CF  paper  RDP-throw-go-PA-3s  

`The plane went  throwing paper slips down' 

\ea%x220
\label{ex:x220}
\gll Oposia  nain  \textstyleEmphasizedVernacularWords{pu-puuk-ap}  uup-e-mik. \\
      \\
\glt
\z

meat  that1  RDP-cut-SS.SEQ  cook-PA-1/3p  

`We cut up the meat (into many pieces) and cooked it.'

Usan differs from Mauwake in that it does not use reduplication very much in verbs, and never in main clause final verbs (\textstyleBibliogBaseChar{Reesink 1987}:116). Also, when reduplication is used to indicate duration or repetition the whole verb word is reduplicated (ibid. 117). In Bargam reduplication occurs but is not very productive. In transitive verbs reduplication indicates plurality of objects, in intransitive verbs plurality of subjects (\textstyleBibliogBaseChar{Hepner 2002}:19). In Maia ``\textstyleBibliogCitationAAAstyleChar{verb roots may be partially or completely reduplicated. Verb reduplication broadly indicates an augmented action which may include a greater, more massive, more intensified or very often repetitive form of the action}'' (\textstyleBibliogBaseChar{Hardin 2002}:50).

\paragraph[Bring-prefixes]{Bring-prefixes}
\hypertarget{RefHeading20281935131865}{}
The prefixes in this group change the directional verbs (see \sectref{sec:3.8.4.4.5}) into transitive verbs with the meaning `bring' or `take'.  \textstyleStyleVernacularWordsItalic{p}\nobreakdash- is a neutral prefix and by far the most common one (\stepcounter{nx}{\thenx}), \textstyleStyleVernacularWordsItalic{amap}\nobreakdash- is used when something is brought out in the open, often with the meaning `bring forth'. Usually there is a clear goal, a person or a place, which may not be mentioned in the clause itself but occurs in an earlier one (\stepcounter{nx}{\thenx}), or is understood from the context (\stepcounter{nx}{\thenx}). If the goal is explicitly mentioned in the clause, the neutral prefix is used (\stepcounter{nx}{\thenx}), (\stepcounter{nx}{\thenx}).  The prefix \textstyleStyleVernacularWordsItalic{aap}\nobreakdash- (\stepcounter{nx}{\thenx}) is very rare and I have been unable to establish whether it really differs from \textstyleStyleVernacularWordsItalic{amap}\nobreakdash- or whether it is just a matter of idiolectal use.\footnote{The bring\nobreakdash-prefixes may have been grammaticalized from a medial verb construction involving the verb \textit{aaw}\nobreakdash- `take'. It is easy to see how \textit{aawep ekap}\nobreakdash- `take (and) come' could have developed into \textit{aapekap}\nobreakdash- `bring' and possibly also into \textit{pekap}-. Another possibility is that it is a result of a related process to that in Usan where the verb \textstyleFootnoteBaseChar{\textit{ba}} `take' has contracted into \textstyleFootnoteBaseChar{\textit{b}}\nobreakdash-, which has combined with verbs of motion and been lexicalized with the meaning of bringing or taking (\textstyleBibliogBaseChar{Reesink 1987}:144-5). The \textstyleFootnoteBaseChar{\textit{amap}}\nobreakdash-prefix may have its origin in the expression \textstyleFootnoteBaseChar{\textit{ama-pa}} `in the sun', which implies `in the open'. There is also a very slight possibility that the \textit{p}\nobreakdash-prefix might be an Austronesian loan, as p(V\textit{)\nobreakdash-} is a common causative or transitivizer prefix in Austronesian languages \citep[61]{Bugenhagen1995}. \textstyleFootnoteBaseChar{But all of this is just conjecture at this point.}}

\ea%x221
\label{ex:x221}
\gll Amina  aaw-ep  Liisa  ame  wia \\
      \\
\glt
\z

pot  take/get-SS.SEQ  Liisa  others  3p.ACC  

\textstyleEmphasizedVernacularWords{p-er}-om-a.

Bpx-go-BEN-BNFY2.IMP.2s

`Get the pot and take it to Liisa and the others.' 

\ea%x430
\label{ex:x430}
\gll Pita  pensil  wiar  or-op  ik-ua  nain  aaw-ep \\
      \\
\glt
\z

Pita  pencil  3.DAT  fall-SS.SEQ  be-PA.3s  that1  take-SS.SEQ

\textstyleEmphasizedVernacularWords{amap-ikiw}-om-aka.

Bpx-go-BEN-BNFY2.IMP.2p

`Take to Pita his pencil that has dropped.'

\ea%x222
\label{ex:x222}
\gll Wiipa  oko  \textstyleEmphasizedVernacularWords{amap-ora}-iwkin  ma-e-k  {\dots} \\
      \\
\glt
\z

daughter  other  Bpx-descend-2/3p.DS  say-PA-3s  

`When they took another daughter down (from the house out in the open), he said{\dots}' 

\ea%x223
\label{ex:x223}
\gll Ni  auwa  maa  \textstyleEmphasizedVernacularWords{p-urup}-om-aka.  \\
      \\
\glt
\z

2p.UNM  father  food  Bpx-ascend-BEN-BNFY2.IMP.2p  

`Take food (up) to father.'

\ea%x224
\label{ex:x224}
\gll Iwera  ir-ap  erup  op-ap  \textstyleEmphasizedVernacularWords{aap-or}-e. \\
      \\
\glt
\z

coconut  go.up-SS.SEQ  two  grab-SS.SEQ  Bpx-descend-IMP.2s  

`Climb the coconut palm, grab two coconuts and bring them down.' 

\subsection{Verb inflection} 
\hypertarget{RefHeading20301935131865}{}
The following table 11 shows those inflectional suffixes for the Mauwake verbs that change with the person and/or number of the subject. All of these are discussed in more detail below.





\begin{tabular}{llllllllll}
\mytoprule


\multicolumn{1}{l}{} & {\bfseries BNFY} & \multicolumn{1}{l}{{\bfseries CNTF}}

 & {\bfseries TENSE} & \multicolumn{2}{l}{{\bfseries PERS./ NUMBER}}

 & \multicolumn{1}{l}{{\bfseries IMPERAT.}}

 & \multicolumn{3}{l}{{\bfseries MEDIAL}}

\\
\multicolumn{1}{l}{} &  & \multicolumn{1}{l}{} & NON-PAST:

\textbf{-}i

PAST:

\textbf{-}E / -a & \multicolumn{1}{l}{PRES}

 & PAST & \multicolumn{1}{l}{} & \multicolumn{3}{l}{SAME SUBJECT}

\\
\multicolumn{1}{l}{1s}

 & -e & -ek &  & \multicolumn{1}{l}{\textbf{-}yem}

 & \textbf{-}m & \multicolumn{1}{l}{-u \textstyleTableEntryChar{(1d)}}

 & \multicolumn{3}{l}{SEQUENTIAL:  -ep/ap}

\\
\hhline{---~------}
2s &  &  &  & \multicolumn{2}{l}{\textbf{-}n}

 & \multicolumn{1}{l}{\textbf{-}e /-a}

 & \multicolumn{3}{l}{SIMULTANEOUS:  -emi/ami}

\\
\hhline{---~------}
\multicolumn{1}{l}{3s}

 & -a &  &  & \multicolumn{1}{l}{\textbf{-}ya}

 & \textbf{-}k & \multicolumn{1}{l}{\textbf{-}inok}

 & \multicolumn{3}{l}{DIFFERENT  SUBJECT}

\\
\hhline{---~------}
1p &  &  &  & \multicolumn{2}{l}{\textbf{-}mik}

 & \multicolumn{1}{l}{\textbf{-}ikua\textbf{} }

 & \multicolumn{2}{l}{\textbf{-}Vmkun  (s \& p)}

 & 1\\
\hhline{-~-~------}
2p &  &  &  & \multicolumn{2}{l}{\textbf{-}man}

 & \multicolumn{1}{l}{\textbf{-}eka\textbf{ /-}aka}

 & \multicolumn{1}{l}{\textbf{-}eya (s)}

 & \multicolumn{1}{l}{\textbf{-}iwkin (p)}

 & 2\\
\hhline{-~-~------}
3p &  &  &  & \multicolumn{2}{l}{\textbf{-}mik}

 & \textbf{-}uk &  &  & 3\\
\hhline{---~------}
\multicolumn{1}{l}{} &  &  &  & \multicolumn{2}{l}{FUTURE}

 & \multicolumn{4}{l}{}\\
\hhline{---~--~~~~}
1s &  &  &  & \multicolumn{2}{l}{\textbf{-}nen}

 &  &  &  & \\
\hhline{-~-~--~~~~}
2s &  &  &  & \multicolumn{2}{l}{\textbf{-}nan}

 &  &  &  & \\
\hhline{-~-~--~~~~}
3s &  &  &  & \multicolumn{2}{l}{\textbf{-}non}

 &  &  &  & \\
\hhline{-~-~--~~~~}
1p &  &  &  & \multicolumn{2}{l}{\textbf{-}yen}

 &  &  &  & \\
\hhline{-~----~~~~}
2p &  & \multicolumn{1}{l}{} & \textbf{-}o (Np) & \multicolumn{2}{l}{\textbf{-}wen}

 &  &  &  & \\
\hhline{-~----~~~~}
3p &  & \multicolumn{1}{l}{} &  & \multicolumn{2}{l}{\textbf{-}kuan}

 &  &  &  & \\
\hhline{-~----~~~~}

\mybottomrule
\end{tabular}



\begin{table}
\caption{Inflectional suffixes of Mauwake verbs}
\label{tab:11}
\end{table}

\subsubsection[Beneficiary]{Beneficiary}
\hypertarget{RefHeading20321935131865}{}
The beneficiary suffix indicates the person the action is done for. Its position is directly after the benefactive suffix, or after the distributive suffix in those few cases where the benefactive comes before the distributive (\sectref{sec:3.8.2.3.3}). It is inflectional rather than derivational because 1) when it is used, nominalization is blocked and 2) it has a paradigm for different persons, even if the paradigm only consists of two members. 

The only two forms for the beneficiary are  \textstyleEmphasizedVernacularWords{\nobreakdash-}\textstyleStyleVernacularWordsItalic{e}  for first or second person singular (\stepcounter{nx}{\thenx}) and \textstyleEmphasizedVernacularWords{\nobreakdash-}\textstyleStyleVernacularWordsItalic{a}  for all the other persons (\stepcounter{nx}{\thenx}). The context often provides more person distinctions, as the plural requires accusative pronouns to precede the verb to indicate the beneficiary, like third person plural in (\stepcounter{nx}{\thenx}). 

\ea%x225
\label{ex:x225}
\gll Wafur-om-\textstyleEmphasizedVernacularWords{e}!   \\
      \\
\glt
\z

throw-BEN-BNFY1.IMP.2s  

`Throw it to me!' 

\ea%x226
\label{ex:x226}
\gll Marasin  wu-om-\textstyleEmphasizedVernacularWords{a}-mik=na  weetak. \\
      \\
\glt
\z

medicine  put-BEN-BNFY2.PA-1/3p=TP  no  

`They put medicine on him but no (it didn't help).'

\ea%x227
\label{ex:x227}
\gll Na-iwkin  \textstyleEmphasizedVernacularWords{wia}  uf-om-\textstyleEmphasizedVernacularWords{a}-mik.  \\
      \\
\glt
\z

say-2/3p.DS  3p.ACC  dance-BEN-BNFY2.PA-1/3p  

`They said so and we danced for them.'

When the beneficiary suffix is followed by a vowel, a mid vowel is deleted adjacent to a low vowel (\stepcounter{nx}{\thenx}) and both a mid and a low vowel are deleted preceding a high vowel (\stepcounter{nx}{\thenx}). In the latter case the person distinction gets neutralized in the singular (\stepcounter{nx}{\thenx}), but not in the plural where the obligatory accusative pronouns maintain the distinction (\stepcounter{nx}{\thenx}). The examples (\stepcounter{nx}{\thenx})-(\stepcounter{nx}{\thenx}) below show how the beneficiary suffix affects the past tense suffix. In (\stepcounter{nx}{\thenx}) a sequence of two identical vowels is reduced to one vowel.

\ea%x228
\label{ex:x228}
\gll aaw-om-\textstyleEmphasizedVernacularWords{ak}-a-m    {{\textless}  aaw-om}-\textstyleEmphasizedVernacularWords{a-ek}-a-m  \\
      \\
\glt
\z

get-BEN-BNFY2.CNTF-PA-1s  

`I would have gotten it for him' 

\ea%x444
\label{ex:x444}
\gll aaw-om-\textstyleEmphasizedVernacularWords{i-non}      {{\textless}  aaw-om-}\textstyleEmphasizedVernacularWords{e-i-non}, aaw-om-\textstyleEmphasizedVernacularWords{a-i-non} \\
      \\
\glt
\z

get-BEN-BNFY.Np-FU.3s

`he will get it for me/you/him/her'

\ea%x1750
\label{ex:x1750}
\gll aaw-om-\textstyleEmphasizedVernacularWords{uk}        {{\textless} aaw-om-}\textstyleEmphasizedVernacularWords{e-uk}, aaw-om-\textstyleEmphasizedVernacularWords{a-uk} \\
      \\
\glt
\z

get-BEN-BNFY.IMP.3p

`let them get it for me/you/him/her'

\ea%x445
\label{ex:x445}
\gll Panewowa  maa  \textstyleEmphasizedVernacularWords{wia}  p-ikiw-om-\textstyleEmphasizedVernacularWords{uk}. \\
      \\
\glt
\z

old  food  3p.ACC  BPx-go-BEN-BNFY.IMP.3p

`Let them take food for the old people.'

\ea%x494
\label{ex:x494}
\gll Uf-\textstyleEmphasizedVernacularWords{o}-k. \\
      \\
\glt
\z

dance-PA-3s

`He danced.'

\ea%x495
\label{ex:x495}
\gll Uf-om-\textstyleEmphasizedVernacularWords{e}-k. \\
      \\
\glt
\z

dance-BEN-BNFY1.PA-3s

`He danced for me/you.'

\ea%x496
\label{ex:x496}
\gll Uf-om-\textstyleEmphasizedVernacularWords{a}-k. \\
      \\
\glt
\z

dance-BEN-BNFY2.PA-3s

`He danced for him.'

\subsubsection[Counterfactual]{Counterfactual}
\hypertarget{RefHeading20341935131865}{}
The counterfactual modality is the only modal distinction made in the verb morphology. It is an expression of the truth value of the statement: something could or would have happened, but did not, or something might be the case but for some reason is not. The counterfactual is marked by the suffix \nobreakdash-\textstyleStyleVernacularWordsItalic{ek} (\stepcounter{nx}{\thenx}) and is only used with the past tense suffix even if the verb refers to the present (\stepcounter{nx}{\thenx}) or future (\stepcounter{nx}{\thenx}) time\textstyleEmphasizedVernacularWords{\textmd{\textit{.}}} The counterfactual is used in both hypothetical and counterfactual conditional clauses (\sectref{sec:8.3.5}). 

\ea%x234
\label{ex:x234}
\gll Lawiliw  akena  waki-\textstyleEmphasizedVernacularWords{ek}-a-m. \\
      \\
\glt
\z

nearly  very  fall-CNTF-PA-1s  

`I very nearly fell.' 

\ea%x433
\label{ex:x433}
\gll Yena  aamun  aakisa  uruf-\textstyleEmphasizedVernacularWords{ek-a}-m=na  kemel-\textstyleEmphasizedVernacularWords{ek-a}-m. \\
      \\
\glt
\z

1s.GEN  yonger.brother  now  see-CNTF-PA-1s=TP  rejoice-CNTF-PA-1s

`If I saw my younger brother now, I would be happy.'

\ea%x434
\label{ex:x434}
\gll Morauta  fan  ik-\textstyleEmphasizedVernacularWords{ek-a}-k=na  uurika  ikiw-ep  \\
      \\
\glt
\z

Morauta  here  be-CNTF-PA-3s=TP  tomorrow  go-SS.SEQ  

maak-\textstyleEmphasizedVernacularWords{ek-a}-mik.

tell-CNTF-PA-1/3p

`If Morauta were here, tomorrow we would go and tell him.'

If there is a beneficiary suffix \textstyleStyleVernacularWordsItalic{-a} preceding the counterfactual, the vowel /e/ of the counterfactual suffix is deleted (\stepcounter{nx}{\thenx}):

\ea%x235
\label{ex:x235}
\gll Maifa  yia  aaw-om-\textstyleEmphasizedVernacularWords{ak}-a-k=na{\dots} \\
      \\
\glt
\z

paper  1p.ACC  get-BEN-BNFY2.CNTF-PA-3s=TP  

`If he had gotten tickets for us{\dots}'

{}

\subsubsection[Mood]{Mood}
\hypertarget{RefHeading20361935131865}{}
Mood in Mauwake is defined as a morphological category of the verb, relating to the pragmatic function of the sentence (cf. Palmer 1986:21). Mauwake has a mixed tense-mood system, where the indicative present, past and future, and the imperative are in contrast. 

The mood distinctions only show in the finite verb. Same-subject medial verbs take the interpretation of their mood from the following finite verb, but different-subject medial verbs may be independent of the final verb as to their mood.

\paragraph[Indicative]{Indicative}
\hypertarget{RefHeading20381935131865}{}
The indicative is the neutral, morphologically unmarked mood.  It is characterized by the tense distinctions between present, past and future, and the person/number distinctions of first, second, and third person in singular and plural. 

\ea%x693
\label{ex:x693}
\gll I  me  yia  damol-a-mik. \\
      \\
\glt
\z

1s.UNM  not  1s.ACC  harm-PA-1/3p

`They didn't harm us.'

\ea%x694
\label{ex:x694}
\gll Aria,  iperowa  opora  wiar  ook-i-yen. \\
      \\
\glt
\z

alright,  middle-aged  talk  3.DAT  follow-Np-FU.1p

`Alright, we'll follow the advice of the middle-aged men.'

\paragraph[Imperative]{Imperative}
\hypertarget{RefHeading20401935131865}{}
The term imperative is used for ``mands'' (\textstyleBibliogBaseChar{Lyons 1977}:745)\footnote{Lyons borrows the term from B.F. Skinner as a useful cover term, without subscribing to Skinner's behaviouristic position.} showing in the verbal morphology, regardless of person. In Mauwake the imperatives form a full paradigm (with the first person singular being replaced with the first person dual), and their syntactic behaviour is similar.  So there is no valid reason to divide them into different categories such as imperatives, jussives and hortatives, just because semantically giving orders to oneself differs from giving orders to an addressee or to a third person.\footnote{For a discussion on this question, see Palmer 1986:109-111.}

There are no tense distinctions in the imperative forms.  The initial (or only) vowel in the second person imperatives is usually /e/, but in very few cases it is /a/.\footnote{The only verbs found with -\textstyleFootnoteBaseChar{\textit{a}}  in the imperative are \textstyleFootnoteBaseChar{\textit{iw}}- `go', \textstyleFootnoteBaseChar{\textit{mik}}- `spear, hit', \textstyleFootnoteBaseChar{\textit{op}}- `hold' and \textstyleFootnoteBaseChar{\textit{pok}}- `sit'.}





\begin{tabular}{ll}
\mytoprule


PERSON/NUMBER & \\
-u & 1d\\
-e  (-a) & 2s\\
-inok & 3s\\
-ikua & 1p\\
-eka  (-aka) & 2p\\
-uk & 3p\\
\mybottomrule
\end{tabular}



\begin{table}
\caption{Imperative suffixes}
\label{tab:12}
\end{table}

\ea%x229
\label{ex:x229}
\gll Or-op  mua  nain  uruf-\textstyleEmphasizedVernacularWords{e}. \\
      \\
\glt
\z

descend-SS.SEQ  man  that1  see-IMP.2s  

`Go down and see that man.'

\ea%x1847
\label{ex:x1847}
\gll Ikoka  amap-urup-eya  op-\textstyleEmphasizedVernacularWords{ikua}. \\
      \\
\glt
\z

later  Bpx-ascend-2/3s.DS  hold-IMP.1p

`Later when he comes up, let's hold/grab him.'

\ea%x230
\label{ex:x230}
\gll Wi  urup-ep  mukuna  nain  umuk-\textstyleEmphasizedVernacularWords{uk}. \\
      \\
\glt
\z

3p.UNM  ascend-SS.SEQ  fire  that1  extinguish-IMP.3p

`Let them go up and extinguish the fire.' 

The imperative differs from the other moods in that it has a dual form in the first person but no singular: 

\ea%x446
\label{ex:x446}
\gll Aria,  i  owowa=ko  or-\textstyleEmphasizedVernacularWords{u}. \\
      \\
\glt
\z

alright,  1p.UNM  village=NF  descend-IMP.1d

`Alright, let's (d.) go down to the village.'

\ea%x1196
\label{ex:x1196}
\gll Yiena  ikos  akena  iw-\textstyleEmphasizedVernacularWords{u}. \\
      \\
\glt
\z

1p.GEN  two.together  truly  go-IMP.1d

`Let's just the two of us go together.'

The initial (or only) vowel in the second person imperative forms is deleted after the beneficiary suffix (\stepcounter{nx}{\thenx})-(\stepcounter{nx}{\thenx}).  

\ea%x432
\label{ex:x432}
\gll Iwera  ir-\textstyleEmphasizedVernacularWords{e.} \\
      \\
\glt
\z

coconut  ascend-IMP.2s

`Climb up the coconut palm (to get coconuts).'

\ea%x431
\label{ex:x431}
\gll Iwera  ir-om-\textstyleEmphasizedVernacularWords{e}. \\
      \\
\glt
\z

coconut  ascend-BEN-BNFY1.IMP.2s

`Climb up the coconut palm for me.'

\ea%x232
\label{ex:x232}
\gll Iwera  ir-om-\textstyleEmphasizedVernacularWords{a.} \\
      \\
\glt
\z

coconut  ascend-BEN-BNFY2.IMP.2s  

`Climb the coconut for him.'

\ea%x233
\label{ex:x233}
\gll Iwera  \textstyleEmphasizedVernacularWords{yia}  ir-om-\textstyleEmphasizedVernacularWords{a}ka. \\
      \\
\glt
\z

coconut  1p.ACC  ascend-BEN-BNFY2.IMP.2p  

`Climb (plural) the coconut for us.'

The semantics of the imperative and the functional aspects of commands are discussed in \sectref{sec:7.3}. On the use of subject pronouns with imperatives, see \sectref{sec:3.5.11}. The imperative forms of the verbs are also used in desiderative (\sectref{sec:8.3.2.1.3}) and conative (\sectref{sec:8.3.2.1.5}) constructions. 

\subsubsection[Tense and person/number in final verbs]{Tense and person/number in final verbs}
\hypertarget{RefHeading20421935131865}{}
Tense is a ``\textstyleBibliogCitationAAAstyleChar{grammaticalized expression of  location in time}'' \citep[9]{Comrie1985}. Mauwake has a straightforward three-tense system in the finite verbs marking past, present and future time reference. The tense system is simple compared with most other Papuan languages, many of which have more than three genuine tense distinctions and/or interaction between tense and status\footnote{`Status' here refers to the distinction between realis and irrealis.} resulting in several ``tenses'' (\textstyleBibliogBaseChar{Foley 1986}:158-63). Of the most closely related languages well studied so far, Usan has five tenses (\textstyleBibliogBaseChar{Reesink 1987}:98) out of which one, uncertain future/subjunctive, is semantically related to irrealis. Maia has a complete status system instead of a tense system, and temporal relations are inferred from the realis or irrealis status and the aspects (\textstyleBibliogBaseChar{Hardin 2002}:55). According to \textstyleBibliogBaseChar{Foley}, \textstyleBibliogCitationAAAstyleChar{``most Papuan languages are tense-dominated [rather than status-dominated]''} (1986:162). In Mauwake the status hardly plays any role at all.

Portmanteau morphemes of the tense and person/number markers are very common in Papuan languages, but having the two distinct from each other is not uncommon either (\textstyleBibliogBaseChar{Foley 1986}:137). The tense and person suffixes are separate morphemes in Mauwake, but have an interesting interplay with each other.

The speech event is taken as the reference point. The tense suffixes in themselves only distinguish between two tenses, past and non-past, and the further distinction between present and future is made by the person/number suffixes.  The person/number suffixes, on the other hand, are the same in past and present tense except for the first and third person singular forms. 

\ea%x1029
\label{ex:x1029}
\gll Unan  \textstyleEmphasizedVernacularWords{aakun-e-mik},  aakisa  \textstyleEmphasizedVernacularWords{aakun-i-mik}  ne    \\
      \\
\glt
\z

yesterday  talk-PA-1/3p  now/today  talk-Np-PR.1/3p  ADD  

uurika  nainiw  \textstyleEmphasizedVernacularWords{aakun-i-yen}.

tomorrow  again  talk-Np-FU.1p

`Yesterday we talked, now/today we talk and tomorrow we'll talk again.'

The non-past marker in the second person plural future form is \textstyleEmphasizedVernacularWords{\nobreakdash-}\textstyleStyleVernacularWordsItalic{o}  instead of \textstyleEmphasizedVernacularWords{\nobreakdash-}\textstyleStyleVernacularWordsItalic{i}  possibly because of  assimilation to the labial consonant /w/ in the person/number suffix. 

The verb conjugation classes determining the past tense suffix vowels are discussed in the section on morphophonology (\sectref{sec:2.3.3.3}). The beneficiary and the counterfactual suffixes influence the past tense suffix in the following way. After the counterfactual the past tense suffix is always \textstyleEmphasizedVernacularWords{\nobreakdash-}\textstyleStyleVernacularWordsItalic{a}. When the beneficiary suffix is present, the vowel of the past tense suffix is assimilated to it.

The following table presents the full paradigms for the tense and person/number suffixes.





\begin{tabular}{llll}
\mytoprule
 & \raggedleft Non-past  -  Present\& {\quad}{\quad}person  \par & \raggedleft Non-past  -  Future\& person\par & Past    -    person\\
1s & -i    -yem & -i    -nen & -a/E    -m\\
2s & -i    -n & -i    -nan & -a/E    -n\\
3s & -i    -ya & -i    -non & -a/E    -k\\
1p & -i    -mik & -i    -yen & -a/E    -mik\\
2p & -i    -man & -o    -wen & -a/E    -man\\
3p & -i    -mik & -i    -kuan & -a/E    -mik\\

\mybottomrule 
\end{tabular}



\begin{table}
\caption{Tense and person/number suffixes}
\label{tab:13}
\end{table}

The person/number marking in the verb distinguishes three persons in both singular and plural. There is no dual number, nor is there inclusive-exclusive distinction in the first person plural form. The plural is marked only for humans, spirits and important animals. The singular form is used for less important and small animals as well as all inanimates:

\ea%x236
\label{ex:x236}
\gll Waa  muuka  arow  ekap-o-\textstyleEmphasizedVernacularWords{k}. \\
      \\
\glt
\z

pig  boy  three  come-PA-3s  

`Three piglets came.'

Besides their primary meaning, the present and future tenses also have secondary meanings. The present tense form of the first or third person plural is used for generic or time-neutral statements (\stepcounter{nx}{\thenx}). For the habitual aspect in the present, the simple present tense (\stepcounter{nx}{\thenx}) is an alternative to the full habitual aspect form.\footnote{Continuous aspect form is required for the past habitual (\sectref{sec:3.8.5.1.1.2}).} 

\ea%x1034
\label{ex:x1034}
\gll Ifa  yia  keraw-i-ya  nain  miira  \textstyleEmphasizedVernacularWords{saawirin-i-mik}. \\
      \\
\glt
\z

snake  1p.ACC  bite-Np-PR.3s  that1  face  become.round-Np-PR.1/3p

 `When a snake bites us, we become dizzy.'

\ea%x1035
\label{ex:x1035}
\gll Nos=ke  anane  urema  efar  \textstyleEmphasizedVernacularWords{ikum-ar-i-n}. \\
      \\
\glt
\z

2s.FC=CF  always  bandicoot  1s.DAT  illicitly-INCH-Np-PR.2s

`You always steal bandicoots from me.'

The future tense in any person form is used for habitual or generic conditionals (8.3.5). The example (\stepcounter{nx}{\thenx}) refers to a traditional custom and is generic, even if the first person form of the verb is used, and the first person pronoun as well.

\ea%x1640
\label{ex:x1640}
\gll Waaya  \textstyleEmphasizedVernacularWords{ika-i-non},  waaya  \textstyleEmphasizedVernacularWords{uup-i-nan},  naap. \\
      \\
\glt
\z

pig  be-Np-FU.3s  pig  cook-Np-FU.2s  thus

`If there is a pig, you will cook it - it is like that.'

\ea%x1641
\label{ex:x1641}
\gll Ikoka  yo  \textstyleEmphasizedVernacularWords{um-i-nen},  muuka  nain  nainiw  wiena  \\
      \\
\glt
\z

later  1s.UNM  die-Np-FU.1s  son  that1  again  3p.GEN

\textstyleEmphasizedVernacularWords{aaw-i-kuan}.

take-Np-FU.3p

`Later if I die (without paying the bride price) they will take the son back.'

The second person singular form of the future tense has two other usages as well. It can be used when referring to generic or habitual  situations,  especially in process descriptions which can also be understood as instructions. For these, the first person plural form of the present tense is much more common, but often the two alternate. The following example describes work involved in harvesting taro roots (and the addressee that the story was told to, had no garden, so the speaker did not refer to her personally). 

\ea%x1038
\label{ex:x1038}
\gll Perek-ami  en-ow(a)  gelemuta  \textstyleEmphasizedVernacularWords{on-i-nan}. \\
      \\
\glt
\z

harvest-SS.SIM  eat-NMZ  little  make-Np-FU.2s

`When you harvest it you make a little feast.'

It is also used for a command, or a statement of obligation:

\ea%x1039
\label{ex:x1039}
\gll Ikoka  kuisow  kuuma  kuisow  \textstyleEmphasizedVernacularWords{yi-i-nan}. \\
      \\
\glt
\z

later  one  stick  one  give.me-Np-FU.2s

`Very soon you have to give me 10 kina.' Or: `Give me 10 kina very soon.'

\subsubsection[Medial verb marking]{Medial verb marking}
\hypertarget{RefHeading20441935131865}{}
The distinction between medial and final verbs is common in Papuan languages (\textstyleBibliogBaseChar{Foley 1986}:11). Especially in the \textstyleAcronymallcaps{TNG} languages the medial verbs are ``\textstyleBibliogCitationAAAstyleChar{very common, universal over a wide area [and the] systems often highly complex}'' (\textstyleBibliogBaseChar{Wurm 1982}:63, also Roberts 1997). The medial verbs typically lack the full tense and person/number marking of the finite verbs. Instead, they usually indicate whether the subject is the same as the subject of the following verb, and/or whether the action of the verb is simultaneous or sequential with the action of the following verb.\footnote{This is called the ``switch-reference system''. \textstyleBibliogBaseChar{The question whether} the system really tracks the topic (pragmatic subject) or the syntactic subject  is discussed further in 8.2.3.} As for person reference, the verbs in the simplest systems only show whether the two subjects are the same or different, but in the most elaborate systems the subjects of both the medial verb and that of the following verb are shown in the medial verb, which is thus even more specific than the finite verb.\footnote{Usan makes a distinction between neutral and future medial verbs, and in both of these there is a division between same-subject and different-subject forms, but not between sequentiality and simultaneity \textstyleBibliogBaseChar{(Reesink 1987}:87-92). Maia only uses medial verbs when a clause has the same subject as the following clause; a distinction is made between simultaneous and sequential actions. When the following clause has a different subject, finite forms plus the contrast clitic \textstyleFootnoteBaseChar{\nobreakdash-(}\textstyleFootnoteBaseChar{\textit{d)i}} is used (\textstyleBibliogBaseChar{Hardin 2002}:87). Amele makes the basic distinction between the same-subject and different-subject medial verbs, and has simultaneous and sequential forms in both. But it also has different-subject simultaneous irrealis forms (\textstyleBibliogBaseChar{Roberts 1987}:275). Particularly the East New Guinea Highlands languages are known for marking the anticipatory subject in their medial verbs. See \citet[40-41]{Franklin1983} for a succinct list of switch-reference characteristics in Papuan languages, and \citet{Roberts1997} for a more comprehensive overview} 

In Mauwake the medial verb system is relatively simple. The suffixes indicate whether the subject of the medial verb stays the same in the following verb as well, and in the ``same subject following'' (\textstyleAcronymallcaps{SS}) verbs there is a further distinction between simultaneous and sequential action. The ``different subject following'' (\textstyleAcronymallcaps{DS}) verbs indicate sequential action; for simultaneous action one needs to use the continuous (\sectref{sec:3.8.5.1.1.2}) or stative aspect (\sectref{sec:3.8.5.1.1.3}). The \textstyleAcronymallcaps{DS} verbs also have some person marking but not as detailed as the finite verbs have.

The two sections below give a general outline of the person reference in medial clauses, but it is discussed in more detail in \sectref{sec:8.2.3}.

Typically, medial verbs have much fewer inflectional possibilities than finite verbs \citep[11]{Foley1986}. This is the case in Mauwake too: mood or tense and full person/number marking cannot be suffixed to the medial verbs. Derivational suffixes, on the other hand, can freely occur on the medial verbs. In Tail-Head linkage a new sentence begins with a medial verb copy of the finite verb that ended the previous sentence (8.2.3.5). Often the derivational morphology of the two verbs is the same, and sometimes the medial verb has less derivation than the final verb; very rarely it has even \textstyleEmphasizedWords{\textsc{more}} (\stepcounter{nx}{\thenx}):

\ea%x237
\label{ex:x237}
\gll Ikiwosa  wiar  pepekim-ep  kaik-a-m.  Kaik-\textstyleEmphasizedVernacularWords{om}-\textbf{a}p{\dots} \\
      \\
\glt
\z

head  3.DAT  measure-SS.SEQ  tie-PA-1s  tie-BEN-BNFY2.SS.SEQ

`I measured her head and tied it (=headdress). I tied it for her and {\dots}'

\paragraph[Same-subject marking]{Same-subject marking}
\hypertarget{RefHeading20461935131865}{}
When the subject of the medial clause is the same as that of the following clause, the verb itself does not give any indication of the person and number of the subject, only that the same subject continues in the next clause.\footnote{For exceptions to this, see \sectref{sec:8.2.3} where the functional aspects of switch reference are discussed.} If the actions are sequential, i.e. the action indicated by the verb in the medial clause precedes that of the following clause, the suffix is \nobreakdash-\textstyleStyleVernacularWordsItalic{ap}  or \nobreakdash-\textstyleStyleVernacularWordsItalic{ep} (\stepcounter{nx}{\thenx}) depending on the conjugation class (\sectref{sec:2.3.3.3}).\footnote{For the second conjugation class verb \textstyleFootnoteBaseChar{\textit{or}}\textit{-} `descend' the suffix is -\textstyleFootnoteBaseChar{\textit{op}}\textstyleEmphasizedVernacularWords{.}}  

\ea%x238
\label{ex:x238}
\gll Owowa  ek-\textstyleEmphasizedVernacularWords{ap},  wailal-\textstyleEmphasizedVernacularWords{ep}  akia  ik-e-k. \\
      \\
\glt
\z

village  go-SS.SEQ  be.hungry-SS.SEQ  banana  roast-PA-3s  

`He went to the village, was hungry and roasted bananas.' 

If the verb has the beneficiary suffix \textstyleStyleVernacularWordsItalic{-a} or \textstyleStyleVernacularWordsItalic{-e} (\sectref{sec:3.8.3.1}), the vowel of the medial verb suffix gets assimilated to it (\stepcounter{nx}{\thenx}), (\stepcounter{nx}{\thenx}).

\ea%x1930
\label{ex:x1930}
\gll {\dots}eka=pa  merena  yasuw-om-\textstyleEmphasizedVernacularWords{e}p...  (cf.  yasuw-\textstyleEmphasizedVernacularWords{a}p) \\
      \\
\glt
\z

water=LOC  foot  wash-BEN-BNFY.1-SS.SEQ

`{\dots} she washed my feet in water (and) {\dots}'

\ea%x1929
\label{ex:x1929}
\gll ...waaya  nain  uup-om-\textstyleEmphasizedVernacularWords{a}p  samapora=pa \\
      \\
\glt
\z

pig  that1  cook-BEN-BNFY2.SS.SEQ  floor=LOC  

wu-ap  maak-e-mik...  (cf.  uup-\textstyleEmphasizedVernacularWords{e}p)

put-SS.SEQ  tell-PA-1/3p

`{\dots} they cooked the pig for him, put it on the floor and told him, {\dots}'

When the medial clause subject is the same as the subject in the following clause but the two actions are simultaneous, or at least overlapping, the suffix is \nobreakdash-\textstyleStyleVernacularWordsItalic{ami} or \nobreakdash-\textstyleStyleVernacularWordsItalic{emi}  (or \nobreakdash-\textstyleStyleVernacularWordsItalic{omi}) according to the conjugation class of the verb. Even if the action of the medial verb may often be \textstyleEmphasizedWords{\textsc{interpreted}} as continuous (\stepcounter{nx}{\thenx}), the suffix in itself only indicates simultaneity (\stepcounter{nx}{\thenx}). Continuous aspect marking may be needed for clarity when continuity is in focus (\stepcounter{nx}{\thenx}).

\ea%x239
\label{ex:x239}
\gll Wi  sawur  ir-\textstyleEmphasizedVernacularWords{ami}  fan  yiar  pok-a-mik. \\
      \\
\glt
\z

3p.UNM  spirit  go-SS.SIM  here  1p.DAT  sit.down-PA-1/3p  

`As the spirits were going they sat down here with us.' 

\ea%x240
\label{ex:x240}
\gll {\dots}ekap-\textstyleEmphasizedVernacularWords{emi}  koora=pa  yia  wua-i-mik.  \\
      \\
\glt
\z

come-SS.SIM  house=LOC  1p.ACC  put-Np-PR.1/3p  

`{\dots}coming (=upon arrival) they put us in the house.' 

\ea%x241
\label{ex:x241}
\gll \textstyleEmphasizedVernacularWords{Soomar-em-ik}\textstyleEmphasizedVernacularWords{-ok}  ifara  oko  uruf-a-k.  \\
      \\
\glt
\z

walk-SS.SIM-be-SS  vine  other  see-PA-3s  

`He was walking and saw another vine. '

The verb \textstyleStyleVernacularWordsItalic{ik}- `be' is different from other verbs in that there is no differentiation between the simultaneous and sequential forms in the same-subject medial verb: in example (\stepcounter{nx}{\thenx}) the actions are simultaneous, in (\stepcounter{nx}{\thenx}) sequential. Also, the verb does not take either one of the normal same-subject suffixes.

\ea%x242
\label{ex:x242}
\gll Owowa=pa  neeke  \textstyleEmphasizedVernacularWords{ik-ok}  mua  maak-ek{\dots} \\
      \\
\glt
\z

village=LOC  there.CF  be-SS  man  tell-PA-3s  

`While they were there in the village she told her husband, {\dots}'

\ea%x243
\label{ex:x243}
\gll No  kaaneke  \textstyleEmphasizedVernacularWords{ik-ok}  kerer-e-n? \\
      \\
\glt
\z

2s.UNM  where.CF  be-SS  appear-PA-2s  

`Where have you been and now come?' 

\paragraph[Different-subject marking]{Different-subject marking}
\hypertarget{RefHeading20481935131865}{}
If the subject of the medial clause is different from that of the following clause, the suffix of the \textstyleAcronymallcaps{DS} verb reflects this. There are some person/number distinctions in these suffixes, even though not as many as in the finite verbs. The first person singular and plural are distinguished from all the other forms; in the other persons the distinction is based on the number: second and third person singular share the same suffix, and second and third person plural likewise.\footnote{There is great variation in this area in Papuan languages. Some only have one form to indicate that the subject changes, others have partial or full differentiation according to the person, some even show the subject of the following clause in the medial verb.}  





\begin{tabular}{lll}
\mytoprule
 & \multicolumn{1}{l}{Singular}

 & Plural\\
1 & \multicolumn{2}{l}{-Vmkun}

\\
2 & \multicolumn{1}{l}{-eya}

 & -iwkin\\
\multicolumn{1}{l}{3}

 &  & \\
\hhline{-~~}

\mybottomrule
\end{tabular}



\begin{table}
\caption{Suffixes marking a different subject in the following clause}
\label{tab:14}
\end{table}

\ea%x244
\label{ex:x244}
\gll Imen-ap  maak-\textstyleEmphasizedVernacularWords{iwkin}  o  miim-o-k. \\
      \\
\glt
\z

find-SS.SEQ  tell-2/3p.DS  3s.UNM  precede-PA-3s  

`They found him and told him, and he went ahead of them.' 

\ea%x245
\label{ex:x245}
\gll Mik-\textstyleEmphasizedVernacularWords{amkun}  me  um-o-k,  wiowa  onaiya  ikiw-em-ik-\textstyleEmphasizedVernacularWords{eya} \\
      \\
\glt
\z

spear-1s/p.DS  not  die-PA-3s  spear  with  go-SS.SIM-be-2/3s.DS  

Olas=ke  war-ek.

Olas=CF  shoot-PA-3s

`When I speared it, it didn't die, (but) as it was going with the spear Olas shot it.'

The suffix  is -\textstyleStyleVernacularWordsItalic{aya} instead of \nobreakdash-\textstyleStyleVernacularWordsItalic{eya}  in a few short conjugation class 1 verbs (\sectref{sec:3.8.4.1}) (\stepcounter{nx}{\thenx})\footnote{The vowel \textstyleFootnoteBaseChar{\textit{-a}}  is somewhat more common in the Muaka dialect group where the 2/3s.DS marker is -\textit{era} instead of \nobreakdash-\textit{eya.}} and in those benefactive verbs where the first vowel of the suffix is assimilated to the preceding vowel of the beneficiary suffix (\stepcounter{nx}{\thenx}). 

\ea%x493
\label{ex:x493}
\gll Iw-\textstyleEmphasizedVernacularWords{aya}  nan  miira  saawirin-e-k. \\
      \\
\glt
\z

enter-2/3s.DS  there  face  become.round-PA-3s

`As [the poison] entered [his liver], he became dizzy.'

\ea%x695
\label{ex:x695}
\gll Aaya=ko  yia  aaw-om-\textstyleEmphasizedVernacularWords{aya}  enim-i-yan.  \\
      \\
\glt
\z

sugarcane=NF  1p.ACC  get-BEN-BNFY2.2/3s.DS  eat-Np-FU.1p

`Get us sugarcane and we'll eat it.'

The different-subject marking \nobreakdash-\textstyleStyleVernacularWordsItalic{eya} is also used with some non-verbs. This seems to be uncommon in PNG languages: in Roberts' (1997:137) survey the very few examples where the switch-reference marking was on non-verbs these were pro-clausal substitutes like a demonstrative or vocative. In Mauwake the \textstyleAcronymallcaps{DS} suffix can be added to nouns (\stepcounter{nx}{\thenx}) or adjectives (\stepcounter{nx}{\thenx}), or the negative adverbs \textstyleStyleVernacularWordsItalic{weetak} and \textstyleStyleVernacularWordsItalic{marew} (\stepcounter{nx}{\thenx}) functioning as predicates in verbless clauses. When it is added to words ending in -\textstyleStyleVernacularWordsItalic{a}, the first vowel in the suffix gets assimilated to this vowel (\stepcounter{nx}{\thenx}).

\ea%x250
\label{ex:x250}
\gll Enakiwa-\textstyleEmphasizedVernacularWords{ya}  me  aaw-e-m. \\
      \\
\glt
\z

half-2/3s.DS  not  take-PA-1s  

`There was (only) half (left), so I didn't take any/it.'

\ea%x251
\label{ex:x251}
\gll Mauwow  maneka-\textstyleEmphasizedVernacularWords{ya}=na  yia  maak-i-non. \\
      \\
\glt
\z

work  big-2/3s.DS=TP  1p.ACC  tell-Np-FU.3s  

`If the work is big, she will tell us.'

\ea%x252
\label{ex:x252}
\gll Soomia  marew-\textstyleEmphasizedVernacularWords{eya}  amap-ep-om-a-m. \\
      \\
\glt
\z

spoon  none-2/3s.DS  BPx-come-BEN-BNFY2.PA-1s  

`She has/had no spoons (lit: there are/were no spoons) so I brought them to her.'

When the different subject marking \nobreakdash-\textstyleStyleVernacularWordsItalic{eya} is added to the adverb \textstyleStyleVernacularWordsItalic{naap} `thus', the outcome is a consecutive connective `therefore, so' (\sectref{sec:3.11.2}).

\paragraph[Tense and medial verbs]{Tense and medial verbs}
\hypertarget{RefHeading20501935131865}{}
The medial verbs have no tense marking, so the tense in a medial clause is interpreted in relation to that of the next finite clause. When the finite clause is in the past tense, both a simultaneous (\stepcounter{nx}{\thenx}) and a sequential medial clause (\stepcounter{nx}{\thenx}) are also understood to be in the past tense.

\ea%x1025
\label{ex:x1025}
\gll Iwera  uruk-am-ika-iwkin  wi  \textstyleEmphasizedVernacularWords{ikiw-emi} \\
      \\
\glt
\z

coconut  drop-SS.SIM-be-2/3p.DS  3p.UNM  go-SS.SIM

\textstyleEmphasizedVernacularWords{aaw-em-ik-e-mik}.

take-SS.SIM-be-PA-1/3p

`They\textsubscript{i} kept dropping coconuts, and they\textsubscript{j} went and got them.'

\ea%x1023
\label{ex:x1023}
\gll Owowa  \textstyleEmphasizedVernacularWords{or-op,  wailal-ep},  akia  \textstyleEmphasizedVernacularWords{ik-e-k}. \\
      \\
\glt
\z

village  descend-SS.SEQ  be.hungry-SS.SEQ  banana  roast-PA-3s

`He came down to the village, was hungry and roasted bananas.'

Since a sequential verb indicates that the action takes place before another action, a sequential medial clause preceding a present tense final clause has to be interpreted to be in the past tense, whereas a simultaneous clause is interpreted to be in the present tense like the final verb.

\ea%x1024
\label{ex:x1024}
\gll Iperuma  nain=ke  mua  \textstyleEmphasizedVernacularWords{puuk-ap}  owora  \textstyleEmphasizedVernacularWords{en-emi} \\
      \\
\glt
\z

eel  that1=CF  man  become-SS.SEQ  betelnut  eat-SS.SIM

afura  \textstyleEmphasizedVernacularWords{buan-em-ika-i-ya}.

lime.container  knock-SS.SIM-be-Np-PR.3s

`The eel has become man, and is eating betelnut and knocking the lime container.'

Both a sequential and a simultaneous medial clause preceding a future final clause are also understood as future clauses. The action in a sequential medial clause takes place before that in the final clause, but it is still in the future (\stepcounter{nx}{\thenx}). The action in a simultaneous clause is partly or fully overlapping with that in the final clause (\stepcounter{nx}{\thenx}).

\ea%x1026
\label{ex:x1026}
\gll Is=ke  maa  uup-emkun  wi  \textstyleEmphasizedVernacularWords{ekap-ep}  \textstyleEmphasizedVernacularWords{enim-i-kuan}. \\
      \\
\glt
\z

1p.FC=CF  food  cook-1s/p.DS  3p.UNM  come-SS.SEQ  eat-Np-FU.3p

`We'll cook the food and they'll come and eat it.' Or: `When we have cooked the food they will come and eat it.'

\ea%x1027
\label{ex:x1027}
\gll Wi  \textstyleEmphasizedVernacularWords{ir-ami}  nia  \textstyleEmphasizedVernacularWords{aaw-emi}  efa  \textstyleEmphasizedVernacularWords{ifakim-i-kuan}. \\
      \\
\glt
\z

3p.UNM  come-SS.SIM  2p.ACC  take-SS.SIM  1s.ACC  kill-Np-FU.3p

`They will come and take you and kill me.'

The medial verb form cannot be used in the following example, because the first verb refers to time preceding the speech event and the second verb to time following it. Final verbs with different tenses have to be used, and in this case it is most natural to place the past tense verb in a relative clause: 

\ea%x1030
\label{ex:x1030}
\gll Mukuna  kerer-e-k  nain  kamenap  umuk-i-yan? \\
      \\
\glt
\z

fire  appear-PA-3s  that1  how  extinguish-Np-FU.1p

`How shall we extinguish the fire that started?'

The medial verbs acquire more absolute-relative tense character of ``past in the past'' \citep[65]{Comrie1985} in those cases where sequential medial clauses are either right-dislocated and placed after the final clause (\stepcounter{nx}{\thenx}) or placed inside another medial clause (\stepcounter{nx}{\thenx}), or when there is a separate time expression referring to earlier time than that indicated by the final verb (\stepcounter{nx}{\thenx}). 

\ea%x1031
\label{ex:x1031}
\gll Rubaruba  nain=ke  ona  emeria  nain  aaw-ep  p-ikiw-o-k, \\
      \\
\glt
\z

Rubaruba  that1=CF  3s.GEN  woman  that1  take-SS.SEQ  BPx-go-PA-3s

\textstyleEmphasizedVernacularWords{iw-iwkin}.

give.him-2/3p.DS

`That Rubaruba took his wife and took her (away), when they had given her to him.'

\ea%x1032
\label{ex:x1032}
\gll Um-eya  merena  ere-erup  [\textstyleEmphasizedVernacularWords{ifara  aaw-ep}]  kaik-ap  \\
      \\
\glt
\z

die-2/3s.DS  leg  RDP-two  vine  get-SS.SEQ  tie-SS.SEQ

nabena  suuw-ap  akua  aaw-ep  or-o-m.

carrying.pole  push-SS.SEQ  shoulder  take-SS.SEQ  descend-PA-1s

 `It died, and I tied its legs in pairs with a vine that I had gotten, and pushed it to the carrying pole and carried it down on my shoulder.

\ea%x1033
\label{ex:x1033}
\gll \textstyleEmphasizedVernacularWords{Iiriw}  inasin  mua  nain=ke  naap  wia  \textstyleEmphasizedVernacularWords{maak-eya} \\
      \\
\glt
\z

earlier  spirit  man  that=CF  thus  3p.ACC  tell-2/3p.DS

wi  naap  on-a-mik.

3p.UNM  thus  do-PA-1/3p

 `The spirit man had earlier told them like that and they did so.'

\subsection{Verb classes} 
\hypertarget{RefHeading20521935131865}{}
Verbs can be divided into classes on the basis of various criteria.  Conjugation classes based on morphological/inflectional criteria are usually arbitrary and unrelated to other parts of the grammar \citep[191]{Anderson1985b}. They are only touched on briefly in the next subsection. Transitivity as a basis of verb classes is discussed in \sectref{sec:3.8.4.2}, and valence-changing operations in \sectref{sec:3.8.4.3}.  Verb classes based on semantic features are described in \sectref{sec:3.8.4.4}.

\subsubsection[Conjugation classes]{Conjugation classes}
\hypertarget{RefHeading20541935131865}{}
In the Mauwake lexicon the verbs are divided into classes 1 and 2 depending on whether they have /a/ or /e\~{o}/ as the past tense suffix. There are morphophonological rules for deriving the past tense marking for most verbs (see \sectref{sec:2.3.3.3}), but since some of the rules are rather complicated, and because they do not cover a number of cases like (\stepcounter{nx}{\thenx}) and (\stepcounter{nx}{\thenx}) below, the division into two separate classes is maintained.

\ea%x253
\label{ex:x253}
\gll miim-\textstyleEmphasizedVernacularWords{a}-k \\
      \\
\glt
\z

hear-PA-3s 

`he heard'

\ea%x254
\label{ex:x254}
\gll miim-\textstyleEmphasizedVernacularWords{o}-k \\
      \\
\glt
\z

precede-PA-3s 

`he went ahead'

In Class 1, transitive verbs outnumber intransitive verbs over four times, but Class 2 is divided almost equally between transitive and intransitive verbs.\footnote{For this count, the verbs formed with the verbalizer \textstyleFootnoteBaseChar{\textit{--ar}} and the causative \textstyleFootnoteBaseChar{\textit{--ow}} were deleted from the total of 857 verbs, since both these suffixes influence the choice of the past tense vowel.}

\ea%x255
\label{ex:x255}
\gll puuk-\textstyleEmphasizedVernacularWords{a}-k    vs.      puk-\textstyleEmphasizedVernacularWords{o}-k  \\
      \\
\glt
\z

cut-PA-3s          burst-PA-3s  

`he cut (it)'          `it burst' 

\ea%x256
\label{ex:x256}
\gll teek-\textstyleEmphasizedVernacularWords{a}-k    vs.      ten-\textstyleEmphasizedVernacularWords{e}-k  \\
      \\
\glt
\z

pluck-PA-3s        collapse-PA-3s

`he plucked (it)'      `it collapsed'  (also: `it broke away')

\subsubsection[Verb classes based on transitivity]{Verb classes based on transitivity}
\hypertarget{RefHeading20561935131865}{}
With the term \textstyleEmphasizedWords{\textsc{transitivity}} of a verb I refer to its \textstyleEmphasizedWords{\textsc{syntactic}} transitivity, i.e. ``\textstyleBibliogCitationAAAstyleChar{the number of overt morpho-syntactically coded arguments it takes}'' (Van Valin and LaPolla 1997:147). 

Intransitive verbs in Mauwake only require a subject, whereas transitive verbs require a direct object as well. This definition differs slightly from that of \citet[397]{Crystal1997}, who defines as transitive verbs those that \textstyleEmphasizedWords{\textsc{can}} take a direct object, and as intransitive those that \textstyleEmphasizedWords{\textsc{cannot}} (emphasis mine). Crystal's definition works for Mauwake when considering prototypical patient/undergoer objects, but it fails in the cases where the syntactic object manifests other roles not required by the semantic structure of the verb.\footnote{The syntactic transitivity of a verb can differ from both its semantic and macrorole transitivity (Van Valin and La Polla 1997).}  

In some languages verb roots can be neutral as to transitivity (Kittil\"a 2002:53), but in Mauwake each verb has a basic transitivity value. Most verbs are either intransitive (\sectref{sec:3.8.4.2.1}) or transitive (\sectref{sec:3.8.4.2.2}). There are only a few ambitransitives (\sectref{sec:3.8.4.2.3}). Mauwake does not have a regular class of ditransitive verbs that would require two objects. Instead, some verbs that are transitive easily allow a second object. And in the small class of the object cross-referencing verbs (\sectref{sec:3.8.4.2.4}), in which the pronominal object is in the verb root, two of the verbs require a second object as well. 

The basic transitivity of a verb can be changed with valence-changing strategies (\sectref{sec:3.8.4.3}). Causative (\sectref{sec:3.8.2.3.1}) and benefactive morphology (\sectref{sec:3.8.2.3.3}) as well as  possessor raising (\sectref{sec:5.3.2.3}) are processes that increase the number of syntactic objects in a clause.

\paragraph[Intransitive verbs]{Intransitive verbs}
\hypertarget{RefHeading20581935131865}{}
In Mauwake the class of basic, or ``ordinary'', intransitive verbs consists of a semantically very diverse group including involuntary processes (\stepcounter{nx}{\thenx}), many motion verbs (\stepcounter{nx}{\thenx}), and some bodily function verbs (\stepcounter{nx}{\thenx}).

\ea%x266
\label{ex:x266}
\gll Fikera  \textstyleEmphasizedVernacularWords{aw-o-k}. \\
      \\
\glt
\z

kunai.grass  burn-PA-3s  

`The kunai grass burned.'

\ea%x267
\label{ex:x267}
\gll Kuuten  ikos  \textstyleEmphasizedVernacularWords{karu-e-mik}. \\
      \\
\glt
\z

Kuuten  with  run-PA-1/3p  

`I ran with Kuuten.'

\ea%x269
\label{ex:x269}
\gll Niir-emi  \textstyleEmphasizedVernacularWords{pisi-e}\textstyleEmphasizedVernacularWords{-k}. \\
      \\
\glt
\z

laugh-SS.SIM  fart-PA-3s  

`He laughed and farted.'

Some experience verbs expressing physiological states are also regular intransitive verbs:

\ea%x1485
\label{ex:x1485}
\gll Maa  enowa  nopa-yiaw-ep  \textstyleEmphasizedVernacularWords{wailal-ep} \\
      \\
\glt
\z

food  eat-NMZ  search-move.around-SS.SEQ  get.hungry-SS.SEQ

ma-e-mik, ``...''

say-PA-1/3p

`They searched  around for food and got hungry and said, ``{\dots}'' '

The verbs derived with the inchoative suffix \nobreakdash-\textstyleStyleVernacularWordsItalic{ar}  (\sectref{sec:3.8.2.2.2}) are mostly intransitive, but a few of them are transitive (\stepcounter{nx}{\thenx}). 

\ea%x271
\label{ex:x271}
\gll Uuw-ap  uuw-ap  \textstyleEmphasizedVernacularWords{lebum-ar-e-m}. \\
      \\
\glt
\z

work-SS.SEQ  work-SS.SEQ  lazy-INCH-PA-1s  

`I worked and worked and got tired.'

\ea%x1486
\label{ex:x1486}
\gll Nan  teeria  \textstyleEmphasizedVernacularWords{manek-ar-e-k},  owowa  pun  manek-ar-e-k. \\
      \\
\glt
\z

there  family  big-INCH-PA-3s  village  also  big-INCH-PA-3s

`The family grew big there, and the village grew big too.'

\ea%x1836
\label{ex:x1836}
\gll Maa  unowa  oram  me  \textstyleEmphasizedVernacularWords{amis}\textstyleEmphasizedVernacularWords{-}\textstyleEmphasizedVernacularWords{ar}\textstyleEmphasizedVernacularWords{-}\textstyleEmphasizedVernacularWords{i}\textstyleEmphasizedVernacularWords{-}\textstyleEmphasizedVernacularWords{mik},  weetak. \\
      \\
\glt
\z

thing  many  just  not  knowledge-INCH-Np-PR.1/3p  no

`We don't just gain knowledge of many things (without learning them), no.'

Climate expressions often use intransitive verbs. There is no separate class of verbs for climate expressions.\footnote{Climate expressions also use directional verbs (\textstyleFootnoteBaseChar{\textit{ipia oraiya} }`the rain descends'), inchoative verbs (\textstyleFootnoteBaseChar{\textit{kokomarek} }`it got dark') and transitive verbs (\textstyleFootnoteBaseChar{\textit{ama fookak } }`the sun split (tr.)').}

\ea%x270
\label{ex:x270}
\gll Aapereka  \textbf{paran-em-ika-i-ya}. \\
      \\
\glt
\z

cloud  rumble-SS.SIM-be-Np-PR.3s

`It is thundering.'

Intransitive clauses are discussed in 5.4.

\paragraph[Transitive verbs]{Transitive verbs}
\hypertarget{RefHeading20601935131865}{}
Transitive verbs require a subject and an object. A [+human] object needs to be marked with an accusative pronoun (\sectref{sec:3.5.3}) regardless of the presence or absence of a separate object \textstyleAcronymallcaps{NP}. 

\ea%x294
\label{ex:x294}
\gll Yaapan  wia  ifakim-e-mik. \\
      \\
\glt
\z

Japan  3p.ACC  kill-PA-1/3p  

`They killed the Japanese.'

Besides the prototypical transitive verbs with an agent subject and a patient-of-change object (Giv\'on 1984:96) like (\stepcounter{nx}{\thenx}) and (\stepcounter{nx}{\thenx}), also many verbs of perception (\stepcounter{nx}{\thenx}), cognition (\stepcounter{nx}{\thenx}) and emotion (\stepcounter{nx}{\thenx}) are transitive.

\ea%x295
\label{ex:x295}
\gll Wiipa  erup  wia  \textstyleEmphasizedVernacularWords{sesek-a-mik}. \\
      \\
\glt
\z

girl  two  3p.ACC  send-PA-1/3p  

`They sent the two girls.'

\ea%x296
\label{ex:x296}
\gll Yo  me  efa  \textstyleEmphasizedVernacularWords{enim-uk}. \\
      \\
\glt
\z

1s.UNM  not  1.ACC  eat-IMP.3p  

`Let them not eat me.'\footnote{This was said in a traditional story by a spirit that was able to change into a man or into an eel, which the people in the story were preparing to eat.}

\ea%x297
\label{ex:x297}
\gll Nomokowa  unowa  aakisa  wia  \textstyleEmphasizedVernacularWords{uruf-i-n.} \\
      \\
\glt
\z

2s/p.brother  many  now  3p.ACC  see-Np-PR.2s  

`Now you see many brothers of yours.

\ea%x298
\label{ex:x298}
\gll Nefa  \textstyleEmphasizedVernacularWords{amis-ar-ep}  ma-i-yem. \\
      \\
\glt
\z

2s.ACC  knowledge-INCH-SS.SEQ  say-Np-PR.1s  

`I am saying (this) because I know you.'

\ea%x299
\label{ex:x299}
\gll Yena  mua=ke  efa  \textstyleEmphasizedVernacularWords{kookal-ep}  manin(a)  uuw-owa  \\
      \\
\glt
\z

1s.GEN  man  1s.ACC  like-SS.SEQ  garden  work-NMZ  

efa  asip-i-ya.

1s.ACC  help-Np-PR.3s

`My husband likes me and helps me in the garden.'

If there is no other overt object available for a transitive verb, the maximally generic noun \textstyleStyleVernacularWordsItalic{maa} `thing'\footnote{The semantic area of \textstyleFootnoteBaseChar{\textit{maa}} is at least as wide that of its English equivalent `thing'. Because it is used so often with verbs denoting eating and preparing food, it has acquired a secondary meaning `food'.} is used as a dummy object. Compare the next two examples: in (\stepcounter{nx}{\thenx}) \textstyleStyleVernacularWordsItalic{maa} is added because of syntactic requirements, whereas in (\stepcounter{nx}{\thenx}) the lack of an overt object indicates a third person singular object.

\ea%x300
\label{ex:x300}
\gll (Yo)  \textstyleEmphasizedVernacularWords{maa}  uruf-i-yem. \\
      \\
\glt
\z

I  thing  see-Np-PR.1s  

`I see.'  (=I see something, or: I can see.) 

\ea%x301
\label{ex:x301}
\gll (Yo)  uruf-i-yem. \\
      \\
\glt
\z

I  see-Np-PR.1s  

`I see him/her/it.'

\ea%x1825
\label{ex:x1825}
\gll Iir  oko  \textstyleEmphasizedVernacularWords{maa}  enim-i-yem,  iir  oko  \textstyleEmphasizedVernacularWords{maa}  me  enim-i-yem. \\
      \\
\glt
\z

time  other  thing  eat-Np-PR.1s  time  other  thing  not  eat-Np-PR.1s

`Sometimes I eat, sometimes I don't eat.'

The language-specific characteristic of syntactic transitivity (Kittil\"a 2002:49-51) is illustrated by a number of verbs that are transitive in Mauwake but intransitive in English:

aner-  `aim (at), refer (to)'

ikum-  `wonder (about)'

kerew-  `be angry (at)'

\ea%x302
\label{ex:x302}
\gll Wi  wia  amukar-emi  me  nefa  \textstyleEmphasizedVernacularWords{aner-a-m}. \\
      \\
\glt
\z

3p.UNM  3p.ACC  scold-SS.SIM  not  2s.ACC  refer.to-PA-1s  

`When I scolded them I didn't refer to you.'

\ea%x303
\label{ex:x303}
\gll Nefa  \textstyleEmphasizedVernacularWords{ikum-am-ika-iwkin}  nan  kerer-e-n. \\
      \\
\glt
\z

2s.ACC  wonder.about-SS.SIM-be-2/3p.DS  there  arrive-PA-2s

`As they were wondering about you, you arrived there.'

There are a a few verbs that require an undergoer object, but usually have recipient object as well. The verbs \textstyleStyleVernacularWordsItalic{ofakow}- `show, teach' and \textstyleStyleVernacularWordsItalic{maak}- `tell' are the most common of these.\footnote{The verb `send' is cross-linguistically typically ditransitive, but in Mauwake it requires the benefactive suffix in order to be able to take a second object.} 

\ea%x1838
\label{ex:x1838}
\gll Tunde  urera  Liisa  ame=ke  [\textstyleEmphasizedVernacularWords{epa}]\textsubscript{O}  [\textstyleEmphasizedVernacularWords{yia}]\textsubscript{O} \\
      \\
\glt
\z

Tuesday  afternoon  Liisa  ASSOC=CF  place  1p.ACC  

\textstyleEmphasizedVernacularWords{ofakowa-y}\textstyleEmphasizedVernacularWords{i}\textstyleEmphasizedVernacularWords{aw}\textstyleEmphasizedVernacularWords{-}\textstyleEmphasizedVernacularWords{e}\textstyleEmphasizedVernacularWords{-}\textstyleEmphasizedVernacularWords{mik}.

show-move.around-PA-1/3p

`On Tuesday afternoon Liisa and the othes showed us around the place.'

\ea%x943
\label{ex:x943}
\gll Nena  panewowa  pun  [\textstyleEmphasizedVernacularWords{wadol  opora}]\textsubscript{O} \textstyleEmphasizedVernacularWords{} [\textstyleEmphasizedVernacularWords{yia}]\textsubscript{O}  \textstyleEmphasizedVernacularWords{maak-i-n.} \\
      \\
\glt
\z

2s.GEN  old  also  lie  talk  1p.ACC  tell-Np-PR.2s

`You yourself -- an old person too! -- tell us lies.'

The verb \textstyleStyleVernacularWordsItalic{wu}- `put' requires both an undergoer object and a locative adverbial: 

\ea%x1837
\label{ex:x1837}
\gll [Sosora  nain]\textsubscript{O}  [pona-pa]\textsubscript{AdvP}  wu-a-mik. \\
      \\
\glt
\z

grass.skirt  that1  riverbank  put-PA-1/3p

`They put those grass skirts on the riverbank.'

The directional verbs (\sectref{sec:3.8.4.4.5}) could be treated as weakly transitive, in which case the goal \textstyleAcronymallcaps{NP}, which is never marked with the locative clitic -\textstyleStyleVernacularWordsxiiptItalic{pa}, could be a locative object. There are two main reasons against this analysis. When the goal of a directional verb is a personal pronoun, the dative case is used rather than the accusative:

\ea%x1870
\label{ex:x1870}
\gll {\dots}ona  wiawi  \textstyleEmphasizedVernacularWords{wiar}  ikiw-o-k. \\
      \\
\glt
\z

3s.GEN  3s/p.father  3.DAT  go-PA-3s

`{\dots}she went to her father.'

Also, if the directional verbs were considered weakly transitive and the the goal a locative object, the locative adverb \textstyleStyleVernacularWordsxiiptItalic{nan} `there' in the following clauses would be treated as a locative adverb in (\stepcounter{nx}{\thenx}) but as a locative object in (\stepcounter{nx}{\thenx}):

\ea%x1871
\label{ex:x1871}
\gll Kerer-ep  \textstyleEmphasizedVernacularWords{nan}  soomare-miaw-e-mik. \\
      \\
\glt
\z

arrive-SS.SEQ  there  walk-move.around-PA-1/3p

`They arrived and walked around there.'

\ea%x1872
\label{ex:x1872}
\gll Or-op  \textstyleEmphasizedVernacularWords{nan}  ikiw-ep  wia  uruf-a-k. \\
      \\
\glt
\z

descend-SS.SEQ  there  go-SS.SEQ  3p.ACC  see-PA-3s

`He went down and went there and saw them.'

Transitive clauses are discussed in \sectref{sec:5.3}.

\paragraph[Ambitransitive verbs]{Ambitransitive verbs}
\hypertarget{RefHeading20621935131865}{}
Although most verb roots in Mauwake are clearly transitive or intransitive, there are a few that are ambitransitive. Many of their English equivalents would be intransitive. Only the following roots have been found to be neutral with regard to transitivity. Of them \textstyleStyleVernacularWordsItalic{ofof}- and \textstyleStyleVernacularWordsItalic{taan}- are of the S=O type, where the intransitive subject is an undergoer; the others are of the S=A type, where the intransitive subject is an actor \citep[124]{Dixon2010b}.

ofof-  `shake'

taan-  `become full'; `fill (a place)'

karu-  `run'; `visit'

om(om)-  `cry'; `mourn (for)'

pepek er-  `be enough'; `suffice'

aakun-  `speak, talk'; `discuss'

\ea%x1827
\label{ex:x1827}
\gll Ifar(a)  makena  wulewul  \textstyleEmphasizedVernacularWords{ofof-i-ya}. \\
      \\
\glt
\z

vine  fruit  wulewul  shake-Np-PR.3s

`The vine fruit (called) \textstyleForeignWords{wulewul} shakes.'

\ea%x1826
\label{ex:x1826}
\gll Maa-ofofona  saarik  \textstyleEmphasizedVernacularWords{wia  ofof-a-k}. \\
      \\
\glt
\z

earthquake  like  3p.ACC  shake-PA-3s

`It shook them like an earthquake.'

\ea%x304
\label{ex:x304}
\gll Ifa  uruf-ap  baurar-ep  \textstyleEmphasizedVernacularWords{karu-or-o-mik}. \\
      \\
\glt
\z

snake  see-SS.SEQ  flee-SS.SEQ  run-descend-PA-1/3p  

`We saw a snake and fled and ran down (to the village).'

\ea%x305
\label{ex:x305}
\gll Epasia=pa  ik-omkun  me  \textstyleEmphasizedVernacularWords{efa}  \textstyleEmphasizedVernacularWords{karu-e-mik.} \\
      \\
\glt
\z

far=LOC  be-1s/p.DS  NEG  1s.ACC  run/visit-PA-1/3p

`When I lived far away, they didn't visit me.'

\ea%x1059
\label{ex:x1059}
\gll En-em-ika-eya  ona  wiamun=ke  uruf-ap  \\
      \\
\glt
\z

eat-SS.SIM-be-2/3s.DS  3s.GEN  younger.sibling=CF  see-SS.SEQ

\textstyleEmphasizedVernacularWords{om-o-k}.

cry-PA-3s

`When he was eating it his younger sibling saw it/him and cried.'

\ea%x1060
\label{ex:x1060}
\gll \textstyleEmphasizedVernacularWords{Efa  om-em-ik-eya}  epa  wiim-o-k. \\
      \\
\glt
\z

1s.ACC  cry-SS.SIM-be-2/3s.DS  place  dawn-PA-3s

`While she was mourning for me it dawned.'

The subject of the adjunct plus verb \textstyleStyleVernacularWordsItalic{pepek er}- `be enough, suffice' is typically inanimate, whereas the object, when there is one, is usually human.  

\ea%x1058
\label{ex:x1058}
\gll Kemuka  nain  \textstyleEmphasizedVernacularWords{pepek  er-eya}  onak  ona  \\
      \\
\glt
\z

string  that1  enough  go-2/3s.DS  3s/p.mother  3s.GEN  

wiar  puuk-a-k.

3.DAT  cut-PA-3s

`When the string was (long) enough, their mother herself cut it.'

\ea%x306
\label{ex:x306}
\gll \textstyleEmphasizedVernacularWords{Wia}  \textstyleEmphasizedVernacularWords{pepek  er-a-k}. \\
      \\
\glt
\z

3p.ACC  enough  come/go-PA-3s  

`It was enough for them.'

\paragraph[Object cross-referencing verbs]{Object cross-referencing verbs}
\hypertarget{RefHeading20641935131865}{}
One feature very common to a small group of verbs in the Trans-New Guinea languages is that the ``\textstyleBibliogCitationAAAstyleChar{verb stem undergoes changes according to the person of the object or beneficiary}'' \citep[62]{Wurm1982}.\footnote{Wurm actually seems to be referring to \textit{recipient} rather than beneficiary, as `give' is the most common of these verbs, and the verb stem changes according to the recipient.} In Mauwake this group consists of only five members. 

I call these verbs object cross-referencing because, besides marking the subject with a suffix like all other verbs do, they also \textstyleEmphasizedWords{\textsc{obligatorily}} mark the object in the verb root. What has clearly been a prefix\footnote{Phonetically this prefix is closer to the unmarked pronouns than the accusative pronouns.} earlier has been grammaticalized as part of the verb itself: there is no neutral root that would not be linked to any particular person.\footnote{When a ``neutral'' form is required, the third person singular is used.} In this respect these verbs differ from all the other verbs in Mauwake. In the case of `give' the verb root \textstyleStyleVernacularWordsItalic{i}- has assimilated into the prefix, so currently the person marking of the recipient object is the only root that there is. Four of the object cross-referencing verbs are listed below.

`give'  `feed'  `follow'  `shoot'

yi-  `give me'  enak  `feed me'  yook-  follow me'  enar-  `shoot me'

ni-  give you  nenak-  feed  you  nook-  follow you  nenar-  shoot you

iw-  give him  onak-  feed him  ook-  follow him  war-  shoot him

yi-  give us  yienak-  feed us  yiok-  follow us  yiar-  shoot us

ni-  give you    nienak-  feed you  niok-  follow you  niar-  shoot you

wi-  give them  wienak-  feed them  wiok-  follow them  wiar-  shoot them

\ea%x334
\label{ex:x334}
\gll Maa  fain  me  \textstyleEmphasizedVernacularWords{iw}-o-k. \\
      \\
\glt
\z

thing  this   not  give.him/her-PA-3s  

`He did not give this thing to him/her.'

\ea%x335
\label{ex:x335}
\gll Waaya  pun  \textstyleEmphasizedVernacularWords{enak}-e-mik. \\
      \\
\glt
\z

pig  too  feed.me-PA-1/3p  

`They also gave me pork to eat.'

\ea%x336
\label{ex:x336}
\gll Amia=iya  \textstyleEmphasizedVernacularWords{nenar}-e-mik=i? \\
      \\
\glt
\z

bow=COM  shoot.you-PA-1/3p=QM

`Did they shoot you with a gun?'

The cross-referenced objects are semantically quite different.  In the verbs \textstyleStyleVernacularWordsItalic{iw}- `give (him)' and \textstyleStyleVernacularWordsItalic{onak}- `feed (him)' it is the recipient,\footnote{\textstyleFootnoteBaseChar{\textit{onak-}} `feed (him)' requires a food term as the undergoer object, so a better translation, but longer, would be `give him (something) to eat'.} in \textstyleStyleVernacularWordsItalic{war}- `shoot (him)' and \textstyleStyleVernacularWordsItalic{ook}- `follow (him)' the undergoer. The verb \textstyleStyleVernacularWordsItalic{wionar}-\footnote{A possible origin for this is \textit{PRON+onaiya+ar}- `become together-with PRON' (Kwan, p.c.)} `hide among (them)' is a special case in two ways: the cross-referenced argument `among a group' is quite untypical as a verbal argument; and only plural forms of this verb can be used because of semantic restrictions. 

yionar-  `hide among us'

nionar-  `hide among you (pl)'

wionar-  `hide among them'

\ea%x337
\label{ex:x337}
\gll Wi  \textstyleEmphasizedVernacularWords{wionar}-ep  pok-ap  ik-ua. \\
      \\
\glt
\z

3p.UNM  hide.among.them-SS.SEQ  sit.down-SS.SEQ  be-PA.3s  

`He sat hiding among them.'

Maia does not have any verbs behaving like this \citep{Hardin2002}, and Hepner only reports one for Bargam: \textstyleForeignWords{\nobreakdash-g}  `give' (2002:87). Usan has three verbs involving a stem change of this kind: \textstyleForeignWords{ut\^ab} `give (him)', \textstyleForeignWords{w\^ab} `shoot' and \textstyleForeignWords{w\^aramb} `hit'\citep[44]{Reesink1987}. \textstyleForeignWords{Ut\^ab}, which coreferences the recipient, has quite strict co-occurrence restrictions with other arguments or even with peripheral elements in the same clause (ibid. 129-30). 

Unlike Usan, in Mauwake the clauses with object cross-referencing verbs can easily have a locative or instrument phrase, and the verb itself can take a benefactive suffix. A sentence like (\stepcounter{nx}{\thenx}) would be possible for instance when sending money to people travelling in the same vehicle as the addressee.  

\ea%x338
\label{ex:x338}
\gll Miiw-aasa=pa  \textstyleEmphasizedVernacularWords{wi-om}-\textstyleEmphasizedVernacularWords{e}. \\
      \\
\glt
\z

land-canoe=LOC  give.them-BEN-BNFY1.IMP.2s  

`Give it to them for me in the car.'

\subsubsection[Valence changes]{Valence changes}
\hypertarget{RefHeading20661935131865}{}
The term \textstyleEmphasizedWords{\textsc{valence}} refers to the number of arguments that have a grammatical relation with the verb. As was mentioned above, almost all of the verb roots in Mauwake have a basic valence of one or two: they take either a subject only (intransitive verbs \sectref{sec:3.8.4.2.1}) or a subject and an object (transitive verbs \sectref{sec:3.8.4.2.2}) as their arguments. There are some ways to change the valence of verbs, even if strategies like passivization and dative shift are not possible in Mauwake.  The valence is increased, when an intransitive verb is made into transitive or a transitive verb into causative with the addition of an causative suffix, or when a benefactive suffix is added to a verb. There are no processes to decrease the syntactic valence of a verb. The \textstyleEmphasizedWords{\textsc{semantic}} valence is decreased when the object of a transitive verb is a reflexive or reciprocal pronoun, since the subject and object have the same referent(s). Subject demotion is another way to decrease the semantic valence. 

\paragraph[Causatives]{Causatives}
\hypertarget{RefHeading20681935131865}{}
The causative always increases the number of arguments a verb can take: the subject of an intransitive verb becomes the object of a transitive verb, and a new subject is added. The causative suffix -\textstyleStyleVernacularWordsItalic{ow} was described above in \sectref{sec:3.8.2.3.1}.  In most cases the meaning of a causative is `to cause someone or something to do something'. The caused `doing' is usually \textstyleEmphasizedWords{\textsc{not} }agentive. 

\ea%x997
\label{ex:x997}
\gll Iwera  nainiw  kaken  iimar-e-k. (Intransitive) \\
      \\
\glt
\z

coconut  again  straight  stand-PA-3s

`The coconut palm stood straight again.'

\ea%x998
\label{ex:x998}
\gll [Eka  napia]\textsubscript{O}  koor  miira=pa  iimar-\textstyleEmphasizedVernacularWords{ow}-a-mik. \\
      \\
\glt
\z

water  bamboo  house  face=LOC  stand-CAUS-PA-1/3p

`We made the bamboo water containers stand in front of the house.'

\ea%x992
\label{ex:x992}
\gll [Wiowa  erup]\textsubscript{O}  ar-\textstyleEmphasizedVernacularWords{ow}-amkun  um-o-k. \\
      \\
\glt
\z

spear  two  become-CAUS-1s/p.DS  die-PA-3s

`I speared it a second time and it (=the pig) died.'(Lit: `I caused a spear to become two and it died.')

The mental state of being angry is expressed via a verb in Mauwake (\stepcounter{nx}{\thenx}), and it can take the causative suffix (\stepcounter{nx}{\thenx}).

\ea%x993
\label{ex:x993}
\gll Kema  bagiwir-a-m. \\
      \\
\glt
\z

liver  be.angry-PA-1s

`I was angry.'

\ea%x994
\label{ex:x994}
\gll Yo  kema  [efa]\textsubscript{O}  bagiwir-\textstyleEmphasizedVernacularWords{ow}-a-n,  yaa! \\
      \\
\glt
\z

1s.UNM  liver  1s.ACC  be.angry-CAUS-PA-2s  INTJ

`Boy, have you made me angry!'

In some cases the causative suffix acts simply as a transitiviser. The subject in (\stepcounter{nx}{\thenx}) does not actually cause the children to grow. Also in this case the suffix  increases the valency of the verb: \textstyleStyleVernacularWordsItalic{arim}- `grow' in (\stepcounter{nx}{\thenx}) is intransitive, but \textstyleStyleVernacularWordsItalic{arimow}- in (\stepcounter{nx}{\thenx}) is transitive and takes an object.

\ea%x995
\label{ex:x995}
\gll Aakisa  arim-o-n,  aakisa  muew-o-n. \\
      \\
\glt
\z

now  grow-PA-2s  now  marry-PA-2s

`Now you have grown, now you have married.'

\ea%x996
\label{ex:x996}
\gll No  nena  maa  fariar-ep  [muuka  nain]\textsubscript{O} \\
      \\
\glt
\z

2s.UNM  2s.GEN  food  abstain-SS.SEQ  son  that1  

arim-\textstyleEmphasizedVernacularWords{ow}-e.

grow-CAUS-IMP.2s

`You yourself have to abstain from (certain) food(s) and bring the son up.'

When the causative suffix is added to the intransitive verb \textstyleStyleVernacularWordsItalic{sail}- `(tell a) lie', its meaning changes into `lie to someone', `cheat'. 

\ea%x448
\label{ex:x448}
\gll Opor(a)  makena  ma-i-yem,  me  [nia]\textsubscript{O}  sail-\textstyleEmphasizedVernacularWords{ow}-iyem. \\
      \\
\glt
\z

talk  true  say-Np-PR.1s  not  2p.ACC  lie-CAUS-PR.1s

`I am telling the truth, I am not cheating you.'

Bring-prefixes (\sectref{sec:3.8.2.4.2}) are another causative strategy, used only with the directional verbs (\sectref{sec:3.8.4.4.5}) and a couple of other motion verbs. The subject of the verb causes the object to move by undertaking the transfer himself/herself.

\ea%x999
\label{ex:x999}
\gll Maa  unowa  ifer  aasa=ke  \textstyleEmphasizedVernacularWords{p}-urup-o-k. \\
      \\
\glt
\z

thing  many  sea  canoe=CF  BPx-ascend-PA-3s

`A lot of things were brought/taken up by ships.'

\ea%x1001
\label{ex:x1001}
\gll O  mua  imen-ap=na  feeke  wia  \textstyleEmphasizedVernacularWords{p}-ekap-eka. \\
      \\
\glt
\z

3s.UNM  man  find-SS.SEQ=TP  here.CF  3p.ACC  BPx-come-IMP.2p

`If/when you find a/any man, bring them/him here.'

\ea%x1000
\label{ex:x1000}
\gll Gomi  kawus  \textstyleEmphasizedVernacularWords{p}-irapar-i-ya. \\
      \\
\glt
\z

east.wind  smoke  BPx-move.to.and.fro-Np-PR.3s

`The east wind moves the smoke around.'

Forming a causative from an agentive verb (\textstyleEmphasizedWords{\textsc{inducive causative}}, Talmy 2007:112) is not done morphologically with an affix but syntactically with a verbal construction involving the nominalized form of the main verb and \textstyleStyleVernacularWordsItalic{suuw}- `push' as the causative auxiliary (5.7.1). 

\ea%x1003
\label{ex:x1003}
\gll O  uruf-ap  op-ap  Yeesus  nomokowa  moke \\
      \\
\glt
\z

3s.UNM  see-SS.SEQ  hold-SS.SEQ  Jesus  tree  slanting  

\textstyleEmphasizedVernacularWords{akua-aaw-om-owa  suuw-a-mik}.

shoulder-take-BEN-NMZ  push-PA-1/3p

`They saw him and took hold of him, and made him carry Jesus' cross on his shoulder.'

\ea%x1002
\label{ex:x1002}
\gll Sira  enuma  \textstyleEmphasizedVernacularWords{ook-owa  nia  suuw-i-mik}. \\
      \\
\glt
\z

custom  new  follow-NMZ  2p.ACC  push-Np-PR.1/3p

`They make you follow new customs/ways.'

In the following examples the three different causative strategies have been applied to the same verb \textstyleStyleVernacularWordsItalic{ikiw}- `go', and in all of them the patient is [+human]. In (\stepcounter{nx}{\thenx}) and (\stepcounter{nx}{\thenx}) the object of the causative verb has no influence on what happens to him/her, but in (\stepcounter{nx}{\thenx}) the object of the inducive causative is active and becomes the actor of the verb resulting from the causation. 

\ea%x1016
\label{ex:x1016}
\gll Ipamsika  mua=ke  \textstyleEmphasizedVernacularWords{ikiw-ow}\textstyleEmphasizedVernacularWords{-a-k}. \\
      \\
\glt
\z

nail  man=CF  go-CAUS-PA-3s  

`A sorcerer (lit: nail man) killed him (lit: caused him to go).'

\ea%x1829
\label{ex:x1829}
\gll Kes  tepak=pa  wu-ap  \textstyleEmphasizedVernacularWords{p-ikiw-e-mik}. \\
      \\
\glt
\z

coffin  inside=LOC  put-SS.SEQ  Bpx-go-PA-1/3p

`They put him inside the coffin and took him (away).'

\ea%x1873
\label{ex:x1873}
\gll Yo  mua  oko  \textstyleEmphasizedVernacularWords{ikiw-owa  suuw-amkun}  ikiw-i-non. \\
      \\
\glt
\z

1s.UNM  man  other  go-NMZ  push-1s/p.DS  go-Np-FU.3s

`I make a man go and he goes.'

\paragraph[  Benefactive]{  Benefactive}
\hypertarget{RefHeading20701935131865}{}
The benefactive form of a verb (\sectref{sec:3.8.2.3.3}) is used when an action is done \textstyleEmphasizedWords{\textsc{for} }someone, for their benefit (\stepcounter{nx}{\thenx}), or in some cases for their detriment (\stepcounter{nx}{\thenx}). With the addition of the benefactive suffix to the verb, the beneficiary  becomes an obligatory argument. The beneficiary is always animate, and usually human. 

\ea%x1004
\label{ex:x1004}
\gll Wi  owow  mua=ke  wilkar  wia \\
      \\
\glt
\z

3p.UNM  village  man=CF  cart  3p.ACC

muf-em-ik-\textstyleEmphasizedVernacularWords{om}-a-mik.

pull-SS.SIM-be-BEN-BNFY2.PA-1/3p

`The village men kept pulling carts for them.'

\ea%x1005
\label{ex:x1005}
\gll Epia  wilin-owa  uruf-ap  bom  yia \\
      \\
\glt
\z

fire(wood)  shine-NMZ  see-SS.SEQ  bomb  1p.ACC  

fuurk-\textstyleEmphasizedVernacularWords{om}-i-kuan.

throw-BEN-Np-FU.3p

`When they see the light from the fire(s) they will throw bombs at us.'

More than one valency-increasing strategy can be applied to a verb simultaneously. In both (\stepcounter{nx}{\thenx}) and (\stepcounter{nx}{\thenx}) the valency of the verb increases from one to three: besides the subject of the original verb, the derived verbs also have both an object and a beneficiary.

\ea%x1007
\label{ex:x1007}
\gll Koor  poka  iimar-\textstyleEmphasizedVernacularWords{ow}-\textstyleEmphasizedVernacularWords{om}-e. \\
      \\
\glt
\z

house  post  stand.up-CAUS-BEN-BNFY1.IMP.2s

`Stand up the house posts for me.'

\ea%x1008
\label{ex:x1008}
\gll Ona  soomia  marew-eya  \textstyleEmphasizedVernacularWords{amap}-ep-\textstyleEmphasizedVernacularWords{om}-a-m. \\
      \\
\glt
\z

3s.GEN  spoon  no(ne)-2/3s.DS  BPx-come-BEN-BNFY2.PA-1s

`She has/had no spoons of her own, so I brought them for her.'

\paragraph[Decreasing semantic valence]{Decreasing semantic valence}
\hypertarget{RefHeading20721935131865}{}
There are no morphological means in Mauwake for decreasing syntactic valence. A verb that is inherently reflexive, like \textstyleStyleVernacularWordsItalic{yaki}- `wash oneself', is intransitive. But the semantic valence of transitive verbs is decreased when they are made either reflexive or reciprocal. Syntactically the reflexive/reciprocal pronoun is an object, but the pronoun refers to the same referent(s) as the subject.  

\ea%x1834
\label{ex:x1834}
\gll Birin-ep  nomokowa  iinan  akena  ikiw-ep  wame \\
      \\
\glt
\z

fly-SS.SEQ  tree  top  very  go-SS.SEQ  3s.REFL  

pipilim-ep  aakun-em-ika-i-non.

hide-SS.SEQ  speak-SS.SIM-be-Np-FU.3s

`It will fly and hide (itself) in the very top of a tree and keep making noise.'

\ea%x1835
\label{ex:x1835}
\gll Osaiwa  aalbok  ikos  uf-owa  na-ep \\
      \\
\glt
\z

bird.of.paradise  black.cuckoo-shrike  together  dance-NMZ  say-SS.SEQ  

ofa  wiam  if-e-mik.

colour  3p.REFL  paint-PA-1/3p

`A bird of paradise and a black cuckoo-shrike wanted to dance together and painted each other with colour.'

A common valence-decreasing device in many languages is the passive voice, which demotes or deletes the subject. In Mauwake verbs there is no passive voice. The standard way of demoting the subject is to have the verb in third person plural form and leave the subject \textstyleAcronymallcaps{NP} unexpressed.\footnote{Cf. the English impersonal ``they'': \textit{They say it is going to be cold tomorrow}.} None of the arguments change their syntactic function. The example (\stepcounter{nx}{\thenx}) comes from a story where the main point was that the people responsible for the fire were never found, and it was not known if only one person was involved or many. 

\ea%x1009
\label{ex:x1009}
\gll Fikera  ikum  \textstyleEmphasizedVernacularWords{kuum-e-mik}  nain  ma-i-yem. \\
      \\
\glt
\z

kunai.grass  illicitly  burn-PA-1/3p  that1  say-Np-PR.1s

`I tell about when the kunai grass was burned (by arson).'

\ea%x1010
\label{ex:x1010}
\gll Nefa  \textstyleEmphasizedVernacularWords{war-iwkin}  naap  ma-e. \\
      \\
\glt
\z

2s.ACC  shoot-2/3p.DS  thus  say-IMP.2s

`If/when you are shot, then say like that.' (Or: `If they shoot you, then say like that.')

Another strategy to demote the subject is to use the same-subject sequential form of the main verb and the auxiliary \textstyleStyleVernacularWordsItalic{ik}- `be' agreeing with the object of the verb. This can only be used when the end result is a state. 

\ea%x1011
\label{ex:x1011}
\gll Nomokowa  puuk-ap  ik-ua. \\
      \\
\glt
\z

tree  cut-SS.SEQ  be-PA.3s

`The tree is cut.'

\subsubsection[Semantically based verb classes ]{Semantically based verb classes} 
\hypertarget{RefHeading20741935131865}{}
Even though the following classification is based on semantic characteristics of the verbs, the verbs within the resulting groups tend to have similarities in their syntactic behaviour as well.

\paragraph[Stative/existential verb ik- ]{Stative/existential verb \textit{ik}-} 
\hypertarget{RefHeading20761935131865}{}
The basic meaning of the stative verb \textstyleStyleVernacularWordsItalic{ik}(\textstyleStyleVernacularWordsItalic{a})- is `be'. The vowel /a/ gets deleted elsewhere except in the present tense and the medial different-subject non-first plural form; in the corresponding singular form the vowel may be optionally deleted (\stepcounter{nx}{\thenx}). 

\ea%x257
\label{ex:x257}
\gll Nan  mukuna=pa  \textstyleEmphasizedVernacularWords{ik(a)-eya}  o  nan  samor  aaw-o-k. \\
      \\
\glt
\z

there  fire=LOC  be-2/3s.DS  3s.UNM  there  badly  get-PA-3s

`They (=bananas) were there on the fire and he really got bad there.'

Like intransitive verbs, it may form a complete clause by itself. Example (\stepcounter{nx}{\thenx}) is from a situation where the speaker was in a plane for the first time, refused to eat and declined any help offered to him.

\ea%x1455
\label{ex:x1455}
\gll \textstyleEmphasizedVernacularWords{Ika-i-nen}. \\
      \\
\glt
\z

be-Np-FU.1s

`I will just be (like this).'

Often it is used for `be/live (somewhere)', and in this use it naturally co-occurs with a locative adverbial:

\ea%x497
\label{ex:x497}
\gll I  naap  koora=pa  \textstyleEmphasizedVernacularWords{ik-e-mik}. \\
      \\
\glt
\z

1p.UNM  thus  house=LOC  be-PA-1/3p

`We were in the house like that.'

Together with the dative pronouns \textstyleStyleVernacularWordsItalic{ik}- is used to form possessive constructions (\stepcounter{nx}{\thenx}) (\sectref{sec:3.5.5}, 5.5.2). 

\ea%x258
\label{ex:x258}
\gll Yo  waaya  arow  \textstyleEmphasizedVernacularWords{efar  ik-ua.} \\
      \\
\glt
\z

1s.UNM  pig  three  1s.DAT  be-PA.3s  

`I have three pigs.'

The function of \textstyleStyleVernacularWordsItalic{ik}- as a copular verb is very restricted. In equative or descriptive clauses it is normally not used in the present tense finite form, but in the past (\stepcounter{nx}{\thenx}) and future (\stepcounter{nx}{\thenx}) tenses it is employed. It could be said, following Giv\'on, that in these clauses its primary function is to be the carrier of the tense (1984:92). 

\ea%x259
\label{ex:x259}
\gll Yo  um-ep  ik-owa  saarik  \textstyleEmphasizedVernacularWords{ik-e-m}. \\
      \\
\glt
\z

1s.UNM  die-SS.SEQ  be-NMZ  like  be-PA-1s

`I was like dead.'

\ea%x1070
\label{ex:x1070}
\gll Ikoka  maa  marew,  eliw  manek=iw  \textstyleEmphasizedVernacularWords{ika-i-nan}. \\
      \\
\glt
\z

later  thing  none  well  big=LIM  be-Np-FU.2s

`Later there will be no problem, you will just be very well.'

In Mauwake it can also be used when the non-verbal predicate is understood to be transitory (\stepcounter{nx}{\thenx}) rather than stable over time:

\ea%x499
\label{ex:x499}
\gll No  kamenap  \textstyleEmphasizedVernacularWords{ika-i-n}? \\
      \\
\glt
\z

2s.UNM  how  be-Np-PR.2s

`How are you?' 

The verb \textstyleStyleVernacularWordsItalic{ik}- `be' is in a class of its own for several reasons. Its morphology is irregular, and so are the semantics of some of its morphology. In (\stepcounter{nx}{\thenx}) the past tense and the person/number marker in the third person singular form are merged into one portmanteau morpheme. An alternative form for the different-subject first person form \textstyleStyleVernacularWordsItalic{ikemkun} is \textstyleStyleVernacularWordsItalic{ikomkun} (\stepcounter{nx}{\thenx}). The same-subject medial form is \textstyleStyleVernacularWordsItalic{ikok} (\stepcounter{nx}{\thenx}), not *\textstyleStyleVernacularWordsItalic{ikep}\textstyleEmphasizedVernacularWords{} and\textstyleEmphasizedVernacularWords{} *\textstyleStyleVernacularWordsItalic{ikemi}\textstyleEmphasizedVernacularWords{\textmd{\textit{.}}}\footnote{\textstyleFootnoteBaseChar{\textit{ikep}} and \textstyleFootnoteBaseChar{\textit{ikemi}} are the same subject medial forms of the homophonous verb \textstyleFootnoteBaseChar{\textit{ik-}} `roast'.} There is no formal differentiation between a simultaneous (\stepcounter{nx}{\thenx}) and a sequential (\stepcounter{nx}{\thenx}) form in the same-subject medial verb. 

\ea%x1931
\label{ex:x1931}
\gll Siowa  nain  kakalt-am-\textstyleEmphasizedVernacularWords{ik}\textstyleEmphasizedVernacularWords{-}\textstyleEmphasizedVernacularWords{emkun}  arim-o-k. \\
      \\
\glt
\z

dog  that1  look.after-SS.SIM-be-1s/p.DS  grow-PA-3s

`I was looking after the dog and it grew.'

\ea%x260
\label{ex:x260}
\gll Naap  \textstyleEmphasizedVernacularWords{ik-ok}  uruf-am-ika-iwkin  wia. \\
      \\
\glt
\z

thus  be-SS  see-SS.SIM-be-2/3p.DS  no

`As he was/stayed like that they were watching him (but) no (=he didn't get better).' 

\ea%x262
\label{ex:x262}
\gll Owowa  ekap-o-k,  amia  mua=pa  \textstyleEmphasizedVernacularWords{ik-ok}. \\
      \\
\glt
\z

village  come-PA-3s  bow  man=LOC  be-SS  

`He came to the village, having been in the police (force).'

It also differs from ordinary intransitive verbs in that in a verb+auxiliary construction it cannot be the main verb, but can be used as the aspectual auxiliary (\stepcounter{nx}{\thenx}) (see also \sectref{sec:3.8.4.5}). But it is like other intransitive verbs in that it can take an causative suffix (\stepcounter{nx}{\thenx}).\footnote{Reesink notes that in Usan the corresponding verb \textstyleFootnoteBaseChar{\textit{igo}} `be' cannot occur with the causative suffix (1987:142). In Mauwake there is no similar restriction.} 

\ea%x261
\label{ex:x261}
\gll Nomokowa  war-ep  miiwa=pa  \textstyleEmphasizedVernacularWords{ik-ow-a-mik.} \\
      \\
\glt
\z

tree  cut-SS.SEQ  ground=LOC  be-CAUS-PA-1/3p

`We cut trees and laid them on the ground' 

The tense distinction is partly neutralized: the past tense form is used for past (\textstyleParagraphChari{\stepcounter{nx}{\thenx}}) and present (\textstyleParagraphChari{\stepcounter{nx}{\thenx}}). The present tense form is not very common and is mainly used for less time-stable situations (\stepcounter{nx}{\thenx}), (\stepcounter{nx}{\thenx}), or to replace the missing  continuous aspect form (\textstyleParagraphChari{\stepcounter{nx}{\thenx}}). The verb \textstyleStyleVernacularWordsItalic{ik}- is used as the regular continuous aspect auxiliary (\sectref{sec:4.4.1}), and as a main verb \textstyleStyleVernacularWordsItalic{ik}- `be' cannot take this auxiliary. 

\ea%x263
\label{ex:x263}
\gll Yo  unan  koka=pa  \textstyleEmphasizedVernacularWords{ik-e-m}. \\
      \\
\glt
\z

1s.UNM  yesterday  jungle=LOC  be-PA-1s  

`Yesterday I was in the jungle.' 

\ea%x264
\label{ex:x264}
\gll Ni  kaaneke  \textstyleEmphasizedVernacularWords{ik-e-man}  oo,  ni  ekap-omak-eka  oo! \\
      \\
\glt
\z

2p.UNM  where  be-PA-2p  oh  2p.UNM  come-DISTR-IMP.2p  oh  

`Wherever you are, come!'

\ea%x1028
\label{ex:x1028}
\gll Mesa  asia  fiker  gone=pa  \textstyleEmphasizedVernacularWords{ika-i-ya}  nain \\
      \\
\glt
\z

winged.bean  wild  kunai.grass  middle=LOC  be-Np-PR.3s  that1  

aaw-em-ik-e-m.

take-SS.SIM-be-PA-1s

`I was picking wild winged bean that was (lit: is) in the middle of the kunai grass.'

\ea%x265
\label{ex:x265}
\gll Yo  nan  \textstyleEmphasizedVernacularWords{ika-i-yem}  nain  yo  nia  asip-i-yem,  {\dots} \\
      \\
\glt
\z

1s.UNM  there  be-Np-PR.1s  that1  1s.UNM  2p.ACC  help-Np-PR.1s

`Now that I am living there I help you, {\dots}'

The verb \textstyleStyleVernacularWordsItalic{ik}-  mainly functions in intransitive clauses, but it is also needed as a copula for those cases where a non-verbal predicate is in a non-present tense. 

\ea%x969
\label{ex:x969}
\gll O  ikoka  somek  mua  maneka  \textstyleEmphasizedVernacularWords{ika-i-non}. \\
      \\
\glt
\z

3p.UNM  later  song  man  big  be-Np-FU.3s

`He will later be the headmaster.'

An equative or descriptive medial clause requires \textstyleStyleVernacularWordsItalic{ik}- as a copula regardless of the tense of the final verb (\stepcounter{nx}{\thenx}).

\ea%x498
\label{ex:x498}
\gll Koora  naap  \textstyleEmphasizedVernacularWords{ik-eya}  uruf-i-mik. \\
      \\
\glt
\z

house  thus  be-2/3s.DS  see-Np-PR.1p

`We see the house as it is like that.'

\paragraph[Position-taking verbs]{Position-taking verbs}
\hypertarget{RefHeading20781935131865}{}
The three position-taking verbs are among the most frequently used verbs in Mauwake: \textstyleStyleVernacularWordsItalic{pok}- `sit down', \textstyleStyleVernacularWordsItalic{iimar}- `stand up' and \textstyleStyleVernacularWordsItalic{in}- `lie down/ fall asleep'. They are essentially punctiliar verbs with an inceptive meaning (\stepcounter{nx}{\thenx}), but they are most typically used in the same-subject sequential form together with the auxiliary \textstyleStyleVernacularWordsItalic{ik}- (\sectref{sec:3.8.4.5}) to convey stative meaning: `sit' (\stepcounter{nx}{\thenx}), `stand', and `lie/sleep'.

\ea%x273
\label{ex:x273}
\gll Kokom-ar-eya  \textstyleEmphasizedVernacularWords{in-e-mik}. \\
      \\
\glt
\z

darkness-INCH-2/3s.DS  lie.down-PA-1/3p  

`When it got dark we went to bed.'

\ea%x274
\label{ex:x274}
\gll Ona  koora=pa  arew-ap  \textstyleEmphasizedVernacularWords{pok-ap  ik-e-mik}. \\
      \\
\glt
\z

3s.GEN  house=LOC  wait-SS.SEQ  sit.down-SS.SEQ  be-PA-1/3p

`We sat and waited (lit: waited and sat) in his house.'

The verb \textstyleStyleVernacularWordsItalic{pok}- is occasionally used without the auxiliary to mean `sit': 

\ea%x1824
\label{ex:x1824}
\gll Neek(e)  \textstyleEmphasizedVernacularWords{pok-aka}. \\
      \\
\glt
\z

there  sit-IMP.2p

`Sit there/Keep sitting there.' (Commonly used as a conversational ``filler'' for people that are already sitting, when there is a lull in the conversation.)

The continuous aspect form of the position-taking verbs is not used with progressive meaning, only with the meaning `habitual' (\sectref{sec:3.8.5.1.1.2}). 

\ea%x275
\label{ex:x275}
\gll Irak-ow  epa=pa  koka=pa  \textstyleEmphasizedVernacularWords{in-em-ik-e-mik}. \\
      \\
\glt
\z

fight-NMZ  time=LOC  jungle=LOC  lie.down-SS.SIM-be-PA-1/3p  

`During the war we used to sleep in the jungle.'

\paragraph[Location verbs]{Location verbs}
\hypertarget{RefHeading20801935131865}{}
The two verbs that have been verbalized from the demonstrative adverbs \textstyleStyleVernacularWordsItalic{fan} `here' and \textstyleStyleVernacularWordsItalic{nan} `there' (\sectref{sec:3.8.2.2.1}), are very restricted in their use. The original meaning of the verbs must refer to arrival at some place, but since they are only used in the past tense, they currently tend to indicate presence at a place rather than movement.\footnote{This may indicate that the past tense used to encode perfectivity (Malcolm Ross, p.c.)} They can even be used with immobile objects (\stepcounter{nx}{\thenx}). 

\ea%x1270
\label{ex:x1270}
\gll No  ikiw-e,  irak-owa  maneka  \textstyleEmphasizedVernacularWords{fan}\textstyleEmphasizedVernacularWords{-}\textstyleEmphasizedVernacularWords{e}\textstyleEmphasizedVernacularWords{-}\textstyleEmphasizedVernacularWords{k}  a. \\
      \\
\glt
\z

2s.UNM  go-IMP.2s  fight-NMZ  big  here-PA-3s  INTJ

`Go (home), the big war is here.'

\ea%x1271
\label{ex:x1271}
\gll Aakisa  i  \textstyleEmphasizedVernacularWords{fan-e-mik}. \\
      \\
\glt
\z

Now  1p.UNM  here-PA-1/3p  

`Now we are / have come here.'

\ea%x1272
\label{ex:x1272}
\gll No  niawi  akena  \textstyleEmphasizedVernacularWords{nan-e-k}. \\
      \\
\glt
\z

2s.UNM  2s/p.father  true  there-PA-3s

`Your real father is there.'

\ea%x1276
\label{ex:x1276}
\gll Aa,  o  koora  \textstyleEmphasizedVernacularWords{fan-e-k}  a. \\
      \\
\glt
\z

INTJ  3s.UNM  house  here-PA-3s  INTJ

`Ah, his house is here.'

\paragraph[Resultative verbs]{Resultative verbs}
\hypertarget{RefHeading20821935131865}{}
The resultative verbs with the meaning `become' are another small group of intransitive verbs. Besides the semantic similarity they also share the syntactic characteristic that, in addition to the subject, they require another argument expressing the result of change. This other obligatory argument is a noun with the verbs \textstyleStyleVernacularWordsItalic{ar}- `become' and \textstyleStyleVernacularWordsItalic{puuk}- `change into',\footnote{This verb is homonymous with the transitive verb \textstyleFootnoteBaseChar{\textit{puuk-}} `cut'. They may be historically related, but synchronically the meanings are quite different.} and a colour adjective with the verb \textstyleStyleVernacularWordsItalic{kir}- `turn'. 

\ea%x276
\label{ex:x276}
\gll Takira  arim-ep  mua  \textstyleEmphasizedVernacularWords{ar-e-k}. \\
      \\
\glt
\z

boy  grow-SS.SEQ  man  become-PA-3s  

`The boy grew and became a man.'

\ea%x277
\label{ex:x277}
\gll Emeria  nain  afa  \textstyleEmphasizedVernacularWords{ar-e-mik}. \\
      \\
\glt
\z

woman  that1  flyng.fox  become-PA-1/3p  

`Those women became flying foxes.'

\ea%x278
\label{ex:x278}
\gll Inasin  mua  ifa  \textstyleEmphasizedVernacularWords{puuk-ap}  solon-ep  {\dots} \\
      \\
\glt
\z

spirit  man  snake  change.into-SS.SEQ  glide-SS.SEQ  

`The spirit man changed into a snake, glided and {\dots}'

\ea%x279
\label{ex:x279}
\gll Oona  kia  \textstyleEmphasizedVernacularWords{kir-em-ik-eya}  uruf-ap  ma-e-k  {\dots} \\
      \\
\glt
\z

bone  white  turn-SS.SIM-be-2/3s.DS  see-SS.SEQ  say-PA-3s  

`She saw that the bones were turning white and said, {\dots}'

The verb \textstyleStyleVernacularWordsItalic{ar}- is mostly used when the subject stays essentially the same but undergoes some change (\stepcounter{nx}{\thenx}). However, it can also be used when the subject  changes into something else (\stepcounter{nx}{\thenx}). The verb \textstyleStyleVernacularWordsItalic{puuk}- is only used in the latter context (\stepcounter{nx}{\thenx}), and it is always an intentional action. It is most common in traditional stories where spirits change into various inanimate things or animate beings. The verb \textstyleStyleVernacularWordsItalic{kir}- is used with most colour terms (\stepcounter{nx}{\thenx}), but for `black' there is a separate verb formed with the inchoative suffix \nobreakdash-\textstyleStyleVernacularWordsItalic{ar}\textstyleEmphasizedVernacularWords{} : \textstyleStyleVernacularWordsItalic{sepenar}-\footnote{This is related to the adjective \textstyleFootnoteBaseChar{\textit{sepa}} `black'.} `become black'. The inchoative suffix (\sectref{sec:3.8.2.2.2}) is the standard device used for verbalizing adjectives. 

\paragraph[Directional verbs]{Directional verbs}
\hypertarget{RefHeading20841935131865}{}
The verbs indicating coming and going are among the most frequent verbs in Mauwake. These verbs have the direction inherent in the verb root. Verbs of this kind are quite common among Papuan languages: in some languages the directional is an affix, in others it is part of the meaning of the root itself \citep[149]{Foley1986};  Mauwake is of the latter type. The directional verb group contains verbs that in many languages would be prototypically intransitive.

Most of these verbs can be translated into English as either `go' or `come', depending on the context. Since the elevation of the goal, the direction of the compass and the distance all influence the choice of the verb, and may conflict with each other, the speaker has some freedom of choice. Also, with regard to proximity, it is a very relative notion how close or far away something is.

ikiw-  `go', `leave' (away from the deictic centre; generic)

iw-    `go' (away from the deictic centre)\footnote{In the Moro area \textstyleFootnoteBaseChar{\textit{iw-}} also has the meaning `enter': \textstyleFootnoteBaseChar{\textit{Marasin kema wiar iwak}}  `The poison entered his liver.'}

ekap-  `come' (towards the deictic centre; generic)

urup-  `go/come up', `ascend' (uphill/away from sea)

or(a)-  `go/come down', `descend' (downhill/towards sea)

ek-    `go (close/east)'

ep-    `come (close/west)'

er-    `go (not close/west/downriver)'

ir-    `come/go (not close/east/upriver)', `climb'

\ea%x280
\label{ex:x280}
\gll Manina  \textstyleEmphasizedVernacularWords{urup-ep}  nan  uuw-ap  owowa  \textstyleEmphasizedVernacularWords{or-o-k}. \\
      \\
\glt
\z

garden  ascend-SS.SEQ  there  work-SS.SEQ  village  descend-PA-3s

`She went up to the garden, worked there and came down to the village.'

\ea%x281
\label{ex:x281}
\gll Fofa  \textstyleEmphasizedVernacularWords{er-ap}  \textstyleEmphasizedVernacularWords{ir-i-mik}. \\
      \\
\glt
\z

market  go.SS.SEQ  come-Np-PR.1/3p  

`We are coming back from the market.' (Lit: `We went west to the market and are coming east.')

The deictic orientation of \textstyleStyleVernacularWordsItalic{ikiw}- `away from speaker/deictic centre' and \textstyleStyleVernacularWordsItalic{ekap}- `towards the speaker/deictic centre'is stricter in Mauwake than in many European languages where the deictic centre especially for `come' can vary considerably. The sentence (\stepcounter{nx}{\thenx}) is all right in Finnish regardless of the location of the speaker, but the corresponding sentence in Mauwake would be acceptable only if the speaker were in Tampere at the time of speaking.

\ea%x282
\label{ex:x282}
\gll Isois\"ani \textstyleForeignWords{tuli}  Tampereelle vuonna 1912.        (Finnish)  \\
      \\
\glt
\z

`My grandfather \textstyleEmphasizedWords{came} to Tampere in 1912.' 

The equivalent of the English `come' in (\stepcounter{nx}{\thenx}) has to be `go' in Mauwake (\stepcounter{nx}{\thenx}). This is discussed further in 6.3. 

\ea%x283
\label{ex:x283}
\gll I'll \textstyleEmphasizedWords{\textsc{come t}}o your place tomorrow.  \\
      \\
\glt
\z



\ea%x284
\label{ex:x284}
\gll Uurika  nefa  uruf-owa  \textstyleEmphasizedVernacularWords{ikiw-i-nen}. \\
      \\
\glt
\z

tomorrow  2s.ACC  see-NMZ  go-Np-FU.1s  

`Tomorrow I'll go to see you.'

When these verbs occur with a locative phrase containing the locative marker (\sectref{sec:3.12.4}), the phrase almost always refers to either source (\stepcounter{nx}{\thenx}), or location/path (\stepcounter{nx}{\thenx}).  The goal is very seldom marked with the locative marker -\textstyleStyleVernacularWordsItalic{pa}; this happens when the goal is important mainly as the location of the next event (\stepcounter{nx}{\thenx}). Also, in (\stepcounter{nx}{\thenx}) \textstyleStyleVernacularWordsItalic{mukuna} `fire' is an untypical goal for a directional verb.

\ea%x285
\label{ex:x285}
\gll \textstyleEmphasizedVernacularWords{Manina}\textstyleEmphasizedVernacularWords{=pa  ekap-ep}  maa  uup-e-mik. \\
      \\
\glt
\z

garden=LOC  come-SS.SEQ  food  cook-PA-1/3p  

`We came from the garden and cooked food.'

\ea%x447
\label{ex:x447}
\gll Iinan  aasa  \textstyleEmphasizedVernacularWords{iinan=pa}  fan  \textstyleEmphasizedVernacularWords{ekap-emi}  {\dots} \\
      \\
\glt
\z

sky  canoe  sky=LOC  here  come-SS.SIM

`The airplane came here in the sky and{\dots}'

\ea%x1878
\label{ex:x1878}
\gll Ne  soran-emi  \textstyleEmphasizedVernacularWords{epia  mukuna}\textstyleEmphasizedVernacularWords{=}\textstyleEmphasizedVernacularWords{pa} \\
      \\
\glt
\z

ADD  get.startled-SS.SIM  firewood  fire=LOC  

\textstyleEmphasizedVernacularWords{or}\textstyleEmphasizedVernacularWords{-}\textstyleEmphasizedVernacularWords{omi}  aw-o-k.

descend-SS.SIM  burn-PA-3s

`And he got startled and fell on the fire and burned himself.'

The directional verbs differ from other verbs in Mauwake in that they can be transitivized with the `bring' prefixes \textstyleStyleVernacularWordsItalic{p}-, \textstyleStyleVernacularWordsItalic{amap}- and \textstyleStyleVernacularWordsItalic{aap}- (\sectref{sec:3.8.2.4.2}) to indicate either bringing or taking something somewhere.

\ea%x286
\label{ex:x286}
\gll Ona  owowa  \textstyleEmphasizedVernacularWords{p-ikiw-ep}  soop-i-yan. \\
      \\
\glt
\z

3s.GEN  village  BPx-go-SS.SEQ  bury-Np-FU.1p  

`We'll take him (=his body) in his village and bury him (there).'

The causative suffix \nobreakdash-\textstyleStyleVernacularWordsItalic{ow}\textstyleEmphasizedWords{} (\sectref{sec:3.8.4.3.1})\textstyleEmphasizedWords{} can be added to the roots; when following a one-syllable root the suffix is often reduplicated, but the meaning is still the same as with a single causative suffix.

\ea%x435
\label{ex:x435}
\gll Purowa  ir-\textstyleEmphasizedVernacularWords{ow}-(\textstyleEmphasizedVernacularWords{ow})-eya  siin-ar-e-k. \\
      \\
\glt
\z

armband  go.up-CAUS-CAUS-2/3s.DS  tight-INCH-PA-3s

`She pushed the armband up and it got tight.'

The directional verbs are very frequent as the second root in serial verbs (\stepcounter{nx}{\thenx}) (\sectref{sec:3.8.5.1.2}) and as the main verb in verb plus auxiliary constructions (\stepcounter{nx}{\thenx}) (\sectref{sec:3.8.5.1.1}). Some of them also enter into adjunct plus verb constructions (\stepcounter{nx}{\thenx}) (\sectref{sec:3.8.5.2}). 

\ea%x287
\label{ex:x287}
\gll Wi  Amerika  ``epa  eliwa''  nae-\textstyleEmphasizedVernacularWords{ekap}-e-mik. \\
      \\
\glt
\z

3p.UNM  America  time  good  say-come-PA-1/3p  

`The Americans came saying, ``peace''.'

\ea%x288
\label{ex:x288}
\gll Wi  Yaapan  saa=iw  \textstyleEmphasizedVernacularWords{ir}-am-ika-i-mik. \\
      \\
\glt
\z

3p.UNM  Japan  sand=INST  go-SS.SIM-be-Np-PR.1/3p  

`The Japanese are going along the beach.'

\ea%x289
\label{ex:x289}
\gll Kemuka  \textstyleEmphasizedVernacularWords{pepek  er-}eya  puuk-a-k. \\
      \\
\glt
\z

string  enough  go-2/3s.DS  cut-PA-3s  

`When the string was (long) enough she cut it.'

The meaning of the verbs \textstyleStyleVernacularWordsItalic{ekap}- `come' and \textstyleStyleVernacularWordsItalic{ikiw}- `go' can be metaphorically extended to time, to signal time spans. The former is used when the time span is extended from the past to the present (\stepcounter{nx}{\thenx}), the latter is more common when the time extends from the present to the future (\stepcounter{nx}{\thenx}), but it can also refer to the past (\stepcounter{nx}{\thenx}).

\ea%x290
\label{ex:x290}
\gll Naap  on-am-ik-e-mik,  \textstyleEmphasizedVernacularWords{ekap-ep } aakisa. \\
      \\
\glt
\z

thus  do-SS.SIM-be-PA-1/3p  come-SS.SEQ  now

`We have been doing like that (all the time) up until now.'

\ea%x437
\label{ex:x437}
\gll No  naap  ik-ok  \textstyleEmphasizedVernacularWords{iki(w-e)p}  mokoma  enuma  iiwawun \\
      \\
\glt
\z

2s.UNM  thus  be-SS  go-SS.SEQ  year  new  altogether  

aakun-i-nan.

talk-Np-FU.2s

`You will be like that (long time) but next year you will talk.'

\ea%x291
\label{ex:x291}
\gll Buren  \textstyleEmphasizedVernacularWords{ife-iki}\textstyleEmphasizedVernacularWords{(}\textstyleEmphasizedVernacularWords{w-e}\textstyleEmphasizedVernacularWords{)}\textstyleEmphasizedVernacularWords{p}  aakisa  arim-o-n. \\
      \\
\glt
\z

ceremonial.liquid  rub-go-SS.SEQ  now  grow-PA-2s  

`You have kept rubbing the \textstyleEmphasizedWords{buren}\textit{}  liquid on (for years), and now you have grown up.'

On the fringe of directional verbs are \textstyleStyleVernacularWordsItalic{kerer}- `arrive', \textstyleStyleVernacularWordsItalic{yiaw}-/\textstyleStyleVernacularWordsItalic{miaw}- `walk/move around, wander' and \textstyleStyleVernacularWordsItalic{irapar}- `move back and forth (aimlessly)', which share some of their grammatical characteristics but not all of them. Of these three verbs, \textstyleStyleVernacularWordsItalic{kerer}- cannot be prefixed with the bring-prefixes, but it mainly occurs with an unmarked goal instead of a locative phrase (\stepcounter{nx}{\thenx}).

\ea%x292
\label{ex:x292}
\gll Emeria  mua  manina  \textstyleEmphasizedVernacularWords{kerer-e-mik}. \\
      \\
\glt
\z

woman  man  garden  arrive-PA-1/3p  

`The people arrived in the garden.'

With the other two a bring-prefix is acceptable (\stepcounter{nx}{\thenx}), but they do not take a goal/path argument. If a locative phrase occurs with them it requires a locative clitic (\stepcounter{nx}{\thenx}).

\ea%x293
\label{ex:x293}
\gll Gomi  kawus  \textstyleEmphasizedVernacularWords{p-irapar-i-ya}. \\
      \\
\glt
\z

east.wind  smoke  BPx-move.back.and.forth-Np-PR.3s  

`The east wind moves/blows the smoke around.'

\ea%x436
\label{ex:x436}
\gll Soora=pa  nan  \textstyleEmphasizedVernacularWords{yiaw-e-mik}. \\
      \\
\glt
\z

jungle=LOC  there  walk.around-PA-1/3p  

`They walked around in the jungle.'

\paragraph[Utterance verbs]{Utterance verbs}
\hypertarget{RefHeading20861935131865}{}
Utterance verbs may be either intransitive (\stepcounter{nx}{\thenx}), ambitransitive (\stepcounter{nx}{\thenx}), (\stepcounter{nx}{\thenx}), or transitive (\stepcounter{nx}{\thenx}).  They may be used to introduce a quote complement, but not to close it. They often occur with one of the `saying' verbs described below (\stepcounter{nx}{\thenx}).

\ea%x309
\label{ex:x309}
\gll Takira  niir-emi  \textstyleEmphasizedVernacularWords{kirir-i-mik}. \\
      \\
\glt
\z

boy  play-SS.SIM  shout-Np-PR.1/3p  

`The boys are playing and shouting.'

\ea%x310
\label{ex:x310}
\gll Wi  iperowa=ke  \textstyleEmphasizedVernacularWords{aakun-ep}  ma-e-mik,  ``{\dots}'' \\
      \\
\glt
\z

3p.UNM  middle.aged=CF  discuss-SS.SEQ  say-PA-1/3p

`The middle-aged men discussed (it) / talked and said, ``{\dots}'' '

\ea%x1927
\label{ex:x1927}
\gll Maapora  kamenap  \textstyleEmphasizedVernacularWords{aakun-i-yan}? \\
      \\
\glt
\z

feast  how  discuss-Np-FU.1p

`How shall we discuss the feast?'

\ea%x311
\label{ex:x311}
\gll Yena  mua  \textstyleEmphasizedVernacularWords{far-e-m},  ``Sarak  oo,  {\dots}'' \\
      \\
\glt
\z

1s.GEN  man  call-PA-1s  Sarak  oh  

`I called to my husband, ``Oh Sarak,{\dots}'' '

The `\textstyleEmphasizedWords{\textsc{saying verbs}}' described in this section below include three, or four, verbs that between them divide the semantic area of `tell/say/speak/think'.  They are frequently used as frame verbs in quote formulas, but they have other functions as well.

maak-/naak-  `tell'

ma-  `say/speak'

na-   `say/speak/think'

The verb \textstyleStyleVernacularWordsItalic{maak}- `tell' is used in the same two main senses as its English equivalent: telling someone \textstyleEmphasizedWords{\textsc{about}} something (\stepcounter{nx}{\thenx}) and telling someone \textstyleEmphasizedWords{\textsc{to do}} something (\stepcounter{nx}{\thenx}). In direct quote formulas it is used mainly preceding a quote (\stepcounter{nx}{\thenx}), not directly following it as a short closing formula. It is not used in indirect quotes at all.

\ea%x312
\label{ex:x312}
\gll Ne  \textstyleEmphasizedVernacularWords{maak-e-mik},  ``Ifa  yia  keraw-i-ya  nain,  {\dots''} \\
      \\
\glt
\z

and  tell-PA-1/3p  snake  1p.ACC  bite-Np-PR.3s  that1

`And they told him, ``When a snake bites us, {\dots}'' '

\ea%x313
\label{ex:x313}
\gll Moma  yia  \textstyleEmphasizedVernacularWords{maak-i-mik}. \\
      \\
\glt
\z

taro  1p.ACC  tell-Np-PR.1/3p  

`They are telling us (to get them) taro roots.'

\ea%x314
\label{ex:x314}
\gll Efa\textstyleFreeTranslationChar{ } \textstyleEmphasizedVernacularWords{maak-ek}\textstyleFreeTranslationChar{, } ``Opora  tep=pa  wu-e.'' \\
      \\
\glt
\z

\textstyleFreeTranslationChar{1s.ACC  tell-PA-3s  talk  tape.recorder=LOC  put-IMP.2s}

`She told me, ``Put the talk on a tape recorder.'' '

When \textstyleStyleVernacularWordsItalic{maak}- closes a direct quote, it requires the manner adverb \textstyleStyleVernacularWordsItalic{naap} `thus' to precede it:

\ea%x315
\label{ex:x315}
\gll ``Aaw-ep  p-ekap-eka,''  \textstyleEmphasizedVernacularWords{naap}  yia \\
      \\
\glt
\z

get-SS.SEQ  BPx-come-IMP.2p  thus  1p.ACC  

\textstyleEmphasizedVernacularWords{maak-em-ik-e-mik}.

tell-SS.SIM-be-PA-1/3p

` ``Bring it'', they were telling us like that.'

The default object for \textstyleStyleVernacularWordsItalic{maak}- is the addressee (\stepcounter{nx}{\thenx}) and a possible second object is the speech itself (\stepcounter{nx}{\thenx}).

\ea%x316
\label{ex:x316}
\gll [Wadol  opora]\textsubscript{O}  [yia]\textsubscript{O}  \textstyleEmphasizedVernacularWords{maak-i-n}. \\
      \\
\glt
\z

lie  talk  1p.ACC  tell-Np-PR.2s  

`You are telling us lies.'

The status of the verb \textstyleStyleVernacularWordsItalic{naak}- is unclear. It is infrequent, and in natural texts only occurs in closing formulas (\stepcounter{nx}{\thenx}). It may have developed as an analogy to the verb pair \textstyleStyleVernacularWordsItalic{ma}-/\textstyleStyleVernacularWordsItalic{na}-.

\ea%x317
\label{ex:x317}
\gll ``No  bom  fain=iw  mera  kuum-e,''  \textstyleEmphasizedVernacularWords{naak-e-mik}. \\
      \\
\glt
\z

2s.UNM  bomb  this=INST  fish  burn-IMP.2s  tell-PA-1/3p  

` ``Blast fish with this bomb,'' they told him.'

With the verb \textstyleStyleVernacularWordsItalic{ma}- `say/speak/tell' the addressee is not in focus, and is hardly ever even mentioned. Instead, the verb requires either an object referring to the speech content (\stepcounter{nx}{\thenx}) or an adverb \textstyleStyleVernacularWordsItalic{naap} `thus' (\stepcounter{nx}{\thenx}) preceding the verb, or a quote complement following it (\stepcounter{nx}{\thenx}). 

\ea%x318
\label{ex:x318}
\gll Yo  yena  yaaya  ifa  ku-o-k  nain  opora \\
      \\
\glt
\z

1s.UNM  1s.GEN  1s/p.uncle  snake  bite-PA-3s  that1  talk  

\textstyleEmphasizedVernacularWords{ma-i-yem.}

say-Np-PR.1s

`I am telling a story about my uncle that was bitten by a snake.'

\ea%x319
\label{ex:x319}
\gll Momora,  no  naap  me  \textstyleEmphasizedVernacularWords{ma-e.} \\
      \\
\glt
\z

Fool  2s.UNM  thus  not  say-IMP.2s  

`Fool, don't say like that.'

\ea%x320
\label{ex:x320}
\gll En-e-mik  na{\footnotemark}   \textstyleEmphasizedVernacularWords{ma-e-mik},  ``Eliwa,  aara  oposia  saarik.'' \\
      \\
\glt
\z

eat-PA-1/3p  ADD  say-PA-1/3p  good  hen  meat  like

`They ate it and said, ``It is good, like chicken meat.'' '

\footnotetext{Tok Pisin \textit{na} `and' is increasingly used instead of the vernacular additive connective \textit{ne}.}

Occasionally the verb can occur without any of the above objects:

\ea%x321
\label{ex:x321}
\gll Yena  oram  \textstyleEmphasizedVernacularWords{ma-i-yem}. \\
      \\
\glt
\z

1s.GEN  just  say-Np-PR.1s  

`I'm just speaking (without any reason ).'

The difference between the verbs \textstyleStyleVernacularWordsItalic{maak}- and \textstyleStyleVernacularWordsItalic{ma}- in regard to the semantic role of a person object is shown clearly in the next example:

\ea%x322
\label{ex:x322}
\gll Naap  \textstyleEmphasizedVernacularWords{yi}\textstyleEmphasizedVernacularWords{a}  \textstyleEmphasizedVernacularWords{ma-i-}\textstyleEmphasizedVernacularWords{kuan}  na-ep  yo \\
      \\
\glt
\z

thus  1p.ACC  say-Np-FU.3p  think-SS.SEQ  1s.UNM  

ariman \textstyleEmphasizedVernacularWords{} \textstyleEmphasizedVernacularWords{nefa}  \textstyleEmphasizedVernacularWords{maak-i-yem}.

openly  2s.ACC  tell-Np-PR.1s

`Thinking that they will \textstyleEmphasizedWords{\textsc{say}} like that \textstyleEmphasizedWords{\textsc{about us}} I'm openly \textstyleEmphasizedWords{\textsc{telling you}} (this).'

The verb \textstyleStyleVernacularWordsItalic{na}- `say/speak/call/think' is the most interesting of the speech verbs. In quote formulas it is only used for closing the quote (\stepcounter{nx}{\thenx}), with or without another utterance verb in an opening formula.

\ea%x323
\label{ex:x323}
\gll {\dots}\textstyleEmphasizedVernacularWords{ma-em-ik-e-mik},  ``Oo,  {\dots}''  \textstyleEmphasizedVernacularWords{na-em-ik-e-mik}. \\
      \\
\glt
\z

{\dots}say-SS.SIM-be-PA-1/3p  oh  ...  say-SS.SIM-be-PA-1/3p  

`...they kept saying, ``Oh...'',  they kept saying (like that).'

\ea%x942
\label{ex:x942}
\gll Amerika  fan  ``Epa  eliwa''  \textstyleEmphasizedVernacularWords{nae-ekap-e-mik}. \\
      \\
\glt
\z

America  here  time  good  say-come-PA-1/3p

`The Americans came saying ``peace''.'

In a Tail-Head type construction (\sectref{sec:8.2.3.5}) it is often used as a generic verb to replace another utterance verb, when normally the first verb would be repeated.\footnote{Other types of verbs, when not repeated in a Tail-Head construction, are replaced with the generic verb \textstyleFootnoteBaseChar{\textit{on-}} `do'.} 

\ea%x324
\label{ex:x324}
\gll Wia  \textstyleEmphasizedVernacularWords{maak-e-mik},  ``Yia  uf-om-aka.'' \\
      \\
\glt
\z

3p.ACC  tell-PA-1/3p  1p.ACC  dance-BEN-BNFY2.IMP.2p  

\textstyleEmphasizedVernacularWords{Na-iwkin}{\dots}

say-2/3p.DS

`They told them, ``Dance for us.'' When they said (that){\dots}'

When \textstyleStyleVernacularWordsItalic{na}- replaces another utterance verb in that way, the replaced verb may influence what semantic argument becomes the object. In (\stepcounter{nx}{\thenx}) \textstyleStyleVernacularWordsItalic{maak}- requires the addressee of the verb as the default object, and in the following sentence with \textstyleStyleVernacularWordsItalic{na}- the same accusative pronoun \textstyleStyleVernacularWordsItalic{wia}\textit{ }still refers to the addressees, even if with \textstyleStyleVernacularWordsItalic{na}-  it would normally refer to the people spoken about.

\ea%x325
\label{ex:x325}
\gll Ekap-emi  \textstyleEmphasizedVernacularWords{wia}  \textstyleEmphasizedVernacularWords{maak-e-mik},  ``Maa  iiw-eka.'' \\
      \\
\glt
\z

come-SS.SIM  3p.ACC  tell-PA-1/3p  food  dish.out-IMP.2p  

\textstyleEmphasizedVernacularWords{Wia  na-iwkin}  ma-e-mik,  ...

3p.ACC  say-2/3p.DS  say-PA-1/3p

`They\textsubscript{i} came and told them\textsubscript{j}, ``Dish out food.'' They\textsubscript{i} said to them\textsubscript{j} like that and they\textsubscript{j} said, {\dots}'

The verb \textstyleStyleVernacularWordsItalic{na}- is also used in a somewhat different sense `call (by some name)'. In (\stepcounter{nx}{\thenx}) the speaker tells that the word used by the Japanese soldiers for `coconut' was \textstyleStyleVernacularWordsItalic{yasi}, a foreign word for her.\footnote{The verb \textit{unuf}-  is used when the calling by name or giving a name is emphasized.}

\ea%x326
\label{ex:x326}
\gll Iwera  ``yasi''  yia  \textstyleEmphasizedVernacularWords{na-em-ik-e-mik}. \\
      \\
\glt
\z

coconut  yasi  1p.ACC  say-SS.SIM-be-PA-1/3p

`They kept calling coconut (by the name) ``yasi'' to us.' 

The ``speaking'' expressed by \textstyleStyleVernacularWordsItalic{na}- can also be internal speech, i.e. thinking (\stepcounter{nx}{\thenx}). This characteristic is quite common to speech verbs in Papuan languages. When the thinking \textstyleEmphasizedWords{\textsc{process}} itself is more in focus, an adjunct plus verb construction \textstyleStyleVernacularWordsItalic{kema} \textstyleStyleVernacularWordsItalic{suuw}- `think' (literally: `push the liver') is used.

\ea%x327
\label{ex:x327}
\gll Maa  eliwa=ke  \textstyleEmphasizedVernacularWords{na-ep}  aaw-e-m. \\
      \\
\glt
\z

thing  good=CF  say-SS.SEQ  get-PA-1s  

`I thought it was a good thing and got it.'

Related to the inner speech is another usage typical of verbs for `saying' in Papuan languages: to convey desire, intention or plan. For this function only the same subject sequential form\textit{} \textstyleStyleVernacularWordsxiiptItalic{naep} is used, and the verb that indicates the desired or intended  action is in a preceding speech complement clause. This is discussed more fully in the section on complements of utterance verbs (8.3.2.1). 

\ea%x328
\label{ex:x328}
\gll [Yo  manina  urup-i-nen]  \textstyleEmphasizedVernacularWords{na-ep}.  \\
      \\
\glt
\z

1s.UNM  garden  ascend-Np-FU.1s  say-SS.SEQ  

`I want to go to the garden.'

\ea%x329
\label{ex:x329}
\gll [Irak-u]  \textstyleEmphasizedVernacularWords{na-ep}  ikiw-e-mik. \\
      \\
\glt
\z

fight-IMP.1d  say-SS.SEQ  go-PA-1/3p

`They went to fight.' (Lit: ` ``Let's fight'' they said/thought and went.')

\ea%x1608
\label{ex:x1608}
\gll [Ununa  owowa  p-or-owa]  \textstyleEmphasizedVernacularWords{na-ep}  maa  eno-wa \\
      \\
\glt
\z

slit.gong  village  Bpx-descend-NMZ  say-SS.SEQ  food  eat-NMZ  

maneka  on-i-kuan.

big  make-Np-FU.3p

`When they want to take the slit gong down to the village they make a big feast.'

In this function \textstyleStyleVernacularWordsxiiptItalic{naep} is becoming less like a regular medial verb. It can occur in sentence-final position, without being right-dislocated (\stepcounter{nx}{\thenx}).  It usually does retain its word stress, but there is a tendency to un-stress and shorten it by dropping the vowel /a/ in speech (\stepcounter{nx}{\thenx}). When the verb in the speech complement clause is in the counterfactual form, all that is sometimes left of \textstyleStyleVernacularWordsItalic{na-ep} is only the suffix, which is then added as a suffix to the other verb (\stepcounter{nx}{\thenx}).

\ea%x1830
\label{ex:x1830}
\gll Ifana  wu-am-ika-i-kuan,  [unuma  wia  miim-u]   \\
      \\
\glt
\z

ear  put-SS.SIM-be-Np-FU.3p  name  3p.ACC  hear-1d.IMP  

\textstyleEmphasizedVernacularWords{n}\textstyleEmphasizedVernacularWords{-}\textstyleEmphasizedVernacularWords{ep}.

say-SS.SEQ

`They\textsubscript{i} are listening carefully (lit: putting their ear), wanting to hear their\textsubscript{j} names.'

\ea%x1609
\label{ex:x1609}
\gll Yo  aakisa  nanar  nain  \textstyleEmphasizedVernacularWords{ma-ek-a-m-{\O}-ep}. \\
      \\
\glt
\z

1s.UNM  now  story  that1  say-CNTF-PA-1s-{\O}-SS.SEQ

`Now I would like to tell that story.'

The verb \textstyleStyleVernacularWordsItalic{na}- quite freely combines with sound words, and a number of these combinations have been lexicalized (\stepcounter{nx}{\thenx}), (\stepcounter{nx}{\thenx}). The onomatopoeic word has become part of the verb, and the vowel /a/ has been deleted from the verb in the process.

\ea%x330
\label{ex:x330}
\gll Oro-mi  \textstyleEmphasizedVernacularWords{bulak  na-i-ya}\textstyleEmphasizedVernacularWords{\textmd{\textit{.}}} \\
      \\
\glt
\z

drop-SS.SIM  plop  say-Np-PR.3s  

`When it drops it says ``plop''.'

\ea%x331
\label{ex:x331}
\gll Siowa  \textstyleEmphasizedVernacularWords{baun-i-ya}.  ({{\textless}}  bau  na-i-ya) \\
      \\
\glt
\z

dog  bark-Np-PR.3s  (  bau  say-Np-PR.3s)

`The dog barks.'

\ea%x332
\label{ex:x332}
\gll Ema  \textstyleEmphasizedVernacularWords{buun-eya}  mua  erup  um-e-mik.  ({{\textless}}  buu  na-eya) \\
      \\
\glt
\z

mountain  erupt-2/3s.DS  man  two  die-PA-1/3p  (  buu  say-2/3s.DS)

`The mountain (=volcano) erupted and two men died.'

In fast speech \textstyleStyleVernacularWordsItalic{na}- is often reduced to \textstyleStyleVernacularWordsItalic{a}- when the verb follows a consonant-final word.

\ea%x333
\label{ex:x333}
\gll ``Uruf-a-mik''  \textstyleEmphasizedVernacularWords{a-e-k}. \\
      \\
\glt
\z

see-PA-1/3p  say-PA-3s  

` ``They saw it,'' he said.'

The medial form \textstyleStyleVernacularWordsItalic{na-eya} is also used as resultative connective `so, therefore' (\sectref{sec:3.11.2}).

\ea%x500
\label{ex:x500}
\gll Iwera  yia  na-em-ik-e-mik.  \textstyleEmphasizedVernacularWords{Naeya}  iwera  wia \\
      \\
\glt
\z

coconut  1p.ACC  say-SS.SIM-be-PA-1/3p  So  coconut  3p.ACC

uruk-am-ik-om-a-mik. 

drop-SS.SIM-be-BEN-BNFY2.PA-1/3p

`They kept speaking to us about coconuts /asking us for coconuts. So we kept dropping coconuts for them.'

\paragraph[Impersonal experience verbs]{Impersonal experience verbs}
\hypertarget{RefHeading20881935131865}{}
This very small group mainly consists of verbs indicating some kind of pain. They look like transitive verbs, but the syntactic subject is inanimate, usually a body part, and the human experiencer is the object. 

gilin-  `smart (v.)'

kokas-  `itch'

liilin-  `sting'

tiitin-  `hurt, ache (generic)'

tukun-  `throb'

sirir-  `ache'

\ea%x1013
\label{ex:x1013}
\gll Maara  efa  \textstyleEmphasizedVernacularWords{tiitin-i-ya}. \\
      \\
\glt
\z

forehead  1s.ACC  hurt-Np-PR.3s

`My head hurts.'/ `I have a headache.' (Lit: `It hurts my forehead.')

\ea%x1014
\label{ex:x1014}
\gll Uuw-ap  uuw-ap  oona=ke  efa  \textstyleEmphasizedVernacularWords{sirir-i-ya}. \\
      \\
\glt
\z

work-SS.SEQ  work-SS.SEQ  bone=CF  1s.ACC  ache-Np-PR.3s

`I have worked and worked, and my bones ache.'

Most of the experience verbs in Mauwake are adjunct plus verb constructions (\sectref{sec:3.8.5.2.1}); a few are ordinary intransitive verbs (\sectref{sec:3.8.4.2.1}). 

\subsubsection[Auxiliary verbs]{Auxiliary verbs}
\hypertarget{RefHeading20901935131865}{}
The small group of auxiliary verbs in Mauwake consists of two ordinary verbs that have also grammaticalized as auxiliaries indicating aspect. In this function the lexical meaning of the verbs is somewhat bleached. The auxiliary is the last verb in a verbal group (\sectref{sec:3.8.5.1}). 

The paradigms of the auxiliaries are similar to those of main verbs. The auxiliary verbs are:

\textstyleAcronymallcaps{AUX:}  \textstyleAcronymallcaps{MEANING:}  \textstyleAcronymallcaps{MAIN VERB FORM:}

ik-    `continuous'  \textstyleAcronymallcaps{SS.SIM}

    `stative'  \textstyleAcronymallcaps{SS.SEQ}

pu- ({{\textless}}wu-)  `completive'   \textstyleAcronymallcaps{SS.SEQ}

The auxiliary \textstyleStyleVernacularWordsItalic{ik}- is very frequent and has several functions.  When it is used with a main verb in the same-subject simultaneous form (\textstyleAcronymallcaps{SS.SIM}), it indicates continuous aspect, which can have either progressive (\stepcounter{nx}{\thenx}) or habitual (\stepcounter{nx}{\thenx}) meaning. For position-taking verbs (\sectref{sec:3.8.4.4.2}) and other semantically punctiliar verbs the habitual interpretation is the only possible one, but for other verbs the context is needed to determine the correct interpretation. 

\ea%x339
\label{ex:x339}
\gll Fikera  aw-em-\textstyleEmphasizedVernacularWords{ik}-eya  uruf-a-k.  (progressive) \\
      \\
\glt
\z

kunai.grass  burn-SS.SIM-be-2/3s.DS  see-PA-3s  

`When the kunai grass was burning she saw it.' (Or: `She saw the kunai grass burning.')

\ea%x340
\label{ex:x340}
\gll I  yabuela  aaw-ep  {\dots}  wi-em-\textstyleEmphasizedVernacularWords{ik}-e-mik.  (habitual) \\
      \\
\glt
\z

1p.UNM  papaya  get-SS.SEQ  {\dots}  give.them-SS.SIM-be-PA-1/3p  

`We kept getting papayas and {\dots} giving them to them.'

When the main verb is in the same-subject sequential form (\textstyleAcronymallcaps{SS.SEQ}), the auxiliary \textstyleStyleVernacularWordsItalic{ik}- indicates stativity (\stepcounter{nx}{\thenx}). With non-punctiliar verbs this form can often be translated into English with a past perfect (\stepcounter{nx}{\thenx}).

\ea%x341
\label{ex:x341}
\gll Pok-ap-\textstyleEmphasizedVernacularWords{ik}-emkun  epa  wiim-o-k.  (stative) \\
      \\
\glt
\z

sit.down-SS.SEQ-be-1s/p.DS  place  dawn-PA-3s  

`As I was sitting it became light.'

\ea%x342
\label{ex:x342}
\gll Ikiw-ep-\textstyleEmphasizedVernacularWords{ik}-eya  ona  emeria=ke  ekap-o-k.  (perfect) \\
      \\
\glt
\z

go-SS.SEQ-be-2/3s.DS  3s.GEN  woman=CF  come-PA-3s  

``After he was/had gone his wife came.'

The auxiliary \textstyleStyleVernacularWordsItalic{pu}- `completive', is obviously derived from \textstyleStyleVernacularWordsItalic{wu}- `put'\footnote{`Put' is one of the verbs commonly used in Papuan languages to indicate completion \citep[145]{Foley1986}.} through assimilation with the final /p/ of the same-subject sequential form in the main verb preceding it. Synchronically, the Mauwake speakers do not recognise the origin of the auxiliary.

\ea%x343
\label{ex:x343}
\gll Maa  en-ep-\textstyleEmphasizedVernacularWords{pu}-ap  soomar-eka. \\
      \\
\glt
\z

food  eat-SS.SEQ-CMPL-SS.SEQ  walk-IMP.2p  

`Having finished eating you may go.' (Lit: `Eat the food and go'.)

\ea%x501
\label{ex:x501}
\gll Nan  efa  wu-ap-\textstyleEmphasizedVernacularWords{pu}-ami  o  Ulingan  ikiw-o-k. \\
      \\
\glt
\z

there1  1s.ACC  put-SS.SEQ-CMPL-SS.SIM  3s.UNM  Ulingan  go-PA-3s

`He left (lit: put) me there and went to Ulingan.'

\subsection{Verbal clusters}
\hypertarget{RefHeading20921935131865}{}
The verbal clusters are described here under verb morphology, because they function as a unit very much like single verbs. There are two kinds of verbal clusters: verbal groups and adjunct plus verb constructions. The definition of a verbal group is from \citet[175]{Halliday1985}:  ``a sequence of words in the primary class of verb''. 

\ea%x347
\label{ex:x347}
\gll Ifara  \textstyleEmphasizedVernacularWords{mokak-ikiw-em-ik-ok}  ifara  oko  uruf-a-k. \\
      \\
\glt
\z

vine  stare-go-SS.SIM-be-SS  vine  other  see-PA-3s

`He kept looking for a vine and saw one vine.'

Adjunct plus verb combinations\footnote{\citet[184]{Halliday1985} calls these ``phrasal verbs''.}  contain a verb (or a verbal group) plus an element from another word class that is obligatory and contributes to the meaning of the verb. 

\ea%x348
\label{ex:x348}
\gll Owora  efar  \textstyleEmphasizedVernacularWords{ikum  aaw-iwkin}  wia  maak-e-m. \\
      \\
\glt
\z

betelnut  1s.DAT  illicitly  get-2/3p.DS  3p.ACC  tell-PA-1s

`They stole my betelnut and I talked to them.'

The status of a verb phrase in Mauwake is somewhat questionable. It is discussed in \sectref{sec:4.5}. 

\subsubsection[Verbal groups]{Verbal groups}
\hypertarget{RefHeading20941935131865}{}
A verbal group consists of two or more verbs that function grammatically and semantically as one unit. The semantic unity within the group varies between different types of verbal groups. 

The verbal groups containing a main verb plus auxiliary have developed by merging clauses as can still be seen from the verbs involved. But since they synchronically function as a unit very much like an individual verb they are treated on the word level. Features that identify them as one close-knit unit are as follows:


\begin{itemize}
\item Shared subject (and object, if relevant)

\item No non-verbal elements intervening between the parts

\item Scope of negation spans over the whole group

\item No coordinators are allowed between the parts

\item Phrasal intonation and pause structure, i.e. no pauses between the words.


\end{itemize}
Mauwake has two kinds of verbal groups. The verbs in the first group consist of a main verb and an aspectual auxiliary. The second group consists of serial verbs, where all the verb stems contribute to the semantic, rather than grammatical, meaning of the verb.

\paragraph[Main verb plus auxiliary: aspect]{Main verb plus auxiliary: aspect}
\hypertarget{RefHeading20961935131865}{}
The importance of tense as a verbal category in Mauwake  shows in its obligatory morphological marking, but aspect is a relatively important category as well. Aspects are `\textstyleBibliogCitationAAAstyleChar{different ways of viewing the internal temporal constituency of a situation}' \citep[3]{Comrie1976}. 

Aspect in Mauwake is expressed periphrastically, through verbal groups that have a main verb and an auxiliary. The main verb, which is in the medial form, largely gives the semantic content to the whole, and the auxiliary adds the grammatical meaning of aspect. In the continuous and stative aspects also the medial form of the \textstyleEmphasizedWords{\textsc{main}} verb contributes to the aspectual meaning. What distinguishes these constructions from medial clauses (8.2) is that the two verbs function as a unit rather than individual verbs, and their phonological stress, intonation and pause pattern is that of a word or phrase rather than a medial clause. 

 As is typical of \textstyleAcronymallcaps{\textup{SOV}} languages, the auxiliary follows the main verb (Greenberg 1966:85, Dryer 2007a:90). The more common of the aspectual auxiliaries is \textstyleStyleVernacularWordsxiiptItalic{ik}- `be', which can combine with two different medial forms. The other aspectual auxiliary is  \textstyleStyleVernacularWordsxiiptItalic{pu}- `completive' (\sectref{sec:3.8.4.5}). 

The neutral, aspectually unmarked verb form is used in Mauwake whenever the speaker chooses not to pay special attention to the internal structure of the situation. It could be claimed that this is a neutral perfective, since the situation is viewed as a whole, but that term would be confusing, as the neutral forms can also be used in clauses that are aspectually habitual (cf. Payne 1997:239). The majority of the verb forms used in all kinds of texts in Mauwake are aspectually neutral.

The marked completive aspect is only used when completion of an action is stressed. The continuous aspect is used for both progressive and habitual actions, and the stative aspect for a state continuing over some time.

{\bfseries
\hypertarget{RefHeading20981935131865}{}
Completive aspect}

When the \textstyleEmphasizedWords{\textsc{completion}} of an action is in focus, the completive aspect is used. It is formed by a main verb in the same-subject sequential form, followed by the auxiliary \textstyleStyleVernacularWordsItalic{pu}- `completive' (\sectref{sec:3.8.4.5}).

\ea%x361
\label{ex:x361}
\gll Ifakim-ep  nomokow  ekeka=pa  \textstyleEmphasizedVernacularWords{sererim-ep-pu-a-k}. \\
      \\
\glt
\z

kill-SS.SEQ  tree  branch=LOC  hang-SS.SEQ-CMPL-PA-3s  

`He killed it and hung it on a tree branch.'

The completive aspect verb is often used in a medial same-subject sequential form, which in itself only indicates sequentiality but often implies completion of the first action as well. 

\ea%x362
\label{ex:x362}
\gll \textstyleEmphasizedVernacularWords{Sererim-ep-pu-ap}  owowa  or-o-k. \\
      \\
\glt
\z

hang-SS.SEQ-CMPL-SS.SEQ  village  descend-PA-3s

`He hung it up and went/came down to the village.'

\ea%x1041
\label{ex:x1041}
\gll Manina  \textstyleEmphasizedVernacularWords{n}\textstyleEmphasizedVernacularWords{op-ap-pu-ap}  nomokowa  war-i-mik. \\
      \\
\glt
\z

garden  burn-SS.SEQ-CMPL-SS.SEQ  tree  cut-Np-PR.1/3p

`We burn (the undergrowth for new) garden and (when it is done we) cut the trees.'

\ea%x1042
\label{ex:x1042}
\gll Nomokowa  \textstyleEmphasizedVernacularWords{war-ep-pu-ap}  arew-i-mik. \\
      \\
\glt
\z

tree  cut-SS.SEQ-CMPL-SS.SEQ  wait-Np-PR.1/3p

`We cut the trees and wait.'

But it is not uncommon either to have the completive aspect with a simultaneous action medial form, when the second action coincides with the completion of the first one:

\ea%x363
\label{ex:x363}
\gll Wia  \textstyleEmphasizedVernacularWords{maak-ep-pu-ami}  i  ikiw-e-mik. \\
      \\
\glt
\z

3p.ACC  tell-SS.SEQ-CMPL-SS.SIM  1p.UNM  go-PA-1/3p

`We told them and went.' 

\ea%x1040
\label{ex:x1040}
\gll Aria  yo  nan  efa  \textstyleEmphasizedVernacularWords{wu-ap-pu-ami}  o \\
      \\
\glt
\z

alright  1s.UNM  there  1s.ACC  put-SS.SEQ-CMPL-SS.SIM  3s.UNM

Ulingan  ikiw-o-k.

Ulingan  go-PA-3s

`Alright he put me there and he went to Ulingan.'

\ea%x1043
\label{ex:x1043}
\gll Maa  en-owa  \textstyleEmphasizedVernacularWords{wakesim-ep-pu-ami}  ikiw-o-k. \\
      \\
\glt
\z

thing  eat-NMZ  cover-SS.SEQ-CMPL-SS.SIM  go-PA-3s

`Covering the food she left.'

The completive aspect form is also used when \textstyleEmphasizedWords{\textsc{momentaneity}} of the action is emphasized:

\ea%x364
\label{ex:x364}
\gll \textstyleEmphasizedVernacularWords{En-ep-pu-ap}  ikiw-e! \\
      \\
\glt
\z

eat-SS.SEQ-CMPL-SS.SEQ  go-IMP.2s

`Get done with your eating and go!'

The origin of the auxiliary, the verb `put', shows in the fact that it cannot be used with non-controlled actions.\footnote{In general, control vs. non-control is not a prominent feature in the verb system in Mauwake, unlike many other Papuan languages (Foley 1986:127, Reesink 1987:128).}

\ea%x365
\label{ex:x365}
\gll *Waki-ep-pu-a-k \\
      \\
\glt
\z

fall-SS.SEQ-CMPL-PA-3s

In process descriptions a medial verb, followed by the verb \textstyleStyleVernacularWordsItalic{weeser}- `finish', which stresses the endpoint of the action, is used more than the completive aspect. This, however, is a case of clause chaining (8.2), not a verbal group.

\ea%x366
\label{ex:x366}
\gll Uup-ep  \textstyleEmphasizedVernacularWords{weeser-eya}  wienak-e-m. \\
      \\
\glt
\z

cook-SS.SEQ  finish-2/3s.DS  feed.them-PA-1s

`I finished cooking it and fed it to them.' [Lit: `I cooked it and when it (=the cooking) was finished I fed it to them.']

{\bfseries
\hypertarget{RefHeading21001935131865}{}
Continuous aspect: progressive and habitual}

Continuity, or duration, is the semantic component shared by the aspects called progressive and habitual in many languages: continuation of the same action or of repeated actions of the same kind \citep[26]{Comrie1986}. The continuous aspect form in Mauwake can have either progressive (\stepcounter{nx}{\thenx}), (\stepcounter{nx}{\thenx}) or habitual (\stepcounter{nx}{\thenx}), (\stepcounter{nx}{\thenx}) interpretation. The main verb is in the same-subject simultaneous medial form, but with the final /i/ deleted, and the auxiliary \textstyleStyleVernacularWordsItalic{ik}- `be' is inflected for tense and person/number (\stepcounter{nx}{\thenx}). 

\ea%x349
\label{ex:x349}
\gll Maa  \textstyleEmphasizedVernacularWords{en-em-ik-omkun}  ama  or-o-k. \\
      \\
\glt
\z

food  eat-SS.SIM-be-1s/p.DS  sun  descend-PA-3s  

`As I was eating the sun went down.'

\ea%x1044
\label{ex:x1044}
\gll Fikera  \textstyleEmphasizedVernacularWords{aw-em-ik-eya}  nain  umuk-i-nen \\
      \\
\glt
\z

kunai.grass  burn-SS.SIM-be-2/3s.DS  that1  extinguish-Np-FU.1s

na-ep  urup-o-k.

say-SS.SEQ  ascend-PA-3s

`The kunai grass was burning, and she went up in order to extinguish it.'

\ea%x350
\label{ex:x350}
\gll Iwera=ke  wia  aruf-eya  \textstyleEmphasizedVernacularWords{ma-em-ik-e-mik},  ``{\dots''} \\
      \\
\glt
\z

coconut=CF  3p.ACC  hit-2/3s.DS  say-SS.SIM-be-PA-1/3p  

`When coconuts hit them, they kept saying, `` ...'' '

\ea%x1045
\label{ex:x1045}
\gll Wi  Yaapan  naap  kuisow=iw  \textstyleEmphasizedVernacularWords{ekap-em-ik-e-mik}. \\
      \\
\glt
\z

3p.UNM  Japan  thus  one=INST  come-SS.SIM-be-PA-1/3p

`The Japanese kept coming like that, one by one.'

For punctiliar verbs the habitual interpretation (\stepcounter{nx}{\thenx}) is the only one possible, whereas for non-punctiliar verbs both habitual and progressive interpretations  are possible.

\ea%x351
\label{ex:x351}
\gll Koka=pa  nan  \textstyleEmphasizedVernacularWords{in-em-ik-e-mik.} \\
      \\
\glt
\z

jungle=LOC  there  lie.down-SS.SIM-be-PA-1/3p

`We kept sleeping in the jungle'

\ea%x1932
\label{ex:x1932}
\gll Owowa  oko  wiam=iya  \textbf{irak-em-ik-e-mik}. \\
      \\
\glt
\z

village  other  3p.ACC=COM  fight-SS.SIM-be-PA-1/3p

`We were fighting (or: kept fighting repeatedly) with the other village.'

  When the verbal group is in the medial form, the progressive interpretation (\stepcounter{nx}{\thenx}) is the more common:

\ea%x353
\label{ex:x353}
\gll Waaya  \textstyleEmphasizedVernacularWords{urup-em-ik-eya } mik-a-m. \\
      \\
\glt
\z

pig  ascend-SS.SIM-be-2/3s.DS  spear-PA-1s

`As the pig was going/coming up I speared it.'

Often the context provides the only clue as to whether the continuous aspect form should be interpreted as progressive or habitual. The example (\stepcounter{nx}{\thenx}) describes a situation where the villagers kept feeding the Japanese soldiers who asked them for food; the sentence (\stepcounter{nx}{\thenx}) is from a text describing a coconut plantation fire and its consequences.

\ea%x354
\label{ex:x354}
\gll Waaya  yia  na-iwkin  waaya  \textstyleEmphasizedVernacularWords{wienak-em-ik-e-mik}. \\
      \\
\glt
\z

pig  1p.ACC  say-2/3p.DS  pig  feed.them-SS.SIM-be-PA-1/3p

`They asked us for pigs and we kept giving them pigs to eat.'

\ea%x355
\label{ex:x355}
\gll Kawus  \textstyleEmphasizedVernacularWords{ir-am-ik-eya}  kuuf-a-k. \\
      \\
\glt
\z

smoke  rise-SS.SIM-be-2/3s.DS  see-PA-3s

`The smoke was rising and she saw it.'

Cross-linguistically the habitual aspect more commonly receives overt marking in the past tense than in the present \citep[154]{Cristofaro2006}. In Mauwake the continuous aspect can be used for habitual in any of the three tenses. The example (\stepcounter{nx}{\thenx}) was said about particular work that the speaker was not involved in continuously; he used to do it time to time because of his position as need arose. The example (\stepcounter{nx}{\thenx}) refers to a couple needing to keep visiting an ailing father. 

\ea%x1063
\label{ex:x1063}
\gll Yo  anane  maneka  naap  \textstyleEmphasizedVernacularWords{mauw-am-ika-i-yem}. \\
      \\
\glt
\z

1s.UNM  always  very  thus  work-SS.SIM-be-Np-PR.1s

`I always/forever keep working like that.'

\ea%x1064
\label{ex:x1064}
\gll O  me  sariar-i-non-(na)  neeke  \textstyleEmphasizedVernacularWords{in-em-ika-i-kuan}. \\
      \\
\glt
\z

3s.UNM  not  get.well-Np-FU.3s-(TP)  there.CF  sleep-SS.SIM-be-Np-FU.3p

`If he doesn't get well, they will keep sleeping/staying \textit{there}.' 

For a clause to have habitual interpretation it is not obligatory to use the continuous aspect form in the verb. For instance in process descriptions, which tell how something is habitually done, the unmarked, aspectually neutral present tense form is more common than the continuous aspect. Three of the four verbs in (\stepcounter{nx}{\thenx}) are aspectually unmarked, although all the clauses have habitual interpretation, describing seclusion customs.

\ea%x1049
\label{ex:x1049}
\gll Moma  ik-owa  \textstyleEmphasizedVernacularWords{enim-i-mik}.  Eka  me  \textstyleEmphasizedVernacularWords{enim-i-mik},  iwer \\
      \\
\glt
\z

taro  roast-NMZ  eat-Np-PR.1/3p  water  not  eat-Np-PR.1/3p  coconut

eka  me  \textstyleEmphasizedVernacularWords{enim-i-mik}.  Aaya  muutiw  \textstyleEmphasizedVernacularWords{en-em-ika-i-mik}.

water  not  eat-Np-PR.1/3p  sugarcane  only  eat-SS.SIM-be-Np-PR.1/3p

`We do not eat roasted taro. We do not drink water or coconut water. We only eat / keep eating sugarcane.'

{\bfseries
\hypertarget{RefHeading21021935131865}{}
Stative aspect}

The same semantic component of continuity is also shared by the other aspect using the auxiliary \textstyleStyleVernacularWordsxiiptItalic{ik}- `be': this time it is a \textstyleEmphasizedWords{\textsc{state}} rather than activity that continues the same over time. In the stative aspect the auxiliary is combined with a main verb that is in the same-subject sequential form. This usage is most common with the position-taking verbs like \textstyleStyleVernacularWordsItalic{pok}- `sit down', \textstyleStyleVernacularWordsItalic{iimar}- `stand up' and \textstyleStyleVernacularWordsItalic{in}- `lie down/ fall asleep'.

\ea%x356
\label{ex:x356}
\gll \textstyleEmphasizedVernacularWords{Pok-ap-ik-omkun } epa  wiim-o-k. \\
      \\
\glt
\z

sit.down-SS.SEQ-be-1s/p.DS  place  dawn-PA-3s  

`As we were sitting it dawned.'

\ea%x1046
\label{ex:x1046}
\gll Yena  koor  miira=pa  \textstyleEmphasizedVernacularWords{iimar-ep-ik-e-m},  {\dots} \\
      \\
\glt
\z

1s.GEN  house  face=LOC  stand.up-SS.SEQ-be-PA-1s

`I was standing in front of my house, {\dots}'

Other punctiliar verbs (\stepcounter{nx}{\thenx}), as well as non-punctiliar verbs can be used in this aspect to indicate the state resulting from an action (\stepcounter{nx}{\thenx}), or process (\stepcounter{nx}{\thenx}), but they are less frequent.

\ea%x357
\label{ex:x357}
\gll Ifakim-eya  \textstyleEmphasizedVernacularWords{pu-ep-ik-eya } om-em-ik-ua. \\
      \\
\glt
\z

kill-2/3s.DS  die-SS.SEQ-be-2/3s.DS  cry-SS.SIM-be-PA.3s

`When she killed him and he was dead, she was crying.'

\ea%x358
\label{ex:x358}
\gll \textstyleEmphasizedVernacularWords{Ikiw-ep-ik-eya}  ona  emeria=ke  ekap-o-k. \\
      \\
\glt
\z

go-SS.SEQ-be-2/3s.DS  3s.GEN  woman=CF  come-PA-3s

`While he was gone his wife came.'

\ea%x1047
\label{ex:x1047}
\gll Ewar  pun  wuun-e-k  ne  epa  \textstyleEmphasizedVernacularWords{reen-ep-ik-ua}. \\
      \\
\glt
\z

west.wind  too  blow-PA-3s  and  place  dry-SS.SEQ-be-PA.3s

`The west wind blew, too, and the ground was dry.'

In the example (\stepcounter{nx}{\thenx}) the continuous form indicates more active waiting process than is the case in (\stepcounter{nx}{\thenx}) with the stative aspect.  In (\stepcounter{nx}{\thenx}) the people were getting impatient with the vehicle that should already have come to get them. The example (\stepcounter{nx}{\thenx}) is from a description of garden work, and part of the work process is the state of patiently waiting for the felled trees and undergrowth to dry. 

\ea%x359
\label{ex:x359}
\gll Arew\textstyleEmphasizedVernacularWords{-am-}ik-omkun  ama  ikur  miiw-aasa  kerer-ek. \\
      \\
\glt
\z

wait-SS.SIM-be-1s/p.DS  sun  five  land-canoe  arrive-PA-3s

`As we were waiting the car arrived at five.'

\ea%x360
\label{ex:x360}
\gll Nomokowa  war-ep-pu-ap  arew\textstyleEmphasizedVernacularWords{-ap-}ika-iwkin \\
      \\
\glt
\z

tree  cut-SS.SEQ-CMPL-SS.SEQ  wait-SS.SEQ-be-2/3p.DS

reen-eya  saama  kuum-i-mik.

dry-2/3s.DS  cleared.bush  burn-Np-PR.1/3p

`They cut the trees and while they are waiting it dries and then they burn the cleared bush.' 

\paragraph[Serial verbs ]{Serial verbs} 
\hypertarget{RefHeading21041935131865}{}
Verbal groups called serial verbs are very common in Papuan languages \citep[116]{Foley1986}. Finding a cross-linguistic definition for serial verbs has proved to be an extremely hard task (Sebba 1987:5, Lord 1993:1). Instead of one definition covering all the possible serial verbs, \citet[19]{Crowley2002} suggests defining these verbs within ``\textstyleBibliogCitationAAAstyleChar{specific typological and linguogenetic groupings}'' for comparative purposes. 

For a working definition I borrow one given by \citet[28]{James1983} describing the serial verbs in Siane, another Papuan language: 

``A serial verb construction consists of two or more verbs which occur in series with neither normal coordinating nor subordinating markers, which share at least some core argument (normally subject and/or object/goal), and which in some sense function together semantically like a single predication''. 

Typically, even if not obligatorily, one of the verbs in the series is finite and the other(s) more or less ``stripped-down''.  In a verb-final language the finite verb is the last one in the series. After describing the serial verb construction in Mauwake I will discuss the question whether serial verbs are actually compound verbs, and the relationship of the serial verbs to main verb + auxiliary verbal groups and medial clauses.

In Mauwake a non-final verb in a serial construction consists of a bare root without any inflection at all. This restriction is tighter than those given for serial verbs in many other languages (Crowley 2002:19, Sebba 1987:86-7, James 1983:28). Each of the verbs in a serial construction contribute to the overall semantic meaning of the predicate. Even if the meaning is not exactly the same as the combination of the same verbs would have in a tight medial verb chain (cf. Payne 1997:310), it does not get bleached either, like that of the auxiliaries.\footnote{Since a serial verb construction has only one main stress it is written as one word in the orthography, but the verb stems are separated by hyphens to make reading easier.}

\ea%x377
\label{ex:x377}
\gll Sama=pa  \textstyleEmphasizedVernacularWords{oro-boon-ek}. \\
      \\
\glt
\z

ladder=LOC  descend-get.loose-PA-3s

`He fell from the ladder.'

The last verb in a series is either a finite verb with tense and person/number inflection, or a medial verb. The arguments are shared by the whole verbal complex, even if they would originally have been associated with only one of the verbs (\stepcounter{nx}{\thenx}). Also negation and obliques (\stepcounter{nx}{\thenx}) are shared. All this points to serial verbs being a nuclear-level phenomenon in Mauwake, rather than a core-level one (Foley and Van Valin 1984:189-193). 

\ea%x378
\label{ex:x378}
\gll Yo  Amerika  wia  \textstyleEmphasizedVernacularWords{akup-ikiw-i-yem}. \\
      \\
\glt
\z

1s.UNM  America  3p.ACC  search-go-Np-PR.1s  

`I am going to look for the Americans. / I go searching the Americans.' 

\ea%x379
\label{ex:x379}
\gll Neeke  \textstyleEmphasizedVernacularWords{aw(e)-or-om-ik-eya}  {\dots} \\
      \\
\glt
\z

there.CF  burn-descend-SS.SIM-be-2/3s.DS

`As it was burning (towards) down \textit{there}{\dots}'

Semantically the verb combinations are of two types. In the more common one a directional or another motion verb follows another verb stem, giving the meaning of \textstyleEmphasizedWords{\textsc{movement}} to the whole (\stepcounter{nx}{\thenx})-(\stepcounter{nx}{\thenx}), and often the meaning of \textstyleEmphasizedWords{\textsc{directionality}} as well (\stepcounter{nx}{\thenx})-(\stepcounter{nx}{\thenx}).\footnote{Cross-linguistically motion and location verbs are very common in serial verbs \citep[9]{Lord1993}.} This is a productive process, as long as the verbs are semantically compatible.

\ea%x438
\label{ex:x438}
\gll Wia  \textstyleEmphasizedVernacularWords{mokak-urup-o-k},  wia  \textstyleEmphasizedVernacularWords{mokak-or-o-k}. \\
      \\
\glt
\z

3p.ACC  stare-ascend-PA-3s  3p.ACC  stare-descend-PA-3s

`He stared them up and down.'

\ea%x381
\label{ex:x381}
\gll Aasa  \textstyleEmphasizedVernacularWords{suuw-or-o-mik}. \\
      \\
\glt
\z

canoe  push-descend-PA-1/3p  

`We pushed the canoe down (towards the sea).'\footnote{Compare this with a medial construction:  \textstyleFootnoteBaseChar{\textit{Aasa suuw-ap or-o-mik}}  `We pushed the canoe and went down (to sea)'}

If the first stem is also a motion verb, it indicates the \textstyleEmphasizedWords{\textsc{manner}} of movement:

\ea%x380
\label{ex:x380}
\gll Merena  kir-ep  \textstyleEmphasizedVernacularWords{segen-ikiw-o-k}. \\
      \\
\glt
\z

foot  turn-SS.SEQ  limp-go-PA-3s  

`He twisted his foot and limped.'

A motion verb in a serial construction can also indicate \textstyleEmphasizedWords{\textsc{temporal continuity}} over a long period of time. In (\stepcounter{nx}{\thenx}) the length of time is emphasized even more by the repetition of the motion verb.

\ea%x439
\label{ex:x439}
\gll \textstyleEmphasizedVernacularWords{Ife-iki}(w-e)\textstyleEmphasizedVernacularWords{p  iki}(w-e)\textstyleEmphasizedVernacularWords{p}  aakisa  arim-o-n. \\
      \\
\glt
\z

rub-go-SS.SEQ  go-SS.SEQ  now  grow-PA-3s

`You kept rubbing it (over the years) and now you have grown up.'

In the second type, any two verbs can, in principle, combine into a serial verb. But this process is less productive, and both the type and token frequency of this type is low when compared with the frequency of the first type. Usually, like in (\stepcounter{nx}{\thenx}) the meaning of the whole is transparent and can be inferred from the meanings of the component roots, but sometimes the semantics are more opaque (\stepcounter{nx}{\thenx}).

\ea%x382
\label{ex:x382}
\gll Emera  \textstyleEmphasizedVernacularWords{kue-puuk-ap}  okaiwi  siowa  onak-e-k. \\
      \\
\glt
\z

sago  bite-cut-SS.SEQ  other.side  dog  feed.him-PA-3s  

`He bit off half of the sago cake and fed it to the dog.'

\ea%x383
\label{ex:x383}
\gll Aakun-emi  \textstyleEmphasizedVernacularWords{mika-kof-a-m}. \\
      \\
\glt
\z

speak-SS.SIM  spear-knock-PA-1s  

`I stumbled in my speech.'

This type of serialization in Mauwake is very close to what \citet[1-5]{James1983} calls \textstyleEmphasizedWords{\textsc{lexical}} serialization. 

A special case among the roots forming serial verbs is \textstyleStyleVernacularWordsItalic{afur}- `do well'/`augmentative', which is not used as an independent verb, only as a second element in a serial verb structure.\footnote{See James 1983:32 for the use of a similar verb, \textstyleFootnoteBaseChar{\textit{ito,}} in Siane.}

\ea%x384
\label{ex:x384}
\gll Koora  ku-owa  \textstyleEmphasizedVernacularWords{amis-ar-afur-a-k}. \\
      \\
\glt
\z

house  build-NMZ  knowledge-INCH-do.well-PA-3s  

`He really knew how to build a house.'

It is quite possible even if not very common to form a three-root serial verb by combining the two types:

\ea%x385
\label{ex:x385}
\gll \textstyleEmphasizedVernacularWords{Mika-fien-ikiw-o-k}. \\
      \\
\glt
\z

hit-push.aside-go-PA-3s  

`He went on countering (an attack).'

It is far more common to have three verbs in a combination where an auxiliary is attached to a serial verb:

\ea%x386
\label{ex:x386}
\gll Naap  \textstyleEmphasizedVernacularWords{amis-ar-ikiw-em-ik-o-wen}. \\
      \\
\glt
\z

thus  knowledge-INCH-go-SS.SIM-be-Np-FU.2p  

`That way you will gain more and more knowledge.'

Combining four or more roots into one verbal group is more of a theoretical possibility than a practical reality.  Examples are easy enough to obtain through elicitation, but very rare in non-elicited texts.

Mauwake does \textstyleEmphasizedWords{\textsc{not}} use serial verbs for a benefactive like many languages do \citep[174-80]{Sebba1987}; it utilizes benefactive morphology for that purpose (\sectref{sec:3.8.2.3.3}, 3.8.3.1). Neither is the serial verb structure used for aspect, as a verb plus auxiliary construction takes care of that. Another function often associated with serial verbs is that of instrument marking, but for that Mauwake uses either an ordinary switch-reference construction or an adverbial phrase (\sectref{sec:4.6.3}).

Distinguishing serial verbs from compound verbs on the one hand and medial clauses on the other is not a problem for Mauwake only, as  serial verbs can behave very much like either \citep[17]{Crowley2002}. Crowley suggests the following continuum of gradually loosening syntactic juncture: verbal compounds {{\textgreater}} nuclear serial verbs {{\textgreater}} core serial verbs {{\textgreater}} clause chains {{\textgreater}} subordinate clauses {{\textgreater}} coordinate clauses (ibid. 18). In the following I will briefly discuss the relationship of serial verbs to adjunct plus verb constructions, to verbal groups consisting of a main verb plus auxiliary, and to medial clauses in Mauwake. 

The serial verbs in Mauwake show the following characteristics of compounding (cf. James 1983:69 regarding Papuan languages). The first verb appears as a mere root (or as a stem, if it has undergone derivation); secondly, the verbs obligatorily share the same arguments; thirdly, the meaning of the whole may differ from the combined meanings of the parts. Furthermore, the stress and intonation contour of a serial verb is that of a single word rather than that of a phrase or a clause. There are two main reasons for calling them serial verbs. The first one is that especially the first type is productive. I also want to link them to a typologically widespread phenomenon instead of looking at them from a strictly language-specific point of view. In this I follow \citet[101]{Margetts1999}, who maintains that ``\textstyleBibliogCitationAAAstyleChar{the term `compound' does not by definition contradict an analysis as serialization}''. A similar position is also strongly defended by \citet[16]{Crowley2002} and by Giv\'on (1991:17).

Because of the tight restriction of ``root only'' for the first element in a serial verb in Mauwake, the main verb plus auxiliary combinations are left outside the group by definition. Another reason for this differential treatment is the fact that different processes seem to be going on in the two groups: grammaticalization in the main verb + \textstyleAcronymallcaps{AUX} group, lexicalization in the true serial verbs.\footnote{In some other languages main verb + AUX constructions are included among serial verbs (e.g. James 1983: 29, Crowley 2002:178).  Farr notes the ``staging'' aspects of the two constructions: in medial verbs the temporal relationship of the two verbs may be specified, but as ``the verbal constituents of SVCs [serial verb constructions] do not specify temporal borders or overlapping relationships, the events they represent can blend into a unit {\dots} and present the SVC is a complex but integrated event'' (1999:174).} 

In Mauwake the clause chaining is structurally midway between serialization and main clause coordination, and may consequently be used instead of either in some cases. The instrumental may in Mauwake be expressed by a `take-instrument-do' structure (\stepcounter{nx}{\thenx}) which in many serializing languages is a serial verb construction \citep[162-74]{Sebba1987}; but in Mauwake there is no good reason to call the structure anything other than a combination of a medial and final clause. This shows more clearly in example (\stepcounter{nx}{\thenx}), which does not pass the rule for verbal groups: ``no non-verbal elements between the parts''. 

\ea%x387
\label{ex:x387}
\gll Fura  \textstyleEmphasizedVernacularWords{aaw-ep}  puuk-a-m. \\
      \\
\glt
\z

knife  take-SS.SEQ  cut-PA-1s  

`I took a knife and cut it.' Or: `I cut it with a knife.'

\ea%x388
\label{ex:x388}
\gll Burir  aaw-ep  nomokowa  unowa  war-e-mik. \\
      \\
\glt
\z

axe  take-SS.SEQ  tree  many  fell-PA-1/3p  

`We took an axe and felled many trees.' Or: `We felled many trees with an axe.'

For Mauwake, I propose the following continuum where the syntactic juncture gradually loosens: serial verb {{\textgreater}} verb + \textstyleAcronymallcaps{AUX} group {{\textgreater}} subordinate+main clause {{\textgreater}} clause chain {{\textgreater}} coordinate main clauses.

The borderline between serial verbs and medial verbs on the one hand, and between verb + \textstyleAcronymallcaps{AUX} groups and medial verbs on the other is not absolutely clear-cut. In (\stepcounter{nx}{\thenx}) the medial verb structure is used instead of a serial verb, even though the two actions are simultaneous, not sequential as indicated by the form of the medial verb.\footnote{Mauwake does not allow same subject simultaneous forms following each other except in a strictly coordinate structure where the verbs do not so much indicate simultaneity with each other as with the final verb.}  

\ea%x389
\label{ex:x389}
\gll Wi  Malala=ke  \textstyleEmphasizedVernacularWords{muf-ep  ekap-emi}{\dots} \\
      \\
\glt
\z

3p.UNM  Malala=TP  pull-SS.SEQ  come-SS.SIM  

`The Malala people came pulling it and{\dots}'

Likewise, the four verbs in (\stepcounter{nx}{\thenx}) describe \textstyleEmphasizedWords{\textsc{one}} protracted action in spite of the sequential form in the medial verbs:

\ea%x390
\label{ex:x390}
\gll Ifa  nain  \textstyleEmphasizedVernacularWords{murar-ep  wiok-ap  ekap-ep} \\
      \\
\glt
\z

snake  that1  follow-SS.SEQ  follow.them-SS.SEQ  come-SS.SEQ

\textstyleEmphasizedVernacularWords{ekap-ep}  owowa  kerer-ek.

come-SS.SEQ  village  arrive-PA-3s

`The snake kept following them and arrived in the village.'

The main verb in verb plus \textstyleAcronymallcaps{AUX} combinations has to be in medial form. The only exception found is the continuous aspect form of the verb \textstyleStyleVernacularWordsItalic{wiaw}- `move around'. The mere root of this verb is used when it is the second verb in a serial structure which then takes an aspectual auxiliary:

\ea%x391
\label{ex:x391}
\gll Ifara  mufe-\textstyleEmphasizedVernacularWords{wiaw}-ik-ok{\dots} \\
      \\
\glt
\z

vine  pull-move.around-be-SS  

`As he was pulling the vine around{\dots}'

\subsubsection[Adjunct plus verb constructions ]{Adjunct plus verb constructions} 
\hypertarget{RefHeading21061935131865}{}
Papuan languages typically enlarge their verb inventories through adjunct plus verb combinations \citep[127]{Foley1986}. Foley only discusses nominal adjuncts, but adverbial adjuncts are commonly used in these structures as well. 

Mauwake is not nearly as productive in the use of the adjunct plus verb  construction as many other Papuan languages. Some of them use almost exclusively generic verbs (Foley 1986:117, Roberts 1987:309, Whitehead 2004:145), whereas others employ a larger set of verbs \citep[62-66]{Farr1999} in these constructions.\footnote{Farr divides the nominals in these constructions into `complements' an `adjuncts'. Korafe does not seem to use adverbial adjuncts in these structures. } 

\paragraph[Nominal adjunct plus verb]{Nominal adjunct plus verb}
\hypertarget{RefHeading21081935131865}{}
The nominal adjuncts look like object \textstyleAcronymallcaps{NP}s, and the origin of at least some of them probably is in object \textstyleAcronymallcaps{NP}s, but currently there are syntactic and semantic differences between the two. An object \textstyleAcronymallcaps{NP} may be separated from the verb by the negator adverb \textstyleStyleVernacularWordsItalic{me}  or by an accusative or a dative pronoun, but a nominal adjunct must immediately precede the verb. The meaning of the nominal adjunct plus verb construction often cannot be derived from the meanings of its constituent parts.  

\ea%x450
\label{ex:x450}
\gll Meta  yia  miim-ap  yia  \textstyleEmphasizedVernacularWords{miira  puuk-ekap-e-mik}. \\
      \\
\glt
\z

fame  1p.ACC  hear-SS.SEQ  1p.ACC  face  cut-come-PA-1/3p

`They heard about us and came to greet us.'

An object \textstyleAcronymallcaps{NP} only occurs with a transitive verb, but a nominal adjunct can also occur with an intransitive verb:

\ea%x451
\label{ex:x451}
\gll Uura  or-op  \textstyleEmphasizedVernacularWords{arua  karu-e-mik}. \\
      \\
\glt
\z

night  descend-SS.SEQ  torch  run-PA-1/3p

`At night we went down to sea and fished with a torch.'

Those nominal adjunct plus verb structures where the verb is transitive look like two-object clauses, and in a few cases behave like them syntactically. In (\stepcounter{nx}{\thenx}) the nominal adjunct \textstyleStyleVernacularWordsItalic{kema} `liver' is in its normal adjunct position, but in (\stepcounter{nx}{\thenx}) it is in object \textstyleAcronymallcaps{NP} position. The basic meanings of the two sentences are the same, but with a different prominence: (\stepcounter{nx}{\thenx}) encodes marked negative focus and (\stepcounter{nx}{\thenx}) verb focus. The clause (\stepcounter{nx}{\thenx}) with an initial theme pronoun \textstyleStyleVernacularWordsItalic{yo} `I' is pragmatically more neutral than the others except in cases where  the initial pronoun receives extra stress. Note the intervening negator also in (\stepcounter{nx}{\thenx}). 

\ea%x452
\label{ex:x452}
\gll Me  efa  \textstyleEmphasizedVernacularWords{kema  suuw-a-k}. \\
      \\
\glt
\z

not  1s.ACC  liver  push-PA-3s

`He did \textstyleEmphasizedWords{\textsc{not}} think of me.'

\ea%x453
\label{ex:x453}
\gll \textstyleEmphasizedVernacularWords{Kema}  me  efa  \textstyleEmphasizedVernacularWords{suuw-a-k}. \\
      \\
\glt
\z

liver  not  1s.ACC  push-PA-3s

`He didn't \textstyleEmphasizedWords{\textsc{think}} of me.'

\ea%x1874
\label{ex:x1874}
\gll Yo  me  efa  \textstyleEmphasizedVernacularWords{kema  suuw-a-k}. \\
      \\
\glt
\z

1s.UNM  not  1s.ACC  liver  push-PA-3s

`He didn't think of me.'

In cases where the adjunct only occurs with a certain verb it is difficult to give it a specified meaning apart from the verb. The same is true for verbs that do not occur independently, only with an adjunct.

\ea%x454
\label{ex:x454}
\gll \textstyleEmphasizedVernacularWords{Naruw  ir-a-mik}. \\
      \\
\glt
\z

?  ascend-PA-1/3p

`They acted silly.'

\ea%x455
\label{ex:x455}
\gll Naap  \textstyleEmphasizedVernacularWords{kema  tuup-am-ika-i-ya}. \\
      \\
\glt
\z

thus  liver  ?-SS.SIM-be-Np-PR.3s

`He is hoping so.'

Most of the verbs in Mauwake indicating physiological or  psychological states and cognition are nominal adjunct plus verb constructions. The verb takes the person marking from the experiencer. The following list gives only a small sample of these constructions, where the most common nominal is \textstyleStyleVernacularWordsItalic{kema} `liver'.\footnote{A good list of these is in \citet[47-63]{Kwan1989}, where she has described a large number of body image concepts formed with \textit{kema} from semantic point of view. For that study the syntactic characteristics of the structures were not relevant.} The second column provides a literal translation. A few more examples of these constructions are in the sentences (\stepcounter{nx}{\thenx})-(\stepcounter{nx}{\thenx}).

kema enekar-  liver catch.fire  `be thirsty'

kema kaalal-  liver float  `be enthusiastic'

kema korin-  liver get.stuck  'be confused'

kema peelal-  liver rot  `be grieved'

kema ten-  liver collapse  `be relieved'

eneka maayar-  tooth become.long  `be hungry for meat'

miira ikiw-  face go  `feel dizzy'

\ea%x1490
\label{ex:x1490}
\gll Uura  \textstyleEmphasizedVernacularWords{uroma  ikiw-e-m}. \\
      \\
\glt
\z

night  stomach  go-PA-1s

`Last night I had diarrhea.'

\ea%x1487
\label{ex:x1487}
\gll \textstyleEmphasizedVernacularWords{Kema  samor-ar-ep}  maa  me  enim-i-yem. \\
      \\
\glt
\z

liver  spoil-INCH-SS.SEQ  food  not  eat-Np-PR.1s

`I am sad and don't eat.'

\ea%x1488
\label{ex:x1488}
\gll ...oko  \textstyleEmphasizedVernacularWords{emina} \textstyleEmphasizedVernacularWords{} \textstyleEmphasizedVernacularWords{urur}\textstyleEmphasizedVernacularWords{-}\textstyleEmphasizedVernacularWords{ep}  soomar-ikiw-i-kuan. \\
      \\
\glt
\z

...other  occiput  drop-SS.SEQ  walk-go-Np-FU.3p

`{\dots}lest they feel ashamed and walk away.'

\ea%x1489
\label{ex:x1489}
\gll Muuka  gelemuta  akena  \textstyleEmphasizedVernacularWords{kema}  me  \textstyleEmphasizedVernacularWords{puk}\textstyleEmphasizedVernacularWords{-}\textstyleEmphasizedVernacularWords{e}\textstyleEmphasizedVernacularWords{-}\textstyleEmphasizedVernacularWords{mik}. \\
      \\
\glt
\z

son  small  very  liver  not  burst-PA-1/3p

`Little boys/children do not think well (yet).'

\paragraph[Adverbial adjunct plus verb]{Adverbial adjunct plus verb}
\hypertarget{RefHeading21101935131865}{}
Adverbial adjuncts also have to precede the verb without any intervening words.  

\ea%x456
\label{ex:x456}
\gll Maamuma  efar  \textstyleEmphasizedVernacularWords{ikum  aaw-e-mik}. \\
      \\
\glt
\z

money  1s.DAT  illicitly  get-PA-1/3p

`They stole money from me.'

\ea%x457
\label{ex:x457}
\gll Maa  me  efa  \textstyleEmphasizedVernacularWords{pepek  er-a-k}. \\
      \\
\glt
\z

food  not  1s.ACC  enough  go-PA-3s

`The food wasn't enough for me.'

Some of the adverbial adjuncts, like \textstyleStyleVernacularWordsItalic{ikum} `illicitly' (\stepcounter{nx}{\thenx}) and \textstyleStyleVernacularWordsItalic{pepek} `enough' (\stepcounter{nx}{\thenx}), also function as independent adverbs, shown by an intervening pronoun (\stepcounter{nx}{\thenx}) and/or negator (\stepcounter{nx}{\thenx}).

\ea%x458
\label{ex:x458}
\gll Yo  oram  \textstyleEmphasizedVernacularWords{ikum}  efa  wu-a-n. \\
      \\
\glt
\z

1s.UNM  for.nothing  illicitly  1s.ACC  put-PA-2s

`You accused me for theft without grounds.'

\ea%x459
\label{ex:x459}
\gll No  \textstyleEmphasizedVernacularWords{pepek}  me  ma-e-n. \\
      \\
\glt
\z

2s.UNM  enough  not  say-PA-2s

`You didn't say right.'

Other adjuncts like \textstyleStyleVernacularWordsItalic{ane} `together' and \textstyleStyleVernacularWordsItalic{anu} `apart', only combine with verbs to form verbal groups, and it is hard to give them an exact meaning; the glosses below are just approximations.

\ea%x460
\label{ex:x460}
\gll Apura  \textstyleEmphasizedVernacularWords{ane  suuw-am-ika-iwkin}  pok-ap  ik-ok \\
      \\
\glt
\z

widow  together  push-SS.SIM-be-2/3p.DS  sit.down-SS.SEQ  be-SS  

om-o-k.

cry-PA-3s

`They were supporting the widow (sitting against her back) and she sat and wailed.'

\ea%x461
\label{ex:x461}
\gll Opora  \textstyleEmphasizedVernacularWords{anu  fien-owa}  me  pepek. \\
      \\
\glt
\z

talk  apart/aside  brush.off-NMZ  not  enough  

`He wasn't able to disregard the talk.'

It was mentioned above that the meanings of the adjunct plus verb combinations are often idiomatic rather than analytically derivable from the meanings of the parts. But this is a somewhat dangerous statement for one to make who comes from outside the speech community. For example, how literally \textstyleStyleVernacularWordsItalic{kema} `liver', which figures very strongly in the adjunct plus verb constructions, is understood to be really involved in the emotional and cognitive processes would need to be established in a separate study.

\section{Adverbs}
\hypertarget{RefHeading21121935131865}{}
Adverbs in Mauwake are a heterogeneous class morphologically, syntactically and semantically. Schachter's (1985:20) definition of adverbs as words functioning ``\textstyleBibliogCitationAAAstyleChar{as modifiers of constituents other than nouns}'' is quite usable for Mauwake. Functionally the adverbs can be divided into four groups. The \textstyleEmphasizedWords{\textsc{material}} adverbs \citep{Ahlman1933}\footnote{Ahlman used the term in classifying adverbs in Finnish, and I find it useful in describing the adverbs in Mauwake as well, since the temporal, locative and manner adverbs share some characteristics which differentiate them from the other adverbs.} form the largest group, which contains the subgroups of locative, temporal and manner adverbs. The second group, that of \textstyleEmphasizedWords{\textsc{intensity}} adverbs,\footnote{In some grammars these form a class of their own, called ``intensifiers''. But that name is somewhat misleading as it may contain words like \textstyleFootnoteBaseChar{\textit{somewhat}} or \textstyleFootnoteBaseChar{\textit{hardly}} which do not intensify the meaning of the adjacent adjective or adverb.} consists of a small group of adverbs that function on phrase level and modify an adjective or adverb. \textstyleEmphasizedWords{\textsc{Sentential}} (or \textstyleEmphasizedWords{\textsc{modal}}) adverbs modify a whole sentence. The last group consists of the two \textstyleEmphasizedWords{\textsc{free}} adverbs \textstyleStyleVernacularWordsItalic{pun} `also' and \textstyleStyleVernacularWordsItalic{muutiw} `only'.

A material adverb may function as the head of an adverbial phrase. In this respect, however, adverbs differ from most other word classes: whereas the head of a \textstyleAcronymallcaps{NP} is usually a noun, that of a \textstyleAcronymallcaps{VP} a verb and an \textstyleAcronymallcaps{AP} an adjective, an adverbial phrase typically either consists of an adverb only, or does not contain an adverb word at all (\sectref{sec:4.6}.). The material and sentential adverbs may be modified by an intensity adverb, in particular by \textstyleStyleVernacularWordsItalic{akena} `very, truly' (\stepcounter{nx}{\thenx}).

\ea%x462
\label{ex:x462}
\gll \textstyleEmphasizedVernacularWords{baliwep}  \textstyleEmphasizedVernacularWords{akena} \\
      \\
\glt
\z

well  very

`very well'

The position of adverbs within a clause is also discussed under adverbial phrase (\sectref{sec:4.6}).

\subsection{Material adverbs}
\hypertarget{RefHeading21141935131865}{}
The material adverbs function as peripherals in a clause. They are divided into locative, temporal, and manner adverbs.  The temporal and manner adverbs may be subdivided into deictic and non-deictic adverbs, and the locative adverbs are practically all deictic; in this they differ from the intensity and modal adverbs, which cannot be deictic.

\subsubsection[Locative adverbs]{Locative adverbs}
\hypertarget{RefHeading21161935131865}{}
All the non-controversial locative adverbs are deictic, and they were discussed above in section on spatial deictics (\sectref{sec:3.6.3}). 

\ea%x1933
\label{ex:x1933}
\gll {\dots}mokoma  kuisow  naap  \textstyleEmphasizedVernacularWords{fan}  yiam=iya  ik-e-mik. \\
      \\
\glt
\z

year  one  thus  here  1p.REFL=COM  be-PA-1/3p

`{\dots}for about a year they were here with us.'

\ea%x1934
\label{ex:x1934}
\gll {\dots}mua  owawiya  \textstyleEmphasizedVernacularWords{neeke}  ik-ok  uruf-ap{\dots}  kiiriw  ep-i-kuan. \\
      \\
\glt
\z

man  with  there.CF  be-SS  see-SS.SEQ  again  come-Np-FU.3p

`{\dots}having been with her husband there and seeing [her father] they will come (back) again.'

The words that are formed with a noun plus the locative clitic \nobreakdash-\textstyleStyleVernacularWordsxiiptItalic{pa} are treated as (adverbial) locative phrases, since they are expandable.

The words \textstyleStyleVernacularWordsxiiptItalic{mamaiya} `near, close' and \textstyleStyleVernacularWordsItalic{epasia} \footnote{\textit{Epasia} has probably developed from \textit{epa asia} `wild place'.} `far (away)' are actually locative nouns, but may be in the process of becoming adverbs. They optionally take the locative clitic \nobreakdash-\textstyleStyleVernacularWordsItalic{pa}, but its presence or absence causes no semantic difference. \textstyleStyleVernacularWordsItalic{Tiil}  `edgewise, close' cannot take the locative clitic. Its use is quite restricted, and it might be more accurately classified as a manner adverb. 

\ea%x467
\label{ex:x467}
\gll \textstyleEmphasizedVernacularWords{Epasia}  ikiw-em-ik-omkun  yia  far-e-k. \\
      \\
\glt
\z

far  go-SS.SIM-be-1s/p.DS  1p.ACC  call-PA-3s

`As we were (still) walking at a distance, he called us.'

\ea%x469
\label{ex:x469}
\gll Fikera  \textstyleEmphasizedVernacularWords{mamaiya=pa}  nan  pok-ap  {\dots} \\
      \\
\glt
\z

kunai.grass  near=LOC  there  sit-SS.SEQ  

`Having sat there near the kunai grass {\dots}'

\ea%x1856
\label{ex:x1856}
\gll Mua  oko=ke  \textstyleEmphasizedVernacularWords{mamaiya}  pok-a-k. \\
      \\
\glt
\z

man  other=CF  near  sit-PA-3s

`Another man slept with her (lit: sat near).'

\ea%x470
\label{ex:x470}
\gll Saapipia  baliwep  me  wu-a-m,  \textstyleEmphasizedVernacularWords{tiil}  wu-a-m. \\
      \\
\glt
\z

trap  well  not  put-PA-1s  on.edge  put-PA-1s

`I didn't put the trap well, I put it right on the edge (of the reef).'

Locative expressions that in some other languages would be expressed through pre- or postpositions or adverbs are formed with locative phrases containing locative relational nouns in Mauwake. 

\ea%x468
\label{ex:x468}
\gll koor  \textstyleEmphasizedVernacularWords{kuenuma}\textstyleEmphasizedVernacularWords{=pa} \\
      \\
\glt
\z

house  underside=LOC

`underneath (lit: in/on the underside of) the house'

\subsubsection[Temporal adverbs]{Temporal adverbs}
\hypertarget{RefHeading21181935131865}{}
The temporal adverbs can be classified semantically as deictic or non-deictic. The meaning of the former is tied to the time of the utterance, whereas the meaning of the latter is independent of it.  Both the deictic and non-deictic temporal adverbs are either specific or non-specific. This grouping is relevant on the syntactic level, as it influences the ordering of multiple temporal adverbials within a clause (\sectref{sec:4.6.2}).

\textstyleEmphasizedWords{\textsc{Deictic}}\textsc{} \textstyleEmphasizedWords{\textsc{specific}} temporal adverbs refer to a certain day in relation to the time of the utterance.\footnote{The only exception to this in the data is \textstyleFootnoteBaseChar{\textit{uurika}}, which in the forms \textstyleFootnoteBaseChar{\textit{uurik ona}}\textstyleEmphasizedVernacularWords{} (lit: `tomorrow place') and \textstyleFootnoteBaseChar{\textit{uurika naap nain}} (lit: `tomorrow thus that') means `the following day' and takes the time of the event as the deictic centre.} They are the following:

aakisa\footnote{\textstyleFootnoteBaseChar{\textit{Aakisa}} `today, now' may be either specific or non-specific.}  `today' 

unan  `yesterday'

erekema  `the day before yesterday'

uurika  `tomorrow'

ere    `the day after tomorrow'

arowona  `third day from today'

\ea%x471
\label{ex:x471}
\gll \textstyleEmphasizedVernacularWords{Unan}  nainiw  yiam  fiirim-e-mik. \\
      \\
\glt
\z

Yesterday  again  1p.REFL  gather-PA-1/3p

`Yesterday we met again.'

\ea%x472
\label{ex:x472}
\gll \textstyleEmphasizedVernacularWords{Uurika}  emeria  manina  ikiw-ep  en-owa  nop-ap \\
      \\
\glt
\z

tomorrow  woman  garden  go-SS.SEQ  eat-NMZ  fetch-SS.SEQ  

or-eka.

descend-IMP.2p

`You women, go to the garden tomorrow and fetch food (and come) down.'

The \textstyleEmphasizedWords{\textsc{deictic non-specific temporals}} refer to a time that is related to the time of the utterance (or in some cases to the time of the event), but is not restricted to a certain day. 

aakisa  `now'

aakisa fain  `nowadays,  now', literally: `now this'

aakisa fan  `just a while ago,  just now (past)', literally: `now here'

aakisa kuisow  `right now,  in a minute' (future), literally: `now one'

eewuar  `not yet'

iirakuma  `a few days ago'

iiriw  `already,  earlier,  long ago'

iiriwiw  `long time ago'

ikoka  `later'

ikoka kuisow  `right now' (future), literally: `later one'

uurik ona  `the following day', literally: `tomorrow place'

wiimar  `later, some other time' 

\ea%x473
\label{ex:x473}
\gll Aria,  no  \textstyleEmphasizedVernacularWords{aakisa}  maa  enim-e. \\
      \\
\glt
\z

alright  2s.UNM  now  thing/food  eat-IMP.2s

`Alright, eat now.'

\ea%x1215
\label{ex:x1215}
\gll \textstyleEmphasizedVernacularWords{Eewuar, } eka  me  saanar-owa  ik-ua. \\
      \\
\glt
\z

not.yet  water  not  dry-NMZ  be-PA.3s

`Not yet, the water hadn't dried.'

\ea%x474
\label{ex:x474}
\gll No  emeria  \textstyleEmphasizedVernacularWords{iiriw}  sesek-a-mik. \\
      \\
\glt
\z

2s.UNM  woman  already  send-PA-1/3p

`We already sent your wife (away).'

Both \textstyleStyleVernacularWordsItalic{ikoka} and \textstyleStyleVernacularWordsItalic{wiimar } mean `later', and they can occasionally be used interchangeably. \textstyleStyleVernacularWordsItalic{Ikoka} is the more common of the two, and has to be used when referring to a later time the same day.  \textstyleStyleVernacularWordsItalic{Wiimar}  always refers to a less specific time somewhere in the future, but the use of \textstyleStyleVernacularWordsItalic{ikoka} is spreading to cover that too. The sentence (\stepcounter{nx}{\thenx}) is from a wedding speech, and it was unlikely that the young couple would be fighting later the very same day.

\ea%x476
\label{ex:x476}
\gll \textstyleEmphasizedVernacularWords{Wiimar}  ikiw-i-yan,  \textstyleEmphasizedVernacularWords{ikoka}  weetak. \\
      \\
\glt
\z

later  go-Np-FU.1p  later  no

`We'll go some other time, not later today.'

\ea%x477
\label{ex:x477}
\gll No  \textstyleEmphasizedVernacularWords{ikoka}  mua  ikos  irak-ep  me  efar  kerer-e. \\
      \\
\glt
\z

2s.UNM  later  man  with  fight-SS.SEQ  not  1s.DAT  arrive-IMP.2s

`Later when you fight with your husband, don't come to me.'

\textstyleStyleVernacularWordsItalic{Aakisa} `now' can be modified to further specify the meaning, as the exact present moment is so short that a word referring to it only is practically useless. \textstyleStyleVernacularWordsItalic{Aakisa kuisow} (lit: `now one') refers to something that \textstyleEmphasizedWords{\textsc{will take place}} `just now', in a moment (\stepcounter{nx}{\thenx}), \textstyleStyleVernacularWordsItalic{aakisa fan} (lit: `now here') refers to something that \textstyleEmphasizedWords{\textsc{has happened}} just now (\stepcounter{nx}{\thenx}) and \textstyleStyleVernacularWordsItalic{aakisa fain} (lit: `now this') compares the present situation with earlier times (\stepcounter{nx}{\thenx}).

\ea%x478
\label{ex:x478}
\gll \textstyleEmphasizedVernacularWords{Aakisa  kuisow}  on-e,  ikoka  weetak. \\
      \\
\glt
\z

now  one  do-IMP.2s  later  no

`Do it right now, not later.'

\ea%x479
\label{ex:x479}
\gll Muuna  kirip-owa  ma-e-mik  nain  \textstyleEmphasizedVernacularWords{aakisa  fan}  kirip-a-mik. \\
      \\
\glt
\z

debt  return-NMZ  say-PA-1/3p  that1  now  here  return-PA-1/3p

`They (only) just now returned the debt they have talked about returning.'

\ea%x480
\label{ex:x480}
\gll Iiriw  miiw-aasa  marew,  \textstyleEmphasizedVernacularWords{aakisa  fain}  miiw-aasa  nepik  akena. \\
      \\
\glt
\z

earlier  land-canoe  none  now  this  land-canoe  crowd  real

`Earlier there were no cars, now(adays) there are lots of cars.'

The interpretation of the \textstyleEmphasizedWords{\textsc{non-deictic}} temporals is not tied to the time of the utterance or to the time of the event. The following ones are \textstyleEmphasizedWords{\textsc{specific}}:

uuriw  `morning'

amirika  `day(time), noon'

urera  `(late) afternoon'

uura  `evening/night'

uur gonegon\footnote{The word \textit{gonegon}, which I have not come across elsewhere, is a partial reduplication of the locative noun \textit{gone} `middle'. As a reduplication it is unusual in that the partial reduplication follows rather than precedes the root.}  `midnight'

epa wiiwim\footnote{This is a back-formation of the expression \textit{epa wii-wiim-ik-ua} [place RDP-dawn-be-PA.3s] `It is/was beginning to dawn'.}  `close to dawn' 

\ea%x698
\label{ex:x698}
\gll \textstyleEmphasizedVernacularWords{Amirika}  ama  kekan-eya  uurar-i-mik. \\
      \\
\glt
\z

day  sun  strong-2/3.DS  rest-Np-PR.1/3p

`During the day (or: at noon) when the sun is strong, we take a rest.'

\ea%x699
\label{ex:x699}
\gll Yaapan=ke  \textstyleEmphasizedVernacularWords{uura}  ifera=pa  nan  pok-om-ow-a-mik. \\
      \\
\glt
\z

Japan=CF  evening/night  sea=LOC  there  sit-BEN-CAUS-PA-1/3p

`In the evening the Japanese made him sit in the sea.'

The following temporal adverbs are both \textstyleEmphasizedWords{\textsc{non-deictic}} and \textstyleEmphasizedWords{\textsc{non-specific}}:

aawurun  `forever'

anane  `always', `every day'

ewur  `soon, quickly, fast'

ewursow  `soon, at once'

iir oko  `once upon a time, at some point' (lit: `(an)other time')

kiikir  `first'

kiiriw  `again'

mokomokoka  `first'

nainiw  `again'    ({{\textless}}nain=iw)

muri\footnote{This is an Austronesian borrowing (M. Ross, p.c.). It also occurs in the verb \textit{murar-} `follow', which has grammaticalized from the adjunct plus verb compound \textit{muri ar-} `behind become'.}  `later, behind'

\ea%x502
\label{ex:x502}
\gll Yo  \textstyleEmphasizedVernacularWords{anane}  naap  mauw-am-ika-i-yem. \\
      \\
\glt
\z

I  always  thus  work-SS.SIM-be-Np-PR.1p

`I always work like that.'

\ea%x504
\label{ex:x504}
\gll Irak-owa  maneka  \textstyleEmphasizedVernacularWords{ewur}  me  imen-ar-e-k. \\
      \\
\glt
\z

fight-NMZ  big  quickly  not  find-INCH-PA-3s

`The big fight/war didn't start quickly.'

\textstyleStyleVernacularWordsItalic{Kiiriw} and \textstyleStyleVernacularWordsItalic{nainiw}  both mean `again', and they can be used interchangeably when referring to repeated action. 

\ea%x697
\label{ex:x697}
\gll Ne  \textstyleEmphasizedVernacularWords{nainiw}  sande  uura  yiam  fiirim-e-mik. \\
      \\
\glt
\z

ADD  again  Sunday  evening  1p.REFL  gather-PA-1/3p

`And again on Sunday evening we gathered together.'

\ea%x1762
\label{ex:x1762}
\gll Ne  \textstyleEmphasizedVernacularWords{kiiriw}  enuma  on-am-ik-e-mik. \\
      \\
\glt
\z

ADD  again  new  make-SS.SIM-be-PA-1/3p

`And again they kept making a new one.'

When some action or event results in a state that is the same or similar as before, even if the action itself is not repeated, only\textstyleStyleVernacularWordsItalic{} \textstyleStyleVernacularWordsItalic{kiiriw} can be used. Thus only \textstyleStyleVernacularWordsItalic{kiiriw} is possible in (\stepcounter{nx}{\thenx}). \textstyleStyleVernacularWordsItalic{Kiiriw} indicates that Jesus is alive again, as he had been before, whereas \textstyleStyleVernacularWordsItalic{nainiw}  would indicate that he had risen from the dead earlier too. 

\ea%x503
\label{ex:x503}
\gll Yeesus  \textstyleEmphasizedVernacularWords{kiiriw}  iikir-a-k. \\
      \\
\glt
\z

Jesus  again  rise-PA-3s

`Jesus rose again (= rose from the dead).'

Also, if the action is the same but the situation changes, \textstyleStyleVernacularWordsItalic{kiiriw} is used. The example (\stepcounter{nx}{\thenx}) describes a situation where a grandmother first sent her younger grandchild, a girl, to listen to a sound. Later she sent the grandson for the same errand; the act of sending was repeated but the person who was sent changed:

\ea%x1761
\label{ex:x1761}
\gll \textstyleEmphasizedVernacularWords{Kiiriw}  morena  iperowa  nain  sesek-a-k. \\
      \\
\glt
\z

again  male  older  that1  send-PA-3s

`Again she sent the elder male (grandchild).'

Occasionally \textstyleStyleVernacularWordsItalic{kiiriw} and \textstyleStyleVernacularWordsItalic{nainiw}  can be used together:

\ea%x700
\label{ex:x700}
\gll Ar-ep  ik-eya  aria  \textstyleEmphasizedVernacularWords{kiiriw}  mua  nain \\
      \\
\glt
\z

become-SS.SEQ  be-2/3s.DS  alright  again  man  that1

\textstyleEmphasizedVernacularWords{nainiw}  urup-o-k.

again  ascend-PA-3s

`When she had become like that, alright the man came up again.'

\subsubsection[Manner adverbs]{Manner adverbs}
\hypertarget{RefHeading21201935131865}{}
The manner adverbial phrase is often manifested by just an adverb word rather than a longer phrase. The same distinction between deictic and non-deictic adverbs that was made with the other material adverbs can be made with the manner adverbs as well. The description of the deictic manner adverbs is in 3.6.4.

\ea%x1935
\label{ex:x1935}
\gll ...maa  oposia  pun  \textstyleEmphasizedVernacularWords{naap}  sesek-a-mik. \\
      \\
\glt
\z

thing  meat  also  thus  sell-PA-1/3p

`{\dots}like that they also sold meat.'

\ea%x1936
\label{ex:x1936}
\gll Soo  nain  \textstyleEmphasizedVernacularWords{feenap}:  era  erup  ik-ua. \\
      \\
\glt
\z

fishtrap  that1  like.this  way  two  be-PA.3s

`The fishtrap (custom) is like this: there are two ways.'

A few of the non-deictic manner adverbs have been derived from adjectives by the deletion of word-final /a/, but this process is not productive. Below is a list of some of the more common non-deictic manner adverbs.

ariman  `openly, publicly'

baliwep/balisow  `well'

damol/samor  `badly, poorly' (from \textstyleStyleVernacularWordsItalic{damola/samora} `bad')

ewur/ewuriw  `quickly'

ikum  `illicitly'

kapi  `askew'

kaken/kakeniw  `straight, correctly'

kekelka  `quietly, gently'

kerew  `strongly'

kokot  `secretly'

momasia  `slowly'  (cf.  adjective \textstyleStyleVernacularWordsItalic{momasia} `slow')

momor  `indiscriminately', `foolishly' (from \textstyleStyleVernacularWordsItalic{momora} `foolish')

pepek  `correctly'

oram/moram  `without reason', `without doing anything'\footnote{This word is difficult to gloss in English; its meaning is close to that of Tok Pisin \textit{nating}.}

orawin  `for the benefit'

\ea%x704
\label{ex:x704}
\gll Naap  yia  ma-i-kuan  na-ep  yo  \textstyleEmphasizedVernacularWords{ariman} \\
      \\
\glt
\z

thus  1p.ACC  say-Np-FU.3p  say/think-SS.SEQ  1s.UNM  openly

nefa  maak-i-yem.

2s.ACC  tell-Np-PR.1s

`I am telling you this openly, thinking that they will say like that about us.'

\ea%x505
\label{ex:x505}
\gll Opaimika  \textstyleEmphasizedVernacularWords{baliwep}  me  wiar  amis-ar-e-m. \\
      \\
\glt
\z

talk  well  not  3.DAT  knowledge-INCH-PA-1s

`I don't/didn't know their language well.'

\ea%x507
\label{ex:x507}
\gll Fikera  \textstyleEmphasizedVernacularWords{ikum}  kuum-e-mik  nain  ma-i-yem. \\
      \\
\glt
\z

kunai.grass  illicitly  burn-PA-1/3p  that1  say-Np-PR.1s

`I tell about that when the kunai grass was burned by arson.'

\ea%x506
\label{ex:x506}
\gll \textstyleEmphasizedVernacularWords{Samor}  akena  aruf-a-mik. \\
      \\
\glt
\z

badly  very  hit-PA-1/3p

`They beat him very badly.'

\subsection{Intensity adverbs}
\hypertarget{RefHeading21221935131865}{}
Intensity adverbs are a small and heterogeneous group of adverbs that modify a verb, an adjective, a quantifier or another adverb. Some of them (\textstyleStyleVernacularWordsItalic{akena, maneka}) are also adjectives, some others (\textstyleStyleVernacularWordsItalic{lawisiw, iiwawun, wenup}) are non-numeral quantifiers (\sectref{sec:3.4.2}) with a second function as intensity adverbs. The distribution is different for each of the intensity adverbs.

akena  `very, really, truly'

iiwawun  `altogether'

kakeniw  `exactly'

lawisiw/lawiliw  `somewhat'

maneka  `very'

oram  `very, just'

pepek  `enough'

wenup  `very'

\ea%x508
\label{ex:x508}
\gll Moma  fain  eliw(a)  \textstyleEmphasizedVernacularWords{oram}. \\
      \\
\glt
\z

taro  this  good  just/very  

`This taro is very good.'

\ea%x510
\label{ex:x510}
\gll Koora  nain  maala  \textstyleEmphasizedVernacularWords{pepek}. \\
      \\
\glt
\z

house  that  long  enough

`That house is long enough.'

\textstyleStyleVernacularWordsItalic{Akena} `really, truly' is more flexible than the other intensity adverbs in that it can modify a word belonging to almost any word class.

\ea%x706
\label{ex:x706}
\gll Eka  mamaiya  \textstyleEmphasizedVernacularWords{akena}  i  yoowa  me  aaw-i-yen \\
      \\
\glt
\z

river  near  very  1p.UNM  hot  not  get-Np-FU.1p

`Very near the river we'll not get hot.'

\ea%x708
\label{ex:x708}
\gll Iikamin  \textstyleEmphasizedVernacularWords{akena=ko}  imen-ar-i-non? \\
      \\
\glt
\z

when  really=NF  find-INCH-Np-FU.3s

`Exactly when is it going to appear?'

\ea%x709
\label{ex:x709}
\gll Sira  samora  piipu-eka  \textstyleEmphasizedVernacularWords{akena}. \\
      \\
\glt
\z

habit  bad  leave-IMP.2p  really

`You (pl) must really leave (your) bad habits.'

\ea%x710
\label{ex:x710}
\gll Yiena  ikos  \textstyleEmphasizedVernacularWords{akena}  iw-u. \\
      \\
\glt
\z

1p.GEN  two.together  really  go-IMP.1d

`Lets's go \textstyleEmphasizedWords{\textsc{just}} the two of us together.'

\ea%x1875
\label{ex:x1875}
\gll Weetak  \textstyleEmphasizedVernacularWords{akena},  i=ko  me  kuum-e-mik. \\
      \\
\glt
\z

no  really,  1p.UNM=NF  not  burn-PA-1/3p

`\textstyleEmphasizedWords{\textsc{Really no}}, we did not burn it.'

\textstyleStyleVernacularWordsItalic{Lawisiw} `somewhat' is different from the rest in that it precedes the expression it modifies, rather than following it.

\ea%x703
\label{ex:x703}
\gll Uuw-owa  nain  \textstyleEmphasizedVernacularWords{lawisiw } yoowa. \\
      \\
\glt
\z

work-NMZ  that1  somewhat  hot/hard

`That work is somewhat hard.'

As an adjective \textstyleStyleVernacularWordsItalic{maneka} `big' is very common, but as an intensity adverb `very' it is very restricted in its distribution. \textstyleStyleVernacularWordsItalic{Maneka} cannot modify a verb, but it can intensify some non-numeral quantifiers like \textstyleStyleVernacularWordsItalic{unowa} `many' and \textstyleStyleVernacularWordsItalic{iiwawun} `altogether', as well as the temporal adverb \textstyleStyleVernacularWordsItalic{anane} `always'. 

\ea%x509
\label{ex:x509}
\gll Yo  anane  \textstyleEmphasizedVernacularWords{maneka}  naap  mauw-am-ika-i-yem. \\
      \\
\glt
\z

I  always  very  thus  work-SS.SIM-be-Np-PR.1s

`I \textstyleEmphasizedWords{\textsc{always}} keep working like that.'

\subsection{Modal adverbs}
\hypertarget{RefHeading21241935131865}{}
The two modal adverbs in Mauwake differ from each other not only semantically, but morphologically and syntactically as well. Modality of a predication is discussed in \sectref{sec:6.1}.

\textstyleStyleVernacularWordsItalic{Eliw}  `all right, well'\footnote{The manner adverb `well' is \textstyleFootnoteBaseChar{\textit{baliwep} }\textstyleFootnoteBaseChar{(\sectref{sec:3.8.1.3})}.} is a deontic adverb and expresses permission or desirability: `it is all right/good that{\dots}'. It can often be translated with the auxiliary `may' in English. It follows the subject, if there is any, but precedes the other clause constituents (\stepcounter{nx}{\thenx}). It may also be in the tail position after the clause, either following a clause that already has \textstyleStyleVernacularWordsItalic{eliw} in it (\stepcounter{nx}{\thenx}), or by itself (\stepcounter{nx}{\thenx}). 

\ea%x514
\label{ex:x514}
\gll Wie  wi  \textstyleEmphasizedVernacularWords{eliw}  wiar  op-i-kuan. \\
      \\
\glt
\z

3s/p.uncle  3p.UNM  well  3.DAT  hold-Np-FU.3p

`Her uncles may get (lit: hold) them (=clay pots) from her.' 

\ea%x515
\label{ex:x515}
\gll \textstyleEmphasizedVernacularWords{Eliw}  Kululu  ma-e-man,  \textstyleEmphasizedVernacularWords{eliw}. \\
      \\
\glt
\z

well  Kululu  say-PA-2p  well

`It is all right that you mentioned Kululu, that is OK.'

\ea%x516
\label{ex:x516}
\gll Nomokowa,  nie  owowa=pa  fan  pok-a-n,  \textstyleEmphasizedVernacularWords{eliw}. \\
      \\
\glt
\z

2s/p.brother  2s/p.uncle  village=LOC  here  sit-PA-2s  well

`It is good/OK that you settled here in your brother's and uncle's village.'

An epistemic modal adverb is the clitic -\textstyleStyleVernacularWordsItalic{yon} (with an alternative form -\textstyleStyleVernacularWordsItalic{nion}), expressing hesitation or non-committal assumption: `perhaps', `maybe', `I suppose'.  It is attached to the predicate, which usually is a verb but can also be non-verbal (\stepcounter{nx}{\thenx}).

\ea%x517
\label{ex:x517}
\gll Maa  me  wu-om-a-mik=\textstyleEmphasizedVernacularWords{yon}. \\
      \\
\glt
\z

thing/food  not  put-BEN-BNFY2.PA-1/3p-perhaps

`Perhaps they didn't put food (aside) for him.'

\ea%x518
\label{ex:x518}
\gll Yo  me  efa  ma-e-n=\textstyleEmphasizedVernacularWords{yon}  aa? \\
      \\
\glt
\z

1s.UNM  not  1s.ACC  say-PA-2s-perhaps  aa

`I suppose you weren't saying it about me?'

\ea%x519
\label{ex:x519}
\gll Ni  kema  puk-owa  marewa=ke=\textstyleEmphasizedVernacularWords{yon}! \\
      \\
\glt
\z

2p.UNM  liver  burst-NMZ  none=CF-perhaps

 `You must be crazy!' (Lit: `I suppose your liver hasn't burst (yet).')

The question word \textstyleStyleVernacularWordsItalic{kamenion} `or what' is related to the modal adverb -\textstyleStyleVernacularWordsItalic{yon} (\sectref{sec:3.9.3}).

\subsection{Free adverbs}
\hypertarget{RefHeading21261935131865}{}
The adverbs \textstyleStyleVernacularWordsItalic{muut(a}\textstyleStyleVernacularWordsItalic{)}/\textstyleStyleVernacularWordsItalic{muutiw} `just/only' and \textstyleStyleVernacularWordsItalic{pun} `also, too' are called free adverbs, as they can move around quite freely and attach themselves to various elements in a clause. \textstyleStyleVernacularWordsItalic{Muutiw}  is a combination of \textstyleStyleVernacularWordsItalic{muut(a)} and the limiter clitic \nobreakdash-\textstyleStyleVernacularWordsItalic{iw},  and it restricts restricts the scope of a preceding noun phrase or adverbial phrase. \textstyleStyleVernacularWordsItalic{Muut(a)} is used almost exclusively with noun phrases.

\ea%x747
\label{ex:x747}
\gll Aaya  \textstyleEmphasizedVernacularWords{muutiw}  en-em-ika-i-mik. \\
      \\
\glt
\z

sugarcane  only  eat-SS.SIM-be-Np-PR.1/3p

`They are only eating sugarcane.' 

\ea%x748
\label{ex:x748}
\gll Ofa  sepa  \textstyleEmphasizedVernacularWords{muutiw } (if-o-k). \\
      \\
\glt
\z

paint  black  only  paint-PA-3s

`He painted with only black paint.'

\ea%x757
\label{ex:x757}
\gll Ewar  wuun-i-ya  nain  \textstyleEmphasizedVernacularWords{muutiw}  miim-i-nan. \\
      \\
\glt
\z

wind  blow-Np-PR.3s  that  only  hear-Np-FU.2s

`You will hear only the wind blowing.'

\ea%x758
\label{ex:x758}
\gll Lotu  koora  Ulingan=pa  \textstyleEmphasizedVernacularWords{muutiw}  ik-ua=i? \\
      \\
\glt
\z

worship  house  Ulingan=LOC  only  be-PA.3s=QM

`Is there a church only at Ulingan?'

\ea%x806
\label{ex:x806}
\gll Aakisa  \textstyleEmphasizedVernacularWords{muutiw}  niir-i-mik. \\
      \\
\glt
\z

today  only  play-Np-PR.1/3p

`They play only today.'

\ea%x807
\label{ex:x807}
\gll Eliw  \textstyleEmphasizedVernacularWords{muutiw}. \\
      \\
\glt
\z

well  only

`It's just all right.'

\ea%x1820
\label{ex:x1820}
\gll Yo  opora  \textstyleEmphasizedVernacularWords{muut}  naap. \\
      \\
\glt
\z

1s.UNM  talk  only  thus

`That's my talk.'

\ea%x1821
\label{ex:x1821}
\gll Uf-owa  erup  \textstyleEmphasizedVernacularWords{muuta}  naap   uf-e-mik. \\
      \\
\glt
\z

dance-NMZ  two  only  thus  dance-PA-1/3p

`We only danced two dances like that.'

\textstyleStyleVernacularWordsItalic{Pun} `also' has even wider distribution than \textstyleStyleVernacularWordsItalic{muutiw}: it can occur following almost any element in a clause.\footnote{\textit{Pun} may be in the process of developing into a clitic.  As a one-syllable word it it often has a weak stress, and some speakers also write it attached to the preceding word with a hyphen, the way clitics are written in the Mauwake orthography.} 

\ea%x749
\label{ex:x749}
\gll Ne  waaya  nain  \textstyleEmphasizedVernacularWords{pun}  afila  marew,  waaya  asia  \textstyleEmphasizedVernacularWords{pun.} \\
      \\
\glt
\z

and  pig  that1  also  grease  no(ne)  pig  wild  also

`And that pig also didn't have fat, (as) it was a wild pig too.'

\ea%x1937
\label{ex:x1937}
\gll Yos  \textstyleEmphasizedVernacularWords{pun}  wie  opora  nainiw  ma-i-yem. \\
      \\
\glt
\z

1s.FC  too  3s/p.uncle  talk  again  say-Np-PR.1s

`I, too, will again give ``uncle-talk'' (=cultural instruction).'

\ea%x750
\label{ex:x750}
\gll Ne  \textstyleEmphasizedVernacularWords{pun}  aakisa  iperowa  korokor  or-owa  sira \\
      \\
\glt
\z

and  also  now  middle.aged  initiation  descend-NMZ  custom

iiriw  wafur-a-mik.

earlier  throw-PA-1/3p

`Also, now the middle-aged people have already rejected the initiation custom.'

\ea%x751
\label{ex:x751}
\gll Iiriw  \textstyleEmphasizedVernacularWords{pun}  miiwa  muuta  nain  irak-owa  marew. \\
      \\
\glt
\z

earlier  also  ground  because.of  that1  fight-NMZ  no(ne)

`Earlier there were also no fights over ground' (or: `Earlier, too, there were no fights over ground.')

\ea%x808
\label{ex:x808}
\gll Teeria  maneka  wadol  opora  mik-a-mik  \textstyleEmphasizedVernacularWords{pun}  naap,  {\dots} \\
      \\
\glt
\z

group  big  lie  talk  hit-PA-1/3p  also  thus

`(When) the big group lied it was also like that, {\dots}'

\section{Negators} 
\hypertarget{RefHeading21281935131865}{}
Mauwake has four negators: \textstyleStyleVernacularWordsItalic{weetak}, \textstyleStyleVernacularWordsItalic{wia}, \textstyleStyleVernacularWordsItalic{me} and \textstyleStyleVernacularWordsItalic{marew}. They are morphologically free and syntactically heterogeneous, each one having its specific position. Of the four negators \textstyleStyleVernacularWordsItalic{me} is positioned before the negated element, while \textstyleStyleVernacularWordsItalic{marew} follows the negated element. \textstyleStyleVernacularWordsItalic{Weetak} and \textstyleStyleVernacularWordsItalic{wia} either form a complete utterance by themselves, or they are sentence-initial when used as negative interjections (\stepcounter{nx}{\thenx}) but clause-final when functioning as non-verbal predicates (\stepcounter{nx}{\thenx}), and when replacing full clauses they take the position of the clause they replace (\stepcounter{nx}{\thenx}), (\stepcounter{nx}{\thenx}). 

\ea%x654
\label{ex:x654}
\gll Maamuma  \textstyleEmphasizedVernacularWords{me}  tuun-owa  ik-e-mik. \\
      \\
\glt
\z

money  not  count-NMZ  be-PA-1/3p

`They haven't counted the money (yet).'

\ea%x1112
\label{ex:x1112}
\gll Mukuna  \textstyleEmphasizedVernacularWords{me}  op-a,  nefa  kuum-i-non! \\
      \\
\glt
\z

fire  not  touch-IMP.2s  2s.ACC  burn-Np-FU.3s

`Don't touch the fire, it will burn you!'

\ea%x655
\label{ex:x655}
\gll I  muuka  \textstyleEmphasizedVernacularWords{marew}. \\
      \\
\glt
\z

1p.UNM  son  no(ne).

`We have no son.'

\ea%x707
\label{ex:x707}
\gll \textstyleEmphasizedVernacularWords{Wia},  me  kookal-i-yem. \\
      \\
\glt
\z

No  not  like-Np-PR.1s  

`No, I don't like it.'

\ea%x1212
\label{ex:x1212}
\gll Yo  uuw-owa  oko  \textstyleEmphasizedVernacularWords{weetak}. \\
      \\
\glt
\z

1s.UNM  work-NMZ  other  no

`I have no other work.'

\ea%x705
\label{ex:x705}
\gll Wafur-a-k  na  \textstyleEmphasizedVernacularWords{weetak},  ufer-a-k. \\
      \\
\glt
\z

throw-PA-3s  but  no,  miss-PA-3s

`He threw it (a spear), but no (=he didn't succeed), he missed (the pig).'

\ea%x1111
\label{ex:x1111}
\gll Akup-a-mik,  akup-a-mik,  \textstyleEmphasizedVernacularWords{wia}. \\
      \\
\glt
\z

search-PA-1/3p  search-PA-1/3p  no

`We searched and searched, but no (=we did not find it).'

According to a rough generalization the most frequent negator \textstyleStyleVernacularWordsItalic{me} is basically a clause and constituent negator. It is also used to negate imperatives. \textstyleStyleVernacularWordsItalic{Weetak} and \textstyleStyleVernacularWordsItalic{wia} are negative interjections or predicates in verbless clauses, and \textstyleStyleVernacularWordsItalic{marew} can negate non-verbal predicates and occasionally noun phrase constituents. \textstyleStyleVernacularWordsItalic{Marew} often has the meaning `none at all'.

\textstyleStyleVernacularWordsItalic{Weetak}\textstyleStyleVernacularWordsItalic{} (\stepcounter{nx}{\thenx}), \textstyleStyleVernacularWordsItalic{wia} and occasionally \textstyleStyleVernacularWordsItalic{marew}, may be intensified by a postposed intensity adverb \textstyleStyleVernacularWordsItalic{akena} `truly, very'. \textstyleStyleVernacularWordsItalic{Me}  can only be intensified as a verbal negator, in which case \textstyleStyleVernacularWordsItalic{akena} comes after the verb rather than after the negator.

\ea%x652
\label{ex:x652}
\gll Ni  niam  erup  kema\textbf{  marew  akena}! \\
      \\
\glt
\z

2p.UNM  2p.REFL  two  liver  no(ne)  really

`The two of you have \textstyleEmphasizedWords{\textsc{really no}} sense at all!'

\ea%x653
\label{ex:x653}
\gll \textstyleEmphasizedVernacularWords{Me}  on-a-m  \textstyleEmphasizedVernacularWords{akena}. \\
      \\
\glt
\z

not  do-PA-1s  really

`I \textstyleEmphasizedWords{\textsc{really didn't}} do it.'

A fuller treatment of the negators is in \sectref{sec:6.2}, where negation as a functional category is discussed.\footnote{Bergh\"all (2006) gives a somewhat more comprehensive treatment of negation in Mauwake, but some of the analysis has changed since the writing of the article.}

\section{Connectives}
\hypertarget{RefHeading21301935131865}{}
The inventory of connectives in Mauwake is small. They are called connectives rather than conjunctions, because conjunctions are normally understood as a class of words, but in Mauwake a connective may be a word or a phrase. The term \textstyleEmphasizedWords{\textsc{conjunction}} is reserved for the conjunctive coordination construction (\sectref{sec:8.1.1}). Many of the connectives also have another primary function. 

The main division is into pragmatic and semantic connectives; all of them are coordinate. Subordination is discussed in \sectref{sec:8.3}. The connectives mostly operate on sentence level, joining clauses (\sectref{sec:8.1}). Almost all of the coordinators also conjoin sentences. Only the pragmatic connectives and the disjunctive connective \textstyleStyleVernacularWordsItalic{e} `or' are able to conjoin elements on the word and phrase levels as well. 

The most typical way of combining clauses is clause chaining through medial verbs, with no connective words at all (\sectref{sec:8.2}). When there are connectives, they are always placed between the two clauses. 

\subsection{Pragmatic connectives}
\hypertarget{RefHeading21321935131865}{}
Instead of clearly specifying the semantic relationship between the units they connect, like semantic connectives do, the pragmatic connectives signal a pragmatic relationship between them.\footnote{For this distinction on pragmatic and semantic connectives I am indebted to Stephen Levinsohn.} In Haspelmath's (2007:8) terms they are `medial [and] prepositive', meaning that they occur between the items they conjoin, and are linked more closely to the following constituent rather than the preceding one.

The connective \textstyleStyleVernacularWordsItalic{ne} `additive' only indicates that something is added to what has just been said.  It can connect word and phrase level units (\sectref{sec:4.1.2}), but is mostly used between clauses (\sectref{sec:8.1}) and even sentences. It is semantically neutral. When it conjoins words (\stepcounter{nx}{\thenx}) or phrases (\stepcounter{nx}{\thenx}), and often when it coordinates clauses (\stepcounter{nx}{\thenx}) or sentences (\stepcounter{nx}{\thenx}), it can be translated into English with `and'. 

\ea%x711
\label{ex:x711}
\gll kumin,  wutkekela  \textstyleEmphasizedVernacularWords{ne}  mera  ... \\
      \\
\glt
\z

hermit.crab  calamari  ADD  fish

`hermit crabs, calamari and fish {\dots}'

\ea%x713
\label{ex:x713}
\gll Inawera  sira  unowa,  \textstyleEmphasizedVernacularWords{ne}  kemena  unowa. \\
      \\
\glt
\z

dream  custom  many  ADD  inside  many

`There are many kinds of dreams, and (they have) many meanings.'

\ea%x714
\label{ex:x714}
\gll \textstyleEmphasizedVernacularWords{Ne}  yo  aakisa  tep=pa  ma-i-yem. \\
      \\
\glt
\z

ADD  1s.UNM  now  tape.recorder=LOC  say-Np-PR.1s

`And now I say it to a tape recorder.'

Words or phrases in lists are most commonly joined by juxtaposition only. If a connective is used, \textstyleStyleVernacularWordsItalic{ne} usually joins the last two (\stepcounter{nx}{\thenx}) coordinands. It is also possible to place the connective(s) closer to the beginning of the list.

\ea%x1359
\label{ex:x1359}
\gll Sesa  nain  waaya  erup  arow  \textstyleEmphasizedVernacularWords{ne}  maamuma  kuuma  erepam  ikur \\
      \\
\glt
\z

price  that1  pig  two  three  ADD  money  stick  four  five

\textstyleEmphasizedVernacularWords{ne}  manar  kuisow,  waa  eneka,  naap  muuka

ADD  forehead.ornament  one  pig  tooth  thus  son  

sesenar-i-nen.

buy-Np-FU.1s

 `(As for) the price, I will buy my son with two-three pigs and forty-fifty kina and a forehead ornament (and) pig's tusk(s), like that.'

There is no emphatic coordinate connective of the type `both {\dots} and' in Mauwake. 

If the propositions connected by \textstyleStyleVernacularWordsItalic{ne} contrast with each other in some way, it may be interpreted as adversative and translated into English with `but'.\footnote{Many Papuan languages have a connective that is glossed `and/but'. I suspect it is an additive connective like \textit{ne}, which is only interpreted as either `and' or `but' according to the content of the clauses conjoined.} In these cases it is always a ``weak'' adversative in contrast to the demonstrative \textstyleStyleVernacularWordsItalic{nain}  used in ``strong'' adversative clauses  (\sectref{sec:8.1.3}).

\ea%x715
\label{ex:x715}
\gll Maa  en-owa  iw-e-mik,  \textstyleEmphasizedVernacularWords{ne}  rais  weetak. \\
      \\
\glt
\z

thing  eat-NMZ  give.him-PA-1/3p  ADD  rice  no

`They gave him food, but not rice.'

\ea%x716
\label{ex:x716}
\gll Wi  me  kuum-e-mik,  \textstyleEmphasizedVernacularWords{ne}  wi  murar-owa=pa \\
      \\
\glt
\z

3p.UNM  not  burn-PA-1/3p  ADD  3p.UNM  follow-NMZ=LOC  

mukuna  nain  kerer-e-k.

fire  that  appear-PA-3s

`They didn't burn it, but the fire started after them.'

In a number of cases either neutral additive or contrastive interpretation is possible:

\ea%x1361
\label{ex:x1361}
\gll Wiam  erup  irak-ep  puk-e-mik,  aalbok=ke \\
      \\
\glt
\z

3p.REFL  two  fight-SS.SEQ  disperse-PA-1/3p  black.cuckoo-shrike=CF

ifera  or-o-k  ne  osaiwa=ke  soor(a)  asia  ikiw-o-k.

sea  descend-PA-3s  ADD  bird.of.paradise=CF  forest  wild  go-PA-3s

`The two of them fought and went their separate ways, the black cuckoo-shrike went down to the coast and/but the bird of paradise went to the wild (rain)forest.'

There are two discourse-marking pragmatic connectives, \textstyleStyleVernacularWordsItalic{aria} and \textstyleStyleVernacularWordsItalic{ne aria}.  They both mark discontinuity in the text.

\textstyleStyleVernacularWordsItalic{Aria} `alright'\footnote{The translation reflects the Tok Pisin word \textit{orait}, which sometimes has a similar discourse function. \textit{Aria} occurs in many Madang languages, and the speakers of those languages tend to use \textit{aria} in Tok Pisin too.}  usually comes sentence-initially, but can also or occur sentence-medially. Its main function is to indicate a break in the topic chain. In (\stepcounter{nx}{\thenx}) the topic changes from the snake to the man, and in (\stepcounter{nx}{\thenx}) from a health extension officer to a group of men:

\ea%x717
\label{ex:x717}
\gll Keraw-eya  \textstyleEmphasizedVernacularWords{aria}  nomokowa  gelemuta  puuk-ap  ifa  nain \\
      \\
\glt
\z

bite-2/3s.DS  alright  tree  small  cut-SS.SEQ  snake  that

ifakim-o-k.

kill-PA-3s

`It (=the snake) bit him, and he cut a small tree and killed the snake.'

\ea%x718
\label{ex:x718}
\gll {\dotso  miim-o-k. } \textstyleEmphasizedVernacularWords{Aria}  wi  kiiriw  neeke  {\dots} \\
      \\
\glt
\z

{\dots}3s.UNM  precede-PA-3s.  Alright  3p.UNM  again  there.CF  ...

`{\dots} he went ahead. (When) they were \textstyleEmphasizedWords{\textsc{there}} again {\dots}'

It often signals the beginning of a turn in a conversation (\stepcounter{nx}{\thenx}), or beginning of a speech (\stepcounter{nx}{\thenx}), again indicating a break with the preceding text. 

\ea%x721
\label{ex:x721}
\gll \textstyleEmphasizedVernacularWords{Aria}  wiipa,  i  yia  uruf-e. \\
      \\
\glt
\z

alright  daughter,  1p.UNM  1p.ACC  see-IMP.2s

`Daughter, look at us.'

\ea%x720
\label{ex:x720}
\gll \textstyleEmphasizedVernacularWords{Aria},  i  owowa=ko  urup-u. \\
      \\
\glt
\z

alright,  1p.UNM  village=NF  ascend-IMP.1d

`Alright, let's go back to the village.'

Even if the topic stays the same, \textstyleStyleVernacularWordsItalic{aria} can be used, especially when there is a contrast between alternatives (\stepcounter{nx}{\thenx}), or sometimes when an expected sequence of events is broken (\stepcounter{nx}{\thenx}).

\ea%x719
\label{ex:x719}
\gll Mua  maneka  maamuma  erup,  \textstyleEmphasizedVernacularWords{aria}  wi  suule  takira \\
      \\
\glt
\z

man  big  money  two  alright  3p.UNM  school  child  

maamuma  kuisow,  naap  omopora  sesenar-e-mik.

money  one  thus  door  buy-PA-1/3p

`The grown men paid two coins (=20 toea) for entrance, the schoolchildren one coin.'

\ea%x722
\label{ex:x722}
\gll Wiawi  onak  urera  maa  uup-e-mik,  \textstyleEmphasizedVernacularWords{aria} \\
      \\
\glt
\z

3s/p.father  3s/p.mother  evening  food  cook-PA-1/3p  alright

maa  me  wu-om-a-mik=yon.

food  not  put-BEN-BNFY2.PA-1/3p-perhaps

`In the evening his parents cooked food, (but) perhaps they didn't put any food for him.'

\textstyleStyleVernacularWordsItalic{Ne aria} `and alright' occurs less often than \textstyleStyleVernacularWordsItalic{aria}, and only sentence-initially. It marks major points of development in the plot of a story. 

\ea%x723
\label{ex:x723}
\gll Naap  wia  maak-e-mik.  \textstyleEmphasizedVernacularWords{Ne  aria},  ifa  nain  murar-ep{\dots} \\
      \\
\glt
\z

thus  3p.ACC  tell-PA-1/3p  ADD  alright  snake  that  follow-SS.SEQ

`They told them like that.  Now, the snake followed them and {\dots}'

Sometimes it also signals return to foreground text (i.e. main story line) after some backgrounded material. 

\subsection{Semantic connectives}
\hypertarget{RefHeading21341935131865}{}
The semantic connectives specify the relationship between two propositions. 

The disjunctive connective \textstyleStyleVernacularWordsItalic{e} `or' can connect not only propositions but words or phrases as well. It is used both for standard (\stepcounter{nx}{\thenx}) and interrogative (\stepcounter{nx}{\thenx}) disjunction\footnote{This terminology is from \citet{Haspelmath2007}.} (\sectref{sec:8.1.2}, 7.2.2). When there are two alternatives, the connective occurs between them. It is also common to have the question marker -\textstyleStyleVernacularWordsItalic{i}  cliticized to the end of the first alternative, especially in questions, but also elsewhere.

\ea%x724
\label{ex:x724}
\gll ama  arow  naap,  \textstyleEmphasizedVernacularWords{e}  erepam  naap,  {\dots} \\
      \\
\glt
\z

sun  three  thus  or  four  thus  

`at about three o'clock, or at about four {\dots}'

\ea%x725
\label{ex:x725}
\gll Emeria=ko  efar  uruf-a-man=\textstyleEmphasizedVernacularWords{i  e}  weetak? \\
      \\
\glt
\z

woman=NF  1s.DAT  see-PA-2p=QM  or  no

`Did you see my wife or not?'

When there are more alternatives than one and the question clitic is present, the connective may be left out altogether (\stepcounter{nx}{\thenx}), or it may occur between the first two alternatives (\stepcounter{nx}{\thenx}). 

\ea%x726
\label{ex:x726}
\gll maa  oposia=i  moma,  emera,  naap \\
      \\
\glt
\z

thing  meat=QM  taro,  sago,  thus

`meat, or taro, or sago, (things) like that'

\ea%x727
\label{ex:x727}
\gll iwer  eka=ki  \textstyleEmphasizedVernacularWords{e}  mauwa=ki,  a  episowa=ki, \\
      \\
\glt
\z

coconut  water=CF.QM  or  what=CF.QM  ah  tobacco=CF.QM,  

ufia=ki {\dots}

betel.pepper=CF.QM

`coconut juice or - ummm - tobacco, or betel pepper {\dots}'

The following consecutive connectives marking effect or result\footnote{It is typical for Papuan languages to mark the effect/result clause rather than the cause/reason clause. For a Papuan language which has several connectives both for result and for reason, see \citet[267-273]{Farr1999}.} are used in sentences where the clauses have a consecutive, i.e. a cause-effect or reason-result relationship: \textstyleStyleVernacularWordsItalic{naapeya/naeya}, \textstyleStyleVernacularWordsItalic{neemi}, and \textstyleStyleVernacularWordsItalic{naap nain}. They can all be glossed with `therefore, (and) so'.

\textstyleStyleVernacularWordsItalic{Naapeya/naeya} is the most generic and frequently used of the four.  \textstyleStyleVernacularWordsItalic{Naapeya} has developed from the manner adverb \textstyleStyleVernacularWordsItalic{naap} `thus' followed by the different-subject marker -\textstyleStyleVernacularWordsItalic{eya}\textstyleParagraphChari{ (\sectref{sec:3.8.3.5.2})};\footnote{Actually this connective in the coastal villages is \textit{naapera}, but because of the language committee's decision to use -\textit{eya} for the 2/3s.DS marker, this form is used here too.} the resulting meaning is `it being thus'. The origin of \textstyleStyleVernacularWordsItalic{naeya} is in the medial different-subject form of the verb \textstyleStyleVernacularWordsItalic{na}- `say, think'. The difference between the two is mainly dialectal, or areal: \textstyleStyleVernacularWordsItalic{naapeya} is used more on the coast, \textstyleStyleVernacularWordsItalic{naeya} in the inland. They are used for marking the effect or result clause in a consecutive sentence. 

\ea%x731
\label{ex:x731}
\gll I  maamuma  marew,  \textstyleEmphasizedVernacularWords{naapeya}  ifera=ko  me  sesenar-e-mik. \\
      \\
\glt
\z

1p.UNM  money  no(ne),  so  salt=NF  not  buy-PA-1/3p

`We didn't have money, so we didn't buy salt.'

\ea%x732
\label{ex:x732}
\gll Ben  uuw-owa  piipu-a-k.  \textstyleEmphasizedVernacularWords{Naapeya}  emina  urur-ep \\
      \\
\glt
\z

Ben  work-NMZ  left-PA-3s  therefore  occiput  fall-SS.SEQ

me  ekap-o-k.

not  come-PA-3s

`Ben has left the work. Therefore he was ashamed to come.'

\ea%x735
\label{ex:x735}
\gll Pika  oona  me  kekan-ow-a-k.  \textstyleEmphasizedVernacularWords{Naeya}  uura  ewar=ke \\
      \\
\glt
\z

wall  bone  not  strong-CAUS-PA-3s  therefore  night  wind=CF  

teek-a-k.

tear-PA-3s

`He didn't strengthen the wall studs. So at night the wind tore it (the house) down.'

\ea%x1413
\label{ex:x1413}
\gll I  miiw-aasa=pa  ekap-e-mik,  \textstyleEmphasizedVernacularWords{naeya}  o  me \\
      \\
\glt
\z

1p.UNM  land-canoe=LOC  come-PA-1/3p  therefore  3s.UNM  not  

yook-a-k.

follow.us-PA-3s

`We came in a car, so he didn't follow/come with us.'

The origin of \textstyleStyleVernacularWordsItalic{naeya} is so transparent that there are many cases where two different  interpretations for \textstyleStyleVernacularWordsItalic{naeya} are acceptable (\stepcounter{nx}{\thenx}). 

\ea%x734
\label{ex:x734}
\gll ``Yo  koka=pa  ik-e-m.''  \textstyleEmphasizedVernacularWords{Na-eya}  Magerka=ke  (ma-e-k){\dots} \\
      \\
\glt
\z

1s.UNM  jungle=LOC  be-PA-1s  say-2/3s.DS  MacArthur  (say-PA-3s)

` ``I was in the jungle.'' He said that, and (or: So) MacArthur said, {\dots}'

But in (\stepcounter{nx}{\thenx}) \textstyleStyleVernacularWordsItalic{naeya} clearly means `therefore' and cannot be interpreted as a medial verb, as the correct verb form in this case would be plural \textstyleStyleVernacularWordsItalic{naiwkin} `they said and{\dots}', not singular \textstyleStyleVernacularWordsItalic{naeya} `you/(s)he said and{\dots}'.

\ea%x733
\label{ex:x733}
\gll Iwera  yia  na-em-ik-e-mik. \\
      \\
\glt
\z

coconut  1s.ACC  say-SS.SIM-be-PA-1/3p

\textstyleEmphasizedVernacularWords{Naeya}  iwera  wia  uruk-am-ik-om-a-mik.

So  coconut  3p.ACC  drop-SS.SIM-be-BEN-BNFY2.PA-1/3p

`They kept asking us for coconuts. So we kept dropping coconuts for them.'

The originally dialectal difference may be developing into a semantic one. In the original text data from three decades ago there is no clear semantic distinction between the use of \textstyleStyleVernacularWordsItalic{naapeya} and \textstyleStyleVernacularWordsItalic{naeya}, but fairly recently when a group with members from different dialects, discussing language matters, produced consecutive clauses, nearly all of the sentences with \textstyleStyleVernacularWordsItalic{naapeya} were cases of cause-effect (\stepcounter{nx}{\thenx}), and all of the sentences with \textstyleStyleVernacularWordsItalic{naeya} were cases of reason and result (\stepcounter{nx}{\thenx}). 

\ea%x1414
\label{ex:x1414}
\gll I  fiirim-owa=pa  ik-emkun  ama  or-o-k, \\
      \\
\glt
\z

1p.UNM  gather-NMZ=LOC  be-1s/p.DS  sun  descend-PA-3s

\textstyleEmphasizedVernacularWords{naapeya}  epa  kokom-ar-e-k.

therefore  place  dark-INCH-PA-3s

`When we were in the meeting the sun went down, so it became dark.'

\ea%x1415
\label{ex:x1415}
\gll I  fiirim-owa=pa  ik-emkun  ama  or-o-k, \\
      \\
\glt
\z

1p.UNM  gather-NMZ=LOC  be-1s/p.DS  sun  descend-PA-3s

\textstyleEmphasizedVernacularWords{naeya}  maa  me  wiar  en-owa  ikiw-o-k.

therefore  food  not  3.DAT  eat-NMZ  go-PA-3s

`When we were in the meeting the sun went down, so he went without eating the food.'

\textstyleStyleVernacularWordsItalic{Naapeya} can also co-occur with the conjunctive coordinators \textstyleStyleVernacularWordsItalic{ne} or \textstyleStyleVernacularWordsItalic{aria}. In argumentation, \textstyleStyleVernacularWordsItalic{ne naapeya} or \textstyleStyleVernacularWordsItalic{aria naapeya} has to be used, when the reason is not confined to one clause but extends to a longer stretch of the discourse.

\ea%x1406
\label{ex:x1406}
\gll \textstyleEmphasizedVernacularWords{Aria}  \textstyleEmphasizedVernacularWords{naapeya}  niena  soomar-owa  ne  aakun-owa  pun \\
      \\
\glt
\z

alright  therefore  2p.GEN  walk-NMZ  ADD  talk-NMZ  also

sira  yi-e-k  nain  kaken=iw  ook-ap  soomar-eka.

custom  give.us-PA-3s  that1  straight=LIM  follow-SS.SEQ  walk-IMP.2p

`So therefore, as concerns your walk and talk too, follow straight the behaviour that he gave us and walk that way.'

\textstyleStyleVernacularWordsItalic{Neemi}  is used only in reasoning. It requires some point of similarity between the antecedent and the result clause.

\ea%x736
\label{ex:x736}
\gll Teeria  fain  K10  wu-a-mik.  \textstyleEmphasizedVernacularWords{Neemi}  wi  teeria  nain  pun \\
      \\
\glt
\z

group  this  K10  put-PA-1/3p  therefore  3p.UNM  group  that1  too

K10  wu-a-mik.

K10  put-PA-1/3p

`This group put down K10. Thefore that group put down K10, too.'

\textstyleStyleVernacularWordsItalic{Naap nain} can be translated into English with `therefore', `in that case', `if so, then'. It is made up of the manner adverb \textstyleStyleVernacularWordsItalic{naap} `thus' and the distal demonstrative \textstyleStyleVernacularWordsItalic{nain} `that'.  It is a strong connective, stressing the fact that the proposition following the connective is a logical conclusion from the preceding proposition.

\ea%x737
\label{ex:x737}
\gll Ni  moma  uup-i-man=i?  \textstyleEmphasizedVernacularWords{Naap  nain}  yo  saa \\
      \\
\glt
\z

2p.UNM  taro  cook-Np-2p=QM  thus  that  1s.UNM  rice  

uup-i-nen.

cook-Np-FU.1s

`Are you cooking taro? In that case I'll cook rice.'

It is much less common in Mauwake to mark the reason clause than the result clause with a connective.  When the reason clause is emphasized, it is marked with the connective \textstyleStyleVernacularWordsItalic{moram} (\textstyleStyleVernacularWordsItalic{wia}) `because' and always follows the result clause rather than preceding it. The origin of the reason connective is in the question word \textstyleStyleVernacularWordsItalic{moram} `why?' and the negator \textstyleStyleVernacularWordsItalic{wia} `no(t)'.\footnote{\textit{Moram} as a reason connective is probably a calque on Tok Pisin \textit{bilong wanem} `why, because'. The negator, which does not influence the meaning of the connective, may have been added in Mauwake to help distinguish the connective from the question word.} The difference between \textstyleStyleVernacularWordsItalic{moram wia} and \textstyleStyleVernacularWordsItalic{moram} is that the former is mainly used across sentence boundary (\stepcounter{nx}{\thenx}), and the latter within a sentence (\stepcounter{nx}{\thenx}).

\ea%x738
\label{ex:x738}
\gll Maamuma  senam  aaw-e-mik.  \textstyleEmphasizedVernacularWords{Moram}  \textstyleEmphasizedVernacularWords{wia},  maa  ele-eliwa \\
      \\
\glt
\z

money  a.lot  get-PA-1/3p  why  not  thing  RDP-good  

sesek-a-mik. 

sell-PA-1/3

`They got a lot of money. (That's) because they sold good things/foods.'

\ea%x739
\label{ex:x739}
\gll Miiw-aasa  muf-owa  me  ikiw-e-mik,  \textstyleEmphasizedVernacularWords{moram}  os=ke  naap \\
      \\
\glt
\z

land-canoe  pull-NMZ  not  go-PA-1/3p  why  3s.FC=CF  thus

ar-eya.

become-2/3s.DS

`We didn't go to fetch a truck, because she had become like that (=died).'

\section{Postpositions and clitics}
\hypertarget{RefHeading21361935131865}{}
Since Mauwake is an \textstyleAcronymallcaps{SOV} language, it is natural that it has postpositions rather than prepositions.  But their number is small: besides the comitative postpositions there are only two others, one for comparison and one indicating reason.

Unlike the postpositions, which are both phonological and grammatical words, clitics are grammatical words that together with the preceding word form one  phonological word. The stress assignment rule does not affect them: they are always unstressed.  If there are any derivational and inflectional suffixes in the host word, the clitics  are added after all of them (Dixon 2010a:221\nobreakdash-2).

The nominal clitics associate with noun phrases and attach themselves phonologically to the last element of the noun phrase. They mark either the case role or the pragmatic function of the \textstyleAcronymallcaps{NP}. The only sentential clitic is the question marker \nobreakdash-\textstyleStyleVernacularWordsItalic{i}.  The modal clitic -\textstyleStyleVernacularWordsItalic{yon} `perhaps' was discussed above in 3.9.3.

The postpositions and clitics are discussed together because of their shared origin in some cases, causing similarity in form, and because some of them have  similarities in function.

\subsection{Comitative clitic and postpositions}
\hypertarget{RefHeading21381935131865}{}
Accompaniment, or a\textstyleEmphasizedWords{\textsc{ comitative}} relationship may be expressed by one clitic or by five different postpositions, three of which are formed with the clitic. 

The comitative clitic is -\textstyleStyleVernacularWordsItalic{iya} `with, and, both {\dots} and'. The clitic may be attached to either of the two related \textstyleAcronymallcaps{NP}s, or to both.

\ea%x775
\label{ex:x775}
\gll Nan  pok-ap-ik-e-mik,  \textstyleEmphasizedVernacularWords{mua=iya  emeria}. \\
      \\
\glt
\z

there  sit-SS.SEQ-be-PA-1/3p  man=COM  woman.

`They were sitting there, (both) husband and wife.'

\ea%x776
\label{ex:x776}
\gll \textstyleEmphasizedVernacularWords{Muuka  wiip=iya}  kerer-e-mik. \\
      \\
\glt
\z

son  daughter=COM  appear-PA-1/3p

`(Both) a son and a daughter appeared.'

\ea%x777
\label{ex:x777}
\gll \textstyleEmphasizedVernacularWords{Bom=iya  kateres=iya,  bom=iya  kateres=iya}  (fuurk-a-mik). \\
      \\
\glt
\z

bomb=COM  cartridge=COM  bomb=COM  cartridge=COM  drop-PA-1/3p

`They dropped (both) bombs and cartridges, (both) bombs and cartridges.'

It combines with pronouns to form comitative pronouns (\sectref{sec:3.5.9}), and the word for `all', \textstyleStyleVernacularWordsItalic{unowiya}, is made up of \textstyleStyleVernacularWordsItalic{unowa} `many' plus the comitative clitic.

Occasionally the clitic can also be used to indicate instrument.  

\ea%x778
\label{ex:x778}
\gll Mauwa  ar-e-n,  \textstyleEmphasizedVernacularWords{amia=iya}  nenar-e-mik=i? \\
      \\
\glt
\z

what  become-PA-2s  bow=COM  shoot.you-PA-1/3p=QM

`What happened to you, did they shoot you with a gun?'

\textstyleStyleVernacularWordsItalic{Owawiya}\textstyleStyleVernacularWordsItalic{/owawik} `with, together with' is used only for humans; it can refer to two or more people. It can also occur by itself (\stepcounter{nx}{\thenx}). The origin of the first part \textstyleStyleVernacularWordsItalic{owaw}- is unknown. The second part is either the comitative clitic -\textstyleStyleVernacularWordsItalic{iya} or the root of the existential verb \textstyleStyleVernacularWordsItalic{ik}\nobreakdash- `be', a reflection of an earlier construction \textstyleStyleVernacularWordsItalic{owawiya ik}\nobreakdash- `be together'. 

\ea%x821
\label{ex:x821}
\gll Yoli  onak  \textstyleEmphasizedVernacularWords{owawiya}  efa  amukar-e-mik. \\
      \\
\glt
\z

Yoli  3s/p.mother  with  1s.ACC  scold-PA-1/3p

`Yoli and his mother scolded me.'

\ea%x822
\label{ex:x822}
\gll \textstyleEmphasizedVernacularWords{Owawiya}  feeke  pok-ap  ik-ok  soomar-ek-eka. \\
      \\
\glt
\z

with  here.CF  sit-SS.SEQ  be-SS  walk-go-IMP.2p

`(First) sit here with us and (then) go.'

\ea%x1817
\label{ex:x1817}
\gll Iikir-ami  onak  \textstyleEmphasizedVernacularWords{owawik}  soomar-e-mik. \\
      \\
\glt
\z

get.up-SS.SIM  3s/p.mother  with  walk-PA-1/3p

`He got up and walked with his mother.'

The postposition \textstyleStyleVernacularWordsItalic{onaiya/onaria}\textstyleStyleVernacularWordsItalic{/onaiyik} may be based on the third person singular genitive pronoun ona and the clitic -\textstyleStyleVernacularWordsItalic{iya}; \textstyleStyleVernacularWordsItalic{onaiyik} also includes the root of the verb \textstyleStyleVernacularWordsItalic{ik}\nobreakdash- `be'.  This postposition is more generic and can be used with noun phrases referring to people (\stepcounter{nx}{\thenx}) but is also common when referring to things (\stepcounter{nx}{\thenx}). When the relationship between the two noun phrases is unequal, \textstyleStyleVernacularWordsItalic{onaiya} may used, like in (\stepcounter{nx}{\thenx}), where the other people carried the sick man. The subject marking on the verb is influenced by how active part the referents of the comitative \textstyleAcronymallcaps{NP} take in the action. When all the participants are active, the subject marking on the verb is plural. The speech in (\stepcounter{nx}{\thenx}) was directed towards the villagers who were instructed to stay away from the Japanese troops. 

\ea%x754
\label{ex:x754}
\gll Mua  unowa  \textstyleEmphasizedVernacularWords{onaiya}  ikiw-e-mik. \\
      \\
\glt
\z

man  many  with  go-PA-1/3p

`We went with many people.'

\ea%x755
\label{ex:x755}
\gll Urom(a)  \textstyleEmphasizedVernacularWords{onaiya}  ik-ua. \\
      \\
\glt
\z

stomach  with  be-PA.3s

`She is/was pregnant.'

\ea%x823
\label{ex:x823}
\gll Mua  napuma  \textstyleEmphasizedVernacularWords{onaiya}  Medebur  ek-a-mik. \\
      \\
\glt
\z

man  sick  with  Medebur  go-PA-1/3p

`They went to Medebur with the sick man '

\ea%x1819
\label{ex:x1819}
\gll No  ara  sepa  ara  kia  \textstyleEmphasizedVernacularWords{onaiyik}  bilik \\
      \\
\glt
\z

2s.UNM  trunk  black  trunk  white  together  mixed

ar-i-nan=na  {\dots}

become-Np-FU.2s=TP

`If you, a black person, are together with the white people mixed with them {\dots}'

Another comitative postposition that mainly refers to things is \textstyleStyleVernacularWordsItalic{feekiya} `with'. It originates from the combination of \textstyleStyleVernacularWordsItalic{feeke} `here' and -\textstyleStyleVernacularWordsItalic{iya} `comitative', but the meaning does not reflect the deictic origin of the initial part. In those rare cases when it is attached to a [+human] \textstyleAcronymallcaps{NP}, the referent of this \textstyleAcronymallcaps{NP} is subordinate to the referent of the other \textstyleAcronymallcaps{NP} and not in control, but still influencing the subject marking of the verb (\stepcounter{nx}{\thenx}). 

\ea%x824
\label{ex:x824}
\gll Mokok  urupa  kaik-i-man  nain  \textstyleEmphasizedVernacularWords{feekiya}  baurar-eka. \\
      \\
\glt
\z

eye  cup  tie-Np-PR.2p  that1  with  flee-IMP.2p

`Flee with your ``eye cups'' (a singsing decoration) still on.'

\ea%x825
\label{ex:x825}
\gll Wiamun  gelemuta  pun  aaw-ep  \textstyleEmphasizedVernacularWords{feekiya}  ikiw-e-mik. \\
      \\
\glt
\z

3s/p.brother  small  also  take-SS.SEQ  with  go-PA-1/3p

`He took his little brother too, and went with him.'

\ea%x1818
\label{ex:x1818}
\gll Maa  eliw  akena  nain  aaw-ep  \textstyleEmphasizedVernacularWords{feekiya}  ikiw-o-k. \\
      \\
\glt
\z

thing  good  very  that1  take-SS.SEQ  with  go-PA-3s

`He took the very good thing and went with it.'

The dual comitative \textstyleStyleVernacularWordsItalic{ikos} `with, together (with)' can only be used when two human participants are referred to (\stepcounter{nx}{\thenx}). It can also occur alone, without a preceding noun phrase, when the participants are known from the person/number suffix in the verb, or from the context (\stepcounter{nx}{\thenx}). The parties are considered equally active, so the verb is always in the plural.

\ea%x752
\label{ex:x752}
\gll Wekera  \textstyleEmphasizedVernacularWords{ikos}  irak-e-mik. \\
      \\
\glt
\z

3s/p.sister  with  fight-PA-1/3p

`He fought with his sister.'

\ea%x753
\label{ex:x753}
\gll \textstyleEmphasizedVernacularWords{Ikos}  ikiw-i-yen. \\
      \\
\glt
\z

with  go-Np-FU.1p

`Let's go together (just the two of us).'

Associative \textstyleStyleVernacularWordsItalic{ame} `with others' is different from the comitative postpositions above in that only one of the parties is specified. The identity of `the others' is left unspecified.

\ea%x826
\label{ex:x826}
\gll Auwa  \textstyleEmphasizedVernacularWords{ame}  wia  maak-eya  res  aaw-ep \\
      \\
\glt
\z

1s/p.father  ASSOC  3p.ACC  tell-2/3s.DS  razor  take-SS.SEQ  

merena  puuk-a-mik.

leg  cut-PA-1/3p

`He told my father and the others, and they took a razor and made a cut on his leg.'

\ea%x827
\label{ex:x827}
\gll Kuuten  \textstyleEmphasizedVernacularWords{ame}=ke  miim-e-mik. \\
      \\
\glt
\z

Kuuten  ASSOC=CF  precede-PA-1/3p

`Kuuten with (some) others went ahead of them.'

\subsection{Reason postposition \textit{muuta (nain)}}
\hypertarget{RefHeading21401935131865}{}
\textstyleStyleVernacularWordsItalic{Muuta}\textstyleStyleVernacularWordsItalic{ (nain)} `because of, for' gives a reason for an action, when the reason is expressed in a noun phrase rather than a full clause. It has developed from the adverb \textstyleStyleVernacularWordsItalic{muuta} `a little, only', and in some cases the meaning `only' is retained with the new function as well (\stepcounter{nx}{\thenx}), (\stepcounter{nx}{\thenx}). The distal-1 demonstrative \textstyleStyleVernacularWordsItalic{nain} `that' is optional, and is left out especially when there is another demonstrative \textstyleStyleVernacularWordsItalic{nain} preceding \textstyleStyleVernacularWordsItalic{muuta} (\stepcounter{nx}{\thenx}).

\ea%x756
\label{ex:x756}
\gll Iiriw  miiwa  \textstyleEmphasizedVernacularWords{muuta  nain}  irak-owa  marew,  oram \\
      \\
\glt
\z

earlier  land  for  that1  fight-NMZ  no(ne)  just  

momor  mauw-am-ik-e-mik.

indiscriminately  work-SS.SIM-be-PA-1/3p

`Earlier there was no fighting for land, they just worked indiscriminately (on any land).'

\ea%x1876
\label{ex:x1876}
\gll Yia  amukar-owa  \textstyleEmphasizedVernacularWords{muuta  nain}  nan  iiriw  ifakim-e-mik. \\
      \\
\glt
\z

1p.ACC  scold-NMZ  for  that1  there  earlier  kill-PA-1/3p

`We killed her earlier (only) because she scolded us (lit: {\dots}for her scolding of us).'

\ea%x759
\label{ex:x759}
\gll Opora  ara  nain  \textstyleEmphasizedVernacularWords{muuta}  ifakim-u  na-ep  on-a-mik. \\
      \\
\glt
\z

Talk  section  that1  for  kill-IMP.1d  say-SS.SEQ  do-PA-1/3p

`(Only) because of that talk they tried to kill him.'

\subsection{Comparison postposition \textit{saarik}}
\hypertarget{RefHeading21421935131865}{}
\textstyleStyleVernacularWordsItalic{Saarik} `like' occurs with noun phrases (\stepcounter{nx}{\thenx}) and with nominalized clauses (\stepcounter{nx}{\thenx}).  It indicates a point of similarity between two essentially \textstyleEmphasizedWords{different} things.

\ea%x760
\label{ex:x760}
\gll Mua  eliwa  \textstyleEmphasizedVernacularWords{saarik}  aakun-e-k. \\
      \\
\glt
\z

man  good  like  speak-PA-3s

`He spoke like a good man.'

\ea%x761
\label{ex:x761}
\gll No  sia  on-owa  \textstyleEmphasizedVernacularWords{saarik}  magimal  puuk-a-n. \\
      \\
\glt
\z

2s.UNM  netbag  make-NMZ  like  vine.sp.  cut-PA-2s

`You cut \textstyleForeignWords{magimal} vine as if you were going to make a netbag.'

For the functional category of comparison, see \sectref{sec:6.5}.

\subsection{Locative clitic -\textit{pa}}
\hypertarget{RefHeading21441935131865}{}
The locative clitic -\textstyleStyleVernacularWordsItalic{pa}  mainly marks locative in noun phrases (\stepcounter{nx}{\thenx}). The most common verb that it collocates with is \textstyleStyleVernacularWordsItalic{ik}- `be' (\stepcounter{nx}{\thenx}). When it occurs with the directional verbs (\sectref{sec:3.8.4.4.5}), it often indicates source (\stepcounter{nx}{\thenx}), but it can also be used for path (\stepcounter{nx}{\thenx}), or for instrument in cases where it has a strong locative meaning as well (\stepcounter{nx}{\thenx}). It is rarely used for goal (\stepcounter{nx}{\thenx}); this is possible in cases where the goal is the location for an event taking place immediately.

\ea%x762
\label{ex:x762}
\gll Pon  piipa  unowa=\textbf{pa}  soomar-em-ik-eya  mik-a-m. \\
      \\
\glt
\z

turtle  seaweed  many=LOC  walk-SS.SIM-be-2/3s.DS  spear-PA-1s

`The turtle was walking among the seaweeds and I speared it.'

\ea%x1880
\label{ex:x1880}
\gll Ona  owowa=\textstyleEmphasizedVernacularWords{pa}  ik-eya  epa  wiim-o-k. \\
      \\
\glt
\z

3s.GEN  village=LOC  be-2/3s.DS  place  dawn-PA-3s

`When he was in his village it dawned.'

\ea%x1877
\label{ex:x1877}
\gll Ifa  maneka=ke  iinan=\textstyleEmphasizedVernacularWords{pa}  or-o-k. \\
      \\
\glt
\z

snake  big=CF  on.top=LOC  descend-PA-3s

`A big snake dropped from above.'

\ea%x770
\label{ex:x770}
\gll Saa=\textstyleEmphasizedVernacularWords{pa}  ir-am-ika-i-mik. \\
      \\
\glt
\z

sand=LOC  come/go-SS.SIM-be-Np-PR.1/3p

`They are coming on/along the beach.'

\ea%x771
\label{ex:x771}
\gll Miiw  aasa=\textstyleEmphasizedVernacularWords{pa}  ikiw-e-mik. \\
      \\
\glt
\z

land  canoe=LOC  go-PA-1/3p

`We went by car/in a car.'

\ea%x1879
\label{ex:x1879}
\gll Mua  nain  ...  eka  kapa\textstyleEmphasizedVernacularWords{=pa}  ir-ap  eka  nain \\
      \\
\glt
\z

man  that1  ...  river  top=LOC  come/go-SS.SEQ  river  that1  

up-o-k.

block-PA-3s

`The man went to the top/source of the river and blocked the river.'

As temporal phrases locate an event in time, they also use the same locative clitic (\stepcounter{nx}{\thenx}).

\ea%x763
\label{ex:x763}
\gll Fraide=\textstyleEmphasizedVernacularWords{pa}  maapora  puk-o-k,  urera. \\
      \\
\glt
\z

Friday=LOC  celebration  burst-PA-3s,  afternoon.

`On Friday the celebration started, in the afternoon.'

It can also be used with an essive meaning, when referring to people's jobs:

\ea%x765
\label{ex:x765}
\gll Yena  mua  owowa  ekap-o-k,  amia  mua=\textstyleEmphasizedVernacularWords{pa}  ik-ok. \\
      \\
\glt
\z

1s.GEN  man  village  come-PA-3s  bow  man=LOC  be-SS 

`My husband came back to the village, having been a policeman.'

The locative clitic has its origin in the word \textstyleStyleVernacularWordsItalic{epa} `place'; the transition vowel [e] can sometimes be heard between the clitic and its host, when the host word ends in a consonant (\stepcounter{nx}{\thenx}).

\ea%x764
\label{ex:x764}
\gll Ne  Sarak  ikos  Gawar=(\textstyleEmphasizedVernacularWords{e})\textstyleEmphasizedVernacularWords{pa}  ik-emkun  yia  maak-e-mik  {\dots} \\
      \\
\glt
\z

ADD  Sarak  with  Gawar=LOC  be-1s/p.DS  1p.ACC  tell-PA-1/3p

`And as Sarak and I were in Gawar, they told us, {\dots} '

\subsection{Instrumental clitic -\textit{iw}}
\hypertarget{RefHeading21461935131865}{}
The instrumental clitic -\textstyleStyleVernacularWordsItalic{iw}\textstyleStyleVernacularWordsItalic{} is used both for concrete (\stepcounter{nx}{\thenx}) and abstract (\stepcounter{nx}{\thenx}) instruments.\footnote{A less emphasized way to add an instrument is to use the chaining structure: `take instrument do something' (\stepcounter{nx}{\thenx}), (\stepcounter{nx}{\thenx}).} 

\ea%x766
\label{ex:x766}
\gll Nomokowa  galua-galua  nain=\textstyleEmphasizedVernacularWords{iw}  biiris  on-am-ik-e-mik \\
      \\
\glt
\z

tree  soft-soft  that1=INST  bridge  make-SS.SIM-be-PA-1/3p

`They kept making bridges with soft timber.'

\ea%x768
\label{ex:x768}
\gll ...wiena  opaimik=\textstyleEmphasizedVernacularWords{iw}  yia  maak-em-ik-e-mik. \\
      \\
\glt
\z

3p.GEN  mouth=INST  1p.ACC  tell-SS.SIM-be-PA-1/3p

`They talked to us in their language.'

It is also used for path and has the meaning `along'; the verb indicates action that continues for some time.

\ea%x767
\label{ex:x767}
\gll Saa=\textstyleEmphasizedVernacularWords{iw}  ir-am-ika-i-mik, ... \\
      \\
\glt
\z

sand=INST  ascend-SS.SIM-be-Np-PR.3p

`They are coming along the beach{\dots}'

The difference between (\stepcounter{nx}{\thenx}) and (\stepcounter{nx}{\thenx}) above is that in (\stepcounter{nx}{\thenx}) the people were coming along the beach at least some of the way, and more specifically at the time of the speaking; whereas (\stepcounter{nx}{\thenx}) indicates that they travelled along the beach more or less the whole way.

The instrumental may also be utilized to indicate manner: 

\ea%x1881
\label{ex:x1881}
\gll Uurik  ona  naap=\textstyleEmphasizedVernacularWords{iw}  iw-ap  poka  aaw-e-mik. \\
      \\
\glt
\z

tomorrow  place  thus=INST  go-SS.SEQ  housepost  get-PA-1/3p

`The following day they went in the same way and got houseposts.'

\ea%x773
\label{ex:x773}
\gll Karu-(o)w=\textstyleEmphasizedVernacularWords{iw}  ekap-o-k. \\
      \\
\glt
\z

run-NMZ=INST  come-PA-3s

`He came running.'

\ea%x1814
\label{ex:x1814}
\gll Ne  ikoka  maa  marew  eliw \textstyleEmphasizedVernacularWords{} manek=\textstyleEmphasizedVernacularWords{iw}  ika-i-nan. \\
      \\
\glt
\z

ADD  later  thing  none  well  big=INST  be-Np-FU.2s

`And later you will have no problems, you will just be very well.'

\ea%x774
\label{ex:x774}
\gll Waaya=ke  anane  wiar  en-ow=\textstyleEmphasizedVernacularWords{iw}  ika-i-ya. \\
      \\
\glt
\z

pig=CF  always  3.DAT  eat-NMZ=INST  be-Np-PR.3s

`A pig stays eating their (taro) all the time.'

Another usage is in those temporal phrases that refer to something taking place repeatedly at the same time: 

\ea%x1882
\label{ex:x1882}
\gll I  amirik=\textstyleEmphasizedVernacularWords{iw}  ...  Gawar  wiar  ikiw-e-mik. \\
      \\
\glt
\z

1p.UNM  daytime=INST  {\dots}  Gawar  3.DAT  go-PA-1/3p

`In the daytime we always went to Gawar {\dots}'

\subsection{Limiter -\textit{iw}}
\hypertarget{RefHeading21481935131865}{}
The limiter clitic -\textstyleStyleVernacularWordsItalic{iw}  `only, just' is homophonous with the instrumental. The two probably are of common origin, but synchronically their meanings and positions in the word are distinct (\sectref{sec:3.12.9}). The limiter does not mark a case but restricts the applicability of the predication to the element that it is attached to.  

\ea%x769
\label{ex:x769}
\gll [Maa  eka]\textsubscript{NP}=\textstyleEmphasizedVernacularWords{iw}  en-ep  en-ep  lebum-ar-i-nan. \\
      \\
\glt
\z

food  water=LIM  eat-SS.SEQ  eat-SS.SEQ  lazy-INCH-Np-FU.2s

`When you keep eating only food cooked with water you become tired of it.'

\ea%x772
\label{ex:x772}
\gll Mua=\textstyleEmphasizedVernacularWords{iw}  pok-aka. \\
      \\
\glt
\z

man=LIM  sit-IMP.2p

`Sit just among the men.'

The limiter clitic may attach itself to genitive and focal pronouns (\sectref{sec:3.5.7}). 

The free adverb \textstyleStyleVernacularWordsItalic{muutiw} `only' is a combination of \textstyleStyleVernacularWordsItalic{muut(a)} `only' and the limiter clitic.

\subsection{Topic and focus markers}
\hypertarget{RefHeading21501935131865}{}
The topic and focus markers indicate the discourse function of the noun phrases that they are attached to.

\subsubsection[Topic markers]{Topic markers}
\hypertarget{RefHeading21521935131865}{}
Of the two topic markers \textstyleStyleVernacularWordsItalic{ena} is fairly low in frequency, and the description given here is only tentative. It seems that \textstyleStyleVernacularWordsItalic{ena} as an independent word originally had a topic marking function, but later the topic clitic \textstyleStyleVernacularWordsxiiptItalic{-(e)na}  developed from it and is now used for highlighted topic (\sectref{sec:9.1.2.4}) in main clauses. \textstyleStyleVernacularWordsItalic{Ena} still marks a topic, but only in relative clauses. It often has a specifying function as well: `the/that (particular one)'.  

\ea%x1681
\label{ex:x1681}
\gll [\textstyleEmphasizedVernacularWords{Mua}  \textstyleEmphasizedVernacularWords{ena}  ma-e-k  nain]  makena  yos. \\
      \\
\glt
\z

man  SPEC  say-PA-3s  that1  true  1s.FC

`The man that he talked about is I.'

In long relative clauses, where it is attached to the \textstyleAcronymallcaps{RelNP}, it helps to distinguish it from all the other \textstyleAcronymallcaps{NP}s in the clause. 

\ea%x1815
\label{ex:x1815}
\gll [\textstyleEmphasizedVernacularWords{Mua}  \textstyleEmphasizedVernacularWords{papako}  \textstyleEmphasizedVernacularWords{ena}  Australia=ke  wia  aaw-ep  wiena \\
      \\
\glt
\z

man  other  SPEC  Australia=CF  3p.ACC  take-SS.SEQ  3p.GEN

feekiya  yiaw-e-mik  nain]  me  epa  fan  irak-owa

with  walk.around-PA-1/3p  that1  not  place  here  fight-NMZ  

uruf-a-mik {\dots}

see-PA-1/3p

`Those other (particular) men whom the Australians took and with whom they walked around did not see the war here in this place {\dots}'

\ea%x1683
\label{ex:x1683}
\gll [\textstyleEmphasizedVernacularWords{I}  mua  owowa=pa  ik-ok  \textstyleEmphasizedVernacularWords{ena}  irakowa  uruf-a-mik \\
      \\
\glt
\z

1p.UNM  man  village=LOC  be-SS  SPEC  fight-NMZ  see-PA-1/3p

nain]  nanar  nain  yo  fan  ma-i-yem.

that1  story  that1  1s.UNM  here  say-Np-PR.1s

`I am telling the story of us (particular) people who stayed in the village and saw the fighting.'

If the head noun of the \textstyleAcronymallcaps{NP} is is recoverable from the context, it may be deleted, leaving behind only \textstyleStyleVernacularWordsItalic{ena}. In (\stepcounter{nx}{\thenx}) the head noun \textstyleStyleVernacularWordsItalic{epira} `bowl(s)' has been omitted.

\ea%x1682
\label{ex:x1682}
\gll [Aakisa  fan  \textstyleEmphasizedVernacularWords{ena}  maneka  wu-a-mik  nain]  eliw, \\
      \\
\glt
\z

now  here  SPEC  big  put-PA-1/3p  that1  well  

wie  wi  eliw  wiar  op-i-kuan.

3s/p.uncle  3p.UNM  well  3.DAT  grab-Np-FU.3p

`Those big (bowls) that we put just now, all right, the uncles may take those from them.'

The following example, taken from Bible translation, has a highlighted topic marker -\textstyleStyleVernacularWordsItalic{na} on the sentence-initial topic \textstyleAcronymallcaps{NP,} which is part of the main clause\textstyleAcronymallcaps{,} and \textstyleStyleVernacularWordsItalic{ena} inside the relative clause:

\ea%x1816
\label{ex:x1816}
\gll Ni  Samaria=na  [\textstyleEmphasizedVernacularWords{o  ena}  me  baliwep  amis-ar-e-man \\
      \\
\glt
\z

2p.UNM  Samaria=TP  3s.UNM  SPEC  not  well  knowledge-INCH

nain]\textsubscript{RC}  lotu  on-i-man.

that1  worship  do-Np-PR.2p

`You Samaritans worship [the one that you do not know well].'

Without \textstyleStyleVernacularWordsItalic{ena} in the relative clause the sentence would mean `You Samaritans do not know him well but (still) worship him.'

The more common topic clitic \textstyleStyleVernacularWordsItalic{-(e)na}\footnote{The clitic has mostly lost the phoneme /e/, but it can sometimes be heard when the host word ends in a voiceless consonant. }  is used to highlight a changed topic, to which attention is drawn. The topic may have been introduced in the immediately preceding clause. The use of this device is infrequent in texts.  It can often be glossed with `as for X'. Highlighted topics are discussed in \sectref{sec:9.1.2.4}.

Example (\stepcounter{nx}{\thenx}) is from a traditional story, where a man has gone hunting and the spirit of his lover comes to his home. When the wife sees her, she knows what the spirit woman has come to look for and comments:

\ea%x779
\label{ex:x779}
\gll Nena  mua=\textstyleEmphasizedVernacularWords{na}  urema  osarena  ikiw-o-k. \\
      \\
\glt
\z

2s.GEN  man=TP  bandicoot  path  go-PA-3s

`(As for) your husband, he went to catch bandicoots.'

In (\stepcounter{nx}{\thenx}) the answer to the question reveals the identity of the person asked about; the topic marker may be used even in a short exchange like this but especially if the text continues to tell more about the topic. 

\ea%x1751
\label{ex:x1751}
\gll Mua  nain  naareke?  Mua  nain=\textbf{na}  owow  saria  maneka=ke. \\
      \\
\glt
\z

Man  that1  who.CF  man  that1=TP  village  headman  big=CF

`Who is that man? -That man is the big village headman.'

In (\stepcounter{nx}{\thenx}) the speaker changes the topic to the addressee after a discussion on something else:

\ea%x780
\label{ex:x780}
\gll Nos=\textstyleEmphasizedVernacularWords{na}? \\
      \\
\glt
\z

2s.FC=TP

`(So,) what about you?'

An important function for the topic marker -\textstyleStyleVernacularWordsItalic{na}  is to mark conditional clauses (\sectref{sec:8.3.5}). This is a common function for topic markers in Papuan languages (Haiman 1978, Reesink 1983b and 1987:242, Foley 1986:203).

\ea%x744
\label{ex:x744}
\gll Opora  wiar  ika-i-ya=\textstyleEmphasizedVernacularWords{na}  eliw  urup-ep  wia \\
      \\
\glt
\z

talk  3.DAT  be-Np-PR.3s=TP  well  ascend-SS.SEQ  3p.ACC  

maak-uk.

tell-IMP.3p

`If they have something to say, they can get up and tell them.'

\ea%x745
\label{ex:x745}
\gll O  emeria  aaw-owa  kookal-ek-a-k=\textstyleEmphasizedVernacularWords{na} \\
      \\
\glt
\z

3s.UNM  woman  get-NMZ  like-CNTF-PA-3s=TP  

iw-ek-a-mik.

give.him-CNTF-PA-1/3p

`If he had liked to get a wife, they would have given him one.'

It is also used in adversative subordinate clauses (\sectref{sec:8.3.4}) when the main clause expresses a frustrated effort or a cancelled expectation (\stepcounter{nx}{\thenx}), or surprise.

\ea%x1399
\label{ex:x1399}
\gll Ikiw-ep  mukuna  nain  umuk-a-mik=\textstyleEmphasizedVernacularWords{na}  me  pepek. \\
      \\
\glt
\z

go-SS.SEQ  fire  that1  extinguish-PA-1/3p=TP  not  enough/able

`We went and (tried to) extinguish the fire, but couldn't.'

\subsubsection[Focus clitics]{Focus clitics}
\hypertarget{RefHeading21541935131865}{}
There are two focus clitics, the contrastive focus marker -(\textstyleStyleVernacularWordsItalic{e})\textstyleStyleVernacularWordsItalic{ke} and the neutral focus marker -\textstyleStyleVernacularWordsItalic{ko}, which has developed from the indefinite \textstyleStyleVernacularWordsItalic{oko} `a certain, other'.\footnote{Most of this section is based on J\"arvinen (1988b:81-96).} The main candidate for the contrastive focus marker is the subject of a noun phrase (\stepcounter{nx}{\thenx}). When the object is fronted as a theme (\sectref{sec:9.1}), the subject usually gets the contrastive focus marking to distinguish it from the object (\stepcounter{nx}{\thenx}).  

\ea%x781
\label{ex:x781}
\gll Iiriw  ifa  marasin=\textstyleEmphasizedVernacularWords{ke}  kekan-e-k. \\
      \\
\glt
\z

earlier  snake  poison=CF  be.strong-PA-3s

`The snake poison had already taken effect.'

\ea%x782
\label{ex:x782}
\gll Episowa  ifa  nain  atua=\textstyleEmphasizedVernacularWords{ke}  en-e-k. \\
      \\
\glt
\z

tobacco  leaf  that1  worm=CF  eat-PA-3s

`The tobacco leaves were eaten by worms.'

Another possible host is the non-verbal predicate of a verbless clause (\sectref{sec:5.6}).

\ea%x783
\label{ex:x783}
\gll Iperuma  nain  me  enim-eka,  inasin  mua=\textstyleEmphasizedVernacularWords{ke}. \\
      \\
\glt
\z

eel  that1  not  eat-IMP.2p  spirit  man=CF

`Do not eat that eel, it is a spirit man.'

There are a few isolated cases where it occurs on some other contrasted element of a clause.

\ea%x784
\label{ex:x784}
\gll Amirika=\textstyleEmphasizedVernacularWords{ke}  eliw  ika-i-yem,  uura=\textstyleEmphasizedVernacularWords{ke}  napum-ar-i-yem. \\
      \\
\glt
\z

noon=CF  well  be-Np-PR.1s  night=CF  sickness-INCH-Np-PR.1s

`At noon I'm well, at night I am sick.'

Contrastive focus as a pragmatic device in a text is discussed in \sectref{sec:9.3.1}.

The neutral focus clitic \textstyleStyleVernacularWordsxiiptItalic{-ko} commonly occurs in irrealis-type clauses:\footnote{Because of this, it was called \textit{Irrealis focus clitic} in J\"arvinen (1988b).} questions, commands, negative clauses, or those with future tense. Unlike the contrastive focus marker, the neutral focus marker can be attached to almost any element of a clause except the final verb. 

\ea%x785
\label{ex:x785}
\gll Mukuna=\textstyleEmphasizedVernacularWords{ko}  op-a-man=i? \\
      \\
\glt
\z

fire=NF  hold-PA-2p=QM

`Did you hold any fire?'

\ea%x786
\label{ex:x786}
\gll Mua  nain=\textstyleEmphasizedVernacularWords{ko}  onak-e! \\
      \\
\glt
\z

man  that1=NF  give.to.eat-IMP.2s

`Give it to that man to eat!'

\ea%x787
\label{ex:x787}
\gll Oposia  en-e-man  nain  yo=\textstyleEmphasizedVernacularWords{ko}  me  uruf-a-m. \\
      \\
\glt
\z

meat  eat-PA-2p  that1  1s.UNM=NF  not  see-PA-1s

`I didn't (even) see the meat that you ate.'

\ea%x788
\label{ex:x788}
\gll Akim-ap=\textstyleEmphasizedVernacularWords{ko}  uruf-i-yen. \\
      \\
\glt
\z

try-SS.SEQ=NF  see-Np-FU.1p

`We'll try and see.'

Neutral focus as a textual device is discussed further in \sectref{sec:9.3.2}.

\subsection{Question marker}
\hypertarget{RefHeading21561935131865}{}
The question marker -\textstyleStyleVernacularWordsItalic{i}  is a sentential clitic, used to form polar questions, and it attaches itself to the clause-final verb or another clause-final element. Its relationship to the alternative connective \textstyleStyleVernacularWordsItalic{e} `or' (\sectref{sec:3.11.2}) is unclear; it is possible that \textstyleStyleVernacularWordsItalic{i} was originally an alternative connective but was also employed as a question marker and became so established in this function that a new  alternative connective \textstyleStyleVernacularWordsItalic{e}  developed. It is not uncommon in \textstyleAcronymallcaps{TNG} Madang languages that the question marker and the alternative connective are either the same or closely related.\footnote{In Usan and Amele the alternative connector and the question marker are the same (Reesink 1987:293, Roberts 1987:99). Maia has \textit{-i} `QM' and \textit{e} `or' like Mauwake \citep[83,159]{Hardin2002}. Bargam has borrowed the Tok Pisin \textit{o} as an alternative connector but has retained the clitic \textit{-e} as the question marker \citep[53,122]{Hepner2002}. Kobon may use the alternative interrogative connector \textit{aka} `or' sentence-finally in leading polar questions, where the speaker expects the addressee to agree with the proposition. }   

\ea%x789
\label{ex:x789}
\gll Sira  nain  piipua-i-nan=\textstyleEmphasizedVernacularWords{i}? \\
      \\
\glt
\z

habit  that1  leave-Np-FU.2s=QM

`Will you give up that habit?'

\ea%x790
\label{ex:x790}
\gll Nobonob  ikiw-e-man  nain,  owowa  eliwa=\textstyleEmphasizedVernacularWords{i}? \\
      \\
\glt
\z

Nobonob  go-PA-2p  that1  village  good=QM

`You went to Nobonob, is it a good village?'

The question marker can also be used in statements, when two or more alternatives are given:

\ea%x791
\label{ex:x791}
\gll Mua  kuisow  manina  erup=\textstyleEmphasizedVernacularWords{i}  (e)  arow=\textstyleEmphasizedVernacularWords{i}  (e)  naap. \\
      \\
\glt
\z

man  one  garden  two=QM  (or)  three=QM  (or)  thus

`A man can have two gardens, or three, like that.'

\subsection{Co-occurrence of the clitics}
\hypertarget{RefHeading21581935131865}{}
It is possible to have two or more clitics attached to the same word; only the topic marker does not allow other clitics with it.  A case marking clitic, forming a constituent with the preceding noun phrase, is placed first. Next comes the focus clitic (\stepcounter{nx}{\thenx}), with the scope over a phrase but not forming a constituent. The limiter \nobreakdash-\textstyleStyleVernacularWordsItalic{iw}  may have a phrase with the focus marking in its scope, so it follows the focus marker.  When the limiter follows another clitic, there is a transition consonant /r/  between the clitics.The modal clitic \nobreakdash-\textstyleStyleVernacularWordsItalic{yon} `perhaps' (\sectref{sec:3.9.3}) may have a scope over a whole predication, so its position is after the limiter. The sentential clitic, with a scope over the whole sentence, comes last. When the contrastive focus marker -\textstyleStyleVernacularWordsItalic{ke} and the question marker -\textstyleStyleVernacularWordsItalic{i} are adjacent they become a portmanteau clitic -\textstyleStyleVernacularWordsItalic{ki}  (\stepcounter{nx}{\thenx}).

\ea%x792
\label{ex:x792}
\gll Fura=\textstyleEmphasizedVernacularWords{iw}=\textstyleEmphasizedVernacularWords{ko}  me  puuk-a-mik. \\
      \\
\glt
\z

knife=INST=NF  not  cut-PA-1/3p

`They didn't cut it with a knife.'

\ea%x795
\label{ex:x795}
\gll Fikera=\textstyleEmphasizedVernacularWords{pa}-r=\textstyleEmphasizedVernacularWords{iw}  fiirim-eka. \\
      \\
\glt
\z

kunai.grass=LOC-{\O}=LIM  gather-IMP.2p

`Gather them right at the \textstyleForeignWords{kunai} grass.'

\ea%x796
\label{ex:x796}
\gll Os=\textstyleEmphasizedVernacularWords{ke}-r=\textstyleEmphasizedVernacularWords{iw}  maa  en-emi  ewur-ar-e-k. \\
      \\
\glt
\z

3s.FC=CF-{\O}=LIM  food  eat-SS.SIM  haste-INCH-PA-3s

`Only he rushed with his food.'

\ea%x793
\label{ex:x793}
\gll Wi  anim  onoma  pun  o  makena  Krais=ke \\
      \\
\glt
\z

3p.UNM  blade  basis  also  3s.UNM  truly  Christ=CF

na-i-mik=\textstyleEmphasizedVernacularWords{yon}=\textstyleEmphasizedVernacularWords{i}?

say-Np-PR.1/3p=perhaps=QM

`Are also the authorities perhaps saying that he truly is Christ?'

\ea%x794
\label{ex:x794}
\gll Mua  fain  Saror  muuka=\textstyleEmphasizedVernacularWords{ki}? \\
      \\
\glt
\z

man  this  Saror  son=CF.QM

`Is this man Saror's son?'

\section{Interjections}
\hypertarget{RefHeading21601935131865}{}
There are a lot of interjections in Mauwake; the following list is not even nearly exhaustive. The pronunciation of interjections may differ from that of other words: intonational variations are greater, and lengthening, even extreme lengthening, of the final vowel is common. Interjections are not part of the normal clause structure, and they usually occur either sentence-initially or finally. Some can also be placed between clauses in a coordinate sentence (\stepcounter{nx}{\thenx}). The glosses given below for the interjections are just rough approximations.  

a      impatience, also used as a filler

aa    `oh'  emphasizes what has been said

ae    `yes'  agreement\footnote{The negators \textit{weetak} and \textit{wia} `no' also function as interjections but are not listed here, because they have other functions as well (see \sectref{sec:3.10} and \sectref{sec:6.2}).} 

aiyoo    distress, disapproval

arika\footnote{This is obviously an imperative derived from the discourse-marker \textit{aria} (\sectref{sec:3.9.1}), as it is only used with second person plural, whereas \textit{aria} is used with all other persons. Elsewhere \textit{aria} has no verb-like qualities.}  `OK, let's go'  exhorting others to get going

awue  `wow  any strong emotion, surprise

ee      delight

ei    `hey!'  being surprised or startled

emawa  `sorry'  expression of empathy, especially grief or pity

emawik  `excuse me'  speech opening in a controversial situation

faa      disgust or astonishment

maa senam  `watch out!'  grave warning (lit: `thing too.much')

oo    `o'  calling someone

oo [oo[241?]]  `yes'  agreement

na      strengthens an imperative

nii?  `really? \textstyleDefinitionE{oh?'  r}\textstyleEncyclopedicinfoE{esponse to hearing something surprising} 

nom\textstylefstandard{  `please'  when repeating a request or command}

noma\textstylefstandard{  `oh dear' } 

sa, se    impatience, disapproval

se-ek  `wow  great happiness

wiisak  `sorry'  mild regret, for minor losses

yaa      impatience

yee    `oh'  recognition, emphasis  

yii    `eek', `oh'  fear, sorrow

\ea%x797
\label{ex:x797}
\gll \textstyleEmphasizedVernacularWords{Aa},  kema=ko  kir-ek-a-n  \textstyleEmphasizedVernacularWords{aa}! \\
      \\
\glt
\z

oh  liver=NF  turn-CNTF-PA-2s  oh

`Oh, if only you had changed your ways (lit: turned your liver)!'

\ea%x798
\label{ex:x798}
\gll \textstyleEmphasizedVernacularWords{Arika},  takira,  yo  yook-eka. \\
      \\
\glt
\z

let's.go  boy  1s.UNM  follow.me-IMP.2p

`Alright boys, follow me (and will get going).'

\ea%x1225
\label{ex:x1225}
\gll Laman  tapala  wu-a-k,  \textstyleEmphasizedVernacularWords{aiyoo}! \\
      \\
\glt
\z

Laman  hat  put-PA-3s  INTJ

`My goodness, Laman put a hat on (and exposed himself and us to the fighter pilots)!'

\ea%x799
\label{ex:x799}
\gll \textstyleEmphasizedVernacularWords{Emawa},  nena  niawi  um-o-k. \\
      \\
\glt
\z

sorry  2s.GEN  2s/p.father  die-PA-3s

`Sorry, your father is dead.'

\ea%x800
\label{ex:x800}
\gll Naap  ma-emi  om-em-ika-i-nan,  \textstyleEmphasizedVernacularWords{na}. \\
      \\
\glt
\z

thus  say-SS.SIM  cry-SS.SIM-be-Np-FU.2s  INTJ

`You must cry saying like that.'

\ea%x801
\label{ex:x801}
\gll Yo  damol-al-e-m  \textstyleEmphasizedVernacularWords{oo},  fiker  fufa  iw-a-m  \textstyleEmphasizedVernacularWords{oo}. \\
      \\
\glt
\z

1s.UNM  bad-INCH-PA-1s  oh  kunai.grass  old.grass  enter-PA-1s  oh

` Oh, I'm in a bad way, I am hiding among the grass.'

\ea%x802
\label{ex:x802}
\gll Ni  kaaneke  ik-e-man  \textstyleEmphasizedVernacularWords{oo},  ni  ekap-omak-eka \\
      \\
\glt
\z

2p.UNM  where  be-PA-2p  oh  2p.UNM  come-DISTR/PL-IMP.2p

\textstyleEmphasizedVernacularWords{oo}!

oh

`O where are you? -- come!'

\ea%x803
\label{ex:x803}
\gll \textstyleEmphasizedVernacularWords{Yii},  ifa=ke  \textstyleEmphasizedVernacularWords{yee}! \\
      \\
\glt
\z

Eek  snake=CF  INTJ

`Eek, that's a snake oh!'

